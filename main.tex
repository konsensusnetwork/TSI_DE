% Options for packages loaded elsewhere
\PassOptionsToPackage{unicode}{hyperref}
\PassOptionsToPackage{hyphens}{url}
%
\documentclass[
  a5paper,
  smalldemyvopaper,10pt,twoside,onecolumn,openright,extrafontsizes,hidelinks]{memoir}

\usepackage{amsmath,amssymb}
\usepackage{iftex}
\ifPDFTeX
  \usepackage[T1]{fontenc}
  \usepackage[utf8]{inputenc}
  \usepackage{textcomp} % provide euro and other symbols
\else % if luatex or xetex
  \usepackage{unicode-math}
  \defaultfontfeatures{Scale=MatchLowercase}
  \defaultfontfeatures[\rmfamily]{Ligatures=TeX,Scale=1}
\fi
\usepackage{lmodern}
\ifPDFTeX\else  
    % xetex/luatex font selection
\fi
% Use upquote if available, for straight quotes in verbatim environments
\IfFileExists{upquote.sty}{\usepackage{upquote}}{}
\IfFileExists{microtype.sty}{% use microtype if available
  \usepackage[]{microtype}
  \UseMicrotypeSet[protrusion]{basicmath} % disable protrusion for tt fonts
}{}
\makeatletter
\@ifundefined{KOMAClassName}{% if non-KOMA class
  \IfFileExists{parskip.sty}{%
    \usepackage{parskip}
  }{% else
    \setlength{\parindent}{0pt}
    \setlength{\parskip}{6pt plus 2pt minus 1pt}}
}{% if KOMA class
  \KOMAoptions{parskip=half}}
\makeatother
\usepackage{xcolor}
\setlength{\emergencystretch}{3em} % prevent overfull lines
\setcounter{secnumdepth}{5}
% Make \paragraph and \subparagraph free-standing
\makeatletter
\ifx\paragraph\undefined\else
  \let\oldparagraph\paragraph
  \renewcommand{\paragraph}{
    \@ifstar
      \xxxParagraphStar
      \xxxParagraphNoStar
  }
  \newcommand{\xxxParagraphStar}[1]{\oldparagraph*{#1}\mbox{}}
  \newcommand{\xxxParagraphNoStar}[1]{\oldparagraph{#1}\mbox{}}
\fi
\ifx\subparagraph\undefined\else
  \let\oldsubparagraph\subparagraph
  \renewcommand{\subparagraph}{
    \@ifstar
      \xxxSubParagraphStar
      \xxxSubParagraphNoStar
  }
  \newcommand{\xxxSubParagraphStar}[1]{\oldsubparagraph*{#1}\mbox{}}
  \newcommand{\xxxSubParagraphNoStar}[1]{\oldsubparagraph{#1}\mbox{}}
\fi
\makeatother


\providecommand{\tightlist}{%
  \setlength{\itemsep}{0pt}\setlength{\parskip}{0pt}}\usepackage{longtable,booktabs,array}
\usepackage{calc} % for calculating minipage widths
% Correct order of tables after \paragraph or \subparagraph
\usepackage{etoolbox}
\makeatletter
\patchcmd\longtable{\par}{\if@noskipsec\mbox{}\fi\par}{}{}
\makeatother
% Allow footnotes in longtable head/foot
\IfFileExists{footnotehyper.sty}{\usepackage{footnotehyper}}{\usepackage{footnote}}
\makesavenoteenv{longtable}
\usepackage{graphicx}
\makeatletter
\def\maxwidth{\ifdim\Gin@nat@width>\linewidth\linewidth\else\Gin@nat@width\fi}
\def\maxheight{\ifdim\Gin@nat@height>\textheight\textheight\else\Gin@nat@height\fi}
\makeatother
% Scale images if necessary, so that they will not overflow the page
% margins by default, and it is still possible to overwrite the defaults
% using explicit options in \includegraphics[width, height, ...]{}
\setkeys{Gin}{width=\maxwidth,height=\maxheight,keepaspectratio}
% Set default figure placement to htbp
\makeatletter
\def\fps@figure{htbp}
\makeatother

% typographical packages
\usepackage{microtype}
\usepackage{setspace}
\tolerance=6000
\hyphenpenalty=1000

% typographical settings for the body text
\setlength{\parskip}{0em}
\setlength{\parindent}{1em}
\linespread{1}

% DEFINITIONS TITLE PAGE / COPYRIGHT
\newcommand{\titleoriginal}{The Sovereign Individual}
\newcommand{\subtitleoriginal}{Mastering the Transition to the Information Age}
\newcommand{\yearoriginal}{1999}
\newcommand{\subtitletranslation}{Der Übergang zum Informationszeitalter}
\newcommand{\yeartranslation}{2024}
\newcommand{\stringtranslation}{Übersetzung}
\newcommand{\stringlicense}{Alle Rechte vorbehalten.}
\newcommand{\stringpublisher}{Verlag}
\newcommand{\ISBNHC}{978-9916-749-25-8}
\newcommand{\ISBNSC}{978-9916-749-26-5}
\newcommand{\ISBNEBOOK}{978-9916-749-27-2}
\newcommand{\ISBNAUDIO}{978-9916-749-29-6}
\newcommand{\press}{Konsensus Network}
\newcommand{\translatorone}{Andreas Tank}
\newcommand{\translators}{
\large\textit{\stringtranslation:}\\
\translatorone\\
}

% PHYSICAL DOCUMENT SETUP
\setstocksize{210mm}{148mm}
\settrimmedsize{210mm}{148mm}{*}
\setbinding{7mm}
\setlrmarginsandblock{15mm}{16mm}{*}
\setulmarginsandblock{16mm}{16mm}{*}
\setlength{\skip\footins}{18pt} % More space between the text and the footnote line

% FONTS
\usepackage{fontspec}
\setmainfont{stone-serif}[
    Path=./fonts/stone-serif-itc-pro/,
    Scale=0.83,
    Extension=.OTF,
    UprightFont=*-Regular,
    BoldFont=*-SemiBd,
    ItalicFont=*-MediumIt,
    BoldItalicFont=*-SemiBdIt
    ]

\setsansfont{stone-sans}[
    Path=./fonts/stone-sans/,
    Scale=0.85,
    Extension=.otf,
    UprightFont=*-Medium,
    BoldFont=*-Semibold,
    ItalicFont=*-MediumItalic,
    BoldItalicFont=*-SemiBoldItalic
    ]

\usepackage{lettrine}
\setcounter{DefaultLines}{3}
\renewcommand{\DefaultLoversize}{0.1}
\renewcommand{\DefaultLraise}{0}
\renewcommand{\LettrineTextFont}{}
\setlength{\DefaultFindent}{\fontdimen2\font}
\setlength{\DefaultNindent}{0em}

% custom second title page
\makeatletter
\newcommand*\halftitlepage{\begingroup % Misericords, T&H p 153
  \setlength\drop{0.1\textheight}
  %\begin{center}
  \vspace*{\drop}
  \rule{\textwidth}{0in}\par
  {\Large\sffamily\thetitle\par}
  \rule{\textwidth}{0in}\par
  \vfill
  %\end{center}
\endgroup}
\makeatother

% custom title page
\makeatletter
\newlength\drop
\newcommand*\titleM{\begingroup % Misericords, T&H p 153
  \setlength\drop{0.15\textheight}
  %\begin{center}
  \vspace*{\drop}
  {\HUGE\sffamily\thetitle\par}
  \vspace{2em}
  {\Large\sffamily\textit\subtitletranslation\par}
  \vspace{4em}
  \rule{5.5cm}{0.3mm}\par
  \vspace{4em}
  {\Large\sffamily\textit\theauthor\par}
  \vspace{6em}
  % {\footnotesize\sffamily\textit\translators\par}
  \vfill
  \includegraphics[width=3.5cm]{figures/knw.png}\par
  %\end{center}
\endgroup}
\makeatother

% copyright page
\makeatletter
\newcommand*\copyrightpage{\begingroup
  \setlength\drop{0.1\textheight}
  \vphantom{just for the drop}
  \vfill
  \begin{footnotesize}
  \noindent \copyright\space \yearoriginal: \theauthor
  \par\noindent \textit{\titleoriginal: \subtitleoriginal}
  \vspace{0.5\baselineskip}
  \par\noindent \copyright\space \yeartranslation\space \stringtranslation: \translatorone
  \par\noindent \textit{\thetitle: \subtitletranslation}
  \vspace{\baselineskip}
  \par\noindent \textit{\stringlicense}
  \vspace{0.5\baselineskip}
  \par\noindent \stringpublisher: \href{https://konsensus.network}{\textit{konsensus.network}}
  \vspace{0.5\baselineskip}
  \par\noindent v1.0.0
  \vspace{0.5\baselineskip}
  \setlength{\parindent}{2em}% default 20pt
  \par\noindent ISBN \ISBNHC \:Hardcover
  \par\hspace{0.28\parindent}\ISBNSC \:Paperback
  \par\hspace{0.28\parindent}\ISBNEBOOK \:E-book\par
  \setlength{\parindent}{0pt}
  \end{footnotesize}
  \vspace{3em}
  \par\noindent \href{https://konsensus.network}{\includegraphics[width=1cm]{figures/freestarfish.png}}
  \par\noindent \href{https://konsensus.network}{\includegraphics[width=3.5cm]{figures/knw.png}}
  \setcounter{footnote}{0}
  \clearpage
\endgroup}
\makeatother

% HEADER AND FOOTER MANIPULATION
% for normal pages
\nouppercaseheads
\headsep = 4mm
\makepagestyle{mystyle} 
\makeevenhead{mystyle}{\scriptsize\sffamily\mdseries\thepage}{}{}
\makeoddhead{mystyle}{\scriptsize\sffamily\mdseries\leftmark}{}{\scriptsize\sffamily\mdseries\thepage}
\makeevenfoot{mystyle}{}{}{}
\makeoddfoot{mystyle}{}{}{}
\makeatletter

% for pages where chapters begin
\makepagestyle{plain}
\makerunningwidth{plain}{\headwidth}
\makeevenfoot{plain}{}{}{}
\makeoddfoot{plain}{}{}{}
\pagestyle{mystyle}

\newif\ifmainmatter
\appto\mainmatter{\mainmattertrue}
\appto\backmatter{\mainmatterfalse}
\appto\appendix{\mainmatterfalse}

\renewcommand\chaptermark[1]{%
  \markboth{\MakeUppercase{%
    \ifmainmatter~\oldstylenums\thechapter.~\fi#1}}{}}%

% TOC
\usepackage[]{tocloft}
\renewcommand{\cftsectiondotsep}{\cftnodots}
\renewcommand{\cftpartfont}{\Large\sffamily\MakeUppercase}
\renewcommand{\cftchapterfont}{\small\sffamily}
\renewcommand{\cftsectionfont}{\Small\sffamily}
\renewcommand{\cftpartpagefont}{\Large\sffamily}
\renewcommand{\cftchapterpagefont}{\small}
\renewcommand{\cftchapterpresnum}{KAPITEL\space}
\renewcommand{\cftchapternumwidth}{7em}
\setlength{\cftchapterindent}{0em}
\setlength{\cftsectionindent}{5em}
\setlength{\cftbeforechapterskip}{0.8em}
\setsecnumdepth{chapter}
\setcounter{tocdepth}{0}


% Redefine footnote presentation
\makeatletter
\renewcommand\@makefntext[1]{%
  \noindent\hb@xt@2em{% <-- Box of fixed size for footnote number and space
    \@thefnmark\quad}% <-- Footnote number followed by a quad space
  \parbox[t]{\dimexpr\linewidth-2em}{#1}% <-- Parbox to control the width of footnote content
}
\makeatother

% layout check and fix
\checkandfixthelayout

% COUNTERS FOOTNOTES
\usepackage{chngcntr}
\counterwithout*{footnote}{chapter}

% TITLE FORMATTING
\usepackage{titlesec}

% Define chapter format with titlesec
\titleformat
    {\chapter}[display]
    {\huge\sffamily} % Main title font style
    {\Large\sffamily\chaptertitlename~\thechapter} % "Chapter N" format
    {0pt} % Space between the chapter number and title
    {\Huge} % Chapter title formatting
    [\vspace{10pt}\Large\textit{\chaptersubtitle}] % Subtitle formatting

% Command to set the subtitle (empty by default)
\newcommand{\chaptersubtitle}{}

% Automatically render the subtitle (if set) after the chapter title
\titleformat{\chapter}[display]
  {\huge\sffamily}
  {\Large\sffamily\chaptertitlename\ \thechapter}
  {0pt}
  {\Huge}
  [\ifx\chaptersubtitle\empty\else\vspace{10pt}\Large\textit{\chaptersubtitle}\fi]

% Command to set subtitle manually after chapter rendering
\newcommand{\setsubtitle}[1]{%
  \renewcommand{\chaptersubtitle}{#1}%
  \chaptermark{\chaptersubtitle} % Update subtitle for header/footer
}

\titleformat
  {\section}[block]
  {\sffamily\large\bfseries}
  {}
  {0pt}
  {}
  
\titlespacing*{\section}{0pt}{2em}{0.5em}

\titleformat{\subsection}{\sffamily\bfseries}{}{}{}
\titlespacing*{\subsection}{0pt}{2em}{0em}

% QUOTE FORMATTING
\renewenvironment{quote}%
               {\list{}{\rightmargin=.6cm\leftmargin=.6cm}%
                \itshape \item[]}% <- The effect of \samepage is local!!!
               {\endlist}

% LAYOUT CHECK AND FIX
\checkandfixthelayout

% CUSTOM TITLE PAGE
\makeatletter
\def\@maketitle{%
  % the half title page
  \pagestyle{empty}
  \halftitlepage
  \cleardoublepage

  % the title page
  \titleM
  \clearpage

  % the copyright page
  \copyrightpage
  \cleardoublepage
  \pagestyle{mystyle}
}
\makeatother
% END PREAMBLE
\makeatletter
\@ifpackageloaded{bookmark}{}{\usepackage{bookmark}}
\makeatother
\makeatletter
\@ifpackageloaded{caption}{}{\usepackage{caption}}
\AtBeginDocument{%
\ifdefined\contentsname
  \renewcommand*\contentsname{Inhaltsverzeichnis}
\else
  \newcommand\contentsname{Inhaltsverzeichnis}
\fi
\ifdefined\listfigurename
  \renewcommand*\listfigurename{Abbildungsverzeichnis}
\else
  \newcommand\listfigurename{Abbildungsverzeichnis}
\fi
\ifdefined\listtablename
  \renewcommand*\listtablename{Tabellenverzeichnis}
\else
  \newcommand\listtablename{Tabellenverzeichnis}
\fi
\ifdefined\figurename
  \renewcommand*\figurename{Abbildung}
\else
  \newcommand\figurename{Abbildung}
\fi
\ifdefined\tablename
  \renewcommand*\tablename{Tabelle}
\else
  \newcommand\tablename{Tabelle}
\fi
}
\@ifpackageloaded{float}{}{\usepackage{float}}
\floatstyle{ruled}
\@ifundefined{c@chapter}{\newfloat{codelisting}{h}{lop}}{\newfloat{codelisting}{h}{lop}[chapter]}
\floatname{codelisting}{Listing}
\newcommand*\listoflistings{\listof{codelisting}{Listingverzeichnis}}
\makeatother
\makeatletter
\makeatother
\makeatletter
\@ifpackageloaded{caption}{}{\usepackage{caption}}
\@ifpackageloaded{subcaption}{}{\usepackage{subcaption}}
\makeatother

\ifLuaTeX
\usepackage[bidi=basic]{babel}
\else
\usepackage[bidi=default]{babel}
\fi
\babelprovide[main,import]{ngerman}
% get rid of language-specific shorthands (see #6817):
\let\LanguageShortHands\languageshorthands
\def\languageshorthands#1{}
\ifLuaTeX
  \usepackage{selnolig}  % disable illegal ligatures
\fi
\usepackage{bookmark}

\IfFileExists{xurl.sty}{\usepackage{xurl}}{} % add URL line breaks if available
\urlstyle{same} % disable monospaced font for URLs
\hypersetup{
  pdftitle={Das souveräne Individuum},
  pdfauthor={James Dale Davidson \& Lord William Rees-Mogg},
  pdflang={de},
  hidelinks,
  pdfcreator={LaTeX via pandoc}}


\title{Das souveräne Individuum}
\usepackage{etoolbox}
\makeatletter
\providecommand{\subtitle}[1]{% add subtitle to \maketitle
  \apptocmd{\@title}{\par {\large #1 \par}}{}{}
}
\makeatother
\subtitle{Der Übergang zum Informationszeitalter}
\author{James Dale Davidson \& Lord William Rees-Mogg}
\date{2024-11-25}

\begin{document}
\frontmatter
\maketitle

\renewcommand*\contentsname{Inhalt}
{
\setcounter{tocdepth}{0}
\tableofcontents
}

\mainmatter
\bookmarksetup{startatroot}

\chapter*{About this book}\label{about-this-book}

\markboth{About this book}{About this book}

\bookmarksetup{startatroot}

\chapter*{Vorwort von Titus Gebel}\label{vorwort-von-titus-gebel}
\addcontentsline{toc}{chapter}{Vorwort von Titus Gebel}

\markboth{Vorwort von Titus Gebel}{Vorwort von Titus Gebel}

Prognosen sind bekanntlich schwierig, insbesondere, je weiter entfernt
die Zukunft ist, die sie betreffen. Wenn aber ein politisches Buch, das
solche Prognosen macht, 25 Jahre nach seinem Erscheinen immer noch
aufgelegt und sogar erstmals ins Deutsche übertragen wird, dann lagen
seine Autoren offenbar in entscheidenden Punkten richtig. Und so ist es
auch hier. Diese Punkte betreffen Demokratie, Nationalstaat, Politik,
Kryptowährungen und den Machtzuwachs des selbstbestimmten Einzelnen. Der
vorhergesagte Wandel hat vielleicht etwas später eingesetzt, als die
Autoren seinerzeit vermutet haben; aber er ist gerade jetzt im vollen
Gange und deshalb ist dieses Buch so bedeutend.

Während der politisch-mediale Hauptstrom der Neunzigerjahre, verkörpert
durch Francis Fukuyama, der Ansicht war, die westlichen Demokratien
seien das Ende der Geschichte, vertreten die Autoren dieses Buches das
genaue Gegenteil: „\emph{Wir gehen davon aus, dass die repräsentative
Demokratie, so wie wir sie heute kennen, verschwinden und durch die neue
Demokratie der Wahlfreiheit \ldots{} ersetzt werden wird.''}

Davon ist in der Tat auszugehen, und zwar im Wesentlichen aus drei
Gründen: Erstens enden herkömmliche Demokratien, ohne Ausnahme, früher
oder später in einer Art von Sozialismus, nämlich nachdem die Mehrheit
herausgefunden hat, dass sie sich Geld in die eigene Tasche wählen kann.
Deshalb wird auch kein Unternehmen auf dieser Welt demokratisch geführt.
Zweitens sind umverteilende Systeme, denen nicht jeder Betroffene vorab
zugestimmt hat, illegitim. Sie sind Verträge zulasten Dritter, eine
Rechtsfigur, die in sämtlichen Privatrechtsordnungen dieser Welt
unzulässig ist. Drittens gibt es keinen ethisch-moralischen Grund, einem
Gesetz unterworfen zu sein, dessen Geltung man nicht zugestimmt hat.
Dafür gibt es keine stichhaltige philosophische Rechtfertigung, wenn wir
von der Gleichwertigkeit aller erwachsenen Menschen ausgehen.

Die historische Entwicklung staatlicher Systeme belegt die
Schlussfolgerung von Davidson und Rees-Mogg ebenfalls: Früher bestimmte
der König, wie ich zu leben habe, jetzt bestimmt die Mehrheit, wie ich
zu leben habe, und in Zukunft werde ich selbst bestimmen, wie ich leben
möchte. Vom Untertanen zum Bürger, vom Bürger zum Kunden.

Auch die Macht der Nationalstaaten schwindet vielerorts und wie von den
Autoren vorhergesagt, zerteilen sich manche in kleinere Einheiten. Es
gibt heute mehr unabhängige Staaten und Gebiete als noch zum Zeitpunkt
des Erscheinens des Buches. Hinzu kommt, dass internationale
Organisation mehr und mehr Entscheidungsbefugnis an sich ziehen, und den
Nationalstaat so auch von außen immer weiter entmachten. Grenzen von
Nationalstaaten spielen immer weniger eine Rolle, wenn von überall
online gearbeitet werden kann. Das ist bereits im großen Maßstab möglich
und erschwert zunehmend die Besteuerung. Selbst wenn China momentan als
gefestigter Nationalstaat erscheint, der vor Kraft kaum laufen kann: ein
verlorener Krieg, etwa um Taiwan, könnte das sehr schnell ändern.

Auch die Prognose, dass wir mehr kleinere Rechtsordnungen sehen werden,
die wie Hongkong nach dem Modell „Ein Land - Zwei Systeme''
funktionieren, war richtig. Zwar wird Hongkong vom Mutterland China
zunehmend wieder absorbiert, aber global gesehen ist das Modell auf dem
Vormarsch. Als das Buch 1999 erstmals erschien, gab es etwa 500
Sonderwirtschaftszonen, die Sonderregelungen in bestimmten Bereichen
vorsehen, heute sind es fast 8000! Jede einzelne Sonderwirtschaftszone
ist dabei bereits ein stillschweigendes Eingeständnis des Staates, dass
seine Regelungen offenbar nicht für alle optimal sind.

Die innovativsten dieser Zonen gehen bereits in Richtung eines echten
Parallelsystems, man denke etwa an das \emph{Dubai International
Financial Center} mit eigener Gerichtsbarkeit und eigenem
Common-Law-System, die Honduranischen ZEDEs, welche erlauben, ein
komplettes Rechtssystem mitsamt eigener Institutionen neu zu schaffen,
oder das Megaprojekt NEOM in Saudi-Arabien, wo der Staat selbst ein
zweites System innerhalb seines Gebietes errichtet, das ausdrücklich
liberalere Regelungen als das Mutterland haben soll.

Auch wenn es derzeit noch nicht so aussieht: auf lange Sicht wird die
Politik, also das Erheben von Wünschen einzelner zum Maßstab für alle,
zurückgehen. Denn Politik ist in jeder Form, die sie annimmt,
kooperationshemmende Intervention. Es kann im Sinne der Freiheit und
Selbstbestimmung daher keine „richtige'' Politik geben.
Freiheitsfördernd ist nur der größtmögliche Verzicht auf Politik.

Die Entstehung von digitalem Geld, das staatlichem Einfluss nicht
unterliegt, wurde von den Autoren bereits mehr als zehn Jahre vor der
Schaffung von Bitcoin vorhergesagt. Noch haben Kryptowährungen das
staatliche Fiat-Geld nicht als Hauptzahlungsmittel abgelöst, aber nach
der nächsten Hyperinflation werden die Karten neu gemischt.

Im Informationszeitalter führen die neuen technischen Entwicklungen zwar
dazu, dass der Staat umfassender überwachen kann. Auf der anderen Seite
ermöglichen sie dem Einzelnen aber auch, durch Verschlüsselung und die
Nutzung von VPNs etc. wirksame Gegenmittel zu entwickeln. Heute
erreichen einzelne Podcaster mehr Menschen als große Fernsehsender mit
ihren Milliardenbudgets. Insgesamt kann der Einzelne mithilfe der
Informationstechnologie deutlich mehr Schlagkraft und Reichweite
entwickeln als noch vor 25 Jahren, und genau das war die Prognose von
Davidson und Rees-Mogg.

Die Autoren haben weiter gewarnt, dass der Nationalstaat all das nicht
kampflos hinnehmen werde. Betrachtet man die Einschränkung der
Grundrechte während der COVID-Zeit, die Klima-Planwirtschaft, die immer
anmaßenderen Anti-Geldwäsche-Maßnahmen, die ein Hundertfaches von dem
kosten, was sie finanziell einbringen, der Versuch, die Meinungsfreiheit
massiv zu beschneiden und die kartellartigen Bemühungen, eine globale
Mindeststeuer einzuführen, wird klar, dass jetzt die schweren Geschütze
aufgefahren werden, um den Bedeutungsverlust des Staates und seiner
zahllosen Kostgänger abzuwenden. Willfährige Wissenschaftler haben
bereits Theorien bereitgestellt, wonach grenzenloses Gelddrucken
überhaupt kein Problem sei. Und wenn gar nichts mehr hilft, wird eben
ein Krieg vom Zaun gebrochen.

Mit keiner dieser Maßnahmen zum Machterhalt wird aber der erreichte
Wohlstand aufrechterhalten werden können. Wissen ist dezentral und je
mehr Freiheit der Einzelne hat, desto mehr Produktivität wird
freigesetzt. Je enger der Würgegriff der Regulierung an den Menschen und
Märkten anliegt, desto geringer wird der Output. Jeder Staat, der das
nicht versteht, ist ein \emph{Ancien Régime} und wird das Schicksal
seiner Vorgänger teilen.

Die meisten Menschen wollen sich nicht Regeln und Vorschriften
unterwerfen, denen sie nicht zugestimmt haben. Die meisten Menschen
wollen nicht für Dinge bezahlen, die sie nicht bestellt haben. Und
vernünftige Menschen brauchen nicht Hunderte oder Tausende von Gesetzen,
um friedlich zusammenzuleben. Stattdessen benötigen die Menschen einen
sicheren Raum, in dem sie leben und mit anderen zusammenarbeiten können,
aber ansonsten in Ruhe gelassen werden.

Die bestehenden politischen Systeme können diese Wünsche nicht mehr
erfüllen. Aus diesem Grund haben neue, kundenorientierte Konzepte wie
Freie Privatstädte eine Chance auf Erfolg, wie von den Autoren
vorhergesagt. Am Ende gehen die Menschen nämlich dorthin, wo sie am
besten behandelt werden. Oder, um mit dem Titel des Buches zu sprechen,
wo sie souveräne Individuen sein können.

\vspace{2em}

--- \textbf{Titus Gebel}

\bookmarksetup{startatroot}

\chapter*{Vorwort von Max Hillebrand}\label{vorwort-von-max-hillebrand}
\addcontentsline{toc}{chapter}{Vorwort von Max Hillebrand}

\markboth{Vorwort von Max Hillebrand}{Vorwort von Max Hillebrand}

In der großen Bilanz der menschlichen Literatur gibt es nur wenige
Bücher, die die Zukunft akkurat voraussagen und uns den Weg in ihrem
hellen Licht der Weitsicht zeigen. „Das Souveräne Individuum'' von James
Dale Davidson und William Rees-Mogg ist ein solches Meisterwerk. Ein
visionäres Buch, das nicht nur das Eintreten des digitalen Zeitalters
vorausgesagt, sondern auch den intellektuellen Grundstein für die
vollkommene Befreiung des Individuums vor dem veralternden Nationalstaat
gelegt hat. Da wir nun 25 Jahre nach dem Anfang der vorhergesagten neuen
Ära stehen, ist es lohnenswert, dieses wegweisende Werk erneut zu
betrachten und die vielfältigen Weisen zu untersuchen, in denen seine
Ideen verwirklicht wurden, insbesondere anhand der verändernden Kraft
von Cypherpunk-Technologien.

Ein Gang durch die Seiten von „Das Souveräne Individuum'' ist eine Reise
durch eine Landschaft der Ideen, die genauso großartig wie vielfältig
ist. Die Autoren, beide bekannt für ihr Wissen in den Gebieten Finanzen
und Ökonomie, spinnen eine überzeugende Erzählung, die historische
Analyse, ökonomische Theorie und technologische Vorhersage miteinander
verbindet. Ihre kombinierten Fachkenntnisse ermöglichen es ihnen, das
Bild einer Welt zu zeichnen, in der das Individuum sich selbst
ermächtigt. Ausgestattet mit den Werkzeugen der digitalen Technologie,
können die Grenzen von Geographie, Kultur und Politik überschritten
werden, um ein beispielloses Freiheits- und Wohlstandsniveau zu
erreichen.

Im Zentrum von „Das Souveräne Individuum'' steht die Idee, dass das
Eintreten der digitalen Technologie die Machtbalance zwischen dem
Einzelnen und dem Staat grundlegend verändert. Die Autoren
argumentieren, dass das Informationszeitalter eine neue Klasse von
Individuen geschaffen hat, die nicht den traditionellen
Autoritätsquellen wie Regierungen und Großkonzernen verpflichtet sind.
Diese „souveränen Individuen'' sind in der Lage, die Macht der digitalen
Technologie zu nutzen, um Wohlstand zu schaffen, Datenschutz zu
gewährleisten und der erstickenden Bevormundung des Nationalstaates zu
entfliehen.

„Das Souveräne Individuum'' prophezeite das Erscheinen von
Digital-Nomaden: Einzelpersonen, die online arbeiten, während sie um die
Welt reisen und sich dafür entscheiden, an jenen Orten zu wohnen, die
ihnen das beste Umfeld bieten. In dieser globalisierten Welt können
Staatsbürgerschaft, Steuerresidenz, Firma-Jurisdiktion, Bankkonten und
Investmentdepots aus verschiedenen Ländern bezogen werden, wodurch
Einzelpersonen ihre Umstände optimieren können, indem sie die besten
verfügbaren Optionen in jeder Gerichtsbarkeit auswählen.

Die Implikationen dieser Vision sind tiefgreifend, da sie die Grundlagen
unseres Verständnisses von Gesellschaft, Politik und Ökonomie
herausfordern. Indem sich „das souveräne Individuum'' als ultimatives
Machtzentrum positioniert, bricht der Einzelne mit traditionellen
Hierarchien der Autorität und begrüßt eine neue Ära des Individualismus
und der Selbstbestimmung. Dies erfordert wiederum tiefgreifende
Auswirkungen auf die Natur der Regierungsführung, die Rolle des Staates
und die Zukunft der Demokratie selbst.

Eines der auffälligsten Merkmale von „Das Souveräne Individuum'' ist
seine weitsichtige Antizipation der verändernden Kraft der
Cypherpunk-Technologie. Die Autoren gehörten zu den ersten, die das
Potenzial der digitalen Technologie erkannten, um ein neues Paradigma
der individuellen Befreiung zu schaffen. Ihre Einsichten sind durch den
bemerkenswerten Fortschritt der Cypherpunk-Bewegung in den vergangenen
Jahren bestätigt worden. Fast dreißig Jahre nach der ersten Ausgabe,
haben wir jetzt die Möglichkeit zu überprüfen, ob unsere heutigen
Technologien auch den Prognosen entsprechen.

Der Aufstieg von Bitcoin, der weltweit ersten dezentralen digitalen
Währung, ist vielleicht das eindrucksvollste Beispiel für die
Verwirklichung der Vision der Autoren. Indem es die Macht der
Kryptographie und des verteilten Konsens nutzt, hat Bitcoin eine neue
Form des Geldes geschaffen, die nicht den Launen der Zentralbanken oder
den Unregelmäßigkeiten des globalen Finanzsystems unterliegt. Mit einer
Bitcoin Full Node kann jeder Nutzer für sich selbst die Regeln seines
eigenen Geldsystems definieren, verifizieren und durchzusetzen, ganz
ohne Vertrauen auf Institutionen oder Erlaubnis der Nationalstaaten.
Dies hat seinerseits eine neue Klasse von Digital-Unternehmern
hervorgebracht, die grenzübergreifend transaktionsfähig sind, ohne auf
das traditionelle Finanzsystem angewiesen zu sein.

Coinjoin ist eine weitere Innovation, die die Privatsphäre von
Bitcoin-Transaktionen verbessert, während das Lightning Network schnelle
und kostengünstige Zahlungen bietet. Dank Bitcoin und Protokollen wie
Cashu und Fedimint erlebt E-Cash eine Renaissance. Cashu ermöglicht es
jedem, ein digitales Geldlager zu betreiben, und beseitigt somit das
Monopol von Depositenbanken. Fedimint verbessert dieses Konzept, indem
es Risiken und Verantwortlichkeiten unter mehreren Betreibern in
verschiedenen Jurisdiktionen verteilt. Diese Technologien zeigen, dass
zusätzliche Protokolle die Souveränität des Benutzers über seine
finanziellen Transaktionen im Bitcoin-Ökosystem exponentiell erhöhen
können.

Die Zero-Knowledge-Kryptographie ist ein weiterer Baustein für eine
freiere Zukunft. Diese Technik ermöglicht es Einzelpersonen, Aussagen zu
beweisen, ohne die konkreten Informationen preiszugeben. Damit können
digitale Dienstleistungen zur Verfügung gestellt werden, ohne dass die
persönlichen Daten der Nutzer veröffentlicht werden.

Peer-to-Peer-Systeme wie BitTorrent ermöglichen Benutzern den direkten
Austausch von Dateien und Ressourcen ohne Zwischenstellen und schaffen
ein neues Paradigma von dezentralisierter Zusammenarbeit und
Wissensaustausch. Zwiebelrouter wie Tor als auch Mixnets wie Nym
ermöglichen verschlüsselte Kommunikation, bei der niemand weiß welcher
Computer sich zu welchem anderen verbindet und ermöglichen freie
Kommunikation ohne Angst vor Überwachung oder Zensur.

Schließlich repräsentiert der Aufstieg von Nostr, einem neuen Protokoll
für den standardisierten Austausch von signierten Daten, das neueste
Gebiet des fortwährenden Strebens nach digitaler Souveränität. Nostr hat
es geschafft, mit Alltagsapplikationen ein Web-of-Trust von
kryptographischen Schlüsseln aufzubauen, welches eine fundamentale Basis
für die souveräne Zukunft ist.

Zusammengefasst ist „Das Souveräne Individuum'' ein bahnbrechendes Werk,
das dem Test der Zeit standhalten konnte und eine Landkarte für das
digitale Zeitalter bietet, die heute genauso relevant ist wie bei ihrer
ersten Veröffentlichung im Jahr 1999. Ihre Autoren, James Dale Davidson
und William Rees-Mogg, waren Visionäre, die die verändernde Kraft der
digitalen Technologie und ihre Auswirkungen auf die Zukunft des
Einzelnen in der Gesellschaft vorhergesagt haben. So wie wir weiterhin
den unerbittlichen Marsch der Cypherpunk-Technologien verfolgen,
erinnern wir uns an die dauerhafte Kraft der in diesen Seiten
enthaltenen Ideen. In einer Welt, die zunehmend von dem Kampf zwischen
individueller Freiheit und staatlicher Kontrolle geprägt ist, bleibt
„Das Souveräne Individuum'' ein Ruf zu Taten, ein Zeugnis der
unerbittlichen Geisteskraft der menschlichen Erfindungskraft und ein
Leuchtturm der Hoffnung für die Zukunft der Menschheit.

\vspace{2em}

--- \textbf{Max Hillebrand}

\bookmarksetup{startatroot}

\chapter{DER ÜBERGANG IN DAS JAHR
2000}\label{der-uxfcbergang-in-das-jahr-2000}

\begin{quote}
„Es fühlt sich an, als stünde etwas Großes bevor: Diagramme
visualisieren das jährliche Bevölkerungswachstum, die Konzentration von
Kohlendioxid in der Atmosphäre, die Anzahl der Webadressen und die
Megabyte pro Dollar. Alle diese Faktoren zeigen eine steil ansteigende
Kurve, die kurz nach dem Jahrhundertwechsel in eine Asymptote übergeht:
die Singularität. Das Ende von allem, was wir kennen. Und der Anfang von
etwas, das wir möglicherweise nie vollständig begreifen werden.``
\footnote{Danny Hillis, \emph{The Millenium Clock}, Wired, Special
  Edition, Herbst 1995, S. 48.} - Danny Hillis
\end{quote}

\section{VORAHNUNGEN}\label{vorahnungen}

Die Jahrtausendwende hat die westliche Vorstellungskraft im letzten
Jahrtausend stark geprägt. Da die Welt zur Zeit des ersten Jahrtausends
nach Christus nicht untergegangen ist, blickten Theologen, Propheten,
Schriftsteller und Wahrsager mit der Erwartung auf das Ende des
Jahrzehnts, dass es etwas Bedeutendes einläuten wird. Sogar Isaac Newton
spekulierte, dass mit dem Jahr 2000 der Weltuntergang bevorstehen würde.
Michel de Nostredame, dessen Prophezeiungen seit ihrer
Erstveröffentlichung 1568 von jeder Generation gelesen werden, sagte für
Juli 1999 das Erscheinen des dritten Antichristen voraus.\footnote{Ericka
  Cheetham, \emph{The Final Prophecies of Nostradamus} (New York:
  Putnam, 1989), S. 424.} Der Schweizer Psychologe Carl Jung, Experte
für das „kollektive Unbewusste``, prophezeite für 1997 den Beginn eines
neuen Zeitalters. Es ist leicht, solche Voraussagen zu belächeln. Dies
gilt auch für die nüchternen Prognosen von Ökonomen wie Dr.~Edward
Yardeni von Deutsche Bank Securities, der voraussagte, dass
Computerstörungen zur Jahrtausendwende „die Weltwirtschaft zum
Stillstand bringen würden``.\footnote{Dr.~Edward Yardeni, \emph{Year
  2000 Recession: ``Prepare for the worst. Hope for the best``}, Version
  5.0, 13. Mai 1998, B1.2.} Ob man das Computerproblem des Jahres 2000
nun als unbegründete Hysterie ansieht, angezettelt von
Computerprogrammierern und IT-Beratern, um ihr Geschäft anzukurbeln,
oder als einen mysteriösen Fall von technischer Entfesselung in
Verbindung mit prophetischer Vorstellungskraft - es lässt sich nicht
leugnen, dass die Gegebenheiten am Vorabend des neuen Jahrtausends mehr
als nur gewöhnliche düstere Zweifel daran wecken, wohin sich die Welt
entwickelt.

Der Optimismus, der die westlichen Gesellschaften der letzten 250 Jahre
geprägt hat, wird schleichend von einer Unruhe bezüglich der Zukunft
verdrängt. Überall sind die Menschen unsicher und besorgt. Man kann es
in ihren Gesichtern sehen. Man kann es aus ihren Gesprächen heraushören.
Dies spiegelt sich sowohl in Umfragen, als auch in Wahlergebnissen
wider. Ganz so, wie unsichtbare, physikalische Veränderungen im
Ionengehalt der Atmosphäre ein aufkommendes Gewitter ankündigen, noch
bevor sich die Wolken verdunkeln und ein Blitz einschlägt, so hängen in
dieser Dämmerung des Jahrtausends Vorboten tiefgreifender Umwälzungen in
der Luft. Eine Person nach der anderen, jede auf ihre eigene Art, nimmt
das drohende Ende einer Lebensweise wahr. Mit Abschluss dieses
Jahrzehnts endet nicht nur ein mörderisches Jahrhundert, sondern auch
ein glorreiches Jahrtausend menschlicher Errungenschaften. Mit dem Jahr
2000 endet eine Ära.

\begin{quote}
„Denn es gibt nichts Verborgenes, das nicht ans Licht gebracht wird, und
nichts Geheimes, das nicht bekannt wird.`` - Matthäus 10:26
\end{quote}

Wir sind der Überzeugung, dass die moderne Phase der westlichen
Zivilisation ihrem Ende entgegengeht. In diesem Buch erklären wir,
warum. Wie viele frühere Werke, stellt es einen Versuch dar, in die
Dunkelheit zu blicken und die unklaren Umrisse und Dimensionen einer
noch kommenden Zukunft zu zeichnen. In diesem Sinne verstehen wir unsere
Arbeit als apokalyptisch, im ursprünglichen Sinne des Wortes.
Apokalypsis bedeutet auf Griechisch „Enthüllung``. Wir sind der Ansicht,
dass eine neue Geschichtsepoche - das Informationszeitalter - kurz vor
seiner „Enthüllung`` steht.

\begin{quote}
„Wir beobachten die Entstehung eines neuen logischen Raums, einer
allgegenwärtigen elektronischen Umgebung, zu der wir alle Zugang haben,
die wir betreten und erleben können. Kurz gesagt, wir erleben die Geburt
einer neuen Form von Gemeinschaft. Die virtuelle Gemeinschaft wird zum
Vorbild für ein säkulares Paradies; so wie Jesus sagte, es gäbe viele
Wohnstätten im Hause seines Vaters, so existieren auch viele virtuelle
Gemeinschaften, die jeweils ihre eigenen Bedürfnisse und Wünsche
reflektieren.`` - Michael Grasso\footnote{Michael Grasso, \emph{The
  Millenium Myth: Love and Death at the End of Time}, Wheaton, Illinois:
  Quest Books, 1995.}
\end{quote}

\section{DIE VIERTE STUFE DER MENSCHLICHEN
GESELLSCHAFT}\label{die-vierte-stufe-der-menschlichen-gesellschaft}

Dieses Buch widmet sich einer neuen Machtrevolution, die den Einzelnen
befreit, indem sie die Zwangsjacke des Nationalstaats des 20.
Jahrhunderts abschüttelt. Innovative Entwicklungen, die uns bisher
unbekannte Veränderungen in der Logik der Gewalt bringen, lassen uns die
Grenzen für die Zukunft neu ziehen. Sofern unsere Vermutungen zutreffen,
stehen wir am Vorabend der bedeutsamsten Revolution, die die Geschichte
je erlebt hat. Mit einer Geschwindigkeit, die nur wenige vorhersehen
können, wird die Mikroprozessortechnik den Nationalstaat untergraben und
zerstören und dabei neue Formen der sozialen Organisation hervorbringen.
Diese Entwicklung wird keineswegs ohne Komplikationen verlaufen.

Die vor uns liegende Herausforderung wird durch die atemberaubende
Geschwindigkeit, mit der sie heranrollt, umso gewaltiger wirken,
besonders im Vergleich zu den Entwicklungen der Vergangenheit. Wenn man
die gesamte Menschheitsgeschichte betrachtet - von den frühesten
Anfängen bis hin zur Gegenwart - lassen sich lediglich drei grundlegende
Phasen des Wirtschaftslebens identifizieren: (1) die Gesellschaften der
Jäger und Sammler; (2) die Agrargesellschaften; und (3) die
Industriegesellschaften. Doch nun zeichnet sich am Horizont eine
vollkommen neue Phase der sozialen Organisation ab, die vierte Stufe:
die Informationsgesellschaften.

Jede vorangegangene Phase der Gesellschaftsentwicklung war einzigartig
in Bezug auf die Evolution und Kontrolle von Gewalt. Wie wir noch im
Detail aufzeigen werden, versprechen Informationsgesellschaften eine
bedeutsame Reduzierung des Einsatzes von Gewalt, teilweise weil sie über
lokale Grenzen hinausreichen. Die virtuelle Realität des Cyberspace, von
Romanautor William Gibson als eine „einvernehmliche Halluzination``
beschrieben, wird sich so weit jenseits der Kontrolle von Tyrannen
erstrecken, wie die Vorstellungskraft das erlaubt. Im neuen Jahrtausend
wird die Bedeutung der Kontrolle über weitreichende Gewalt bei weitem
geringer sein als zu irgendeinem Zeitpunkt seit der Französischen
Revolution. Das wird weitreichende Konsequenzen haben. Eine davon wird
der Anstieg der Kriminalität sein. Während der Ertrag aus organisierter
Gewalt in großem Stil schrumpft, ist es wahrscheinlich, dass die Profite
aus Gewalt in kleinem Stil stark ansteigen werden. Gewalt wird
zufälliger und örtlich begrenzt sein. Das organisierte Verbrechen wird
zunehmen. Wir werden erklären, weshalb das so ist.

Eine weitere logische Konsequenz des nachlassenden Hangs zur Gewalt ist
das Verschwinden der Politik. Viele Indikatoren lassen vermuten, dass
das Beharren auf den staatsbürgerlichen Mythen des Nationalstaates des
20. Jahrhunderts rapide abnimmt. Der Tod des Kommunismus ist nur das
auffälligste Beispiel hierfür. Der moralische Verfall und die zunehmende
Korruption in den höchsten Ebenen westlicher Regierungen sind kein
Zufallsprodukt, wie wir in der Tiefe aufzeigen werden. Dies ist ein
Beleg dafür, dass die Möglichkeiten des Nationalstaates ausgeschöpft
sind. Selbst viele seiner Anführer glauben nicht mehr an die Floskeln,
die sie verkünden. Und auch der Rest nimmt sie ihnen nicht mehr ab.

\subsection{Geschichte wiederholt
sich}\label{geschichte-wiederholt-sich}

Diese Situation erinnert stark an vergangene Ereignisse. Immer wenn
technologische Veränderungen die alten Strukturen von den neuen
treibenden Kräften der Wirtschaft entkoppelt haben, verschieben sich die
moralischen Maßstäbe. Die Menschen beginnen, diejenigen, die die alten
Institutionen beherrschen, mit wachsender Verachtung zu betrachten.
Diese verbreitete Ablehnung setzt häufig ein, lange bevor die Menschen
eine schlüssige Ideologie des Wandels formulieren. So war es auch im
späten fünfzehnten Jahrhundert, als die mittelalterliche Kirche die
dominierende Institution des Feudalismus war. Trotz des Volksvertrauens
in die „Heiligkeit des geistlichen Amtes`` wurden sowohl hohe als auch
niedere Geistliche extrem verachtet - eine Einstellung, die
bemerkenswert der heutigen Haltung der Bevölkerung gegenüber Politikern
und Bürokraten ähnelt.\footnote{Johan Huizinga, \emph{The Waning of the
  Middle Ages}, trans. E Hopman (London: Penguin Books, 1990), S. 172.}

Wir glauben, dass wir viel von dem Jahrhundert, in dem das Leben voll
und ganz von organisierter Religion geprägt war, und von der heutigen
Zeit, in der die Welt von der Politik dominiert wird, lernen können. Die
Kosten für die Aufrechterhaltung der institutionalisierten Religion am
Ende des fünfzehnten Jahrhunderts hatten einen historischen Höchststand
erreicht - ähnlich wie heute die Kosten für die Unterstützung der
Regierung ein rekordverdächtiges Ausmaß angenommen haben.

Wir wissen, was mit der organisierten Religion aufgrund der
Nachwirkungen der Schießpulverrevolution passiert ist. Technologische
Entwicklungen haben damals starke Anreize geschaffen, religiöse
Institutionen zu verkleinern und ihre Kosten zu reduzieren. Eine
vergleichbare technologische Revolution wird zu Beginn des neuen
Jahrtausends auch eine radikale Verkleinerung der Nationalstaaten zur
Folge haben.

\begin{quote}
„Heute, nach über einem Jahrhundert elektronischer Technologie, haben
wir unser zentrales Nervensystem praktisch weltweit erweitert und dabei
sowohl räumliche als auch zeitliche Barrieren, zumindest in Bezug auf
unseren Planeten, überwunden.`` \footnote{Marshall McLuhan,
  \emph{Understanding Media}, New York: Signet, 1964, S. 19.}
\end{quote}

\subsection{Die informationelle
Revolution}\label{die-informationelle-revolution}

In dem Maße, wie die großen Systeme immer schneller zusammenbrechen,
lässt der systematische Zwang, der Wirtschaft und Einkommensverteilung
steuert, nach. Die Effizienz beim Organisieren sozialer Einrichtungen
wird schnell an Bedeutung gewinnen und somit wichtiger als
Machtstrukturen werden. Das bedeutet, dass Provinzen und selbst Städte,
die effektiv Eigentumsrechte durchsetzen und für Rechtssicherheit sorgen
können, ohne viele Ressourcen zu verbrauchen, im Informationszeitalter
eine tragfähige Souveränität erlangen werden, wie es in den letzten fünf
Jahrhunderten nicht vorkam. In der digitalen Welt, dem Cyberspace, wird
ein völlig neuer Wirtschaftssektor entstehen, der unabhängig von
physischer Gewalt agiert. Die deutlichsten Vorteile davon werden der
„kognitiven Elite`` zu Gute kommen, die zunehmend über nationale Grenzen
hinweg handelt. Diese Elite ist bereits in Städten wie Frankfurt,
London, New York, Buenos Aires, Los Angeles, Tokyo und Hongkong
gleichermaßen heimisch. Die Einkommensunterschiede innerhalb der
einzelnen Länder werden größer, während sie zwischen den Ländern
abnehmen.

Das selbstbestimmte Individuum untersucht die sozialen und finanziellen
Auswirkungen dieses revolutionären Umbruchs. Es liegt uns am Herzen, Sie
dabei zu unterstützen, die Potenziale dieser neuen Epoche optimal zu
nutzen und dabei nicht von ihren Folgen überrollt zu werden. Sollte auch
nur die Hälfte unserer Prognosen eintreffen, steht uns eine Veränderung
bevor, deren Ausmaß in der Geschichte beispiellos ist.

Der Jahreswechsel 2000 wird nicht nur die Weltwirtschaft grundlegend
verändern, sondern dies auch schneller bewerkstelligen als jeder andere
vorangegangene Paradigmenwechsel. Im Gegensatz zur landwirtschaftlichen
Revolution wird die informationelle Revolution nicht Jahrtausende
brauchen, um ihre volle Wirkung zu entfalten. Und anders als bei der
industriellen Revolution werden sich ihre Auswirkungen nicht über
Jahrhunderte hinweg ziehen. Die informationelle Revolution vollzieht
sich innerhalb einer Lebensspanne.

Darüber hinaus wird diese Veränderung nahezu überall gleichzeitig
geschehen. Technische und wirtschaftliche Innovationen werden nicht mehr
auf bestimmte Gebiete begrenzt sein. Der Wandel wird allgegenwärtig
sein. Und er wird einen so fundamentalen Bruch mit der Vergangenheit
darstellen, dass die magische Welt der Götter, wie sie sich frühe
Agrarvölker wie die alten Griechen vorstellten, beinahe zum Leben
erweckt wird. In einem viel größeren Ausmaß, als es sich die meisten
heute eingestehen möchten, könnte es schwierig oder sogar unmöglich
sein, viele der aktuellen Institutionen ins neue Jahrtausend zu retten.
Wenn sich die Informationsgesellschaften formen, werden sie sich von den
Industriegesellschaften ebenso stark unterscheiden, wie das alte
Griechenland von der Welt der Höhlenbewohner abwich.

\section{PROMETHEUS ENTFESSELT: DER AUFSTIEG DES SELBSTBESTIMMTEN
INDIVIDUUMS}\label{prometheus-entfesselt-der-aufstieg-des-selbstbestimmten-individuums}

\begin{quote}
„Mir ist keine ermutigendere Tatsache bekannt als die unbestreitbare
Fähigkeit des Menschen, sein Leben durch bewusste Anstrengung zu
bereichern`` - Henry David Thoreau.
\end{quote}

Der anstehende Wandel birgt sowohl Vor- als auch Nachteile. Der Vorteil
ist, dass die informationelle Revolution Individuen stärker befreien
wird als je zuvor. Erstmals werden all diejenigen, die in der Lage sind,
sich eigenständig weiterzubilden, fast vollkommen frei darin sein, ihre
eigene Arbeit zu gestalten und das Maximum an Nutzen aus ihrer
persönlichen Produktivität zu ziehen. Genialität wird sich entfalten und
sich sowohl von Regierungsunterdrückung als auch von den Fesseln
rassistischer und ethnischer Vorurteile lösen. In der
Informationsgesellschaft wird niemand, der tatsächlich fähig ist, von
den ungeschliffenen Meinungen anderer gebremst werden. Es wird
unerheblich sein, was der Großteil der Menschen weltweit über Ihre
Rasse, Ihr Aussehen, Ihr Alter, Ihre sexuellen Vorlieben oder Ihre
Frisur denkt. In der Cyberwirtschaft wird man Sie nicht einmal sehen.
Die Unattraktiven, die Übergewichtigen, die Alten und die Behinderten
werden unter denselben Voraussetzungen wie die Jungen und Schönen
konkurrieren -- nämlich in der vollkommen farbenblinden Anonymität der
neuen Grenzen des Cyberspace.

\subsection{Aus Ideen wird Reichtum}\label{aus-ideen-wird-reichtum}

Leistung, ganz gleich wo sie erbracht wird, wird künftig stärker belohnt
als je zuvor. In einer Umwelt, in der die wertvollste Ressource nicht
mehr materielles Kapital, sondern die eigenen Ideen sind, hat jeder, der
klug denkt, das Potenzial, wohlhabend zu sein. Das Informationszeitalter
wird das Zeitalter der steigenden Mobilität sein. Es wird den Milliarden
von Menschen in Teilen der Welt, die bisher nie voll am Wohlstand der
Industriegesellschaft partizipieren konnten, deutlich mehr
Chancengleichheit bieten. Ihre klügsten, erfolgreichsten und
ehrgeizigsten Vertreter werden sich zu wahrhaft eigenständigen
Individuen entwickeln.

Zunächst werden zwar nur einige wenige die vollständige finanzielle
Souveränität erlangen, aber das schmälert keineswegs die Vorzüge der
finanziellen Unabhängigkeit. Die Tatsache, dass nicht jeder das gleiche
Vermögen ansammelt, bedeutet nicht, dass der Versuch, reich zu werden,
vergeblich oder sinnlos ist. Auf jeden Milliardär kommen 25.000
Millionäre. Wenn Sie Millionär und kein Milliardär sind, sind Sie
deswegen nicht arm. Auch in Zukunft wird einer der Maßstäbe Ihres
finanziellen Erfolgs nicht nur darin bestehen, wie viele Nullen Sie zu
Ihrem Nettovermögen hinzufügen können, sondern darin, ob Sie Ihre
Geschäfte so strukturieren können, dass Sie vollständige individuelle
Autonomie und Unabhängigkeit erreichen. Je mehr Finesse Sie an den Tag
legen, desto weniger Anstrengung werden Sie benötigen, um die
finanzielle Fluchtgeschwindigkeit zu erreichen. Selbst Personen mit eher
bescheidenen Ressourcen können sich hocharbeiten, wenn der Einfluss der
Politik auf die Weltwirtschaft abnimmt. Eine nie zuvor dagewesene
finanzielle Unabhängigkeit wird in Ihrem Leben oder dem Ihrer Kinder ein
erreichbares Ziel sein.

Auf dem Gipfel der Produktivität gehen diese selbstbestimmten Individuen
miteinander in den Wettstreit und interagieren unter Bedingungen, die an
die Verbindung zwischen den Göttern in der griechischen Mythologie
erinnern. Der schwer greifbare Olymp des kommenden Jahrtausends wird im
Cyberspace liegen - einem Bereich ohne physische Existenz, der dennoch
das Potenzial hat, die größte Wirtschaft der Welt im zweiten Jahrzehnt
dieses Jahrtausends zu entwickeln. Bis 2025 wird die Cyber-Ökonomie
viele Millionen Teilnehmer verzeichnen. Einige davon werden sich mit
einem Vermögen von jeweils über 10 Milliarden Dollar ebenso bereichern
wie Bill Gates. Die Cyber-Armen werden diejenigen sein, die weniger als
200.000 Dollar pro Jahr verdienen. Es wird keine Cyber-Sozialhilfe,
keine Cyber-Steuern und keine Cyber-Regierung geben. Nicht China,
sondern die Cyber-Ökonomie könnte das dominierende Wirtschaftsphänomen
der nächsten dreißig Jahre darstellen.

Die gute Nachricht ist, dass Politiker ebenso wenig Kontrolle,
Unterdrückung und Regulierung des größten Teils des Handels in dieser
neuen Welt ausüben können, wie die Gesetzgeber der antiken griechischen
Stadtstaaten in der Lage waren, den Bart von Zeus zu stutzen. Das ist
eine positive Nachricht für die Reichen und noch bessere Neuigkeit für
die weniger Reichen. Die von der Politik geschaffenen Hindernisse und
Belastungen wirken eher hinderlich auf das Erlangen von Reichtum als auf
das Erhalten desselben. Die Zurückhaltung bei der Anwendung von Gewalt
und die Dezentralisierung der Zuständigkeiten schaffen neue
Möglichkeiten für jeden tatkräftigen und ambitionierten Menschen, vom
Abflauen der politischen Macht zu profitieren. Selbst Konsumenten
staatlicher Dienstleistungen können davon profitieren, wenn Unternehmer
die Vorteile von Wettbewerb weiter fördern. Bislang bedeutete der
Wettbewerb zwischen Gerichtsbarkeiten in der Regel einen Wettbewerb der
Gewalt zur Durchsetzung der Herrschaft einer vorherrschenden Gruppe.
Folglich wurde viel Erfindergeist von Wettbewerb zwischen
Gerichtsbarkeiten in militärische Bestrebungen kanalisiert. Jedoch wird
die Cyberökonomie den Wettbewerb in Bezug auf staatliche
Dienstleistungen unter neuen Bedingungen fördern. Eine Zunahme von
Gerichtsbarkeiten bedeutet mehr Möglichkeiten für das Ausprobieren neuer
Methoden zur Durchsetzung von Verträgen und um die Sicherheit von
Personen und Eigentum auf neue Art und Weise zu garantieren. Die
Befreiung eines großen Teils der Weltwirtschaft von politischer
Kontrolle wird alle verbliebenen Regierungsformen dazu zwingen, unter
Bedingungen zu arbeiten, die stark an die Marktwirtschaft angelehnt
sind. Letztendlich werden sie kaum eine andere Wahl haben, als die
Bevölkerung in den von ihnen betreuten Gebieten eher wie Kunden zu
behandeln und weniger so, wie organisierte Kriminelle die Opfer ihrer
Erpressung behandeln.

\subsection{Jenseits der Politik}\label{jenseits-der-politik}

Was in der Mythologie als die Domäne der Götter galt, wird für den
Einzelnen zur erreichbaren Option - ein Leben jenseits der Macht von
Königen und Ratsherren. Zuerst zu Hunderten, dann zu Tausenden und
schließlich zu Millionen werden Menschen die Ketten der Politik
abstreifen. Dabei werden sie die Natur der Regierungen verändern, den
Raum des Zwanges reduzieren und den Bereich der privaten Kontrolle über
Ressourcen erweitern.

Das erneute Auftauchen des selbstbestimmten Individuums wird einmal mehr
die geheimnisvolle, prophetische Macht des Mythos unterstreichen. Die
frühen Agrargesellschaften hatten nur wenig Kenntnis von den
Naturgesetzen und nahmen an, dass „Kräfte, die wir heute als
übernatürlich bezeichnen würden``, weit verbreitet seien. Diese Kräfte
wurden teils von Menschen, teils von „leibhaftigen menschlichen
Göttern`` genutzt, die menschenähnlich aussahen und auf eine Weise mit
ihnen interagierten, die Sir James George Frazer in „The Golden Bough``
als „große Demokratie`` beschrieb.\footnote{James George Frazer,
  \emph{The Golden Bough: A Study in Magic and Religion} (New York:
  Macmillan, 1951), S. 105.}

Als sich die Menschen der Antike ausmalten, dass die Nachkommen des Zeus
mitten unter ihnen weilten, war ihr Glaube an Magie stark. Zusammen mit
anderen primitiven Agrargesellschaften teilten sie eine tiefe Ehrfurcht
vor der Natur sowie die abergläubische Annahme, dass natürliche
Phänomene durch individuelle Willenskraft, also durch Magie, beeinflusst
werden konnten. In diesem Kontext hatte ihr Verständnis von der Natur
und ihren Göttern nichts an sich, was selbstbewusst und prophetisch
genannt werden könnte. Es lag weit außerhalb ihrer Vorstellungskraft,
die zukünftige Mikrotechnologie zu erahnen. Sie konnten sich nicht
ausmalen, wie diese Tausende von Jahren später die individuelle
Produktivität verändern würde. Sie hätten sicher nicht vorhersehen
können, wie sie das Gleichgewicht von Macht und Effizienz verschieben
und damit die Art und Weise revolutionieren würde, wie Reichtum
geschaffen und erhalten wird. Doch das, was sie sich ausdachten, als sie
ihre Mythen webten, hallt auf merkwürdige Weise in der Welt nach, die
Sie aller Wahrscheinlichkeit nach erleben werden.

\subsection{Alt.Abrakadabra}\label{alt.abrakadabra}

Das „Abrakadabra`` magischer Beschwörungen zum Beispiel erinnert
merkwürdigerweise an ein Passwort, das für den Zugriff auf einen
Computer benötigt wird. In gewisser Hinsicht hat die
Hochgeschwindigkeitsberechnung bereits ermöglicht, die Magie des
Flaschengeistes nachzuempfinden. Frühe Generationen dieser „digitalen
Diener`` gehorchen bereits den Anweisungen derjenigen, die die Computer
steuern, in denen sie eingeschlossen sind - genauso wie Flaschengeister
in versiegelten Wunderlampen. Die virtuelle Realität der
Informationstechnologie wird das Spektrum menschlicher Wünsche erweitern
und nahezu jede erdenkliche Vorstellung zu einer scheinbaren Realität
machen. Telepräsenz wird Lebewesen die Fähigkeit verleihen, Entfernungen
mit übernatürlicher Geschwindigkeit zu überbrücken und Ereignisse aus
der Ferne zu verfolgen - ähnlich wie es den Göttern Hermes und Apollo in
der griechischen Mythologie zugeschrieben wurde. Die selbstbestimmten
Individuen des Informationszeitalters werden, ähnlich wie die Götter der
antiken und primitiven Mythen, mit der Zeit eine Art „diplomatische
Immunität`` gegenüber den meisten politischen Problemen genießen, die
sterbliche Menschen in den meisten Zeiten und an den meisten Orten
heimsuchen.

Das neue selbstbestimmte Individuum wird ähnlich agieren wie die
Gottheiten aus den Mythen. Im gleichen physischen Umfeld wie der
normale, unterworfene Bürger, allerdings in einem eigenen Bereich der
Politik. Mit seinen bedeutend größeren Ressourcen und der Unabhängigkeit
von vielen Formen von Zwang hat das selbstbestimmte Individuum die
Macht, zum neuen Jahrtausend Regierungen umzubauen und Volkswirtschaften
neu einzurichten. Es ist fast unvorstellbar, welche weitreichenden
Auswirkungen dieser Wandel haben wird.

\subsection{Genius und Nemesis}\label{genius-und-nemesis}

Für alle, die menschlichen Ehrgeiz und Erfolg schätzen, wird das
Informationszeitalter eine Belohnung bereitstellen. Das ist zweifellos
die beste Neuigkeit seit vielen Generationen. Aber es gibt auch einen
Wermutstropfen: Mit dem Siegeszug der individuellen Autonomie und der
echten Chancengleichheit auf Leistungsbasis wird ein neues
Gesellschaftsmodell entstehen, welches enorme Belohnungen für Leistung
und größte individuelle Freiheit mit sich bringt. Damit wird jeder
Einzelne viel mehr Eigenverantwortung tragen müssen, als es zur Zeit der
Industrialisierung der Fall war. Zusätzlich wird dieser Wandel den
ungerechtfertigten Vorteil des Lebensstandards, den die Bewohner der
fortschrittlichen Industrienationen im gesamten 20. Jahrhundert genossen
haben, reduzieren. Während diese Zeilen entstehen, verfügen die obersten
15 Prozent der Weltbevölkerung über ein durchschnittliches jährliches
Pro-Kopf-Einkommen von 21.000 Dollar. Die übrigen 85 Prozent kommen
durchschnittlich auf nur 1.000 Dollar jährlich. Unter den neuen
Bedingungen des Informationszeitalters wird sich diese enorme, aus der
Vergangenheit angehäufte Vorteilslage zwangsläufig auflösen.

Dies wird dazu führen, dass die Fähigkeit der Nationalstaaten, Einkommen
in großem Umfang umzuverteilen, zusammenbricht. Die
Informationstechnologie nährt einen dramatisch verstärkten Wettbewerb
zwischen den Rechtsordnungen. Wenn Technologie zunehmend mobil wird und
Transaktionen vermehrt im Cyberspace stattfinden, werden Regierungen
kaum mehr in der Lage sein, für ihre Dienstleistungen mehr zu verlangen,
als sie den Zahlenden wert sind. Jeder, der über einen Laptop und eine
Satellitenverbindung verfügt, wird in der Lage sein, fast jeden
Informationshandel an jedem beliebigen Ort abzuwickeln. Dazu zählen auch
fast alle Finanztransaktionen im Wert von mehreren Billionen Dollar.

Das heißt, dass man künftig nicht mehr dazu gezwungen ist, in einem Land
mit hoher Steuerlast zu leben, um ein hohes Einkommen zu erzielen. In
einer Zukunft, in der der größte Teil des Wohlstands überall verdient
und auch ausgegeben werden kann, werden Regierungen, die versuchen,
überhöhte Preise für den Wohnsitz einzufordern, ihre besten Steuerzahler
verlieren. Sollten unsere Überlegungen zutreffen, und davon sind wir
überzeugt, wird der Nationalstaat, wie wir ihn heute kennen, in seiner
jetzigen Form nicht bestehen bleiben.

\section{DAS ENDE DER
NATIONALSTAATEN}\label{das-ende-der-nationalstaaten}

Veränderungen, die die Dominanz etablierter Institutionen untergraben,
können sowohl beängstigend als auch bedrohlich sein. Genauso wie
Monarchen, Fürsten, Päpste und Machthaber zu Beginn der Neuzeit
erbarmungslos um den Erhalt ihrer gewohnten Privilegien kämpften, so
setzen auch heute Regierungen oft verdeckt und willkürlich Gewalt ein,
in dem Versuch, den Lauf der Zeit aufzuhalten. Durch die technologischen
Herausforderungen geschwächt, behandelt der Staat die autonomen
Individuen, seine ehemaligen Bürger, mit der gleichen Skrupellosigkeit
und Diplomatie, die er bisher gegenüber anderen Regierungen gezeigt hat.
Der Beginn dieser neuen Phase der Geschichte wurde am 20. August 1998
eingeläutet, als die Vereinigten Staaten Tomahawk-Marschflugkörper im
Wert von etwa 200 Millionen Dollar gegen Ziele abfeuerten, die angeblich
mit dem saudischen Exil-Millionär Osama bin Laden verknüpft waren. Bin
Laden war die erste Person in der Geschichte, deren Satellitentelefon
Ziel von Marschflugkörpern wurde. Gleichzeitig zerstörte die USA eine
Arzneimittelfabrik in Khartum, Sudan, angeblich als Vergeltung gegen Bin
Laden. Das Auftreten von Bin Laden als größter Feind der Vereinigten
Staaten markiert einen signifikanten Wechsel in der Kriegsdynamik. Eine
einzelne Person, die allerdings über Hunderte von Millionen Dollar
verfügt, kann nun als glaubhafte Bedrohung für die größte Militärmacht
des Industriezeitalters angesehen werden. In Aussagen, die an die
Propaganda aus Zeiten des Kalten Kriegs gegen die Sowjetunion erinnern,
präsentierten der US-Präsident und seine nationalen Sicherheitsberater
Bin Laden, eine Privatperson, als transnationalen Terroristen und
Hauptgegner der Vereinigten Staaten.

Die gleiche militärische Logik, die Osama bin Laden zum Hauptgegner der
USA machte, wird sich auch innerhalb der Beziehungen zwischen
Regierungen und ihren Bürgern etablieren. Immer strengere
Durchsetzungsmethoden werden die logische Konsequenz aus dem Entstehen
einer neuen Art von Verhandlungen zwischen Regierungen und
Einzelpersonen sein. Technologie wird die Einzelpersonen
selbstbestimmter machen als jemals zuvor. Und genau so werden sie auch
behandelt werden. Manchmal gewaltsam als Feinde, manchmal als
gleichberechtigte Verhandlungspartner, manchmal als Verbündete. Sie
können noch so rücksichtslos vorgehen, insbesondere während der
Übergangszeit wird es ihnen jedoch wenig Nutzen bringen, das Finanzamt
mit der CIA zu fusionieren. Sie werden zunehmend gezwungen sein, mit
autonom handelnden Individuen zu verhandeln, deren Ressourcen nicht mehr
so einfach kontrollierbar sind.

Die Informationsrevolution zieht Veränderungen nach sich, die nicht nur
zu einer finanziellen Krise für Regierungen führen, sondern auch zum
Zerfall großer Strukturen. Im zwanzigsten Jahrhundert sahen wir bereits
den Untergang von vierzehn Imperien. Dieser Prozess ist Teil einer
Entwicklung, die letztlich zur Auflösung des Nationalstaates selbst
führen wird. Staaten werden sich der zunehmenden Autonomie des Einzelnen
anpassen müssen. So wird die Steuerkapazität voraussichtlich um 50 bis
70 Prozent sinken. Dies dürfte kleinere Rechtsgebiete erfolgreicher
machen. Die Aufgabe, wettbewerbsfähige Bedingungen zu schaffen, um
kompetente Menschen und ihr Vermögen anzulocken, wird in Enklaven
leichter zu bewältigen sein als auf kontinentaler Ebene.

Wir glauben, dass mit dem fortschreitenden Zerfall des modernen
Nationalstaates die Barbaren der Neuzeit immer mehr die Macht im
Hintergrund übernehmen werden. Gruppierungen wie die russische Mafia,
die die Überreste der ehemaligen Sowjetunion aufsammeln, andere
ethnische Verbrechersyndikate, Nomenklaturen\footnote{Nomenklaturen sind
  die verwurzelten Eliten, die die frühere Sowjetunion und andere
  staatlich gelenkte Volkswirtschaften beherrschten.}, Drogenbosse und
abtrünnige Geheimdienste werden ihre eigenen Gesetze schaffen. Das tun
sie bereits. Viel mehr als allgemein bekannt, haben diese modernen
Barbaren bereits die Strukturen des Nationalstaates unterwandert, ohne
sein Erscheinungsbild signifikant zu verändern. Sie sind Mikroparasiten,
die sich von einem sterbenden System nähren. Diese Gruppen sind ebenso
gewalttätig und skrupellos wie ein Staat im Krieg, wenden jedoch
staatliche Techniken auf kleinerer Ebene an. Ihr wachsender Einfluss und
ihre Macht sind Teil der Verkleinerung des politischen Rahmens. Die
Mikroprozessortechnik reduziert die Größe, die diese Gruppen erreichen
müssen, um in der Anwendung und Kontrolle von Gewalt effektiv zu sein.
Mit dem Fortschreiten dieser technologischen Revolution wird räuberische
Gewalt immer mehr außerhalb der zentralen Kontrolle organisiert werden.
Auch die Bemühungen zur Eindämmung von Gewalt werden sich auf eine Weise
entwickeln, die mehr von der Effizienz als von der Größe der Macht
abhängt.

\subsection{Geschichte in umgekehrter
Reihenfolge}\label{geschichte-in-umgekehrter-reihenfolge}

Der Vorgang, wie der Nationalstaat in den letzten fünf Jahrhunderten
gewachsen ist, wird durch die neue Dynamik des Informationszeitalters
umgekehrt. Lokale Machtzentren treten erneut in den Vordergrund, während
sich der Staatsapparat in fragmentierte, sich überschneidende
Souveränitätsgebiete auflöst.\footnote{Für mehr Einzelheiten über
  fragmentierte Souveränitäten als Vorläufer und Alternative zum
  Nationalstaat, siehe Charles Tilly, Coercion, Capital and European
  States AD 990-1992 (Oxford: Blackwell, 1993).} Die zunehmende Macht
der organisierten Kriminalität ist nur ein Beispiel für diese
Entwicklung. Multinationale Unternehmen sehen sich bereits genötigt,
alle Aufgaben bis auf die essentiellen an Subunternehmen auszulagern.
Einige Konglomerate wie AT\&T, Unisys und ITT haben sich in mehrere
Firmen aufgespalten, um rentabler wirtschaften zu können. Der
Nationalstaat wird sich, ähnlich wie ein schwerfälliges Konglomerat,
auflösen - vermutlich jedoch erst, wenn er durch Finanzkrisen dazu
gezwungen ist.

Nicht nur die Machtverhältnisse weltweit wandeln sich, sondern auch die
Arbeitswelt unterliegt einem starken Wandel. Dies bedeutet wiederum,
dass sich unweigerlich die Art und Weise ändern wird, wie Unternehmen
arbeiten. Das Konzept des „virtuellen Unternehmens`` ist ein Zeichen für
diese einschneidenden Veränderungen, die durch sinkende Informations-
und Transaktionskosten begünstigt wurden. Wir analysieren, welche
Auswirkungen die Informationsrevolution auf die Auflösung von
Unternehmen und das Ende des „sicheren Arbeitsplatzes`` hat. Im
Informationszeitalter wird aus dem „Arbeitsplatz`` eine
Aufgabenstellung, die es zu lösen gilt, statt einer Position, die man
einfach „hat``. Die Mikroprozessortechnik hat ganz neue Perspektiven für
wirtschaftliche Aktivitäten jenseits von territorialen Grenzen
erschlossen. Diese Überwindung von Grenzen und Territorien könnte
womöglich die revolutionärste Entwicklung sein, seit Adam und Eva
aufgrund des Urteils ihres Schöpfers das Paradies verlassen mussten: „Im
Schweiße deines Angesichts sollst du dein Brot essen.`` Während die
Technologie die Werkzeuge, die wir verwenden, revolutioniert,
hinterlässt sie auch unsere Gesetze als veraltet, modelliert unsere
Moral neu und verändert unsere Wahrnehmungen. In diesem Buch wird
erklärt, wie dies geschieht.

Die Mikroprozessortechnik und die rapide Weiterentwicklung der
Kommunikationsmittel ermöglichen es den Menschen schon heute, ihren
Arbeitsort frei zu wählen. Transaktionen über das Internet oder das
World Wide Web können verschlüsselt und in naher Zukunft nahezu
unentdeckt von Steuerbehörden durchgeführt werden. Steuerfreies Geld
vermehrt sich bereits jetzt im Ausland deutlich schneller als
inländische Gelder, die weiterhin der hohen Steuerbelastung des
Nationalstaates aus dem 20. Jahrhundert ausgesetzt sind. Nach der
Jahrtausendwende wird sich ein Großteil des Welthandels in den
Cyberspace verlagern, eine Region, über die Regierungen nicht mehr
Kontrolle ausüben werden, als sie es über den Meeresboden oder die
äußeren Planeten tun. Im Cyberspace werden die Drohungen physischer
Gewalt, die seit jeher Grundpfeiler der Politik sind, der Vergangenheit
angehören. Im Cyberspace begegnen sich die Milden und die Mächtigen auf
einer Ebene. Der Cyberspace ist die ultimative Offshore-Rechtsordnung:
Eine steuerfreie Wirtschaft. Bermuda im Himmel, geschmückt mit
Diamanten.

Wenn dieses größte Steuerparadies überhaupt erst einmal auf die
Wirtschaft losgelassen wird, können alle Fonds im Endeffekt als
Offshore-Fonds auf Wunsch ihrer Besitzer betrachtet werden. Die
Auswirkungen werden enorm sein. Der Staat hat sich darauf eingestellt,
seine Steuerzahler zu behandeln wie ein Bauer seine Kühe: er hält sie
auf einer Weide, um sie zu melken. Doch bald werden diese Kühe Flügel
entwickeln.

\subsection{Die Rache des
Nationalstaats}\label{die-rache-des-nationalstaats}

Gleich einem aufgebrachten Bauern wird der Staat zweifellos anfangs zu
verzweifelten Mitteln greifen, um seine abwandernde Herde zu
kontrollieren und einzugrenzen. Unauffällige und sogar gewaltsame
Maßnahmen werden angewandt, um den Zugang zu befreienden Technologien zu
begrenzen. Diese Mittel werden jedoch, wenn überhaupt, nur vorübergehend
erfolgreich sein. Der Nationalstaat des zwanzigsten Jahrhunderts, mit
all seiner Anmaßung, wird verhungern, sobald seine Steuereinnahmen
zurückgehen.

Wenn der Staat es nicht schafft, seine Ausgaben durch höhere
Steuereinkünfte zu decken, greift er auf andere, verzweifeltere
Maßnahmen zurück. Eine solche Maßnahme besteht darin, Geld zu drucken.
Regierungen haben sich daran gewöhnt, ein Monopol auf die eigene Währung
zu besitzen, die sie nach Belieben abwerten können. Diese willkürliche
Inflation war ein markantes Kennzeichen der Geldpolitiken aller Länder
im 20. Jahrhundert. Selbst die Deutsche Mark, die stärkste nationale
Währung der Nachkriegszeit, verlor von 1949 bis Ende Juni 1995 71
Prozent ihres Wertes. Im gleichen Zeitraum verlor der US-Dollar 84
Prozent seines Wertes.\footnote{Der deutsche GPI-Index lag am 31.
  Dezember 1948 bei 33,20 und am 30. Juni 1995 bei 112,90, was einer
  durchschnittlichen jährlichen Abwertung von 2,7\% entspricht. Der VPI
  der USA lag am 31. Dezember 1948 bei 24 und am 30. Juni 1995 bei
  152,50. Die kumulierte Inflation in den USA betrug in diesem Zeitraum
  635\%.} Diese Inflation hatte den gleichen Effekt wie eine Steuer auf
jeden, der Geld besitzt. Wie wir später erörtern werden, wird die
Inflation als mögliche Einnahmequelle durch das Aufkommen von digitalem
Geld weitgehend ausgeschaltet. Neue Technologien ermöglichen es
Vermögensinhabern, nationale Monopole zu umgehen, die in der Neuzeit das
Geld herausgegeben und reguliert haben. Die Finanzkrisen, die 1997 und
1998 Asien, Russland und andere aufstrebende Länder traf, zeigen, dass
nationale Währungen und nationale Bonitätseinstufungen nicht zur
reibungslosen Funktion des globalen Wirtschaftssystems beitragen.
Insbesondere die Tatsache, dass die Souveränitätsbedingungen
vorschreiben, dass alle Transaktionen innerhalb eines Landes in der
nationalen Währung abgewickelt werden müssen, schafft Anfälligkeit für
Fehlentscheidungen der Zentralbanker und Angriffe von Spekulanten, die
in einem Land nach dem anderen zu deflationären Krisen geführt haben. Im
Informationszeitalter wird es den Menschen ermöglicht, Cyberwährungen zu
nutzen und so ihre wirtschaftliche Unabhängigkeit zu erklären. Wenn
jeder in der Lage ist, seine eigene Geldpolitik über das Internet zu
betreiben, wird die Kontrolle des Staates über die Druckmaschinen des
Industriezeitalters an Bedeutung verlieren. Ihre bisherige Bedeutung für
die Kontrolle des globalen Reichtums wird durch mathematische
Algorithmen, die keine physische Existenz haben, übertroffen. Im neuen
Jahrtausend wird digitales Geld, das von privaten Märkten kontrolliert
wird, das von den Regierungen ausgegebene Papiergeld ersetzen. Nur die
Armen werden Opfer der Inflation und des sich anschließenden
Zusammenbruchs in die Deflation, die eine Folge des künstlichen Hebels
ist, den das Fiat-Geld der Wirtschaft gewährt.

Da ihnen herkömmliche Methoden der Besteuerung und Inflationierung nicht
zur Verfügung stehen, werden Regierungen, selbst in traditionell
bürgerlichen Ländern, unangenehm auffallen. Wenn die Einkommensteuer
nicht mehr erhoben werden kann, erfahren ältere und repressivere
Methoden der Steuereintreibung eine Wiedergeburt. Die extreme Form der
Kapitalertragssteuer -- faktisch oder durch offene Geiselnahme - wird
von Regierungen ins Spiel gebracht, die verzweifelt versuchen, das
Fliehen des Reichtums aus ihren Griffen zu verhindern. Wer Pech hat,
wird herausgegriffen und auf nahezu mittelalterliche Art und Weise
gefügig gemacht. Unternehmen, die Dienstleistungen anbieten, die die
individuelle Autonomie fördern, werden infiltriert, sabotiert und
gestört. Die willkürliche Beschlagnahmung von Eigentum, die in den
Vereinigten Staaten bereits üblich ist und dort wöchentlich
fünftausendmal vorkommt, wird noch größere Verbreitung finden.
Regierungen werden Menschenrechte verletzen, die freie
Informationsverbreitung zensieren, nützliche Technologien sabotieren und
Schlimmeres. Aus den gleichen Gründen, aus denen die im Untergang
begriffene Sowjetunion erfolglos versucht hat, den Zugang zu Computern
und Kopiermaschinen einzuschränken, werden westliche Regierungen
versuchen, die Cyberökonomie mit totalitären Methoden zu unterdrücken.

\section{DIE RÜCKKEHR DER LUDDITEN}\label{die-ruxfcckkehr-der-ludditen}

Diese Methoden könnten sich bei bestimmten Bevölkerungsgruppen als
beliebt erweisen. Denn was für viele wie eine gute Nachricht von
individueller Befreiung und Autonomie klingt, könnte von jenen, die
nicht zur intellektuellen Elite gehören, als schlechte Nachricht
aufgefasst werden. Der größte Widerstand könnte von durchschnittlich
begabten Menschen in den aktuell wohlhabenden Ländern erwartet werden.
Sie sind es vor allem, die die Informationstechnologie als eine
Bedrohung ihres Lebensstils wahrnehmen könnten. Die Nutznießer des
organisierten Zwangs, einschließlich der Millionen Menschen, die von
staatlich umverteilten Einkommen leben, könnten das neu gewonnene
Freiheitsstreben selbstbestimmter Individuen als störend empfinden. Ihre
Missbilligung wird das alte Sprichwort verdeutlichen, das besagt: „Wo du
stehst, wird dadurch bestimmt, wo du sitzt``.

\begin{quote}
„Manchmal habe ich mich gefragt, wie ich so tiefe Trauer um das
Schicksal einiger weniger Männer empfinden konnte, die ich nie
persönlich kennenlernte und die Hunderte von Kilometern entfernt in
einem Stadion gegen eine andere Gruppe ebenfalls unbekannter Menschen
spielten. Die Antwort ist einfach. Ich liebte meine Mannschaft. Trotz
des Risikos war diese emotionale Investition ihren Preis wert. Der Sport
brachte mein Blut in Wallung, erregte mich, brachte mein Herz zum
Klopfen. Ich genoss den Nervenkitzel, wenn es wirklich um etwas ging.
Das Leben fühlte sich während eines Wettkampfs einfach intensiver an.``
- Craig Lambert
\end{quote}

Es wäre allerdings irreführend, sämtliche negativen Emotionen, die
während der anstehenden Übergangskrise aufkommen werden, lediglich dem
Wunsch zuzuschreiben, auf Kosten anderer zu leben. Das Ausmaß ist
deutlich größer. Bei genauerer Betrachtung der Beschaffenheit
menschlicher Gesellschaften lässt sich erahnen, dass die bevorstehende
Reaktion der Neu-Ludditen durchaus eine missverstandene moralische
Komponente aufweisen wird. Man könnte das als bloßen Wunsch beschreiben,
der mit einem moralischen Toupet versehen ist. In diesem Zusammenhang
beleuchten wir sowohl die moralischen als auch die moralisierenden
Aspekte der Übergangskrise. Egoistisches Streben ist weit weniger
motivierend als selbstgerechter Zorn. Obwohl die bürgerlichen Mythen des
20. Jahrhunderts immer mehr an Bedeutung einbüßen, haben sie nach wie
vor ihre Anhänger. Jeder, der im 20. Jahrhundert aufwuchs, wurde mit den
Aufgaben und Pflichten eines Bürgers dieser Zeit vertraut gemacht. Die
verbliebenen moralischen Imperative der Industriegesellschaft werden
zumindest einige neo-ludditische Angriffe auf Informationstechnologien
provozieren.

In diesem Kontext drückt die zu erwartende Gewalt zumindest teilweise
das aus, was wir als „moralischen Anachronismus`` bezeichnen. Dies
bedeutet, dass moralische Regeln, die in einer bestimmten
Wirtschaftsphase entstanden sind, auf Situationen angewendet werden, die
aus einer anderen Phase stammen. Jede Epoche der Gesellschaft benötigt
ihre eigenen moralischen Regeln, um den Individuen dabei zu helfen, die
typischen Anreizfallen, die die Entscheidungen prägen, die sie in ihrem
jeweiligen Lebensstil treffen müssen, zu meistern. Genau wie eine
bäuerliche Gesellschaft nicht nach den Moralvorstellungen einer
nomadischen Eskimogruppe leben könnte, kann die Informationsgesellschaft
nicht den moralischen Imperativen gerecht werden, die geschaffen wurden,
um den Erfolg eines militanten Industriestaates des 20. Jahrhunderts zu
fördern. Wir werden erklären, warum das so ist.

In den kommenden Jahren wird ein moralischer Anachronismus in den
westlichen Kernländern ähnliche Erscheinungsformen zeigen, wie wir sie
über die letzten fünf Jahrhunderte in den Randgebieten beobachten
konnten. Westliche Kolonisatoren und militärische Expeditionen lösten
solche Krisen aus, wenn sie auf einheimische Jäger- und Sammlergruppen
oder auf Gesellschaften stießen, deren Lebensweise noch
landwirtschaftlich geprägt war. Dabei sorgte die Einführung neuer
Technologien in einem anachronistischen Kontext für Verwirrung und
moralische Krisen. Der Erfolg christlicher Missionare bei der Bekehrung
von Millionen Einheimischen lässt sich zu einem erheblichen Teil auf
diese lokalen Krisen zurückführen, die durch das abrupte Aufkommen neuer
Machtstrukturen ausgelöst wurden. Solche Zusammenstöße ereigneten sich
vom 16. bis zu den ersten Jahrzehnten des 20. Jahrhunderts immer wieder.
Ähnliche Konflikte erwarten wir zu Beginn des neuen Jahrtausends, wenn
informationstechnologiegesteuerte Gesellschaften die industriell
orientierten ablösen.

\subsection{Die Sehnsucht nach Zwang}\label{die-sehnsucht-nach-zwang}

Die Idee des Erstarkens des selbstbestimmten Individuums wird sicher
nicht von allen als aufregende neue Ära in der Geschichte gesehen
werden, auch nicht von denen, die am meisten davon profitieren. Jeder
wird seine Bedenken haben. Und viele werden jede Neuerung, die auf
Kosten der territorialen Nationalstaaten geht, ablehnen. Es ist ein
menschlicher Instinkt, dass radikale Änderungen oft als dramatischer
Rückschritt wahrgenommen werden. Vor fünfhundert Jahren hätten die
Höflinge, die sich um den Herzog von Burgund geschart hatten, die neuen
Entwicklungen, die den Feudalismus untergruben, sicherlich als Übel
betrachtet. Sie waren überzeugt, dass sich die Welt in einer
beschleunigten Abwärtsspirale befand, genau in dem Moment, als spätere
Historiker eine Explosion des menschlichen Potenzials in der Renaissance
erkannten. Auf ähnliche Weise mag das, was aus der Sicht des nächsten
Jahrtausends einmal als neue Renaissance gesehen werden könnte, für die
erschöpften Seelen des zwanzigsten Jahrhunderts angsteinflößend
erscheinen.

Es ist sehr wahrscheinlich, dass manche, die sich durch neue
Entwicklungen angegriffen fühlen, sowie viele, die sie benachteiligen,
mit Unbehagen reagieren werden. Ihre Sehnsucht nach Zwang wird sich
wahrscheinlich in Gewalttaten ausdrücken. Begegnungen mit diesen neuen
„Ludditen`` könnten den Übergang zu radikalen, neuen Formen der sozialen
Organisation zu einer schlechten Nachricht für alle machen. Gehen Sie
lieber in Deckung. Da die Geschwindigkeit des Wandels die moralische und
ökonomische Anpassungsfähigkeit vieler Generationen überfordert, ist mit
heftigem und entrüstetem Widerstand gegen die informationelle Revolution
zu rechnen, trotz ihres enormen Potenzials, die Zukunft zu entfesseln.

Sie müssen diese Unannehmlichkeiten begreifen und sich darauf
einstellen. Eine Übergangskrise steht uns bevor. Deflationsschocks,
vergleichbar mit der asiatischen Infektion, die 1997 und 1998 von
Ostasien aus auf Russland und andere Schwellenländer übergriff, werden
sporadisch auftreten, wenn sich die veralteten nationalen und
internationalen Institutionen, die aus dem Industriezeitalter übrig
geblieben sind, als untauglich für die Herausforderungen der neuen,
dezentralisierten, transnationalen Wirtschaft erweisen. Die neuen
Informations- und Kommunikationstechnologien sind eine größere Bedrohung
für den modernen Staat als jede politische Herausforderung seiner
Vorherrschaft seit den Zeiten der Seefahrt von Kolumbus. Das ist
bedeutend, da die Machthaber selten friedlich auf Entwicklungen reagiert
haben, die ihre Autorität infrage stellen. Das wird auch jetzt nicht der
Fall sein.

Der Konflikt zwischen Neuem und Altem wird die ersten Jahre des neuen
Jahrtausends bestimmen. Unsere Prognose sieht sowohl große Risiken als
auch große Chancen voraus, sowie eine Zeit, in der die Zivilgesellschaft
in einigen Bereichen stark schrumpft, in anderen hingegen eine nie da
gewesene Ausdehnung erfährt. Autonome Individuen und bankrotte,
verzweifelte Regierungen werden sich zunehmend über einen neuen Graben
hinweg gegenüberstehen. Wir rechnen mit einer radikalen Umstrukturierung
des Souveränitätsbegriffs und dem faktischen Ende der Politik, bis
dieser Übergang abgeschlossen ist. Aus unausweichlichen Gründen, die wir
in diesem Buch ausführlich besprechen werden, wird die
Informationstechnologie die Fähigkeit des Staates untergraben, mehr für
seine Dienstleistungen zu verlangen, als diese für diejenigen, die sie
finanzieren, tatsächlich wert sind.

\begin{quote}
„Die Regierungen werden sich mit der Frage auseinandersetzen müssen, was
Souveränität eigentlich bedeutet.`` - Robert Martin
\end{quote}

\subsection{Souveränität durch
Märkte}\label{souveruxe4nituxe4t-durch-muxe4rkte}

In einem Umfang, den man sich vor einem Jahrzehnt kaum hätte ausmalen
können, gewinnen Individuen durch die Mechanismen des Marktes zunehmend
Autonomie gegenüber territorial gebundenen Nationalstaaten. Alle
Nationalstaaten sehen sich dem Bankrott und der raschen Erosion ihrer
Autorität gegenüber. So mächtig sie auch sein mögen, die Macht, die
ihnen bleibt, ist die Macht zu zerstören, nicht zu befehlen. Ihre
Interkontinentalraketen und Flugzeugträger sind bereits Relikte, so
beeindruckend und nutzlos wie die letzten Schlachtrösser des
Feudalismus.

Die Informationstechnologie hat die Kapazität, die Märkte dramatisch zu
erweitern, indem sie die Art und Weise, wie Vermögenswerte generiert und
geschützt werden, revolutioniert. Dies ist zweifellos bahnbrechend und
dürfte für die Industriegesellschaft umwälzender sein, als es das
Aufkommen von Schießpulver für die feudale Agrarwelt war. Die
Verwandlung, die mit dem Jahr 2000 einhergeht, kündigt die
Kommerzialisierung von Souveränität und den Untergang der Politik an, so
wie das Schießpulver einst das Ende des auf Eiden basierenden
Feudalismus bedeutete. Die Staatsbürgerschaft könnte denselben Pfad wie
das Rittertum beschreiten.

Wir sind davon überzeugt, dass das Zeitalter persönlicher
wirtschaftlicher Souveränität beginnt. Genau wie Stahlwerke,
Telefongesellschaften, Bergwerke und Eisenbahnen, die einst staatlich
betrieben wurden, weltweit rasch privatisiert wurden, stehen Sie kurz
davor, die ultimative Form der Privatisierung zu erleben - die
umfassende Entnationalisierung des Individuums. Das selbstbestimmte
Individuum des neuen Jahrtausends wird kein Wirtschaftssubjekt des
Staates mehr darstellen, keine Position in der Bilanz des
Finanzministeriums. Nach dem Übergang ins neue Jahrtausend werden die
entnationalisierten Bürger gar keine „Bürger`` mehr sein, sondern
Kunden.

\subsection{Reichweite übertrifft
Grenzen}\label{reichweite-uxfcbertrifft-grenzen}

Durch die Kommerzialisierung der Souveränität könnten die Konditionen
von Staatsbürgerschaften innerhalb von Nationalstaaten veraltet sein,
ähnlich wie Rittereide nach dem Untergang des Feudalismus. Statt als
steuerzahlende Bürger mit einem mächtigen Staat zu interagieren, werden
die souveränen Individuen des 21. Jahrhunderts Kunden von Regierungen
sein, die aus einem „neuen logischen Raum`` heraus agieren. Sie könnten
die geringstmögliche Regierungsführung aushandeln und dafür einen
vertraglichen Preis zahlen. Die Regierungen des Informationszeitalters
könnten sich nach anderen Prinzipien organisieren als denen, an die wir
in den vergangenen Jahrhunderten gewöhnt waren. Einige Gerichtsbarkeiten
und Souveränitätsdienstleistungen könnten durch „unterstützendes
Matching`` entstehen, einem System, bei dem affinitätsbezogene Faktoren,
inklusive kommerzieller Affinitäten, die Grundlage für Loyalität
innerhalb virtueller Gerichtsbarkeiten schaffen. In seltenen Fällen
könnten diese neuen Souveränitäten Überreste mittelalterlicher
Organisationen wie dem 900 Jahre alten Souveränen Malteserorden sein.
Der Orden, eine Vereinigung reicher Katholiken mit derzeit 10.000
Mitgliedern, hat ein jährliches Einkommen von mehreren Milliarden Euro,
gibt eigene Pässe, Briefmarken und Geld aus und unterhält volle
diplomatische Beziehungen mit siebzig Ländern. Während wir dies
schreiben, verhandelt der Orden mit der Republik Malta über die Rückgabe
von Fort St.~Angelo. Besitz des Forts könnte den fehlenden territorialen
Aspekt bereitstellen, der es den Rittern ermöglichen würde, als
Souveränität anerkannt zu werden. Die Malteserritter könnten erneut zu
einem souveränen Kleinststaat avancieren, der durch ihre lange
Geschichte sofort legitimiert würde. Es war das Fort St.~Angelo, von dem
aus die Malteserritter die Türken in der großen Schlacht von 1565
schlugen. Tatsächlich beherrschten die Malteserritter von Fort
St.~Angelo aus viele Jahre lang Malta, bis sie 1798 von Napoleon
vertrieben wurden. Falls sie in den kommenden Jahren zurückkehren, wäre
dies ein klares Indiz dafür, dass das moderne Nationalstaatensystem,
eingeführt nach der Französischen Revolution, nur eine Zwischenstation
auf der längeren historischen Strecke gewesen ist, in der es üblich war,
dass zahlreiche Arten von Souveränität parallel existierten.

Ein weiteres, völlig unterschiedliches Modell für eine postmoderne
Souveränität, die auf unterstützendem Matching basiert, ist das
Iridium-Satellitentelefonnetz. Auf den ersten Blick scheint es
sonderbar, einen Mobilfunkdienst als eine Form von Souveränität zu
betrachten. Iridium ist jedoch bereits von internationalen Behörden als
virtuelle Souveränität anerkannt. Vielleicht wissen Sie, dass Iridium
ein globaler Mobilfunkdienst ist, der es seinen Nutzern erlaubt, Anrufe
unter einer einzigen Nummer zu empfangen, unabhängig davon, wo auf der
Welt sie sich gerade befinden, sei es in Featherston, Neuseeland, oder
im bolivianischen Chaco. Damit Anrufe an Iridium-Nutzer weltweit
weitergeleitet werden können, mussten die internationalen
Telekommunikationsbehörden angesichts der Struktur der globalen
Telekommunikation zustimmen, Iridium als virtuelles Land mit einer
eigenen Landesvorwahl zu anerkennen: 8816. Von einem virtuellen Land,
das Satellitentelefonnutzer zusammenfasst, ist es nur ein kleiner
Schritt zur Souveränität für stärker vernetzte, grenzüberschreitende
virtuelle Gemeinschaften im World Wide Web. Die Bandbreite, das bedeutet
die Übertragungskapazität eines Kommunikationsmediums, hat seit der
Erfindung der Transistoren schneller zugenommen als die Rechenleistung.
Wenn dieser Trend zu größerer Bandbreite anhält, was wir für
wahrscheinlich halten, dann wird es nur noch wenige Jahre nach der
Jahrtausendwende dauern, bis die Bandbreite groß genug ist, um das
„Metaverse`` technisch zu ermöglichen, eine alternative Cyberspace-Welt,
die der Science-Fiction-Autor Neal Stephenson konzipiert hat.
Stephensons „Metaverse`` ist eine dichte virtuelle Gemeinschaft mit
ihren eigenen Gesetzen. Wir sind davon überzeugt, dass es unvermeidlich
ist, dass die Teilnehmer der zunehmend florierenden Cyber-Ökonomie eine
Ablösung von den veralteten Gesetzen der Nationalstaaten anstreben und
erlangen werden. Die neuen Cyber-Gemeinschaften werden mindestens ebenso
wohlhabend und durchsetzungsstark sein wie der Souveräne Malteserorden.
Angesichts ihrer weitreichenden Kommunikations- und
Informationskriegsfähigkeiten werden sie sogar noch besser in der Lage
sein, ihre Interessen durchzusetzen. Wir untersuchen auch weitere
Modelle fragmentierter Souveränität, bei denen kleine Gruppen effektiv
die Souveränität schwacher Nationalstaaten pachten und eigene
Wirtschaftsoasen schaffen können, ähnlich den gegenwärtig existierenden
Freihäfen und Freihandelszonen.

Um die Beziehungen zwischen selbstbestimmten Individuen und dem, was von
der Regierung übrigbleibt, abzubilden, wird ein neues moralisches
Vokabular benötigt. Es ist davon auszugehen, dass viele Menschen, die
als „Bürger`` der Nationalstaaten des zwanzigsten Jahrhunderts
aufgewachsen sind, bei diesem Paradigmenwechsel vor den Kopf gestoßen
werden. Das Ende der Nationen und die „Entnationalisierung des
Individuums`` werden etablierte Begriffe wie „gleicher Schutz durch das
Gesetz`` ins Wanken bringen, die auf Machtverhältnissen basieren, die in
naher Zukunft überflüssig sein könnten. Da virtuelle Gemeinschaften mehr
und mehr Zusammenhalt finden, fordern sie, dass ihre Mitglieder nach
ihren eigenen Regeln zur Verantwortung gezogen werden, nicht nach denen
der Nationalstaaten, in denen sie zufälligerweise leben. So wie es in
der Antike und im Mittelalter der Fall war, werden innerhalb desselben
geografischen Gebietes wieder mehrere Rechtssysteme nebeneinander
existieren.

Genauso wie die Bestrebungen, die Macht gepanzerter Ritter zu erhalten,
angesichts der Feuerwaffen zum Scheitern verurteilt waren, so sind auch
moderne Vorstellungen von Nationalismus und Staatsbürgerschaft dazu
bestimmt, durch fortschrittliche Mikrotechnologie untergraben zu werden.
Tatsächlich werden sie letztendlich zur Farce, wie die heiligen
Prinzipien des Feudalismus im 15. Jahrhundert, die im 16. Jahrhundert
der Lächerlichkeit preisgegeben wurden. Die hochgehaltenen
Bürgerschaftskonzepte des 20. Jahrhunderts werden für kommende
Generationen nach der Jahrtausendwende zu komischen Anachronismen. Der
Don Quijote des 21. Jahrhunderts wird kein Ritter sein, der für die
Wiederauferstehung des Feudalismus kämpft, sondern ein Bürokrat in einem
braunen Anzug - ein Steuereintreiber, der nach einem Bürger sucht, den
er prüfen kann.

\section{WIEDERBELEBUNG DER
MARSCHGESETZE}\label{wiederbelebung-der-marschgesetze}

Selten betrachten wir Regierungen als Wettbewerber, zumindest nicht in
einem umfassenden Sinn, sodass unser modernes Verständnis von Umfang und
Potenzial der Souveränität eingeschränkt ist. In der Vergangenheit, als
es für Gruppierungen schwieriger war, ein stabiles Gewaltmonopol mit
Zwang durchzusetzen, war die Macht häufig zersplittert, Zuständigkeiten
überschnitten sich und viele verschiedene Einheiten verkörperten ein
oder mehrere Merkmale der Souveränität. Oft besaß der nominelle
Herrscher kaum Macht. Heutzutage befinden sich Regierungen, die
schwächer sind als Nationalstaaten, in einem ständigen Wettbewerb um die
Durchsetzung eines Gewaltmonopols in einem bestimmten Gebiet. Dieser
Wettbewerb hat zu Anpassungen in Bezug auf die Zurückhaltung von Gewalt
und der Gewinnung von Gefolgschaft geführt, die bald wieder aktuell sein
werden.

Wenn Fürsten und Könige nur geringe Macht hatten und die Ansprüche von
einer oder mehreren Gruppen an einer Grenze kollidierten, passierte es
oft, dass keine der beiden Gruppen die andere dominieren konnte. Im
Mittelalter gab es zahlreiche Grenz- oder sogenannte „Marsch``-Regionen,
an denen Machtbereiche aufeinandertrafen. Solche konfliktbehafteten
Grenzverläufe prägten die europäischen Grenzgebiete über Jahrzehnte,
teils sogar Jahrhunderte hinweg. Man denke nur an die Grenzen zwischen
keltischen und englischen Gebieten in Irland, zwischen Wales und
England, Schottland und England, Italien und Frankreich, Frankreich und
Spanien, Deutschland und den slawischen Randgebieten Mitteleuropas,
sowie zwischen den christlichen Königreichen Spaniens und dem
islamischen Königreich Granada. In diesen Marschregionen entstanden
unterschiedliche institutionelle und rechtliche Strukturen, so wie wir
sie vermutlich im nächsten Jahrtausend wiederfinden werden. Wegen des
konkurrierenden Anspruchs zweier Oberhäupter zahlten die Bewohner der
Marschregionen selten Steuern. Weiterhin hatten sie meist das Privileg
selbst zu entscheiden, welchen Gesetzen sie sich unterwerfen wollten -
eine Wahl, die durch rechtliche Modelle wie „Bekenntnis`` und
„Pfändung`` realisiert wurde, die in der heutigen Zeit weitgehend in
Vergessenheit geraten sind. Wir sind jedoch der Ansicht, dass solche
Konzepte in den Rechtssystemen zukünftiger Informationsgesellschaften
eine bedeutende Rolle einnehmen werden.

\subsection{Jenseits der
Nationalität}\label{jenseits-der-nationalituxe4t}

Bevor der Nationalstaat existierte, war es oft unmöglich, die Anzahl der
Souveränitäten weltweit genau festzustellen, da sie sich auf komplexe
Art und Weise überschnitten und unterschiedliche Formen der Macht
ausübten. Das wird wiederkehren. Die Grenzziehung zwischen den einzelnen
Territorien wurde innerhalb der Strukturen der Nationalstaaten meist
deutlich festgelegt. Doch im Zeitalter der Informationstechnologie
beginnen diese Grenzen wieder zu verschwimmen. Mit dem Anbrechen des
neuen Jahrtausends erleben wir eine erneute Fragmentierung der
Souveränität. Es entstehen neue Strukturen, die einige, aber nicht alle
Eigenschaften aufweisen, die wir traditionell mit Regierungen
assoziieren.

Einige dieser neu entstandenen Organisationen, wie beispielsweise die
Tempelritter und andere religiöse Militärorden des Mittelalters, konnten
beträchtlichen Reichtum und militärische Stärke aufweisen, ohne ein fest
umrissenes Territorium zu kontrollieren. Sie orientierten sich bei ihrer
Organisation nicht an nationalen Kriterien. Die Mitglieder und
Führungspersonen solcher religiösen Gemeinschaften, die im Mittelalter
in einigen Teilen Europas eine souveräne Autorität ausübten, bezogen
ihre Autorität keineswegs aus einer nationalen Identität. Sie gehörten
allen möglichen Ethnien an und betonten, dass ihre Loyalität Gott galt,
nicht den nationalen Gemeinsamkeiten ihrer Mitglieder.

\subsection{Handelsrepubliken im
Cyberspace}\label{handelsrepubliken-im-cyberspace}

Auch die Wiederkehr von Händlerbünden und wohlhabenden Einzelpersonen
mit gewisser souveräner Macht, wie der Hanse im Mittelalter, lässt sich
beobachten. Die Hanse, die auf französischen und flämischen Messen aktiv
war, erweiterte sich auf Kaufleute aus sechzig Städten.\footnote{Janet
  L. Abu-Lughod, \emph{Before European Hegemony: The World System A.D.
  1250-1350} (Oxford: Oxford University Press, 1991), S. 62.} Die
``Hanseatic League``, wie sie im Englischen redundanterweise genannt
wird (die wörtliche Übersetzung wäre „Liga-Liga``), war eine
Organisation germanischer Kaufmannsgilden, die ihren Mitgliedern Schutz
bot und Handelsabkommen aushandelte. Sie hatten in vielen nord- und
osteuropäischen Städten quasistaatliche Befugnisse. Solche Institutionen
werden im neuen Jahrtausend den sterbenden Nationalstaat ersetzen, indem
sie in einer unsicheren Welt Schutz bieten und bei der Durchsetzung von
Verträgen helfen.

Kurz gesagt, diejenigen, die noch immer an den bürgerlichen Mythen der
Industriegesellschaft des 20. Jahrhunderts festhalten, werden von der
Zukunft wohl enttäuscht werden. Hierzu zählen auch die Illusionen der
Sozialdemokratie, die einst die talentiertesten Köpfe inspirierte und
antrieb. Sie beruhen auf der Annahme, dass sich Gesellschaften genau so
entwickeln würden, wie es die Regierungen für sinnvoll halten -
vorzugsweise als Reaktion auf Meinungsumfragen und sorgfältig gezählte
Stimmen. Doch das war nie so zutreffend, wie man es sich vor fünfzig
Jahren vorgestellt hat. Heutzutage wirkt diese Sichtweise wie ein
Anachronismus, ein Relikt der Industrialisierung, ähnlich einem rostigen
Schornstein. Jene bürgerlichen Mythen reflektieren nicht nur eine
Denkweise, die gesellschaftliche Probleme als technisch lösbare
Herausforderungen sieht; sie zeugen auch von einer trügerischen
Sicherheit, dass Ressourcen und Individuen in der Zukunft genauso leicht
politisch manipulierbar sein werden wie im 20. Jahrhundert. Wir zweifeln
daran. Es werden die Kräfte des Marktes und nicht die politischen
Mehrheiten sein, die Gesellschaften dazu zwingen, sich so zu
transformieren, wie es von der breiten Öffentlichkeit weder verstanden
noch gutgeheißen wird. So wird die naive Vorstellung, Geschichte sei
einfach das, was die Menschen sich wünschen, als äußerst irreführend
entlarvt werden.

Es wird von entscheidender Bedeutung sein, dass Sie die Welt mit neuen
Augen betrachten. Das bedeutet, dass Sie Ihre Perspektive ändern und
Dinge, die Sie wahrscheinlich als selbstverständlich angesehen haben,
neu analysieren müssen. Nur so können Sie zu einem neuen Verständnis
gelangen. Sollten Sie in einer Ära, in der herkömmliche Denkmuster an
Realitätsbezug verlieren, nicht über diese hinausdenken können, werden
Sie aller Wahrscheinlichkeit nach Opfer einer wachsenden
Orientierungslosigkeit werden. Orientierungslosigkeit kann zu
Fehlentscheidungen führen, die Ihr Unternehmen, Ihre Investitionen und
Ihren Lebensstil ernsthaft gefährden können.

\begin{quote}
„Das Universum belohnt uns, wenn wir seine Geheimnisse erkennen, während
es uns bestraft, wenn wir sie nicht verstehen. Wenn wir verstehen, wie
das Universum funktioniert, laufen unsere Pläne glatt und wir fühlen uns
wohl. Wenn wir jedoch versuchen zu fliegen, indem wir von einer Klippe
springen und mit den Armen rudern, wird uns das Universum töten.``
\footnote{Jack Cohen and Ian Stewart, \emph{The Collapse of Chaos} (New
  York: Viking, 1994).} - Jack Cohen und Ian Stewart
\end{quote}

\subsection{Neue Sichtweisen}\label{neue-sichtweisen}

Um sich auf die bevorstehende Welt einzustellen, ist es wichtig zu
verstehen, warum sie sich anders entwickeln wird, als die meisten
Experten vorhersagen. Man muss die versteckten Triebkräfte des Wandels
genau analysieren. Wir haben versucht, dies durch einen
unkonventionellen Ansatz zu erreichen, den wir als Studie der
Megapolitik bezeichnen. In zwei unserer früheren Werke, „Blood in the
Streets`` und \emph{The Great Reckoning}, haben wir argumentiert, dass
die hauptsächlichen Veränderungen nicht in politischen Manifesten oder
in den Aussagen verstorbener Ökonomen zu finden sind, sondern in den
weniger offensichtlichen Faktoren, die die Grenzen der Macht
verschwimmen lassen. Minimale Veränderungen in Klima, Topographie,
Mikroben oder Technologie können die Logik der Gewalt verändern. Sie
beeinflussen die Art und Weise, wie Menschen ihr Leben organisieren und
sich schützen.

Beachten Sie bitte, dass unser Ansatz, die ständigen Veränderungen in
der Welt zu begreifen, grundlegend anders ist, als der der meisten
Prognosesteller. Wir beanspruchen nicht, Experten zu sein, in dem Sinne,
dass wir über bestimmte „Themen`` mehr wissen als diejenigen, die ihr
Lebenswerk in die Kultivierung spezialisierter Kenntnisse investiert
haben. Stattdessen betrachten wir die Dinge von außen. Wir sind mit den
Themen vertraut, über die wir Vorhersagen treffen. Dabei geht es uns vor
allem darum aufzuzeigen, wo die Grenzen des Notwendigen gezogen werden.
Wenn diese Grenzen verschoben werden, verändert sich zwangsläufig auch
die Gesellschaft, egal, wie sehr sich die Menschen auch das Gegenteil
wünschen.

Wir sind der Meinung, dass der Schlüssel zum Verständnis der
gesellschaftlichen Entwicklungen im Begreifen der Faktoren liegt, die
Kosten und Nutzen der Anwendung von Gewalt bestimmen. Jede menschliche
Gesellschaft - von Jägergruppen bis zu Imperien -- wird von
megapolitischen Faktoren geprägt, welche die geltenden „Naturrechte``
bestimmen. Das Leben ist stets und überall hochkomplex. Schaf und Löwe
halten ein filigranes Gleichgewicht und interagieren an der Schwelle zum
Chaos. Wären Löwen plötzlich schneller, könnten sie Beutetiere fangen,
die ihnen bisher entkommen sind. Wären Schafen auf einmal Flügel
gewachsen, würden die Löwen verhungern. Die Fähigkeit, Gewalt auszuüben
und sich gegen sie zur Wehr zu setzen, ist die entscheidende Variable,
die jenes Leben am Rande der Ordnung verändert.

Wir rücken die Gewalt aus triftigem Grund in den Fokus unserer
Megapolitik-Theorie. Die Beherrschung von Gewalt stellt die zentrale
Herausforderung dar, mit der jede Gesellschaft konfrontiert ist. Wie wir
bereits in \emph{The Great Reckoning} dargelegt haben:

\begin{quote}
„Der Grund dafür, dass Menschen häufig zur Gewalt greifen, ist, dass
sich dies oft auszahlt. In gewissem Sinne ist es das Einfachste, was ein
Mann tun kann, wenn er Geld begehrt - es einfach zu nehmen. Dies gilt
gleichermaßen für eine Armee von Männern, die ein Ölfeld in Besitz
nimmt, wie auch für einen einsamen Kriminellen, der sich eine Geldbörse
greift. Wie William Playfair treffend formulierte, hat die Macht „stets
den einfachsten Weg zum Reichtum gesucht, indem sie diejenigen angriff,
die ihn besitzen``.

Eine der großen Herausforderungen des Wohlstands liegt genau darin, dass
räuberische Gewalt unter bestimmten Bedingungen sehr einträglich sein
kann. Krieg verändert alles. Er ändert die Spielregeln. Er beeinflusst
die Verteilung von Vermögen und Einkommen. Er hat sogar Macht darüber,
wer lebt und wer stirbt. Gerade die Tatsache, dass sich Gewalt lohnt,
macht sie so schwer zu kontrollieren.`` \footnote{Siehe James Dale
  Davidson und Lord William Rees-Mogg, \emph{The Great Reckoning},
  Zweite Ausgabe (New York: Simon \& Schuster, 1993), S. 53.}
\end{quote}

Diese Denkweise ermöglichte es uns, eine Reihe von Entwicklungen
vorherzusehen, die besserwisserische Experten als unmöglich abtaten. So
versuchten wir beispielsweise mit „Blood in the Streets``, das Anfang
1987 veröffentlicht wurde, die Anfänge der großen Megapolitischen
Revolution festzuhalten, die jetzt in vollem Gange ist. Damals
argumentierten wir, dass technologischer Wandel das Machtgleichgewicht
weltweit destabilisiert. Einige unserer zentralen Thesen sind:

\begin{itemize}
\item
  Wir äußerten die Vermutung, dass die amerikanische Vormachtstellung
  auf dem absteigenden Ast sei, was zu wirtschaftlichen Ungleichheiten
  und Krisensituationen führen würde - einschließlich eines neuerlichen
  Börsencrashs nach dem Muster von 1929. Die Experten widerlegten fast
  einstimmig die Möglichkeit eines solchen Szenarios. Doch schon ein
  halbes Jahr später, im Oktober 1987, wurden die Finanzmärkte vom
  intensivsten Ausverkauf des Jahrhunderts erschüttert.
\item
  Wir empfahlen unseren Lesern, auf den Zusammenbruch des Kommunismus zu
  warten. Wieder einmal wurden wir von Experten belächelt. Doch dann
  geschah 1989 das, was „niemand vorhergesehen hatte``. Die Berliner
  Mauer fiel und Revolutionen fegten die kommunistischen Regime von den
  baltischen Staaten bis Bukarest hinfort.
\item
  Wir erläuterten, warum das multiethnische Imperium, das die
  bolschewistische Nomenklatur von den Zaren übernommen hat,
  „unvermeidlich zerfallen`` würde. Ende Dezember 1991 wurde das Banner
  mit Hammer und Sichel zum letzten Mal vom Kreml eingeholt, als die
  Sowjetunion aufhörte zu existieren.
\item
  Inmitten des massiven Rüstungsanstiegs unter Reagan behaupteten wir,
  die Welt stehe kurz vor einer umfangreichen Abrüstung. Dies wurde
  vielfach als unwahrscheinlich oder sogar als absurd abgetan. Doch die
  nachfolgenden sieben Jahre führten zur weitreichendsten Abrüstung seit
  dem Ende des Ersten Weltkriegs.
\item
  Es gab eine Zeit, in der Experten aus Nordamerika und Europa Japan als
  Beleg dafür anführten, dass Regierungen in der Lage seien, die Märkte
  erfolgreich zu manipulieren. Unsere Prognose lautete jedoch anders.
  Wir sagten voraus, dass der Boom auf dem japanischen Finanzmarkt in
  einem Crash enden würde. Kurz nach dem Fall der Berliner Mauer erlebte
  der japanische Aktienmarkt einen drastischen Absturz und verlor fast
  die Hälfte seines Wertes. Bis heute sind wir der Überzeugung, dass der
  finale Absturz den Wertverlust von 89 Prozent, den die Wall Street
  nach dem Crash von 1929 erlitt, erreichen oder sogar übertreffen
  könnte.
\item
  Zu einer Zeit, in der von der Mittelklassefamilie bis hin zu den
  weltweit größten Immobilieninvestoren fast jeder davon ausging, dass
  Immobilienpreise nur steigen und nicht fallen könnten, haben wir vor
  einem bevorstehenden Immobiliencrash gewarnt. Innerhalb von vier
  Jahren verloren Immobilieninvestoren weltweit über eine Billion
  Dollar, als die Immobilienwerte ins Rutschen kamen.
\item
  Lange bevor es für Experten offensichtlich wurde, haben wir in „Blood
  in the Streets`` darauf hingewiesen, dass das Arbeitseinkommen
  gesunken ist und auch weiterhin sinken wird. Jetzt, fast ein Jahrzehnt
  später, erwacht die Welt endlich aus ihrem Dornröschenschlaf und
  erkennt diese Wahrheit. Die Durchschnittsstundenlöhne in den
  Vereinigten Staaten sind mittlerweile niedriger als zur Zeit der
  zweiten Eisenhower-Administration. Der durchschnittliche
  Jahresstundenlohn betrug 1993, bereinigt um Inflation, 18.808 Dollar.
  Im Vergleich dazu lag der entsprechende Lohn 1957, als Eisenhower
  seine zweite Amtszeit antrat, bei 18.903 Dollar.
\end{itemize}

Obwohl die Hauptthemen von „Blood in the Streets`` im Nachhinein
erstaunlich zutreffend waren, wurden sie vor einigen Jahren von den
Bewahrern etablierten Denkens noch als völliger Unsinn betrachtet. Ein
Newsweek-Rezensent von 1987 reflektierte die verschlossene
intellektuelle Atmosphäre der spätindustriellen Gesellschaft, indem er
unsere Analyse als „einen undurchdachten Angriff auf die Vernunft``
abtat.

Man könnte annehmen, dass Newsweek und ähnliche Publikationen im Laufe
der Zeit erkannt haben, dass unsere Analysen aufschlussreiche Einblicke
in die Dynamiken der sich wandelnden Welt gegeben haben. Aber
keineswegs, das war nicht der Fall. Die erste Ausgabe von \emph{The
Great Reckoning} wurde mit derselben spöttischen Feindseligkeit
aufgenommen wie „Blood in the Streets``. Keine geringere Autorität als
das Wall Street Journal lehnte unsere Analyse kategorisch ab und
bezeichnete sie als das Geschwätz einer „bekloppten Tante``.

Trotz diesem leichten Schmunzeln zeigten sich die Themen von \emph{The
Great Reckoning} in Wahrheit weniger absurd, als die Bewahrer der
Orthodoxie es zunächst darstellen wollten.

Wir haben unsere Analyse zum Ende der Sowjetunion vertieft und uns damit
auseinandergesetzt, warum Russland und die anderen ehemaligen
Sowjetrepubliken einer Zukunft voller zunehmender Unruhen,
Hyperinflation, und fallendem Lebensstandard entgegenblicken.

\begin{itemize}
\item
  Wir haben dargelegt, warum die 1990er Jahre als ein Jahrzehnt des
  Stellenabbaus in Erinnerung bleiben werden, einschließlich des ersten
  globalen Stellenabbaus sowohl in der Regierung als auch in
  Unternehmen.
\item
  Wir prognostizierten zudem, dass es zu einer weitreichenden
  Neudefinition der Rahmenbedingungen für die Umverteilung von Einkommen
  kommen würde, einschließlich drastischer Kürzungen des
  Leistungsstandards. Von Kanada bis Schweden zeigten sich Anzeichen
  einer Finanzkrise, und amerikanische Politiker verkündeten das „Ende
  des Sozialstaats, so wie wir ihn kennen``.
\item
  Wir haben vorausgesehen und erläutert, warum sich die vermeintliche
  „neue Weltordnung`` letztendlich als „neue Weltunordnung``
  herausstellen würde. Lange bevor die Gräueltaten in Bosnien die Medien
  beherrschten, warnten wir schon davor, dass Jugoslawien in einen
  Bürgerkrieg versinken könnte.
\item
  Bevor Somalia ins Chaos abglitt, erläuterten wir, wieso der
  bevorstehende Zusammenbruch von Regierungen in Afrika zur Folge haben
  würde, dass einige Länder dort praktisch unter Zwangsverwaltung
  gestellt werden müssten.
\item
  Wir haben vorausgesagt und begründet, warum der militante Islam den
  Marxismus als führende Ideologie der Auseinandersetzung mit dem Westen
  ablösen würde. Jahre bevor der Bombenanschlag in Oklahoma stattfand
  und der Versuch unternommen wurde das World Trade Center in die Luft
  zu jagen, haben wir erklärt, warum die USA mit einer Zunahme des
  Terrorismus konfrontiert werden würden.
\item
  Bevor die Schlagzeilen über die Unruhen in Städten wie Los Angeles und
  Toronto die Runde machten, haben wir darüber gesprochen, warum die
  Entstehung krimineller Subkulturen innerhalb städtischer Minderheiten
  den Nährboden für weitreichende kriminelle Gewalt bildet.
\item
  Wir prognostizierten auch „die letzte Depression des zwanzigsten
  Jahrhunderts``, die 1989 in Asien ihren Anfang nahm und sich vom Rand
  hin zum Zentrum des globalen Systems hin ausbreitete. Wir waren der
  Meinung, dass der japanische Aktienmarkt den Weg der Wall Street nach
  1929 einschlagen würde, was zu einem Kreditkollaps und einer
  Depression führen würde. Wenngleich massive staatliche Eingriffe in
  Japan und anderenorts vorübergehend verhinderten, dass die Märkte den
  Rückgang der Kreditbedingungen vollständig widerspiegelten, hat dies
  die wirtschaftliche Krisensituation lediglich verschoben und
  verschärft, was den Druck für wettbewerbsbedingte Abwertungen sowie
  einen systembedingten Kreditkollaps der Art, der in den 30er Jahren zu
  weltweiten Wirtschaftseinbrüchen geführt hatte, erhöht.
\end{itemize}

In \emph{The Great Reckoning} wurden auch einige kontroverse Thesen
formuliert, die sich bisher noch nicht bewahrheitet haben oder nicht den
von uns vorhergesagten Entwicklungsstatus erreicht haben:

\begin{itemize}
\item
  Wir prophezeiten, dass der japanische Aktienmarkt dem Pfad der Wall
  Street nach 1929 folgen und dies zu einem Kreditkollaps sowie einer
  Depression führen würde. Obwohl die Arbeitslosenraten in Spanien,
  Finnland und einigen weiteren Ländern die der 1930er Jahre sogar
  überstiegen und zahlreiche Länder, einschließlich Japan, lokale
  Depressionen durchliefen, hat es bislang noch keinen systemischen
  Kreditkollaps gegeben, der die Wirtschaften weltweit zum Einsturz
  brachte wie in den 1930er Jahren.
\item
  Wir waren der Ansicht, dass der Zusammenbruch des Führungssystems in
  der ehemaligen Sowjetunion dazu führen könnte, dass Atomwaffen in die
  Hände von Kleinstaaten, Terroristen und kriminellen Banden fallen.
  Dies ist glücklicherweise nicht eingetreten, zumindest nicht in dem
  von uns befürchteten Ausmaß. Presseberichten zufolge hat der Iran zwar
  mehrere taktische Atomwaffen auf dem Schwarzmarkt erworben, und
  deutsche Behörden konnten mehrere Versuche, nukleares Material zu
  verkaufen, vereiteln. Doch es gab keine angekündigte Stationierung
  oder Verwendung von Atomwaffen aus den Beständen der ehemaligen
  Sowjetunion.
\item
  Wir haben erläutert, warum der „Krieg gegen die Drogen`` in Ländern,
  in denen Drogen weit verbreitet sind - insbesondere in den USA -
  tatsächlich dazu dient, Polizei- und Justizsysteme zu unterwandern.
  Angesichts von zig Milliarden Dollar an verdeckten Monopolgewinnen,
  die jährlich erzielt werden, haben Drogenhändler sowohl das nötige
  Kapital als auch die Motivation, selbst stabil scheinende Länder zu
  korrumpieren. Obwohl die Weltmedien hin und wieder über die
  Durchdringung des amerikanischen politischen Systems mit Drogengeldern
  auf höchstem Niveau berichtet haben, ist die gesamte Geschichte noch
  nicht vollständig ans Licht gekommen.
\end{itemize}

\subsection{Hinsehen wenn andere
wegsehen}\label{hinsehen-wenn-andere-wegsehen}

Trotz aller Aspekte, in denen unsere Prognosen falsch lagen oder im
Licht der aktuellen Erkenntnisse falsch zu sein scheinen, besteht unsere
Bilanz die Überprüfung. Vieles, was voraussichtlich eine Rolle in der
Wirtschaftsgeschichte der 1990er Jahre spielen wird, war bereits in
\emph{The Great Reckoning} vorhergesehen und erläutert worden. Unsere
Vorhersagen waren oft nicht bloß einfache Hochrechnungen oder
Fortführungen von Trends, sondern Prognosen über wesentliche
Abweichungen von dem, was seit dem Zweiten Weltkrieg als normal galt.
Wir hatten gewarnt, dass sich die 1990er Jahre drastisch von den
vorhergehenden fünf Jahrzehnten unterscheiden würden. Wenn wir die
Nachrichten der Jahre 1991 bis 1995 betrachten, stellt sich heraus, dass
viele Themen aus \emph{The Great Reckoning} nahezu täglich bestätigt
wurden.

Wir sehen diese Entwicklungen nicht als isolierte Schwierigkeiten oder
sporadische Probleme, sondern als Erschütterungen und Beben, die entlang
derselben Bruchlinie auftreten. Die bestehende Ordnung wird durch ein
megapolitisches Erdbeben erschüttert, das sowohl die Institutionen
revolutionieren wird, als auch die Art und Weise, wie aufgeklärte
Menschen die Welt wahrnehmen, drastisch verändern wird.

Obwohl Gewalt eine zentrale Rolle in der Funktionsweise unserer Welt
spielt, wird sie erstaunlicherweise oft übersehen. Die meisten
politischen Analysten und Wirtschaftswissenschaftler behandeln Gewalt
wie eine unwesentliche Störung, ähnlich einer Fliege, die um einen
Kuchen schwirrt, anstatt sie als den Koch zu betrachten, der den Kuchen
überhaupt erst gebacken hat.

\subsection{Ein weiterer megapolitischer
Pionier}\label{ein-weiterer-megapolitischer-pionier}

Tatsächlich wurde über die Rolle der Gewalt in der Geschichte so wenig
nachgedacht, dass die Bibliografie zur megapolitischen Analyse auf ein
einziges Blatt Papier passen würde. In unserem Buch \emph{The Great
Reckoning} griffen wir die Argumentation eines fast vollständig in
Vergessenheit geratenen Klassikers der Megapolitik-Analyse auf: „An
Enquiry into the Permanent Causes of the Decline and Fall of Powerful
and Wealthy Nations`` von William Playfair aus dem Jahr 1805, die wir
weiter ausarbeiten. Einer unserer Ausgangspunkte ist das Werk von
Frederic C. Lane. Lane war ein mittelalterlicher Historiker, der in den
1940er und 1950er Jahren mehrere eindringliche Schriften zur Rolle der
Gewalt in der Geschichte verfasste. Seine womöglich umfassendste Arbeit
war „Economic Consequences of Organized Violence``, die 1958 im Journal
of Economic History veröffentlicht wurde. Nur eine Handvoll
professioneller Wirtschaftswissenschaftler und Historiker haben sie
gelesen, und scheinbar haben die meisten ihre Tragweite nicht erkannt.
Ähnlich wie Playfair schrieb auch Lane für ein Publikum, das es zu
seiner Zeit noch nicht gab.

\subsection{Einsichten für das
Informationszeitalter}\label{einsichten-fuxfcr-das-informationszeitalter}

Lane veröffentlichte seine Abhandlung über Gewalt und die
wirtschaftliche Relevanz des Krieges weit vor dem Aufkommen des
Informationszeitalters. Ohne Zweifel schrieb er seine Thesen nicht im
Hinblick auf Mikroprozessortechnologie oder die anderen technologischen
Umbrüche, die wir derzeit erleben. Gleichwohl stellen seine Erkenntnisse
über Gewalt einen Referenzrahmen dar, der uns dabei hilft zu verstehen,
wie die Gesellschaft im Zuge der Informationsrevolution neu geformt
wird.

Das Fenster in die Zukunft, das Lane öffnete, war paradoxerweise eines,
durch das er einen Blick in die Vergangenheit warf. Als Historiker,
spezialisiert auf das Mittelalter und insbesondere auf die Handelsstadt
Venedig, deren Reichtum in einer brutalen Welt auf- und abebbt, wandte
er seine Gedanken dem Aufstieg und Niedergang von Venedig zu. Diese
Betrachtungen lenkten seinen Fokus auf Themen, die uns wertvolle
Einblicke in die Zukunft gewähren können. Er erkannte, dass die Art und
Weise, wie Gewalt organisiert und kontrolliert wird, entscheidend ist
für „den Umgang mit begrenzten Ressourcen``.\footnote{Frederic C. Lane,
  \emph{Economic Consequences of Organized Violence}, The Journal of
  Economic History Vol. 18, No.~4 (Dezember 1958), S. 402.}

Wir sind überzeugt, dass Lanes Analyse über den konkurrierenden Einsatz
von Gewalt uns viel Aufschluss darüber geben kann, wie das Leben im
Informationszeitalter sich aller Wahrscheinlichkeit nach verändern wird.
Aber rechnen Sie nicht damit, dass die meisten Menschen ein so
altertümliches und abstraktes Argument zur Kenntnis nehmen, geschweige
denn ihm folgen. Während die Aufmerksamkeit der Welt auf unehrliche
Diskussionen und exzentrische Persönlichkeiten gelenkt wird, bleiben die
Fehlentwicklungen der Megapolitik meist unbemerkt. Der durchschnittliche
Nordamerikaner hat wahrscheinlich hundertmal mehr Aufmerksamkeit auf O.
J. Simpson gerichtet als auf die neuen Mikrotechnologien, die seinen
Arbeitsplatz überflüssig machen und das politische System untergraben,
auf das er für seine Arbeitslosenunterstützung angewiesen ist.

\section{DIE EITELKEIT DER WÜNSCHE}\label{die-eitelkeit-der-wuxfcnsche}

Die Neigung, das Wesentliche zu übersehen, findet sich nicht nur bei den
Stubenhockern vor dem Fernseher. Herkömmliche Denkmodelle aller Art
klammern sich an die Illusion des Nationalstaates und dass es die
Ansichten der Menschen wären, die die Welt verändern. Angeblich
versierte Analysten verstricken sich in Erklärungen und Prognosen, die
bedeutende historische Entwicklungen so deuten, als wären sie auf Wunsch
herbeigeführt worden. Ein besonders eindrückliches Beispiel für diese
Art von Argumentation fand sich auf der Meinungsseite der New York
Times, als wir gerade dabei waren, den Artikel „Goodbye, Nation-State,
Hello...What?{}`` von Nicholas Colchester\footnote{Nicholas Colehester,
  \emph{Goodbye Nation-State, Hello \ldots{} What?}, New York Times, 17.
  Juli 1994, Abschnitt 4, S. 17.} zu verfassen. Nicht nur, dass das
Thema - der Untergang des Nationalstaates - genau unser Thema ist; der
Autor steht auch exemplarisch dafür, wie weit unsere Gedankengänge von
der Norm abweichen. Colchester ist alles andere als ein Dummkopf. Er war
Redaktionsleiter bei der Economist Intelligence Unit. Wenn also jemand
einen realistischen Überblick über die Welt haben sollte, dann er.
Trotzdem weist sein Artikel mehrmals ausdrücklich darauf hin, dass „das
Aufkommen einer internationalen Regierung`` nun „logischerweise
unaufhaltsam ist``.

Warum? Weil der Nationalstaat ins Straucheln geraten ist und die
Kontrolle über die wirtschaftlichen Kräfte verloren hat.

Wir sind der Ansicht, dass diese Annahme schlichtweg absurd ist. Der
Gedanke, eine neue Regierungsform würde einfach deshalb entstehen, weil
eine andere gescheitert ist, ist ein Trugschluss. Hätte diese
Argumentation Bestand, dann hätten Länder wie Haiti und Zaire längst
eine bessere Regierung, lediglich gestützt auf die Tatsache, dass deren
bisherige Regierungsführung so offenkundig unzureichend war.

Die Perspektive Colchesters, die unter den wenigen, die sich in
Nordamerika und Europa mit solchen Themen beschäftigen, weitgehend
vertreten wird, vernachlässigt völlig die maßgeblichen, megapolitischen
Kräfte, welche bestimmen, welche politischen Systeme tatsächlich
tragfähig sind. Dies bildet den Kern dieses Buches. Betrachtet man die
Technologien, die das neue Jahrtausend prägen, so erscheint es sehr viel
wahrscheinlicher, dass wir nicht auf eine Weltregierung zusteuern,
sondern auf eine Mikroregierung oder sogar Zustände, die sich der
Anarchie annähern.

Für jede ernsthafte Untersuchung zum Einfluss von Gewalt auf die
Festsetzung von Handlungsregeln wurden dutzende Bücher über die
Feinheiten von Weizensubventionen und noch hunderte mehr über die teils
undurchsichtigen Facetten der Geldpolitik verfasst. Ein Großteil dieses
Mangels an Reflexion über die Schlüsselfragen, die tatsächlich den
Verlauf der Geschichte bestimmen, spiegelt vermutlich die relative
Stabilität der Machtverhältnisse der letzten Jahrhunderte wider. Ein
Vogel, der auf dem Rücken eines Nilpferdes eingeschlafen ist, denkt
nicht daran, seinen Aussichtspunkt zu verlieren, bis das Nilpferd sich
tatsächlich bewegt. Träume, Mythen und Phantasien nehmen in den
sogenannten Sozialwissenschaften einen viel größeren Stellenwert ein,
als wir gemeinhin annehmen.

Dies zeigt sich besonders in der umfangreichen Literatur zum Thema
wirtschaftliche Gerechtigkeit. Für jede Seite, die einer sorgfältigen
Analyse darüber gewidmet ist, wie Gewalt die Gesellschaft prägt und
somit die Grenzen festlegt, innerhalb derer die Wirtschaft agieren muss,
sind unzählige Worte über wirtschaftliche Gerechtigkeit und
Ungerechtigkeit verfasst und gesprochen worden. Allerdings setzen
Diskussionen über wirtschaftliche Gerechtigkeit im modernen Kontext
voraus, dass die Gesellschaft von einem Zwangsinstrument dominiert wird,
das so mächtig ist, dass es die guten Dinge des Lebens entziehen und
umverteilen kann. Solch eine Macht hat es nur für wenige Generationen in
der Neuzeit gegeben. Diese Macht ist jedoch aktuell im Schwinden.

\subsection{Big Brother und soziale
Sicherheit}\label{big-brother-und-soziale-sicherheit}

Die industrielle Technologie verlieh den Regierungen im 20. Jahrhundert
mehr Kontrollmöglichkeiten als jemals zuvor. Auf eine gewisse Weise
schien es unvermeidlich, dass die Regierungen die Kontrolle so effektiv
monopolisieren würden, dass kaum noch Raum für individuelle Autonomie
übrigbleiben würde. Um die Mitte des Jahrhunderts rechnete niemand mit
einem Triumph des selbstbestimmten Individuums.

Einige der schärfsten Denker Mitte des 20. Jahrhunderts waren aufgrund
der vorherrschenden Ansichten überzeugt, dass die Tendenz des
Nationalstaates zur Machtkonzentration in eine totalitäre Kontrolle über
alle Lebensbereiche münden würde. In George Orwells „1984`` (1949)
kämpfte der Einzelne vergeblich darum, unter dem wachsamen Auge von Big
Brother Autonomie und Selbstachtung zu wahren. Es schien eine
aussichtslose Angelegenheit zu sein. Friedrich August von Hayeks „Der
Weg zur Knechtschaft`` (1944) nahm einen wissenschaftlicheren Ansatz,
indem er argumentierte, dass durch eine neue Form der wirtschaftlichen
Kontrollmacht, die den Staat zum absoluten Herrscher macht, die Freiheit
verloren geht. Diese Arbeiten wurden verfasst, bevor Mikroprozessoren
aufkamen, welche eine Fülle von Technologien hervorgebracht haben, die
die Fähigkeit von kleinen Gruppen und sogar Individuen stärken,
unabhängig von einer zentralen Autorität zu agieren.

So scharfsinnig Geister wie Hayek und Orwell auch gewesen sein mögen, so
neigten sie doch zum Pessimismus. Geschichte steckt voller
Überraschungen. Der totalitäre Kommunismus hat kaum das Jahr 1984
überstanden. Im nächsten Jahrtausend könnte zwar eine neue Form der
Knechtschaft entstehen, sollten Regierungen es schaffen, die befreienden
Kräfte der Mikrotechnologie zu unterdrücken. Es ist jedoch weitaus
wahrscheinlicher, dass wir eine beispiellose Gelegenheit und Autonomie
für das Individuum erleben werden. Das, was unsere Eltern als Bedrohung
wahrgenommen haben, könnte sich als unwichtig erweisen. Das, was sie als
stabile und dauerhafte Elemente des gesellschaftlichen Lebens ansahen,
scheint zum Untergang bestimmt zu sein. Dort, wo die Notwendigkeit
Einschränkungen für menschliche Entscheidungen schafft, passen wir uns
an und organisieren unser Leben entsprechend neu.

\subsection{Die Gefahren der
Vorhersage}\label{die-gefahren-der-vorhersage}

Unbestritten riskieren wir einen Verlust unseres geringen Maßes an
Würde, wenn wir versuchen, weitreichende Veränderungen in der
Organisation des Lebens und der Kultur, die es zusammenhält,
vorherzusagen und zu erläutern. Die meisten Prognosen sind zum Scheitern
verurteilt und wirken mit der Zeit geradezu lächerlich. Und je
tiefgreifender die Veränderungen sind, die vorausgesagt werden, umso
häufiger liegen die Vorhersagen peinlich falsch. Die Welt geht nicht
unter. Das Ozon verschwindet nicht einfach. Die angekündigte Eiszeit
wandelt sich in eine globale Erwärmung. Trotz aller gegenteiligen
Befürchtungen ist immer noch Öl im Tank. Mr.~Antrobus, die Hauptfigur in
„Wir sind noch einmal davongekommen``, trotzt der Kälte, überlebt Kriege
sowie drohende wirtschaftliche Katastrophen und wird alt, ohne sich um
die wiederholten Warnungen der Experten zu kümmern.

Die meisten Versuche, die Zukunft „vorherzusagen``, entpuppen sich
schnell als komödiantische Misserfolge. Selbst dort, wo das
Eigeninteresse einen starken Anstoß für klares Denken bietet, ist unsere
Perspektive auf die Zukunft oft verblendet. Im Jahr 1903 verkündete das
Unternehmen Mercedes beispielsweise, dass es „auf der ganzen Welt
niemals eine Million Automobile geben würde``. Der Grund dafür war, dass
es unwahrscheinlich erschien, weltweit eine Million Handwerker zu
Chauffeuren ausbilden zu können.\footnote{Norman Macrae,
  \emph{Governments in Decline}, Cato Policy Report, Juli/August 1992,
  S. 10.}

Das zu erkennen, sollte uns verstummen lassen. Tut es aber nicht. Wir
scheuen uns nicht, uns dem gebührenden Spott auszusetzen. Selbst wenn
wir einen gewaltigen Fehler begehen, dürfen nachfolgende Generationen
nach Herzenslust über uns lachen - vorausgesetzt natürlich, irgendjemand
erinnert sich noch an unsere Worte. Wer mutig genug ist, einen neuen
Gedanken zu formulieren, muss auch bereit sein, sich eventuell zu irren.
Wir sind keinesfalls so unnachgiebig und unbrauchbar, dass wir Angst
davor hätten, uns zu irren. Ganz im Gegenteil. Wir würden es immer
bevorzugen, Ideen aufzuwerfen, die für Sie hilfreich sein könnten,
anstatt diese aufgrund der Angst zu unterdrücken, sie könnten im
Nachhinein als überzogen oder peinlich erscheinen.

Wie Arthur C. Clarke treffend bemerkte, scheitern Vorhersagen für die
Zukunft meistens aus zwei Gründen: „Zu wenig Nerven und zu wenig
Vorstellungskraft``.\footnote{Arthur C. Clarke, \emph{Profiles of the
  Future: An Enquiry into the Limits of the Possible} (London: Victor
  Gollancz Ltd., 1962), S. 13.} Er merkte an: „Zu wenig Nerven scheint
häufiger zu sein. Es tritt ein, wenn der angehende Prophet trotz aller
relevanten Fakten nicht in der Lage ist, die unausweichliche
Schlussfolgerung zu erkennen. Einige solcher Vorhersagefehler sind so
absurd, dass sie nahezu unglaublich erscheinen.`` \footnote{Ebenda.}

Sollte unsere Untersuchung der informationellen Revolution nicht von
Erfolg gekrönt sein, was unweigerlich der Fall sein wird, dann liegt das
eher an unserem Mangel an Vorstellungskraft als an fehlenden Nerven.
Zukunftsprognosen waren stets ein wagemutiges Unterfangen, dem nicht
ohne berechtigte Skepsis begegnet wird. Es könnte sein, dass die Zeit
zeigt, dass unsere Schlussfolgerungen komplett daneben liegen. Wir
behaupten nicht, wie Nostradamus prophetische Fähigkeiten zu besitzen.
Unsere Vorhersagen basieren nicht auf dem Rühren eines Zauberstabs in
einer Wasserschale oder dem Erstellen von Horoskopen. Ebenso wenig
bedienen wir uns kryptischer Verse. Unser Anliegen ist es, Ihnen eine
nüchterne und unvoreingenommene Analyse von Themen zu präsentieren, die
für Sie von großer Bedeutung sein könnten.

Wir sehen es als unsere Aufgabe, unsere Ansichten kundzutun, auch wenn
sie auf den ersten Blick als ketzerisch betrachtet werden könnten.
Gerade weil sie sonst möglicherweise ungehört blieben. In der
geschlossenen Gedankenwelt der spätindustriellen Gesellschaft haben
Ideen nicht den gleichen freien Lauf, wie sie es in etablierten Medien
haben sollten.

Dieses Buch wurde mit einem konstruktiven Gedankenansatz verfasst. Es
ist das dritte, das wir zusammen geschrieben haben, und es analysiert
verschiedene Phasen des umfassenden Wandels, der gegenwärtig
stattfindet. Ähnlich wie „Blood in the Streets`` und \emph{The Great
Reckoning} stellt es eine Denkübung dar. Es beleuchtet das Ende der
Industriegesellschaft und deren Umgestaltung in neue Formen. In den
kommenden Jahren rechnen wir mit erstaunlichen Paradoxien. Einerseits
gibt es Zeugnisse der Manifestation einer neuen Form der Freiheit,
inklusive der Entstehung des selbstbestimmten Individuums. Sie können
eine fast vollständige Entfaltung der Produktivität erwarten.
Gleichzeitig erwarten wir den Untergang des modernen Nationalstaates.
Viele der Gleichheitsgarantien, die die Menschen im Westen im
zwanzigsten Jahrhundert für selbstverständlich hielten, werden mit ihm
zugrunde gehen. Wir gehen davon aus, dass die repräsentative Demokratie,
so wie wir sie heute kennen, verschwinden und durch die neue Demokratie
der Wahlfreiheit auf dem Cybermarktplatz ersetzt werden wird. Wenn
unsere Annahmen korrekt sind, wird die Politik des nächsten Jahrhunderts
wesentlich vielfältiger, aber weniger bedeutend sein als die, an die wir
uns gewöhnt haben.

Wir sind überzeugt, dass unsere Argumentation nachvollziehbar ist,
obwohl sie durch einige Bereiche führt, die man als intellektuelle
Äquivalente zu abgelegenen Dörfern und Problemvierteln bezeichnen
könnte. Wenn unsere Aussagen an manchen Stellen unklar erscheinen, liegt
das nicht daran, dass wir raffiniert sind oder uns der gefeierten
Mehrdeutigkeit derer bedienen, die vorgeben, die Zukunft durch
kryptische Aussagen voraussagen zu können. Wir sind keine Wortverdreher.
Sollten unsere Argumente unverständlich sein, dann deswegen, weil wir es
nicht geschafft haben, überzeugende Ideen klar verständlich zu Papier
gebracht zu haben. Im Gegensatz zu vielen Zukunftsvisionären möchten
wir, dass Sie unseren Denkansatz verstehen und nachvollziehen können. Er
basiert nicht auf übernatürlichen Fantasien oder den Bewegungen der
Planeten, sondern auf nüchterner, altmodischer Logik. Aus rein logischen
Gründen sind wir der Meinung, dass Mikroprozessoren den Nationalstaat
unausweichlich untergraben und zerstören und gleichzeitig neue Formen
sozialer Organisation hervorbringen werden. Es ist sowohl notwendig als
auch möglich, dass Sie zumindest einige Aspekte der neuen Lebensform
erahnen, die vielleicht schneller eintritt, als Sie es für möglich
halten.

\subsection{Die Ironie einer vorhergesagten
Zukunft}\label{die-ironie-einer-vorhergesagten-zukunft}

Seit Jahrhunderten gilt das Ende des letzten Jahrtausends als
entscheidender Wendepunkt in der Geschichte. Vor über 850 Jahren legte
der Heilige Malachias das Jahr 2000 als Zeitpunkt des Jüngsten Gerichts
fest. Edgar Cayce, ein amerikanischer Hellseher, prophezeite im Jahr
1934, die Erde würde sich im Jahr 2000 um ihre Achse drehen, wodurch
Kalifornien zersplittern und sowohl New York City als auch Japan
überschwemmt werden würden. 1980 verkündete Hideo Itokawa, ein
japanischer Raketentechniker, dass die Konstellation der Planeten im
„Großen Kreuz`` am 18. August 1999 eine weitreichende Umweltzerstörung
zur Folge haben und das Ende des menschlichen Lebens auf der Erde
besiegeln würde.\footnote{A. T. Mann, \emph{Millennium Prophecies:
  Predictions for the Year 2000} (Shafiesbury, England: Element Books,
  1992), S. 88ff.}

Apokalyptische Visionen sind ein gefundenes Fressen für Spötter. Zwar
mag das Jahr 2000 aufgrund seiner runden Zahlenform beeindruckend
wirken, was jedoch nichts weiter als ein willkürliches Produkt des
westlich-christlichen Kalenders ist. Andere Kalender und Zeitsysteme
berechnen Jahrhunderte und Jahrtausende aufgrund unterschiedlicher
Ausgangspunkte. Gemäß islamischem Kalender zum Beispiel entspricht das
Jahr 2000 nach Christus dem Jahr 1378. So unspektakulär wie ein Jahr nur
klingen kann. Der chinesische Kalender hingegen, welcher sich alle 60
Jahre wiederholt, bezeichnet das Jahr 2000 n.~Chr. schlichtweg als ein
weiteres Jahr des Drachen, das Teil eines anhaltenden Zyklus ist,
welcher tausende von Jahren in die Vergangenheit reicht. Das Jahr 2000
steht jedoch nicht ausschließlich für theologische Vorhersagen. Seine
Bedeutung ist nicht nur in der christlichen Tradition verwurzelt,
sondern ebenso durch die IT-Einschränkungen des Jahrhunderts. Das
berüchtigte Jahr-2000-Problem oder „Y2K`` stellt einen potentiell
katastrophalen Logikfehler in milliardenfachem Computercode dar und
könnte durchaus das Potential zum Untergangsszenario haben, indem
industrielle Abläufe um die Jahrtausendwende empfindlich beeinträchtigt
werden. Viele Computer und Mikroprozessoren basieren noch immer auf
Software aus den Anfangszeiten der Computertechnologie, als
Speicherplatz mit Kosten von bis zu 600.000 Dollar pro Megabyte
wertvoller als Gold war. Um teuren Speicherplatz zu schonen, nutzten
damalige Programmierer zweistellige Jahreszahlen, die lediglich die
letzten beiden Ziffern des jeweiligen Jahres enthielten. Diese Praxis,
zweistellige Datumsangaben zu verwenden, wurde in der
Großrechnersoftware und sogar bei den sogenannten eingebetteten Chips,
den Mikroprozessoren, die zur Steuerung von so ziemlich allem
herangezogen wurden -- von Videorekordern bis zu Autostartsystemen,
Sicherheitssystemen, Telefonen, Vermittlungssystemen, die das
Telefonnetz regeln, Prozess- und Kontrollsystemen in Fabriken sowie
Kraftwerken, Ölraffinerien, chemischen Werken, Pipelines und vielem mehr
-- nahezu flächendeckend eingeführt. So wurde das Jahr 1999 kurzerhand
auf „99`` reduziert. Das Problem allerdings liegt darin, was geschieht,
wenn 00 für das Jahr 2000 erscheint. Viele Computer interpretieren dies
als 1900, was dazu führen könnte, dass viele nicht umgerüstete Computer
und andere digitale Geräte das Jahr 2000 in den Datumsfeldern nicht
richtig einordnen können.

Das Ergebnis wird eine massive Belastung durch Datenverfälschung sein
und ein zufälliges Muster für neue Potenziale in der
Informationskriegsführung setzen. Im Zeitalter der Information könnten
potenzielle Gegner Schaden anrichten, indem sie „Logikbomben`` zünden,
die die Funktion wichtiger Systeme sabotieren, indem sie die Daten, auf
denen ihre Funktionen basieren, schädigen. Bei einer Militärübung wäre
es beispielsweise nicht notwendig, ein Flugzeug abzuschießen, sofern man
dessen Betriebsdaten beschädigen könnte. Datenbeschädigung kann das
Funktionieren einer modernen Gesellschaft beinahe so stark
beeinträchtigen wie physische Angriffe. Dass dies weitreichende Folgen
haben kann, sollte offensichtlich sein. Die Londoner „Mail`` berichtete
am 14. Dezember 1997, dass Fluggesellschaften weltweit hunderte von
Flügen für den 1. Januar 2000 streichen wollten, da sie befürchteten,
dass die Flugverkehrskontrollsysteme ausfallen könnten.\footnote{Yardeni,
  op. cit., S. 45.} Zu den potenziellen Problemen zählen nicht nur die
Flugverkehrssysteme, sondern auch datumsensitive Funktionen in den
Flugzeugen selbst. Boeing zufolge müssten viele Flugzeuge auf das Jahr
2000 umgerüstet werden. Viele Geräte könnten Probleme haben, wenn sie
versuchen, ein Ereignis mit einem ungültigen Datum zu registrieren. Die
computergesteuerten Fly-by-Wire-Systeme, die zum Betrieb von Flugzeugen
verwendet werden, könnten fehlerhaft funktionieren, wenn sie so
programmiert sind, dass sie schlussfolgern, wichtige Wartungsarbeiten
seien zuletzt im Jahr 1900 durchgeführt worden. Diese könnten sogar in
eine Fehler-Schleife geraten und sich selbst abschalten.

Die potenziell tödlichen Folgen einer zeitlichen Logikbombe, die nicht
konforme Kontrollsysteme lahmlegt, könnten die Jahrtausendwende aus
unschönen Gründen zu einem denkwürdigen Zeitpunkt machen. Bedenken Sie,
dass viele Geräte, die Sie täglich nutzen, in eine Fehlerschleife
geraten und sich abschalten könnten - selbst wenn Sie das Glück haben
sollten, sich nicht in der Luft zu befinden, wenn das neue Jahrtausend
anbricht.

Es wäre ratsam, Unfälle zu vermeiden, die entweder durch
Herzschrittmacher, die nicht mit dem Jahr 2000 kompatibel sind, oder
einfach bloß durch betrunkene Silvesterfeiernde verursacht werden
könnten, denn was Herzschrittmacher ausfallen lässt, könnte auch das
Telekommunikationssystem außer Betrieb setzen und das Eintreffen eines
Krankenwagens verhindern. Solange Sie nicht in Brasilien oder der
Ukraine leben, sind Sie es gewohnt, den Hörer abzuheben oder Ihr
Autotelefon einzuschalten und automatisch ein Freizeichen zu hören.
Glücklicherweise müssen Sie nur selten die technischen Details des
Telefonsystems verstehen. Es stellt sich jedoch heraus, dass die
Telefonvermittlungsstellen und Router stark datumsabhängig sind. Jeder
Anruf wird mit Datum und Uhrzeit protokolliert, was für die Berechnung
der Gesprächsdauer zur Rechnungsstellung essenziell ist. Wenn Sie am 31.
Dezember 1999 um 23:59:30 Uhr ein einminütiges Gespräch führen und das
System um 00:00 Uhr feststellt, dass Ihr Gespräch eine negative Dauer
von mehr als 99 Jahren hatte, könnte das zu Fehlerschleifen und
Ausfällen führen. Zwar investieren die Telekommunikationsunternehmen
viel Geld, um ihre Vermittlungsstellen aufzurüsten und sie fit für das
Jahr 2000 zu machen, und es ist anzunehmen, dass die lokalen Anbieter
dasselbe tun, aber sollte auch nur ein kleines Unternehmen den
Anforderungen nicht gerecht werden und ausfallen, könnte das das gesamte
Netzwerk beeinträchtigen. Sie können sich glücklich schätzen, wenn Sie
am 1. Januar 2000 ein Freizeichen erhalten.

Um es mit den Worten des Experten für das Jahr 2000, Peter de Jager, zu
sagen: „Wenn wir die Möglichkeit verlieren, einen Anruf zu tätigen, dann
verlieren wir alles. Es würde elektronische Geldüberweisungen, den
Handel und Bankfilialen betreffen.`` Die Folgen von Fehlern, die im Jahr
2000 auftreten könnten, könnten sogar noch weitreichender sein.

Heutzutage kann niemand mit Gewissheit vorhersagen, wie groß der
Einfluss des Jahr-2000-Problems auf wichtige Systeme sein wird.
Eingebettete Systeme, die sich nicht umprogrammieren lassen und daher
ersetzt werden müssen, wenn sie aufgrund von Datumsproblemen nicht mehr
funktionieren, finden sich in Karten, Lastwagen und Bussen, die nach
1976 gebaut wurden. (Vielleicht werden Sie nicht in einen Unfall mit
Fahrzeugen verwickelt, die von Personen mit nicht konformen
Herzschrittmachern gefahren werden, da deren Fahrzeuge möglicherweise
nicht starten werden). Eingebettete Systeme sind weit verbreitet, selbst
in Kraftwerken, Wasser- und Abwasseranlagen, medizinischen Geräten,
Militärausrüstung, Flugzeugen, Ölplattformen, Öltankern, Alarmsystemen
und Aufzügen. Auch wenn viele Mikroprozessorsysteme keine
datumsabhängigen Funktionen ausführen, könnten sie möglicherweise von
einer Uhr abhängig sein, die aufgrund des „Millennium-Bugs`` Probleme
bereiten könnte.

\section{GROSSRECHNER UND DIE
JAHR-2000-ZEITBOMBE}\label{grossrechner-und-die-jahr-2000-zeitbombe}

Die umfangreichen Befehls- und Kontrollsysteme von Behörden und
Großunternehmen mit hohem Transaktionsvolumen auf Großrechnern standen
ursprünglich im Zentrum der sogenannten Jahr-2000-Problematik. Da diese
Systeme auf leistungsfähigen Maschinen betrieben werden, deren Mehrzahl
an Software schon Jahrzehnte alt ist und häufig Inkompatibilitäten
aufweist, lag der Schwerpunkt der von Peter de Jager Anfang der 90er
Jahre erstmals geäußerten Warnungen vor dem Jahr-2000-Problem auf der
Notwendigkeit, die Betriebssysteme der zentralen Großrechner mit
Mehrprozessorarchitektur zu aktualisieren. Herr de Jager äußerte die
Befürchtung, dass es möglicherweise nicht genügend Programmierer gibt,
die mit COBOL, der klassischen Großrechner-Sprache, vertraut sind, um
die notwendigen Patches und Reparaturen für datumsabhängigen Code
durchzuführen, selbst wenn jedes Unternehmen und jede Regierungsbehörde
mit einem anfälligen System bereits vor einigen Jahren ein
Crash-Programm initiiert hätte. Da dies jedoch nicht der Fall war und
viele Betreiber datumsabhängiger Informationssysteme erst kürzlich
begonnen haben, ihre Schwachstellen zu analysieren, lässt sich mit an
Sicherheit grenzender Wahrscheinlichkeit vorhersagen, dass eine Menge
von Großrechnersystemen nicht ausreichend auf einen reibungslosen
Betrieb ins Jahr 2000 vorbereitet sein werden.

Dies stellt zweifellos ein enormes Problem dar, denn in der aktuellen
Struktur der Wirtschaft ist keine Alternative zur Datenverarbeitung
durch Computer vorhanden. Die meisten Firmen, die groß genug sind, um
einen Großrechner für ihre Geschäftsabwicklungen zu benötigen, sind auf
ein Transaktionsvolumen angewiesen, das von den veralteten
Papiersystemen des 19. Jahrhunderts unmöglich zu bewältigen wäre. Wären
diese Unternehmen gezwungen, wieder auf Papier umzusteigen, könnten sie
nur einen Bruchteil ihres gewohnten Transaktionsvolumens bewältigen. Der
Einbruch in den Einnahmen, der sich aus einer solchen Verringerung des
Geschäftsvolumens ergäbe, würde das Überleben aller Unternehmen, mit
Ausnahme der am stärksten kapitalisierten, aufs Spiel setzen.

Nahezu alle finanzbezogenen Aspekte - Fakturierungssysteme, Einkaufs-
und Gehaltsabrechnungen, Lagerverwaltung und die Einhaltung gesetzlicher
Vorschriften - könnten ins Chaos gestürzt werden. Unmengen an Daten
könnten verloren gehen, wenn Computer abstürzen oder infolge des
Jahr-2000-Problems fehlerhafte Daten produzieren. In manchen Fällen
könnte es sich sogar als Glücksfall erweisen, wenn die Systeme sofort
abstürzen, anstatt ihre Daten schrittweise zu beschädigen, bis
gravierende Störungen auf das Problem aufmerksam machen. Was geschieht
mit Dateien, die von einem Backup-Programm vom 7. April
\textquotesingle99 auf eine Version vom 1. April \textquotesingle00
kopiert werden? Wer kann das schon sagen? Wird ein Computer eine am 4.
Januar „1900`` getätigte Versicherungszahlung als Anzeichen deuten, dass
die Zahlung seit einem Jahrhundert überfällig ist, was zu ihrer
Annullierung und Löschung aus den Akten führt? Werden Bank- und
Finanzcomputer versuchen, hundert Jahre Zinsen auf Kredite zu berechnen,
die das neue Jahrtausend überspannen? Werden Ihre Banken und
Broker-Firmen genaue Aufzeichnungen Ihrer Kontostände führen und Ihnen
zeitnah Zugang zu Ihren Geldern gewähren? Dies sind nur einige der
faszinierenden Fragen, die sich im Zusammenhang mit dem
Jahr-2000-Problem ergeben werden.

\begin{quote}
„Dies ist der womöglich verheerendste Aspekt des Jahr-2000-Problems. Wir
sprechen hier nicht über die Unannehmlichkeiten, die entstehen, wenn Ihr
Gehalt ein paar Tage verspätet eintrifft. Hierbei handelt es sich um den
Teil, bei dem buchstäblich das Blut auf den Straßen fließt.`` - Dr.~Leon
Kappelman, stellvertretender Vorsitzender der Jahr-2000-Arbeitsgruppe
der Gesellschaft für Informationsmanagement.
\end{quote}

Ganz oben auf der Liste all Ihrer Sorgen sollte die Frage stehen: Was
geschieht, wenn der Strom wegen der Probleme, die das Jahrtausendproblem
verursacht, ausfällt? Selbst die widerstandsfähigsten Systeme, die gar
nicht von der Jahr-2000-Problematik betroffen sind, würden ohne Strom
nicht funktionieren: Ihr Kühlschrank, Ihr Gefrierschrank, vielleicht
sogar Ihre Heizung. Jahr-2000-Probleme könnten wichtige Zutritts- und
Kontrollfunktionen in Atomkraftwerken beeinträchtigen. Beispielsweise
tragen Mitarbeiter in Atomkraftwerken Dosimeter, die die
Strahlenbelastung messen, der sie während ihres Aufenthalts in der
Anlage ausgesetzt sind. Diese Geräte werden regelmäßig analysiert und
die Daten zur Strahlenbelastung in einem Computersystem gespeichert, das
den Zutritt des Personals zur Anlage überwacht. Es leuchtet ein, dass
ein Ausfall dieser Kontrollcomputer selbst die ausgefeiltesten
Überwachungsmechanismen, die einen sicheren Betrieb und ordnungsgemäße
Wartungsarbeiten gewährleisten sollen, zunichtemachen würde. Aber noch
beunruhigender ist die in einem Memo der Kommission für die Regulierung
von Kernkraftwerken festgestellte Tatsache, dass viele „nicht
sicherheitskritische, aber dennoch wichtige computerbasierte Systeme,
vor allem Datenbanken und Datenerfassungen, die für den Betrieb der
Anlage notwendig sind``, datumsabhängig sind.

Konventionelle Kraftwerke sind keineswegs unempfindlich gegenüber
Störungen, die im Zuge des Jahrtausendwechsels auftreten können.
Insbesondere sind kohlebetriebene Kraftwerke anfällig für Ausfälle des
Oberflächentransportsystems, das Kohle zu den Kesseln bringt. In der
winterlichen Heizperiode 1997-1998 wurden Betreiber von Kohlekraftwerken
dazu gezwungen, ihre Leistung in einigen Fällen zu reduzieren. Dies
geschah, da die Bahnlieferungen von westlicher Kohle aufgrund der Fusion
der Eisenbahnsysteme Southern Pacific und Union Pacific verlangsamt
wurden. Das Problem rührte von Unverträglichkeiten zwischen den
Computersteuerungs- und Abfertigungssystemen her, die von den beiden
Eisenbahngesellschaften eingesetzt wurden. Ein Sprecher der Union
Pacific beschrieb die Integration der beiden Systeme als „Albtraum``,
und das trotz der Tatsache, dass Union Pacific Technologies als
Branchenführer in der Entwicklung von computergestützten
Transportkontrollsystemen galt. Aufgrund von Programmierschwierigkeiten
war es der Eisenbahngesellschaft nicht möglich, die Bewegungen ihrer
Güterwagen genau zu verfolgen. Dass Union Pacific die Übernahme von
Southern Pacific nicht erfolgreich bewältigen konnte, deutet darauf hin,
welche Probleme auftreten könnten, wenn die zeitlichen Logikbomben des
Jahres 2000 das Transportwesen, die Energieproduktion und andere
Wirtschaftsbereiche stören.

Die größte Sorge hinsichtlich des Stromnetzes entsteht jedoch aus der
Notwendigkeit, dass das gesamte System einer sensiblen Überwachung und
Computersteuerung unterliegt, um Strom von Überschuss- zu
Defizitregionen zu leiten. Dieser Ablauf muss sorgsam per Computer
überwacht werden, um Stromspitzen und Systemausfälle zu verhindern. Alle
durchgeführten Stromübertragungen werden mit Datum und Uhrzeit
dokumentiert, ähnlich wie bei einer Telefonverbindung. Die Verbindungen
selbst werden mittels robuster mechanischer Relais realisiert, welche
allerdings von Computersystemen gesteuert werden. Diese
Computersteuerungen, unerlässlich für die Lastverteilung, können aus
denselben Gründen wie Telekommunikationsnetzwerke ausfallen. Tatsächlich
sind die Systeme zur Steuerung der Lastverteilung in Nordamerika
miteinander über T-1-Leitungen und Telefon-Mikrowellenverbindungen
vernetzt. Wenn also das Telefonnetz ausfällt, ist es durchaus
wahrscheinlich, dass auch der Stromausfall folgt. Und wie die
Erfahrungen aus Kanada im Januar 1998 zeigen, kann es eine große
Herausforderung sein, das System erneut zum Laufen zu bringen, sobald
der Strom in einem größeren Gebiet ausfällt. Ein Blackout kann
unangenehm lange andauern.

\section{Y2K UND DIE ATOMWAFFEN}\label{y2k-und-die-atomwaffen}

Ein Stromausfall inmitten des Winters wäre für moderne Volkswirtschaften
eine Katastrophe, ganz zu schweigen von den potenziellen
Gesundheitsrisiken, insbesondere für diejenigen, die auf elektrische
Heizsysteme oder medizinische Geräte angewiesen sind. Aber die
schlimmste denkbare Situation könnte noch gravierender sein. John
Koskinen, zur damaligen Zeit Leiter des Y2K Conversion Council unter
Präsident Clinton, äußerte die Besorgnis, dass die Waffensysteme des
US-Militärs am 31. Dezember 1999 um Mitternacht versagen könnten.
Koskinen möchte zwar keine unverhältnismäßige Panik schüren, betont
jedoch: „Man muss sich darüber Gedanken machen``. Eine spezifische Sorge
hinsichtlich der Nuklearraketen sei, „wenn die Daten nicht funktionieren
und sie tatsächlich abgefeuert werden``.

Natürlich würden diese Bedenken in gleichem oder sogar größerem Maße auf
russische Atomraketen zutreffen. Der Finanzkollaps Russlands hat die
Aufrüstung auf Jahrtausend-Fähigkeiten noch problematischer gemacht als
in den USA. Zudem gibt es Anzeichen dafür, dass Russland die Problematik
der Jahrtausendwende noch nicht ernst genug nimmt. Auch wenn man
inständig hofft, dass es nicht zu unbeabsichtigten Raketenstarts kommt,
sollte es kaum Zweifel daran geben, dass der Übergang ins Jahr 2000 das
Potential hat, die globale Unsicherheit zu verschärfen. Und das aus
keinem anderen Grund als der möglichen Funktionsstörung der
militärischen Kommunikationssysteme in vielen Ländern. Wie Koskinen es
formuliert: „Wenn man in einem Land sitzt und plötzlich nicht mehr
sicher ist, was vor sich geht und die Kommunikation nicht mehr
reibungslos funktioniert, wird man noch nervöser.`` Fügen Sie das also
zu Ihrer Liste der „Y2K-Sorgen`` hinzu. Die zeitliche Logikbombe könnte
den Abschuss von tatsächlich explosiven Bomben auslösen -- eine
Tatsache, die die Gefahren einer Informationskriegsführung für
zentralisierte Befehls- und Kontrollsysteme verdeutlicht.

Sollten Terroristen das Ziel haben, ein zentralisiertes System
anzugreifen, könnten sie als Termin hierfür den 31. Dezember 1999 ins
Auge fassen, da zu diesem Zeitpunkt viele Systeme äußerst verwundbar
sein könnten. Dabei geht es nicht allein darum, dass die Kommunikation
mutmaßlich gestört sein wird - womöglich fällt der Strom aus, Fahrzeuge
springen nicht mehr an, die Notrufsysteme von Polizei, Feuerwehr und
Krankenwagen versagen und so weiter. Auch viele andere Funktionen, die
wir für selbstverständlich halten, wie zum Beispiel die
Flugverkehrskontrolle, könnten zum Erliegen kommen. Ohne Strom gibt es
kein Wasser aus dem Hahn, Abwassersysteme würden zusammenbrechen.
Verkehrsampeln könnten ausfallen. Nur wenige Stunden nach einem
vollständigen Verkehrskollaps könnten die Lebensmittelläden leergeräumt
(oder geplündert) sein. Basierend auf jüngsten Erfahrungen in
amerikanischen Städten lässt sich mutmaßen, dass ein Mangel an Strom,
Wasser und Wärme für viele Menschen, kein Licht und eine unzuverlässige
Kommunikation mit den Notdiensten - einschließlich Polizei und Feuerwehr
- letztlich zum Zusammenbruch der Zivilisation führen könnte. Zwar kann
niemand verbindlich prognostizieren, welche Auswirkungen das
Jahr-2000-Problem haben wird, doch könnten Plünderungen und
Ausschreitungen auf den Straßen die Folge sein, insbesondere wenn
bekannt wird, dass es wahrscheinlich zu weitreichenden Ausfällen bei
Gehalts-, Sozialhilfe- und Rentenauszahlungen kommen könnte.

\begin{quote}
„Wir werden nicht länger das sein, was wir einst waren, sondern
beginnen, uns zu verändern.`` - Joachim de Fiore\footnote{Zitiert in
  Frooso, op. cit., S. 40.}
\end{quote}

Die düsteren Prophezeiungen für das neue Jahrtausend basieren nicht
zwangsläufig auf einer christlich geprägten Theologie, doch sie lassen
sich gut in die Jahrtausendtradition von Joachim de Fiore einordnen.
Seine Meditationen überzeugten ihn davon, dass Christus lediglich den
„zweiten Wendepunkt der Geschichte`` darstellt und sich „ein weiterer
unweigerlich entfalten würde``.\footnote{Ebenda.} So argumentiert auch
der Philosoph Michael Grosso, der behauptet, dass die informationelle
Revolution die Menschheitsgeschichte in Richtung der Verwirklichung der
prophetischen Vision der westlichen Welt lenkt. Er bezeichnet dies als
„Technokalypse``. Unabhängig davon, ob die technologische Entwicklung
auf irgendeine Weise von den Visionen des neuen Jahrtausends beeinflusst
wird oder nicht, stellt das Jahr-2000-Phänomen ein Artefakt der
vorherrschenden westlichen Zeitvorstellung dar. Auf ungewöhnliche Weise
könnte es Träume, Fantasien und Visionen oder numerische Deutungen von
Visionen ergänzen, wie etwa Newtons Erläuterung der Prophezeiungen
Daniels. Solche intuitiven Sprünge beginnen stets mit der Betrachtung
der Geburt Christi als zentralem historischen Ereignis. Sie werden durch
die psychologische Kraft großer runder Zahlen verstärkt, die jeden
Händler in ihren Bann zieht. Das Jahr 2000 kann nicht umhin, ein
Brennpunkt für die Fantasie intuitiver Menschen zu werden.

Kritiker mögen diese Prophezeiungen als lächerlich beiseite fegen, ohne
sich mit den mehrdeutigen und umstrittenen theologischen Konzepten der
Apokalypse und des Jüngsten Gerichts auseinanderzusetzen, die diesen
Visionen eine gewaltige Kraft verleihen. Interessanterweise übertrumpft
der Jahr-2000-Computerbug jedoch die rechnerischen Fehler, die ansonsten
die Bedeutung des Jahres 2000 selbst im christlichen Rahmen zu entwerten
scheinen. Das Jahr 2000 hat das Potential, einen Wendepunkt für die
nächste Phase der Geschichte darzustellen, allein aufgrund der
vorgezogenen Ankunft des neuen Jahrtausends. Streng genommen beginnt das
nächste Jahrtausend erst im Jahr 2001. Das Jahr 2000 markiert lediglich
das zweitausendste Jahr nach Christi Geburt. Genau genommen wäre dies
der Fall gewesen, wenn Christus im ersten Jahr der christlichen Ära zur
Welt gekommen wäre. Aber das war nicht der Fall. Im Jahr 533, als das
Geburtsdatum von Christus die Gründung Roms als Grundlage für die
Zeitrechnung nach westlichem Kalender ersetzte, machten die Mönche, die
diese neue Konvention einführten, einen Fehler in Bezug auf das
Geburtsjahr Christi. Heutzutage wird angenommen, dass er im Jahr 4 v.
Chr. geboren wurde. Unter dieser Prämisse, wären die zweitausend Jahre
seit seiner Geburt irgendwann im Jahr 1997 vollendet gewesen. Das
erklärt Carl Jungs scheinbar kurioses Anfangsdatum für den Beginn eines
neuen Zeitalters.

Sie dürfen gern schmunzeln, aber wir verachten oder ignorieren
keinesfalls das intuitive Verständnis der Geschichte. Obgleich unsere
Argumentation auf Logik und nicht auf Annahmen beruht, sind wir dennoch
vom prophetischen Potential des menschlichen Bewusstseins beeindruckt.
Immer wieder bestätigen sich die Visionen von Außenseitern, Sehern und
Heiligen. So könnte es auch bei der Wende des Jahres 2000 der Fall sein.
Dieses Datum, das sich seit langem in den Köpfen des Westens verankert
hat, scheint einen Wendepunkt zu markieren, der zumindest partiell
bestätigt, dass Geschichte ein Schicksal hat. Wir können nicht erklären,
warum das so ist, doch wir sind überzeugt davon, dass es so ist.

Unsere Intuition lässt uns glauben, dass Geschichte einem Schicksal
folgt, und dass freier Wille und Determinismus nur zwei Seiten der
gleichen Medaille sind. Es scheint, als würden die menschlichen
Interaktionen, die die Geschichte prägen, von einer Art Schicksal
geleitet werden. Menschen verhalten sich genauso wie ein
Elektronenplasma, ein dichtes Gas aus Elektronen -- als komplexes
System. Die individuelle Bewegungsfreiheit der Elektronen passt
überraschend gut zu hochgradig organisiertem kollektivem Verhalten. Wie
David Ohm einmal über ein Elektronenplasma sagte, so ist auch die
menschliche Geschichte „ein hochgradig organisiertes System, das als
Einheit agiert``.

Um die Funktionsweise der Welt zu begreifen, muss man sich ein
realistisches Bild davon machen, wie sich die menschliche Gesellschaft
den mathematischen Gesetzen natürlicher Prozesse unterordnet. Die
Realität verläuft nicht linear, aber die Erwartungen der meisten
Menschen schon. Um das Wesen des Wandels zu durchdringen, sollte man
sich bewusst machen, dass menschliche Gesellschaften, ähnlich wie andere
komplexe Systeme in der Natur, durch wiederkehrende Zyklen und
Diskontinuitäten geprägt sind. Das heißt, bestimmte geschichtliche
Merkmale neigen dazu, sich zu wiederholen und die signifikantesten
Veränderungen treten, wenn sie denn auftauchen, eher schlagartig als
allmählich auf.

Im menschlichen Leben gibt es zahlreiche Zyklen, doch scheint ein
mysteriöser Fünfhundert-Jahres-Zyklus bedeutsame Meilensteine in der
Geschichte der westlichen Zivilisation zu bestimmen. Während wir uns dem
Jahr 2000 nähern, werden wir mit der merkwürdigen Beobachtung
konfrontiert, dass das letzte Jahrzehnt jedes Jahrhunderts, das durch
fünf teilbar ist, einen signifikanten Wandel in der westlichen
Zivilisation eingeläutet hat. Es bildet ein Muster von Tod und
Wiedergeburt, das neue Stadien der gesellschaftlichen Organisation
kennzeichnet, so wie Tod und Geburt den Zyklus der menschlichen
Generationen beschreibt. Dieses Phänomen lässt sich mindestens bis ins
Jahr 500 v. Chr. zurückverfolgen, als die griechische Demokratie mit den
Verfassungsreformen des Kleisthenes im Jahr 508 v. Chr. ihren Anfang
nahm. Die darauffolgenden fünf Jahrhunderte stellten eine Periode des
Wachstums und der Intensivierung der antiken Ökonomie dar, die ihren
Höhepunkt in der Geburt Christi im Jahr 4 v. Chr. fand. Dies war auch
die Zeit des größten Wohlstands der antiken Wirtschaft, in der die
Zinssätze ihren niedrigsten Stand vor der Neuzeit erreichten.

Im Verlauf der fünf darauffolgenden Jahrhunderte nahm der Wohlstand
schrittweise ab, was schlussendlich zum Zusammenbruch des römischen
Reiches gegen Ende des 5. Jahrhunderts n.~Chr. führte. Es lohnt sich,
William Playfairs Zusammenfassung zu zitieren: „Als Rom auf dem Gipfel
seiner Macht angelangt war... konnte man feststellen, dass dies zur Zeit
der Geburt Christi war, also während der Herrschaft des Augustus, und
ebenso stellt man fest, dass es bis 490 n.~Chr. stetig zurückging.`` Zum
besagten Zeitpunkt lösten sich die letzten Legionen auf und die
westliche Welt versank in den Wirren des dunklen
Mittelalters.\footnote{William Playfair, \emph{An Inquiry into the
  Permanent Causes of the Decline and Fall of Powerful and Wealthy
  Nations: Designed to Shew How the Prosperity of the British Empire May
  be Prolonged} (London: Greenland and Norris, 1805), S.79.}

In den kommenden fünf Jahrhunderten erlebte die Wirtschaft einen
erheblichen Verfall, der Fernhandel erlag einer Stagnation, Städte
wurden entvölkert, Geld verschwand aus dem Umlauf und Kunst und
Alphabetisierung gingen nahezu vollständig verloren. Mit dem
Zusammenbruch des römischen Reiches im Westen und dem daraus
resultierenden Mangel an wirksamem Recht entstanden primitivere Regeln
zur Streitbeilegung. Die Blutrache gewann gegen Ende des fünften
Jahrhunderts mehr und mehr an Relevanz. Der erste historisch belegte
Gerichtsprozess fand schließlich im Jahr 500 statt.

Einmal mehr, vor einem Jahrtausend, ereignete sich im letzten Jahrzehnt
des zehnten Jahrhunderts eine „monumentale Veränderung der sozialen und
wirtschaftlichen Systeme``. Eine dieser Übergänge, wahrscheinlich die am
wenigsten bekannte, die feudale Revolution, begann inmitten einer
Periode voller wirtschaftlicher und politischer Turbulenzen. Guy Bois,
ein Professor für Mittelaltergeschichte an der Universität Paris,
argumentiert in seinem Buch „Umbruch im Jahr 1000``, dass dieser Wandel
am Ende des zehnten Jahrhunderts den völligen Zusammenbruch der Reste
alter Institutionen und das Entstehen von etwas Neuem aus der Anarchie
des Feudalismus bedeutete.\footnote{Guy Bois, \emph{The Transformation
  of the Year One Thousand: The Village of Lournard from Antiquity to
  Feudalism} (Manchester, England: Manchester University Press, 1992).}
Raoul Glaber formuliert es so: „Man sagte, die gesamte Welt schüttelte
einvernehmlich die Trümmer der Antike ab.`` \footnote{Ebenda, S. 150.}
Das neu aufkommende System ermöglichte eine allmähliche Wiederbelebung
des wirtschaftlichen Wachstums. In den fünf Jahrhunderten, die wir heute
als Mittelalter bezeichnen, erlebten wir eine Renaissance des Geldes und
des internationalen Handels. Auch die Arithmetik, das Lesen und
Schreiben und ein Bewusstsein für Zeit wurden wiederentdeckt.

Im letzten Jahrzehnt des 15. Jahrhunderts erreichte Europa einen
weiteren Meilenstein. An diesem Punkt hatte es das durch die Pest
verursachte demographische Defizit überwunden und übernahm unmittelbar
im Anschluss fast die gesamte Kontrolle über die restliche Welt. Der
Übergang in dieses neue Zeitalter, gekennzeichnet durch die
„Schießpulverrevolution``, die „Renaissance`` und die „Reformation``,
wurde mit dem Einmarsch Karls VIII. in Italien unter Einsatz von neuen
Bronzekanonen einprägsam eingeleitet. Dies ging einher mit der Öffnung
Europas zur Welt, verkörpert durch Christoph Kolumbus' Reise nach
Amerika im Jahr 1492. Diese Öffnung für die neue Welt startete das
bisher dramatischste wirtschaftliche Wachstum in der menschlichen
Geschichte, führte zum Umbruch von Physik und Astronomie und in der
Konsequenz zur Entstehung moderner Wissenschaften. Und die durch diese
Epoche generierten Ideen wurden mit Hilfe der neuartigen Technologie der
Druckerpresse weitflächig verbreitet.

Wir stehen nun am Vorabend eines weiteren Jahrtausendwechsels. Die
großen Befehls- und Kontrollsysteme, die aus dem Industriezeitalter
stammen, könnten mit dem Glockenschlag der tausendjährigen Mitternacht
kollabieren, ähnlich wie eine einspännige Kutsche. Unabhängig davon, ob
die sogenannte „Jahr-2000-Logikbombe`` einen unmittelbaren Zusammenbruch
der Industriegesellschaft herbeiführt oder nicht - ihre Tage sind
gezählt. Wir gehen davon aus, dass das Aufkeimen der
Informationsgesellschaft tiefgreifende Veränderungen auf der Welt
bewirkt, welche in diesem Buch dargelegt werden sollen. Es steht Ihnen
natürlich frei, dies in Frage zu stellen; kein Zyklus, der nur zweimal
in einem Jahrtausend stattfindet, hat genügend Wiederholungen geboten,
um statistisch signifikant zu sein. Selbst erheblich kürzere Zyklen sind
von Ökonomen mit Skepsis betrachtet worden, die nach statistisch
stichhaltigeren Beweisen verlangten. Professor Dennis Robertson merkte
einst an, dass wir „besser einige Jahrhunderte abwarten sollten, bevor
wir uns der Existenz von Vierjahres- sowie Acht- bis
Zehnjahres-Handelszyklen sicher sind``.\footnote{Zitiert in S. B. Saul,
  \emph{The Myth of the Great Depression} (London: Macmillan, 1985), S.
  10.} Nach diesem Maßstab müsste Professor Robertson sein Urteil rund
dreißigtausend Jahre aufschieben, um sicherzustellen, dass der
Fünfhundertjahreszyklus kein statistischer Zufall ist. Wir sind weniger
dogmatisch und eher bereit anzuerkennen, dass die Muster der Realität
komplexer sind als die statischen und linearen Gleichgewichtsmodelle,
die die meisten Wirtschaftswissenschaftler betrachten.

Wir sind der Überzeugung, dass das Jahr 2000 mehr als nur eine weitere
zweckdienliche Teilung im unendlichen Zeitgefüge bedeutet. Wir glauben,
es wird einen Wendepunkt zwischen der alten und der herannahenden neuen
Welt darstellen. Das Industriezeitalter neigt sich rapide dem Ende zu
und paradoxerweise könnte der Untergang durch die ursprünglich hohen
Kosten für Computerspeicher beschleunigt worden sein, die zur
weitflächigen Implementierung von zweistelligen Datumsfeldern führten.
Als Hallerith-Lochkarten lediglich achtzig Zeichen speichern konnten,
schien eine Verkürzung der Datumsangaben sinnvoll. Entgegen der
Befürchtungen der frühen Programmierer hat ihre Verkürzung des
Datumsfelds jedoch vier Jahrzehnte bis zum Jahrtausendende überdauert,
als eine ungewollte zeitliche Logikbombe, die große Teile der
Industriegesellschaft zerstören könnte. Das Office of Management and
Budget der US-Regierung beschrieb das Problem in „Getting Federal
Computers Ready for 2000``, einem Bericht vom 7. Februar 1997. Das OMB
kommt hinsichtlich von Computern zu dem Schluss: „Wenn sie nicht
repariert oder ersetzt werden, werden sie zur Jahrtausendwende auf eine
von drei Arten versagen: Sie werden gültige Eingaben ablehnen, falsche
Ergebnisse berechnen oder schlichtweg nicht funktionieren.`` Diese drei
Szenarien könnten gemeinsam die Industriegesellschaft lahmlegen. Die
Massenproduktionstechnologie wird unweigerlich von neuer
Miniaturisierungstechnologie verdrängt werden. Eine kurzfristige Krise
würde diesen Prozess nur beschleunigen. Mit der neuen
Informationstechnologie ist eine neue Wissenschaft der nichtlinearen
Dynamik entstanden, deren verblüffende Schlussfolgerungen lediglich Teil
eines noch zu strickenden umfangreichen Weltbildes sind. Wir leben im
Computerzeitalter, aber unsere Träume werden immer noch am Webstuhl
gesponnen. Wir leben weiterhin in den Metaphern und Denkmustern des
Industrialismus. Unsere Politik verläuft immer noch entlang der
industriellen Spaltung zwischen Rechts und Links, wie sie von Denkern
wie Adam Smith und Karl Marx skizziert wurde, die starben, bevor
praktisch alle heute lebenden Menschen geboren wurden.\footnote{Adam
  Smith starb 1790, Karl Marx 1883.} Die industrielle Weltanschauung,
die die Funktionsprinzipien der industriellen Wissenschaft umfasst, ist
immer noch der intuitiv wahrgenommene „gesunde Menschenverstand`` der
unterrichteten Meinung. Unsere Hypothese ist, dass der „gesunde
Menschenverstand`` des Industriezeitalters in vielen Bereichen nicht
mehr greifen wird, da sich die Welt verändert.

Über 85 Jahre nach dem Tag im Jahr 1911, an dem Oswald Spengler die
Vision eines bevorstehenden Weltkriegs und des „Untergangs des
Abendlands`` hatte, erleben auch wir einen „historischen
Paradigmenwechsel, der sich genau an jenem Punkt vollzieht, der ihm vor
Jahrhunderten vorherbestimmt war.`` \footnote{Oswald Spengler, \emph{The
  Decline of the West}, Übersetzung ins Englische von Charles Francis
  Atkinson, zitiert in I. F. Clark, \emph{The Pattern of Expectation},
  1644-2001 (London: Jonathan Cape, 1979), S. 220.} Wie Spengler
prophezeien auch wir den nahenden Untergang der westlichen Zivilisation
und damit den Zusammenbruch der Weltordnung, die die vergangenen fünf
Jahrhunderte dominierte, seit Kolumbus nach Westen gesegelt ist, um den
Kontakt zur Neuen Welt herzustellen. Im Gegensatz zu Spengler sehen wir
jedoch das Aufkommen einer neuen Phase der westlichen Zivilisation im
heraufziehenden neuen Jahrtausend.

\bookmarksetup{startatroot}

\chapter*{Nachwort: Dezentralisierung und das Gesetz des abnehmenden
Ertrags}\label{nachwort-dezentralisierung-und-das-gesetz-des-abnehmenden-ertrags}
\addcontentsline{toc}{chapter}{Nachwort: Dezentralisierung und das
Gesetz des abnehmenden Ertrags}

\markboth{Nachwort: Dezentralisierung und das Gesetz des abnehmenden
Ertrags}{Nachwort: Dezentralisierung und das Gesetz des abnehmenden
Ertrags}

\begin{quote}
„Was über seine Maße aufgebläht ist, wird unweigerlich
zusammenbrechen\ldots{} Was konzentriert, kohärent und mit seiner
Vergangenheit verbunden ist, ist mächtig. Was verschwenderisch, geteilt
und aufgebläht ist, verrottet und fällt zu Boden. Je aufgeblähter es
ist, desto heftiger der Fall.'' --- Robert Greene und Joost Elffers,
Power: Die 48 Gesetze der Macht\footnote{Cited in \emph{Wired}, March
  1999, p.~33.}
\end{quote}

Bisher ist die Geschichte der menschlichen Gesellschaften durch eine
Tendenz gekennzeichnet, sich in Richtung größerer „Komplexität'' oder
sozialpolitischer Kontrolle zu entwickeln. Kleine Jäger- und
Sammlergruppierungen entwickelten sich zu landwirtschaftlichen Staaten,
die schließlich größeren industriellen Nationalstaaten Platz machten.
Wie der Archäologe und Historiker Joseph A. Tainter in seinem Werk
\emph{The Collapse of Complex Societies} schreibt: „Die Geschichte der
Menschheit insgesamt ist durch eine scheinbar unaufhaltsame Tendenz zu
höheren Komplexitäts-, Spezialisierungs- und sozialpolitischen
Kontrollniveaus gekennzeichnet. \ldots{}``\footnote{Joseph A. Tainer,
  \emph{The Collapse of Complex Societies} (Cambridge, Mass.: Cambridge
  University Press, 1988), p.~3.} Doch nun verspricht das Aufkommen der
nächsten Entwicklungsstufe der Wirtschaft, der Informationsgesellschaft,
eine Umkehrung des scheinbar „unaufhaltsamen Trends'' zu höheren Graden
der Zentralisierung.

Tainers Arbeit wirft viele interessante Fragen auf, die für die Themen
dieses Buches relevant sind. Beispielsweise, wenn Tainer recht hat mit
der Annahme, dass die Dezentralisierung von Kontrolle und eine
verringerte Ressourcenumverteilung einen Zusammenbruch implizieren, dann
ist es unwahrscheinlich, dass der industrielle Nationalstaat in seiner
gegenwärtigen Form langfristig mit dezentralisierten Mikrostaaten
koexistieren kann, die souveräne Individuen beherbergen. Die
Nationalstaaten sind möglicherweise nicht in der Lage, mit
gleichbleibenden, geschweige denn verringerten Ressourcen auszukommen.
Wie Tainer detailliert beschreibt, tritt bei hypertrophierten Systemen,
die ihr Potenzial erschöpft haben -- wie wir glauben, dass es bei den
Nationalstaaten heute der Fall ist -- häufig „das Gesetz des abnehmenden
Grenzertrags'' ein. In „vielen entscheidenden Bereichen'' sinken die
Erträge aus vermehrten Investitionen in die zentralisierte
sozialpolitische Kontrolle oder werden sogar negativ. Dies erklärt das
Phänomen der „Parkinsonschen Gesetze'', bei dem die Zahl der
Angestellten und die Kosten für den Betrieb der britischen Admiralität
im zwanzigsten Jahrhundert in die Höhe schnellen, während die Anzahl der
Schiffe in der britischen Marine dramatisch zurückgeht.

Ähnliche Erscheinungsformen des „Gesetzes der abnehmenden Erträge'' sind
sicherlich zu beobachten, während das zwanzigste Jahrhundert sich dem
Ende zuneigt, sowohl in den Vereinigten Staaten als auch in anderen
führenden Volkswirtschaften. Wie Roger Lane, Professor für
Sozialwissenschaften am Haverford College, in seinem Aufsatz „On the
Social Meaning of Homicide Trends in America'' schrieb: „Die alten
Institutionen der sozialen Kontrolle -- Gesetze, Schulen, Polizei,
Gefängnisse -- haben an Wirksamkeit verloren, trotz häufiger Zuführungen
von Arbeitskräften und Geld.''\footnote{Roger Lane, \emph{On the Social
  Meaning of Homicide Trends in America}, in \emph{Violence in America},
  Vol. 1, ed.~Ted Robert Gurr (Newbury Park: Sage Publications, 1989),
  p.~57.} Es gibt eindeutige Belege für steigende Kosten, die mit den
Gesamterfordernissen der Staatsführung verbunden sind. So sind die
Gesamteinnahmen aus Steuern von 27,8 Prozent des mittleren Einkommens in
den USA im Jahr 1957 auf 37,6 Prozent im Jahr 1997 gestiegen.\footnote{See
  Robert Higgs, \emph{A Carnival of Taxation}, The Independent Review,
  Volume III, Number 3, Winter 1999, p.~437.} Das ist ein starkes Indiz,
wenn nicht sogar ein eindeutiger Beweis für abnehmende Grenzerträge im
gesamten Bereich staatlicher Aktivitäten in den Vereinigten Staaten.

In der Vergangenheit waren stark abnehmende Grenzerträge häufig Vorboten
eines Zusammenbruchs. Die Argumentation dieses Buches besagt, dass die
gesteigerte Fähigkeit von Individuen, ihre Transaktionen und
Vermögenswerte vor räuberischen Steuern zu schützen, einen Rückgang der
Umverteilung von Ressourcen zur Folge hat, sowie eine geringere zentrale
soziale Kontrolle, weniger Regulierung und Vereinheitlichung und
letztlich eine Dezentralisierung des Territoriums. All diese
Entwicklungen haben sich historisch in Form eines „Kollaps''
manifestiert. In Tainers Worten bezeichnet „Kollaps'' den Zustand, wenn
ein zentralisiertes Kontrollsystem nicht mehr das wert ist, was es
kostet.

\begin{quote}
„Wann immer wir ein Schwellenphänomen beobachten, sei es in physischen,
biologischen oder sozialen Systemen, wird die Konfiguration des Systems
in dem Moment, in dem die Schwelle erreicht wird, instabil. Bereits die
geringste, selbst infinitesimale Veränderung in der Konfiguration des
Systems kann daher eine Veränderung im Verhalten eines einzelnen
Individuums, egal wie klein sie auch sein mag, auslösen, die in einem
instabilen sozialen Konfigurationsprozess zu einer begrenzten und
manchmal radikalen Veränderung führt.'' --- Nicholas Rashevsky,
\emph{Looking at History Through Mathematics}\footnote{Nicholas
  Rashevsky, \emph{Looking at History Through Mathematics} (Cambridge,
  Massachusetts: MIT Press, 1968), p.~119.}
\end{quote}

Während die meisten individuellen Anpassungen an Veränderungen
zugegebenermaßen marginal und evolutionär geprägt sind, kann es durchaus
revolutionäre „Paradigmenwechsel'' geben. Manchmal stürzen sogar große
Reiche als Folge hiervon. Die Grenzerträge aus weiteren Investitionen in
zentrale Kontrolle können so überwältigend negativ werden, dass es für
die meisten Individuen nicht mehr wirtschaftlich rational ist, das alte
System weiterhin zu unterstützen. Tainter erklärt den Fall des Römischen
Reiches mit diesen Worten: „Wenn man den Berichten Glauben schenken
darf, begrüßt zumindest ein Teil der überbesteuerten Landbevölkerung
offen die Erleichterung, die sie sich von den Barbaren erhofften, um die
Lasten der römischen Herrschaft loszuwerden. Ein wesentlich größerer
Teil war offensichtlich apathisch gegenüber dem bevorstehenden
Zusammenbruch\ldots{} Die Kosten des Imperiums waren dramatisch
gestiegen, während angesichts der Erfolge der Barbaren der Schutz, den
der Staat vielen seiner Bürger bieten konnte, zunehmend ineffektiv
wurde. Für viele gab es schlichtweg keine verbleibenden Vorteile des
Imperiums, da sowohl Barbaren als auch Steuerbeamte ihr Land
durchquerten und verwüsteten. Wie Gunderson feststellt: ‚\ldots{} der
Nettowert lokaler Autonomie überstieg den der Mitgliedschaft im
Imperium.' Komplexität brachte keine Vorteile mehr, die die Zersetzung
übertrafen, stattdessen kostete sie erheblich mehr.'' \footnote{Tainter,
  op. cit., pp 150-51.}

Tainter zitiert andere Autoritäten, um seine These zu untermauern, dass
ein Zusammenbruch „eine entsprechende Steigerung des Grenzertrags von
sozialen Investitionen mit sich bringen kann.''

\begin{quote}
„Zosimus, ein Schriftsteller aus der zweiten Hälfte des fünften
Jahrhunderts n.~Chr., schrieb über Thessalien und Makedonien, dass
‚\ldots{} infolge dieser Steuererhebungen Stadt und Land voller Klagen
und Beschwerden waren und alle die Barbaren anriefen und um ihre Hilfe
baten.' \ldots{} ‚Im fünften Jahrhundert,' schlussfolgert R.M. Adams,
‚waren die Menschen bereit, die Zivilisation selbst aufzugeben, um dem
erschreckenden Steuerdruck zu entkommen.'\,'' \footnote{Tainter, op.
  cit., p.~147.}
\end{quote}

Rashevskys Analyse der „Rolle des Determinismus versus Indeterminismus''
in der Geschichte hebt die Verwundbarkeit von Systemen für radikale
Veränderungen hervor, die sogar durch eine einzige Person ausgelöst
werden können, wenn das System instabil wird und einen „Schwellenwert''
erreicht. Wenn die Bedingungen für Veränderungen günstig sind
(beispielsweise wenn die Grenzerträge für die Unterstützung eines
zentralen Systems nicht mehr „überlegene Vorteile gegenüber der
Zersetzung'' bieten), ist die Möglichkeit eines radikalen Wandels so
stark, dass praktisch jeder ihn herbeiführen kann. Rashevsky schreibt:
„Das Individuum, das eine endliche Veränderung herbeiführt, muss kein
\emph{außergewöhnliches} Individuum sein. Es kann \emph{jedes}
Individuum sein. Die Situation verhält sich analog zu einem
physikalischen System, in dem an einem Punkt der Instabilität eine
zufällige Verschiebung eines der Billionen identischen Moleküle einen
endlichen Übergang zu einem stabilen Zustand auslöst.''\footnote{Rashevsky,
  op. cit., pp.~119-20.}

Wir können nicht genau bestimmen, wer den Zusammenbruch des überdehnten
Nationalstaatensystems herbeiführen wird oder wann dies geschehen wird.
Aber aus den Analysen von Tainter und Rashevsky über die Dynamik des
sozialen Wandels können wir einen bevorstehenden Zusammenbruch erahnen.
Die am weitesten entwickelten und bislang erfolgreichen Nationalstaaten
zeichnen sich alle durch schrumpfende Bevölkerungen und massive, nicht
finanzierte Rentenverpflichtungen aus. Ohne beispiellose Einwanderung
aus unterentwickelten Ländern oder einen unerwarteten Zustrom von
Engeln, die bereit sind, Überstunden zu leisten und konfiskatorische
Steuersätze zu bezahlen, werden die führenden Staaten in Europa,
Nordamerika und Australasien bei den Einnahmen weit hinter dem
zurückbleiben, was nötig wäre, um die derzeitigen sozialen Leistungen
aufrechtzuerhalten. Versicherungsmathematiker prognostizieren steigende
Steuern und sinkende Nutzen, d.h. abnehmende Grenzerträge, insbesondere
für Unternehmer, die einen unverhältnismäßig hohen Anteil der Steuerlast
tragen.

Die Zahlen des amerikanischen Finanzamts zeigen, dass im Jahr 1997 ein
Zehntel Prozent der Amerikaner den Großteil der Einkommensteuern in den
Vereinigten Staaten zahlte. Es sind genau diese Personen, denen
effiziente Mini-Souveränitäten neue Möglichkeiten für ihren Wohnsitz zu
vernachlässigbaren Steuern bieten können. Der Unterschied zwischen den
Schutzkosten einer kommerzialisierten Souveränität und den
ausbeuterischen Steuern, die von den alten Nationalstaaten auferlegt
werden, könnte sich auf viele Millionen oder sogar mehrere Milliarden
Dollar an lebenslangem Einkommen summieren.

Die herkömmliche Mikroökonomie beruht auf der Annahme, dass Individuen,
die einen 100-Dollar-Schein auf der Straße sehen, ihn aufheben werden.
Die Möglichkeiten, Millionen oder Milliarden zu sparen, wären
zehntausend- oder sogar millionenfach überzeugender. Menschen werden in
der beschriebenen Weise handeln, wenn sie vor der Wahl stehen, ihre
kostspielige Loyalität gegenüber Institutionen aufrechtzuerhalten, die
von abnehmenden Grenzerträgen betroffen sind, oder sich neuen Strukturen
zuzuwenden, die weniger verlangen und mehr versprechen.

\begin{quote}
„Von allen 36 Möglichkeiten, aus der Klemme zu kommen, ist der beste
Weg: einfach weggehen.'' --- Chinesisches Sprichwort
\end{quote}

Das Argument dieses Buches legt eindrücklich dar, warum es an der Zeit
ist, Ihr Kapital neu zu verteilen, sofern Sie über welches verfügen. Die
Staatsbürgerschaft ist überholt. Um Ihr lebenslanges Einkommen zu
optimieren und zu einem souveränen Individuum zu werden, müssen Sie
vielmehr Kunde von Regierung oder Schutzdiensten werden, anstatt sich
als Bürger zu verstehen. Anstatt die Steuerlast zu tragen, die Ihnen von
gierigen Politikern auferlegt wird, sind Sie besser aufgestellt, um im
Informationszeitalter erfolgreich zu sein, indem Sie sich die Freiheit
verschaffen, einen privaten Steuervertrag auszuhandeln, der Sie
verpflichtet, nicht mehr für staatliche Dienstleistungen zu zahlen, als
sie Ihnen tatsächlich wert sind.

Basierend auf der Geschichte anderer dominierender Systeme, die vor dem
Zusammenbruch stehen, werden diejenigen, die sich für das
\emph{Ultimative Refugium} entscheiden und frühzeitig aussteigen, davon
profitieren. Dies zeigt sich bereits in der Flut von Gesetzen, die in
den 1990er Jahren verabschiedet wurden, um Amerikaner zu bestrafen, die
ihre Staatsbürgerschaft niederlegen. Die Gefahren einer
nationalistischen Reaktion auf die Krise des Nationalstaats machen es
wichtig, den Spielraum für Tyrannei und Unheil nicht zu unterschätzen.
Ungeachtet der Tatsache, dass das Recht auf Auswanderung in der
Unabhängigkeitserklärung der USA verankert ist, ist es wahrscheinlich,
dass die USA eine der tyrannischeren Jurisdiktionen sein werden, die das
Aufkommen einer kommerzialisierten Souveränität blockieren werden. Sie
sollten stets darauf achten, Ihr Geld nicht in einer Jurisdiktion zu
belassen, die sich das Recht anmaßt, Sie, Ihre Kinder oder Ihre Enkel
einzuziehen.

Egal wo Sie derzeit wohnen oder welche Nationalität Sie haben, um Ihren
Reichtum zu optimieren, sollten Sie anstreben, hauptsächlich in einem
anderen Land als dem Ihrer ersten Staatsbürgerschaft zu leben, während
Sie den Großteil Ihres Geldes in einer dritten Jurisdiktion anlegen,
idealerweise in einem Steuerparadies.

Um sich besser mit den Alternativen vertraut zu machen, empfehlen wir
Ihnen, umfassend zu reisen und attraktive Orte zu besuchen, an denen Sie
im Notfall das Recht auf Aufenthalt sichern möchten.

Wenn Sie wirklich ambitioniert sind, möchten Sie möglicherweise sogar
eine eigene Minisouveränität schaffen. In den Anhängen finden Sie
Kontakte, die Ihnen dabei helfen können, Ihre eigene steuerfreie Zone
oder Zona Franca zu verhandeln, von einer anerkannten Regierung, die
bereit ist, unter bestimmten Umständen Souveränität zu übertragen.

Nehmen wir an, Sie stehen noch am Anfang\ldots{}

Aber nehmen wir mal an, Sie stimmen den Grundannahmen dieses Buches zu
und sind begeistert von der Aussicht auf das Informationszeitalter,
haben jedoch nicht das nötige Kapital, um die Chancen zu nutzen und von
der kommerzialisierten Souveränität zu profitieren. Was tun Sie dann?

Jedes Rezept für einfachen Erfolg wird zwangsläufig enttäuschen. Die
Möglichkeiten, erfolgreich zu sein, sind infolge der
Informationsrevolution in Hülle und Fülle vorhanden. Welche dieser
Möglichkeiten für Sie die richtige ist, lässt sich nicht pauschal sagen.
Wenn Sie entschlossen sind, Kapital anzusammeln, um Ihr volles Potenzial
als souveräne Person zu entfalten, sollten Sie es sich zur Priorität
machen, die Werke verschiedener Gurus zu studieren und zu bewerten, die
hilfreiche Tipps zum Erfolg anbieten.

Jede gute Fachbuchhandlung oder einer der Online-Buchhändler, wie zum
Beispiel Amazon, bietet eine große Auswahl an Ratgebern zum Thema
Erfolg. Lesen Sie so viele wie möglich, nicht mit der Vorstellung, dass
eine bestimmte Reihe von Regeln Ihnen automatisch finanzielle
Unabhängigkeit verschafft, sondern mit dem Verständnis, dass Erfolg eine
Entscheidung ist. Wenn Sie erfolgreich sein möchten, müssen Sie sich mit
der Perspektive und den Gewohnheiten ausstatten, die erfolgreiche
Menschen auszeichnen.

Wenn Sie sich noch in der Phase der Berufswahl befinden, widerstehen Sie
der Versuchung, zu der einfachen Schlussfolgerung zu gelangen, dass der
beste Weg zum Erfolg im Informationszeitalter darin besteht,
Computerprogrammierer zu werden. Ja, es stimmt, dass Programmierer
während der Informationsrevolution, die im letzten Viertel des
zwanzigsten Jahrhunderts stattfand, stark nachgefragt wurden. Doch mit
zunehmender Rechenleistung hat sich auch die künstliche Intelligenz
rasant entwickelt. Ein Unternehmen namens Authorgenics hat bereits
gezeigt, dass es in der Lage ist, objektorientierte Software ohne
Programmierer zu erstellen. Sie werden nicht gut bezahlt werden, wenn
Sie etwas studiert haben, das auch mit Aladdins Lampe erledigt werden
kann. Das Problem bei der Spezialisierung auf Software oder ein anderes
schnelllebiges Feld im Zentrum der Informationsrevolution ist, dass Ihr
Fachgebiet bald veraltet sein könnte.

Dies unterstreicht die Weisheit der traditionellen liberalen Bildung,
die darauf abzielte, die Studierenden zu ermutigen, ihre kritischen
Fähigkeiten und Denkfertigkeiten zu entwickeln. Erfolg im
Geschäftsleben, wie in den meisten Lebensbereichen, hängt davon ab,
Probleme lösen zu können. Wenn Sie lernen, wie man Probleme löst, steht
Ihnen eine vielversprechende Karriere bevor. Egal, wo Sie leben, es gibt
zahlreiche Probleme, die auf Lösungen warten. In den meisten Fällen sind
diejenigen, die von Lösungen profitieren könnten oder deren Probleme Sie
angehen, bereit, Sie gut zu bezahlen, um diese Lösungen zu finden.

\bookmarksetup{startatroot}

\chapter*{Anhang: Anlaufstellen zum Erreichen von mehr
Unabhängigkei}\label{anhang-anlaufstellen-zum-erreichen-von-mehr-unabhuxe4ngigkei}
\addcontentsline{toc}{chapter}{Anhang: Anlaufstellen zum Erreichen von
mehr Unabhängigkei}

\markboth{Anhang: Anlaufstellen zum Erreichen von mehr
Unabhängigkei}{Anhang: Anlaufstellen zum Erreichen von mehr
Unabhängigkei}

\href{https://steuernsindraub.com/}{\textbf{steuernsindraub.com}}

\vspace{1em}

sind im deutschsprachigen Internet so etwas wie die Gelben Seiten des
Libertarismus. Dort finden Sie nicht nur einen Überblick über die
aktuelle Steuerlage in Deutschland, sondern auch Verlinkungen zu Social
Media Kanälen, Videos und Veranstaltungen.

\vspace{1em}

\href{https://Staatenlos.ch}{\textbf{Staatenlos.ch}}

\vspace{1em}

bietet einen direkten Überblick zum Auswandern, auf individuelle
Bedürfnisse angepasst. Das Team rund um Christoph Heuermann sind die
Experten, die Ihnen zeigen, wie Sie Ihr Leben in völliger Freiheit und
nach eigenen Wünschen gestalten können. Grenzen existieren ab jetzt
nicht mehr. Hier findet sich alles, was Sie über die Themen
Steuervermeidung und Internationalisierung wissen müssen.

\vspace{1em}

\href{https://Free-Cities.org}{\textbf{Free-Cities.org}}

\vspace{1em}

befasst sich mit dem Aufbau selbstregulierter Territorien, die
individuelle Rechte und Freiheit hochhalten. Freie Städte haben in der
Regel einen besonderen Rechtsstatus innerhalb ihres Gaststaates, der
ihnen die Autonomie gewährt, Entscheidungen zu treffen, die ihre lokale
Gemeinschaft betreffen. Diese Autonomie kann es den Freien Städten
ermöglichen, ihre eigenen Gesetze zu erlassen, Vorschriften einzuführen,
öffentliche Mittel zu beschaffen und öffentliche Dienstleistungen
unabhängig zu erbringen.

Freie Städte unterscheiden sich von anderen Arten von autonomen Gebieten
dadurch, dass sie den Schwerpunkt auf die Wahrung der individuellen
Rechte und Freiheiten legen. Im Gegensatz zu vielen anderen
Sonderrechtssystemen, die heute in der Welt üblich sind - wie z. B.
Freihandelszonen - konzentrieren sich Freie Städte darauf, mehr Freiheit
und ein besseres Leben für die gesamte Wohnbevölkerung zu bieten,
anstatt nur Vorteile für Unternehmen zu schaffen.

\vspace{1em}

\href{https://Free-communities.org}{\textbf{Free-communities.org}}

\vspace{1em}

sind bereits etablierte oder in der Entstehung befindliche Projekte des
freien Zusammenlebens.

\vspace{1em}

\href{https://Einundzwanzig.space}{\textbf{Einundzwanzig.space}}

\vspace{1em}

versammelt die deutschsprachige Bitcoin-Community. Unter dem Reiter
Community/Meetups findet sich eine Karte mit sämtlichen Stammtischen in
Deutschland, Österreich und der Schweiz, bei denen sich Bitcoiner
regelmäßig treffen und sich über diverse Themen im Bitcoinuniversum
austauschen.

\vspace{1em}

\href{https://Ef-magazin.de}{\textbf{Ef-magazin.de}}

\vspace{1em}

ist eine etablierte Bastion der Freiheit. Das Magazin eigentümlich frei
erscheint seit 1998 und widmet sich dem Thema Libertarismus und
individuelle Freiheit.

\vspace{1em}

\href{https://Misesde.org}{\textbf{Misesde.org}}

\vspace{1em}

sammelt Artikel, Veranstaltungen und Veröffentlichungen rund um die
österreichische Schule der Nationalökonomie. Die gesamte Theorie rund um
das Thema individuelle Freiheit aus wirtschaftlicher Perspektive ist
hier zu finden.

\vspace{1em}

\href{https://Atlas-Initiative.de}{\textbf{Atlas-Initiative.de}}

\vspace{1em}

schreibt sich das Wertegerüst einer freiheitlichen Gesellschaft auf die
Fahnen und unterstützt diverse freiheitliche Projekte.


\backmatter


\end{document}
