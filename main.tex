% Options for packages loaded elsewhere
\PassOptionsToPackage{unicode}{hyperref}
\PassOptionsToPackage{hyphens}{url}
%
\documentclass[
  a5paper,
  smalldemyvopaper,10pt,twoside,onecolumn,openright,extrafontsizes,hidelinks]{memoir}

\usepackage{amsmath,amssymb}
\usepackage{iftex}
\ifPDFTeX
  \usepackage[T1]{fontenc}
  \usepackage[utf8]{inputenc}
  \usepackage{textcomp} % provide euro and other symbols
\else % if luatex or xetex
  \usepackage{unicode-math}
  \defaultfontfeatures{Scale=MatchLowercase}
  \defaultfontfeatures[\rmfamily]{Ligatures=TeX,Scale=1}
\fi
\usepackage{lmodern}
\ifPDFTeX\else  
    % xetex/luatex font selection
\fi
% Use upquote if available, for straight quotes in verbatim environments
\IfFileExists{upquote.sty}{\usepackage{upquote}}{}
\IfFileExists{microtype.sty}{% use microtype if available
  \usepackage[]{microtype}
  \UseMicrotypeSet[protrusion]{basicmath} % disable protrusion for tt fonts
}{}
\makeatletter
\@ifundefined{KOMAClassName}{% if non-KOMA class
  \IfFileExists{parskip.sty}{%
    \usepackage{parskip}
  }{% else
    \setlength{\parindent}{0pt}
    \setlength{\parskip}{6pt plus 2pt minus 1pt}}
}{% if KOMA class
  \KOMAoptions{parskip=half}}
\makeatother
\usepackage{xcolor}
\setlength{\emergencystretch}{3em} % prevent overfull lines
\setcounter{secnumdepth}{5}
% Make \paragraph and \subparagraph free-standing
\makeatletter
\ifx\paragraph\undefined\else
  \let\oldparagraph\paragraph
  \renewcommand{\paragraph}{
    \@ifstar
      \xxxParagraphStar
      \xxxParagraphNoStar
  }
  \newcommand{\xxxParagraphStar}[1]{\oldparagraph*{#1}\mbox{}}
  \newcommand{\xxxParagraphNoStar}[1]{\oldparagraph{#1}\mbox{}}
\fi
\ifx\subparagraph\undefined\else
  \let\oldsubparagraph\subparagraph
  \renewcommand{\subparagraph}{
    \@ifstar
      \xxxSubParagraphStar
      \xxxSubParagraphNoStar
  }
  \newcommand{\xxxSubParagraphStar}[1]{\oldsubparagraph*{#1}\mbox{}}
  \newcommand{\xxxSubParagraphNoStar}[1]{\oldsubparagraph{#1}\mbox{}}
\fi
\makeatother


\providecommand{\tightlist}{%
  \setlength{\itemsep}{0pt}\setlength{\parskip}{0pt}}\usepackage{longtable,booktabs,array}
\usepackage{calc} % for calculating minipage widths
% Correct order of tables after \paragraph or \subparagraph
\usepackage{etoolbox}
\makeatletter
\patchcmd\longtable{\par}{\if@noskipsec\mbox{}\fi\par}{}{}
\makeatother
% Allow footnotes in longtable head/foot
\IfFileExists{footnotehyper.sty}{\usepackage{footnotehyper}}{\usepackage{footnote}}
\makesavenoteenv{longtable}
\usepackage{graphicx}
\makeatletter
\def\maxwidth{\ifdim\Gin@nat@width>\linewidth\linewidth\else\Gin@nat@width\fi}
\def\maxheight{\ifdim\Gin@nat@height>\textheight\textheight\else\Gin@nat@height\fi}
\makeatother
% Scale images if necessary, so that they will not overflow the page
% margins by default, and it is still possible to overwrite the defaults
% using explicit options in \includegraphics[width, height, ...]{}
\setkeys{Gin}{width=\maxwidth,height=\maxheight,keepaspectratio}
% Set default figure placement to htbp
\makeatletter
\def\fps@figure{htbp}
\makeatother

% typographical packages
\usepackage{microtype}
\usepackage{setspace}
\tolerance=6000
\hyphenpenalty=1000

% typographical settings for the body text
\setlength{\parskip}{0em}
\setlength{\parindent}{1em}
\linespread{1}

% DEFINITIONS TITLE PAGE / COPYRIGHT
\newcommand{\titleoriginal}{The Sovereign Individual}
\newcommand{\subtitleoriginal}{Mastering the Transition to the Information Age}
\newcommand{\yearoriginal}{2023}
\newcommand{\subtitletranslation}{Der Übergang zum Informationszeitalter}
\newcommand{\yeartranslation}{2024}
\newcommand{\stringtranslation}{Übersetzung}
\newcommand{\stringlicense}{Alle Rechte vorbehalten.}
\newcommand{\stringpublisher}{Verlag}
\newcommand{\ISBNHC}{978-9916-749-25-8}
\newcommand{\ISBNSC}{978-9916-749-26-5}
\newcommand{\ISBNEBOOK}{978-9916-749-27-2}
\newcommand{\ISBNAUDIO}{978-9916-749-29-6}
\newcommand{\press}{Konsensus Network}
\newcommand{\translatorone}{Andreas Tank}
\newcommand{\translators}{
\large\textit{\stringtranslation:}\\
\translatorone\\
}

% PHYSICAL DOCUMENT SETUP
\setstocksize{210mm}{148mm}
\settrimmedsize{210mm}{148mm}{*}
\setbinding{7mm}
\setlrmarginsandblock{15mm}{15mm}{*}
\setulmarginsandblock{15mm}{16mm}{*}
\setlength{\skip\footins}{20pt} % More space between the text and the footnote line

% FONTS
\usepackage{fontspec}
\setmainfont{stone-serif}[
    Path=./fonts/stone-serif-itc-pro/,
    Scale=0.83,
    Extension=.OTF,
    UprightFont=*-Regular,
    BoldFont=*-SemiBd,
    ItalicFont=*-MediumIt,
    BoldItalicFont=*-SemiBdIt
    ]

\setsansfont{stone-sans}[
    Path=./fonts/stone-sans/,
    Scale=0.85,
    Extension=.otf,
    UprightFont=*-Medium,
    BoldFont=*-Semibold,
    ItalicFont=*-MediumItalic,
    BoldItalicFont=*-SemiBoldItalic
    ]

\usepackage{lettrine}
\setcounter{DefaultLines}{3}
\renewcommand{\DefaultLoversize}{0.1}
\renewcommand{\DefaultLraise}{0}
\renewcommand{\LettrineTextFont}{}
\setlength{\DefaultFindent}{\fontdimen2\font}
\setlength{\DefaultNindent}{0em}

%\usepackage[footskip=8mm]{geometry}

% custom second title page
\makeatletter
\newcommand*\halftitlepage{\begingroup % Misericords, T&H p 153
  \setlength\drop{0.1\textheight}
  \begin{center}
  \vspace*{\drop}
  \rule{\textwidth}{0in}\par
  {\Large\sffamily\thetitle\par}
  \rule{\textwidth}{0in}\par
  \vfill
  \end{center}
\endgroup}
\makeatother

% custom title page
\makeatletter
\newlength\drop
\newcommand*\titleM{\begingroup % Misericords, T&H p 153
  \setlength\drop{0.15\textheight}
  \begin{center}
  \vspace*{\drop}
  {\huge\sffamily\thetitle\par}
  \vspace{2em}
  {\normalsize\sffamily\textit\subtitletranslation\par}
  \vspace{2em}
  \rule{5.5cm}{0.3mm}\par
  \vspace{2em}
  {\Large\sffamily\textit\theauthor\par}
  \vspace{3em}
  {\footnotesize\sffamily\textit\translators\par}
  \vfill
  \includegraphics[width=3.5cm]{figures/knw.png}\par
  \end{center}
\endgroup}
\makeatother

% copyright page
\makeatletter
\newcommand*\copyrightpage{\begingroup
  \setlength\drop{0.1\textheight}
  \vphantom{just for the drop}
  \vfill
  \begin{scriptsize}
  \noindent \copyright\space \yearoriginal: \theauthor
  \par\noindent \textit{\titleoriginal: \subtitleoriginal}
  \vspace{0.5\baselineskip}
  \par\noindent \copyright\space \yeartranslation\space \stringtranslation: \translatorone
  \par\noindent \textit{\thetitle: \subtitletranslation}
  \vspace{\baselineskip}
  \par\noindent \textit{\stringlicense}
  \vspace{0.5\baselineskip}
  \par\noindent \stringpublisher: \href{https://konsensus.network}{\textit{konsensus.network}}
  \vspace{0.5\baselineskip}
  \par\noindent v1.0.0
  \vspace{0.5\baselineskip}
  \setlength{\parindent}{2em}% default 20pt
  \par\noindent ISBN \ISBNHC \:Hardcover
  \par\hspace{0.28\parindent}\ISBNSC \:Paperback
  \par\hspace{0.28\parindent}\ISBNEBOOK \:E-book\par
  \setlength{\parindent}{0pt}
  \end{scriptsize}
  \vspace{3em}
  \par\noindent \href{https://konsensus.network}{\large\MakeUppercase \press \hspace{3em} \includegraphics[width=1cm]{figures/freestarfish.png}}
  \setcounter{footnote}{0}
  \clearpage
\endgroup}
\makeatother

% HEADER AND FOOTER MANIPULATION
% for normal pages
\nouppercaseheads
\headsep = 3mm
\makepagestyle{mystyle} 
\makeevenhead{mystyle}{\scriptsize\sffamily\mdseries\thepage}{}{}
\makeoddhead{mystyle}{{\scriptsize\sffamily\mdseries\leftmark}}{}{\scriptsize\sffamily\mdseries\thepage}
\makeevenfoot{mystyle}{}{}{}
\makeoddfoot{mystyle}{}{}{}
\makeatletter

% for pages where chapters begin
\makepagestyle{plain}
\makerunningwidth{plain}{\headwidth}
\makeevenfoot{plain}{}{}{}
\makeoddfoot{plain}{}{}{}
\pagestyle{mystyle}

\newif\ifmainmatter
\appto\mainmatter{\mainmattertrue}
\appto\backmatter{\mainmatterfalse}
\appto\appendix{\mainmatterfalse}

\renewcommand\chaptermark[1]{%
  \markboth{\MakeUppercase{%
    \ifmainmatter~\oldstylenums\thechapter.~\fi#1}}{}}%

% TOC
\usepackage[]{tocloft}
\renewcommand{\cftsectiondotsep}{\cftnodots}
\renewcommand{\cftpartfont}{\Large\sffamily\MakeUppercase}
\renewcommand{\cftchapterfont}{\small\sffamily}
\renewcommand{\cftsectionfont}{\Small\sffamily}
\renewcommand{\cftpartpagefont}{\Large\sffamily}
\renewcommand{\cftchapterpagefont}{\small}
\renewcommand{\cftchapterpresnum}{KAPITEL\space}
\renewcommand{\cftchapternumwidth}{7em}
\setlength{\cftchapterindent}{0em}
\setlength{\cftsectionindent}{7em}
\setlength{\cftbeforechapterskip}{0.8em}
\setsecnumdepth{chapter}
\setcounter{tocdepth}{0}


% Redefine footnote presentation
\makeatletter
\renewcommand\@makefntext[1]{%
  \noindent\hb@xt@2em{% <-- Box of fixed size for footnote number and space
    \@thefnmark\quad}% <-- Footnote number followed by a quad space
  \parbox[t]{\dimexpr\linewidth-2em}{#1}% <-- Parbox to control the width of footnote content
}
\makeatother

% layout check and fix
\checkandfixthelayout

% COUNTERS FOOTNOTES
\usepackage{chngcntr}
\counterwithout*{footnote}{chapter}

% TITLE FORMATTING
\usepackage{titlesec}

% Define chapter format with titlesec
\titleformat
    {\chapter}[display]
    {\huge\sffamily} % Main title font style
    {\Large\sffamily\chaptertitlename~\thechapter} % "Chapter N" format
    {0pt} % Space between the chapter number and title
    {\Huge} % Chapter title formatting
    [\vspace{10pt}\Large\textit{\chaptersubtitle}] % Subtitle formatting

% Command to set the subtitle (empty by default)
\newcommand{\chaptersubtitle}{}

% Automatically render the subtitle (if set) after the chapter title
\titleformat{\chapter}[display]
  {\huge\sffamily}
  {\Large\sffamily\chaptertitlename\ \thechapter}
  {0pt}
  {\Huge}
  [\ifx\chaptersubtitle\empty\else\vspace{10pt}\Large\textit{\chaptersubtitle}\fi]

% Command to set subtitle manually after chapter rendering
\newcommand{\setsubtitle}[1]{%
  \renewcommand{\chaptersubtitle}{#1}%
  \chaptermark{\chaptersubtitle} % Update subtitle for header/footer
}

\titleformat
  {\section}[block]
  {\sffamily\large\bfseries}
  {}
  {0pt}
  {}
  
\titlespacing*{\section}{0pt}{2em}{0.5em}

\titleformat{\subsection}{\sffamily\bfseries}{}{}{}
\titlespacing*{\subsection}{0pt}{2em}{0em}

% QUOTE FORMATTING
\renewenvironment{quote}%
               {\list{}{\rightmargin=.6cm\leftmargin=.6cm}%
                \itshape \item[]}% <- The effect of \samepage is local!!!
               {\endlist}

% LAYOUT CHECK AND FIX
\checkandfixthelayout

% CUSTOM TITLE PAGE
\makeatletter
\def\@maketitle{%
  % the half title page
  \pagestyle{empty}
  \halftitlepage
  \cleardoublepage

  % the title page
  \titleM
  \clearpage

  % the copyright page
  \copyrightpage
  \cleardoublepage
  \pagestyle{mystyle}
}
\makeatother
% END PREAMBLE
\makeatletter
\@ifpackageloaded{bookmark}{}{\usepackage{bookmark}}
\makeatother
\makeatletter
\@ifpackageloaded{caption}{}{\usepackage{caption}}
\AtBeginDocument{%
\ifdefined\contentsname
  \renewcommand*\contentsname{Inhaltsverzeichnis}
\else
  \newcommand\contentsname{Inhaltsverzeichnis}
\fi
\ifdefined\listfigurename
  \renewcommand*\listfigurename{Abbildungsverzeichnis}
\else
  \newcommand\listfigurename{Abbildungsverzeichnis}
\fi
\ifdefined\listtablename
  \renewcommand*\listtablename{Tabellenverzeichnis}
\else
  \newcommand\listtablename{Tabellenverzeichnis}
\fi
\ifdefined\figurename
  \renewcommand*\figurename{Abbildung}
\else
  \newcommand\figurename{Abbildung}
\fi
\ifdefined\tablename
  \renewcommand*\tablename{Tabelle}
\else
  \newcommand\tablename{Tabelle}
\fi
}
\@ifpackageloaded{float}{}{\usepackage{float}}
\floatstyle{ruled}
\@ifundefined{c@chapter}{\newfloat{codelisting}{h}{lop}}{\newfloat{codelisting}{h}{lop}[chapter]}
\floatname{codelisting}{Listing}
\newcommand*\listoflistings{\listof{codelisting}{Listingverzeichnis}}
\makeatother
\makeatletter
\makeatother
\makeatletter
\@ifpackageloaded{caption}{}{\usepackage{caption}}
\@ifpackageloaded{subcaption}{}{\usepackage{subcaption}}
\makeatother

\ifLuaTeX
\usepackage[bidi=basic]{babel}
\else
\usepackage[bidi=default]{babel}
\fi
\babelprovide[main,import]{ngerman}
% get rid of language-specific shorthands (see #6817):
\let\LanguageShortHands\languageshorthands
\def\languageshorthands#1{}
\ifLuaTeX
  \usepackage{selnolig}  % disable illegal ligatures
\fi
\usepackage{bookmark}

\IfFileExists{xurl.sty}{\usepackage{xurl}}{} % add URL line breaks if available
\urlstyle{same} % disable monospaced font for URLs
\hypersetup{
  pdftitle={Das souveräne Individuum},
  pdfauthor={James Dale Davidson \& Lord William Rees-Mogg},
  pdflang={de},
  hidelinks,
  pdfcreator={LaTeX via pandoc}}


\title{Das souveräne Individuum}
\usepackage{etoolbox}
\makeatletter
\providecommand{\subtitle}[1]{% add subtitle to \maketitle
  \apptocmd{\@title}{\par {\large #1 \par}}{}{}
}
\makeatother
\subtitle{Der Übergang zum Informationszeitalter}
\author{James Dale Davidson \& Lord William Rees-Mogg}
\date{2024-09-10}

\begin{document}
\frontmatter
\maketitle

\renewcommand*\contentsname{Inhalt}
{
\setcounter{tocdepth}{0}
\tableofcontents
}

\mainmatter
\bookmarksetup{startatroot}

\chapter*{About this book}\label{about-this-book}
\addcontentsline{toc}{chapter}{About this book}

\markboth{About this book}{About this book}

\bookmarksetup{startatroot}

\chapter{DER ÜBERGANG IN DAS JAHR
2000}\label{der-uxfcbergang-in-das-jahr-2000}

\begin{quote}
„Es fühlt sich an, als stünde etwas Großes bevor: Diagramme
visualisieren das jährliche Bevölkerungswachstum, die Konzentration von
Kohlendioxid in der Atmosphäre, die Anzahl der Webadressen und die
Megabyte pro Dollar. Alle diese Faktoren zeigen eine steil ansteigende
Kurve, die kurz nach dem Jahrhundertwechsel in eine Asymptote übergeht:
die Singularität. Das Ende von allem, was wir kennen. Und der Anfang von
etwas, das wir möglicherweise nie vollständig begreifen werden.``
\footnote{Danny Hillis, \emph{The Millenium Clock}, Wired, Special
  Edition, Herbst 1995, S. 48.} - Danny Hillis
\end{quote}

\section{VORAHNUNGEN}\label{vorahnungen}

Die Jahrtausendwende hat die westliche Vorstellungskraft im letzten
Jahrtausend stark geprägt. Da die Welt zur Zeit des ersten Jahrtausends
nach Christus nicht untergegangen ist, blickten Theologen, Propheten,
Schriftsteller und Wahrsager mit der Erwartung auf das Ende des
Jahrzehnts, dass es etwas Bedeutendes einläuten wird. Sogar Isaac Newton
spekulierte, dass mit dem Jahr 2000 der Weltuntergang bevorstehen würde.
Michel de Nostredame, dessen Prophezeiungen seit ihrer
Erstveröffentlichung 1568 von jeder Generation gelesen werden, sagte für
Juli 1999 das Erscheinen des dritten Antichristen voraus.\footnote{Ericka
  Cheetham, \emph{The Final Prophecies of Nostradamus} (New York:
  Putnam, 1989), S. 424.} Der Schweizer Psychologe Carl Jung, Experte
für das „kollektive Unbewusste``, prophezeite für 1997 den Beginn eines
neuen Zeitalters. Es ist leicht, solche Voraussagen zu belächeln. Dies
gilt auch für die nüchternen Prognosen von Ökonomen wie Dr.~Edward
Yardeni von Deutsche Bank Securities, der voraussagte, dass
Computerstörungen zur Jahrtausendwende „die Weltwirtschaft zum
Stillstand bringen würden``.\footnote{Dr.~Edward Yardeni, \emph{Year
  2000 Recession: ``Prepare for the worst. Hope for the best``}, Version
  5.0, 13. Mai 1998, B1.2.} Ob man das Computerproblem des Jahres 2000
nun als unbegründete Hysterie ansieht, angezettelt von
Computerprogrammierern und IT-Beratern, um ihr Geschäft anzukurbeln,
oder als einen mysteriösen Fall von technischer Entfesselung in
Verbindung mit prophetischer Vorstellungskraft - es lässt sich nicht
leugnen, dass die Gegebenheiten am Vorabend des neuen Jahrtausends mehr
als nur gewöhnliche düstere Zweifel daran wecken, wohin sich die Welt
entwickelt.

Der Optimismus, der die westlichen Gesellschaften der letzten 250 Jahre
geprägt hat, wird schleichend von einer Unruhe bezüglich der Zukunft
verdrängt. Überall sind die Menschen unsicher und besorgt. Man kann es
in ihren Gesichtern sehen. Man kann es aus ihren Gesprächen heraushören.
Dies spiegelt sich sowohl in Umfragen, als auch in Wahlergebnissen
wider. Ganz so, wie unsichtbare, physikalische Veränderungen im
Ionengehalt der Atmosphäre ein aufkommendes Gewitter ankündigen, noch
bevor sich die Wolken verdunkeln und ein Blitz einschlägt, so hängen in
dieser Dämmerung des Jahrtausends Vorboten tiefgreifender Umwälzungen in
der Luft. Eine Person nach der anderen, jede auf ihre eigene Art, nimmt
das drohende Ende einer Lebensweise wahr. Mit Abschluss dieses
Jahrzehnts endet nicht nur ein mörderisches Jahrhundert, sondern auch
ein glorreiches Jahrtausend menschlicher Errungenschaften. Mit dem Jahr
2000 endet eine Ära.

\begin{quote}
„Denn es gibt nichts Verborgenes, das nicht ans Licht gebracht wird, und
nichts Geheimes, das nicht bekannt wird.`` - Matthäus 10:26
\end{quote}

Wir sind der Überzeugung, dass die moderne Phase der westlichen
Zivilisation ihrem Ende entgegengeht. In diesem Buch erklären wir,
warum. Wie viele frühere Werke, stellt es einen Versuch dar, in die
Dunkelheit zu blicken und die unklaren Umrisse und Dimensionen einer
noch kommenden Zukunft zu zeichnen. In diesem Sinne verstehen wir unsere
Arbeit als apokalyptisch, im ursprünglichen Sinne des Wortes.
Apokalypsis bedeutet auf Griechisch „Enthüllung``. Wir sind der Ansicht,
dass eine neue Geschichtsepoche - das Informationszeitalter - kurz vor
seiner „Enthüllung`` steht.

\begin{quote}
„Wir beobachten die Entstehung eines neuen logischen Raums, einer
allgegenwärtigen elektronischen Umgebung, zu der wir alle Zugang haben,
die wir betreten und erleben können. Kurz gesagt, wir erleben die Geburt
einer neuen Form von Gemeinschaft. Die virtuelle Gemeinschaft wird zum
Vorbild für ein säkulares Paradies; so wie Jesus sagte, es gäbe viele
Wohnstätten im Hause seines Vaters, so existieren auch viele virtuelle
Gemeinschaften, die jeweils ihre eigenen Bedürfnisse und Wünsche
reflektieren.`` - Michael Grasso\footnote{Michael Grasso, \emph{The
  Millenium Myth: Love and Death at the End of Time}, Wheaton, Illinois:
  Quest Books, 1995.}
\end{quote}

\section{DIE VIERTE STUFE DER MENSCHLICHEN
GESELLSCHAFT}\label{die-vierte-stufe-der-menschlichen-gesellschaft}

Dieses Buch widmet sich einer neuen Machtrevolution, die den Einzelnen
befreit, indem sie die Zwangsjacke des Nationalstaats des 20.
Jahrhunderts abschüttelt. Innovative Entwicklungen, die uns bisher
unbekannte Veränderungen in der Logik der Gewalt bringen, lassen uns die
Grenzen für die Zukunft neu ziehen. Sofern unsere Vermutungen zutreffen,
stehen wir am Vorabend der bedeutsamsten Revolution, die die Geschichte
je erlebt hat. Mit einer Geschwindigkeit, die nur wenige vorhersehen
können, wird die Mikroverarbeitung den Nationalstaat untergraben und
zerstören und dabei neue Formen der sozialen Organisation hervorbringen.
Diese Entwicklung wird keineswegs ohne Komplikationen verlaufen.

Die vor uns liegende Herausforderung wird durch die atemberaubende
Geschwindigkeit, mit der sie heranrollt, umso gewaltiger wirken,
besonders im Vergleich zu den Entwicklungen der Vergangenheit. Wenn man
die gesamte Menschheitsgeschichte betrachtet - von den frühesten
Anfängen bis hin zur Gegenwart - lassen sich lediglich drei grundlegende
Phasen des Wirtschaftslebens identifizieren: (1) die Gesellschaften der
Jäger und Sammler; (2) die Agrargesellschaften; und (3) die
Industriegesellschaften. Doch nun zeichnet sich am Horizont eine
vollkommen neue Phase der sozialen Organisation ab, die vierte Stufe:
die Informationsgesellschaften.

Jede vorangegangene Phase der Gesellschaftsentwicklung war einzigartig
in Bezug auf die Evolution und Kontrolle von Gewalt. Wie wir noch im
Detail aufzeigen werden, versprechen Informationsgesellschaften eine
bedeutsame Reduzierung des Einsatzes von Gewalt, teilweise weil sie über
lokale Grenzen hinausreichen. Die virtuelle Realität des Cyberspace, von
Romanautor William Gibson als eine „einvernehmliche Halluzination``
beschrieben, wird sich so weit jenseits der Kontrolle von Tyrannen
erstrecken, wie die Vorstellungskraft das erlaubt. Im neuen Jahrtausend
wird die Bedeutung der Kontrolle über weitreichende Gewalt bei weitem
geringer sein als zu irgendeinem Zeitpunkt seit der Französischen
Revolution. Das wird weitreichende Konsequenzen haben. Eine davon wird
der Anstieg der Kriminalität sein. Während der Ertrag aus organisierter
Gewalt in großem Stil schrumpft, ist es wahrscheinlich, dass die Profite
aus Gewalt in kleinem Stil stark ansteigen werden. Gewalt wird
zufälliger und örtlich begrenzt sein. Das organisierte Verbrechen wird
zunehmen. Wir werden erklären, weshalb das so ist.

Eine weitere logische Konsequenz des nachlassenden Hangs zur Gewalt ist
das Verschwinden der Politik. Viele Indikatoren lassen vermuten, dass
das Beharren auf den staatsbürgerlichen Mythen des Nationalstaates des
20. Jahrhunderts rapide abnimmt. Der Tod des Kommunismus ist nur das
auffälligste Beispiel hierfür. Der moralische Verfall und die zunehmende
Korruption in den höchsten Ebenen westlicher Regierungen sind kein
Zufallsprodukt, wie wir in der Tiefe aufzeigen werden. Dies ist ein
Beleg dafür, dass die Möglichkeiten des Nationalstaates ausgeschöpft
sind. Selbst viele seiner Anführer glauben nicht mehr an die Floskeln,
die sie verkünden. Und auch der Rest nimmt sie ihnen nicht mehr ab.

\subsection{Geschichte wiederholt
sich}\label{geschichte-wiederholt-sich}

Diese Situation erinnert stark an vergangene Ereignisse. Immer wenn
technologische Veränderungen die alten Strukturen von den neuen
treibenden Kräften der Wirtschaft entkoppelt haben, verschieben sich die
moralischen Maßstäbe. Die Menschen beginnen, diejenigen, die die alten
Institutionen beherrschen, mit wachsender Verachtung zu betrachten.
Diese verbreitete Ablehnung setzt häufig ein, lange bevor die Menschen
eine schlüssige Ideologie des Wandels formulieren. So war es auch im
späten fünfzehnten Jahrhundert, als die mittelalterliche Kirche die
dominierende Institution des Feudalismus war. Trotz des Volksvertrauens
in die „Heiligkeit des geistlichen Amtes`` wurden sowohl hohe als auch
niedere Geistliche extrem verachtet - eine Einstellung, die
bemerkenswert der heutigen Haltung der Bevölkerung gegenüber Politikern
und Bürokraten ähnelt.\footnote{Johan Huizinga, \emph{The Waning of the
  Middle Ages}, trans. E Hopman (London: Penguin Books, 1990), S. 172.}

Wir glauben, dass wir viel von dem Jahrhundert, in dem das Leben voll
und ganz von organisierter Religion geprägt war, und von der heutigen
Zeit, in der die Welt von der Politik dominiert wird, lernen können. Die
Kosten für die Aufrechterhaltung der institutionalisierten Religion am
Ende des fünfzehnten Jahrhunderts hatten einen historischen Höchststand
erreicht - ähnlich wie heute die Kosten für die Unterstützung der
Regierung ein rekordverdächtiges Ausmaß angenommen haben.

Wir wissen, was mit der organisierten Religion aufgrund der
Nachwirkungen der Schießpulverrevolution passiert ist. Technologische
Entwicklungen haben damals starke Anreize geschaffen, religiöse
Institutionen zu verkleinern und ihre Kosten zu reduzieren. Eine
vergleichbare technologische Revolution wird zu Beginn des neuen
Jahrtausends auch eine radikale Verkleinerung der Nationalstaaten zur
Folge haben.

\begin{quote}
„Heute, nach über einem Jahrhundert elektronischer Technologie, haben
wir unser zentrales Nervensystem praktisch weltweit erweitert und dabei
sowohl räumliche als auch zeitliche Barrieren, zumindest in Bezug auf
unseren Planeten, überwunden.`` \footnote{Marshall McLuhan,
  \emph{Understanding Media}, New York: Signet, 1964, S. 19.}
\end{quote}

\subsection{Die informationelle
Revolution}\label{die-informationelle-revolution}

In dem Maße, wie die großen Systeme immer schneller zusammenbrechen,
lässt der systematische Zwang, der Wirtschaft und Einkommensverteilung
steuert, nach. Die Effizienz beim Organisieren sozialer Einrichtungen
wird schnell an Bedeutung gewinnen und somit wichtiger als
Machtstrukturen werden. Das bedeutet, dass Provinzen und selbst Städte,
die effektiv Eigentumsrechte durchsetzen und für Rechtssicherheit sorgen
können, ohne viele Ressourcen zu verbrauchen, im Informationszeitalter
eine tragfähige Souveränität erlangen werden, wie es in den letzten fünf
Jahrhunderten nicht vorkam. In der digitalen Welt, dem Cyberspace, wird
ein völlig neuer Wirtschaftssektor entstehen, der unabhängig von
physischer Gewalt agiert. Die deutlichsten Vorteile davon werden der
„kognitiven Elite`` zu Gute kommen, die zunehmend über nationale Grenzen
hinweg handelt. Diese Elite ist bereits in Städten wie Frankfurt,
London, New York, Buenos Aires, Los Angeles, Tokyo und Hongkong
gleichermaßen heimisch. Die Einkommensunterschiede innerhalb der
einzelnen Länder werden größer, während sie zwischen den Ländern
abnehmen.

Das selbstbestimmte Individuum untersucht die sozialen und finanziellen
Auswirkungen dieses revolutionären Umbruchs. Es liegt uns am Herzen, Sie
dabei zu unterstützen, die Potenziale dieser neuen Epoche optimal zu
nutzen und dabei nicht von ihren Folgen überrollt zu werden. Sollte auch
nur die Hälfte unserer Prognosen eintreffen, steht uns eine Veränderung
bevor, deren Ausmaß in der Geschichte beispiellos ist.

Der Jahreswechsel 2000 wird nicht nur die Weltwirtschaft grundlegend
verändern, sondern dies auch schneller bewerkstelligen als jeder andere
vorangegangene Paradigmenwechsel. Im Gegensatz zur landwirtschaftlichen
Revolution wird die informationelle Revolution nicht Jahrtausende
brauchen, um ihre volle Wirkung zu entfalten. Und anders als bei der
industriellen Revolution werden sich ihre Auswirkungen nicht über
Jahrhunderte hinweg ziehen. Die informationelle Revolution vollzieht
sich innerhalb einer Lebensspanne.

Darüber hinaus wird diese Veränderung nahezu überall gleichzeitig
geschehen. Technische und wirtschaftliche Innovationen werden nicht mehr
auf bestimmte Gebiete begrenzt sein. Der Wandel wird allgegenwärtig
sein. Und er wird einen so fundamentalen Bruch mit der Vergangenheit
darstellen, dass die magische Welt der Götter, wie sie sich frühe
Agrarvölker wie die alten Griechen vorstellten, beinahe zum Leben
erweckt wird. In einem viel größeren Ausmaß, als es sich die meisten
heute eingestehen möchten, könnte es schwierig oder sogar unmöglich
sein, viele der aktuellen Institutionen ins neue Jahrtausend zu retten.
Wenn sich die Informationsgesellschaften formen, werden sie sich von den
Industriegesellschaften ebenso stark unterscheiden, wie das alte
Griechenland von der Welt der Höhlenbewohner abwich.

\section{PROMETHEUS ENTFESSELT: DER AUFSTIEG DES SELBSTBESTIMMTEN
INDIVIDUUMS}\label{prometheus-entfesselt-der-aufstieg-des-selbstbestimmten-individuums}

\begin{quote}
„Mir ist keine ermutigendere Tatsache bekannt als die unbestreitbare
Fähigkeit des Menschen, sein Leben durch bewusste Anstrengung zu
bereichern`` - Henry David Thoreau.
\end{quote}

Der anstehende Wandel birgt sowohl Vor- als auch Nachteile. Der Vorteil
ist, dass die informationelle Revolution Individuen stärker befreien
wird als je zuvor. Erstmals werden all diejenigen, die in der Lage sind,
sich eigenständig weiterzubilden, fast vollkommen frei darin sein, ihre
eigene Arbeit zu gestalten und das Maximum an Nutzen aus ihrer
persönlichen Produktivität zu ziehen. Genialität wird sich entfalten und
sich sowohl von Regierungsunterdrückung als auch von den Fesseln
rassistischer und ethnischer Vorurteile lösen. In der
Informationsgesellschaft wird niemand, der tatsächlich fähig ist, von
den ungeschliffenen Meinungen anderer gebremst werden. Es wird
unerheblich sein, was der Großteil der Menschen weltweit über Ihre
Rasse, Ihr Aussehen, Ihr Alter, Ihre sexuellen Vorlieben oder Ihre
Frisur denkt. In der Cyberwirtschaft wird man Sie nicht einmal sehen.
Die Unattraktiven, die Übergewichtigen, die Alten und die Behinderten
werden unter denselben Voraussetzungen wie die Jungen und Schönen
konkurrieren -- nämlich in der vollkommen farbenblinden Anonymität der
neuen Grenzen des Cyberspace.

\subsection{Aus Ideen wird Reichtum}\label{aus-ideen-wird-reichtum}

Leistung, ganz gleich wo sie erbracht wird, wird künftig stärker belohnt
als je zuvor. In einer Umwelt, in der die wertvollste Ressource nicht
mehr materielles Kapital, sondern die eigenen Ideen sind, hat jeder, der
klug denkt, das Potenzial, wohlhabend zu sein. Das Informationszeitalter
wird das Zeitalter der steigenden Mobilität sein. Es wird den Milliarden
von Menschen in Teilen der Welt, die bisher nie voll am Wohlstand der
Industriegesellschaft partizipieren konnten, deutlich mehr
Chancengleichheit bieten. Ihre klügsten, erfolgreichsten und
ehrgeizigsten Vertreter werden sich zu wahrhaft eigenständigen
Individuen entwickeln.

Zunächst werden zwar nur einige wenige die vollständige finanzielle
Souveränität erlangen, aber das schmälert keineswegs die Vorzüge der
finanziellen Unabhängigkeit. Die Tatsache, dass nicht jeder das gleiche
Vermögen ansammelt, bedeutet nicht, dass der Versuch, reich zu werden,
vergeblich oder sinnlos ist. Auf jeden Milliardär kommen 25.000
Millionäre. Wenn Sie Millionär und kein Milliardär sind, sind Sie
deswegen nicht arm. Auch in Zukunft wird einer der Maßstäbe Ihres
finanziellen Erfolgs nicht nur darin bestehen, wie viele Nullen Sie zu
Ihrem Nettovermögen hinzufügen können, sondern darin, ob Sie Ihre
Geschäfte so strukturieren können, dass Sie vollständige individuelle
Autonomie und Unabhängigkeit erreichen. Je mehr Finesse Sie an den Tag
legen, desto weniger Anstrengung werden Sie benötigen, um die
finanzielle Fluchtgeschwindigkeit zu erreichen. Selbst Personen mit eher
bescheidenen Ressourcen können sich hocharbeiten, wenn der Einfluss der
Politik auf die Weltwirtschaft abnimmt. Eine nie zuvor dagewesene
finanzielle Unabhängigkeit wird in Ihrem Leben oder dem Ihrer Kinder ein
erreichbares Ziel sein.

Auf dem Gipfel der Produktivität gehen diese selbstbestimmten Individuen
miteinander in den Wettstreit und interagieren unter Bedingungen, die an
die Verbindung zwischen den Göttern in der griechischen Mythologie
erinnern. Der schwer greifbare Olymp des kommenden Jahrtausends wird im
Cyberspace liegen - einem Bereich ohne physische Existenz, der dennoch
das Potenzial hat, die größte Wirtschaft der Welt im zweiten Jahrzehnt
dieses Jahrtausends zu entwickeln. Bis 2025 wird die Cyber-Ökonomie
viele Millionen Teilnehmer verzeichnen. Einige davon werden sich mit
einem Vermögen von jeweils über 10 Milliarden Dollar ebenso bereichern
wie Bill Gates. Die Cyber-Armen werden diejenigen sein, die weniger als
200.000 Dollar pro Jahr verdienen. Es wird keine Cyber-Sozialhilfe,
keine Cyber-Steuern und keine Cyber-Regierung geben. Nicht China,
sondern die Cyber-Ökonomie könnte das dominierende Wirtschaftsphänomen
der nächsten dreißig Jahre darstellen.

Die gute Nachricht ist, dass Politiker ebenso wenig Kontrolle,
Unterdrückung und Regulierung des größten Teils des Handels in dieser
neuen Welt ausüben können, wie die Gesetzgeber der antiken griechischen
Stadtstaaten in der Lage waren, den Bart von Zeus zu stutzen. Das ist
eine positive Nachricht für die Reichen und noch bessere Neuigkeit für
die weniger Reichen. Die von der Politik geschaffenen Hindernisse und
Belastungen wirken eher hinderlich auf das Erlangen von Reichtum als auf
das Erhalten desselben. Die Zurückhaltung bei der Anwendung von Gewalt
und die Dezentralisierung der Zuständigkeiten schaffen neue
Möglichkeiten für jeden tatkräftigen und ambitionierten Menschen, vom
Abflauen der politischen Macht zu profitieren. Selbst Konsumenten
staatlicher Dienstleistungen können davon profitieren, wenn Unternehmer
die Vorteile von Wettbewerb weiter fördern. Bislang bedeutete der
Wettbewerb zwischen Gerichtsbarkeiten in der Regel einen Wettbewerb der
Gewalt zur Durchsetzung der Herrschaft einer vorherrschenden Gruppe.
Folglich wurde viel Erfindergeist von Wettbewerb zwischen
Gerichtsbarkeiten in militärische Bestrebungen kanalisiert. Jedoch wird
die Cyberökonomie den Wettbewerb in Bezug auf staatliche
Dienstleistungen unter neuen Bedingungen fördern. Eine Zunahme von
Gerichtsbarkeiten bedeutet mehr Möglichkeiten für das Ausprobieren neuer
Methoden zur Durchsetzung von Verträgen und um die Sicherheit von
Personen und Eigentum auf neue Art und Weise zu garantieren. Die
Befreiung eines großen Teils der Weltwirtschaft von politischer
Kontrolle wird alle verbliebenen Regierungsformen dazu zwingen, unter
Bedingungen zu arbeiten, die stark an die Marktwirtschaft angelehnt
sind. Letztendlich werden sie kaum eine andere Wahl haben, als die
Bevölkerung in den von ihnen betreuten Gebieten eher wie Kunden zu
behandeln und weniger so, wie organisierte Kriminelle die Opfer ihrer
Erpressung behandeln.

\subsection{Jenseits der Politik}\label{jenseits-der-politik}

Was in der Mythologie als die Domäne der Götter galt, wird für den
Einzelnen zur erreichbaren Option - ein Leben jenseits der Macht von
Königen und Ratsherren. Zuerst zu Hunderten, dann zu Tausenden und
schließlich zu Millionen werden Menschen die Ketten der Politik
abstreifen. Dabei werden sie die Natur der Regierungen verändern, den
Raum des Zwanges reduzieren und den Bereich der privaten Kontrolle über
Ressourcen erweitern.

Das erneute Auftauchen des selbstbestimmten Individuums wird einmal mehr
die geheimnisvolle, prophetische Macht des Mythos unterstreichen. Die
frühen Agrargesellschaften hatten nur wenig Kenntnis von den
Naturgesetzen und nahmen an, dass „Kräfte, die wir heute als
übernatürlich bezeichnen würden``, weit verbreitet seien. Diese Kräfte
wurden teils von Menschen, teils von „leibhaftigen menschlichen
Göttern`` genutzt, die menschenähnlich aussahen und auf eine Weise mit
ihnen interagierten, die Sir James George Frazer in „The Golden Bough``
als „große Demokratie`` beschrieb.\footnote{James George Frazer,
  \emph{The Golden Bough: A Study in Magic and Religion} (New York:
  Macmillan, 1951), S. 105.}

Als sich die Menschen der Antike ausmalten, dass die Nachkommen des Zeus
mitten unter ihnen weilten, war ihr Glaube an Magie stark. Zusammen mit
anderen primitiven Agrargesellschaften teilten sie eine tiefe Ehrfurcht
vor der Natur sowie die abergläubische Annahme, dass natürliche
Phänomene durch individuelle Willenskraft, also durch Magie, beeinflusst
werden konnten. In diesem Kontext hatte ihr Verständnis von der Natur
und ihren Göttern nichts an sich, was selbstbewusst und prophetisch
genannt werden könnte. Es lag weit außerhalb ihrer Vorstellungskraft,
die zukünftige Mikrotechnologie zu erahnen. Sie konnten sich nicht
ausmalen, wie diese Tausende von Jahren später die individuelle
Produktivität verändern würde. Sie hätten sicher nicht vorhersehen
können, wie sie das Gleichgewicht von Macht und Effizienz verschieben
und damit die Art und Weise revolutionieren würde, wie Reichtum
geschaffen und erhalten wird. Doch das, was sie sich ausdachten, als sie
ihre Mythen webten, hallt auf merkwürdige Weise in der Welt nach, die
Sie aller Wahrscheinlichkeit nach erleben werden.

\subsection{Alt.Abrakadabra}\label{alt.abrakadabra}

Das „Abrakadabra`` magischer Beschwörungen zum Beispiel erinnert
merkwürdigerweise an ein Passwort, das für den Zugriff auf einen
Computer benötigt wird. In gewisser Hinsicht hat die
Hochgeschwindigkeitsberechnung bereits ermöglicht, die Magie des
Flaschengeistes nachzuempfinden. Frühe Generationen dieser „digitalen
Diener`` gehorchen bereits den Anweisungen derjenigen, die die Computer
steuern, in denen sie eingeschlossen sind - genauso wie Flaschengeister
in versiegelten Wunderlampen. Die virtuelle Realität der
Informationstechnologie wird das Spektrum menschlicher Wünsche erweitern
und nahezu jede erdenkliche Vorstellung zu einer scheinbaren Realität
machen. Telepräsenz wird Lebewesen die Fähigkeit verleihen, Entfernungen
mit übernatürlicher Geschwindigkeit zu überbrücken und Ereignisse aus
der Ferne zu verfolgen - ähnlich wie es den Göttern Hermes und Apollo in
der griechischen Mythologie zugeschrieben wurde. Die selbstbestimmten
Individuen des Informationszeitalters werden, ähnlich wie die Götter der
antiken und primitiven Mythen, mit der Zeit eine Art „diplomatische
Immunität`` gegenüber den meisten politischen Problemen genießen, die
sterbliche Menschen in den meisten Zeiten und an den meisten Orten
heimsuchen.

Das neue selbstbestimmte Individuum wird ähnlich agieren wie die
Gottheiten aus den Mythen. Im gleichen physischen Umfeld wie der
normale, unterworfene Bürger, allerdings in einem eigenen Bereich der
Politik. Mit seinen bedeutend größeren Ressourcen und der Unabhängigkeit
von vielen Formen von Zwang hat das selbstbestimmte Individuum die
Macht, zum neuen Jahrtausend Regierungen umzubauen und Volkswirtschaften
neu einzurichten. Es ist fast unvorstellbar, welche weitreichenden
Auswirkungen dieser Wandel haben wird.

\subsection{Genius und Nemesis}\label{genius-und-nemesis}

Für alle, die menschlichen Ehrgeiz und Erfolg schätzen, wird das
Informationszeitalter eine Belohnung bereitstellen. Das ist zweifellos
die beste Neuigkeit seit vielen Generationen. Aber es gibt auch einen
Wermutstropfen: Mit dem Siegeszug der individuellen Autonomie und der
echten Chancengleichheit auf Leistungsbasis wird ein neues
Gesellschaftsmodell entstehen, welches enorme Belohnungen für Leistung
und größte individuelle Freiheit mit sich bringt. Damit wird jeder
Einzelne viel mehr Eigenverantwortung tragen müssen, als es zur Zeit der
Industrialisierung der Fall war. Zusätzlich wird dieser Wandel den
ungerechtfertigten Vorteil des Lebensstandards, den die Bewohner der
fortschrittlichen Industrienationen im gesamten 20. Jahrhundert genossen
haben, reduzieren. Während diese Zeilen entstehen, verfügen die obersten
15 Prozent der Weltbevölkerung über ein durchschnittliches jährliches
Pro-Kopf-Einkommen von 21.000 Dollar. Die übrigen 85 Prozent kommen
durchschnittlich auf nur 1.000 Dollar jährlich. Unter den neuen
Bedingungen des Informationszeitalters wird sich diese enorme, aus der
Vergangenheit angehäufte Vorteilslage zwangsläufig auflösen.

Dies wird dazu führen, dass die Fähigkeit der Nationalstaaten, Einkommen
in großem Umfang umzuverteilen, zusammenbricht. Die
Informationstechnologie nährt einen dramatisch verstärkten Wettbewerb
zwischen den Rechtsordnungen. Wenn Technologie zunehmend mobil wird und
Transaktionen vermehrt im Cyberspace stattfinden, werden Regierungen
kaum mehr in der Lage sein, für ihre Dienstleistungen mehr zu verlangen,
als sie den Zahlenden wert sind. Jeder, der über einen Laptop und eine
Satellitenverbindung verfügt, wird in der Lage sein, fast jeden
Informationshandel an jedem beliebigen Ort abzuwickeln. Dazu zählen auch
fast alle Finanztransaktionen im Wert von mehreren Billionen Dollar.

Das heißt, dass man künftig nicht mehr dazu gezwungen ist, in einem Land
mit hoher Steuerlast zu leben, um ein hohes Einkommen zu erzielen. In
einer Zukunft, in der der größte Teil des Wohlstands überall verdient
und auch ausgegeben werden kann, werden Regierungen, die versuchen,
überhöhte Preise für den Wohnsitz einzufordern, ihre besten Steuerzahler
verlieren. Sollten unsere Überlegungen zutreffen, und davon sind wir
überzeugt, wird der Nationalstaat, wie wir ihn heute kennen, in seiner
jetzigen Form nicht bestehen bleiben.

\section{DAS ENDE DER
NATIONALSTAATEN}\label{das-ende-der-nationalstaaten}

Veränderungen, die die Dominanz etablierter Institutionen untergraben,
können sowohl beängstigend als auch bedrohlich sein. Genauso wie
Monarchen, Fürsten, Päpste und Machthaber zu Beginn der Neuzeit
erbarmungslos um den Erhalt ihrer gewohnten Privilegien kämpften, so
setzen auch heute Regierungen oft verdeckt und willkürlich Gewalt ein,
in dem Versuch, den Lauf der Zeit aufzuhalten. Durch die technologischen
Herausforderungen geschwächt, behandelt der Staat die autonomen
Individuen, seine ehemaligen Bürger, mit der gleichen Skrupellosigkeit
und Diplomatie, die er bisher gegenüber anderen Regierungen gezeigt hat.
Der Beginn dieser neuen Phase der Geschichte wurde am 20. August 1998
eingeläutet, als die Vereinigten Staaten Tomahawk-Marschflugkörper im
Wert von etwa 200 Millionen Dollar gegen Ziele abfeuerten, die angeblich
mit dem saudischen Exil-Millionär Osama bin Laden verknüpft waren. Bin
Laden war die erste Person in der Geschichte, deren Satellitentelefon
Ziel von Marschflugkörpern wurde. Gleichzeitig zerstörte die USA eine
Arzneimittelfabrik in Khartum, Sudan, angeblich als Vergeltung gegen Bin
Laden. Das Auftreten von Bin Laden als größter Feind der Vereinigten
Staaten markiert einen signifikanten Wechsel in der Kriegsdynamik. Eine
einzelne Person, die allerdings über Hunderte von Millionen Dollar
verfügt, kann nun als glaubhafte Bedrohung für die größte Militärmacht
des Industriezeitalters angesehen werden. In Aussagen, die an die
Propaganda aus Zeiten des Kalten Kriegs gegen die Sowjetunion erinnern,
präsentierten der US-Präsident und seine nationalen Sicherheitsberater
Bin Laden, eine Privatperson, als transnationalen Terroristen und
Hauptgegner der Vereinigten Staaten.

Die gleiche militärische Logik, die Osama bin Laden zum Hauptgegner der
USA machte, wird sich auch innerhalb der Beziehungen zwischen
Regierungen und ihren Bürgern etablieren. Immer strengere
Durchsetzungsmethoden werden die logische Konsequenz aus dem Entstehen
einer neuen Art von Verhandlungen zwischen Regierungen und
Einzelpersonen sein. Technologie wird die Einzelpersonen
selbstbestimmter machen als jemals zuvor. Und genau so werden sie auch
behandelt werden. Manchmal gewaltsam als Feinde, manchmal als
gleichberechtigte Verhandlungspartner, manchmal als Verbündete. Sie
können noch so rücksichtslos vorgehen, insbesondere während der
Übergangszeit wird es ihnen jedoch wenig Nutzen bringen, das Finanzamt
mit der CIA zu fusionieren. Sie werden zunehmend gezwungen sein, mit
autonom handelnden Individuen zu verhandeln, deren Ressourcen nicht mehr
so einfach kontrollierbar sind.

Die Informationsrevolution zieht Veränderungen nach sich, die nicht nur
zu einer finanziellen Krise für Regierungen führen, sondern auch zum
Zerfall großer Strukturen. Im zwanzigsten Jahrhundert sahen wir bereits
den Untergang von vierzehn Imperien. Dieser Prozess ist Teil einer
Entwicklung, die letztlich zur Auflösung des Nationalstaates selbst
führen wird. Staaten werden sich der zunehmenden Autonomie des Einzelnen
anpassen müssen. So wird die Steuerkapazität voraussichtlich um 50 bis
70 Prozent sinken. Dies dürfte kleinere Rechtsgebiete erfolgreicher
machen. Die Aufgabe, wettbewerbsfähige Bedingungen zu schaffen, um
kompetente Menschen und ihr Vermögen anzulocken, wird in Enklaven
leichter zu bewältigen sein als auf kontinentaler Ebene.

Wir glauben, dass mit dem fortschreitenden Zerfall des modernen
Nationalstaates die Barbaren der Neuzeit immer mehr die Macht im
Hintergrund übernehmen werden. Gruppierungen wie die russische Mafia,
die die Überreste der ehemaligen Sowjetunion aufsammeln, andere
ethnische Verbrechersyndikate, Nomenklaturen\footnote{Nomenklaturen sind
  die verwurzelten Eliten, die die frühere Sowjetunion und andere
  staatlich gelenkte Volkswirtschaften beherrschten.}, Drogenbosse und
abtrünnige Geheimdienste werden ihre eigenen Gesetze schaffen. Das tun
sie bereits. Viel mehr als allgemein bekannt, haben diese modernen
Barbaren bereits die Strukturen des Nationalstaates unterwandert, ohne
sein Erscheinungsbild signifikant zu verändern. Sie sind Mikroparasiten,
die sich von einem sterbenden System nähren. Diese Gruppen sind ebenso
gewalttätig und skrupellos wie ein Staat im Krieg, wenden jedoch
staatliche Techniken auf kleinerer Ebene an. Ihr wachsender Einfluss und
ihre Macht sind Teil der Verkleinerung des politischen Rahmens. Die
Mikroverarbeitung reduziert die Größe, die diese Gruppen erreichen
müssen, um in der Anwendung und Kontrolle von Gewalt effektiv zu sein.
Mit dem Fortschreiten dieser technologischen Revolution wird räuberische
Gewalt immer mehr außerhalb der zentralen Kontrolle organisiert werden.
Auch die Bemühungen zur Eindämmung von Gewalt werden sich auf eine Weise
entwickeln, die mehr von der Effizienz als von der Größe der Macht
abhängt.

\subsection{Geschichte in umgekehrter
Reihenfolge}\label{geschichte-in-umgekehrter-reihenfolge}

Der Vorgang, wie der Nationalstaat in den letzten fünf Jahrhunderten
gewachsen ist, wird durch die neue Dynamik des Informationszeitalters
umgekehrt. Lokale Machtzentren treten erneut in den Vordergrund, während
sich der Staatsapparat in fragmentierte, sich überschneidende
Souveränitätsgebiete auflöst.\footnote{Für mehr Einzelheiten über
  fragmentierte Souveränitäten als Vorläufer und Alternative zum
  Nationalstaat, siehe Charles Tilly, Coercion, Capital and European
  States AD 990-1992 (Oxford: Blackwell, 1993).} Die zunehmende Macht
der organisierten Kriminalität ist nur ein Beispiel für diese
Entwicklung. Multinationale Unternehmen sehen sich bereits genötigt,
alle Aufgaben bis auf die essentiellen an Subunternehmen auszulagern.
Einige Konglomerate wie AT\&T, Unisys und ITT haben sich in mehrere
Firmen aufgespalten, um rentabler wirtschaften zu können. Der
Nationalstaat wird sich, ähnlich wie ein schwerfälliges Konglomerat,
auflösen - vermutlich jedoch erst, wenn er durch Finanzkrisen dazu
gezwungen ist.

Nicht nur die Machtverhältnisse weltweit wandeln sich, sondern auch die
Arbeitswelt unterliegt einem starken Wandel. Dies bedeutet wiederum,
dass sich unweigerlich die Art und Weise ändern wird, wie Unternehmen
arbeiten. Das Konzept des „virtuellen Unternehmens`` ist ein Zeichen für
diese einschneidenden Veränderungen, die durch sinkende Informations-
und Transaktionskosten begünstigt wurden. Wir analysieren, welche
Auswirkungen die Informationsrevolution auf die Auflösung von
Unternehmen und das Ende des „sicheren Arbeitsplatzes`` hat. Im
Informationszeitalter wird aus dem „Arbeitsplatz`` eine
Aufgabenstellung, die es zu lösen gilt, statt einer Position, die man
einfach „hat``. Die Mikroverarbeitung hat ganz neue Perspektiven für
wirtschaftliche Aktivitäten jenseits von territorialen Grenzen
erschlossen. Diese Überwindung von Grenzen und Territorien könnte
womöglich die revolutionärste Entwicklung sein, seit Adam und Eva
aufgrund des Urteils ihres Schöpfers das Paradies verlassen mussten: „Im
Schweiße deines Angesichts sollst du dein Brot essen.`` Während die
Technologie die Werkzeuge, die wir verwenden, revolutioniert,
hinterlässt sie auch unsere Gesetze als veraltet, modelliert unsere
Moral neu und verändert unsere Wahrnehmungen. In diesem Buch wird
erklärt, wie dies geschieht.

Die Mikroverarbeitung und die rapide Weiterentwicklung der
Kommunikationsmittel ermöglichen es den Menschen schon heute, ihren
Arbeitsort frei zu wählen. Transaktionen über das Internet oder das
World Wide Web können verschlüsselt und in naher Zukunft nahezu
unentdeckt von Steuerbehörden durchgeführt werden. Steuerfreies Geld
vermehrt sich bereits jetzt im Ausland deutlich schneller als
inländische Gelder, die weiterhin der hohen Steuerbelastung des
Nationalstaates aus dem 20. Jahrhundert ausgesetzt sind. Nach der
Jahrtausendwende wird sich ein Großteil des Welthandels in den
Cyberspace verlagern, eine Region, über die Regierungen nicht mehr
Kontrolle ausüben werden, als sie es über den Meeresboden oder die
äußeren Planeten tun. Im Cyberspace werden die Drohungen physischer
Gewalt, die seit jeher Grundpfeiler der Politik sind, der Vergangenheit
angehören. Im Cyberspace begegnen sich die Milden und die Mächtigen auf
einer Ebene. Der Cyberspace ist die ultimative Offshore-Rechtsordnung:
Eine steuerfreie Wirtschaft. Bermuda im Himmel, geschmückt mit
Diamanten.

Wenn dieses größte Steuerparadies überhaupt erst einmal auf die
Wirtschaft losgelassen wird, können alle Fonds im Endeffekt als
Offshore-Fonds auf Wunsch ihrer Besitzer betrachtet werden. Die
Auswirkungen werden enorm sein. Der Staat hat sich darauf eingestellt,
seine Steuerzahler zu behandeln wie ein Bauer seine Kühe: er hält sie
auf einer Weide, um sie zu melken. Doch bald werden diese Kühe Flügel
entwickeln.

\subsection{Die Rache des
Nationalstaats}\label{die-rache-des-nationalstaats}

Gleich einem aufgebrachten Bauern wird der Staat zweifellos anfangs zu
verzweifelten Mitteln greifen, um seine abwandernde Herde zu
kontrollieren und einzugrenzen. Unauffällige und sogar gewaltsame
Maßnahmen werden angewandt, um den Zugang zu befreienden Technologien zu
begrenzen. Diese Mittel werden jedoch, wenn überhaupt, nur vorübergehend
erfolgreich sein. Der Nationalstaat des zwanzigsten Jahrhunderts, mit
all seiner Anmaßung, wird verhungern, sobald seine Steuereinnahmen
zurückgehen.

Wenn der Staat es nicht schafft, seine Ausgaben durch höhere
Steuereinkünfte zu decken, greift er auf andere, verzweifeltere
Maßnahmen zurück. Eine solche Maßnahme besteht darin, Geld zu drucken.
Regierungen haben sich daran gewöhnt, ein Monopol auf die eigene Währung
zu besitzen, die sie nach Belieben abwerten können. Diese willkürliche
Inflation war ein markantes Kennzeichen der Geldpolitiken aller Länder
im 20. Jahrhundert. Selbst die Deutsche Mark, die stärkste nationale
Währung der Nachkriegszeit, verlor von 1949 bis Ende Juni 1995 71
Prozent ihres Wertes. Im gleichen Zeitraum verlor der US-Dollar 84
Prozent seines Wertes.\footnote{Der deutsche GPI-Index lag am 31.
  Dezember 1948 bei 33,20 und am 30. Juni 1995 bei 112,90, was einer
  durchschnittlichen jährlichen Abwertung von 2,7\% entspricht. Der VPI
  der USA lag am 31. Dezember 1948 bei 24 und am 30. Juni 1995 bei
  152,50. Die kumulierte Inflation in den USA betrug in diesem Zeitraum
  635\%.} Diese Inflation hatte den gleichen Effekt wie eine Steuer auf
jeden, der Geld besitzt. Wie wir später erörtern werden, wird die
Inflation als mögliche Einnahmequelle durch das Aufkommen von digitalem
Geld weitgehend ausgeschaltet. Neue Technologien ermöglichen es
Vermögensinhabern, nationale Monopole zu umgehen, die in der Neuzeit das
Geld herausgegeben und reguliert haben. Die Finanzkrisen, die 1997 und
1998 Asien, Russland und andere aufstrebende Länder traf, zeigen, dass
nationale Währungen und nationale Bonitätseinstufungen nicht zur
reibungslosen Funktion des globalen Wirtschaftssystems beitragen.
Insbesondere die Tatsache, dass die Souveränitätsbedingungen
vorschreiben, dass alle Transaktionen innerhalb eines Landes in der
nationalen Währung abgewickelt werden müssen, schafft Anfälligkeit für
Fehlentscheidungen der Zentralbanker und Angriffe von Spekulanten, die
in einem Land nach dem anderen zu deflationären Krisen geführt haben. Im
Informationszeitalter wird es den Menschen ermöglicht, Cyberwährungen zu
nutzen und so ihre wirtschaftliche Unabhängigkeit zu erklären. Wenn
jeder in der Lage ist, seine eigene Geldpolitik über das Internet zu
betreiben, wird die Kontrolle des Staates über die Druckmaschinen des
Industriezeitalters an Bedeutung verlieren. Ihre bisherige Bedeutung für
die Kontrolle des globalen Reichtums wird durch mathematische
Algorithmen, die keine physische Existenz haben, übertroffen. Im neuen
Jahrtausend wird digitales Geld, das von privaten Märkten kontrolliert
wird, das von den Regierungen ausgegebene Papiergeld ersetzen. Nur die
Armen werden Opfer der Inflation und des sich anschließenden
Zusammenbruchs in die Deflation, die eine Folge des künstlichen Hebels
ist, den das Fiat-Geld der Wirtschaft gewährt.

Da ihnen herkömmliche Methoden der Besteuerung und Inflationierung nicht
zur Verfügung stehen, werden Regierungen, selbst in traditionell
bürgerlichen Ländern, unangenehm auffallen. Wenn die Einkommensteuer
nicht mehr erhoben werden kann, erfahren ältere und repressivere
Methoden der Steuereintreibung eine Wiedergeburt. Die extreme Form der
Kapitalertragssteuer -- faktisch oder durch offene Geiselnahme - wird
von Regierungen ins Spiel gebracht, die verzweifelt versuchen, das
Fliehen des Reichtums aus ihren Griffen zu verhindern. Wer Pech hat,
wird herausgegriffen und auf nahezu mittelalterliche Art und Weise
gefügig gemacht. Unternehmen, die Dienstleistungen anbieten, die die
individuelle Autonomie fördern, werden infiltriert, sabotiert und
gestört. Die willkürliche Beschlagnahmung von Eigentum, die in den
Vereinigten Staaten bereits üblich ist und dort wöchentlich
fünftausendmal vorkommt, wird noch größere Verbreitung finden.
Regierungen werden Menschenrechte verletzen, die freie
Informationsverbreitung zensieren, nützliche Technologien sabotieren und
Schlimmeres. Aus den gleichen Gründen, aus denen die im Untergang
begriffene Sowjetunion erfolglos versucht hat, den Zugang zu Computern
und Kopiermaschinen einzuschränken, werden westliche Regierungen
versuchen, die Cyberökonomie mit totalitären Methoden zu unterdrücken.

\section{DIE RÜCKKEHR DER LUDDITEN}\label{die-ruxfcckkehr-der-ludditen}

Diese Methoden könnten sich bei bestimmten Bevölkerungsgruppen als
beliebt erweisen. Denn was für viele wie eine gute Nachricht von
individueller Befreiung und Autonomie klingt, könnte von jenen, die
nicht zur intellektuellen Elite gehören, als schlechte Nachricht
aufgefasst werden. Der größte Widerstand könnte von durchschnittlich
begabten Menschen in den aktuell wohlhabenden Ländern erwartet werden.
Sie sind es vor allem, die die Informationstechnologie als eine
Bedrohung ihres Lebensstils wahrnehmen könnten. Die Nutznießer des
organisierten Zwangs, einschließlich der Millionen Menschen, die von
staatlich umverteilten Einkommen leben, könnten das neu gewonnene
Freiheitsstreben selbstbestimmter Individuen als störend empfinden. Ihre
Missbilligung wird das alte Sprichwort verdeutlichen, das besagt: „Wo du
stehst, wird dadurch bestimmt, wo du sitzt``.

\begin{quote}
„Manchmal habe ich mich gefragt, wie ich so tiefe Trauer um das
Schicksal einiger weniger Männer empfinden konnte, die ich nie
persönlich kennenlernte und die Hunderte von Kilometern entfernt in
einem Stadion gegen eine andere Gruppe ebenfalls unbekannter Menschen
spielten. Die Antwort ist einfach. Ich liebte meine Mannschaft. Trotz
des Risikos war diese emotionale Investition ihren Preis wert. Der Sport
brachte mein Blut in Wallung, erregte mich, brachte mein Herz zum
Klopfen. Ich genoss den Nervenkitzel, wenn es wirklich um etwas ging.
Das Leben fühlte sich während eines Wettkampfs einfach intensiver an.``
- Craig Lambert
\end{quote}

Es wäre allerdings irreführend, sämtliche negativen Emotionen, die
während der anstehenden Übergangskrise aufkommen werden, lediglich dem
Wunsch zuzuschreiben, auf Kosten anderer zu leben. Das Ausmaß ist
deutlich größer. Bei genauerer Betrachtung der Beschaffenheit
menschlicher Gesellschaften lässt sich erahnen, dass die bevorstehende
Reaktion der Neu-Ludditen durchaus eine missverstandene moralische
Komponente aufweisen wird. Man könnte das als bloßen Wunsch beschreiben,
der mit einem moralischen Toupet versehen ist. In diesem Zusammenhang
beleuchten wir sowohl die moralischen als auch die moralisierenden
Aspekte der Übergangskrise. Egoistisches Streben ist weit weniger
motivierend als selbstgerechter Zorn. Obwohl die bürgerlichen Mythen des
20. Jahrhunderts immer mehr an Bedeutung einbüßen, haben sie nach wie
vor ihre Anhänger. Jeder, der im 20. Jahrhundert aufwuchs, wurde mit den
Aufgaben und Pflichten eines Bürgers dieser Zeit vertraut gemacht. Die
verbliebenen moralischen Imperative der Industriegesellschaft werden
zumindest einige neo-ludditische Angriffe auf Informationstechnologien
provozieren.

In diesem Kontext drückt die zu erwartende Gewalt zumindest teilweise
das aus, was wir als „moralischen Anachronismus`` bezeichnen. Dies
bedeutet, dass moralische Regeln, die in einer bestimmten
Wirtschaftsphase entstanden sind, auf Situationen angewendet werden, die
aus einer anderen Phase stammen. Jede Epoche der Gesellschaft benötigt
ihre eigenen moralischen Regeln, um den Individuen dabei zu helfen, die
typischen Anreizfallen, die die Entscheidungen prägen, die sie in ihrem
jeweiligen Lebensstil treffen müssen, zu meistern. Genau wie eine
bäuerliche Gesellschaft nicht nach den Moralvorstellungen einer
nomadischen Eskimogruppe leben könnte, kann die Informationsgesellschaft
nicht den moralischen Imperativen gerecht werden, die geschaffen wurden,
um den Erfolg eines militanten Industriestaates des 20. Jahrhunderts zu
fördern. Wir werden erklären, warum das so ist.

In den kommenden Jahren wird ein moralischer Anachronismus in den
westlichen Kernländern ähnliche Erscheinungsformen zeigen, wie wir sie
über die letzten fünf Jahrhunderte in den Randgebieten beobachten
konnten. Westliche Kolonisatoren und militärische Expeditionen lösten
solche Krisen aus, wenn sie auf einheimische Jäger- und Sammlergruppen
oder auf Gesellschaften stießen, deren Lebensweise noch
landwirtschaftlich geprägt war. Dabei sorgte die Einführung neuer
Technologien in einem anachronistischen Kontext für Verwirrung und
moralische Krisen. Der Erfolg christlicher Missionare bei der Bekehrung
von Millionen Einheimischen lässt sich zu einem erheblichen Teil auf
diese lokalen Krisen zurückführen, die durch das abrupte Aufkommen neuer
Machtstrukturen ausgelöst wurden. Solche Zusammenstöße ereigneten sich
vom 16. bis zu den ersten Jahrzehnten des 20. Jahrhunderts immer wieder.
Ähnliche Konflikte erwarten wir zu Beginn des neuen Jahrtausends, wenn
informationstechnologiegesteuerte Gesellschaften die industriell
orientierten ablösen.

\subsection{Die Sehnsucht nach Zwang}\label{die-sehnsucht-nach-zwang}

Die Idee des Erstarkens des selbstbestimmten Individuums wird sicher
nicht von allen als aufregende neue Ära in der Geschichte gesehen
werden, auch nicht von denen, die am meisten davon profitieren. Jeder
wird seine Bedenken haben. Und viele werden jede Neuerung, die auf
Kosten der territorialen Nationalstaaten geht, ablehnen. Es ist ein
menschlicher Instinkt, dass radikale Änderungen oft als dramatischer
Rückschritt wahrgenommen werden. Vor fünfhundert Jahren hätten die
Höflinge, die sich um den Herzog von Burgund geschart hatten, die neuen
Entwicklungen, die den Feudalismus untergruben, sicherlich als Übel
betrachtet. Sie waren überzeugt, dass sich die Welt in einer
beschleunigten Abwärtsspirale befand, genau in dem Moment, als spätere
Historiker eine Explosion des menschlichen Potenzials in der Renaissance
erkannten. Auf ähnliche Weise mag das, was aus der Sicht des nächsten
Jahrtausends einmal als neue Renaissance gesehen werden könnte, für die
erschöpften Seelen des zwanzigsten Jahrhunderts angsteinflößend
erscheinen.

Es ist sehr wahrscheinlich, dass manche, die sich durch neue
Entwicklungen angegriffen fühlen, sowie viele, die sie benachteiligen,
mit Unbehagen reagieren werden. Ihre Sehnsucht nach Zwang wird sich
wahrscheinlich in Gewalttaten ausdrücken. Begegnungen mit diesen neuen
„Ludditen`` könnten den Übergang zu radikalen, neuen Formen der sozialen
Organisation zu einer schlechten Nachricht für alle machen. Gehen Sie
lieber in Deckung. Da die Geschwindigkeit des Wandels die moralische und
ökonomische Anpassungsfähigkeit vieler Generationen überfordert, ist mit
heftigem und entrüstetem Widerstand gegen die informationelle Revolution
zu rechnen, trotz ihres enormen Potenzials, die Zukunft zu entfesseln.

Sie müssen diese Unannehmlichkeiten begreifen und sich darauf
einstellen. Eine Übergangskrise steht uns bevor. Deflationsschocks,
vergleichbar mit der asiatischen Infektion, die 1997 und 1998 von
Ostasien aus auf Russland und andere Schwellenländer übergriff, werden
sporadisch auftreten, wenn sich die veralteten nationalen und
internationalen Institutionen, die aus dem Industriezeitalter übrig
geblieben sind, als untauglich für die Herausforderungen der neuen,
dezentralisierten, transnationalen Wirtschaft erweisen. Die neuen
Informations- und Kommunikationstechnologien sind eine größere Bedrohung
für den modernen Staat als jede politische Herausforderung seiner
Vorherrschaft seit den Zeiten der Seefahrt von Kolumbus. Das ist
bedeutend, da die Machthaber selten friedlich auf Entwicklungen reagiert
haben, die ihre Autorität infrage stellen. Das wird auch jetzt nicht der
Fall sein.

Der Konflikt zwischen Neuem und Altem wird die ersten Jahre des neuen
Jahrtausends bestimmen. Unsere Prognose sieht sowohl große Risiken als
auch große Chancen voraus, sowie eine Zeit, in der die Zivilgesellschaft
in einigen Bereichen stark schrumpft, in anderen hingegen eine nie da
gewesene Ausdehnung erfährt. Autonome Individuen und bankrotte,
verzweifelte Regierungen werden sich zunehmend über einen neuen Graben
hinweg gegenüberstehen. Wir rechnen mit einer radikalen Umstrukturierung
des Souveränitätsbegriffs und dem faktischen Ende der Politik, bis
dieser Übergang abgeschlossen ist. Aus unausweichlichen Gründen, die wir
in diesem Buch ausführlich besprechen werden, wird die
Informationstechnologie die Fähigkeit des Staates untergraben, mehr für
seine Dienstleistungen zu verlangen, als diese für diejenigen, die sie
finanzieren, tatsächlich wert sind.

\begin{quote}
„Die Regierungen werden sich mit der Frage auseinandersetzen müssen, was
Souveränität eigentlich bedeutet.`` - Robert Martin
\end{quote}

\subsection{Souveränität durch
Märkte}\label{souveruxe4nituxe4t-durch-muxe4rkte}

In einem Umfang, den man sich vor einem Jahrzehnt kaum hätte ausmalen
können, gewinnen Individuen durch die Mechanismen des Marktes zunehmend
Autonomie gegenüber territorial gebundenen Nationalstaaten. Alle
Nationalstaaten sehen sich dem Bankrott und der raschen Erosion ihrer
Autorität gegenüber. So mächtig sie auch sein mögen, die Macht, die
ihnen bleibt, ist die Macht zu zerstören, nicht zu befehlen. Ihre
Interkontinentalraketen und Flugzeugträger sind bereits Relikte, so
beeindruckend und nutzlos wie die letzten Schlachtrösser des
Feudalismus.

Die Informationstechnologie hat die Kapazität, die Märkte dramatisch zu
erweitern, indem sie die Art und Weise, wie Vermögenswerte generiert und
geschützt werden, revolutioniert. Dies ist zweifellos bahnbrechend und
dürfte für die Industriegesellschaft umwälzender sein, als es das
Aufkommen von Schießpulver für die feudale Agrarwelt war. Die
Verwandlung, die mit dem Jahr 2000 einhergeht, kündigt die
Kommerzialisierung von Souveränität und den Untergang der Politik an, so
wie das Schießpulver einst das Ende des auf Eiden basierenden
Feudalismus bedeutete. Die Staatsbürgerschaft könnte denselben Pfad wie
das Rittertum beschreiten.

Wir sind davon überzeugt, dass das Zeitalter persönlicher
wirtschaftlicher Souveränität beginnt. Genau wie Stahlwerke,
Telefongesellschaften, Bergwerke und Eisenbahnen, die einst staatlich
betrieben wurden, weltweit rasch privatisiert wurden, stehen Sie kurz
davor, die ultimative Form der Privatisierung zu erleben - die
umfassende Entnationalisierung des Individuums. Das selbstbestimmte
Individuum des neuen Jahrtausends wird kein Wirtschaftssubjekt des
Staates mehr darstellen, keine Position in der Bilanz des
Finanzministeriums. Nach dem Übergang ins neue Jahrtausend werden die
entnationalisierten Bürger gar keine „Bürger`` mehr sein, sondern
Kunden.

\subsection{Reichweite übertrifft
Grenzen}\label{reichweite-uxfcbertrifft-grenzen}

Durch die Kommerzialisierung der Souveränität könnten die Konditionen
von Staatsbürgerschaften innerhalb von Nationalstaaten veraltet sein,
ähnlich wie Rittereide nach dem Untergang des Feudalismus. Statt als
steuerzahlende Bürger mit einem mächtigen Staat zu interagieren, werden
die souveränen Individuen des 21. Jahrhunderts Kunden von Regierungen
sein, die aus einem „neuen logischen Raum`` heraus agieren. Sie könnten
die geringstmögliche Regierungsführung aushandeln und dafür einen
vertraglichen Preis zahlen. Die Regierungen des Informationszeitalters
könnten sich nach anderen Prinzipien organisieren als denen, an die wir
in den vergangenen Jahrhunderten gewöhnt waren. Einige Gerichtsbarkeiten
und Souveränitätsdienstleistungen könnten durch „unterstützendes
Matching`` entstehen, einem System, bei dem affinitätsbezogene Faktoren,
inklusive kommerzieller Affinitäten, die Grundlage für Loyalität
innerhalb virtueller Gerichtsbarkeiten schaffen. In seltenen Fällen
könnten diese neuen Souveränitäten Überreste mittelalterlicher
Organisationen wie dem 900 Jahre alten Souveränen Malteserorden sein.
Der Orden, eine Vereinigung reicher Katholiken mit derzeit 10.000
Mitgliedern, hat ein jährliches Einkommen von mehreren Milliarden Euro,
gibt eigene Pässe, Briefmarken und Geld aus und unterhält volle
diplomatische Beziehungen mit siebzig Ländern. Während wir dies
schreiben, verhandelt der Orden mit der Republik Malta über die Rückgabe
von Fort St.~Angelo. Besitz des Forts könnte den fehlenden territorialen
Aspekt bereitstellen, der es den Rittern ermöglichen würde, als
Souveränität anerkannt zu werden. Die Malteserritter könnten erneut zu
einem souveränen Kleinststaat avancieren, der durch ihre lange
Geschichte sofort legitimiert würde. Es war das Fort St.~Angelo, von dem
aus die Malteserritter die Türken in der großen Schlacht von 1565
schlugen. Tatsächlich beherrschten die Malteserritter von Fort
St.~Angelo aus viele Jahre lang Malta, bis sie 1798 von Napoleon
vertrieben wurden. Falls sie in den kommenden Jahren zurückkehren, wäre
dies ein klares Indiz dafür, dass das moderne Nationalstaatensystem,
eingeführt nach der Französischen Revolution, nur eine Zwischenstation
auf der längeren historischen Strecke gewesen ist, in der es üblich war,
dass zahlreiche Arten von Souveränität parallel existierten.

Ein weiteres, völlig unterschiedliches Modell für eine postmoderne
Souveränität, die auf unterstützendem Matching basiert, ist das
Iridium-Satellitentelefonnetz. Auf den ersten Blick scheint es
sonderbar, einen Mobilfunkdienst als eine Form von Souveränität zu
betrachten. Iridium ist jedoch bereits von internationalen Behörden als
virtuelle Souveränität anerkannt. Vielleicht wissen Sie, dass Iridium
ein globaler Mobilfunkdienst ist, der es seinen Nutzern erlaubt, Anrufe
unter einer einzigen Nummer zu empfangen, unabhängig davon, wo auf der
Welt sie sich gerade befinden, sei es in Featherston, Neuseeland, oder
im bolivianischen Chaco. Damit Anrufe an Iridium-Nutzer weltweit
weitergeleitet werden können, mussten die internationalen
Telekommunikationsbehörden angesichts der Struktur der globalen
Telekommunikation zustimmen, Iridium als virtuelles Land mit einer
eigenen Landesvorwahl zu anerkennen: 8816. Von einem virtuellen Land,
das Satellitentelefonnutzer zusammenfasst, ist es nur ein kleiner
Schritt zur Souveränität für stärker vernetzte, grenzüberschreitende
virtuelle Gemeinschaften im World Wide Web. Die Bandbreite, das bedeutet
die Übertragungskapazität eines Kommunikationsmediums, hat seit der
Erfindung der Transistoren schneller zugenommen als die Rechenleistung.
Wenn dieser Trend zu größerer Bandbreite anhält, was wir für
wahrscheinlich halten, dann wird es nur noch wenige Jahre nach der
Jahrtausendwende dauern, bis die Bandbreite groß genug ist, um das
„Metaverse`` technisch zu ermöglichen, eine alternative Cyberspace-Welt,
die der Science-Fiction-Autor Neal Stephenson konzipiert hat.
Stephensons „Metaverse`` ist eine dichte virtuelle Gemeinschaft mit
ihren eigenen Gesetzen. Wir sind davon überzeugt, dass es unvermeidlich
ist, dass die Teilnehmer der zunehmend florierenden Cyber-Ökonomie eine
Ablösung von den veralteten Gesetzen der Nationalstaaten anstreben und
erlangen werden. Die neuen Cyber-Gemeinschaften werden mindestens ebenso
wohlhabend und durchsetzungsstark sein wie der Souveräne Malteserorden.
Angesichts ihrer weitreichenden Kommunikations- und
Informationskriegsfähigkeiten werden sie sogar noch besser in der Lage
sein, ihre Interessen durchzusetzen. Wir untersuchen auch weitere
Modelle fragmentierter Souveränität, bei denen kleine Gruppen effektiv
die Souveränität schwacher Nationalstaaten pachten und eigene
Wirtschaftsoasen schaffen können, ähnlich den gegenwärtig existierenden
Freihäfen und Freihandelszonen.

Um die Beziehungen zwischen selbstbestimmten Individuen und dem, was von
der Regierung übrigbleibt, abzubilden, wird ein neues moralisches
Vokabular benötigt. Es ist davon auszugehen, dass viele Menschen, die
als „Bürger`` der Nationalstaaten des zwanzigsten Jahrhunderts
aufgewachsen sind, bei diesem Paradigmenwechsel vor den Kopf gestoßen
werden. Das Ende der Nationen und die „Entnationalisierung des
Individuums`` werden etablierte Begriffe wie „gleicher Schutz durch das
Gesetz`` ins Wanken bringen, die auf Machtverhältnissen basieren, die in
naher Zukunft überflüssig sein könnten. Da virtuelle Gemeinschaften mehr
und mehr Zusammenhalt finden, fordern sie, dass ihre Mitglieder nach
ihren eigenen Regeln zur Verantwortung gezogen werden, nicht nach denen
der Nationalstaaten, in denen sie zufälligerweise leben. So wie es in
der Antike und im Mittelalter der Fall war, werden innerhalb desselben
geografischen Gebietes wieder mehrere Rechtssysteme nebeneinander
existieren.

Genauso wie die Bestrebungen, die Macht gepanzerter Ritter zu erhalten,
angesichts der Feuerwaffen zum Scheitern verurteilt waren, so sind auch
moderne Vorstellungen von Nationalismus und Staatsbürgerschaft dazu
bestimmt, durch fortschrittliche Mikrotechnologie untergraben zu werden.
Tatsächlich werden sie letztendlich zur Farce, wie die heiligen
Prinzipien des Feudalismus im 15. Jahrhundert, die im 16. Jahrhundert
der Lächerlichkeit preisgegeben wurden. Die hochgehaltenen
Bürgerschaftskonzepte des 20. Jahrhunderts werden für kommende
Generationen nach der Jahrtausendwende zu komischen Anachronismen. Der
Don Quijote des 21. Jahrhunderts wird kein Ritter sein, der für die
Wiederauferstehung des Feudalismus kämpft, sondern ein Bürokrat in einem
braunen Anzug - ein Steuereintreiber, der nach einem Bürger sucht, den
er prüfen kann.

\section{WIEDERBELEBUNG DER
MARSCHGESETZE}\label{wiederbelebung-der-marschgesetze}

Selten betrachten wir Regierungen als Wettbewerber, zumindest nicht in
einem umfassenden Sinn, sodass unser modernes Verständnis von Umfang und
Potenzial der Souveränität eingeschränkt ist. In der Vergangenheit, als
es für Gruppierungen schwieriger war, ein stabiles Gewaltmonopol mit
Zwang durchzusetzen, war die Macht häufig zersplittert, Zuständigkeiten
überschnitten sich und viele verschiedene Einheiten verkörperten ein
oder mehrere Merkmale der Souveränität. Oft besaß der nominelle
Herrscher kaum Macht. Heutzutage befinden sich Regierungen, die
schwächer sind als Nationalstaaten, in einem ständigen Wettbewerb um die
Durchsetzung eines Gewaltmonopols in einem bestimmten Gebiet. Dieser
Wettbewerb hat zu Anpassungen in Bezug auf die Zurückhaltung von Gewalt
und der Gewinnung von Gefolgschaft geführt, die bald wieder aktuell sein
werden.

Wenn Fürsten und Könige nur geringe Macht hatten und die Ansprüche von
einer oder mehreren Gruppen an einer Grenze kollidierten, passierte es
oft, dass keine der beiden Gruppen die andere dominieren konnte. Im
Mittelalter gab es zahlreiche Grenz- oder sogenannte „Marsch``-Regionen,
an denen Machtbereiche aufeinandertrafen. Solche konfliktbehafteten
Grenzverläufe prägten die europäischen Grenzgebiete über Jahrzehnte,
teils sogar Jahrhunderte hinweg. Man denke nur an die Grenzen zwischen
keltischen und englischen Gebieten in Irland, zwischen Wales und
England, Schottland und England, Italien und Frankreich, Frankreich und
Spanien, Deutschland und den slawischen Randgebieten Mitteleuropas,
sowie zwischen den christlichen Königreichen Spaniens und dem
islamischen Königreich Granada. In diesen Marschregionen entstanden
unterschiedliche institutionelle und rechtliche Strukturen, so wie wir
sie vermutlich im nächsten Jahrtausend wiederfinden werden. Wegen des
konkurrierenden Anspruchs zweier Oberhäupter zahlten die Bewohner der
Marschregionen selten Steuern. Weiterhin hatten sie meist das Privileg
selbst zu entscheiden, welchen Gesetzen sie sich unterwerfen wollten -
eine Wahl, die durch rechtliche Modelle wie „Bekenntnis`` und
„Pfändung`` realisiert wurde, die in der heutigen Zeit weitgehend in
Vergessenheit geraten sind. Wir sind jedoch der Ansicht, dass solche
Konzepte in den Rechtssystemen zukünftiger Informationsgesellschaften
eine bedeutende Rolle einnehmen werden.

\subsection{Jenseits der
Nationalität}\label{jenseits-der-nationalituxe4t}

Bevor der Nationalstaat existierte, war es oft unmöglich, die Anzahl der
Souveränitäten weltweit genau festzustellen, da sie sich auf komplexe
Art und Weise überschnitten und unterschiedliche Formen der Macht
ausübten. Das wird wiederkehren. Die Grenzziehung zwischen den einzelnen
Territorien wurde innerhalb der Strukturen der Nationalstaaten meist
deutlich festgelegt. Doch im Zeitalter der Informationstechnologie
beginnen diese Grenzen wieder zu verschwimmen. Mit dem Anbrechen des
neuen Jahrtausends erleben wir eine erneute Fragmentierung der
Souveränität. Es entstehen neue Strukturen, die einige, aber nicht alle
Eigenschaften aufweisen, die wir traditionell mit Regierungen
assoziieren.

Einige dieser neu entstandenen Organisationen, wie beispielsweise die
Tempelritter und andere religiöse Militärorden des Mittelalters, konnten
beträchtlichen Reichtum und militärische Stärke aufweisen, ohne ein fest
umrissenes Territorium zu kontrollieren. Sie orientierten sich bei ihrer
Organisation nicht an nationalen Kriterien. Die Mitglieder und
Führungspersonen solcher religiösen Gemeinschaften, die im Mittelalter
in einigen Teilen Europas eine souveräne Autorität ausübten, bezogen
ihre Autorität keineswegs aus einer nationalen Identität. Sie gehörten
allen möglichen Ethnien an und betonten, dass ihre Loyalität Gott galt,
nicht den nationalen Gemeinsamkeiten ihrer Mitglieder.

\subsection{Handelsrepubliken im
Cyberspace}\label{handelsrepubliken-im-cyberspace}

Auch die Wiederkehr von Händlerbünden und wohlhabenden Einzelpersonen
mit gewisser souveräner Macht, wie der Hanse im Mittelalter, lässt sich
beobachten. Die Hanse, die auf französischen und flämischen Messen aktiv
war, erweiterte sich auf Kaufleute aus sechzig Städten.\footnote{Janet
  L. Abu-Lughod, \emph{Before European Hegemony: The World System A.D.
  1250-1350} (Oxford: Oxford University Press, 1991), S. 62.} Die
``Hanseatic League``, wie sie im Englischen redundanterweise genannt
wird (die wörtliche Übersetzung wäre „Liga-Liga``), war eine
Organisation germanischer Kaufmannsgilden, die ihren Mitgliedern Schutz
bot und Handelsabkommen aushandelte. Sie hatten in vielen nord- und
osteuropäischen Städten quasistaatliche Befugnisse. Solche Institutionen
werden im neuen Jahrtausend den sterbenden Nationalstaat ersetzen, indem
sie in einer unsicheren Welt Schutz bieten und bei der Durchsetzung von
Verträgen helfen.

Kurz gesagt, diejenigen, die noch immer an den bürgerlichen Mythen der
Industriegesellschaft des 20. Jahrhunderts festhalten, werden von der
Zukunft wohl enttäuscht werden. Hierzu zählen auch die Illusionen der
Sozialdemokratie, die einst die talentiertesten Köpfe inspirierte und
antrieb. Sie beruhen auf der Annahme, dass sich Gesellschaften genau so
entwickeln würden, wie es die Regierungen für sinnvoll halten -
vorzugsweise als Reaktion auf Meinungsumfragen und sorgfältig gezählte
Stimmen. Doch das war nie so zutreffend, wie man es sich vor fünfzig
Jahren vorgestellt hat. Heutzutage wirkt diese Sichtweise wie ein
Anachronismus, ein Relikt der Industrialisierung, ähnlich einem rostigen
Schornstein. Jene bürgerlichen Mythen reflektieren nicht nur eine
Denkweise, die gesellschaftliche Probleme als technisch lösbare
Herausforderungen sieht; sie zeugen auch von einer trügerischen
Sicherheit, dass Ressourcen und Individuen in der Zukunft genauso leicht
politisch manipulierbar sein werden wie im 20. Jahrhundert. Wir zweifeln
daran. Es werden die Kräfte des Marktes und nicht die politischen
Mehrheiten sein, die Gesellschaften dazu zwingen, sich so zu
transformieren, wie es von der breiten Öffentlichkeit weder verstanden
noch gutgeheißen wird. So wird die naive Vorstellung, Geschichte sei
einfach das, was die Menschen sich wünschen, als äußerst irreführend
entlarvt werden.

Es wird von entscheidender Bedeutung sein, dass Sie die Welt mit neuen
Augen betrachten. Das bedeutet, dass Sie Ihre Perspektive ändern und
Dinge, die Sie wahrscheinlich als selbstverständlich angesehen haben,
neu analysieren müssen. Nur so können Sie zu einem neuen Verständnis
gelangen. Sollten Sie in einer Ära, in der herkömmliche Denkmuster an
Realitätsbezug verlieren, nicht über diese hinausdenken können, werden
Sie aller Wahrscheinlichkeit nach Opfer einer wachsenden
Orientierungslosigkeit werden. Orientierungslosigkeit kann zu
Fehlentscheidungen führen, die Ihr Unternehmen, Ihre Investitionen und
Ihren Lebensstil ernsthaft gefährden können.

\begin{quote}
„Das Universum belohnt uns, wenn wir seine Geheimnisse erkennen, während
es uns bestraft, wenn wir sie nicht verstehen. Wenn wir verstehen, wie
das Universum funktioniert, laufen unsere Pläne glatt und wir fühlen uns
wohl. Wenn wir jedoch versuchen zu fliegen, indem wir von einer Klippe
springen und mit den Armen rudern, wird uns das Universum töten.``
\footnote{Jack Cohen and Ian Stewart, \emph{The Collapse of Chaos} (New
  York: Viking, 1994).} - Jack Cohen und Ian Stewart
\end{quote}

\subsection{Neue Sichtweisen}\label{neue-sichtweisen}

Um sich auf die bevorstehende Welt einzustellen, ist es wichtig zu
verstehen, warum sie sich anders entwickeln wird, als die meisten
Experten vorhersagen. Man muss die versteckten Triebkräfte des Wandels
genau analysieren. Wir haben versucht, dies durch einen
unkonventionellen Ansatz zu erreichen, den wir als Studie der
Megapolitik bezeichnen. In zwei unserer früheren Werke, „Blood in the
Streets`` und \emph{The Great Reckoning}, haben wir argumentiert, dass
die hauptsächlichen Veränderungen nicht in politischen Manifesten oder
in den Aussagen verstorbener Ökonomen zu finden sind, sondern in den
weniger offensichtlichen Faktoren, die die Grenzen der Macht
verschwimmen lassen. Minimale Veränderungen in Klima, Topographie,
Mikroben oder Technologie können die Logik der Gewalt verändern. Sie
beeinflussen die Art und Weise, wie Menschen ihr Leben organisieren und
sich schützen.

Beachten Sie bitte, dass unser Ansatz, die ständigen Veränderungen in
der Welt zu begreifen, grundlegend anders ist, als der der meisten
Prognosesteller. Wir beanspruchen nicht, Experten zu sein, in dem Sinne,
dass wir über bestimmte „Themen`` mehr wissen als diejenigen, die ihr
Lebenswerk in die Kultivierung spezialisierter Kenntnisse investiert
haben. Stattdessen betrachten wir die Dinge von außen. Wir sind mit den
Themen vertraut, über die wir Vorhersagen treffen. Dabei geht es uns vor
allem darum aufzuzeigen, wo die Grenzen des Notwendigen gezogen werden.
Wenn diese Grenzen verschoben werden, verändert sich zwangsläufig auch
die Gesellschaft, egal, wie sehr sich die Menschen auch das Gegenteil
wünschen.

Wir sind der Meinung, dass der Schlüssel zum Verständnis der
gesellschaftlichen Entwicklungen im Begreifen der Faktoren liegt, die
Kosten und Nutzen der Anwendung von Gewalt bestimmen. Jede menschliche
Gesellschaft - von Jägergruppen bis zu Imperien -- wird von
megapolitischen Faktoren geprägt, welche die geltenden „Naturrechte``
bestimmen. Das Leben ist stets und überall hochkomplex. Schaf und Löwe
halten ein filigranes Gleichgewicht und interagieren an der Schwelle zum
Chaos. Wären Löwen plötzlich schneller, könnten sie Beutetiere fangen,
die ihnen bisher entkommen sind. Wären Schafen auf einmal Flügel
gewachsen, würden die Löwen verhungern. Die Fähigkeit, Gewalt auszuüben
und sich gegen sie zur Wehr zu setzen, ist die entscheidende Variable,
die jenes Leben am Rande der Ordnung verändert.

Wir rücken die Gewalt aus triftigem Grund in den Fokus unserer
Megapolitik-Theorie. Die Beherrschung von Gewalt stellt die zentrale
Herausforderung dar, mit der jede Gesellschaft konfrontiert ist. Wie wir
bereits in \emph{The Great Reckoning} dargelegt haben:

\begin{quote}
„Der Grund dafür, dass Menschen häufig zur Gewalt greifen, ist, dass
sich dies oft auszahlt. In gewissem Sinne ist es das Einfachste, was ein
Mann tun kann, wenn er Geld begehrt - es einfach zu nehmen. Dies gilt
gleichermaßen für eine Armee von Männern, die ein Ölfeld in Besitz
nimmt, wie auch für einen einsamen Kriminellen, der sich eine Geldbörse
greift. Wie William Playfair treffend formulierte, hat die Macht „stets
den einfachsten Weg zum Reichtum gesucht, indem sie diejenigen angriff,
die ihn besitzen``.

Eine der großen Herausforderungen des Wohlstands liegt genau darin, dass
räuberische Gewalt unter bestimmten Bedingungen sehr einträglich sein
kann. Krieg verändert alles. Er ändert die Spielregeln. Er beeinflusst
die Verteilung von Vermögen und Einkommen. Er hat sogar Macht darüber,
wer lebt und wer stirbt. Gerade die Tatsache, dass sich Gewalt lohnt,
macht sie so schwer zu kontrollieren.`` \footnote{Siehe James Dale
  Davidson und Lord William Rees-Mogg, \emph{The Great Reckoning},
  Zweite Ausgabe (New York: Simon \& Schuster, 1993), S. 53.}
\end{quote}

Diese Denkweise ermöglichte es uns, eine Reihe von Entwicklungen
vorherzusehen, die besserwisserische Experten als unmöglich abtaten. So
versuchten wir beispielsweise mit „Blood in the Streets``, das Anfang
1987 veröffentlicht wurde, die Anfänge der großen Megapolitischen
Revolution festzuhalten, die jetzt in vollem Gange ist. Damals
argumentierten wir, dass technologischer Wandel das Machtgleichgewicht
weltweit destabilisiert. Einige unserer zentralen Thesen sind:

\begin{itemize}
\item
  Wir äußerten die Vermutung, dass die amerikanische Vormachtstellung
  auf dem absteigenden Ast sei, was zu wirtschaftlichen Ungleichheiten
  und Krisensituationen führen würde - einschließlich eines neuerlichen
  Börsencrashs nach dem Muster von 1929. Die Experten widerlegten fast
  einstimmig die Möglichkeit eines solchen Szenarios. Doch schon ein
  halbes Jahr später, im Oktober 1987, wurden die Finanzmärkte vom
  intensivsten Ausverkauf des Jahrhunderts erschüttert.
\item
  Wir empfahlen unseren Lesern, auf den Zusammenbruch des Kommunismus zu
  warten. Wieder einmal wurden wir von Experten belächelt. Doch dann
  geschah 1989 das, was „niemand vorhergesehen hatte``. Die Berliner
  Mauer fiel und Revolutionen fegten die kommunistischen Regime von den
  baltischen Staaten bis Bukarest hinfort.
\item
  Wir erläuterten, warum das multiethnische Imperium, das die
  bolschewistische Nomenklatur von den Zaren übernommen hat,
  „unvermeidlich zerfallen`` würde. Ende Dezember 1991 wurde das Banner
  mit Hammer und Sichel zum letzten Mal vom Kreml eingeholt, als die
  Sowjetunion aufhörte zu existieren.
\item
  Inmitten des massiven Rüstungsanstiegs unter Reagan behaupteten wir,
  die Welt stehe kurz vor einer umfangreichen Abrüstung. Dies wurde
  vielfach als unwahrscheinlich oder sogar als absurd abgetan. Doch die
  nachfolgenden sieben Jahre führten zur weitreichendsten Abrüstung seit
  dem Ende des Ersten Weltkriegs.
\item
  Es gab eine Zeit, in der Experten aus Nordamerika und Europa Japan als
  Beleg dafür anführten, dass Regierungen in der Lage seien, die Märkte
  erfolgreich zu manipulieren. Unsere Prognose lautete jedoch anders.
  Wir sagten voraus, dass der Boom auf dem japanischen Finanzmarkt in
  einem Crash enden würde. Kurz nach dem Fall der Berliner Mauer erlebte
  der japanische Aktienmarkt einen drastischen Absturz und verlor fast
  die Hälfte seines Wertes. Bis heute sind wir der Überzeugung, dass der
  finale Absturz den Wertverlust von 89 Prozent, den die Wall Street
  nach dem Crash von 1929 erlitt, erreichen oder sogar übertreffen
  könnte.
\item
  Zu einer Zeit, in der von der Mittelklassefamilie bis hin zu den
  weltweit größten Immobilieninvestoren fast jeder davon ausging, dass
  Immobilienpreise nur steigen und nicht fallen könnten, haben wir vor
  einem bevorstehenden Immobiliencrash gewarnt. Innerhalb von vier
  Jahren verloren Immobilieninvestoren weltweit über eine Billion
  Dollar, als die Immobilienwerte ins Rutschen kamen.
\item
  Lange bevor es für Experten offensichtlich wurde, haben wir in „Blood
  in the Streets`` darauf hingewiesen, dass das Arbeitseinkommen
  gesunken ist und auch weiterhin sinken wird. Jetzt, fast ein Jahrzehnt
  später, erwacht die Welt endlich aus ihrem Dornröschenschlaf und
  erkennt diese Wahrheit. Die Durchschnittsstundenlöhne in den
  Vereinigten Staaten sind mittlerweile niedriger als zur Zeit der
  zweiten Eisenhower-Administration. Der durchschnittliche
  Jahresstundenlohn betrug 1993, bereinigt um Inflation, 18.808 Dollar.
  Im Vergleich dazu lag der entsprechende Lohn 1957, als Eisenhower
  seine zweite Amtszeit antrat, bei 18.903 Dollar.
\end{itemize}

Obwohl die Hauptthemen von „Blood in the Streets`` im Nachhinein
erstaunlich zutreffend waren, wurden sie vor einigen Jahren von den
Bewahrern etablierten Denkens noch als völliger Unsinn betrachtet. Ein
Newsweek-Rezensent von 1987 reflektierte die verschlossene
intellektuelle Atmosphäre der spätindustriellen Gesellschaft, indem er
unsere Analyse als „einen undurchdachten Angriff auf die Vernunft``
abtat.

Man könnte annehmen, dass Newsweek und ähnliche Publikationen im Laufe
der Zeit erkannt haben, dass unsere Analysen aufschlussreiche Einblicke
in die Dynamiken der sich wandelnden Welt gegeben haben. Aber
keineswegs, das war nicht der Fall. Die erste Ausgabe von \emph{The
Great Reckoning} wurde mit derselben spöttischen Feindseligkeit
aufgenommen wie „Blood in the Streets``. Keine geringere Autorität als
das Wall Street Journal lehnte unsere Analyse kategorisch ab und
bezeichnete sie als das Geschwätz einer „bekloppten Tante``.

Trotz diesem leichten Schmunzeln zeigten sich die Themen von \emph{The
Great Reckoning} in Wahrheit weniger absurd, als die Bewahrer der
Orthodoxie es zunächst darstellen wollten.

Wir haben unsere Analyse zum Ende der Sowjetunion vertieft und uns damit
auseinandergesetzt, warum Russland und die anderen ehemaligen
Sowjetrepubliken einer Zukunft voller zunehmender Unruhen,
Hyperinflation, und fallendem Lebensstandard entgegenblicken.

\begin{itemize}
\item
  Wir haben dargelegt, warum die 1990er Jahre als ein Jahrzehnt des
  Stellenabbaus in Erinnerung bleiben werden, einschließlich des ersten
  globalen Stellenabbaus sowohl in der Regierung als auch in
  Unternehmen.
\item
  Wir prognostizierten zudem, dass es zu einer weitreichenden
  Neudefinition der Rahmenbedingungen für die Umverteilung von Einkommen
  kommen würde, einschließlich drastischer Kürzungen des
  Leistungsstandards. Von Kanada bis Schweden zeigten sich Anzeichen
  einer Finanzkrise, und amerikanische Politiker verkündeten das „Ende
  des Sozialstaats, so wie wir ihn kennen``.
\item
  Wir haben vorausgesehen und erläutert, warum sich die vermeintliche
  „neue Weltordnung`` letztendlich als „neue Weltunordnung``
  herausstellen würde. Lange bevor die Gräueltaten in Bosnien die Medien
  beherrschten, warnten wir schon davor, dass Jugoslawien in einen
  Bürgerkrieg versinken könnte.
\item
  Bevor Somalia ins Chaos abglitt, erläuterten wir, wieso der
  bevorstehende Zusammenbruch von Regierungen in Afrika zur Folge haben
  würde, dass einige Länder dort praktisch unter Zwangsverwaltung
  gestellt werden müssten.
\item
  Wir haben vorausgesagt und begründet, warum der militante Islam den
  Marxismus als führende Ideologie der Auseinandersetzung mit dem Westen
  ablösen würde. Jahre bevor der Bombenanschlag in Oklahoma stattfand
  und der Versuch unternommen wurde das World Trade Center in die Luft
  zu jagen, haben wir erklärt, warum die USA mit einer Zunahme des
  Terrorismus konfrontiert werden würden.
\item
  Bevor die Schlagzeilen über die Unruhen in Städten wie Los Angeles und
  Toronto die Runde machten, haben wir darüber gesprochen, warum die
  Entstehung krimineller Subkulturen innerhalb städtischer Minderheiten
  den Nährboden für weitreichende kriminelle Gewalt bildet.
\item
  Wir prognostizierten auch „die letzte Depression des zwanzigsten
  Jahrhunderts``, die 1989 in Asien ihren Anfang nahm und sich vom Rand
  hin zum Zentrum des globalen Systems hin ausbreitete. Wir waren der
  Meinung, dass der japanische Aktienmarkt den Weg der Wall Street nach
  1929 einschlagen würde, was zu einem Kreditkollaps und einer
  Depression führen würde. Wenngleich massive staatliche Eingriffe in
  Japan und anderenorts vorübergehend verhinderten, dass die Märkte den
  Rückgang der Kreditbedingungen vollständig widerspiegelten, hat dies
  die wirtschaftliche Krisensituation lediglich verschoben und
  verschärft, was den Druck für wettbewerbsbedingte Abwertungen sowie
  einen systembedingten Kreditkollaps der Art, der in den 30er Jahren zu
  weltweiten Wirtschaftseinbrüchen geführt hatte, erhöht.
\end{itemize}

In \emph{The Great Reckoning} wurden auch einige kontroverse Thesen
formuliert, die sich bisher noch nicht bewahrheitet haben oder nicht den
von uns vorhergesagten Entwicklungsstatus erreicht haben:

\begin{itemize}
\item
  Wir prophezeiten, dass der japanische Aktienmarkt dem Pfad der Wall
  Street nach 1929 folgen und dies zu einem Kreditkollaps sowie einer
  Depression führen würde. Obwohl die Arbeitslosenraten in Spanien,
  Finnland und einigen weiteren Ländern die der 1930er Jahre sogar
  überstiegen und zahlreiche Länder, einschließlich Japan, lokale
  Depressionen durchliefen, hat es bislang noch keinen systemischen
  Kreditkollaps gegeben, der die Wirtschaften weltweit zum Einsturz
  brachte wie in den 1930er Jahren.
\item
  Wir waren der Ansicht, dass der Zusammenbruch des Führungssystems in
  der ehemaligen Sowjetunion dazu führen könnte, dass Atomwaffen in die
  Hände von Kleinstaaten, Terroristen und kriminellen Banden fallen.
  Dies ist glücklicherweise nicht eingetreten, zumindest nicht in dem
  von uns befürchteten Ausmaß. Presseberichten zufolge hat der Iran zwar
  mehrere taktische Atomwaffen auf dem Schwarzmarkt erworben, und
  deutsche Behörden konnten mehrere Versuche, nukleares Material zu
  verkaufen, vereiteln. Doch es gab keine angekündigte Stationierung
  oder Verwendung von Atomwaffen aus den Beständen der ehemaligen
  Sowjetunion.
\item
  Wir haben erläutert, warum der „Krieg gegen die Drogen`` in Ländern,
  in denen Drogen weit verbreitet sind - insbesondere in den USA -
  tatsächlich dazu dient, Polizei- und Justizsysteme zu unterwandern.
  Angesichts von zig Milliarden Dollar an verdeckten Monopolgewinnen,
  die jährlich erzielt werden, haben Drogenhändler sowohl das nötige
  Kapital als auch die Motivation, selbst stabil scheinende Länder zu
  korrumpieren. Obwohl die Weltmedien hin und wieder über die
  Durchdringung des amerikanischen politischen Systems mit Drogengeldern
  auf höchstem Niveau berichtet haben, ist die gesamte Geschichte noch
  nicht vollständig ans Licht gekommen.
\end{itemize}

\subsection{Hinsehen wenn andere
wegsehen}\label{hinsehen-wenn-andere-wegsehen}

Trotz aller Aspekte, in denen unsere Prognosen falsch lagen oder im
Licht der aktuellen Erkenntnisse falsch zu sein scheinen, besteht unsere
Bilanz die Überprüfung. Vieles, was voraussichtlich eine Rolle in der
Wirtschaftsgeschichte der 1990er Jahre spielen wird, war bereits in
\emph{The Great Reckoning} vorhergesehen und erläutert worden. Unsere
Vorhersagen waren oft nicht bloß einfache Hochrechnungen oder
Fortführungen von Trends, sondern Prognosen über wesentliche
Abweichungen von dem, was seit dem Zweiten Weltkrieg als normal galt.
Wir hatten gewarnt, dass sich die 1990er Jahre drastisch von den
vorhergehenden fünf Jahrzehnten unterscheiden würden. Wenn wir die
Nachrichten der Jahre 1991 bis 1995 betrachten, stellt sich heraus, dass
viele Themen aus \emph{The Great Reckoning} nahezu täglich bestätigt
wurden.

Wir sehen diese Entwicklungen nicht als isolierte Schwierigkeiten oder
sporadische Probleme, sondern als Erschütterungen und Beben, die entlang
derselben Bruchlinie auftreten. Die bestehende Ordnung wird durch ein
megapolitisches Erdbeben erschüttert, das sowohl die Institutionen
revolutionieren wird, als auch die Art und Weise, wie aufgeklärte
Menschen die Welt wahrnehmen, drastisch verändern wird.

Obwohl Gewalt eine zentrale Rolle in der Funktionsweise unserer Welt
spielt, wird sie erstaunlicherweise oft übersehen. Die meisten
politischen Analysten und Wirtschaftswissenschaftler behandeln Gewalt
wie eine unwesentliche Störung, ähnlich einer Fliege, die um einen
Kuchen schwirrt, anstatt sie als den Koch zu betrachten, der den Kuchen
überhaupt erst gebacken hat.

\subsection{Ein weiterer megapolitischer
Pionier}\label{ein-weiterer-megapolitischer-pionier}

Tatsächlich wurde über die Rolle der Gewalt in der Geschichte so wenig
nachgedacht, dass die Bibliografie zur megapolitischen Analyse auf ein
einziges Blatt Papier passen würde. In unserem Buch \emph{The Great
Reckoning} griffen wir die Argumentation eines fast vollständig in
Vergessenheit geratenen Klassikers der Megapolitik-Analyse auf: „An
Enquiry into the Permanent Causes of the Decline and Fall of Powerful
and Wealthy Nations`` von William Playfair aus dem Jahr 1805, die wir
weiter ausarbeiten. Einer unserer Ausgangspunkte ist das Werk von
Frederic C. Lane. Lane war ein mittelalterlicher Historiker, der in den
1940er und 1950er Jahren mehrere eindringliche Schriften zur Rolle der
Gewalt in der Geschichte verfasste. Seine womöglich umfassendste Arbeit
war „Economic Consequences of Organized Violence``, die 1958 im Journal
of Economic History veröffentlicht wurde. Nur eine Handvoll
professioneller Wirtschaftswissenschaftler und Historiker haben sie
gelesen, und scheinbar haben die meisten ihre Tragweite nicht erkannt.
Ähnlich wie Playfair schrieb auch Lane für ein Publikum, das es zu
seiner Zeit noch nicht gab.

\subsection{Einsichten für das
Informationszeitalter}\label{einsichten-fuxfcr-das-informationszeitalter}

Lane veröffentlichte seine Abhandlung über Gewalt und die
wirtschaftliche Relevanz des Krieges weit vor dem Aufkommen des
Informationszeitalters. Ohne Zweifel schrieb er seine Thesen nicht im
Hinblick auf Mikroprozessortechnologie oder die anderen technologischen
Umbrüche, die wir derzeit erleben. Gleichwohl stellen seine Erkenntnisse
über Gewalt einen Referenzrahmen dar, der uns dabei hilft zu verstehen,
wie die Gesellschaft im Zuge der Informationsrevolution neu geformt
wird.

Das Fenster in die Zukunft, das Lane öffnete, war paradoxerweise eines,
durch das er einen Blick in die Vergangenheit warf. Als Historiker,
spezialisiert auf das Mittelalter und insbesondere auf die Handelsstadt
Venedig, deren Reichtum in einer brutalen Welt auf- und abebbt, wandte
er seine Gedanken dem Aufstieg und Niedergang von Venedig zu. Diese
Betrachtungen lenkten seinen Fokus auf Themen, die uns wertvolle
Einblicke in die Zukunft gewähren können. Er erkannte, dass die Art und
Weise, wie Gewalt organisiert und kontrolliert wird, entscheidend ist
für „den Umgang mit begrenzten Ressourcen``.\footnote{Frederic C. Lane,
  \emph{Economic Consequences of Organized Violence}, The Journal of
  Economic History Vol. 18, No.~4 (Dezember 1958), S. 402.}

Wir sind überzeugt, dass Lanes Analyse über den konkurrierenden Einsatz
von Gewalt uns viel Aufschluss darüber geben kann, wie das Leben im
Informationszeitalter sich aller Wahrscheinlichkeit nach verändern wird.
Aber rechnen Sie nicht damit, dass die meisten Menschen ein so
altertümliches und abstraktes Argument zur Kenntnis nehmen, geschweige
denn ihm folgen. Während die Aufmerksamkeit der Welt auf unehrliche
Diskussionen und exzentrische Persönlichkeiten gelenkt wird, bleiben die
Fehlentwicklungen der Megapolitik meist unbemerkt. Der durchschnittliche
Nordamerikaner hat wahrscheinlich hundertmal mehr Aufmerksamkeit auf O.
J. Simpson gerichtet als auf die neuen Mikrotechnologien, die seinen
Arbeitsplatz überflüssig machen und das politische System untergraben,
auf das er für seine Arbeitslosenunterstützung angewiesen ist.

\section{DIE EITELKEIT DER WÜNSCHE}\label{die-eitelkeit-der-wuxfcnsche}

Die Neigung, das Wesentliche zu übersehen, findet sich nicht nur bei den
Stubenhockern vor dem Fernseher. Herkömmliche Denkmodelle aller Art
klammern sich an die Illusion des Nationalstaates und dass es die
Ansichten der Menschen wären, die die Welt verändern. Angeblich
versierte Analysten verstricken sich in Erklärungen und Prognosen, die
bedeutende historische Entwicklungen so deuten, als wären sie auf Wunsch
herbeigeführt worden. Ein besonders eindrückliches Beispiel für diese
Art von Argumentation fand sich auf der Meinungsseite der New York
Times, als wir gerade dabei waren, den Artikel „Goodbye, Nation-State,
Hello...What?{}`` von Nicholas Colchester\footnote{Nicholas Colehester,
  \emph{Goodbye Nation-State, Hello . . . What?}, New York Times, 17.
  Juli 1994, Abschnitt 4, S. 17.} zu verfassen. Nicht nur, dass das
Thema - der Untergang des Nationalstaates - genau unser Thema ist; der
Autor steht auch exemplarisch dafür, wie weit unsere Gedankengänge von
der Norm abweichen. Colchester ist alles andere als ein Dummkopf. Er war
Redaktionsleiter bei der Economist Intelligence Unit. Wenn also jemand
einen realistischen Überblick über die Welt haben sollte, dann er.
Trotzdem weist sein Artikel mehrmals ausdrücklich darauf hin, dass „das
Aufkommen einer internationalen Regierung`` nun „logischerweise
unaufhaltsam ist``.

Warum? Weil der Nationalstaat ins Straucheln geraten ist und die
Kontrolle über die wirtschaftlichen Kräfte verloren hat.

Wir sind der Ansicht, dass diese Annahme schlichtweg absurd ist. Der
Gedanke, eine neue Regierungsform würde einfach deshalb entstehen, weil
eine andere gescheitert ist, ist ein Trugschluss. Hätte diese
Argumentation Bestand, dann hätten Länder wie Haiti und Zaire längst
eine bessere Regierung, lediglich gestützt auf die Tatsache, dass deren
bisherige Regierungsführung so offenkundig unzureichend war.

Die Perspektive Colchesters, die unter den wenigen, die sich in
Nordamerika und Europa mit solchen Themen beschäftigen, weitgehend
vertreten wird, vernachlässigt völlig die maßgeblichen, megapolitischen
Kräfte, welche bestimmen, welche politischen Systeme tatsächlich
tragfähig sind. Dies bildet den Kern dieses Buches. Betrachtet man die
Technologien, die das neue Jahrtausend prägen, so erscheint es sehr viel
wahrscheinlicher, dass wir nicht auf eine Weltregierung zusteuern,
sondern auf eine Mikroregierung oder sogar Zustände, die sich der
Anarchie annähern.

Für jede ernsthafte Untersuchung zum Einfluss von Gewalt auf die
Festsetzung von Handlungsregeln wurden dutzende Bücher über die
Feinheiten von Weizensubventionen und noch hunderte mehr über die teils
undurchsichtigen Facetten der Geldpolitik verfasst. Ein Großteil dieses
Mangels an Reflexion über die Schlüsselfragen, die tatsächlich den
Verlauf der Geschichte bestimmen, spiegelt vermutlich die relative
Stabilität der Machtverhältnisse der letzten Jahrhunderte wider. Ein
Vogel, der auf dem Rücken eines Nilpferdes eingeschlafen ist, denkt
nicht daran, seinen Aussichtspunkt zu verlieren, bis das Nilpferd sich
tatsächlich bewegt. Träume, Mythen und Phantasien nehmen in den
sogenannten Sozialwissenschaften einen viel größeren Stellenwert ein,
als wir gemeinhin annehmen.

Dies zeigt sich besonders in der umfangreichen Literatur zum Thema
wirtschaftliche Gerechtigkeit. Für jede Seite, die einer sorgfältigen
Analyse darüber gewidmet ist, wie Gewalt die Gesellschaft prägt und
somit die Grenzen festlegt, innerhalb derer die Wirtschaft agieren muss,
sind unzählige Worte über wirtschaftliche Gerechtigkeit und
Ungerechtigkeit verfasst und gesprochen worden. Allerdings setzen
Diskussionen über wirtschaftliche Gerechtigkeit im modernen Kontext
voraus, dass die Gesellschaft von einem Zwangsinstrument dominiert wird,
das so mächtig ist, dass es die guten Dinge des Lebens entziehen und
umverteilen kann. Solch eine Macht hat es nur für wenige Generationen in
der Neuzeit gegeben. Diese Macht ist jedoch aktuell im Schwinden.

\subsection{Big Brother und soziale
Sicherheit}\label{big-brother-und-soziale-sicherheit}

Die industrielle Technologie verlieh den Regierungen im 20. Jahrhundert
mehr Kontrollmöglichkeiten als jemals zuvor. Auf eine gewisse Weise
schien es unvermeidlich, dass die Regierungen die Kontrolle so effektiv
monopolisieren würden, dass kaum noch Raum für individuelle Autonomie
übrigbleiben würde. Um die Mitte des Jahrhunderts rechnete niemand mit
einem Triumph des selbstbestimmten Individuums.

Einige der schärfsten Denker Mitte des 20. Jahrhunderts waren aufgrund
der vorherrschenden Ansichten überzeugt, dass die Tendenz des
Nationalstaates zur Machtkonzentration in eine totalitäre Kontrolle über
alle Lebensbereiche münden würde. In George Orwells „1984`` (1949)
kämpfte der Einzelne vergeblich darum, unter dem wachsamen Auge von Big
Brother Autonomie und Selbstachtung zu wahren. Es schien eine
aussichtslose Angelegenheit zu sein. Friedrich August von Hayeks „Der
Weg zur Knechtschaft`` (1944) nahm einen wissenschaftlicheren Ansatz,
indem er argumentierte, dass durch eine neue Form der wirtschaftlichen
Kontrollmacht, die den Staat zum absoluten Herrscher macht, die Freiheit
verloren geht. Diese Arbeiten wurden verfasst, bevor Mikroprozessoren
aufkamen, welche eine Fülle von Technologien hervorgebracht haben, die
die Fähigkeit von kleinen Gruppen und sogar Individuen stärken,
unabhängig von einer zentralen Autorität zu agieren.

So scharfsinnig Geister wie Hayek und Orwell auch gewesen sein mögen, so
neigten sie doch zum Pessimismus. Geschichte steckt voller
Überraschungen. Der totalitäre Kommunismus hat kaum das Jahr 1984
überstanden. Im nächsten Jahrtausend könnte zwar eine neue Form der
Knechtschaft entstehen, sollten Regierungen es schaffen, die befreienden
Kräfte der Mikrotechnologie zu unterdrücken. Es ist jedoch weitaus
wahrscheinlicher, dass wir eine beispiellose Gelegenheit und Autonomie
für das Individuum erleben werden. Das, was unsere Eltern als Bedrohung
wahrgenommen haben, könnte sich als unwichtig erweisen. Das, was sie als
stabile und dauerhafte Elemente des gesellschaftlichen Lebens ansahen,
scheint zum Untergang bestimmt zu sein. Dort, wo die Notwendigkeit
Einschränkungen für menschliche Entscheidungen schafft, passen wir uns
an und organisieren unser Leben entsprechend neu.

\subsection{Die Gefahren der
Vorhersage}\label{die-gefahren-der-vorhersage}

Unbestritten riskieren wir einen Verlust unseres geringen Maßes an
Würde, wenn wir versuchen, weitreichende Veränderungen in der
Organisation des Lebens und der Kultur, die es zusammenhält,
vorherzusagen und zu erläutern. Die meisten Prognosen sind zum Scheitern
verurteilt und wirken mit der Zeit geradezu lächerlich. Und je
tiefgreifender die Veränderungen sind, die vorausgesagt werden, umso
häufiger liegen die Vorhersagen peinlich falsch. Die Welt geht nicht
unter. Das Ozon verschwindet nicht einfach. Die angekündigte Eiszeit
wandelt sich in eine globale Erwärmung. Trotz aller gegenteiligen
Befürchtungen ist immer noch Öl im Tank. Mr.~Antrobus, die Hauptfigur in
„Wir sind noch einmal davongekommen``, trotzt der Kälte, überlebt Kriege
sowie drohende wirtschaftliche Katastrophen und wird alt, ohne sich um
die wiederholten Warnungen der Experten zu kümmern.

Die meisten Versuche, die Zukunft „vorherzusagen``, entpuppen sich
schnell als komödiantische Misserfolge. Selbst dort, wo das
Eigeninteresse einen starken Anstoß für klares Denken bietet, ist unsere
Perspektive auf die Zukunft oft verblendet. Im Jahr 1903 verkündete das
Unternehmen Mercedes beispielsweise, dass es „auf der ganzen Welt
niemals eine Million Automobile geben würde``. Der Grund dafür war, dass
es unwahrscheinlich erschien, weltweit eine Million Handwerker zu
Chauffeuren ausbilden zu können.\footnote{Norman Macrae,
  \emph{Governments in Decline}, Cato Policy Report, Juli/August 1992,
  S. 10.}

Das zu erkennen, sollte uns verstummen lassen. Tut es aber nicht. Wir
scheuen uns nicht, uns dem gebührenden Spott auszusetzen. Selbst wenn
wir einen gewaltigen Fehler begehen, dürfen nachfolgende Generationen
nach Herzenslust über uns lachen - vorausgesetzt natürlich, irgendjemand
erinnert sich noch an unsere Worte. Wer mutig genug ist, einen neuen
Gedanken zu formulieren, muss auch bereit sein, sich eventuell zu irren.
Wir sind keinesfalls so unnachgiebig und unbrauchbar, dass wir Angst
davor hätten, uns zu irren. Ganz im Gegenteil. Wir würden es immer
bevorzugen, Ideen aufzuwerfen, die für Sie hilfreich sein könnten,
anstatt diese aufgrund der Angst zu unterdrücken, sie könnten im
Nachhinein als überzogen oder peinlich erscheinen.

Wie Arthur C. Clarke treffend bemerkte, scheitern Vorhersagen für die
Zukunft meistens aus zwei Gründen: „Zu wenig Nerven und zu wenig
Vorstellungskraft``.\footnote{Arthur C. Clarke, \emph{Profiles of the
  Future: An Enquiry into the Limits of the Possible} (London: Victor
  Gollancz Ltd., 1962), S. 13.} Er merkte an: „Zu wenig Nerven scheint
häufiger zu sein. Es tritt ein, wenn der angehende Prophet trotz aller
relevanten Fakten nicht in der Lage ist, die unausweichliche
Schlussfolgerung zu erkennen. Einige solcher Vorhersagefehler sind so
absurd, dass sie nahezu unglaublich erscheinen.`` \footnote{Ebenda.}

Sollte unsere Untersuchung der informationellen Revolution nicht von
Erfolg gekrönt sein, was unweigerlich der Fall sein wird, dann liegt das
eher an unserem Mangel an Vorstellungskraft als an fehlenden Nerven.
Zukunftsprognosen waren stets ein wagemutiges Unterfangen, dem nicht
ohne berechtigte Skepsis begegnet wird. Es könnte sein, dass die Zeit
zeigt, dass unsere Schlussfolgerungen komplett daneben liegen. Wir
behaupten nicht, wie Nostradamus prophetische Fähigkeiten zu besitzen.
Unsere Vorhersagen basieren nicht auf dem Rühren eines Zauberstabs in
einer Wasserschale oder dem Erstellen von Horoskopen. Ebenso wenig
bedienen wir uns kryptischer Verse. Unser Anliegen ist es, Ihnen eine
nüchterne und unvoreingenommene Analyse von Themen zu präsentieren, die
für Sie von großer Bedeutung sein könnten.

Wir sehen es als unsere Aufgabe, unsere Ansichten kundzutun, auch wenn
sie auf den ersten Blick als ketzerisch betrachtet werden könnten.
Gerade weil sie sonst möglicherweise ungehört blieben. In der
geschlossenen Gedankenwelt der spätindustriellen Gesellschaft haben
Ideen nicht den gleichen freien Lauf, wie sie es in etablierten Medien
haben sollten.

Dieses Buch wurde mit einem konstruktiven Gedankenansatz verfasst. Es
ist das dritte, das wir zusammen geschrieben haben, und es analysiert
verschiedene Phasen des umfassenden Wandels, der gegenwärtig
stattfindet. Ähnlich wie „Blood in the Streets`` und \emph{The Great
Reckoning} stellt es eine Denkübung dar. Es beleuchtet das Ende der
Industriegesellschaft und deren Umgestaltung in neue Formen. In den
kommenden Jahren rechnen wir mit erstaunlichen Paradoxien. Einerseits
gibt es Zeugnisse der Manifestation einer neuen Form der Freiheit,
inklusive der Entstehung des selbstbestimmten Individuums. Sie können
eine fast vollständige Entfaltung der Produktivität erwarten.
Gleichzeitig erwarten wir den Untergang des modernen Nationalstaates.
Viele der Gleichheitsgarantien, die die Menschen im Westen im
zwanzigsten Jahrhundert für selbstverständlich hielten, werden mit ihm
zugrunde gehen. Wir gehen davon aus, dass die repräsentative Demokratie,
so wie wir sie heute kennen, verschwinden und durch die neue Demokratie
der Wahlfreiheit auf dem Cybermarktplatz ersetzt werden wird. Wenn
unsere Annahmen korrekt sind, wird die Politik des nächsten Jahrhunderts
wesentlich vielfältiger, aber weniger bedeutend sein als die, an die wir
uns gewöhnt haben.

Wir sind überzeugt, dass unsere Argumentation nachvollziehbar ist,
obwohl sie durch einige Bereiche führt, die man als intellektuelle
Äquivalente zu abgelegenen Dörfern und Problemvierteln bezeichnen
könnte. Wenn unsere Aussagen an manchen Stellen unklar erscheinen, liegt
das nicht daran, dass wir raffiniert sind oder uns der gefeierten
Mehrdeutigkeit derer bedienen, die vorgeben, die Zukunft durch
kryptische Aussagen voraussagen zu können. Wir sind keine Wortverdreher.
Sollten unsere Argumente unverständlich sein, dann deswegen, weil wir es
nicht geschafft haben, überzeugende Ideen klar verständlich zu Papier
gebracht zu haben. Im Gegensatz zu vielen Zukunftsvisionären möchten
wir, dass Sie unseren Denkansatz verstehen und nachvollziehen können. Er
basiert nicht auf übernatürlichen Fantasien oder den Bewegungen der
Planeten, sondern auf nüchterner, altmodischer Logik. Aus rein logischen
Gründen sind wir der Meinung, dass Mikroprozessoren den Nationalstaat
unausweichlich untergraben und zerstören und gleichzeitig neue Formen
sozialer Organisation hervorbringen werden. Es ist sowohl notwendig als
auch möglich, dass Sie zumindest einige Aspekte der neuen Lebensform
erahnen, die vielleicht schneller eintritt, als Sie es für möglich
halten.

\subsection{Die Ironie einer vorhergesagten
Zukunft}\label{die-ironie-einer-vorhergesagten-zukunft}

Seit Jahrhunderten gilt das Ende des letzten Jahrtausends als
entscheidender Wendepunkt in der Geschichte. Vor über 850 Jahren legte
der Heilige Malachias das Jahr 2000 als Zeitpunkt des Jüngsten Gerichts
fest. Edgar Cayce, ein amerikanischer Hellseher, prophezeite im Jahr
1934, die Erde würde sich im Jahr 2000 um ihre Achse drehen, wodurch
Kalifornien zersplittern und sowohl New York City als auch Japan
überschwemmt werden würden. 1980 verkündete Hideo Itokawa, ein
japanischer Raketentechniker, dass die Konstellation der Planeten im
„Großen Kreuz`` am 18. August 1999 eine weitreichende Umweltzerstörung
zur Folge haben und das Ende des menschlichen Lebens auf der Erde
besiegeln würde.\footnote{A. T. Mann, \emph{Millennium Prophecies:
  Predictions for the Year 2000} (Shafiesbury, England: Element Books,
  1992), S. 88ff.}

Apokalyptische Visionen sind ein gefundenes Fressen für Spötter. Zwar
mag das Jahr 2000 aufgrund seiner runden Zahlenform beeindruckend
wirken, was jedoch nichts weiter als ein willkürliches Produkt des
westlich-christlichen Kalenders ist. Andere Kalender und Zeitsysteme
berechnen Jahrhunderte und Jahrtausende aufgrund unterschiedlicher
Ausgangspunkte. Gemäß islamischem Kalender zum Beispiel entspricht das
Jahr 2000 nach Christus dem Jahr 1378. So unspektakulär wie ein Jahr nur
klingen kann. Der chinesische Kalender hingegen, welcher sich alle 60
Jahre wiederholt, bezeichnet das Jahr 2000 n.~Chr. schlichtweg als ein
weiteres Jahr des Drachen, das Teil eines anhaltenden Zyklus ist,
welcher tausende von Jahren in die Vergangenheit reicht. Das Jahr 2000
steht jedoch nicht ausschließlich für theologische Vorhersagen. Seine
Bedeutung ist nicht nur in der christlichen Tradition verwurzelt,
sondern ebenso durch die IT-Einschränkungen des Jahrhunderts. Das
berüchtigte Jahr-2000-Problem oder „Y2K`` stellt einen potentiell
katastrophalen Logikfehler in milliardenfachem Computercode dar und
könnte durchaus das Potential zum Untergangsszenario haben, indem
industrielle Abläufe um die Jahrtausendwende empfindlich beeinträchtigt
werden. Viele Computer und Mikroprozessoren basieren noch immer auf
Software aus den Anfangszeiten der Computertechnologie, als
Speicherplatz mit Kosten von bis zu 600.000 Dollar pro Megabyte
wertvoller als Gold war. Um teuren Speicherplatz zu schonen, nutzten
damalige Programmierer zweistellige Jahreszahlen, die lediglich die
letzten beiden Ziffern des jeweiligen Jahres enthielten. Diese Praxis,
zweistellige Datumsangaben zu verwenden, wurde in der
Großrechnersoftware und sogar bei den sogenannten eingebetteten Chips,
den Mikroprozessoren, die zur Steuerung von so ziemlich allem
herangezogen wurden -- von Videorekordern bis zu Autostartsystemen,
Sicherheitssystemen, Telefonen, Vermittlungssystemen, die das
Telefonnetz regeln, Prozess- und Kontrollsystemen in Fabriken sowie
Kraftwerken, Ölraffinerien, chemischen Werken, Pipelines und vielem mehr
-- nahezu flächendeckend eingeführt. So wurde das Jahr 1999 kurzerhand
auf „99`` reduziert. Das Problem allerdings liegt darin, was geschieht,
wenn 00 für das Jahr 2000 erscheint. Viele Computer interpretieren dies
als 1900, was dazu führen könnte, dass viele nicht umgerüstete Computer
und andere digitale Geräte das Jahr 2000 in den Datumsfeldern nicht
richtig einordnen können.

Das Ergebnis wird eine massive Belastung durch Datenverfälschung sein
und ein zufälliges Muster für neue Potenziale in der
Informationskriegsführung setzen. Im Zeitalter der Information könnten
potenzielle Gegner Schaden anrichten, indem sie „Logikbomben`` zünden,
die die Funktion wichtiger Systeme sabotieren, indem sie die Daten, auf
denen ihre Funktionen basieren, schädigen. Bei einer Militärübung wäre
es beispielsweise nicht notwendig, ein Flugzeug abzuschießen, sofern man
dessen Betriebsdaten beschädigen könnte. Datenbeschädigung kann das
Funktionieren einer modernen Gesellschaft beinahe so stark
beeinträchtigen wie physische Angriffe. Dass dies weitreichende Folgen
haben kann, sollte offensichtlich sein. Die Londoner „Mail`` berichtete
am 14. Dezember 1997, dass Fluggesellschaften weltweit hunderte von
Flügen für den 1. Januar 2000 streichen wollten, da sie befürchteten,
dass die Flugverkehrskontrollsysteme ausfallen könnten.\footnote{Yardeni,
  op. cit., S. 45.} Zu den potenziellen Problemen zählen nicht nur die
Flugverkehrssysteme, sondern auch datumsensitive Funktionen in den
Flugzeugen selbst. Boeing zufolge müssten viele Flugzeuge auf das Jahr
2000 umgerüstet werden. Viele Geräte könnten Probleme haben, wenn sie
versuchen, ein Ereignis mit einem ungültigen Datum zu registrieren. Die
computergesteuerten Fly-by-Wire-Systeme, die zum Betrieb von Flugzeugen
verwendet werden, könnten fehlerhaft funktionieren, wenn sie so
programmiert sind, dass sie schlussfolgern, wichtige Wartungsarbeiten
seien zuletzt im Jahr 1900 durchgeführt worden. Diese könnten sogar in
eine Fehler-Schleife geraten und sich selbst abschalten.

Die potenziell tödlichen Folgen einer zeitlichen Logikbombe, die nicht
konforme Kontrollsysteme lahmlegt, könnten die Jahrtausendwende aus
unschönen Gründen zu einem denkwürdigen Zeitpunkt machen. Bedenken Sie,
dass viele Geräte, die Sie täglich nutzen, in eine Fehlerschleife
geraten und sich abschalten könnten - selbst wenn Sie das Glück haben
sollten, sich nicht in der Luft zu befinden, wenn das neue Jahrtausend
anbricht.

Es wäre ratsam, Unfälle zu vermeiden, die entweder durch
Herzschrittmacher, die nicht mit dem Jahr 2000 kompatibel sind, oder
einfach bloß durch betrunkene Silvesterfeiernde verursacht werden
könnten, denn was Herzschrittmacher ausfallen lässt, könnte auch das
Telekommunikationssystem außer Betrieb setzen und das Eintreffen eines
Krankenwagens verhindern. Solange Sie nicht in Brasilien oder der
Ukraine leben, sind Sie es gewohnt, den Hörer abzuheben oder Ihr
Autotelefon einzuschalten und automatisch ein Freizeichen zu hören.
Glücklicherweise müssen Sie nur selten die technischen Details des
Telefonsystems verstehen. Es stellt sich jedoch heraus, dass die
Telefonvermittlungsstellen und Router stark datumsabhängig sind. Jeder
Anruf wird mit Datum und Uhrzeit protokolliert, was für die Berechnung
der Gesprächsdauer zur Rechnungsstellung essenziell ist. Wenn Sie am 31.
Dezember 1999 um 23:59:30 Uhr ein einminütiges Gespräch führen und das
System um 00:00 Uhr feststellt, dass Ihr Gespräch eine negative Dauer
von mehr als 99 Jahren hatte, könnte das zu Fehlerschleifen und
Ausfällen führen. Zwar investieren die Telekommunikationsunternehmen
viel Geld, um ihre Vermittlungsstellen aufzurüsten und sie fit für das
Jahr 2000 zu machen, und es ist anzunehmen, dass die lokalen Anbieter
dasselbe tun, aber sollte auch nur ein kleines Unternehmen den
Anforderungen nicht gerecht werden und ausfallen, könnte das das gesamte
Netzwerk beeinträchtigen. Sie können sich glücklich schätzen, wenn Sie
am 1. Januar 2000 ein Freizeichen erhalten.

Um es mit den Worten des Experten für das Jahr 2000, Peter de Jager, zu
sagen: „Wenn wir die Möglichkeit verlieren, einen Anruf zu tätigen, dann
verlieren wir alles. Es würde elektronische Geldüberweisungen, den
Handel und Bankfilialen betreffen.`` Die Folgen von Fehlern, die im Jahr
2000 auftreten könnten, könnten sogar noch weitreichender sein.

Heutzutage kann niemand mit Gewissheit vorhersagen, wie groß der
Einfluss des Jahr-2000-Problems auf wichtige Systeme sein wird.
Eingebettete Systeme, die sich nicht umprogrammieren lassen und daher
ersetzt werden müssen, wenn sie aufgrund von Datumsproblemen nicht mehr
funktionieren, finden sich in Karten, Lastwagen und Bussen, die nach
1976 gebaut wurden. (Vielleicht werden Sie nicht in einen Unfall mit
Fahrzeugen verwickelt, die von Personen mit nicht konformen
Herzschrittmachern gefahren werden, da deren Fahrzeuge möglicherweise
nicht starten werden). Eingebettete Systeme sind weit verbreitet, selbst
in Kraftwerken, Wasser- und Abwasseranlagen, medizinischen Geräten,
Militärausrüstung, Flugzeugen, Ölplattformen, Öltankern, Alarmsystemen
und Aufzügen. Auch wenn viele Mikroprozessorsysteme keine
datumsabhängigen Funktionen ausführen, könnten sie möglicherweise von
einer Uhr abhängig sein, die aufgrund des „Millennium-Bugs`` Probleme
bereiten könnte.

\section{GROSSRECHNER UND DIE
JAHR-2000-ZEITBOMBE}\label{grossrechner-und-die-jahr-2000-zeitbombe}

Die umfangreichen Befehls- und Kontrollsysteme von Behörden und
Großunternehmen mit hohem Transaktionsvolumen auf Großrechnern standen
ursprünglich im Zentrum der sogenannten Jahr-2000-Problematik. Da diese
Systeme auf leistungsfähigen Maschinen betrieben werden, deren Mehrzahl
an Software schon Jahrzehnte alt ist und häufig Inkompatibilitäten
aufweist, lag der Schwerpunkt der von Peter de Jager Anfang der 90er
Jahre erstmals geäußerten Warnungen vor dem Jahr-2000-Problem auf der
Notwendigkeit, die Betriebssysteme der zentralen Großrechner mit
Mehrprozessorarchitektur zu aktualisieren. Herr de Jager äußerte die
Befürchtung, dass es möglicherweise nicht genügend Programmierer gibt,
die mit COBOL, der klassischen Großrechner-Sprache, vertraut sind, um
die notwendigen Patches und Reparaturen für datumsabhängigen Code
durchzuführen, selbst wenn jedes Unternehmen und jede Regierungsbehörde
mit einem anfälligen System bereits vor einigen Jahren ein
Crash-Programm initiiert hätte. Da dies jedoch nicht der Fall war und
viele Betreiber datumsabhängiger Informationssysteme erst kürzlich
begonnen haben, ihre Schwachstellen zu analysieren, lässt sich mit an
Sicherheit grenzender Wahrscheinlichkeit vorhersagen, dass eine Menge
von Großrechnersystemen nicht ausreichend auf einen reibungslosen
Betrieb ins Jahr 2000 vorbereitet sein werden.

Dies stellt zweifellos ein enormes Problem dar, denn in der aktuellen
Struktur der Wirtschaft ist keine Alternative zur Datenverarbeitung
durch Computer vorhanden. Die meisten Firmen, die groß genug sind, um
einen Großrechner für ihre Geschäftsabwicklungen zu benötigen, sind auf
ein Transaktionsvolumen angewiesen, das von den veralteten
Papiersystemen des 19. Jahrhunderts unmöglich zu bewältigen wäre. Wären
diese Unternehmen gezwungen, wieder auf Papier umzusteigen, könnten sie
nur einen Bruchteil ihres gewohnten Transaktionsvolumens bewältigen. Der
Einbruch in den Einnahmen, der sich aus einer solchen Verringerung des
Geschäftsvolumens ergäbe, würde das Überleben aller Unternehmen, mit
Ausnahme der am stärksten kapitalisierten, aufs Spiel setzen.

Nahezu alle finanzbezogenen Aspekte - Fakturierungssysteme, Einkaufs-
und Gehaltsabrechnungen, Lagerverwaltung und die Einhaltung gesetzlicher
Vorschriften - könnten ins Chaos gestürzt werden. Unmengen an Daten
könnten verloren gehen, wenn Computer abstürzen oder infolge des
Jahr-2000-Problems fehlerhafte Daten produzieren. In manchen Fällen
könnte es sich sogar als Glücksfall erweisen, wenn die Systeme sofort
abstürzen, anstatt ihre Daten schrittweise zu beschädigen, bis
gravierende Störungen auf das Problem aufmerksam machen. Was geschieht
mit Dateien, die von einem Backup-Programm vom 7. April
\textquotesingle99 auf eine Version vom 1. April \textquotesingle00
kopiert werden? Wer kann das schon sagen? Wird ein Computer eine am 4.
Januar „1900`` getätigte Versicherungszahlung als Anzeichen deuten, dass
die Zahlung seit einem Jahrhundert überfällig ist, was zu ihrer
Annullierung und Löschung aus den Akten führt? Werden Bank- und
Finanzcomputer versuchen, hundert Jahre Zinsen auf Kredite zu berechnen,
die das neue Jahrtausend überspannen? Werden Ihre Banken und
Broker-Firmen genaue Aufzeichnungen Ihrer Kontostände führen und Ihnen
zeitnah Zugang zu Ihren Geldern gewähren? Dies sind nur einige der
faszinierenden Fragen, die sich im Zusammenhang mit dem
Jahr-2000-Problem ergeben werden.

\begin{quote}
„Dies ist der womöglich verheerendste Aspekt des Jahr-2000-Problems. Wir
sprechen hier nicht über die Unannehmlichkeiten, die entstehen, wenn Ihr
Gehalt ein paar Tage verspätet eintrifft. Hierbei handelt es sich um den
Teil, bei dem buchstäblich das Blut auf den Straßen fließt.`` - Dr.~Leon
Kappelman, stellvertretender Vorsitzender der Jahr-2000-Arbeitsgruppe
der Gesellschaft für Informationsmanagement.
\end{quote}

Ganz oben auf der Liste all Ihrer Sorgen sollte die Frage stehen: Was
geschieht, wenn der Strom wegen der Probleme, die das Jahrtausendproblem
verursacht, ausfällt? Selbst die widerstandsfähigsten Systeme, die gar
nicht von der Jahr-2000-Problematik betroffen sind, würden ohne Strom
nicht funktionieren: Ihr Kühlschrank, Ihr Gefrierschrank, vielleicht
sogar Ihre Heizung. Jahr-2000-Probleme könnten wichtige Zutritts- und
Kontrollfunktionen in Atomkraftwerken beeinträchtigen. Beispielsweise
tragen Mitarbeiter in Atomkraftwerken Dosimeter, die die
Strahlenbelastung messen, der sie während ihres Aufenthalts in der
Anlage ausgesetzt sind. Diese Geräte werden regelmäßig analysiert und
die Daten zur Strahlenbelastung in einem Computersystem gespeichert, das
den Zutritt des Personals zur Anlage überwacht. Es leuchtet ein, dass
ein Ausfall dieser Kontrollcomputer selbst die ausgefeiltesten
Überwachungsmechanismen, die einen sicheren Betrieb und ordnungsgemäße
Wartungsarbeiten gewährleisten sollen, zunichtemachen würde. Aber noch
beunruhigender ist die in einem Memo der Kommission für die Regulierung
von Kernkraftwerken festgestellte Tatsache, dass viele „nicht
sicherheitskritische, aber dennoch wichtige computerbasierte Systeme,
vor allem Datenbanken und Datenerfassungen, die für den Betrieb der
Anlage notwendig sind``, datumsabhängig sind.

Konventionelle Kraftwerke sind keineswegs unempfindlich gegenüber
Störungen, die im Zuge des Jahrtausendwechsels auftreten können.
Insbesondere sind kohlebetriebene Kraftwerke anfällig für Ausfälle des
Oberflächentransportsystems, das Kohle zu den Kesseln bringt. In der
winterlichen Heizperiode 1997-1998 wurden Betreiber von Kohlekraftwerken
dazu gezwungen, ihre Leistung in einigen Fällen zu reduzieren. Dies
geschah, da die Bahnlieferungen von westlicher Kohle aufgrund der Fusion
der Eisenbahnsysteme Southern Pacific und Union Pacific verlangsamt
wurden. Das Problem rührte von Unverträglichkeiten zwischen den
Computersteuerungs- und Abfertigungssystemen her, die von den beiden
Eisenbahngesellschaften eingesetzt wurden. Ein Sprecher der Union
Pacific beschrieb die Integration der beiden Systeme als „Albtraum``,
und das trotz der Tatsache, dass Union Pacific Technologies als
Branchenführer in der Entwicklung von computergestützten
Transportkontrollsystemen galt. Aufgrund von Programmierschwierigkeiten
war es der Eisenbahngesellschaft nicht möglich, die Bewegungen ihrer
Güterwagen genau zu verfolgen. Dass Union Pacific die Übernahme von
Southern Pacific nicht erfolgreich bewältigen konnte, deutet darauf hin,
welche Probleme auftreten könnten, wenn die zeitlichen Logikbomben des
Jahres 2000 das Transportwesen, die Energieproduktion und andere
Wirtschaftsbereiche stören.

Die größte Sorge hinsichtlich des Stromnetzes entsteht jedoch aus der
Notwendigkeit, dass das gesamte System einer sensiblen Überwachung und
Computersteuerung unterliegt, um Strom von Überschuss- zu
Defizitregionen zu leiten. Dieser Ablauf muss sorgsam per Computer
überwacht werden, um Stromspitzen und Systemausfälle zu verhindern. Alle
durchgeführten Stromübertragungen werden mit Datum und Uhrzeit
dokumentiert, ähnlich wie bei einer Telefonverbindung. Die Verbindungen
selbst werden mittels robuster mechanischer Relais realisiert, welche
allerdings von Computersystemen gesteuert werden. Diese
Computersteuerungen, unerlässlich für die Lastverteilung, können aus
denselben Gründen wie Telekommunikationsnetzwerke ausfallen. Tatsächlich
sind die Systeme zur Steuerung der Lastverteilung in Nordamerika
miteinander über T-1-Leitungen und Telefon-Mikrowellenverbindungen
vernetzt. Wenn also das Telefonnetz ausfällt, ist es durchaus
wahrscheinlich, dass auch der Stromausfall folgt. Und wie die
Erfahrungen aus Kanada im Januar 1998 zeigen, kann es eine große
Herausforderung sein, das System erneut zum Laufen zu bringen, sobald
der Strom in einem größeren Gebiet ausfällt. Ein Blackout kann
unangenehm lange andauern.

\section{Y2K UND DIE ATOMWAFFEN}\label{y2k-und-die-atomwaffen}

Ein Stromausfall inmitten des Winters wäre für moderne Volkswirtschaften
eine Katastrophe, ganz zu schweigen von den potenziellen
Gesundheitsrisiken, insbesondere für diejenigen, die auf elektrische
Heizsysteme oder medizinische Geräte angewiesen sind. Aber die
schlimmste denkbare Situation könnte noch gravierender sein. John
Koskinen, zur damaligen Zeit Leiter des Y2K Conversion Council unter
Präsident Clinton, äußerte die Besorgnis, dass die Waffensysteme des
US-Militärs am 31. Dezember 1999 um Mitternacht versagen könnten.
Koskinen möchte zwar keine unverhältnismäßige Panik schüren, betont
jedoch: „Man muss sich darüber Gedanken machen``. Eine spezifische Sorge
hinsichtlich der Nuklearraketen sei, „wenn die Daten nicht funktionieren
und sie tatsächlich abgefeuert werden``.

Natürlich würden diese Bedenken in gleichem oder sogar größerem Maße auf
russische Atomraketen zutreffen. Der Finanzkollaps Russlands hat die
Aufrüstung auf Jahrtausend-Fähigkeiten noch problematischer gemacht als
in den USA. Zudem gibt es Anzeichen dafür, dass Russland die Problematik
der Jahrtausendwende noch nicht ernst genug nimmt. Auch wenn man
inständig hofft, dass es nicht zu unbeabsichtigten Raketenstarts kommt,
sollte es kaum Zweifel daran geben, dass der Übergang ins Jahr 2000 das
Potential hat, die globale Unsicherheit zu verschärfen. Und das aus
keinem anderen Grund als der möglichen Funktionsstörung der
militärischen Kommunikationssysteme in vielen Ländern. Wie Koskinen es
formuliert: „Wenn man in einem Land sitzt und plötzlich nicht mehr
sicher ist, was vor sich geht und die Kommunikation nicht mehr
reibungslos funktioniert, wird man noch nervöser.`` Fügen Sie das also
zu Ihrer Liste der „Y2K-Sorgen`` hinzu. Die zeitliche Logikbombe könnte
den Abschuss von tatsächlich explosiven Bomben auslösen -- eine
Tatsache, die die Gefahren einer Informationskriegsführung für
zentralisierte Befehls- und Kontrollsysteme verdeutlicht.

Sollten Terroristen das Ziel haben, ein zentralisiertes System
anzugreifen, könnten sie als Termin hierfür den 31. Dezember 1999 ins
Auge fassen, da zu diesem Zeitpunkt viele Systeme äußerst verwundbar
sein könnten. Dabei geht es nicht allein darum, dass die Kommunikation
mutmaßlich gestört sein wird - womöglich fällt der Strom aus, Fahrzeuge
springen nicht mehr an, die Notrufsysteme von Polizei, Feuerwehr und
Krankenwagen versagen und so weiter. Auch viele andere Funktionen, die
wir für selbstverständlich halten, wie zum Beispiel die
Flugverkehrskontrolle, könnten zum Erliegen kommen. Ohne Strom gibt es
kein Wasser aus dem Hahn, Abwassersysteme würden zusammenbrechen.
Verkehrsampeln könnten ausfallen. Nur wenige Stunden nach einem
vollständigen Verkehrskollaps könnten die Lebensmittelläden leergeräumt
(oder geplündert) sein. Basierend auf jüngsten Erfahrungen in
amerikanischen Städten lässt sich mutmaßen, dass ein Mangel an Strom,
Wasser und Wärme für viele Menschen, kein Licht und eine unzuverlässige
Kommunikation mit den Notdiensten - einschließlich Polizei und Feuerwehr
- letztlich zum Zusammenbruch der Zivilisation führen könnte. Zwar kann
niemand verbindlich prognostizieren, welche Auswirkungen das
Jahr-2000-Problem haben wird, doch könnten Plünderungen und
Ausschreitungen auf den Straßen die Folge sein, insbesondere wenn
bekannt wird, dass es wahrscheinlich zu weitreichenden Ausfällen bei
Gehalts-, Sozialhilfe- und Rentenauszahlungen kommen könnte.

\begin{quote}
„Wir werden nicht länger das sein, was wir einst waren, sondern
beginnen, uns zu verändern.`` - Joachim de Fiore\footnote{Zitiert in
  Frooso, op. cit., S. 40.}
\end{quote}

Die düsteren Prophezeiungen für das neue Jahrtausend basieren nicht
zwangsläufig auf einer christlich geprägten Theologie, doch sie lassen
sich gut in die Jahrtausendtradition von Joachim de Fiore einordnen.
Seine Meditationen überzeugten ihn davon, dass Christus lediglich den
„zweiten Wendepunkt der Geschichte`` darstellt und sich „ein weiterer
unweigerlich entfalten würde``.\footnote{Ebenda.} So argumentiert auch
der Philosoph Michael Grosso, der behauptet, dass die informationelle
Revolution die Menschheitsgeschichte in Richtung der Verwirklichung der
prophetischen Vision der westlichen Welt lenkt. Er bezeichnet dies als
„Technokalypse``. Unabhängig davon, ob die technologische Entwicklung
auf irgendeine Weise von den Visionen des neuen Jahrtausends beeinflusst
wird oder nicht, stellt das Jahr-2000-Phänomen ein Artefakt der
vorherrschenden westlichen Zeitvorstellung dar. Auf ungewöhnliche Weise
könnte es Träume, Fantasien und Visionen oder numerische Deutungen von
Visionen ergänzen, wie etwa Newtons Erläuterung der Prophezeiungen
Daniels. Solche intuitiven Sprünge beginnen stets mit der Betrachtung
der Geburt Christi als zentralem historischen Ereignis. Sie werden durch
die psychologische Kraft großer runder Zahlen verstärkt, die jeden
Händler in ihren Bann zieht. Das Jahr 2000 kann nicht umhin, ein
Brennpunkt für die Fantasie intuitiver Menschen zu werden.

Kritiker mögen diese Prophezeiungen als lächerlich beiseite fegen, ohne
sich mit den mehrdeutigen und umstrittenen theologischen Konzepten der
Apokalypse und des Jüngsten Gerichts auseinanderzusetzen, die diesen
Visionen eine gewaltige Kraft verleihen. Interessanterweise übertrumpft
der Jahr-2000-Computerbug jedoch die rechnerischen Fehler, die ansonsten
die Bedeutung des Jahres 2000 selbst im christlichen Rahmen zu entwerten
scheinen. Das Jahr 2000 hat das Potential, einen Wendepunkt für die
nächste Phase der Geschichte darzustellen, allein aufgrund der
vorgezogenen Ankunft des neuen Jahrtausends. Streng genommen beginnt das
nächste Jahrtausend erst im Jahr 2001. Das Jahr 2000 markiert lediglich
das zweitausendste Jahr nach Christi Geburt. Genau genommen wäre dies
der Fall gewesen, wenn Christus im ersten Jahr der christlichen Ära zur
Welt gekommen wäre. Aber das war nicht der Fall. Im Jahr 533, als das
Geburtsdatum von Christus die Gründung Roms als Grundlage für die
Zeitrechnung nach westlichem Kalender ersetzte, machten die Mönche, die
diese neue Konvention einführten, einen Fehler in Bezug auf das
Geburtsjahr Christi. Heutzutage wird angenommen, dass er im Jahr 4 v.
Chr. geboren wurde. Unter dieser Prämisse, wären die zweitausend Jahre
seit seiner Geburt irgendwann im Jahr 1997 vollendet gewesen. Das
erklärt Carl Jungs scheinbar kurioses Anfangsdatum für den Beginn eines
neuen Zeitalters.

Sie dürfen gern schmunzeln, aber wir verachten oder ignorieren
keinesfalls das intuitive Verständnis der Geschichte. Obgleich unsere
Argumentation auf Logik und nicht auf Annahmen beruht, sind wir dennoch
vom prophetischen Potential des menschlichen Bewusstseins beeindruckt.
Immer wieder bestätigen sich die Visionen von Außenseitern, Sehern und
Heiligen. So könnte es auch bei der Wende des Jahres 2000 der Fall sein.
Dieses Datum, das sich seit langem in den Köpfen des Westens verankert
hat, scheint einen Wendepunkt zu markieren, der zumindest partiell
bestätigt, dass Geschichte ein Schicksal hat. Wir können nicht erklären,
warum das so ist, doch wir sind überzeugt davon, dass es so ist.

Unsere Intuition lässt uns glauben, dass Geschichte einem Schicksal
folgt, und dass freier Wille und Determinismus nur zwei Seiten der
gleichen Medaille sind. Es scheint, als würden die menschlichen
Interaktionen, die die Geschichte prägen, von einer Art Schicksal
geleitet werden. Menschen verhalten sich genauso wie ein
Elektronenplasma, ein dichtes Gas aus Elektronen -- als komplexes
System. Die individuelle Bewegungsfreiheit der Elektronen passt
überraschend gut zu hochgradig organisiertem kollektivem Verhalten. Wie
David Ohm einmal über ein Elektronenplasma sagte, so ist auch die
menschliche Geschichte „ein hochgradig organisiertes System, das als
Einheit agiert``.

Um die Funktionsweise der Welt zu begreifen, muss man sich ein
realistisches Bild davon machen, wie sich die menschliche Gesellschaft
den mathematischen Gesetzen natürlicher Prozesse unterordnet. Die
Realität verläuft nicht linear, aber die Erwartungen der meisten
Menschen schon. Um das Wesen des Wandels zu durchdringen, sollte man
sich bewusst machen, dass menschliche Gesellschaften, ähnlich wie andere
komplexe Systeme in der Natur, durch wiederkehrende Zyklen und
Diskontinuitäten geprägt sind. Das heißt, bestimmte geschichtliche
Merkmale neigen dazu, sich zu wiederholen und die signifikantesten
Veränderungen treten, wenn sie denn auftauchen, eher schlagartig als
allmählich auf.

Im menschlichen Leben gibt es zahlreiche Zyklen, doch scheint ein
mysteriöser Fünfhundert-Jahres-Zyklus bedeutsame Meilensteine in der
Geschichte der westlichen Zivilisation zu bestimmen. Während wir uns dem
Jahr 2000 nähern, werden wir mit der merkwürdigen Beobachtung
konfrontiert, dass das letzte Jahrzehnt jedes Jahrhunderts, das durch
fünf teilbar ist, einen signifikanten Wandel in der westlichen
Zivilisation eingeläutet hat. Es bildet ein Muster von Tod und
Wiedergeburt, das neue Stadien der gesellschaftlichen Organisation
kennzeichnet, so wie Tod und Geburt den Zyklus der menschlichen
Generationen beschreibt. Dieses Phänomen lässt sich mindestens bis ins
Jahr 500 v. Chr. zurückverfolgen, als die griechische Demokratie mit den
Verfassungsreformen des Kleisthenes im Jahr 508 v. Chr. ihren Anfang
nahm. Die darauffolgenden fünf Jahrhunderte stellten eine Periode des
Wachstums und der Intensivierung der antiken Ökonomie dar, die ihren
Höhepunkt in der Geburt Christi im Jahr 4 v. Chr. fand. Dies war auch
die Zeit des größten Wohlstands der antiken Wirtschaft, in der die
Zinssätze ihren niedrigsten Stand vor der Neuzeit erreichten.

Im Verlauf der fünf darauffolgenden Jahrhunderte nahm der Wohlstand
schrittweise ab, was schlussendlich zum Zusammenbruch des römischen
Reiches gegen Ende des 5. Jahrhunderts n.~Chr. führte. Es lohnt sich,
William Playfairs Zusammenfassung zu zitieren: „Als Rom auf dem Gipfel
seiner Macht angelangt war... konnte man feststellen, dass dies zur Zeit
der Geburt Christi war, also während der Herrschaft des Augustus, und
ebenso stellt man fest, dass es bis 490 n.~Chr. stetig zurückging.`` Zum
besagten Zeitpunkt lösten sich die letzten Legionen auf und die
westliche Welt versank in den Wirren des dunklen
Mittelalters.\footnote{William Playfair, \emph{An Inquiry into the
  Permanent Causes of the Decline and Fall of Powerful and Wealthy
  Nations: Designed to Shew How the Prosperity of the British Empire May
  be Prolonged} (London: Greenland and Norris, 1805), S.79.}

In den kommenden fünf Jahrhunderten erlebte die Wirtschaft einen
erheblichen Verfall, der Fernhandel erlag einer Stagnation, Städte
wurden entvölkert, Geld verschwand aus dem Umlauf und Kunst und
Alphabetisierung gingen nahezu vollständig verloren. Mit dem
Zusammenbruch des römischen Reiches im Westen und dem daraus
resultierenden Mangel an wirksamem Recht entstanden primitivere Regeln
zur Streitbeilegung. Die Blutrache gewann gegen Ende des fünften
Jahrhunderts mehr und mehr an Relevanz. Der erste historisch belegte
Gerichtsprozess fand schließlich im Jahr 500 statt.

Einmal mehr, vor einem Jahrtausend, ereignete sich im letzten Jahrzehnt
des zehnten Jahrhunderts eine „monumentale Veränderung der sozialen und
wirtschaftlichen Systeme``. Eine dieser Übergänge, wahrscheinlich die am
wenigsten bekannte, die feudale Revolution, begann inmitten einer
Periode voller wirtschaftlicher und politischer Turbulenzen. Guy Bois,
ein Professor für Mittelaltergeschichte an der Universität Paris,
argumentiert in seinem Buch „Umbruch im Jahr 1000``, dass dieser Wandel
am Ende des zehnten Jahrhunderts den völligen Zusammenbruch der Reste
alter Institutionen und das Entstehen von etwas Neuem aus der Anarchie
des Feudalismus bedeutete.\footnote{Guy Bois, \emph{The Transformation
  of the Year One Thousand: The Village of Lournard from Antiquity to
  Feudalism} (Manchester, England: Manchester University Press, 1992).}
Raoul Glaber formuliert es so: „Man sagte, die gesamte Welt schüttelte
einvernehmlich die Trümmer der Antike ab.`` \footnote{Ebenda, S. 150.}
Das neu aufkommende System ermöglichte eine allmähliche Wiederbelebung
des wirtschaftlichen Wachstums. In den fünf Jahrhunderten, die wir heute
als Mittelalter bezeichnen, erlebten wir eine Renaissance des Geldes und
des internationalen Handels. Auch die Arithmetik, das Lesen und
Schreiben und ein Bewusstsein für Zeit wurden wiederentdeckt.

Im letzten Jahrzehnt des 15. Jahrhunderts erreichte Europa einen
weiteren Meilenstein. An diesem Punkt hatte es das durch die Pest
verursachte demographische Defizit überwunden und übernahm unmittelbar
im Anschluss fast die gesamte Kontrolle über die restliche Welt. Der
Übergang in dieses neue Zeitalter, gekennzeichnet durch die
„Schießpulverrevolution``, die „Renaissance`` und die „Reformation``,
wurde mit dem Einmarsch Karls VIII. in Italien unter Einsatz von neuen
Bronzekanonen einprägsam eingeleitet. Dies ging einher mit der Öffnung
Europas zur Welt, verkörpert durch Christoph Kolumbus' Reise nach
Amerika im Jahr 1492. Diese Öffnung für die neue Welt startete das
bisher dramatischste wirtschaftliche Wachstum in der menschlichen
Geschichte, führte zum Umbruch von Physik und Astronomie und in der
Konsequenz zur Entstehung moderner Wissenschaften. Und die durch diese
Epoche generierten Ideen wurden mit Hilfe der neuartigen Technologie der
Druckerpresse weitflächig verbreitet.

Wir stehen nun am Vorabend eines weiteren Jahrtausendwechsels. Die
großen Befehls- und Kontrollsysteme, die aus dem Industriezeitalter
stammen, könnten mit dem Glockenschlag der tausendjährigen Mitternacht
kollabieren, ähnlich wie eine einspännige Kutsche. Unabhängig davon, ob
die sogenannte „Jahr-2000-Logikbombe`` einen unmittelbaren Zusammenbruch
der Industriegesellschaft herbeiführt oder nicht - ihre Tage sind
gezählt. Wir gehen davon aus, dass das Aufkeimen der
Informationsgesellschaft tiefgreifende Veränderungen auf der Welt
bewirkt, welche in diesem Buch dargelegt werden sollen. Es steht Ihnen
natürlich frei, dies in Frage zu stellen; kein Zyklus, der nur zweimal
in einem Jahrtausend stattfindet, hat genügend Wiederholungen geboten,
um statistisch signifikant zu sein. Selbst erheblich kürzere Zyklen sind
von Ökonomen mit Skepsis betrachtet worden, die nach statistisch
stichhaltigeren Beweisen verlangten. Professor Dennis Robertson merkte
einst an, dass wir „besser einige Jahrhunderte abwarten sollten, bevor
wir uns der Existenz von Vierjahres- sowie Acht- bis
Zehnjahres-Handelszyklen sicher sind``.\footnote{Zitiert in S. B. Saul,
  \emph{The Myth of the Great Depression} (London: Macmillan, 1985), S.
  10.} Nach diesem Maßstab müsste Professor Robertson sein Urteil rund
dreißigtausend Jahre aufschieben, um sicherzustellen, dass der
Fünfhundertjahreszyklus kein statistischer Zufall ist. Wir sind weniger
dogmatisch und eher bereit anzuerkennen, dass die Muster der Realität
komplexer sind als die statischen und linearen Gleichgewichtsmodelle,
die die meisten Wirtschaftswissenschaftler betrachten.

Wir sind der Überzeugung, dass das Jahr 2000 mehr als nur eine weitere
zweckdienliche Teilung im unendlichen Zeitgefüge bedeutet. Wir glauben,
es wird einen Wendepunkt zwischen der alten und der herannahenden neuen
Welt darstellen. Das Industriezeitalter neigt sich rapide dem Ende zu
und paradoxerweise könnte der Untergang durch die ursprünglich hohen
Kosten für Computerspeicher beschleunigt worden sein, die zur
weitflächigen Implementierung von zweistelligen Datumsfeldern führten.
Als Hallerith-Lochkarten lediglich achtzig Zeichen speichern konnten,
schien eine Verkürzung der Datumsangaben sinnvoll. Entgegen der
Befürchtungen der frühen Programmierer hat ihre Verkürzung des
Datumsfelds jedoch vier Jahrzehnte bis zum Jahrtausendende überdauert,
als eine ungewollte zeitliche Logikbombe, die große Teile der
Industriegesellschaft zerstören könnte. Das Office of Management and
Budget der US-Regierung beschrieb das Problem in „Getting Federal
Computers Ready for 2000``, einem Bericht vom 7. Februar 1997. Das OMB
kommt hinsichtlich von Computern zu dem Schluss: „Wenn sie nicht
repariert oder ersetzt werden, werden sie zur Jahrtausendwende auf eine
von drei Arten versagen: Sie werden gültige Eingaben ablehnen, falsche
Ergebnisse berechnen oder schlichtweg nicht funktionieren.`` Diese drei
Szenarien könnten gemeinsam die Industriegesellschaft lahmlegen. Die
Massenproduktionstechnologie wird unweigerlich von neuer
Miniaturisierungstechnologie verdrängt werden. Eine kurzfristige Krise
würde diesen Prozess nur beschleunigen. Mit der neuen
Informationstechnologie ist eine neue Wissenschaft der nichtlinearen
Dynamik entstanden, deren verblüffende Schlussfolgerungen lediglich Teil
eines noch zu strickenden umfangreichen Weltbildes sind. Wir leben im
Computerzeitalter, aber unsere Träume werden immer noch am Webstuhl
gesponnen. Wir leben weiterhin in den Metaphern und Denkmustern des
Industrialismus. Unsere Politik verläuft immer noch entlang der
industriellen Spaltung zwischen Rechts und Links, wie sie von Denkern
wie Adam Smith und Karl Marx skizziert wurde, die starben, bevor
praktisch alle heute lebenden Menschen geboren wurden.\footnote{Adam
  Smith starb 1790, Karl Marx 1883.} Die industrielle Weltanschauung,
die die Funktionsprinzipien der industriellen Wissenschaft umfasst, ist
immer noch der intuitiv wahrgenommene „gesunde Menschenverstand`` der
unterrichteten Meinung. Unsere Hypothese ist, dass der „gesunde
Menschenverstand`` des Industriezeitalters in vielen Bereichen nicht
mehr greifen wird, da sich die Welt verändert.

Über 85 Jahre nach dem Tag im Jahr 1911, an dem Oswald Spengler die
Vision eines bevorstehenden Weltkriegs und des „Untergangs des
Abendlands`` hatte, erleben auch wir einen „historischen
Paradigmenwechsel, der sich genau an jenem Punkt vollzieht, der ihm vor
Jahrhunderten vorherbestimmt war.`` \footnote{Oswald Spengler, \emph{The
  Decline of the West}, Übersetzung ins Englische von Charles Francis
  Atkinson, zitiert in I. F. Clark, The Pattern of Expectation,
  1644-2001 (London: Jonathan Cape, 1979), S. 220.} Wie Spengler
prophezeien auch wir den nahenden Untergang der westlichen Zivilisation
und damit den Zusammenbruch der Weltordnung, die die vergangenen fünf
Jahrhunderte dominierte, seit Kolumbus nach Westen gesegelt ist, um den
Kontakt zur Neuen Welt herzustellen. Im Gegensatz zu Spengler sehen wir
jedoch das Aufkommen einer neuen Phase der westlichen Zivilisation im
heraufziehenden neuen Jahrtausend.

\bookmarksetup{startatroot}

\chapter{MEGAPOLITISCHE VERÄNDERUNGEN IM HISTORISCHEN
KONTEXT}\label{megapolitische-veruxe4nderungen-im-historischen-kontext}

\begin{quote}
„In der Geschichte wie in der Natur sind Geburt und Tod gleichermaßen
ausgeglichen.`` - Johan Huizinga\footnote{Huizinga, ebenda, S. 7.}
\end{quote}

\section{DER VERFALL DER MODERNEN
WELT}\label{der-verfall-der-modernen-welt}

Aus unserer Sicht erleben Sie nichts Geringeres als den Verfall des
modernen Zeitalters. Es ist eine Entwicklung, die von einer gnadenlosen,
aber verborgenen Logik angetrieben wird. Es ist mehr als wir im
Allgemeinen verstehen und mehr als CNN und die Zeitungen uns erzählen.
Das nächste Jahrtausend wird nicht mehr „modern`` sein. Wir sagen dies
nicht, um anzudeuten, dass Sie einer wilden oder rückständigen Zukunft
gegenüberstehen, obwohl das möglich ist, sondern um zu betonen, dass die
jetzt beginnende Phase der Geschichte qualitativ anders sein wird als
die, in die Sie hineingeboren wurden.

Etwas Neues kommt. So wie sich landwirtschaftliche Gesellschaften
grundlegend von Jäger-und-Sammler-Gruppen unterschieden haben, wie sich
industrielle Gesellschaften radikal von feudalen oder kleinbäuerlichen
Agrarsystemen unterschieden haben, wird auch die zukünftige Welt, die
uns erwartet, eine radikale Abkehr von allem darstellen, was wir bisher
gesehen haben.

Im neuen Jahrtausend wird das wirtschaftliche und politische Leben nicht
mehr in gigantischem Maßstab unter der Herrschaft des Nationalstaates
organisiert sein, wie es während der modernen Jahrhunderte der Fall war.
Die Zivilisation, die Ihnen den Weltkrieg, das Fließband, die soziale
Sicherheit, die Einkommensteuer, das Deodorant und den Minibackofen
brachte, stirbt aus. Das Deodorant und der Minibackofen könnten
überleben. Die anderen nicht. Wie ein alter und einst mächtiger Mann,
hat der Nationalstaat eine Zukunft, die in Jahren und Tagen, und nicht
mehr in Jahrhunderten und Jahrzehnten, bemessen ist.

Regierungen haben bereits viel von ihrer Macht, zu regulieren und zu
erzwingen, verloren. Der Zusammenbruch des Kommunismus markierte das
Ende eines langen Zyklus von fünf Jahrhunderten, währenddessen die
Größenordnung der Macht die Effizienz in der Organisation der Regierung
überwältigte. Es war eine Zeit, in der die Renditen für Gewalt hoch
waren und weiter stiegen. Sie sind es nicht mehr. Ein Phasenübergang von
weltgeschichtlichen Ausmaßen hat bereits begonnen. Tatsächlich könnte
der zukünftige Geschichtsschreiber, der den Niedergang und den Fall des
einstigen modernen Zeitalters im nächsten Jahrtausend niederschreibt,
erklären, dass es bereits zu Ende gegangen ist, während Sie gerade
dieses Buch lesen. Rückblickend könnte er, so wie wir, auch sagen, dass
es mit dem Fall der Berliner Mauer im Jahr 1989 endete. Oder mit dem
Zusammenbruch der Sowjetunion im Jahr 1991. Beide Daten könnten als
maßgebliches Ereignis in der Entwicklung der Zivilisation stehen, das
Ende dessen, was wir heute als modernes Zeitalter kennen.

Die vierte Stufe der menschlichen Entwicklung kommt, und vielleicht ist
das am wenigsten voraussagbare Merkmal der neue Name, unter dem sie
bekannt sein wird. Nennen Sie es „Post-Modern.`` Nennen Sie es die
„Cyber-Gesellschaft`` oder das „Informationszeitalter``. Oder erfinden
Sie Ihren eigenen Namen. Niemand weiß, welches Etikett auf die nächste
Phase der Geschichte geklebt werden wird.

Wir wissen nicht einmal, ob der gerade endende fünfhundertjährige
Abschnitt der Geschichte weiterhin als „modern`` betrachtet werden wird.
Wenn zukünftige Historiker irgendetwas über Wortableitungen wissen, wird
dem wahrscheinlich nicht so sein. Ein passenderer Titel könnte
„Staatszeitalter`` oder „Das Zeitalter der Gewalt`` sein. Aber ein
solcher Name würde außerhalb des zeitlichen Spektrums fallen, das
derzeit die Epochen der Geschichte definiert. „Modern`` bezieht sich
laut Wörterbuch auf die Gegenwart und die jüngste Vergangenheit, im
Gegensatz zur fernen Vergangenheit... Im historischen Gebrauch wird es
üblicherweise (im Gegensatz zu antik und mittelalterlich) auf die Zeit
im Anschluss an das Mittelalter angewendet.`` \footnote{\emph{The
  Compact Edition of the Oxford English Dictionary}, Vol. 1 (Oxford:
  Oxford University Press, 1971), S. 1828.}

Menschen im Westen begannen sich erst bewusst als „modern`` zu
bezeichnen, als sie begriffen, dass das mittelalterliche Zeitalter
vorbei war. Vor 1500 hatte niemand die feudalen Jahrhunderte als eine
„mittlere`` Periode in der westlichen Zivilisation angesehen. Der Grund
dafür liegt auf der Hand: bevor ein Zeitalter vernünftigerweise als in
der „Mitte`` zwischen zwei anderen historischen Epochen gesehen werden
kann, muss es bereits zu Ende gegangen sein. Diejenigen, die während der
feudalen Jahrhunderte lebten, hätten sich nicht vorstellen können
Bewohner eines Durchgangshauses zwischen Antike und moderner
Zivilisation zu sein, bis ihnen bewusst wurde, dass nicht nur das
Mittelalter vorbei war, sondern dass sich die mittelalterliche
Zivilisation stark von der Zivilisation des dunklen Zeitalters oder der
Antike unterschied.\footnote{Michael Hicks, \emph{Bastard Feudalism}
  (London: Longmans, 1995), S. 1.}

Menschliche Kulturen haben blinde Flecken. Uns fehlt der Wortschatz, um
Paradigmenwechsel, speziell diejenigen, die gerade um uns herum
stattfinden, zu beschreiben. Ungeachtet der vielen dramatischen
Veränderungen, die sich seit der Zeit von Moses ereignet haben, haben
sich nur ein paar Ketzer die Mühe gemacht, darüber nachzudenken, wie die
Übergänge von einer Phase der Zivilisation zur anderen tatsächlich
ablaufen.

Wie werden sie ausgelöst? Was haben sie gemeinsam? Welche Muster können
Ihnen helfen zu erkennen, wann sie beginnen und wann sie enden? Wann
werden Großbritannien oder die Vereinigten Staaten untergehen? Das sind
Fragen, für die man nur schwer konventionelle Antworten finden würde.

\subsection{Das Tabu der
Zukunftsprognosen}\label{das-tabu-der-zukunftsprognosen}

„Von außen`` auf ein bestehendes System zu schauen, gleicht einem
Bühnenarbeiter, der versucht, einen Dialog mit einer Figur in einem
Theaterstück zu führen. Es bricht eine Konvention, die dem System hilft,
zu funktionieren. Jede gesellschaftliche Ordnung enthält unter ihren
großen Tabus die Vorstellung, dass Menschen, die darin leben, nicht
darüber nachdenken sollten, wie sie enden wird und welche neuen Regeln
das neue System, das ihren Platz einnimmt, beherrschen könnten. Implizit
ist jedes bestehende System das letzte oder das einzige System, das
jemals existieren wird. Nicht dass dies so offen ausgesprochen würde.
Nur wenige, die jemals ein Geschichtsbuch gelesen haben, würden eine
solche Annahme für realistisch halten, wenn sie ausformuliert wäre.
Dennoch, genau das ist die Konvention, die die Welt beherrscht. Jedes
soziale System, egal wie fest es sich an die Macht klammert, tut so, als
ob seine Regeln niemals überholt werden könnten. Sie haben das letzte
Wort. Oder vielleicht das einzige Wort. Primitive Geister gehen stets
davon aus, dass ihre Art, das Leben zu organisieren, die einzige
Möglichkeit ist. Wirtschaftlich komplexere Systeme, die ein Verständnis
für Geschichte miteinbeziehen, stellen sich in der Regel an deren
Spitze. Ob sie chinesische Beamte am Hof des Kaisers, die marxistische
Nomenklatura in Stalins Kreml oder Mitglieder des Repräsentantenhauses
in Washington sind, die Machthaber stellen sich entweder gar keine
Geschichte vor oder setzen sich an die Spitze der Geschichte, in einer
überlegenen Position verglichen mit allen, die vor ihnen kamen, und an
vorderster Front für alles, was noch kommen mag.

Das gilt aus nahezu unvermeidlichen Gründen. Je offensichtlicher es ist,
dass ein System seinem Ende nahe ist, desto mehr zögern die Menschen,
sich an seine Gesetze zu halten. Jede soziale Organisation wird daher
dazu neigen, Analysen, die ihr Ende vorhersehen, zu entmutigen oder
herunterzuspielen. Dies allein hilft sicherzustellen, dass die großen
Übergänge in der Geschichte selten erkannt werden, während sie
stattfinden. Wenn Sie nichts über die Zukunft wissen, können Sie sicher
sein, dass dramatische Veränderungen weder von konventionellen Denkern
begrüßt noch beworben werden.

Sie können sich nicht auf herkömmliche Informationsquellen verlassen, um
Ihnen eine objektive und zeitnahe Warnung darüber zu geben, wie sich die
Welt verändert und warum. Wenn Sie den großen Übergang, der gerade im
Gange ist, verstehen wollen, haben Sie kaum eine andere Wahl, als sich
selbst ein Bild davon zu machen.

\subsection{Jenseits des
Offensichtlichen}\label{jenseits-des-offensichtlichen}

Das bedeutet, über das Offensichtliche hinauszuschauen. Die
Aufzeichnungen zeigen, dass selbst Übergänge, die im Nachhinein
unbestreitbar real sind, möglicherweise erst Jahrzehnte oder sogar
Jahrhunderte nach ihrem Auftreten anerkannt werden. Denken Sie an den
Fall Roms. Es war wahrscheinlich die wichtigste historische Entwicklung
im ersten Jahrtausend der christlichen Ära. Doch noch lange nach Roms
Untergang wurde die Fiktion, dass es überlebt hätte, der Öffentlichkeit
ebenso präsentiert, wie Lenins einbalsamierter Leichnam. Niemand, der
für sein Verständnis der „Nachrichten`` auf die Behauptungen von Beamten
angewiesen war, hätte je erfahren, dass Rom gefallen war, bis lange
nachdem diese Information irrelevant geworden war.

Der Grund für diese Situation war nicht allein die Unzulänglichkeit der
Kommunikation in der alten Welt. Das Ergebnis wäre wahrscheinlich das
gleiche gewesen, hätte CNN auf wundersame Weise bereits existiert und im
September 476 seine Videobänder laufen lassen. Das ist der Zeitpunkt,
als der letzte römische Kaiser im Westen, Romulus Augustulus, in Ravenna
gefangen genommen und zur Zwangspension in eine Villa in Kampania
gebracht wurde. Selbst wenn Wolf Blitzer damals im Jahr 476 mit
Minikameras die Nachrichten aufgezeichnet hätte, ist es
unwahrscheinlich, dass er oder jemand anderes es gewagt hätte, diese
Ereignisse als das Ende des römischen Reiches zu bezeichnen. Das ist
aber natürlich genau das, was spätere Historiker gesagt haben, was
passiert sei.

Vermutlich hätten die CNN-Redakteure eine Schlagzeilengeschichte mit dem
Titel „Rom ist heute Abend gefallen`` nicht genehmigt. Die Machthaber
leugneten, dass Rom gefallen war. Leute, die ihre „Nachrichten``
verhökern, sind selten Partisanen von Kontroversen auf eine Weise, die
ihren eigenen Gewinn untergraben würde. Sie könnten es sein. Vielleicht
sogar mit feurigem Eifer. Aber sie berichten selten über
Schlussfolgerungen, die ihre Abonnenten dazu bringen könnten, ihre
Abonnements zu kündigen und sich zurückzuziehen. Aus diesem Grund hätten
wahrscheinlich nur wenige den Fall Roms gemeldet, selbst wenn es
technologisch möglich gewesen wäre. Experten wären hervorgetreten und
hätten gesagt, dass es lächerlich wäre, von einem Fall Roms zu sprechen.
Etwas anderes zu sagen, wäre schlecht für das Geschäft gewesen und
vielleicht auch schlecht für die Gesundheit derjenigen, die Bericht
erstatteten. Die Machthaber im Rom des späten fünften Jahrhunderts waren
Barbaren, und sie leugneten, dass Rom gefallen war.

Aber es handelte sich nicht bloß um einen Fall, bei dem die Autoritäten
sagten: „Melden Sie das nicht, sonst töten wir Sie.`` Ein Teil des
Problems bestand darin, dass Rom bis in die späten Jahrzehnte des
fünften Jahrhunderts so degeneriert war, dass sein „Fall`` den meisten
Menschen, die es miterlebten, tatsächlich entging. In der Tat war es
eine Generation später, als Graf Marcellinus erstmals andeutete, dass
„das weströmische Reich mit Augustulus unterging.`` \footnote{Ebenda, S.
  102.} Viele weitere Jahrzehnte vergingen, vielleicht Jahrhunderte,
bevor es eine allgemeine Anerkennung gab, dass das römische Reich im
Westen nicht mehr existierte. Mit Sicherheit glaubte Karl der Große,
dass er im Jahr 800 der rechtmäßige römische Kaiser war.

Der Punkt ist nicht, dass Karl der Große und alle, die nach 476 in
konventionellen Begriffen über das Römische Reich dachten, Narren waren.
Ganz im Gegenteil. Die Charakterisierung sozialer Entwicklungen ist oft
mehrdeutig. Wenn die Macht dominanter Institutionen ins Spiel gebracht
wird, um eine günstige Schlussfolgerung zu verstärken, selbst wenn diese
hauptsächlich auf Vortäuschung basiert, würde nur jemand mit starkem
Charakter und starker Meinung es wagen, ihr zu widersprechen. Wenn man
versucht, sich in die Lage eines Römers des späten fünften Jahrhunderts
zu versetzen, ist es leicht vorstellbar, wie verlockend es gewesen wäre,
zu schlussfolgern, dass sich nichts verändert hat. Diese Gewissheit war
der optimistische Schluss. Anders zu denken hätte beängstigend sein
können. Und warum sollte man zu einer beängstigenden Schlussfolgerung
kommen, wenn eine beruhigende zur Verfügung steht?

Immerhin könnte behauptet werden, dass die Geschäfte wie gewohnt
weitergeführt würden. In der Vergangenheit war das der Fall. Die
römische Armee, insbesondere die Grenzgarnisonen, waren seit
Jahrhunderten barbarisiert.\footnote{Siehe S. A. Cook et al., eds.,
  \emph{The Cambridge Ancient History}, Vol. 12 (Cambridge: Cambridge
  University Press, 1971), pp.~208-22.} Im dritten Jahrhundert wurde es
zur regulären Praxis, dass die Armee einen neuen Kaiser ausrief. Im
vierten Jahrhundert waren sogar die Offiziere Germanen und oft
Analphabeten.\footnote{Ebenda, pp.~209-20.} Es hatte viele gewaltsame
Umstürze von Kaisern gegeben, bevor Romulus Augustulus vom Thron
entfernt wurde. Sein Abgang könnte seinen Zeitgenossen nicht anders
erschienen sein als viele andere Umwälzungen in einer chaotischen Zeit.
Und er wurde mit einer Pension abgefertigt. Die Tatsache, dass er eine
Pension erhielt, wenn auch nur für eine kurze Zeit, bevor er ermordet
wurde, war die Beruhigung, dass das System überlebt hatte. Optimistisch
betrachtet hat Odoaker, der Romulus Augustulus absetzte, das Reich eher
vereint als zerstört. Odoaker, der Sohn von Attilas Gehilfen Edekon, war
ein kluger Mann. Er rief sich nicht selbst zum Kaiser aus. Stattdessen
versammelte er den Senat und überredete seine allzu beeinflussbaren
Mitglieder dazu, die Kaiserwürde und somit die Souveränität über das
gesamte Reich Zeno, dem oströmischen Kaiser im weit entfernten Byzanz
anzubieten. Odoaker sollte lediglich Zenos Patricius sein, um Italien zu
regieren.

Wie Will Durant in \emph{The Story of Civilization} schrieb, erschienen
diese Veränderungen nicht als der „Fall Roms``, sondern lediglich als
„vernachlässigbare Verschiebungen an der Oberfläche des nationalen
Panoramas.`` \footnote{Will Durant, \emph{The Story of Civilization},
  Vol. 4, \emph{The Age of Faith} (New York: Simon \& Schuster, 1950),
  S. 43.} Als Rom fiel, behauptete Odoaker, dass Rom weiter bestand. Er,
wie fast jeder andere auch, war begierig darauf so zu tun, als ob sich
nichts verändert hätte. Sie wussten, dass „die Herrlichkeit Roms``,
weitaus besser war als die Barbarei, die ihren Platz einnahm. Selbst die
Barbaren dachten so. Wie C. W. Previte-Orton in \emph{The Shorter
Cambridge Medieval History} schrieb, war das Ende des fünften
Jahrhunderts, als „die Kaiser durch barbarische deutsche Könige ersetzt
worden waren``, eine Zeit der „hartnäckigen Fantasterei.`` \footnote{C.
  W. Previte-Orton, \emph{The Shorter Cambridge Medieval History}, Vol.
  1 (Cambridge: Cambridge University Press, 1971), S. 102.}

\subsection{„Hartnäckige
Fantasterei``}\label{hartnuxe4ckige-fantasterei}

Diese „Fantasterei`` beinhaltete die Bewahrung der Fassade des alten
Systems, selbst als seine Essenz „durch Barbarei verunstaltet`` worden
war.\footnote{Ebenda, S. 131.} Die alten Regierungsformen blieben die
gleichen, als der letzte Kaiser durch einen barbarischen
„Stellvertreter`` ersetzt wurde. Der Senat tagte immer noch. „Die
prätorianische Präfektur und andere hohe Ämter blieben bestehen und
wurden von prominenten Römern besetzt.`` \footnote{Ebenda, S. 137.}
Konsuln wurden wie üblich für ein Jahr nominiert. „Die römische
Zivilverwaltung überlebte unversehrt.`` \footnote{Ebenda.} Tatsächlich
blieb sie in einigen Aspekten bis zur Geburt des Feudalismus am Ende des
zehnten Jahrhunderts intakt. Bei öffentlichen Anlässen wurden weiterhin
alte kaiserliche Insignien verwendet. Das Christentum war nach wie vor
die Staatsreligion. Die Barbaren gaben vor, immer noch dem östlichen
Kaiser in Konstantinopel und den Traditionen des römischen Rechts die
Treue zu schulden. Wie Durant es ausdrückte „war das große Imperium im
Westen nicht mehr existent``.\footnote{Durant, ebenda, S. 43.}

\subsection{Was nun?}\label{was-nun}

Das weit entfernte Beispiel des Falls von Rom ist aus verschiedenen
Gründen relevant, wenn man die Bedingungen der heutigen Welt betrachtet.
Die meisten Bücher über die Zukunft sind eigentlich Bücher über die
Gegenwart. Wir haben versucht, diesen Mangel zu beheben, indem wir
dieses Buch in erster Linie als ein Buch über die Vergangenheit
gestalten. Wir glauben, dass Sie wahrscheinlich eine bessere Perspektive
darauf erhalten, was die Zukunft bereithält, wenn wir wichtige
megapolitische Argumente zur Logik der Gewalt mit realen Beispielen aus
der Vergangenheit illustrieren. Geschichte ist eine erstaunliche
Lehrerin. Die Erzählungen sind interessanter als wir sie erfinden
könnten. Und viele der interessantesten beziehen sich auf den Fall von
Rom. Sie dokumentieren wichtige Lektionen, die für Ihre Zukunft im
Informationszeitalter relevant sein könnten.

Zunächst einmal ist der Fall Roms eines der lebendigsten Beispiele in
der Geschichte dafür, was in einem größeren Übergang passierte, als der
Umfang der Regierung zusammenbrach. Die Übergänge des Jahres 1000
beinhalteten ebenfalls den Zusammenbruch der zentralen Autorität und
taten dies in einer Weise, die die Komplexität und den Umfang der
wirtschaftlichen Aktivität erhöhte. Die Schießpulverrevolution am Ende
des fünfzehnten Jahrhunderts führte zu großen Veränderungen in den
Institutionen, die dazu neigten, den Umfang der Regierung eher zu
erhöhen als zu verringern. Heute, zum ersten Mal seit tausend Jahren,
untergraben und zerstören megapolitische Bedingungen im Westen
Regierungen und viele andere Institutionen, die in großem Umfang
agieren.

Natürlich hatte der Zusammenbruch der Regierungsgröße am Ende des
römischen Reiches ganz andere Ursachen als jene, die zum Beginn des
Informationszeitalters existierten. Ein Teil der Ursache, warum Rom
fiel, liegt einfach darin, dass es über das Maß hinausgewachsen war, in
dem die Ökonomien der Gewalt aufrechterhalten werden konnten. Die Kosten
für die Besatzung der weit entfernten Grenzen des Reiches überstiegen
die wirtschaftlichen Vorteile, die eine antike Agrarwirtschaft
unterstützen konnte. Die Last von Steuern und Regulierung, die zur
Finanzierung des militärischen Aufwands benötigt wurde, stieg über die
Tragfähigkeit der Wirtschaft hinaus. Korruption wurde zur Normalität.
Wie der Historiker Ramsay MacMullen dokumentiert hat, widmeten sich die
Militärkommandanten stark der Verfolgung „illegaler Gewinne aus ihrem
Kommando``.\footnote{Ramsay MacMullen, \emph{Corruption and the Decline
  of Rome} (New Haven: Yale University Press, 1988), S. 192.} Dies
erreichten sie, indem sie die Bevölkerung ausbeuteten, was der
Beobachter des vierten Jahrhunderts, Synesius, als „den Krieg in
Friedenszeiten, einen fast schlimmeren als den Barbarenkrieg und
resultierend aus der Disziplinlosigkeit des Militärs und der Gier der
Offiziere`` beschrieb.\footnote{Zitiert ebenda, S. 193.}

Ein weiterer wichtiger Faktor, der zum Zusammenbruch Roms beitrug, war
das demographische Defizit, das durch die Antoninische Pest verursacht
wurde. Der Rückgang der römischen Bevölkerung in vielen Gebieten trug
offensichtlich zur wirtschaftlichen und militärischen Schwäche bei.
Nichts dergleichen ist heute passiert, zumindest noch nicht. Aus einer
längerfristigen Perspektive heraus betrachtet, könnte die Geißel neuer
„Pandemien`` die Herausforderungen der technologischen Entwicklung im
neuen Jahrtausend verschärfen. Die beispiellose Zunahme der menschlichen
Bevölkerung im zwanzigsten Jahrhundert schafft ein verlockendes Ziel für
schnell mutierende Mikroparasiten. Ängste vor dem Ebola-Virus oder
ähnlichen Krankheiten, die städtische Bevölkerungsgruppen befallen
könnten, sind möglicherweise begründet. Doch dies ist nicht der Ort, um
die Koevolution von Menschen und Krankheiten zu diskutieren. So
interessant dieses Thema auch ist, geht es uns an dieser Stelle nicht
darum, warum Rom fiel, oder ob die Welt heute einigen der gleichen
Einflüsse ausgesetzt ist, die zum Niedergang Roms beigetragen haben. Es
geht um etwas anderes -- nämlich die Art und Weise, wie die großen
Veränderungen in der Geschichte wahrgenommen, oder vielmehr, während sie
geschehen, missverstanden werden.

Die Menschen sind stets und überall zu einem gewissen Grad konservativ.
Dies impliziert eine Zurückhaltung, in Bezug auf die Auflösung von
geschätzten gesellschaftlichen Konventionen, den Umsturz der anerkannten
Institutionen und dem Trotzen gegenüber Gesetzen und Werten, die ihnen
als Orientierung dienen. Nur wenige sind bereit zu erwägen, dass
scheinbar geringfügige Veränderungen im Klima oder der Technologie oder
in einer anderen Variablen irgendwie verantwortlich sein könnten, die
Verbindungen zur Welt ihrer Väter zu kappen. Die Römer zögerten, die
sich um sie herum entfaltenden Veränderungen anzuerkennen. Und so geht
es uns auch.

Ob wir es anerkennen oder nicht, wir durchleben einen Wechsel der
historischen Jahreszeit, eine Transformation in der Art und Weise, wie
die Menschen ihren Lebensunterhalt organisieren und sich verteidigen,
die so tiefgreifend ist, dass sie unweigerlich die gesamte Gesellschaft
verändern wird. Der Wandel wird so tiefgreifend sein, dass man, um ihn
zu verstehen, fast nichts als selbstverständlich ansehen muss. Ihnen
wird nahezu bei jeder Gelegenheit suggeriert, dass die kommenden
Informationsgesellschaften der Industriegesellschaft, in der Sie
aufgewachsen sind, sehr ähnlich sein werden. Wir bezweifeln das. Die
Mikroverarbeitung wird den Mörtel in den Steinen auflösen. Sie wird die
Logik der Gewalt so tiefgreifend verändern, dass sie dramatisch die Art
und Weise verändert, wie Menschen ihren Lebensunterhalt organisieren und
sich verteidigen. Dennoch wird die Tendenz bestehen, die
Unvermeidlichkeit dieser Veränderungen herunterzuspielen, oder ihre
Erwünschtheit zu diskutieren, als ob es in der Hand der industriellen
Institutionen läge, die Entwicklung der Geschichte zu bestimmen.

\subsection{Die große Illusion}\label{die-grouxdfe-illusion}

Autoren, die in vielerlei Hinsicht besser informiert sind als wir,
können uns dennoch in Bezug auf die Zukunft in die Irre führen, weil sie
die Funktionsweise von Gesellschaften viel zu oberflächlich untersuchen.
Zum Beispiel haben David Kline und Daniel Burstein ein gut
recherchiertes Werk mit dem Titel „Road Warriors: Dreams and Nightmares
Along the Information Highway`` verfasst. Es steckt voller
bewundernswerter Details, doch viele dieser Details werden dazu
verwendet, eine Illusion zu unterstützen, nämlich die Idee, „dass Bürger
gemeinsam und bewusst die spontanen wirtschaftlichen und natürlichen
Prozesse, die um sie herum stattfinden, gestalten können.`` \footnote{Zitiert
  aus David Kline und Daniel Burstein, \emph{Is Government Obsolete?},
  Wired, Januar 1996, S. 105.} Obwohl es möglicherweise nicht
offensichtlich ist, entspricht dies der Aussage, dass der Feudalismus
hätte überleben können, wenn sich jeder erneut der Ritterlichkeit
verschrieben hätte. Niemand in einem Gericht des späten fünfzehnten
Jahrhunderts hätte eine solche Ansicht abgelehnt. Tatsächlich wäre es
Ketzerei gewesen, dies zu tun. Aber es wäre auch völlig irreführend
gewesen. Wie eine Schlange, die versucht, ihre Zukunft in ihrer alten
Haut zu finden.

Die grundlegenden Ursachen für Veränderungen sind genau diejenigen, die
nicht bewusst kontrolliert werden können. Sie sind die Faktoren, die die
Bedingungen verändern, unter denen Gewalt Vorteile bringt. Tatsächlich
sind sie so fern von jeglichen offensichtlichen Mitteln zur
Manipulation, dass sie nicht einmal Gegenstand politischer Manöver sind,
in einer Welt, die von Politik durchdrungen ist. Niemand hat jemals auf
einer Demonstration gerufen: „Erhöhen Sie die Skaleneffekte im
Produktionsprozess``. Kein Banner hat jemals die Forderung erhoben:
„Erfinden Sie ein Waffensystem, das die Bedeutung der Infanterie
erhöht``. Kein politischer Kandidat hat je versprochen „das
Gleichgewicht zwischen der Effizienz und dem Ausmaß von Schutz der
Gewalt zu verändern``. Solche Parolen wären lächerlich, gerade weil
diese Ziele jenseits der Fähigkeit des Einzelnen liegen, sie bewusst zu
beeinflussen. Dennoch bestimmen diese Variablen, wie wir erkennen
werden, in weit größerem Maße wie die Welt funktioniert, als jede
politische Plattform.

Wenn man ernsthaft darüber nachdenkt, sollte es offensichtlich sein,
dass wichtige Übergänge in der Geschichte selten hauptsächlich von
menschlichen Wünschen angetrieben werden. Sie geschehen nicht, weil die
Menschen von einer Lebensweise genug haben und plötzlich eine andere
bevorzugen. Eine kurze Überlegung zeigt warum. Wenn das, was Menschen
denken und wünschen, die einzigen Determinanten dessen wären, was
passiert, dann müssten alle abrupten Veränderungen in der Geschichte
durch wilde Stimmungsschwankungen erklärt werden, die in keiner
Verbindung zu irgendwelchen Veränderungen in den tatsächlichen
Lebensbedingungen stehen. Tatsächlich geschieht dies jedoch nie. Nur in
Fällen von medizinischen Problemen, die nur wenige Menschen betreffen,
sehen wir willkürliche Stimmungsschwankungen, die scheinbar vollständig
von jeder objektiven Ursache getrennt sind.

Normalerweise entscheiden große Menschengruppen sich nicht plötzlich
dafür, ihren Lebensstil aufzugeben, nur weil sie es amüsant finden. Kein
Jäger und Sammler hat jemals gesagt: „Ich bin es leid, in
prähistorischen Zeiten zu leben, ich würde das Leben eines Bauern in
einem Bauerndorf vorziehen.`` Jeder entscheidende Umschwung in
Verhaltensmustern und Werten ist unweigerlich eine Reaktion auf eine
tatsächliche Veränderung der Lebensbedingungen. In diesem Sinne sind die
Menschen zumindest immer realistisch. Wenn sich ihre Ansichten plötzlich
ändern, deutet das wahrscheinlich darauf hin, dass sie mit einer
Abweichung von den bekannten Bedingungen konfrontiert wurden: Einer
Invasion, einer Pandemie, einer plötzlichen klimatischen Veränderung
oder einer technologischen Revolution, die ihre Lebensgrundlage oder
ihre Fähigkeit zur Selbstverteidigung verändert hat.

Entgegen menschlicher Wunschvorstellungen widersprechen entscheidende
historische Veränderungen häufig der Sehnsucht der meisten Menschen nach
Stabilität. Wenn Veränderungen eintreten, verursachen sie typischerweise
eine weit verbreitete Desorientierung, insbesondere bei denjenigen, die
Einkommen oder sozialen Status verlieren. Sie werden vergeblich in
Meinungsumfragen oder anderen Stimmungsmessungen nach einem Verständnis
dafür suchen, wie der bevorstehende megapolitische Übergang sich
wahrscheinlich entwickeln wird.

\section{LEBEN OHNE WEITSICHT}\label{leben-ohne-weitsicht}

Wenn wir die große Übergangsphase, die um uns herum stattfindet, nicht
wahrnehmen, liegt das teilweise daran, dass wir sie nicht sehen wollen.
Unsere nomadischen Vorfahren hätten vielleicht genauso starrsinnig sein
können, aber sie hatten eine bessere Ausrede. Vor zehntausend Jahren
hätte niemand die Folgen der Agrarrevolution voraussehen können.
Tatsächlich hätte niemand etwas über das Finden der nächsten Mahlzeit
hinaus erahnen können. Als die Landwirtschaft begann, gab es keine
Aufzeichnungen vergangener Ereignisse, aus denen man Perspektiven für
die Zukunft hätte gewinnen können. Es gab nicht einmal ein westliches
Verständnis von Zeit, das in geordnete Einheiten wie Sekunden, Minuten,
Stunden, Tage usw. unterteilt gewesen wäre, um die Jahre zu messen. Die
Nomaden lebten in der „ewigen Gegenwart``, ohne Kalender und ohne
jegliche schriftliche Aufzeichnungen. Sie hatten keine Wissenschaft und
kein anderes intellektuelles Werkzeug, um Ursache und Wirkung über ihre
eigenen Intuitionen hinaus zu verstehen. Was die Zukunftsplanung
betrifft, waren unsere urzeitlichen Vorfahren blind. Um es mit einer
biblischen Metapher auszudrücken: Sie hatten noch nicht von der Frucht
der Erkenntnis gegessen.

\subsection{Aus der Vergangenheit
lernen}\label{aus-der-vergangenheit-lernen}

Glücklicherweise haben wir einen besseren Aussichtspunkt. Die
vergangenen fünfhundert Generationen haben uns analytische Fähigkeiten
verliehen, die unseren Vorfahren fehlten. Wissenschaft und Mathematik
haben uns dabei geholfen, viele Geheimnisse der Natur zu entschlüsseln
und uns ein Verständnis von Ursache und Wirkung gegeben, das im
Vergleich zu dem der frühen Sammler und Jäger geradezu magisch anmutet.
Berechnungsalgorithmen, die durch Hochgeschwindigkeitscomputer
entwickelt wurden, haben neue Einblicke in die Funktionsweise komplexer,
dynamischer Systeme wie der menschlichen Wirtschaft ermöglicht. Die
mühsame Entwicklung der politischen Ökonomie selbst -- obwohl sie weit
davon entfernt ist, perfekt zu sein -- hat das Verständnis der Faktoren,
die menschliches Handeln beeinflussen, geschärft. Insbesondere ist die
Erkenntnis wichtig, dass Menschen zu allen Zeiten und an allen Orten
dazu neigen, auf Anreize zu reagieren. Nicht immer so mechanisch, wie
Ökonomen es sich vorstellen, aber sie reagieren. Kosten und Belohnungen
sind relevant. Veränderungen der äußeren Bedingungen, die die
Belohnungen erhöhen oder die Kosten für bestimmtes Verhalten senken,
führen vermehrt zu diesem Verhalten, sofern alle anderen Bedingungen
gleichbleiben.

\subsection{Anreize sind wichtig}\label{anreize-sind-wichtig}

Die Tatsache, dass Menschen dazu neigen, auf Kosten und Belohnungen zu
reagieren, ist ein wesentliches Element der Vorhersage. Man kann mit
hoher Sicherheit sagen, dass wenn man einen Hundert-Dollar-Schein auf
der Straße fallen lässt, ihn jemand bald aufheben wird, egal ob man in
New York, Mexico City, oder Moskau ist. Das ist nicht so trivial, wie es
scheint. Es zeigt, warum die klugen Leute, die sagen, dass Vorhersagen
unmöglich sind, unrecht haben. Jede Prognose, die die Auswirkungen von
Anreizen auf das Verhalten genau antizipiert, wird wahrscheinlich
generell korrekt sein. Und je größer die erwartete Veränderung in Bezug
auf Kosten und Belohnungen ist, desto weniger trivial ist wahrscheinlich
die implizierte Prognose.

Die weitreichendsten Prognosen aller Zeiten werden wahrscheinlich aus
der Erkenntnis der Implikationen sich verändernder megapolitischer
Variablen hervorgehen. Gewalt ist die ultimative Grenze für das
Verhalten; wenn Sie also verstehen können, wie sich die Logik der Gewalt
verändern wird, können Sie auf nützliche Weise vorhersagen, wo Menschen
in der Zukunft das Äquivalent von Hundert-Dollar-Scheinen fallen lassen
oder aufheben werden.

Wir meinen damit nicht, dass Sie sich über die Unsicherheit sicher sein
können. Wir können Ihnen nicht sagen, wie Sie die Gewinnzahlen der
Lotterie oder irgendein wirklich zufälliges Ereignis vorhersagen können.
Wir haben keine Möglichkeit zu wissen, wann oder ob ein Terrorist eine
Atombombe in Manhattan zünden wird. Oder ob ein Asteroid Saudi-Arabien
treffen wird. Wir können das Kommen einer neuen Eiszeit, einen
plötzlichen Vulkanausbruch oder das Auftreten einer neuen Krankheit
nicht vorhersagen. Die Anzahl unbekannter Ereignisse, die den Lauf der
Geschichte verändern könnten, ist groß. Aber Sicherheit über die
Unsicherheit ist etwas ganz anderes als die Auswirkungen dessen
herauszuarbeiten, was bereits bekannt ist. Wenn Sie in weiter Entfernung
einen Blitz sehen, können Sie mit hoher Gewissheit vorhersagen, dass der
Donner kommen wird. Die Vorhersage der Folgen von megapolitischen
Übergängen beinhaltet viel längere Zeitrahmen und weniger sichere
Zusammenhänge, aber es handelt sich um eine ähnliche Art der Übung.

Megapolitische Katalysatoren für Veränderungen erscheinen in der Regel
lange bevor sich ihre Konsequenzen manifestieren. Es dauerte fünftausend
Jahre, bis die vollen Auswirkungen der Agrarrevolution zum Vorschein
kamen. Der Übergang von einer Agrargesellschaft zu einer
Industriegesellschaft basierend auf Produktion und chemischer Energie
entfaltete sich schneller. Es dauerte Jahrhunderte. Der Übergang zur
Informationsgesellschaft wird noch schneller stattfinden, wahrscheinlich
innerhalb einer Lebensspanne. Dennoch, selbst wenn man die Verkürzung
der Geschichte berücksichtigt, kann man erwarten, dass Jahrzehnte
vergehen, bevor die volle megapolitische Auswirkung der bestehenden
Informationstechnologie realisiert wird.

\subsection{Große und kleine megapolitische
Übergänge}\label{grouxdfe-und-kleine-megapolitische-uxfcberguxe4nge}

Dieses Kapitel analysiert einige der gemeinsamen Merkmale von
megapolitischen Übergängen. In den folgenden Kapiteln betrachten wir die
Agrarrevolution und den Übergang von der Landwirtschaft zur
Fabrikarbeit, die zweite der vorherigen großen Phasenwechsel, genauer.
Innerhalb der landwirtschaftlichen Phase der Zivilisation gab es viele
kleinere megapolitische Übergänge, wie zum Beispiel den Fall Roms und
die feudale Revolution des Jahres 1000. Diese markierten das Aufkommen
und den Niedergang der Machtverhältnisse, während Regierungen aufstiegen
und fielen und die Erlöse aus der Landwirtschaft von einer Gruppe zur
anderen übergingen. Die Besitzer von großflächigen Anwesen im römischen
Reich, Kleinbauern im europäischen Mittelalter und die Herren und
Leibeigenen der Feudalzeit ernährten sich alle von den gleichen Feldern.
Sie lebten unter sehr unterschiedlichen Regierungen aufgrund der
kumulativen Auswirkungen verschiedener Technologien, Klimaschwankungen
und der disruptiven Einflüsse von Krankheiten.

Unser Ziel ist es nicht, all diese Veränderungen gründlich zu erklären.
Wir geben nicht vor, dies zu tun, obwohl wir einige Beispiele skizziert
haben, wie sich verändernde megapolitische Variablen die Art und Weise
verändert haben, wie Macht in der Vergangenheit ausgeübt wurde.
Regierungen sind gewachsen und geschrumpft, da megapolitische
Schwankungen die Kosten für die Ausübung von Macht gesenkt oder erhöht
haben.

Hier sind einige zusammenfassende Punkte, die Sie im Hinterkopf behalten
sollten, wenn Sie versuchen, die Informationsrevolution zu verstehen:

\begin{enumerate}
\def\labelenumi{\arabic{enumi}.}
\item
  Eine Verschiebung der megapolitischen Grundlagen der Macht vollzieht
  sich in der Regel weit vor den eigentlichen Revolutionen in der
  Anwendung der Macht.
\item
  Die Einkommen sinken in der Regel, wenn ein größerer Übergang beginnt,
  oft weil eine Gesellschaft sich selbst krisenanfällig gemacht hat,
  indem sie die Ressourcen aufgrund des Bevölkerungsdrucks
  marginalisiert hat.
\item
  Der Blick „außerhalb`` eines Systems ist in der Regel tabu. Die
  Menschen sind häufig blind für die Logik der Gewalt in der bestehenden
  Gesellschaft; daher sind sie auch fast immer blind für Veränderungen
  innerhalb dieser Logik, ob versteckt oder offensichtlich.
  Megapolitische Übergänge werden selten erkannt, bevor sie stattfinden.
\item
  Große Übergänge sind immer mit einer kulturellen Revolution verbunden
  und führen in der Regel zu Konflikten zwischen den Anhängern der alten
  und der neuen Werte.
\item
  Megapolitische Übergänge sind nie populär, weil sie mühsam erworbenes
  intellektuelles Kapital antiquieren und etablierte moralische
  Imperative in Frage stellen. Sie werden nicht auf allgemeine Nachfrage
  hin vollzogen, sondern als Reaktion auf veränderte externe
  Bedingungen, die die Logik der Gewalt im lokalen Umfeld verändern.
\item
  Übergänge zu neuen Formen der Organisation des Lebensunterhalts oder
  zu neuen Regierungsformen sind zunächst auf die Gebiete beschränkt, in
  denen die megapolitischen Katalysatoren am Werk sind.
\item
  Mit der möglichen Ausnahme der Anfangsphase der Landwirtschaft waren
  Übergänge in der Vergangenheit immer mit Perioden des sozialen Chaos
  und erhöhter Gewalt aufgrund von Desorientierung und Zusammenbruch des
  alten Systems verbunden.
\item
  Korruption, moralischer Verfall und Ineffizienz scheinen Merkmale des
  Endstadiums eines Systems zu sein.
\item
  Die wachsende Bedeutung der Technologie bei der Gestaltung der Logik
  der Gewalt hat zu einer Beschleunigung der Geschichte geführt, so dass
  bei jedem aufeinanderfolgenden Übergang weniger Zeit zur Anpassung
  bleibt als je zuvor.
\end{enumerate}

\subsection{Die Geschichte beschleunigt
sich}\label{die-geschichte-beschleunigt-sich}

Da sich die Ereignisse heute oftmals schneller entwickeln als während
früherer Transformationen, könnte ein frühzeitiges Verständnis davon,
wie sich die Welt verändern wird, weitaus nützlicher für Sie sein als es
für Ihre Vorfahren zu einem vergleichbaren Zeitpunkt in der
Vergangenheit gewesen wäre. Selbst wenn die ersten Bauern auf wundersame
Weise die vollen megapolitischen Implikationen des Ackerbaus verstanden
hätten, wären diese Informationen praktisch nutzlos gewesen, da tausende
von Jahren vergehen sollten, bis der Übergang in die nächste Phase der
Gesellschaft komplett vollzogen war.

Das ist heutzutage nicht mehr so. Die Geschichte hat sich beschleunigt.
Prognosen, die die megapolitischen Auswirkungen neuer Technologien
korrekt vorhersagen, dürften heute weitaus nützlicher sein. Wenn wir die
Implikationen des aktuellen Übergangs zur Informationsgesellschaft
genauso detailliert erfassen können, wie jemand mit aktuellem Wissen die
Implikationen vergangener Übergänge zur Landwirtschaft und zur
Fabrikarbeit hätte verstehen können, sollte diese Information heute ein
Vielfaches wert sein. Einfach ausgedrückt: der Handlungshorizont für
megapolitische Prognosen hat sich auf seinen nützlichsten Bereich
innerhalb einer einzelnen Lebensspanne reduziert.

\begin{quote}
„Blickt man über die Jahrhunderte zurück oder betrachtet man sogar nur
die Gegenwart, so lässt sich deutlich beobachten, dass viele Männer
ihren Lebensunterhalt, oft einen sehr guten Lebensunterhalt, aus ihrer
besonderen Fähigkeit gezogen haben, die Waffen der Gewalt anzuwenden,
und dass ihre Aktivitäten einen sehr großen Teil dazu beigetragen haben,
welche Verwendung knappen Ressourcen zugefügt wurde.`` - Frederic C.
Lane\footnote{Lane, \emph{Economic Consequences of Organized Violence},
  ebenda.}
\end{quote}

Unsere Studie über Megapolitik versucht genau dies zu tun -- die
Konsequenzen der sich ändernden Faktoren, welche die Grenzen, innerhalb
derer Gewalt ausgeübt wird, verändern, herauszuarbeiten.

Diese megapolitischen Faktoren bestimmen im Wesentlichen, wann und wo
Gewalt sich auszahlt. Außerdem helfen sie bei der
Informationsbeschaffung einer marktgerechten Einkommensverteilung. Wie
der Wirtschaftshistoriker Frederic Lane so gekonnt formulierte, spielt
die Art und Weise, wie Gewalt organisiert und kontrolliert wird, eine
große Rolle bei der Bestimmung „welche Verwendungen knappen Ressourcen
zugefügt wurde.`` \footnote{Ebenda.}

\section{EIN CRASHKURS IN
MEGAPOLITIK}\label{ein-crashkurs-in-megapolitik}

Das Konzept der Megapolitik ist mächtig. Es hilft dabei, einige der
größten Geheimnisse der Geschichte zu beleuchten: wie Regierungen
aufsteigen und fallen und welche Art von Institutionen sie werden; das
Timing und Ergebnis von Kriegen; Muster wirtschaftlichen Wohlstands und
Verfalls. Durch die Erhöhung oder Senkung von Kosten und Belohnungen für
die Projektion von Macht, beherrscht die Megapolitik die Fähigkeit der
Menschen, ihren Willen anderen aufzuzwingen. Dies war seit den frühesten
menschlichen Gesellschaften der Fall. Und das ist auch immer noch so.
Wir haben viele der wichtigen, versteckten megapolitischen Faktoren
erforscht, die die Entwicklung der Geschichte in \emph{Blood in the
Streets} und \emph{The Great Reckoning} bestimmen. Der Schlüssel zur
Enthüllung der Auswirkungen megapolitischer Veränderungen liegt im
Verständnis der Faktoren, die Revolutionen in der Anwendung von Gewalt
auslösen. Diese Variablen können in vier Kategorien eingeteilt werden:
Topographie, Klima, Mikroben und Technologie.

1. Topographie ist ein entscheidender Faktor, wie die Tatsache beweist,
dass die Kontrolle von Gewalt auf offener See nie so monopolisiert war
wie an Land. Die Gesetze einer Regierung haben dort nie exklusiv
gegolten. Dies ist von größter Bedeutung, um zu verstehen, wie sich die
Organisation von Gewalt und Schutz entwickeln wird, wenn die Wirtschaft
in den Cyberspace abwandert.

Topographie spielte zusammen mit dem Klima eine bedeutende Rolle in der
frühen Geschichte. Die ersten Staaten entstanden auf
Überschwemmungsgebieten die von Wüste umgeben waren, wie beispielsweise
in Mesopotamien und Ägypten, wo Wasser zur Bewässerung reichlich
vorhanden war, während die umliegenden Regionen zu trocken waren, um
kleinbäuerliche Landwirtschaft zu unterstützen. Unter solchen
Bedingungen standen einzelne Landwirte vor extrem hohen Kosten, wenn sie
nicht an der Erhaltung der politischen Struktur mitwirkten. Ohne
Bewässerung, die nur im großen Maßstab gewährleistet werden konnte,
würden die Pflanzen nicht wachsen. Keine Ernte bedeutete Hunger. Die
Bedingungen, die diejenigen, die das Wasser in einer Wüste
kontrollierten, in eine Position der Stärke brachten, führten zu einer
despotischen und reichen Regierung.

Wie wir in \emph{The Great Reckoning} analysiert haben, spielten auch
topographische Bedingungen eine entscheidende Rolle für den Wohlstand
der Landwirte im alten Griechenland und ermöglichten dieser Region, die
Wiege der westlichen Demokratie zu werden. Angesichts der primitiven
Transportbedingungen, die vor dreitausend Jahren im Mittelmeerraum
herrschten, war es nahezu unmöglich, für Personen, die mehr als ein paar
Kilometer vom Meer entfernten lebten, mit der Produktion der
reichhaltigen Ernte von Oliven und Trauben in der antiken Welt zu
konkurrieren. Wenn Öl und Wein über gewisse Entfernung an Land
transportiert werden mussten, waren die Transportkosten so hoch, dass
sie nicht mit Gewinn verkauft werden konnten. Die ausgedehnten Linien
der griechischen Küste bedeuteten, dass die meisten Gebiete
Griechenlands nicht mehr als dreißig Kilometer vom Meer entfernt waren.
Dies verschaffte den griechischen Bauern einen entscheidenden Vorteil
gegenüber ihren potenziellen Wettbewerbern in Binnenländern.

Aufgrund dieses Handelsvorteils mit hochwertigen Produkten erzielten die
griechischen Bauern hohe Einkommen, obwohl sie nur kleine Landparzellen
besaßen. Diese hohen Einkommen ermöglichten es ihnen, kostspielige
Rüstungen zu erwerben. Die berühmten Hopliten des antiken Griechenlands
waren Bauern oder Landbesitzer, die sich auf eigene Kosten bewaffneten.
Die gut bewaffneten und hochmotivierten griechischen Hopliten stellten
eine ernstzunehmende militärische Macht dar und konnten nicht ignoriert
werden. Die topografischen Bedingungen bildeten das Fundament der
griechischen Demokratie, genauso wie andere Gegebenheiten zur Entstehung
der orientalischen Despotien in Ägypten und anderen Orten führten.

2. Klima trägt ebenfalls dazu bei, die Grenzen zu setzen, innerhalb
derer rohe Gewalt ausgeübt werden kann. Ein Klimawandel war der Auslöser
für den ersten großen Übergang von der Nahrungssuche zum Ackerbau.

Das Ende der letzten Eiszeit, vor ungefähr dreizehntausend Jahren,
führte zu einer radikalen Veränderung der Vegetation. Beginnend im nahen
Osten, wo die Eiszeit zuerst zurückging, führte ein allmählicher Anstieg
von Temperatur und Niederschlägen zur Ausbreitung von Wäldern in
Gebieten, die zuvor Graslandschaften waren. Insbesondere die rasche
Ausbreitung von Buchenwäldern begrenzte die menschliche Ernährung
erheblich. Wie Susan Alling Gregg es in \emph{Foragers and Farmers}
ausdrückte:

„Die Ansammlung von Buchenwäldern muss schwerwiegende Folgen für die
lokalen menschlichen, pflanzlichen und tierischen Populationen gehabt
haben. Das Blätterdach eines Eichenwaldes ist relativ offen und lässt
viel Sonnenlicht auf den Waldboden fallen. Ein üppiger Unterwuchs aus
gemischten Sträuchern, Kräutern und Gräsern entwickelt sich und die
Vielfalt der Pflanzen unterstützt eine Vielzahl von Wildtieren. Im
Gegensatz dazu ist das Blätterdach eines Buchenwaldes geschlossen und
der Waldboden ist stark beschattet. Abgesehen von einem Schub von
Frühjahrsblühern vor dem Austreten der Blätter, findet man nur
schattentolerante Seggen, Farne und einige Gräser.`` \footnote{Susan
  AIling Gregg, \emph{Foragers and Farmers: Population Interaction and
  Agricultural Expansion in Prehistoric Europe} (Chicago: University of
  Chicago Press, 1988), S. 9.}

Mit der Zeit breiteten sich dichte Wälder auf den offenen Ebenen aus und
zogen sich durch ganz Europa bis in die östlichen Steppen.\footnote{Stephen
  Boyden, \emph{Western Civilization in Biological Perspective} (Oxford:
  Clarendon Press, 1987), S. 89. Siehe ebenfalls Marvin Harris,
  \emph{Cannibals and Kings} (New York: Vintage, 1978), pp.~29-32.} Die
Wälder verringerten die Weidefläche, die zur Unterstützung großer Tiere
zur Verfügung stand, was es für die Bevölkerung der menschlichen Sammler
zunehmend schwierig machte, sich zu ernähren.

Die Population der Jäger und Sammler war während der Wohlstandszeit der
Eiszeit so stark angewachsen, dass sie sich nicht mehr durch die
schrumpfenden Herden großer Säugetiere, von denen viele Arten bis zur
Ausrottung gejagt wurden, ernähren konnte. Der Übergang zur
Landwirtschaft war kein Wunsch, sondern eine unter Notwendigkeit
angenommene Improvisation, um Engpässe in der Ernährung auszugleichen.
Die Nahrungssuche blieb weiterhin in den weiter nördlich gelegenen
Gebieten vorherrschend, wo der Erwärmungstrend den Lebensraum großer
Säugetiere nicht beeinträchtigte, sowie in tropischen Regenwäldern, in
denen der globale Erwärmungstrend nicht die perverse Auswirkung hatte,
dass die Nahrungsversorgung verringert wurde. Seit dem Aufkommen der
Landwirtschaft ist es weit häufiger, dass Veränderungen durch die
Abkühlung anstatt durch die Erwärmung des Klimas verursacht werden.

Ein bescheidenes Verständnis der Dynamik des Klimawandels in vergangenen
Gesellschaften könnte sich als nützlich erweisen, falls das Klima
weiterhin Schwankungen unterliegt. Wenn man weiß, dass ein Rückgang um
ein Grad Celsius im Durchschnitt die Wachstumsperiode um drei bis vier
Wochen verkürzt und die maximale Anbauhöhe um 150 Meter reduziert, dann
hat man eine Vorstellung von den Randbedingungen, die das Handeln der
Menschen in der Zukunft einschränken werden.\footnote{Geoffrey Parker
  und Lesley Ni. Smith, eds., \emph{The General Crisis of the
  Seventeenth Century} (London: Routledge \& Kegan Paul, 1985), S. 8.}
Mit diesem Wissen können Sie Veränderungen von Getreidepreisen bis hin
zum Wert von Landflächen prognostizieren. Sie könnten sogar in der Lage
sein, informierte Schlussfolgerungen über den wahrscheinlichen Einfluss
fallender Temperaturen auf die realen Einkommen und die politische
Stabilität zu ziehen. In der Vergangenheit wurden Regierungen gestürzt,
wenn mehrjährige Ernteausfälle die Lebensmittelpreise in die Höhe
trieben und das verfügbare Einkommen schrumpfen ließen.

Es ist beispielsweise kein Zufall, dass das siebzehnte Jahrhundert, das
kälteste in der modernen Epoche, auch eine Zeit weltweiter Revolutionen
war. Eine verborgene megapolitische Ursache dieser Unzufriedenheit war
das drastisch kältere Wetter. Tatsächlich war es so kalt, dass der Wein
auf dem Tisch des „Sonnenkönigs`` in Versailles gefror. Verkürzte
Wachstumsperioden führten zu Ernteausfällen und untergruben das reale
Einkommen. Aufgrund des kühleren Wetters begann der Wohlstand in eine
lange globale Depression abzurutschen, die um 1620 einsetzte. Sie erwies
sich als drastisch destabilisierend. Die Wirtschaftskrise des
siebzehnten Jahrhunderts führte dazu, dass die Welt von Aufständen
überflutet wurde, viele davon sammelten sich im Jahr 1648, genau
zweihundert Jahre vor einem anderen und berühmteren Zyklus von
Rebellionen. Zwischen 1640 und 1650 gab es Aufstände in Irland,
Schottland, England, Portugal, Katalonien, Frankreich, Moskau, Neapel,
Sizilien, Brasilien, Böhmen, der Ukraine, Österreich, Polen, Schweden,
den Niederlanden und der Türkei. Sogar China und Japan wurden von
Unruhen erfasst.

Es könnte auch sein, dass es kein Zufall war, dass der Merkantilismus im
siebzehnten Jahrhundert während einer Periode des schrumpfenden Handels
dominierte. Die wirtschaftliche Abschottung war vielleicht am Ende des
Jahrhunderts am stärksten ausgeprägt, „als eine schreckliche Hungersnot
eintrat.`` \footnote{See Charles Woolsey Cole, \emph{French
  Mercantilism: 1683-1700} (New York: Octagon Books, 197.1), S. 6.} Bis
zum achtzehnten Jahrhundert, insbesondere nach 1750, begannen wärmere
Temperaturen und höhere Ernteerträge, die realen Einkommen in Westeuropa
ausreichend zu steigern, um die Nachfrage nach Industriegütern zu
erweitern. Freiere Marktregelungen wurden angenommen. Dies führte zu
einem selbstverstärkenden Wirtschaftswachstum, da die Industrie in
größerem Maßstab ausgedehnt wurde, was allgemein als die industrielle
Revolution bezeichnet wird. Die wachsende Bedeutung von Technologie und
produzierten Waren verringerte den Einfluss des Wetters auf die
Wirtschaftszyklen.

Auch heute sollte man jedoch nicht unterschätzen, welche Auswirkungen
plötzlich kälteres Wetter auf die Senkung der realen Einkommen haben
kann -- sogar in wohlhabenden Regionen wie Nordamerika. Es besteht eine
starke Tendenz bei Gesellschaften, sich selbst krisenanfällig zu machen,
wenn die bestehende Struktur der Institutionen ihr Potenzial erschöpft
hat. In der Vergangenheit hat sich diese Neigung oft durch
Bevölkerungszunahmen manifestiert, die die Tragfähigkeit des Landes an
ihre Grenzen brachten. Dies geschah sowohl vor dem Übergang zum Jahr
1000 als auch erneut am Ende des fünfzehnten Jahrhunderts. Der Rückgang
des realen Einkommens, verursacht durch Missernten und niedrigere
Erträge, spielte in beiden Fällen eine bedeutende Rolle bei der
Zerstörung der vorherrschenden Institutionen. Heute manifestiert sich
die Marginalisierung in den Kreditmärkten für Verbraucher. Wenn
plötzlich kälteres Wetter die Ernteerträge verringern und die
verfügbaren Einkommen senken würde, könnte dies zu Schuldenausfällen
sowie zu Steuerrevolten führen. Wenn die Vergangenheit uns einen
Leitfaden gibt, könnten sowohl wirtschaftlicher Stillstand als auch
politische Instabilität die Folge sein.

3. Mikroben übertragen die Fähigkeit, Schaden anzurichten oder sich vor
Schaden zu immunisieren, auf eine Art und Weise, die oft die Art der
Machtausübung bestimmt hat. Dies war sicherlich bei der europäischen
Eroberung der neuen Welt der Fall, wie wir in \emph{The Great Reckoning}
untersucht haben. Die europäischen Siedler, die aus sesshaften, von
Krankheiten geplagten Agrargesellschaften kamen, brachten eine relative
Immunität gegen Kinderkrankheiten wie Masern mit. Die Indianer, auf die
sie trafen, lebten größtenteils in dünn besiedelten Gruppen von
Sammlern. Sie besaßen diese Immunität nicht und wurden dezimiert. Die
höchste Sterblichkeitsrate trat oft schon vor der vermehrten Ankunft der
Weißen auf, da die Indianer, die an den Küsten auf Europäer trafen, mit
Infektionen ins Landesinnere reisten.

Es gibt auch mikrobiologische Barrieren für die Ausübung von Macht. In
\emph{Blood in the Streets} haben wir die Rolle diskutiert, welche
potente Malaria-Stämme dabei spielten, das tropische Afrika über viele
Jahrhunderte hinweg unverletzlich gegenüber Invasionen durch weiße
Männer zu machen. Vor der Entdeckung von Chinin, Mitte des neunzehnten
Jahrhunderts, konnten weiße Armeen in malariagefährdeten Regionen nicht
überleben, egal wie überlegen ihre Waffen gewesen sein mögen.

Die Wechselwirkung zwischen Menschen und Mikroben hat auch wichtige
demographische Effekte hervorgebracht, die die Kosten und Belohnungen
von Gewalt veränderten. Wenn Schwankungen in der Sterblichkeit aufgrund
von Epidemien, Hungersnöten oder anderen Ursachen hoch sind, sinkt die
relative Todesrate im Krieg. Die abnehmende Häufigkeit von Sterbefällen
seit dem 16. Jahrhundert trägt zur Erklärung von kleineren Familien und
letztlich der weitaus geringeren Toleranz gegenüber plötzlichem Tod im
Krieg im Vergleich zu früher bei. Dies hat den Effekt gehabt, die
Toleranz für Imperialismus zu senken und die Kosten für die Durchsetzung
von Macht in Gesellschaften mit niedrigen Geburtenraten zu erhöhen.

Zeitgenössische Gesellschaften, die aus kleinen Familien bestehen,
finden selbst geringe Zahlen von Gefallenen in einer Schlacht oft
unerträglich. Im Gegensatz dazu zeigten sich frühere Gesellschaften den
mortalitätsbedingten Kosten des Imperialismus gegenüber viel toleranter.
Vor diesem Jahrhundert brachten die meisten Eltern viele Kinder zur
Welt, von denen einige erwartungsgemäß plötzlich und unerwartet durch
Krankheiten sterben würden. In einer Ära, in der der frühe Tod an der
Tagesordnung war, stellten sich angehende Soldaten und ihre Familien den
Gefahren des Schlachtfeldes mit weniger Widerstand.

„Die Maschinerie ist aggressiv. Der Weber wird zu einem Netz, der
Maschinist zu einer Maschine. Wenn du keine Werkzeuge benutzt, benutzen
sie dich.`` - Emerson

4. Technologie hat bei der Bestimmung von Kosten und Belohnungen der
Machtausübung in den modernen Jahrhunderten bei weitem die größte Rolle
gespielt. Die Argumentation dieses Buches geht davon aus, dass dies auch
weiterhin der Fall sein wird. Technologie hat mehrere entscheidende
Dimensionen:

A. Gleichgewicht zwischen Angriff und Verteidigung. Das Gleichgewicht
zwischen Angriff und Verteidigung, das durch die vorherrschende
Waffentechnologie impliziert wird, hilft dabei, das Ausmaß der
politischen Organisation zu bestimmen. Wenn die Angriffsfähigkeiten
steigen, dominiert die Fähigkeit, Macht auf Distanz zu projizieren,
Jurisdiktionen neigen dazu sich zu konsolidieren, und es entstehen in
größerem Maßstab Regierungen. In anderen Zeiten, wie den jetzigen,
steigt die Verteidigungsfähigkeit. Das macht es teurer, Macht außerhalb
der Kerngebiete zu projizieren. Jurisdiktionen neigen dazu, sich
aufzulösen, und große Regierungen zerbrechen in kleinere Einheiten.

B. Gleichheit und die Vorherrschaft der Infanterie. Ein Hauptmerkmal,
das den Grad der Gleichheit unter den Bürgern bestimmt, ist die Art der
Waffentechnologie. Waffen, die vergleichsweise billig sind, von Laien
verwendet werden können und die militärische Bedeutung der Infanterie
erhöhen, tendieren dazu, die Machtverhältnisse auszugleichen. Als Thomas
Jefferson schrieb, dass „alle Menschen gleich geschaffen sind``, sagte
er etwas, was viel wahrer war als eine ähnliche Aussage Jahrhunderte
zuvor hätte scheinen können. Ein Bauer mit seinem Jagdgewehr war nicht
nur so gut bewaffnet wie der typische britische Soldat mit seiner Brown
Bess, er war sogar besser bewaffnet. Der Bauer konnte mit dem Gewehr aus
größerer Distanz und mit größerer Genauigkeit auf den Soldaten schießen,
als der Soldat zurückfeuern konnte. Dies war ein deutlich anderer
Umstand als im Mittelalter, als ein Bauer mit einer Mistgabel -- mehr
hätte er sich nicht leisten können -- kaum darauf hoffen konnte, gegen
einen schwer bewaffneten Ritter zu Pferde Stand zu halten. Niemand
schrieb 1276, dass „alle Menschen gleich geschaffen sind``. Zu dieser
Zeit waren die Menschen im offensichtlichsten Sinne nicht gleich. Ein
einzelner Ritter übte viel mehr rohe Gewalt aus als Dutzende von Bauern
zusammen.

C. Vorteile und Nachteile der Skalierung von Gewalt. Eine weitere
Variable, die hilft zu bestimmen, ob es wenige große Regierungen oder
viele kleine gibt, ist die Skalierung der Organisation, die zur
Entfaltung der vorherrschenden Waffen benötigt wird. Wenn es steigende
Erträge durch Gewalt gibt, ist es lohnender, Regierungen in großem
Maßstab zu betreiben, und diese neigen dazu, größer zu werden. Wenn eine
kleine Gruppe effektive Mittel zur Abwehr eines Angriffs durch eine
große Gruppe hat, was im Mittelalter der Fall war, neigt die
Souveränität dazu, zu zerbrechen. Kleine, unabhängige Autoritäten
übernehmen viele Funktionen der Regierung. Wie wir in einem späteren
Kapitel näher ausführen, glauben wir, dass das Informationszeitalter die
Ära der Cybersoldaten einläuten wird, die Vorboten der Dezentralisierung
sein werden. Cybersoldaten könnten nicht nur von Nationalstaaten,
sondern auch von sehr kleinen Organisationen und sogar Einzelpersonen
eingesetzt werden. Die Kriege des nächsten Jahrtausends werden einige
fast blutlose Schlachten beinhalten, die mit Computern geführt werden.

D. Skaleneffekte in der Produktion. Ein weiterer wichtiger Faktor, der
Einfluss darauf hat, ob die letztendliche Macht lokal oder aus der Ferne
ausgeübt wird, ist die Rate der vorherrschenden Unternehmen, durch die
die Menschen ihren Lebensunterhalt verdienen. Wenn entscheidende
Unternehmen nur dann optimal funktionieren können, wenn sie in einem
umfassenden Handelsgebiet großflächig organisiert sind, können
Regierungen, die sich ausweiten, um ein solches Umfeld für die unter
ihrem Schutz stehenden Unternehmen bereitzustellen, genug zusätzlichen
Reichtum abschöpfen, um die Kosten für die Aufrechterhaltung eines
großen politischen Systems zu decken. Unter solchen Bedingungen
funktioniert die gesamte Weltwirtschaft in der Regel effektiver, wenn
eine höchste Weltmacht alle anderen dominiert, wie es das Britische
Empire im neunzehnten Jahrhundert tat. Aber manchmal kombinieren sich
die megapolitischen Variablen, um fallende Skaleneffekte zu erzeugen.
Wenn die wirtschaftlichen Vorteile der Aufrechterhaltung eines großen
Handelsraums schwinden, können größere Regierungen, die zuvor von den
Vorteilen größerer Handelsgebiete profitierten, damit beginnen
auseinanderzubrechen -- selbst wenn die Balance der Waffentechnik
zwischen Angriff und Verteidigung ansonsten weitgehend erhalten bleibt.

E. Verbreitung der Technologie. Ein weiterer Faktor, der zur
Machtangleichung beiträgt, ist das Ausmaß der Verbreitung entscheidender
Technologien. Wenn Waffen oder Produktionsmittel effektiv gehortet oder
monopolisiert werden können, neigen sie dazu, die Macht zu
zentralisieren. Sogar Technologien, die im Wesentlichen defensiven
Charakter haben, wie das Maschinengewehr, erwiesen sich als potente
Angriffswaffen, die während der Zeit, in der sie nicht weit verbreitet
waren, zu einer steigenden Größe der Regierung beitrugen.

Als die europäischen Mächte Ende des neunzehnten Jahrhunderts ein
Monopol auf Maschinengewehre besaßen, konnten sie diese Waffen gegen
Völker in der Peripherie einsetzen, um ihre kolonialen Reiche dramatisch
zu erweitern. Später, im zwanzigsten Jahrhundert, als Maschinengewehre,
insbesondere nach dem Zweiten Weltkrieg, weit verbreitet waren, wurden
sie eingesetzt, um die Macht der Imperien zu zerstören. Unter ansonsten
gleichen Bedingungen tendiert Macht dazu, umso breiter verteilt zu sein,
je weitflächiger entscheidende Technologien verteilt sind und umso
kleiner ist die optimale Größe der Regierung.

\section{DIE GESCHWINDIGKEIT MEGAPOLITISCHER
VERÄNDERUNGEN}\label{die-geschwindigkeit-megapolitischer-veruxe4nderungen}

Obwohl Technologie heute bei weitem der wichtigste Faktor ist und
anscheinend immer mehr an Bedeutung gewinnt, haben alle vier großen
megapolitischen Faktoren in der Vergangenheit eine Rolle dabei gespielt,
den Umfang zu bestimmen, in dem Macht ausgeübt werden konnte.

Zusammen bestimmen diese Faktoren, ob die Rendite von Gewalt weiter
steigt, wenn Gewalt in größerem Ausmaß angewendet wird. Dies bestimmt
die Bedeutung des Ausmaßes der Feuerkraft gegenüber der Effizienz in der
Nutzung von Ressourcen. Es beeinflusst auch stark die
Einkommensverteilung auf dem Markt. Die Frage lautet: Welche Rolle
werden sie in der Zukunft innehaben? Ein Schlüssel zur Abschätzung einer
Antwort liegt in der Erkenntnis, dass diese megapolitischen Variablen
sich mit dramatisch unterschiedlichen Geschwindigkeiten verändern.

Die Topographie ist durch die gesamte aufgezeichnete Geschichte hindurch
fast unverändert geblieben. Abgesehen von geringfügigen lokalen
Auswirkungen durch die Versandung von Häfen, Geländeauffüllung oder
Erosion, ist die Topographie der Erde fast dieselbe wie zu der Zeit, als
Adam und Eva aus dem Garten Eden auszogen. Und es ist wahrscheinlich,
dass es so bleiben wird, bis eine weitere Eiszeit die Landschaften der
Kontinente neu formt oder irgendein anderes dramatisches Ereignis die
Oberfläche der Erde stört. Auf einer tieferen Skala scheinen geologische
Zeitalter sich zu verschieben, vielleicht als Reaktion auf große
Meteoriteneinschläge, über einen Zeitraum von 10 bis 40 Millionen
Jahren. Irgendwann könnte es erneut geologische Umwälzungen geben, die
die Topographie unseres Planeten erheblich verändern. Wenn das passiert,
können Sie sicher davon ausgehen, dass sowohl die Fußball- als auch die
Ski-Saison abgesagt werden.

Das Klima schwankt weitaus aktiver als die Topographie. In den letzten
Millionen Jahren war der Klimawandel für den Großteil der bekannten
Variationen in den Merkmalen der Erdoberfläche verantwortlich. Während
der Eiszeiten schnitten Gletscher neue Täler, veränderten den Verlauf
der Flüsse, trennten Inseln von Kontinenten oder verbanden sie durch
Absenkung des Meeresspiegels miteinander.

Schwankungen im Klima haben in der Geschichte eine bedeutende Rolle
gespielt, zunächst durch die Auslösung der Agrarrevolution nach dem Ende
der letzten Eiszeit und später durch die Destabilisierung von
Regierungen in Zeiten kälterer Temperaturen und Dürren.

In letzter Zeit gab es Bedenken hinsichtlich der möglichen Auswirkungen
der „globalen Erwärmung``. Diese Bedenken können nicht einfach abgetan
werden. Doch aus einem langfristigen Blickwinkel erscheint das Risiko
einer Verschiebung hin zu einem kälteren, nicht zu einem wärmeren Klima
wahrscheinlicher. Studien von Temperaturschwankungen, basierend auf der
Analyse von Sauerstoffisotopen in Kernproben vom Meeresboden, zeigen,
dass die gegenwärtige Periode die zweitwärmste seit mehr als 2 Millionen
Jahren ist.\footnote{Chris Scarre, ed., \emph{Past Worlds: The Times
  Atlas of Archaeology} (New York: Random House, 1995), S. 58.} Sollten
sich die Temperaturen abkühlen, wie sie es im siebzehnten Jahrhundert
taten, könnte dies megapolitisch destabilisierend wirken. Aktuelle
Alarmmeldungen über die globale Erwärmung könnten in diesem Sinne
beruhigen. In dem Maße, in dem sie wahr sind, gewährleisten sie, dass
die Temperaturen weiterhin innerhalb des ungewöhnlich warmen und relativ
harmlosen Bereichs schwanken, den wir in den letzten drei Jahrhunderten
erlebt haben.

Die Geschwindigkeit, mit der sich der Einfluss von Mikroben auf die
Ausübung von Macht verändert, ist ein größeres Rätsel. Mikroben können
sehr schnell mutieren. Dies gilt insbesondere für Viren. Die gewöhnliche
Erkältung zum Beispiel mutiert auf nahezu kaleidoskopische Weise. Obwohl
diese Mutationen schnell voranschreiten, war ihre Auswirkung auf die
Verschiebung der Grenzen, innerhalb derer Macht ausgeübt wird, weit
weniger abrupt als technologische Veränderungen. Warum? Ein Teil der
Antwort liegt darin, dass das natürliche Gleichgewicht dazu neigt, es
für Mikroben vorteilhaft zu machen, Wirtspopulationen zu infizieren,
aber nicht zu zerstören. Sehr aggressive Infektionen, die ihre Wirte zu
schnell töten, neigen dazu, sich im Prozess selbst zu eliminieren. Das
Überleben von Mikroparasiten hängt davon ab, dass sie für die Wirte, in
die sie eindringen, nicht zu schnell oder gleichmäßig tödlich sind.

Das bedeutet natürlich nicht, dass es keine tödlichen Ausbrüche von
Krankheiten geben kann, die das Gleichgewicht der Macht verändern.
Solche Episoden haben in der Geschichte eine hervorgehobene Rolle
gespielt. Der schwarze Tod raffte große Teile der Bevölkerung von
Eurasien hinweg und versetzte der internationalen Wirtschaft des
vierzehnten Jahrhunderts einen vernichtenden Schlag.

\subsection{Was hätte sein können}\label{was-huxe4tte-sein-kuxf6nnen}

Die Geschichte kann sowohl in Bezug auf das, was hätte sein können, als
auch auf das, was war, verstanden werden. Wir wissen nicht, warum
Mikroparasiten nicht weiterhin Unheil in der menschlichen Gesellschaft
während der Neuzeit anrichten können sollten. Beispielsweise ist es
möglich, dass sich mikrobiologische Barrieren für die Ausübung von
Macht, vergleichbar mit Malaria aber virulenter, hätten aufbauen können
und so die westliche Invasion der Peripherie aufgehalten hätten. Die
ersten mutigen portugiesischen Abenteurer, die in afrikanische Gewässer
segelten, hätten sich beispielsweise ein tödliches Retrovirus, eine
ansteckendere Version von AIDS, einfangen können, das die Eröffnung der
neuen Handelsroute nach Asien gestoppt hätte, bevor sie überhaupt
begonnen hatte. Auch Kolumbus und die ersten Wellen von Siedlern in der
neuen Welt hätten auf Krankheiten stoßen können, die sie auf die gleiche
Weise dezimiert hätten, wie die indigenen lokalen Bevölkerungen durch
Masern und andere westliche Kinderkrankheiten betroffen waren. Doch
nichts dergleichen geschah, was die Intuition unterstreicht, dass die
Geschichte ein Schicksal hat.

Mikroben haben in der modernen Ära viel weniger zur Behinderung der
Machtkonsolidierung beigetragen als zu deren Förderung. Westliche
Truppen und Siedler in der Peripherie fanden oft heraus, dass die
technologischen Vorteile, die es ihnen ermöglichten, Macht auszuüben,
von mikrobiologischen unterstrichen wurden. Die Westler waren mit
unsichtbaren biologischen Waffen ausgestattet, ihrer relativen Immunität
gegen Kinderkrankheiten, die die einheimischen Völker häufig
verwüsteten. Dies gab Reisenden aus dem Westen einen deutlichen Vorteil,
den ihre Gegner aus weniger dicht besiedelten Gebieten nicht besaßen.
Wie sich die Ereignisse entwickelten, erfolgte der Krankheitstransfer
fast ausschließlich in eine Richtung - von Europa nach außen. Es gab
keine äquivalente Übertragung von Krankheiten in die andere Richtung,
vom Rand zum Kern.

Als mögliches Gegenbeispiel haben einige behauptet, dass westliche
Entdecker die Syphilis aus der neuen Welt nach Europa brachten. Dies ist
umstritten. Wenn dies jedoch wahr wäre, stellte es kein signifikantes
Hindernis für die Ausübung von Macht dar. Der Haupteffekt der Syphilis
bestand darin, die sexuellen Moralvorstellungen im Westen zu verändern.
Vom Ende des fünfzehnten Jahrhunderts bis zum letzten Viertel des
zwanzigsten Jahrhunderts wurde der Einfluss von Mikroben auf die
industrielle Gesellschaft immer harmloser. Ungeachtet der persönlichen
Tragödien und des Unglücks, die durch Ausbrüche von Tuberkulose, Polio
und Grippe verursacht wurden, traten in der modernen Zeit keine neuen
Krankheiten auf, die annähernd den megapolitischen Einfluss der
Antoninischen Pest oder des Schwarzen Todes hatten. Die Verbesserung der
öffentlichen Gesundheit sowie das Aufkommen von Impfungen und
Gegenmitteln reduzierten im Allgemeinen die Bedeutung von infektiösen
Mikroben während der modernen Periode und erhöhten dadurch die relative
Bedeutung der Technologie bei der Festlegung der Grenzen, innerhalb
derer Macht ausgeübt wurde.

Das kürzliche Auftauchen von AIDS und die Warnungen vor der potenziellen
Ausbreitung exotischer Viren sind Hinweise darauf, dass die Rolle von
Mikroben in der Zukunft möglicherweise nicht so megapolitisch harmlos
sein könnte, wie sie es in den letzten fünfhundert Jahren war. Aber wann
oder ob eine neue Pandemie die Welt befallen wird, ist unbekannt. Eine
Eruption von Mikroparasiten, wie einer viralen Pandemie, würde
wahrscheinlich eher die megapolitische Dominanz der Technologie stören
als drastische Veränderungen im Klima oder in der Topographie.

Wir haben keine Möglichkeit, drastische Abweichungen von der Natur des
Lebens auf der Erde, wie wir sie kennen, zu überwachen oder
vorauszusehen. Wir drücken die Daumen und gehen davon aus, dass die
wesentlichen megapolitischen Variablen im nächsten Jahrtausend eher
technologischer als mikrobiologischer Natur sein werden. Wenn das Glück
weiterhin auf der Seite der Menschheit steht, wird Technologie weiterhin
als die führende megapolitische Variable an Bedeutung zunehmen.

Das war jedoch nicht immer so, wie eine Überprüfung der ersten großen
Megapolitik-Transformation, der Agrarrevolution, klar zeigt.

\setsubtitle{Die Landwirtschaftliche Revolution und die Kultivierung der Gewalt}

\bookmarksetup{startatroot}

\chapter{ÖSTLICH VON EDEN}\label{uxf6stlich-von-eden}

\begin{quote}
``Und der Herr sprach zu Kain: Wo ist Abel, dein Bruder? Und er
antwortete: Ich weiß es nicht: Bin ich meines Bruders Hüter? Und er
sagte: Was hast du getan? Die Stimme des Blutes deines Bruders schreit
zu mir aus dem Erdboden.`` - GENESIS 4:9-10
\end{quote}

Vor fünfhundert Generationen begann die erste grundlegende Veränderung
in der Organisation der menschlichen Gesellschaft.\footnote{Boyden,
  ebenda, S. 4} Unsere Vorfahren in mehreren Regionen haben widerwillig
grobe Werkzeuge, geschärfte Pfähle und improvisierte Hacken aufgenommen
und sind zur Arbeit gegangen. Als sie die ersten Kulturen säten, legten
sie auch ein neues Fundament für die Macht in der Welt. Die
Agrarrevolution war die erste große ökonomische und soziale Revolution.
Sie begann mit der Vertreibung aus Eden und schritt so langsam voran,
dass die Landwirtschaft zu Beginn des zwanzigsten Jahrhunderts das Jagen
und Sammeln in allen geeigneten Gebieten der Welt noch nicht völlig
verdrängt hatte. Experten glauben, dass selbst im Nahen Osten, wo die
Landwirtschaft zuerst entstand, sie in einem „langen inkrementellen
Prozess`` eingeführt wurde, der „möglicherweise fünftausend Jahre oder
länger`` gedauert hat.\footnote{Gregg, ebenda, xv.}

Es mag übertrieben erscheinen, einen Prozess, der über Jahrtausende
hinweg stattfand, als „Revolution`` zu bezeichnen. Dennoch war genau das
der Beginn der Landwirtschaft: eine langsame Revolution, die das
menschliche Leben veränderte, indem sie die Logik der Gewalt veränderte.
Überall dort, wo die Landwirtschaft Wurzeln schlug, wurde Gewalt zu
einem wichtigeren Merkmal des sozialen Lebens. Hierarchien, die
geschickt darin waren, Gewalt zu manipulieren oder zu kontrollieren,
begannen, die Gesellschaft zu dominieren.

Das Verständnis der Agrarrevolution ist der erste Schritt zum
Verständnis der Informationsrevolution. Die Einführung von
Bodenbearbeitung und Ernte ist ein Paradigma dafür, wie eine scheinbar
einfache Verschiebung in der Beschaffenheit von Arbeit die Organisation
der Gesellschaft radikal verändern kann. Schaut man diese vergangene
Revolution im richtigen Licht an, ist man viel besser in der Lage,
Vorhersagen zu treffen, wie die Geschichte sich aufgrund der neuen Logik
der Gewalt entwickeln könnte, die mit Mikroprozessoren eingeführt wurde.

Um die revolutionäre Natur der Landwirtschaft zu würdigen, benötigt man
zuerst ein Bild davon, wie die urzeitliche Gesellschaft funktionierte,
bevor der Ackerbau eingeführt wurde. Wir haben dies in \emph{The Great
Reckoning} untersucht und bieten unten einen weiteren Überblick. Jäger-
und Sammlergesellschaften waren die einzigen Formen sozialer
Organisation während eines langen, prähistorischen Dornröschenschlafs,
in dem das menschliche Leben sich von Generation zu Generation wenig
oder gar nicht veränderte. Anthropologen behaupten, dass die Menschen
seit ihrem Auftreten auf der Erde zu 99 Prozent Jäger und Sammler waren.
Entscheidend für den langen Erfolg und letztendlichen Misserfolg der
Jäger- und Sammlergruppen ist die Tatsache, dass sie in kleiner Zahl
über ein sehr weites Gebiet operieren mussten.

Jäger und Sammler konnten nur dort überleben, wo die Bevölkerungsdichte
gering war. Um zu verstehen warum, denken Sie über die Probleme nach,
die größere Gruppen verursacht hätten. Zum einen hätte eine Gruppe von
tausend Jägern, die sich gemeinsam durch eine Landschaft bewegten, ein
solches Getöse gemacht, dass sie das Wild, das sie zu fangen suchten,
weggescheucht hätten. Noch schlimmer, wenn eine kleine Armee von Jägern
es gelegentlich geschafft hätte, eine riesige Herde Wild zu stellen,
hätte die Nahrung, die sie geerntet hätten, einschließlich Früchten und
essbarer Pflanzen, die in freier Wildbahn gefunden wurden, nicht lange
ausreichend bleiben können. Eine große Gruppe von Sammlern hätte die
Landschaft durch Überernte verwüstet, genau wie eine hungernde Armee im
Dreißigjährigen Krieg. Daher mussten Jagdgruppen, um Übermaß zu
vermeiden, klein sein. Wie Stephen Boyden in „Western Civilization in
Biological Perspective`` schreibt, „bestehen die meisten
Jäger-und-Sammler-Gruppen aus zwanzig bis fünfzig Einzelpersonen.``
\footnote{Boyden, ebenda, S. 62}

Auf zehntausend Hektar in einem gemäßigten Klima zu leben ist heute ein
Luxus, den sich nur die sehr reichen leisten können. Eine Familie von
Jägern und Sammlern hätte kaum mit weniger überleben können. Sie
benötigten im Allgemeinen Tausende von Hektar pro Person, selbst in den
Gebieten, die am fruchtbarsten waren. Dies legt nahe, warum das Wachstum
der menschlichen Bevölkerung in besonders landwirtschaftlich günstigen
Perioden die Grundlage für Bevölkerungskrisen geschaffen haben könnte.
Da zur Unterstützung einer einzelnen Person so viel Land benötigt wurde,
mussten die Bevölkerungsdichten der Jagd- und Sammelgesellschaften
unglaublich gering sein. Vor der Landwirtschaft siedelten Menschen etwa
so dicht wie Bären.

Mit geringfügigen Unterschieden ähnelte die menschliche Ernährung der
von Bären. Sammlergesellschaften waren auf Nahrung angewiesen, die im
offenen Gelände oder in nahe gelegenen Gewässern gesammelt wurde. Obwohl
einige Sammler Fischer waren, waren die meisten Jäger, die für ein
Drittel bis ein Fünftel ihrer Ernährung auf Proteine aus großen
Säugetieren angewiesen waren. Mit Ausnahme einiger weniger einfacher
Werkzeuge und Gegenstände, die sie bei sich trugen, hatten Jäger und
Sammler kaum Technologie zur Verfügung. Sie hatten meistens keine
Möglichkeit, große Mengen Fleisch oder andere Lebensmittel für die
spätere Verwendung effektiv zu lagern. Die meisten Lebensmittel mussten
bald konsumiert oder zum Verderben übrig gelassen werden. Das bedeutet
natürlich nicht, dass nicht einige Jäger und Sammler verdorbenes Essen
gegessen haben. Eskimos, so berichtet Boyden, „sollen eine besondere
Vorliebe für verdorbenes Essen haben.`` \footnote{Ebenda, S. 67} Er
wiederholt Beobachtungen von Experten, dass Eskimos „Fischköpfe
vergraben und sie verfaulen lassen, bis die Knochen die gleiche
Konsistenz wie das Fleisch erreichen. Sie kneten dann die stinkende
Masse zu einer Paste und essen sie``; sie genießen auch die „fetten,
madigen Larven der Karibufliege, die roh serviert werden...
Hirschexkremente, die wie Beeren gekaut werden... und Knochenmark, das
älter als ein Jahr ist, wimmelt von Maden.`` \footnote{Ebenda}

Neben solchen Delikatessen entwickelten Jäger und Sammler kaum
Nahrungsüberschüsse. Wie der Anthropologe Gregg anmerkt: „Mobile
Bevölkerungsgruppen lagern generell keine Nahrungsmittel für saisonale
oder unerwartete Mängel an Ressourcen ein``. Folglich hatten Jäger und
Sammler wenig zu stehlen. Eine Arbeitsteilung, die eine Spezialisierung
auf den Einsatz von Gewalt einschloss, war in Situationen, in denen
überschüssige Nahrung nicht gespeichert werden konnte, untragbar. Die
Logik der Jagd verlangte auch, dass Gewalt innerhalb von Jäger- und
Sammlergruppen nie über einen kleinen Rahmen hinaus ansteigen konnte, da
die Gruppen selbst klein bleiben mussten.

Die geringe Größe der Sammlergruppen war auf eine andere Weise von
Vorteil. Mitglieder solcher kleinen Gruppen hätten sich gegenseitig
genauestens gekannt, ein Faktor, der sie effektiver in der
Zusammenarbeit machte. Entscheidungsfindung wird schwieriger, wenn die
Anzahl steigt, denn die Gefahr von Meinungsverschiedenheiten nimmt zu.
Man braucht nur darüber nachzudenken, wie schwierig es ist, ein Dutzend
Leute dafür zu organisieren, essen zu gehen. Man stelle sich vor, wie
aussichtslos es gewesen wäre, Hunderte oder Tausende von Personen zu
organisieren, die auf einem beweglichen Fest herumlaufen sollten. Da sie
keine längerfristige und separate politische Organisation oder
Bürokratie hatten, die durch die Spezialisierung für den Krieg benötigt
wird, mussten Jäger- und Sammlergruppen auf Überzeugungsarbeit und
Konsensbildung angewiesen sein. Diese Prinzipien funktionieren am besten
in kleinen Gruppen mit einer relativ gelassenen Einstellung.

Ob Jäger- und Sammlergruppen gelassen waren, ist umstritten. Sir Henry
Maine spricht von der „universellen Kriegsführung des primitiven
Menschen``. In seinen Worten: „Es ist nicht der Frieden, der natürlich
und primitiv ist, sondern der Krieg.`` \footnote{Zitiert in E. J. P
  Veale, \emph{Advance to Barbarism: The Development of Total Warfore}
  (New York: Devin-Adair, 1968), S. 37} Diese Ansicht wurde durch die
Arbeit von Evolutionsbiologen bestätigt. R. Paul Shaw und Yuwa Wong
bemerken: „Es gibt starke Anhaltspunkte dafür, dass viele Verletzungen,
die an den Überresten von Australopithecus, Homo erectus und Homo
sapiens aus und vor der vierten Eiszeit Europas sichtbar sind, auf
Kämpfe zurückzuführen sind.`` \footnote{R. Paul Shaw und Yuwa Wong,
  \emph{Genetic Seeds of Warfare: Evolution, Nationalism and Patriotism}
  (Boston: Unwin Hyman, 1989), S. 4} Aber andere bezweifeln das.
Experten wie Stephen Boyden argumentieren, dass primitive Gruppen in der
Regel nicht kriegerisch oder gewalttätig waren. Soziale Konventionen
wurden entwickelt, um interne Spannungen zu verringern und das Teilen
der Jagdbeute zu erleichtern. Insbesondere in Gebieten, in denen
Menschen größere Beutetiere jagten, die für einen einzelnen Jäger schwer
zu erlegen waren, entstanden religiöse und soziale Lehren, um die
Umverteilung von erjagtem Wild innerhalb der gesamten Gruppe zu
erleichtern. Die oberste Priorität beim Teilen der kalorienhaltigen
Ressourcen galt den anderen Jägern. Die Notwendigkeit, nicht das
Mitgefühl, war der Ansporn. Der erste Anspruch auf die Ressourcen wurde
von den wirtschaftlich kompetentesten und militärisch stärksten
ausgeübt, nicht von den Kranken und Schwachen. Zweifellos war ein
wichtiger Einflussfaktor für diese Priorität die Tatsache, dass Jäger in
der Blüte ihres Lebens auch die militärisch stärksten Mitglieder der
kleinen Gruppe waren. Durch die Garantie eines ersten Anspruchs auf die
Jagdbeute minimierte die Gruppe potenziell tödliche interne
Streitereien.

Solange die Bevölkerungsdichten gering blieben, waren die Götter der
Jäger und Sammler keine kriegerischen Götter, sondern Verkörperungen von
natürlichen Kräften oder den Tieren, die sie jagten. Der Mangel an
Kapital und offene Grenzen machten Krieg in den meisten Fällen
überflüssig. Es gab wenige Nachbarn außerhalb der eigenen kleinen
Familie oder des Clans, die eine Bedrohung darstellten. Da die Jäger und
Sammler ständig auf der Suche nach Nahrung umherwanderten, wurden
persönliche Besitztümer über das absolute Minimum hinaus zu einer
Belastung. Diejenigen mit wenigen Besitztümern waren zwangsläufig wenig
von Eigentumsdelikten betroffen. Wenn Konflikte aufkamen, waren die
strittigen Parteien oft bereit, einfach weiterzuziehen, da sie wenig in
einen bestimmten Ort investiert hatten. Flucht war eine einfache Lösung
für persönliche Fehden oder übertriebene Forderungen anderer Art. Doch
das bedeutet nicht, dass die frühen Menschen friedlich waren. Sie
könnten gewalttätig und unangenehm gewesen sein, in einem Ausmaß, das
wir uns kaum vorstellen können. Aber wenn sie Gewalt anwendeten, dann
meist aus persönlichen Gründen oder, was noch schlimmer sein könnte, zum
Spaß.

Die Lebensweise der Jäger und Sammler hing davon ab, dass sie in kleinen
Gruppen lebten, die kaum Spielraum für eine Arbeitsteilung jenseits der
Geschlechterrollen ließen. Sie verfügten über keine organisierten
Regierungen, meistens gab es keine dauerhaften Siedlungen und keine
Möglichkeiten, Reichtum anzuhäufen. Selbst derart grundlegende Elemente
einer Zivilisation wie eine geschriebene Sprache waren in der
ursprünglichen Wirtschaft unbekannt. Ohne eine geschriebene Sprache gab
es keine formalen Dokumente und keine Geschichte.

\subsection{Übermaß}\label{uxfcbermauxdf}

Die Dynamik des Sammelns und Jagens schuf völlig unterschiedliche
Anreize zur Arbeit als jene, an die wir uns seit dem Aufkommen der
Landwirtschaft gewöhnt haben. Die Kapitalerfordernisse für ein Leben als
Sammler und Jäger waren minimal. Einige primitive Werkzeuge und Waffen
genügten. Es gab keinen Anlass für Investitionen, nicht einmal
Privatgrundbesitz, außer gelegentlich in Steinbrüchen, in denen
Feuerstein oder Speckstein abgebaut wurde.\footnote{Siehe Carleton S.
  Coon, \emph{The Hunting Peoples} (New York: Nick Lyons Books, 1971),
  S. 275} Wie die Anthropologin Susan Alling Gregg in „Foragers and
Farmers`` schrieb, wurde „Eigentum und Zugang zu Ressourcen von der
Gruppe gemeinschaftlich gehalten``.\footnote{Gregg, ebenda, S. 23} Mit
wenigen Ausnahmen, wie Fischern, die an Seen wohnen, zogen Sammler und
Jäger meistens umher, hatten also keine feste Bleibe. Da sie keine
permanenten Heime hatten, hatten sie wenig Notwendigkeit, schwer zu
arbeiten um Besitz zu erwerben oder zu erhalten. Sie hatten weder
Hypotheken noch Steuern zu zahlen oder Möbel zu kaufen. Ihre wenigen
Konsumgüter waren Tierhäute und persönlicher Schmuck, hergestellt von
Mitgliedern der eigenen Gruppe. Es gab wenig Anreiz, etwas zu erwerben
oder anzuhäufen, was man als Geld hätte betrachten können, weil es kaum
etwas zu kaufen gab. Unter solchen Bedingungen war das Sparen für die
Sammler und Jäger kaum mehr als ein rudimentäres Konzept.

Ohne Grund zum Verdienen und fast keiner Arbeitsteilung, muss das
Konzept von schwerer Arbeit als Tugend den Jagd- und Sammelgruppen fremd
gewesen sein. Außer in Zeiten von ungewöhnlichen Entbehrungen, wenn
anhaltende Anstrengungen erforderlich waren, um etwas zu Essen zu
finden, wurde wenig Arbeit geleistet, da wenig benötigt wurde. Es war
buchstäblich nichts zu gewinnen, indem man über das absolute Minimum
hinaus arbeitete, das für das Überleben erforderlich war. Für die
Mitglieder einer typischen Jagd- und Sammelgruppe bedeutete das, nur
etwa acht bis fünfzehn Stunden pro Woche zu arbeiten. Da die Arbeit
eines Jägers das Nahrungsangebot nicht erhöhte, sondern nur verringern
konnte, trug jemand, der übermäßig arbeitete, um mehr Tiere zu töten
oder mehr Früchte zu pflücken, als vor dem Verderben gegessen werden
konnten, nichts zum Wohlstand bei. Im Gegenteil, übermäßiges Töten
verringerte die Aussichten auf Nahrung in der Zukunft und hatte somit
einen nachteiligen Einfluss auf das Wohlbefinden der Gruppe. Deshalb
wurden einige Sammler, zum Beispiel bei den Eskimos, bestraft oder
ausgestoßen, wenn sie übermäßig viele Tiere töteten.

Das Beispiel der Eskimos, die das Übermaß bestrafen, ist besonders
aufschlussreich, denn sie hätten weit mehr als andere in der Lage sein
können, Fleisch durch Einfrieren zu lagern. Außerdem wäre es machbar
gewesen, zumindest eine gewisse Lagerung für Öle bereitzustellen, die
aus großen Meereslebewesen gewonnen wurden. Die Tatsache, dass die
Sammler allgemein entschieden, dies nicht zu tun, spiegelt ihre weitaus
passivere Interaktionen mit der Natur wider. Es kann auch den Grad
aufzeigen, inwieweit kognitive und geistige Prozesse durch Kultur
beeinflusst sind. Beschränkungen beim Lernen und Verhalten in komplexen
Umgebungen machen die Übernahme mancher Strategien weitaus schwieriger,
als es sonst erscheinen mag. Wie R. Paul Shaw und Yuwa Wong geschrieben
haben: „Da sich Lebensräume in vielerlei Hinsicht unterscheiden,
variiert auch die Lernverzerrung.``

Aus dieser Perspektive gesehen, brachte das Aufkommen der Landwirtschaft
mehr als nur eine Änderung in der Ernährung mit sich; es leitete auch
eine große Revolution in der Organisation des wirtschaftlichen Lebens
und der Kultur ein und verwandelte die Logik der Gewalt. Die
Landwirtschaft schuf groß angelegte Kapitalanlagen in Form von Land und
manchmal in Bewässerungssystemen. Die von den Bauern angelegten
Getreidevorräte und domestizierten Tiere waren wertvolle Vermögenswerte.
Sie konnten gelagert, gehortet und gestohlen werden. Da die Anbauflächen
während der gesamten Wachstumsperiode, von der Aussaat bis zur Ernte,
gepflegt werden mussten, wurde eine Flucht vor Bedrohungen, insbesondere
in trockenen Regionen, in denen die Möglichkeiten zum Anbau begrenzt
waren, wenig attraktiv. Da die Flucht schwieriger wurde, nahmen die
Möglichkeiten für organisierte Erpressungen und Plünderungen zu. Bauern
waren Ernteräubern ausgesetzt, was allmählich das Ausmaß der
Kriegsführung erhöhte.

Dies führte dazu, dass die Größe von Gesellschaften zunahm, da
Gewaltkonflikte meistens von der größeren Gruppe gewonnen wurden. Als
der Wettbewerb um Land und die Kontrolle über dessen Ertrag intensiver
wurde, wurden Gesellschaften sesshafter. Eine Arbeitsteilung wurde
deutlicher. Beschäftigung und Sklaverei entstanden zum ersten Mal.
Bauern und Hirten spezialisierten sich auf die Nahrungsmittelproduktion,
Töpfer produzierten Behälter zur Lagerung von Nahrungsmitteln, Priester
beteten für Regen und ertragreiche Ernten. Gewaltspezialisten, die
Vorläufer von Regierungen, widmeten sich zunehmend der Plünderung und
dem Schutz vor Plünderungen. Zusammen mit den Priestern wurden sie zu
den ersten reichen Personen in der Geschichte.

In den frühen Stadien landwirtschaftlicher Gesellschaften kamen diese
Krieger dazu, einen Teil der jährlichen Ernte als Preis für ihren Schutz
zu kontrollieren. In Gegenden, in denen die Bedrohungen minimal waren,
konnten selbstständige Bauern manchmal einen relativ großen Grad an
Autonomie bewahren. Mit steigender Bevölkerungsdichte und intensiverem
Wettbewerb um Nahrung, vor allem in Regionen um Wüsten, wo fruchtbares
Land sehr wertvoll war, konnte die Kriegergruppe einen großen Anteil der
Gesamtproduktion einnehmen. Mit den Einnahmen aus dieser Abgabe
gründeten diese Krieger die ersten Staaten, welche bis zu 25 Prozent der
Getreideernte und die Hälfte des Zuwachses an Nutztierherden übernahmen.
Die Landwirtschaft erhöhte daher dramatisch die Bedeutung von Zwang und
Gewalt. Der Anstieg an Ressourcen, die geplündert werden konnten, führte
auch zu einem großen Anstieg an Plünderungen.

Es dauerte Jahrtausende, bis sich die vollständige Logik der
Agrarrevolution entfaltete. Eine lange Zeit lebten dünn siedelnde
Bevölkerungen von Bauern in gemäßigten Regionen vielleicht noch ähnlich
wie ihre Vorfahren, die noch sammelten und jagten. Dort, wo Land und
Niederschläge reichlich waren, bewirtschafteten Bauern kleinere Kulturen
ohne viel gewaltsame Einmischung. Doch als im Laufe von mehreren tausend
Jahren die Bevölkerungszahlen stiegen, wurden selbst Bauern in dünn
besiedelten Regionen von gelegentlichen Plünderungen heimgesucht, was
manchmal dazu führte, dass sie nicht genug Saatgut hatten, um die Ernte
des nächsten Jahres neu zu säen. Wettbewerbsgetriebenes Plündern oder
Anarchie waren ein mögliches Extrem, ebenso wie ungeschützte
Gemeinschaften, die ohne eine spezialisierte Organisation lebten, um die
Gewalt zu monopolisieren.

Im Verlauf der Zeit setzte sich die in der Landwirtschaft inne liegende
Gewaltlogik über ein immer größer werdendes Gebiet durch. Die Regionen,
in denen Ackerbau und Viehzucht ohne die Raubzüge der Regierung
fortgesetzt werden konnten, zogen sich auf wenige wirklich entlegene
Gebiete zurück. Die Kafir-Regionen in Afghanistan, um ein extremes
Beispiel zu nennen, leisteten bis zum letzten Jahrzehnt des neunzehnten
Jahrhunderts Widerstand gegen die Einführung einer Regierung. Aber
dadurch wurden sie Jahrhunderte zuvor in eine ziemlich militante
Gesellschaft umgewandelt, die entlang von Verwandtschaftslinien
organisiert war. Solche Vereinbarungen waren nicht fähig, Kräfte im
großen Maßstab zu mobilisieren. Bis die Briten moderne Waffen in die
Region brachten, blieben die Kafire in ihren abgelegenen Tälern Bashgal
und Waigal unabhängig, weil ihre Zufluchtsorte durch topografische
Besonderheiten, hohe Berge und Wüsten vor Eindringlingen von außen
geschützt waren.\footnote{Für weitere Details über die Kafire, siehe
  Schuyler Jones, \emph{Men of Influence in Nuristan} (London: Seminar
  Press, 1974).}

Mit der Zeit hinterließ die grundlegende Logik der Agrarrevolution ihre
Spuren in den Gesellschaften, in denen die Landwirtschaft Fuß fasste.
Die Landwirtschaft erhöhte drastisch den Rahmen, in dem menschliche
Gemeinschaften entstehen konnten. Vor etwa zehntausend Jahren begannen
die ersten Städte zu entstehen. Obwohl sie nach heutigen Maßstäben
winzig waren, bildeten sie die Zentren der ersten „Zivilisationen``, ein
Wort, das von „civis`` abgeleitet ist, was auf Latein „Bürger`` oder
„Stadtbewohner`` bedeutet. Da die Landwirtschaft Vermögenswerte schuf,
die es zu plündern, beziehungsweise zu schützen galt, entstand auch ein
Bedarf an Inventarbuchhaltung. Man kann nicht besteuern, wenn man nicht
in der Lage ist, Aufzeichnungen zu führen und Quittungen auszustellen.
Die in der Buchhaltung verwendeten Symbole wurden die Grundlagen der
geschriebenen Sprache, eine Innovation, die es unter Jägern und Sammlern
nie gegeben hatte.

Die Landwirtschaft erweiterte auch den Horizont, innerhalb dessen
Menschen Probleme lösen mussten. Jagdbanden lebten innerhalb eines
unmittelbaren Zeithorizonts. Sie unternahmen selten Projekte, die länger
als ein paar Tage dauerten. Aber das Pflanzen und Einbringen einer Ernte
dauerte Monate. Längerfristige Projekte zwangen die Landwirte dazu, ihre
Aufmerksamkeit auf die Sterne zu richten. Detaillierte astronomische
Beobachtungen waren eine Voraussetzung für die Erstellung von Almanachen
und Kalendern, die als Leitfaden dienten, wann am besten gepflanzt und
geerntet werden sollte. Mit dem Aufkommen der Landwirtschaft erweiterten
sich die Horizonte der Jäger.

\section{EIGENTUM}\label{eigentum}

Der Übergang zu einer sesshaften landwirtschaftlichen Gesellschaft
führte zur Entstehung von Privateigentum. Natürlich würde niemand
zufrieden sein, eine ganze Wachstumsperiode durchzuarbeiten, um eine
Ernte zu produzieren, nur um zu sehen, dass jemand anders daherkam und
erntete, was er produziert hatte. Die Idee des Eigentums entstand als
unvermeidliche Folge der Landwirtschaft. Aber die Klarheit des Konzepts
des Privateigentums wurde durch die Logik der Gewalt getrübt, die auch
die Einführung der Landwirtschaft begleitete. Die Entstehung des
Eigentums wurde durcheinander gebracht durch die Tatsache, dass die
megapolitische Macht von Einzelpersonen nicht mehr in der Art gleich
war, wie sie es in Sammlergesellschaften gewesen war, wo jeder gesunde
erwachsene Mann ein Jäger war, so gut bewaffnet wie jeder andere.
Landwirtschaft führte zur Spezialisierung von Gewalt. Gerade weil es nun
etwas zum Stehlen gab, machte Landwirtschaft Investitionen in bessere
Waffen rentabel. Das Ergebnis war Raub, in vielerlei Hinsicht hoch
organisiert.

Die Mächtigen waren nun in der Lage, eine neue Form des Raubes zu
organisieren: ein Monopol auf Gewalt, beziehungsweise eine Regierung.
Dies unterschied die Gesellschaften deutlich und schuf völlig
verschiedene Umstände für diejenigen, die von der Plünderung
profitierten, und die große Masse an Armen, die die Felder bearbeiteten.
Die wenigen, die die militärische Macht kontrollierten, konnten nun
reich werden, zusammen mit anderen, die bei ihnen in der Gunst standen.
Die Gott-Könige und ihre Verbündeten, die verschiedenen kleineren,
lokalen Machthaber, die die ersten Nahoststaaten regierten, genossen
eine nahezu moderne Form des Eigentums verglichen mit der großen Masse,
die unter ihnen arbeitete.

Natürlich ist es anachronistisch, in den frühen landwirtschaftlichen
Gesellschaften einen Unterschied zwischen privatem und öffentlichem
Reichtum zu sehen. Der regierende Gott-König hatte die vollen Ressourcen
des Staates in einer Weise zu seiner Verfügung, die kaum vom Besitz
eines weitläufigen Anwesens zu unterscheiden war. Ganz wie in der
feudalen Periode der europäischen Geschichte war jeglicher Besitz der
Oberherrschaft höherer Machthaber unterworfen. Diejenigen weiter unten
in der Hierarchiekette mussten mit einer Verringerung ihres Eigentums
nach den Launen des Herrschers rechnen.

Doch zu sagen, dass der Machthaber nicht durch das Gesetz eingeschränkt
war, bedeutet nicht, dass er sich leisten konnte, alles zu nehmen, was
ihm gefiel. Kosten und Belohnungen wirkten genauso stark auf die
Freiheit des Pharaos ein wie heute auf den Premierminister von Kanada.
Und der Pharao war durch die Schwierigkeiten von Transport und
Kommunikation weitaus stärker eingeschränkt als heutige Führer. Einfach
nur die Beute von einem Ort zum nächsten zu transportieren, insbesondere
wenn die Beute hauptsächlich in Form von landwirtschaftlichen
Erzeugnissen gemessen wurde, ging häufig mit Verlusten durch Verderb und
Diebstahl einher. Die Ausweitung der Zahl von Beamten, die einander
überprüften, reduzierte zwar den Verlust durch Diebstahl, erhöhte jedoch
die Gesamtkosten, die der Pharao tragen musste. Dezentrale Autorität,
welche unter bestimmten Umständen die Produktion optimierte, führte auch
zu stärkeren lokalen Mächten, die manchmal zu vollwertigen
Herausforderern um die Kontrolle der Dynastie heranwuchsen. Selbst
orientalische Despoten waren keineswegs frei, zu tun und zu lassen, was
sie wollten. Sie hatten keine Wahl, als das Gleichgewicht der reinen
Macht zu erkennen, wie sie es vorfanden.

Obwohl alle, einschließlich der Reichen, einer willkürlichen Enteignung
unterlagen, konnten einige eigenes Eigentum ansammeln. Damals wie heute
widmete der Staat einen Großteil seiner Einnahmen den öffentlichen
Arbeiten. Projekte wie Bewässerungssysteme, religiöse Denkmäler und
Krypten für die Könige boten Architekten und Handwerkern
Einkommensmöglichkeiten. Einige gut situierte Individuen konnten
beträchtliches privates Eigentum ansammeln. Tatsächlich zeichnet ein
großer Teil der überlebenden Keilschrifttafeln aus Sumer, einer frühen
mesopotamischen Zivilisation, verschiedene Handelsakte auf, von denen
die meisten den Übergang von Eigentumstiteln beinhalteten.

Es gab privates Eigentum in den frühen landwirtschaftlichen
Gesellschaften, jedoch selten am unteren Ende der sozialen Pyramide. Die
übergroße Mehrheit der Bevölkerung bestand aus Bauern, die zu arm waren,
um viel Reichtum anzuhäufen. Tatsächlich waren, bis auf wenige
Ausnahmen, die meisten Bauern bis in die Neuzeit hinein so arm, dass sie
ständig Gefahr liefen, an Hunger zu sterben, sobald Trockenheit,
Überschwemmungen oder Schädlinge die Ernteerträge reduzierten. Daher
waren die Bauern gezwungen, ihre Angelegenheiten so zu organisieren,
dass sie die Risiken in schlechten Jahren minimierten. In der
weitläufigen und verarmten Bevölkerungsschicht galt eine primitivere
Organisation des Eigentums. Sie erhöhte die Überlebenschance auf Kosten
der Einschränkung der Möglichkeit, Kapital anzuhäufen und im
Wirtschaftssystem aufzusteigen.

\subsection{Bauernversicherung}\label{bauernversicherung}

Die Form, die dieser Handel annahm, war die Annahme dessen, was
Anthropologen und Sozialhistoriker als das „geschlossene Dorf``
beschreiben. Fast jede bäuerliche Gesellschaft in vorindustriellen
Zeiten hatte als Hauptform der Wirtschaftsorganisation das „geschlossene
Dorf``. Im Unterschied zu moderneren Formen der Wirtschaftsorganisation,
in denen Individuen dazu neigen, mit vielen Käufern und Verkäufern auf
einem offenen Markt zu handeln, schlossen sich die Haushalte des
geschlossenen Dorfes zusammen, um wie eine informelle Körperschaft oder
eine große Familie zu agieren, nicht auf einem offenen Markt, sondern in
einem geschlossenen System, in dem alle wirtschaftlichen Transaktionen
des Dorfes dazu neigten, mit einem einzelnen Monopolisten getätigt zu
werden - dem örtlichen Grundbesitzer oder seinen Beauftragten unter den
Dorfhäuptlingen. Das Dorf als Ganzes schloss mit dem Grundherrn einen
Vertrag über einen hohen Anteil an der Ernte und nicht über einen festen
Pachtzins ab, in der Regel gegen Sachleistungen. Die prozentuale Miete
bedeutete, dass der Pächter einen Teil des Risikos einer schlechten
Ernte übernahm. Natürlich nahm der Pächter auch den größeren Teil des
potenziellen Gewinns. Typischerweise stellten Vermieter auch das Saatgut
bereit.

Diese Praxis minimierte auch die Gefahr von Hungersnöten. Sie verlangte,
dass der Gutsherr, anstatt des Bauern, einen überproportionalen Anteil
seiner Ernte sparte. Da die landwirtschaftlichen Erträge in der
Vergangenheit in vielen Gebieten erschreckend niedrig waren, mussten oft
zwei ausgesäte Samen für drei geerntete gepflanzt werden. Unter solchen
Bedingungen würde eine schlechte Ernte Hungersnot bedeuten. Die Bauern
bevorzugten vernünftigerweise eine Regelung, die den Gutsherren dazu
verpflichtete, in ihr Überleben zu investieren. Auf Kosten des Kaufs zu
monopolisierten Preisen, des günstigen Verkaufs und der Bereitstellung
von Naturallohn erhöhten die Bauern ihre Überlebenschancen. Ein
ähnlicher Impuls führte bei den typischen Bauern in einer geschlossenen
Dorfwirtschaft dazu, auf die Sicherheit des freien Eigentums zu
verzichten. Indem sie sich dem Dorfoberhaupt auslieferte, verbesserte
eine Bauernfamilie ihre Chancen, von der regelmäßigen Neuverteilung der
Felder zu profitieren. Nicht selten nahm das Dorfoberhaupt die besten
Felder für sich und seine Begünstigten. Aber das war ein Risiko, das die
Bauern in Kauf nehmen mussten, um die Überlebensversicherung zu
genießen, die das durcheinandergebrachte dörfliche Eigentum der Felder
bot. In Zeiten, in denen die Ernteerträge erbärmlich niedrig waren,
könnte der Unterschied in den Wachstumsbedingungen von Feldern, die nur
hundert Meter voneinander entfernt waren, zwischen Hunger und Überleben
entscheiden. Bauern entschieden sich häufig für die Regelung, die das
Risiko minimierte, auch auf Kosten der Aufgabe jeglicher Hoffnung auf
gesteigerten Wohlstand.

Im Allgemeinen war risikoscheues Verhalten bei allen Gruppen, die am
Rande des Überlebens agierten, weit verbreitet. Die bloße
Herausforderung des Überlebens in vorindustriellen Gesellschaften
beschränkte stets das Verhalten der Armen. Ein interessantes Merkmal
dieser Risikoaversion, das wir in \emph{The Great Reckoning}
untersuchen, ist, dass sie die Bandbreite friedlicher wirtschaftlicher
Verhaltensweisen einschränkte, die den Einzelnen gesellschaftlich
gestattet waren. Tabus und soziale Beschränkungen limitierten
Experimente und innovatives Verhalten, selbst zu den offensichtlichen
Kosten, potenziell vorteilhafte Veränderungen in festgelegten
Vorgehensweisen aufzugeben. Dies war eine rationale Reaktion auf die
Tatsache, dass Experimente die Variabilität der Ergebnisse vermehren.

Größere Variabilität bedeutet nicht nur potenziell größere Gewinne,
sondern - was für diejenigen, die am Rande des Überlebens stehen, noch
bedrohlicher ist - potenziell ruinöse Verluste. Ein großer Teil der
kulturellen Energie armer Bauerngesellschaften wurde immer darauf
verwendet, die Experimentierfreudigkeit zu unterdrücken. Diese
Unterdrückung war gewissermaßen ihr Ersatz für Versicherungspolicen.
Hätten sie Versicherungen oder ausreichende Ersparnisse, um ihre
Experimente selbst zu versichern, wären solch starke soziale Tabus nicht
notwendig, um das Überleben zu gewährleisten.

Kulturen sind keine Geschmacksfragen, sondern Anpassungssysteme an
spezifische Umstände, die in anderen Kontexten irrelevant oder sogar
kontraproduktiv sein können. Menschen leben in einer Vielzahl von
Lebensräumen. Die große Anzahl potenzieller Nischen, in denen wir leben,
erfordert Verhaltensvariationen, die zu komplex sind, um durch Instinkt
bestimmt zu werden. Daher ist das Verhalten kulturell programmiert. Für
die überwiegende Mehrheit in vielen landwirtschaftlichen Gesellschaften
hat die Kultur sie auf das Überleben programmiert, jedoch nur auf ein
Überleben in einer Umgebung, in der der Luxus der Teilnahme an offenen
Märkten anderen vorbehalten war.

Persönliche Fähigkeiten und persönliche Entscheidungen - das
individuelle „Streben nach Glück`` im modernen Sinne - wurden durch
Tabus und soziale Einschränkungen unterdrückt, die unter den Armen immer
am stärksten waren. Solche Einschränkungen wurden nur unter großen
Schwierigkeiten in Gesellschaften mit begrenzter Produktivität
überwunden. Dort wo die landwirtschaftliche Produktivität höher war, wie
zum Beispiel im antiken Griechenland, traten kleinere megapolitische
Revolutionen auf. Besitz nahm modernere Formen an. „Allod``, oder freies
Eigen, entstand. Land neigte dazu, gegen eine feste Gebühr verpachtet zu
werden und der Pächter trug das wirtschaftliche Risiko sowie einen
höheren Anteil am Gewinn, wenn die Ernte gut war. Höhere Ersparnisse
ermöglichen eine Selbstversicherung für risikoreicheres Verhalten. Unter
solchen Bedingungen konnten freie Bauern über den Stand der Bauernschaft
hinaussteigen und manchmal sogar unabhängigen Reichtum ansammeln.

Die Tendenz, dass sich marktähnliche Eigentumsrechte und Beziehungen
nahe der Spitze einer wirtschaftlichen Hierarchie oder in selteneren
Fällen in der gesamten Wirtschaft entwickeln, während Gesellschaften aus
der Armut hervorgehen, ist ein wichtiges Merkmal der sozialen
Organisation. Es ist genauso wichtig zu bemerken, dass die gängigste
Organisation der Agrargesellschaft in der Geschichte grundsätzlich
feudal war, mit Marktbeziehungen an der Spitze und einem geschlossenen
Dorfsystem am unteren Ende. Die große Mehrheit der Bauern war fast in
allen vormodernen Agrargesellschaften an das Land gebunden. Solange die
landwirtschaftliche Produktivität niedrig blieb, oder höhere
Produktivität vom Zugang zu zentralisierten hydraulischen Systemen
abhing, blieben Freiheit und Eigentumsrechte der einzelnen Bauern am
unteren Ende minimal. Unter solchen Bedingungen herrschten feudale
Formen von Eigentum. Das Land wurde durch Pachtverträge und nicht durch
Freilanderwerb gehalten. In der Regel waren Verkaufs-, Schenkungs- und
Erbschaftsrechte eingeschränkt.

Feudalismus in seinen verschiedenen Formen war nicht nur eine Reaktion
auf die ständig vorhandene Gefahr von räuberischer Gewalt. Es war auch
eine Reaktion auf erschreckend niedrige Produktivitätsraten. Die beiden
Aspekte gingen in landwirtschaftlichen Gesellschaften häufig Hand in
Hand. Das eine trug häufig zum anderen bei. Wenn die öffentliche
Autorität zusammenbrach, neigten Eigentumsrechte und Wohlstand dazu,
entsprechend zurückzugehen. Ein Zusammenbruch der Produktivität
untergrub auch oft die Autorität. Obwohl nicht jede Dürre oder
klimatische Veränderung zum Zusammenbruch der öffentlichen Autorität
führte, geschah dies doch in vielen Fällen.

\section{DIE FEUDALE REVOLUTION DES JAHRES
1000}\label{die-feudale-revolution-des-jahres-1000}

So war es auch bei der Transformation um das Jahr 1000, die die feudale
Revolution auslöste.\footnote{Siehe Bois, ebenda.} Zu dieser Zeit
unterschieden sich die megapolitischen und wirtschaftlichen Bedingungen
in wichtigen Punkten von denen, die wir normalerweise mit dem
Mittelalter assoziieren. In den ersten Jahrhunderten nach dem Fall Roms
schrumpfte die Wirtschaft Westeuropas. Die germanischen Königreiche, die
in den Gebieten des ehemaligen Römischen Reiches Wurzeln schlugen,
übernahmen viele Funktionen des römischen Staates, jedoch auf einem viel
weniger ambitionierten Niveau. Die Infrastruktur wurde mehr oder weniger
vernachlässigt. Mit den Jahrhunderten verfielen Brücken und Aquädukte
und wurden unbrauchbar. Römische Münzen wurden zwar noch verwendet,
verschwanden aber praktisch aus dem Umlauf. Grundstücksmärkte, die in
der Römerzeit geblüht hatten, trockneten mehr oder weniger aus. Städte,
die Zentren der römischen Verwaltung gewesen waren, verschwanden
praktisch zusammen mit der Besteuerungskraft des Staates. Und so
verschwand fast jedes andere Anhängsel der Zivilisation.

Das sogenannte „Dunkle Zeitalter`` wurde aus gutem Grund so genannt. Die
Alphabetisierungsrate sank so sehr, dass jeder, der lesen und schreiben
konnte, nahezu Straffreiheit für fast jedes Verbrechen erwarten konnte,
einschließlich Mord. Künstlerische, wissenschaftliche und technische
Fähigkeiten, die zu römischen Zeiten hochentwickelt waren, verschwanden.
Von Straßenbau bis hin zu Veredelungstechniken für Reben und Obstbäume
stellte Westeuropa viele Techniken ein, die einst gut bekannt waren und
auf hohem Niveau praktiziert wurden. Selbst so alte Gerätschaften wie
die Töpferscheiben verschwanden an vielen Orten. Bergbaubetriebe wurden
zurückgefahren. Die Metallurgie ging zurück. Bewässerungsanlagen im
Mittelmeerraum verfielen aufgrund von fehlender
Instandhaltung.\footnote{Siehe Frances und Joseph Gies, \emph{Cathedral,
  Forge, und Waterwheel: Technology and Invention in the Middle Ages}
  (New York: HarperCollins, 1994), S. 40.} Wie der Historiker Georges
Duby feststellte, „war Europa am Ende des sechsten Jahrhunderts ein
zutiefst unzivilisierter Ort.`` \footnote{Zitiert ebenda, S. 42.} Obwohl
es eine kurze Renaissance der zentralen Autorität unter der Herrschaft
von Karl dem Großen um das Jahr 800 gab, zerfiel alles bald wieder nach
seinem Tod.

Ein überraschender Nebeneffekt dieser trostlosen Landschaft war die
Tatsache, dass der Zusammenbruch des römischen Staates wahrscheinlich
über mehrere Jahrhunderte den Lebensstandard der Kleinbauern erhöhte.
Die germanischen Königreiche, die während des Dunklen Zeitalters
Westeuropa dominierten, integrierten einige der relativ entspannten
sozialen Merkmale, die ihren Stammesvorfahren gemeinsam waren, wie zum
Beispiel die rechtliche Gleichstellung der Freibauern. Infolgedessen
waren die Kleinbauern im dunklen Zeitalter weitaus freier als sie es in
den feudalistischen Jahrhunderten sein sollten. Daraus können wir auch
schließen, dass sie wohlhabender waren. Wie wir weiter oben bei der
Untersuchung der Logik von Eigentumsformen unter verschiedenen
Produktivitätsbedingungen analysiert haben, ging Eigentum in freier
Haltung historisch gesehen Hand in Hand mit dem relativen Wohlstand von
Kleinbauern. Die geschlossenen Dorf- und Feudalformen des Eigentums
tendierten dazu, dort aufzutreten, wo die Fähigkeit der Kleinbauern,
ihren Lebensunterhalt zu verdienen, zweifelhafter war.

Sicherlich kostete der Quasi-Zusammenbruch des Handels während des
Dunklen Zeitalters den Kleinbauern die Vorteile von Handel und größeren
Märkten. Der Niedergang der Städte untergrub die Geldwirtschaft, aber es
bedeutete auch, dass die ländliche Bevölkerung nicht mehr aufgerufen
war, die erdrückende Last der Bürokratie zu unterstützen. Wie Guy Bois
schrieb, war die römische Stadt eine parasitäre Gemeinschaft, kein
Zentrum der Produktion: „Im römischen Zeitalter war die dominante
Funktion einer Stadt politischer Art. Sie lebte hauptsächlich von den
Einnahmen, die sie durch die Landsteuer aus ihrer Umgebung abzweigten.
Die Stadt selbst produzierte wenig oder nichts zum Wohl der umliegenden
Landschaft.`` \footnote{Bois, ebenda, S. 78.} Der Zusammenbruch der
römischen Autorität befreite die Landwirte im Wesentlichen von Steuern,
die „zwischen einem Viertel und einem Drittel des Bruttoertrages des
Landes absaugten, ohne dabei die verschiedenen Eintreibungen zu
berücksichtigen, die kleine und mittlere Landbesitzer erlitten.``
\footnote{Ebenda, S. 118.} Die Steuern waren so belastend, dass sie
manchmal durch Hinrichtung durchgesetzt wurden und das Verlassen der
Eigentümer ihrer eigenen Grundstücke (Deserti agri) weit verbreitet war.
Die Barbaren ließen diese Steuern barmherzig verfallen.

\subsection{Deserti agri}\label{deserti-agri}

Die von einer Regierung verursachten Lasten wurden durch die
barbarischen Eroberungen so stark reduziert, dass eine Möglichkeit
geschaffen wurde, dass die Armen Grundbesitz erwerben und behalten
konnten. Einige der Deserti agri, beziehungsweise verlassenen
Bauernhöfe, die von Besitzern aufgrund von Raubsteuern in den letzten
Jahren des Römischen Reiches aufgegeben wurden, wurden wieder in die
Produktion gebracht. Trotz der rauen Umstände der Zeit und der Tatsache,
dass die Erträge nach heutigen Maßstäben lächerlich niedrig waren, waren
die dunklen Jahrhunderte eine Zeit relativer Prosperität für Europas
Kleinbauern. Tatsächlich waren sie in einer stärkeren Position, als sie
es bis zur modernen Epoche wieder sein würden. Einerseits waren weniger
Hände verfügbar, um das fruchtbare Land zu bestellen, da große Teile
davon aus dem Anbau genommen worden waren. Pest, Kriege und die Flucht
von Besitzern vor dem zusammenbrechenden Römischen Reich hatten Gebiete,
die zuvor unter Anbau standen, erheblich entvölkert. Ein weiterer
Vorteil, den Kleinbauern im Dunklen Zeitalter genossen, entstand aus der
Einführung neuer Landwirtschaftstechnologie im sechsten Jahrhundert: dem
schweren Pflug, der oft auf Rädern montiert wurde. In Kombination mit
einem verbesserten Geschirr, das es den Bauern erlaubte, mehrere Ochsen
einzusetzen, machte die neue Technologie das Roden von Waldland in
Nordeuropa wesentlich einfacher.\footnote{Gies, ebenda, S. 45.}

Unter solchen Bedingungen schrumpfte der Markt für
Landwirtschaftsflächen fast bis zur Unkenntlichkeit. Neues Land für den
Anbau konnte einfach durch Abholzung gewonnen und ein Teil jeder neuen
Parzelle mit den zuständigen örtlichen Behörden geteilt werden. Dieser
Prozess, bekannt als Rodung, ermöglichte nach dem Fall Roms
jahrhundertelang ein bequemes Bevölkerungswachstum. Mit den wärmeren
Temperaturen im achten Jahrhundert wurde die Landwirtschaft in den dünn
besiedelten nördlichen Regionen produktiver und das Roden besonders
attraktiv.

Die Anführer der germanischen Stämme, die ehemalige römische Gebiete
erobert hatten, hatten sich als Großgrundbesitzer etabliert. Der
Großteil der restlichen Bevölkerung bewirtschaftete kleine Flächen,
allerdings unter ganz anderen Bedingungen als später im Feudalismus. Die
wohlhabenderen Landbesitzer oder Meister stellten etwa 7-10 Prozent der
Bevölkerung. Es war wohl so, dass vor dem Jahr 1000 zwei Drittel der
Dorfbewohner in einem typischen Gebiet Frankreichs Grundbesitzer
waren.\footnote{Bois, ebenda, S. 116.} Sie besaßen etwa die Hälfte des
gesamten kultivierten Landes.\footnote{Ebenda, S. 26.} Leibeigene gab es
nur wenige. Coloni oder Pachtbauern machten nicht mehr als 5 Prozent der
Bevölkerung aus. Die Sklaverei gab es immer noch, aber in viel kleinerem
Maßstab als in der Römerzeit.

Die germanischen Königreiche in der Nachfolge Roms wurden militärisch
von allen freien Männern verteidigt, die dem Aufruf des örtlichen
Vertreters des Königs, des Grafen, nachkamen, um Waffen zu tragen.
Selbst von „kleinen und mittleren Grundbesitzern`` wurde erwartet, dass
sie sich zusammenschließen und einen aus ihrer Mitte zum Kampf in der
Infanterie senden.\footnote{Ebenda, S. 64.} Im Edikt von Pitres befahl
Karl der Kahlköpfige allen, die es sich leisten konnten, beritten zur
Schlacht zu erscheinen. Papst Gregor II. hatte ein Jahrhundert zuvor
versucht, dieses militärische Gebot voranzutreiben, indem er 732 den
menschlichen Verzehr von Pferdefleisch verbot.\footnote{Gies, ebenda, S.
  47.} Aber es gab bislang wenig Unterschied im Status oder Recht
zwischen der Infanterie der Freigrundbesitzer und der Kavallerie. Alle
freien Männer nahmen an lokalen Gerichtsversammlungen teil und konnten
den Grafen, ein Amt, das seit der spätrömischen Zeit existierte, um die
Beilegung von Streitigkeiten bitten. Es gab keinen Adel im eigentlichen
Sinne.

\begin{quote}
``Ein soziales Phänomen, das als Massenphänomen plötzlich neu am
Horizont in den 980ern auftauchte, war der soziale Abwärtstrend. Seine
ersten Opfer waren die kleinen Allod-Inhaber.`` \footnote{Bois, ebenda,
  S. 52.} - Guy Bois
\end{quote}

Im Verlauf des Dunklen Zeitalters kam es jedoch zu einigen
Entwicklungen, die die Beziehungen, die die Unabhängigkeit der freien
Bauern und Grundeigentümer in den germanischen Königreichen, die nach
dem Fall Roms an die Macht kamen, gewährleistet hatten,
destabilisierten:

\begin{enumerate}
\def\labelenumi{\arabic{enumi}.}
\item
  Die Bevölkerung erholte sich allmählich, was zu einem größeren Druck
  auf die Landnutzung führte. Über mehrere Jahrhunderte hinweg wurde
  viel von dem fruchtbarsten, noch unbeanspruchten Land, insbesondere in
  Nordeuropa, in Produktion gebracht. Die steigende Bevölkerungszahl der
  Bauern im Verhältnis zur verfügbaren Landmenge verringerte den Wert
  der einzelnen Landarbeiter. Die meisten Grundeigentumstitel wurden
  durch Erbschaft in immer kleinere Parzellen zerlegt. Während des
  Dunklen Zeitalters neigten die Kinder dazu, gleichberechtigt an den
  Nachlässen ihrer Eltern teilzuhaben. Die Fragmentierung des Besitzes
  zu Zeiten steigender Bevölkerungszahlen führte dazu, dass Land wieder
  eine kostbare Ressource wurde, und führte zur Wiederbelebung aktiver
  Landmärkte um die Mitte des zehnten Jahrhunderts.
\item
  In den letzten Jahrzehnten des zehnten Jahrhunderts wurde das Wetter
  plötzlich kälter, mit verheerenden Auswirkungen auf die
  landwirtschaftliche Produktion. Drei aufeinanderfolgende Missernten
  führten von 982 bis 984 zu einer schweren Hungersnot. Eine weitere
  Hungersnot folgte nach einer weiteren Missernte im Jahr
  994.\footnote{Ebenda, S. 150.} Dann verschlimmerte sich das Problem
  der sinkenden Ernteerträge im Jahr 997 durch eine Pestepidemie, die
  besonders kleine Familienbetriebe hart traf, da diesen die Ressourcen
  fehlten, um die Arbeitskraft verlorener Familienmitglieder zu
  ersetzen. Diese geballten Misserfolge und Katastrophen trieben die
  freien Bauern zunächst in die Verschuldung. Als sich die Erträge nicht
  erholten, konnten sie ihre Hypotheken nicht mehr bedienen.
\item
  Die Machtverhältnisse wurden zunehmend durch die wachsende Bedeutung
  der schweren Kavallerie destabilisiert. Die mittelalterliche
  Historikerin Frances Gies beschreibt die Transformation des
  gepanzerten Kavalleristen zum mittelalterlichen Ritter:
\end{enumerate}

\begin{quote}
Ursprünglich war der Ritter eine Persönlichkeit von mittlerem Status,
die durch sein teures Pferd und seine Rüstung emporgehoben wurde und
langsam seine Position in der Gesellschaft verbesserte, bis er Teil des
Adels wurde. Obwohl die Ritter die niedrigste Rangordnung der
Oberschicht bildeten, erwarb die Ritterschaft einen einzigartigen Ruf,
der die Erhebung in den Ritterstand zu einer vom Hochadel und sogar vom
Königtum begehrten Ehre machte. Dieser Ruf war hauptsächlich das Produkt
der Politik der Kirche, die Ritterschaft zu christianisieren, indem sie
die Zeremonie des Ritterschlags heiligte und einen Verhaltenskodex
unterstützte, der als Rittertum bekannt ist - ein Kodex, der vielleicht
öfter verletzt als geehrt wurde, aber einen unbestreitbaren Einfluss auf
das Denken und Verhalten der Nachwelt ausübte.\footnote{Gies, ebenda, S.
  2.}
\end{quote}

Wie wir in \emph{The Great Reckoning} erzählt haben, hat die Erfindung
des Steigbügels dem bewaffneten Ritter zu Pferd eine bemerkenswerte
Angriffsfähigkeit verliehen. Er konnte nun mit voller Geschwindigkeit
angreifen und wurde beim Aufprall seiner Lanze auf ein Ziel nicht aus
dem Sattel geworfen. Der militärische Wert der schweren Kavallerie wurde
durch eine asiatische Erfindung, dem genagelten Hufeisen, weiter erhöht,
das im zehnten Jahrhundert nach Westeuropa vordrang. Dies verbesserte
die Strapazierfähigkeit des Pferdes auf der Straße
zusätzlich.\footnote{Ebenda, S. 46.} Auch der Kontursattel, der den
Umgang mit schweren Waffen erleichterte, der Sporn und das
Kandarengebiss, mit dem ein Reiter das Pferd mit einer Hand lenken
konnte während er kämpfte, trugen zur erhöhten Wirksamkeit des
bewaffneten Ritters bei.\footnote{Ebenda, S. 56f.} Zusammen haben diese
augenscheinlich geringen technologischen Innovationen die militärische
Bedeutung der Kleinbauern stark abgewertet, die es sich nicht leisten
konnten, Kriegspferde zu halten und sich zu bewaffnen. Die
kostengünstigeren der speziell für den Krieg gezüchteten Pferde, die
großen Streitrosse, genannt Destrier, waren vier Ochsen oder vierzig
Schafe wert. Die teureren Kriegspferde kosteten zehn Ochsen oder hundert
Schafe. Rüstungen kosteten ebenfalls einen Betrag, den sich kein
Kleinbauer leisten konnte, vergleichbar mit einem Preis von sechzig
Schafen.\footnote{Ebenda, S. 58.}

\begin{enumerate}
\def\labelenumi{\arabic{enumi}.}
\setcounter{enumi}{3}
\tightlist
\item
  Die Tatsache, dass das kalte Wetter, Missernten, Hungersnöte und
  Seuchen im Vorfeld des Jahres 1000 auftraten, spielte ebenfalls eine
  Rolle für das Verhalten vieler Menschen, die davon überzeugt waren,
  dass das Ende der Welt oder die Wiederkunft nahe bevorstanden. Fromme
  oder verängstigte Groß- und Kleingrundbesitzer schenkten ihr Land der
  Kirche, um sich auf die Apokalypse vorzubereiten.
\end{enumerate}

\subsection{„Nur ein Armer verkauft
Land``}\label{nur-ein-armer-verkauft-land}

Die instabilen Bedingungen des späten zehnten Jahrhunderts ebneten den
Weg für die feudale Revolution. Häufige Missernten und Katastrophen
ließen die freien Kleinbauern in Schulden versinken. Als sich die
Ernteerträge nicht erholten, standen die Freibauern vor einer
verzweifelten Situation. Märkte üben immer den größten Druck auf die
schwächsten Teilnehmer aus. Tatsächlich ist das ein Teil ihrer Tugend.
Sie fördern die Effizienz, indem sie Vermögenswerte aus schwachen Händen
entfernen. Aber im Europa des späten zehnten Jahrhunderts war
subsistenzbasierte Landwirtschaft praktisch die einzige Beschäftigung.
Familien, die ihr Land verloren, verloren ihre einzige
Überlebensmöglichkeit. Angesichts dieser unappetitlichen Aussicht,
entschieden viele oder die meisten der Grundbesitzer während der
Feudalrevolution, ihre Felder wegzugeben. In den Worten von Guy Bois:
„Der einzige sichere Weg für einen Bauern, das Land, das er bearbeitete,
zu behalten, war es, das Eigentum daran der Kirche zu überlassen, damit
er seinen Nutzen daran behalten konnte.`` \footnote{Bois, ebenda, S. 87.}
Andere gaben einen Teil oder all ihr Land an wohlhabendere Bauern ab,
denen sie vertrauten, entweder freundliche Nachbarn oder Verwandte.

Diese Eigentumsübergaben wurden unter der Bedingung getätigt, dass der
Bauer, seine Familie und seine Nachkommen auf den Feldern zum Arbeiten
blieben. Die armen Bauern genossen auch die gegenseitige Unterstützung
der mächtigeren Besitzer, nun der „Adligen``, die sich Pferde und
Rüstungen leisten konnten und so die vergrößerten Ländereien schützten.
Ein solches Angebot kann aus Sicht des neuen Leibeigenen als Mittelweg
zwischen fortlaufendem wirtschaftlichem Besitz und Zwangsvollstreckung
angesehen werden. Meistens war es ein Angebot, das er nicht ablehnen
konnte.

Der Produktivitätsrückgang brachte die armen Bauern nicht nur in eine
verzweifelte wirtschaftliche Lage; er löste auch einen Anstieg der
räuberischen Gewalt aus, der die Sicherheit des Eigentums untergrub.
Diejenigen, die nicht über die Mittel verfügten, um sich einen Teil der
verfügbaren und unzureichenden Versorgung mit Pferden und Futter zu
sichern, sahen sich und ihren Besitz plötzlich nicht mehr in Sicherheit.
Um ihr Dilemma in heutigen Begriffen auszudrücken, wäre es so, als
würden Sie gezwungen, sich heute mit einer neuen Art von Waffe
auszustatten, aber die Kosten dafür wären 100.000 Dollar. Wenn Sie
diesen Preis nicht bezahlen können, wären Sie der Gnade derjenigen
ausgeliefert, die es können.

Innerhalb weniger Jahre brach die Fähigkeit des Königs und der Gerichte,
Ordnung durchzusetzen, zusammen.\footnote{Ebenda. Auch wenn die genaue
  Abfolge der Ereignisse während der feudalen Revolution aufgrund der
  spärlichen Aufzeichnungen schwer zu rekonstruieren ist, erscheint uns
  die von Guy Bois vorgeschlagene These in ihren Grundzügen als
  wahrscheinlich richtig. Sie ist nicht nur an sich plausibel, sondern
  ergibt auch aus den ansonsten anomalen Fakten einen Sinn und passt
  außerdem zu unseren Theorien.} Jeder mit Rüstung und Pferd konnte nun
sein eigenes Gesetz werden. Das Ergebnis war eine spätmittelalterliche
Version von Blade Runner, eine Auseinandersetzung und Plünderung, gegen
die die rechtmäßigen Behörden machtlos waren. Plünderungen und Angriffe
durch bewaffnete Ritter störten das Landleben. Es ist jedoch keineswegs
offensichtlich, dass alle Opfer dieser Plünderungen die Armen waren. Im
Gegenteil, die älteren, körperlich schwächeren oder schlecht
vorbereiteten unter den größeren Landbesitzern stellten attraktivere
Ziele dar. Sie hatten mehr zum Stehlen.

Es war kein Zufall, dass dies gerade in dem Moment geschah, als durch
kälteres Wetter, Hungersnöte und Pest ein knappes Angebot an Ressourcen
entstand. Die megapolitischen Bedingungen, die den Zusammenbruch der
Autorität begünstigten, hatten schon einige Zeit bestanden. Ihr
Potenzial, die Machtverhältnisse in der Gesellschaft zu verändern, wurde
jedoch erst realisiert, als eine Krise ausgelöst wurde. Missernten und
Hungersnöte scheinen genau das getan zu haben. Obwohl die genaue Abfolge
der Ereignisse schwer zu rekonstruieren ist, scheint es so, dass das
Plündern zumindest teilweise durch eine verzweifelte Lage ausgelöst
wurde. Einmal entfesselt, wurde offensichtlich, dass niemand die Kraft
mobilisieren konnte, um die Gewalt zu stoppen. Die überwiegende Mehrheit
der schlecht bewaffneten Bauern konnte sicherlich wenig tun. Selbst
Dutzende von Bauern zu Fuß wären von einem einzigen bewaffneten Ritter
zu Pferd übertroffen worden. Die Freibauern, ebenso wie die bestehenden
Autoritäten, die Könige mit ihren Grafen, waren machtlos, um zu
verhindern, dass lokales Land von bewaffneten Kriegern beschlagnahmt
wurde.

\subsection{„Der Friede Gottes``}\label{der-friede-gottes}

In diesen verzweifelten Bedingungen half die Kirche, den Feudalismus zu
starten, durch ihre Bemühungen, einen Waffenstillstand auf dem
gewalttätigen Land zu verhandeln. Der Historiker Guy Bois beschrieb die
Situation auf diese Weise: „Die Ohnmacht der politischen Behörden war so
groß, dass die Kirche einsprang, um den Versuch zur Wiederherstellung
der Ordnung zu unterstützen, die in einer Bewegung als ‚Der Friede
Gottes' bekannt wurde. ‚Friedensräte' verhängten eine Reihe von
Verboten, die mit Bannflüchen sanktioniert wurden; große
‚Friedensversammlungen' nahmen die Eide der Krieger entgegen. Die
Bewegung hatte ihren Ursprung im Süden Frankreichs (Rat von Charroux im
Jahr 989, Rat von Narbonne im Jahr 990), bevor sie sich allmählich
ausbreitete...`` \footnote{Ebenda, S. 136.}

Der Kompromiss, den die Kirche einging, beinhaltete die Anerkennung der
Oberherrschaft bewaffneter Ritter in lokalen Gemeinschaften im Austausch
gegen ein Ende oder eine Mäßigung der Gewalt und des Plünderns.
Grundbuchtitel, die nach der Welle der Gewalt im späten zehnten
Jahrhundert plötzlich verfasst wurden, trugen plötzlich den Titel
„nobilis`` oder „miles`` als Zeichen der Herrschaft. Der Adel als eine
separate Sphäre wurde durch die feudale Revolution geschaffen.
Eigentumshandel, der ein paar Jahre zuvor den gleichen Personen
zugeordnet war, hatte eine solche Unterscheidung nicht
aufgelistet.\footnote{Ebenda, S. 57ff und passim.}

Angesichts der sinkenden Produktivität und der wirtschaftlichen
Unsicherheit der Kleinbauern führte die megapolitische Macht der
bewaffneten Ritter unweigerlich zu Eigentumsbesitz durch feudale Pacht.
Bis zum Ende des ersten Viertels des elften Jahrhunderts waren die
freien Bauern weitgehend verschwunden. Ihre freien Ländereien betrugen
nur noch einen Bruchteil ihrer früheren Größe und wurden nun nur noch in
Teilzeit bewirtschaftet. Die kleinen Bauern oder ihre Nachkommen waren
Leibeigene, die den Großteil ihrer Zeit damit verbrachten, auf den
Anwesen der feudalen und kirchlichen Herren zu arbeiten.

Das Zerfallen der Ordnung, die die feudale Revolution begleitete, führte
zu Anpassungen im Verhalten, die den Feudalismus stärkten. Darunter ein
Anstieg im Burgenbau. Burgen waren im Gefolge von Wikingerüberfällen im
9. Jahrhundert erstmals als primitive hölzerne Strukturen in
Nordwesteuropa aufgetaucht. Ursprünglich waren sie Kommandozentren für
karolingische Beamte, nach der feudalen Revolution wurden sie jedoch zu
Erbbesitz. Diese frühen Festungen waren weit primitiver als sie später
werden würden, aber sie waren dennoch schwer anzugreifen. Einmal
errichtet, konnten Burgen nur mit größter Mühe niedergebrannt werden.
Mit der Ausbreitung der Burgen auf dem Land wurde es immer
unwahrscheinlicher, dass der König oder seine Grafen die lokale
Vorherrschaft der Herren wirksam angreifen konnten.

\subsection{Beiträge der Kirche zur
Produktivität}\label{beitruxe4ge-der-kirche-zur-produktivituxe4t}

Feudalismus war die Reaktion der Agrargesellschaft auf den Zusammenbruch
der Ordnung in Zeiten niedriger Produktivität. Während der frühen
Stadien des Feudalismus spielte die Kirche eine wichtige und
wirtschaftlich produktive Rolle. Zu den Beiträgen der Kirche gehörten:

\begin{enumerate}
\def\labelenumi{\arabic{enumi}.}
\tightlist
\item
  In einem Umfeld, in dem die militärische Macht dezentralisiert war,
  war die Kirche in einzigartiger Weise in der Lage, den Frieden
  aufrechtzuerhalten und Ordnungsregeln zu entwickeln, die über
  zersplitterte, lokale Souveränitäten hinausgingen. Dies ist eine
  Aufgabe, die keine weltliche Macht zu erfüllen vermochte. Die
  Beobachtungen der großen religiösen Autorität A. R. Radcliffe-Brown
  sind hier unmittelbar relevant. Er wies darauf hin, dass „die soziale
  Funktion einer Religion unabhängig von ihrer Wahrheit oder Falschheit
  ist.`` Selbst solche, die „absurd und abstoßend sind, wie die einiger
  wilder Stämme, können wichtige und wirksame Teile der sozialen
  Maschinerie sein.`` \footnote{A. R. Radcliffe-Brown, \emph{Religion
    and Society} in \emph{Structure and Function in Primitive Society}
    (London: Cohen \& West, 1952), S. 153ff.} Dies war bei der Kirche in
  der Frühphase des Feudalismus sicherlich der Fall. Sie trug dazu bei,
  Regeln zu schaffen, wie es nur eine Religion konnte, die es den
  Menschen ermöglichte, Anreizfallen und Verhaltensdilemmata zu
  überwinden. Einige dieser Dilemmas waren moralische Dilemmas, die für
  das gesamte menschliche Leben gelten. Bei anderen handelte es sich um
  lokale Dilemmata, die nur unter den vorherrschenden megapolitischen
  Bedingungen auftraten. Der mittelalterlichen Kirche kam bei der
  Wiederherstellung der Ordnung auf dem Lande in den letzten Jahren des
  zehnten Jahrhunderts eine besondere Rolle zu. Indem sie den lokalen
  Behörden religiöse und zeremonielle Unterstützung gewährte, senkte die
  Kirche die Kosten für die Errichtung zumindest schwacher lokaler
  Gewaltmonopole. Indem sie auf diese Weise half, Ordnung zu schaffen,
  trug die Kirche zu den Bedingungen bei, die letztlich zu stabileren
  Machtkonstellationen führten.
\end{enumerate}

Die Kirche spielte auch eine lange Zeit danach noch eine Rolle bei der
Eindämmung der privaten Kriege und Ausschreitungen, die sonst von den
weltlichen Behörden nicht in den Griff zu bekommen waren. Die relative
Bedeutung der Kirche im Vergleich zu weltlichen Behörden spiegelt sich
in der Tatsache wider, dass bis zum elften Jahrhundert die wichtigste
administrative Einheit in den meisten Teilen Westeuropas die Pfarrei und
nicht länger die alten zivilen Einheiten, der Ager und Pagus (Stadt),
die seit der Römerzeit bis ins dunkle Mittelalter fortbestanden hatten,
waren.\footnote{Bois, ebenda, S. 36.}

\begin{enumerate}
\def\labelenumi{\arabic{enumi}.}
\setcounter{enumi}{1}
\item
  Die Kirche war die Hauptquelle für den Erhalt und die Vermittlung
  technischen Wissens und Informationen. Die Kirche unterstützte
  Universitäten und bot die minimale Bildung, die die mittelalterliche
  Gesellschaft genoss. Die Kirche stellte auch einen Mechanismus zum
  Vervielfältigen von Büchern und Handschriften zur Verfügung,
  einschließlich fast aller zeitgenössischen Informationen über
  Landwirtschaft und Tierhaltung. Die Skriptorien der
  Benediktinerklöster können als eine alternative Technologie zu
  Druckerpressen verstanden werden, die noch nicht existierten. So
  kostspielig und ineffizient die Skriptorien auch waren, sie waren
  praktisch der einzige Mechanismus zur Vervielfältigung und Bewahrung
  von schriftlichem Wissen in der feudalen Periode.
\item
  Teilweise, weil ihre Hofverwalter lesen konnten, leistete die Kirche
  einen großen Beitrag zur Steigerung der Produktivität der europäischen
  Landwirtschaft, insbesondere in den frühen Stufen des Feudalismus. Vor
  dem dreizehnten Jahrhundert konnten fast alle Hofverwalter von
  Laienherren, die fast alle Analphabeten waren, sich nur durch ein
  ausgeklügeltes Markierungssystem Notizen machen. Auch wenn sie
  vielleicht geschickte Landwirte waren, waren sie nicht in der Lage,
  von Verbesserungen in den Produktionsmethoden zu profitieren, die sie
  nicht selbst erfinden oder mit eigenen Augen sehen konnten. Die Kirche
  war daher unerlässlich für die Verbesserung der Qualität von Getreide,
  Früchten und Zuchtbetrieben. Aufgrund ihres ausgedehnten Besitzes, der
  sich über den gesamten europäischen Kontinent erstreckte, konnte die
  Kirche das produktivste Saatgut und den besten Zuchtbestand in Gebiete
  schicken, in denen die Produktion hinterherhinkte. Der Bedarf an
  sakramentalem Wein im Norden Europas veranlasste die Mönche, mit
  robusten Traubensorten zu experimentieren, die in kälteren Klimazonen
  überleben konnten. Die Kirche half auch auf andere Weise, die
  Produktivität der mittelalterlichen Landwirtschaft zu erhöhen. Viele
  der der Kirche während der feudalen Revolution gespendeten
  wirtschaftlich kleinen Parzellen wurden umgestaltet, um sie leichter
  bewirtschaften zu können. Die Kirche bot auch zusätzliche
  Dienstleistungen an, die kleine landwirtschaftliche Gemeinschaften
  benötigten. In vielen Gebieten mahlten kircheneigene Mühlen Getreide
  zu Mehl.
\item
  Die Kirche übernahm viele Funktionen, die heute von der Regierung
  ausgeführt werden, einschließlich der Bereitstellung öffentlicher
  Infrastruktur. Dies war ein Teil der Art und Weise, wie die Kirche
  dazu beigetragen hat, das zu überwinden, was
  Wirtschaftswissenschaftler „Die Tragik der Allmende`` in einer Zeit
  fragmentierter Autorität nennen. Spezifische religiöse Orden der
  frühmittelalterlichen Kirche widmeten sich angewandten technischen
  Aufgaben, wie dem Öffnen von Straßen, dem Wiederaufbau gefallener
  Brücken und der Reparatur verfallener römischer Aquädukte. Sie rodeten
  auch Land, bauten Dämme und entwässerten Sümpfe. Ein neuer
  klösterlicher Orden, die Kartäuser, bohrte den ersten „artesischen``
  Brunnen in Artois, Frankreich. Mit Hilfe der Schlagbohrtechnik gruben
  sie ein kleines Loch tief genug, um einen Brunnen zu schaffen, der
  keine Pumpe benötigte.\footnote{Gies, ebenda, S. 112.} Die
  Zisterzienser übernahmen den Bau und die Wartung von unsicheren
  Seemauern und Deichen in den Tieflandgebieten Europas. Bauern
  überschrieben Land an Zisterzienserklöster und mieteten es zurück,
  während die Mönche die volle Verantwortung für die Instandhaltung und
  Reparatur übernahmen. Die Zisterzienser waren auch führend bei der
  Entwicklung wasserbetriebener Maschinen, die für so verbreitete Zwecke
  wie „Stampfen, Heben, Mahlen und Pressen`` eingesetzt
  wurden.\footnote{Ebenda, S. 114.} Das Kloster von Clairvaux grub einen
  zwei Meilen langen Kanal vom Fluss Aube aus.\footnote{Ebenda, S. 117.}
  Die Kirche intervenierte auch, um neue Straßen und Brücken zu bauen,
  wo sich die Bevölkerungszentren außerhalb des Bereichs der alten
  römischen Garnisonsstraßen verlagert hatten. Bischöfe gewährten
  Ablässe an lokale Herren, die Flussübergänge bauen oder reparieren und
  Hospize für Reisende unterhalten würden. Ein Orden von Mönchen, der
  von St.~Benezet gegründet wurde, die Freres Pontifes, oder „Brüder der
  Brücke``, bauten einige der längsten Brücken, die es damals gab,
  einschließlich der Pont d\textquotesingle Avignon, eine massive
  zwanzigbogige Struktur über die Rhone mit einer kombinierten Kapelle
  und Mautstelle an einem Ende. Sogar die Londoner Brücke, die bis ins
  neunzehnte Jahrhundert stand, wurde von einem Kaplan gebaut und
  teilweise durch einen Beitrag von 1.000 Mark vom päpstlichen Legaten
  finanziert.\footnote{Die Details über Brücken und Infrastruktur stammt
    hauptsächlich aus ebenda, S. 148ff.}
\item
  Die Kirche half auch bei der Entstehung eines komplexeren Marktes. Der
  Bau von Kathedralen unterscheidet sich beispielsweise in seiner Art
  von der öffentlichen Infrastruktur, wie Brücken und Aquädukten. Im
  Prinzip wurden Kirchenstrukturen zumeist nur für religiöse Dienste
  genutzt und nicht als Durchgangsstraßen für den Handel. Dennoch sollte
  nicht vergessen werden, dass der Bau von Kirchen und Kathedralen half,
  Märkte für viele handwerkliche und ingenieurtechnische Fähigkeiten zu
  schaffen und zu vertiefen. Genauso wie die Militärausgaben des
  Nationalstaates während des Kalten Krieges unbeabsichtigt zur
  Entstehung des Internets beitrugen, so führte der Bau der
  mittelalterlichen Kathedralen zu anderen Auswirkungen, der Inkubation
  des Handels. Die Kirche war ein Hauptkunde der Bauhandwerke und
  Handwerkskunst. Kirchliche Käufe von Silber für Abendmahlsdienste,
  Leuchter und Kunstwerke zur Dekoration von Kirchen halfen, einen Markt
  für Luxusgüter zu schaffen, der sonst nicht existiert hätte.
\end{enumerate}

In vielerlei Hinsicht half die Kirche dabei, die Heftigkeit der von
bewaffneten Rittern während und nach der „feudalen Revolution``
entfesselten Gewalt zu mildern. Vor allem in den frühen Jahrhunderten
des Feudalismus trug die Kirche erheblich dazu bei, die Produktivität
der Agrarökonomie zu verbessern. Sie war eine essenzielle Institution,
perfekt angepasst an die Bedürfnisse der Agrargesellschaft am Ende des
Dunklen Zeitalters.

\subsection{Verletzlichkeit gegenüber
Gewalt}\label{verletzlichkeit-gegenuxfcber-gewalt}

In „dreißig oder vierzig Jahren heftiger Unruhen war die
Feudalrevolution des Jahres 1000`` \footnote{Bois, ebenda, S. 136.},
ebenso wie der Fall Roms fünf Jahrhunderte zuvor, ein einzigartiges
Ereignis, verursacht durch ein komplexes Zusammenspiel von Einflüssen.
Doch in einer Hinsicht spiegeln der Triumph von bösen Männern und die
von ihnen verursachten Unterdrückungen die grundsätzliche Verwundbarkeit
der Agrargesellschaft gegenüber Gewalt perfekt wider. Im Gegensatz zur
Phase der menschlichen Existenz, in der das Leben vom Sammeln und Jagen
geprägt war, führte die Landwirtschaft zu einem Quantensprung in
organisierter Gewalt und Unterdrückung.

Dies spiegelte sich schon in den frühesten Zeiten in den kämpferischeren
Kulturen der Bauernvölker wider. Die Götter der frühen
landwirtschaftlichen Gesellschaften waren Regen- und Flutgötter, deren
Funktionen das Interesse dieser Gesellschaften an Faktoren
widerspiegelten, die den Ernteertrag bestimmten. Der Bringer von Regen
oder Wasser war oft auch der Kriegsgott, den die ersten Könige, die vor
allem Kriegsherren waren, anriefen.\footnote{Siehe Norman Cohn,
  \emph{Cosmos, Chaos, and the World to Come: The Ancient Roots of the
  Apocalyptic Faith} (New Haven: Yale University Press, 1993), Kapitel.
  1-3, insbesondere S. 60.}

Die enge Verbindung zwischen Landwirtschaft und Kriegsführung spiegelte
sich in der religiösen Vorstellungskraft von Menschen wider, deren Leben
durch die Innovationen der Agrarrevolution verwandelt wurde. Die
Vertreibung aus dem Garten Eden kann als symbolische Darstellung der
gesellschaftlichen Transformation von der Nahrungssuche zur
Landwirtschaft gesehen werden, von einem freien Leben mit Nahrung, die
mit wenig Aufwand aus der Fülle der Natur gepflückt wurde, zu einem
Leben mit harter Arbeit.

\section{DAS VERLORENE PARADIES}\label{das-verlorene-paradies}

Die Landwirtschaft hat die Menschheit auf einen völlig neuen Kurs
gebracht. Die ersten Bauern pflanzten tatsächlich die Samen der
Zivilisation. Aus ihrer harten Arbeit entstanden Städte, Armeen,
Arithmetik, Astronomie, Verliese, Wein und Whisky, das geschriebene
Wort, Könige, Sklaverei und Krieg. Trotz aller Dramatik, die die
Landwirtschaft dem Leben verleihen sollte, scheint der Abschied von der
urzeitlichen Wirtschaft von Anfang an rundweg unbeliebt gewesen zu sein.
Zeugnis dafür gibt die Erzählung, die im Buch Genesis erhalten ist,
welche die Verstoßung aus dem Paradies erzählt. Die biblische Parabel
vom Garten Eden ist eine liebevolle Erinnerung an das bequeme Leben des
Sammlers in der Wildnis. Gelehrte deuten darauf hin, dass das Wort
„Eden`` aus einem sumerischen Wort für „Wildnis`` abgeleitet zu sein
scheint.\footnote{Bruce Ni. Metzger und Michael D. Coogan, eds.,
  \emph{The Oxford Companion to the Bible} (Oxford: Oxford University
  Press, 1993), p.178.}

Der Übergang vom freien und spärlich besiedelten Leben in der Wildnis zu
einem sesshaften Leben in einem landwirtschaftlichen Dorf war Thema
tiefen Bedauerns, das nicht nur in der Bibel zum Ausdruck kam, sondern
sich auch in der anhaltenden Verärgerung der Menschheit zeigt, morgens
aufzustehen und zur Arbeit zu gehen. Wie Stephen Boyden in „Western
Civilization in Biological Perspective`` schrieb, war die neue Art zu
leben, die mit der Landwirtschaft einherging, „evodeviant`` \footnote{Boyden,
  ebenda, S. 118.}. Vor dem Beginn der Landwirtschaft lebten Tausende
von menschlichen Generationen wie Adam im Garten Eden, eingeladen von
seinem Schöpfer: „Von jedem Baum des Gartens darfst du einfach essen.``
Jäger und Sammler hatten keine Felder zu pflegen, keine Herde zu hüten,
keine Steuern zu zahlen. Wie Landstreicher gingen die Jäger und Sammler,
wohin sie wollten, arbeiteten wenig und waren niemandem Rechenschaft
schuldig.

Mit der Landwirtschaft begann eine neue Lebensweise, und zwar unter ganz
anderen Bedingungen. „Dornen und Disteln lässt sie auch für dich
wachsen; und du musst die Pflanzen des Feldes essen; Mit Schweiß auf
deinem Gesicht wirst du Brot essen.`` Die Landwirtschaft war harte
Arbeit. Die Erinnerung an das Leben vor dem Ackerbau war die eines
verlorenen Paradieses.

Viel mehr als sie es sich hätten vorstellen können, haben Bauern neue
Bedingungen geschaffen, die die Logik der Gewalt drastisch veränderten.
Es ist kein Zufall, dass das Buch Genesis Kain, den ersten Mörder, als
„einen Bearbeiter des Bodens`` darstellt. Tatsächlich ist es Teil der
unheimlichen prophetischen Kraft der Bibel, dass ihre Geschichte Hirten
anvertraut wurde, die sofort verstanden, wie die Landwirtschaft der
Gewalt Auftrieb gab. In wenigen Versen fasst die biblische Erzählung
Logiken zusammen, die Tausende von Jahren zur Umsetzung brauchten. Die
Landwirtschaft war ein Brutkasten für Streitigkeiten. Sie schuf
immobilisiertes Kapital in großem Maßstab, steigerte dadurch den Gewinn
aus Gewalt und erhöhte dramatisch die Herausforderung, Vermögenswerte zu
schützen. Die Landwirtschaft machte sowohl Kriminalität als auch
Regierungen zum ersten Mal zu lohnenden Unternehmungen.

\setsubtitle{Parallelen zwischen dem altersschwachen Niedergang der Heiligen Mutter Kirche und dem Nanny-Staat}

\bookmarksetup{startatroot}

\chapter{DIE LETZTEN TAGE DER
POLITIK}\label{die-letzten-tage-der-politik}

\begin{quote}
„Ich glaube auch - und hoffe, dass Politik und Wirtschaft in Zukunft
nicht mehr so wichtig sein werden wie in der Vergangenheit; es wird die
Zeit kommen, in der die meisten unserer heutigen Kontroversen über diese
Themen so trivial und bedeutungslos erscheinen werden wie die
theologischen Debatten, in denen die eifrigsten Köpfe des Mittelalters
ihre Energien vergeudeten.`` \footnote{Clarke, ebenda, S. 9.} Arthur C.
Clarke
\end{quote}

Vom bevorstehenden Tod der Politik zu sprechen, kann je nach Veranlagung
lächerlich oder optimistisch erscheinen. Doch genau das ist es, was die
Informationsrevolution wahrscheinlich bringen wird. Für Leser, die in
einem politikdurchtränkten Jahrhundert aufgewachsen sind, mag die
Vorstellung, dass das Leben ohne Politik ablaufen könnte, wie ein
Hirngespinst erscheinen, so als würde man behaupten, man könne leben,
indem man nur Nährstoffe aus der Luft aufnimmt. Doch die Politik im
modernen Sinne, also die Beschäftigung mit der Kontrolle und
Rationalisierung der Staatsgewalt, ist größtenteils eine moderne
Erfindung. Wir glauben, dass sie mit der modernen Welt enden wird, so
wie das Gewirr von feudalen Pflichten und Verpflichtungen, das die
Aufmerksamkeit der Menschen im Mittelalter in Anspruch nahm, mit dem
Mittelalter endete. Während der Feudalzeit, so der Historiker Martin van
Creveld, „gab es keine Politik (der Begriff selbst musste erst noch
erfunden werden und geht erst auf das sechzehnte Jahrhundert
zurück)``.\footnote{Martin van Creveld, \emph{The Transformation of War}
  (New York: The Free Press, 1991), S. 52.}

Der Gedanke, dass die Politik, wie wir sie heute kennen, vor der Neuzeit
nicht existierte, mag überraschen, vor allem wenn man bedenkt, dass
Aristoteles zu Zeiten Alexanders des Großen einen Aufsatz mit diesem
Titel verfasst hatte. Aber sehen Sie genau hin. Wörter, die in antiken
Texten verwendet werden, sind nicht unbedingt zeitgenössische Konzepte.
Aristoteles schrieb auch eine Abhandlung mit dem Titel Sophistische
Widerlegungen, ein Begriff, der heute ungefähr so bedeutungslos ist wie
Politik im Mittelalter. Das Wort war einfach nicht in Gebrauch. Es ist
im englischen Wortschatz das erste Mal im Jahr 1529
aufgetaucht.\footnote{\emph{The Compact Edition of The Oxford English
  Dictionary}, ebenda, S. 1074.} Schon damals schien „Politik``
abwertend gemeint gewesen zu sein, abgeleitet von dem altfranzösischen
Wort politique, mit dem man „Opportunisten und Rückgratlose``
bezeichnete.\footnote{Siehe T.C. Onions, ed., \emph{The Oxford
  Dictionary of English Etymology} (Oxford: Oxford University Press,
  1966), S. 693.}

Es dauerte fast zweitausend Jahre, bis Aristoteles' latentes Konzept mit
der Bedeutung auftauchte, die wir heute kennen. Warum? Bevor die moderne
Welt Aristoteles' Wort sinnvoll nutzen konnte, waren megapolitische
Bedingungen erforderlich, die den Ertrag von Gewalt dramatisch erhöhten.
Die Schießpulverrevolution, die wir in The Great Reckoning analysiert
haben, hat genau das bewirkt. Sie steigerte die Rendite von Gewalt weit
über das hinaus, was sie jemals gewesen war. Dadurch wurde die Frage,
wer den Staat kontrolliert, wichtiger als je zuvor. Logischerweise und
unweigerlich entstand die Politik aus dem Kampf um die Kontrolle der
stark gestiegenen Beute der Macht.

Die Politik begann vor fünf Jahrhunderten mit der Frühphase der
Industrialisierung. Jetzt liegt sie im Sterben. Eine weit verbreitete
Abneigung gegen Politik und Politiker durchzieht die Welt. Man sieht es
an den Nachrichten und Spekulationen über die verborgenen Details von
Whitewater und den schlecht getarnten Mord an Vincent Foster. Sie sehen
es in zahlreichen anderen Skandalen, die Präsident Bill Clinton
betreffen. Sie sehen es in Berichten über die Veruntreuung von Geldern
aus dem Postamt durch führende Kongressabgeordnete. Sie sehen es in
Skandalen, die zu Rücktritten im Umfeld von John Major führten, und in
ähnlichen Skandalen in Frankreich, die zwei aktuelle Premierminister,
Eduard Balladur und Alain Juppe, betrafen. Noch größere Skandale wurden
in Italien aufgedeckt, wo der siebenmalige Ministerpräsident Giulio
Andreotti auf die Anklagebank gebracht wurde, um sich unter anderem
wegen Verbindungen zur Mafia und der Anordnung des Mordes an Mino
Pecorelli, einem Enthüllungsjournalisten, vor Gericht zu verantworten.
Weitere Skandale haben den Ruf des spanischen Premierministers Filipe
Gonzales beschädigt. Korruptionsvorwürfe kosteten vier japanische
Premierminister in den ersten fünf Jahren der 1990er Jahre ihr Amt. Das
kanadische Justizministerium beschuldigte in einem Schreiben an die
Schweizer Behörden den ehemaligen Premierminister Brian Mulroney, beim
Verkauf von Airbus-Flugzeugen an Air Canada im Wert von 1,8 Milliarden
Dollar Schmiergelder erhalten zu haben.\footnote{John Urquhart,
  \emph{Former Premier Sues Canada for Libel in Probe of Alleged Airbus
  Kickbacks}, Wall Street Journal, 21. November 1995, S. A11.} Willy
Claes, der Generalsekretär der NATO, wurde aufgrund von
Korruptionsvorwürfen zum Rücktritt gezwungen. Selbst in Schweden musste
Mona Sahlm, stellvertretende Ministerpräsidentin und voraussichtliche
Ministerpräsidentin, zurücktreten, nachdem ihr vorgeworfen wurde, mit
den Kreditkarten der Regierung Windeln und andere Haushaltswaren gekauft
zu haben. Fast überall, wo man sich in Ländern mit reifen
Wohlfahrtsstaaten, die einst als gut regiert galten, hinwendet, hassen
die Menschen ihre politischen Führer.

\subsection{Verachtung als
Frühindikator}\label{verachtung-als-fruxfchindikator}

Moralische Empörung gegen korrupte Führer ist kein isoliertes
historisches Phänomen, sondern ein häufiger Vorläufer des Wandels. Sie
tritt immer wieder auf, wenn eine Epoche einer anderen weicht. Immer
dann, wenn der technologische Wandel die alten Formen von den neuen
bewegenden Kräften der Wirtschaft abgekoppelt hat, verschieben sich die
moralischen Maßstäbe, und die Menschen beginnen, diejenigen, die die
alten Institutionen leiten, mit wachsender Verachtung zu begegnen. Diese
weit verbreitete Abscheu zeigt sich, lange bevor die Menschen eine neue
kohärente Ideologie des Wandels entwickeln. Während wir schreiben, gibt
es noch kaum Anzeichen für eine artikulierte Ablehnung der Politik. Das
wird später kommen. Die meisten ihrer Zeitgenossen sind noch nicht auf
die Idee gekommen, dass ein Leben ohne Politik möglich ist. Was wir in
den letzten Jahren des zwanzigsten Jahrhunderts haben, ist
unartikulierte Verachtung.

Etwas Ähnliches geschah im späten fünfzehnten Jahrhundert, aber damals
war es eher die Religion als die Politik, die sich im Prozess der
Verkleinerung befand. Trotz des weit verbreiteten Glaubens an die
„Heiligkeit des geistlichen Amtes`` \footnote{Huizinga, ebenda, S.172.}
wurden sowohl die höheren als auch die niederen Ränge des Klerus in
höchstem Maße verachtet, nicht unähnlich der Haltung der Bevölkerung
gegenüber Politikern und Bürokraten heute. Es wurde weithin angenommen,
dass der höhere Klerus korrupt, weltlich und käuflich sei. Und das nicht
ohne Grund. Mehrere Päpste des fünfzehnten Jahrhunderts prahlten offen
mit ihren unehelichen Kindern. Der niedere Klerus genoss ein noch
geringeres Ansehen, denn er wucherte auf dem Land und in der Stadt,
bettelte um Almosen und bot häufig an, Gottes Gnade und die Vergebung
der Sünden an jeden zu verkaufen, der dafür Geld hinblätterte.

Unter der „Kruste der oberflächlichen Frömmigkeit`` \footnote{Ebenda, S.
  150.} befand sich ein korruptes und immer schlechter funktionierendes
System. Viele verloren den Respekt vor denen, die es leiteten, lange
bevor jemand sich traute offen zu sagen, dass es nicht funktionierte.
Ein von Religion durchdrungenes Leben, das keinen Unterschied zwischen
dem Geistlichen und dem Zeitlichen machte, hatte seine Möglichkeiten
erschöpft. Sein Ende war unvermeidlich, lange bevor Luther seine 95
Thesen an die Kirchentür in Wittenberg nagelte.

\section{EINE SÄKULARE REFORMATION}\label{eine-suxe4kulare-reformation}

Wir glauben, dass die Reaktion gegen die Sättigungspolitik einen
ähnlichen Weg einschlägt.

Der Untergang der Sowjetunion und die Ablehnung des Sozialismus sind
Teil eines umfassenden Musters der Entpolitisierung, das die Welt
durchzieht. Dies zeigt sich jetzt am deutlichsten in einer wachsenden
Verachtung für diejenigen, die die Regierungen der Welt führen. Sie wird
nur zum Teil von der Erkenntnis angetrieben, dass sie korrupt sind und
dazu neigen, „Ablässe`` aus politischen Schwierigkeiten im Austausch für
Wahlkampfspenden oder besondere Hilfe bei Rohstoffgeschäften zu
verkaufen, um ihre persönlichen Finanzen aufzubessern.

Die Reaktion gegen Politiker ist auch durch die zunehmende Erkenntnis
motiviert, dass vieles von dem, was sie mit großem Aufwand tun, sinnlos
ist, so wie die Organisation einer weiteren Bußwallfahrt, bei der die
Menschen barfuß durch den Schnee laufen, oder die Gründung eines
weiteren Bettelmönchsordens im späten 15. Jahrhundert wenig dazu
beigetragen hätte, die Produktivität zu erhöhen oder den Druck auf den
Lebensstandard zu erleichtern.

\subsection{Die letzten Tage der Heiligen Mutter
Kirche}\label{die-letzten-tage-der-heiligen-mutter-kirche}

Am Ende des Mittelalters war die monolithische Kirche als Institution
altersschwach und kontraproduktiv geworden - ein deutlicher Unterschied
zu ihrem positiven wirtschaftlichen Beitrag fünf Jahrhunderte zuvor. Wie
wir im letzten Kapitel untersucht haben, spielte die Kirche Ende des
zehnten Jahrhunderts eine führende Rolle bei der Herstellung von Ordnung
und der Erleichterung des wirtschaftlichen Aufschwungs nach der
Anarchie, die das Ende des Dunklen Zeitalters gekennzeichnet hatte. Zu
dieser Zeit war die Kirche für das Überleben einer großen Zahl von
kleinen Grundbesitzern und Leibeigenen, die den Großteil der
westeuropäischen Bevölkerung ausmachten, unverzichtbar. Am Ende des
fünfzehnten Jahrhunderts war die Kirche zu einem großen Hemmschuh für
die Produktivität geworden. Die Lasten, die sie der Bevölkerung
aufbürdete, drückten den Lebensstandard nach unten.

Das Gleiche kann man heute über den Nationalstaat sagen. Er war eine
notwendige Anpassung an die neuen megapolitischen Bedingungen, die durch
die Schießpulverrevolution vor fünf Jahrhunderten geschaffen wurden. Der
Nationalstaat erweiterte die Reichweite der Märkte und verdrängte die
zersplitterten lokalen Autoritäten zu einer Zeit, als umfassendere
Handelsgebiete große Gewinne brachten. Die Tatsache, dass sich die
Kaufleute fast überall in Europa spontan mit dem Monarchen im Zentrum
verbündeten, als dieser versuchte, seine Autorität zu festigen, ist an
sich schon ein deutlicher Beweis dafür, dass der Nationalstaat in seiner
frühen Form gut für das Geschäft war. Er trug dazu bei, den Handel von
den Lasten zu befreien, die ihm von Feudalherren und lokalen Magnaten
aufgebürdet wurden.

In einer Welt, in der die Renditen für Gewalt hoch waren und weiter
stiegen, war der Nationalstaat eine nützliche Institution. Doch fünf
Jahrhunderte später, am Ende dieses Jahrtausends, haben sich die
megapolitischen Bedingungen geändert. Die Gewaltrenditen sinken, und der
Nationalstaat ist, wie die Kirche in der Dämmerung des Mittelalters, ein
Anachronismus, der zu einem Hemmschuh für Wachstum und Produktivität
geworden ist.

Wie die Kirche damals hat auch der Nationalstaat heute seine
Möglichkeiten ausgeschöpft. Er ist bankrott, eine Institution, die sich
zu einem senilen Extrem entwickelt hat. Wie die Kirche damals war er
fünf Jahrhunderte lang die vorherrschende Form der sozialen
Organisation. Da er die Bedingungen, die ihn ins Leben gerufen haben,
überlebt hat, ist er reif für den Untergang. Und sie wird fallen. Die
Technologie ist dabei, eine Revolution in der Machtausübung auszulösen,
die den Nationalstaat ebenso sicher zerstören wird, wie das Schießpulver
und der Buchdruck das Monopol der mittelalterlichen Kirche zerstört
haben.

Wenn unsere Überlegungen richtig sind, wird der Nationalstaat durch neue
Formen der Souveränität ersetzt werden, von denen einige in der
Geschichte einzigartig sind, andere an die Stadtstaaten und
mittelalterlichen Handelsrepubliken der vormodernen Welt erinnern. Was
alt war, wird nach dem Jahr 2000 neu sein. Und was unvorstellbar war,
wird alltäglich sein. In dem Maße, in dem die Technologie an Bedeutung
gewinnt, werden die Regierungen feststellen, dass sie wie Unternehmen um
Einkommen konkurrieren müssen und für ihre Dienstleistungen nicht mehr
verlangen dürfen, als sie den Menschen, die dafür bezahlen, wert sind.
Die vollen Auswirkungen dieses Wandels sind nahezu unvorstellbar.

\section{DAMALS UND HEUTE}\label{damals-und-heute}

Etwas Ähnliches hätte man auch vor fünfhundert Jahren, an der Wende zum
fünfzehnten Jahrhundert, sagen können. Damals wie heute stand die
westliche Zivilisation an der Schwelle zu einem bedeutsamen Wandel.
Obwohl es kaum jemand wusste, lag die mittelalterliche Gesellschaft im
Sterben. Ihr Tod wurde weder allgemein erwartet noch verstanden.
Nichtsdestotrotz war die vorherrschende Stimmung von tiefer
Niedergeschlagenheit geprägt. Dies ist am Ende einer Epoche üblich, wenn
die konventionellen Denker spüren, dass die Dinge auseinanderfallen,
dass „der Falke den Falkner nicht hören kann``. Doch ihre geistige
Trägheit ist oft zu groß, um die Auswirkungen der sich abzeichnenden
Machtkonstellationen zu begreifen. Der Mittelalterhistoriker Johan
Huizinga schrieb über die ausklingenden Tage des Mittelalters: „Die
Chronisten des fünfzehnten Jahrhunderts sind fast alle die Dummköpfe
einer absoluten Fehleinschätzung ihrer Zeit gewesen, deren wirkliche
bewegende Kräfte ihnen entgangen sind.`` \footnote{Ebenda, S. 56.}

\subsection{Betrogene Mythen}\label{betrogene-mythen}

Größere Veränderungen in der zugrundeliegenden Dynamik der Macht neigen
dazu, konventionelle Denker zu verwirren, weil sie Mythen entlarven, die
die alte Ordnung rationalisieren, aber keine wirkliche Erklärungskraft
haben. Am Ende des Mittelalters, wie auch heute, klaffte eine besonders
große Lücke zwischen den überlieferten Mythen und der Realität. Huizinga
sagte über die Europäer im späten fünfzehnten Jahrhundert: „Ihr ganzes
Ideensystem war von der Fiktion durchdrungen, dass das Rittertum die
Welt beherrscht``.\footnote{Ebenda, S. 65.} Dies wird von der heutigen
Annahme, dass sie von Abstimmungen und Beliebtheitswettbewerben
beherrscht wird, noch übertroffen. Beide Behauptungen halten einer
genauen Prüfung nicht stand. In der Tat ist die Vorstellung, dass der
Lauf der Geschichte durch demokratische Abstimmungen bestimmt wird,
genauso albern wie die mittelalterliche Vorstellung, dass sie durch
einen ausgefeilten Kodex der Sitten, genannt Ritterlichkeit, bestimmt
wird.

Die Tatsache, dass diese Behauptung an Ketzerei grenzt, zeigt, wie weit
das konventionelle Denken von einem realistischen Verständnis der
Machtdynamik in der spätindustriellen Gesellschaft entfernt ist. Dies
ist ein Thema, das wir in diesem Buch genau untersuchen. Unserer Ansicht
nach war das Wahlrecht eher eine Folge als eine Ursache der
megapolitischen Bedingungen, die den modernen Nationalstaat
hervorgebracht haben. Die Massendemokratie und das Konzept der
Staatsbürgerschaft blühten auf, als der Nationalstaat wuchs. Sie werden
ins Wanken geraten, wenn der Nationalstaat ins Wanken gerät, was in
Washington ebenso viel Bestürzung hervorrufen wird, wie die Erosion des
Rittertums am Hof des Herzogs von Burgund vor fünfhundert Jahren.

\section{PARALLELEN ZWISCHEN RITTERLICHKEIT UND
STAATSBÜRGERSCHAFT}\label{parallelen-zwischen-ritterlichkeit-und-staatsbuxfcrgerschaft}

Wenn Sie verstehen können, wie und warum die Bedeutung ritterlicher Eide
mit dem Übergang zu einer industriellen Organisation der Gesellschaft
schwand, werden Sie besser in der Lage sein zu erkennen, wie die
Staatsbürgerschaft, wie wir sie heute kennen, im Informationszeitalter
schwinden könnte. Beide dienten einer ähnlichen Funktion. Sie
erleichterten die Ausübung von Macht unter zwei ganz unterschiedlichen
megapolitischen Bedingungen.

Feudale Eide herrschten in einer Zeit vor, in der die
Verteidigungstechnik im Vordergrund stand, die Souveränität zersplittert
war und Privatpersonen und Körperschaften ihre eigene militärische Macht
ausübten. Vor der Schießpulverrevolution wurden Kriege normalerweise von
kleinen Kontingenten bewaffneter Männer geführt. Selbst die mächtigsten
Monarchen verfügten nicht über ein militum perpetuum, ein stehendes
Heer. Sie bezogen ihre militärische Unterstützung von ihren Vasallen,
den Großfürsten, die sich wiederum auf ihre Vasallen, die Kleinfürsten,
stützten, die wiederum auf ihre Vasallen, die Ritter, zurückgriffen. Die
gesamte Loyalitätskette reichte von der Hierarchie bis hinunter zur
Person mit dem niedrigsten sozialen Status, die als würdig erachtet
wurde, Waffen zu tragen.

\subsection{Uniformen oder
Abweichungen?}\label{uniformen-oder-abweichungen}

Im Gegensatz zu einem modernen Heer marschierte ein mittelalterliches
Heer, vor der Einführung des Bürgerrechts, nicht in Uniformen auf das
Schlachtfeld. Im Gegenteil, jeder Gefolgsmann oder Vasall, jeder Ritter,
Baron oder Lord unterschiedlichen Ranges hatte seine eigene
unverwechselbare Tracht, die seinen Platz in der Hierarchie
widerspiegelte. Anstelle von Uniformen gab es Unterschiede, die die
vertikale Struktur der Gesellschaft betonten, in der jeder Stand
unterschiedlich war. Wie Huizinga sagte, unterschieden sich die
mittelalterlichen Krieger durch „äußere Zeichen von ... Unterschieden:
Trachten, Farben, Abzeichen, Schlachtrufe.`` \footnote{Ebenda, S. 22.}

Kriege wurden auch nicht nur von Regierungen oder Nationen geführt.
Martin van Creveld hat darauf hingewiesen, dass die moderne Vorstellung
vom Krieg, wie sie von Strategen wie Carl von Clausewitz stilisiert
wurde, die Realität vormoderner Konflikte falsch wiedergibt. Van Creveld
schreibt:

\begin{quote}
Tausend Jahre lang nach dem Fall Roms wurden bewaffnete Konflikte von
verschiedenen gesellschaftlichen Gruppen ausgetragen. Dazu gehörten
barbarische Stämme, die Kirche, Feudalherren jeden Ranges, freie Städte
und sogar Privatpersonen. Auch die „Armeen`` jener Zeit entsprachen
nicht denen, die wir heute kennen; es ist sogar schwierig, ein Wort zu
finden, das ihnen gerecht wird. Die Kriege wurden von Scharen von
Gefolgsleuten geführt, die sich militärische Gewänder anzogen und ihrem
Herrn folgten.\footnote{van Creveld, ebenda, S. 52.}
\end{quote}

Unter diesen Bedingungen war es für den Fürsten offensichtlich von
entscheidender Bedeutung, dass seine Gefolgsleute tatsächlich „ihr
militärisches Gewand anzogen und folgten``. Daher die starke Betonung
des ritterlichen Eides.

Die Ehre des mittelalterlichen Ritters und die Pflicht des
wehrpflichtigen Soldaten hatten parallele Funktionen. Der
mittelalterliche Mensch war durch Eide an Einzelpersonen und die Kirche
gebunden, so wie der moderne Mensch durch seine Staatsbürgerschaft an
den Nationalstaat gebunden ist. Die Verletzung eines Eides war im
Mittelalter gleichbedeutend mit Verrat. Die Menschen im Spätmittelalter
gingen bis zum Äußersten, um die Verletzung von Eiden zu vermeiden, so
wie Millionen moderner Bürger in den Weltkriegen bis zum Äußersten
gingen und Maschinengewehrnester angriffen, um ihre Pflichten als
Staatsbürger zu erfüllen.

Sowohl die Ritterlichkeit als auch das Bürgertum fügten dem einfachen
Kalkül eine zusätzliche Dimension hinzu, die ansonsten uninformierte
Menschen davon abhalten würde, sich auf ein Schlachtfeld zu begeben und
dort zu bleiben, wenn es schwierig wird. Sowohl Ritterlichkeit als auch
Staatsbürgerschaft verleiteten die Menschen dazu, zu töten und den Tod
zu riskieren. Nur anspruchsvolle und überzogene Werte, die durch
führende Institutionen stark gestärkt werden, können diese Funktion
erfüllen.

\subsection{Umgehung der
Kosten-Nutzen-Analyse}\label{umgehung-der-kosten-nutzen-analyse}

Der Erfolg und das Überleben eines jeden Systems hängen von seiner
Fähigkeit ab, in Konflikt- und Krisenzeiten militärische Anstrengungen
zu unternehmen. Es liegt auf der Hand, dass die Entscheidung eines
mittelalterlichen Ritters oder eines Gefreiten in den Schützengräben des
Ersten Weltkriegs, sein Leben in der Schlacht zu riskieren,
wahrscheinlich nicht auf einer nüchternen Kosten-Nutzen-Rechnung
beruhte. Selten sind Kriege so einfach zu führen, oder die Belohnungen
für diejenigen, die die Hauptlast der Kämpfe tragen, überschatten die
möglichen Kosten so sehr, dass ein Heer von Wirtschaftsoptimierern
rekrutiert werden könnte, um auf das Schlachtfeld zu eilen. In fast
jedem Krieg und in der Tat in den meisten Schlachten gibt es Momente, in
denen sich das Blatt innerhalb eines Herzschlags wenden kann. Studenten
der Militärgeschichte wissen, dass der Unterschied zwischen Niederlage
und Sieg oft durch den Mut, die Tapferkeit und die Grausamkeit, mit der
einzelne Soldaten ihre Aufgabe angehen, bestimmt wird. Wenn die
kämpfenden Männer nicht bereit sind, für ein Stück Boden zu sterben, das
nach dem Ende der Schlacht keinen Pfifferling mehr wert ist, dann werden
sie sich wahrscheinlich nicht gegen einen ansonsten ebenbürtigen Gegner
durchsetzen.

Dies hat wichtige Auswirkungen. Je wirksamer Souveränitäten die
Abtrünnigkeit begrenzen und militärische Anstrengungen fördern, desto
wahrscheinlicher ist es, dass sie sich militärisch durchsetzen. In der
Kriegsführung veranlassen die nützlichsten Wertesysteme die Menschen
dazu, sich so zu verhalten, wie es kurzfristiges rationales Kalkül
ausschließen würde. Keine Organisation könnte militärische Kräfte
wirksam mobilisieren, wenn die Menschen, die sie in die Schlacht
schickt, sich frei fühlen würden, zu berechnen, wo ihr eigener Vorteil
liegt, und sich dementsprechend dem Kampf anzuschließen oder
wegzulaufen. In diesem Fall würden sie fast nie kämpfen. Nur unter den
günstigsten Umständen oder in der größten Verzweiflung würde sich ein
vernünftiger Mensch auf der Grundlage einer kurzfristigen
Kosten-Nutzen-Analyse auf eine potenziell tödliche Schlacht einlassen.
Vielleicht würde der Homo oeconomicus an einem sonnigen Tag kämpfen,
wenn die Kräfte auf seiner Seite überwältigend, der Feind schwach und
die potenziellen Vorteile einer Schlacht verlockend wären. Mag sein.
Vielleicht würde er aber auch kämpfen, wenn er von marodierenden
Kannibalen in die Enge getrieben wird.

Aber das sind extreme Umstände. Was ist mit den häufigeren Bedingungen
der Kriegsführung, die weder so attraktiv sind, dass sie einer
Kosten-Nutzen-Analyse standhalten würden, noch so verzweifelt, dass sie
keinen Ausweg bieten? Hier leisten Konzepte wie Ritterlichkeit und
Staatsbürgerschaft einen wichtigen Beitrag zum erfolgreichen Einsatz
militärischer Macht. Lange bevor eine Schlacht beginnt, müssen die
herrschenden Organisationen den Einzelnen davon überzeugen, dass die
Einhaltung bestimmter Pflichten gegenüber dem Herrn oder dem
Nationalstaat wichtiger ist als das Leben selbst. Die Mythen und
Rationalisierungen, die Gesellschaften einsetzen, um die
Risikobereitschaft auf dem Schlachtfeld zu fördern, sind ein
wesentlicher Bestandteil ihrer militärischen Stärke.

Um wirksam zu sein, müssen diese Mythen auf die vorherrschenden
megapolitischen Bedingungen zugeschnitten sein. Die Fiktion, dass das
Rittertum die Welt regiert, hat heute keine Bedeutung mehr, vor allem
nicht in einer Stadt wie New York. Aber es war der hochgehaltene Mythos
des Feudalismus. Er rechtfertigte und rationalisierte das
Schuldverhältnis, das jeden unter der Herrschaft der Kirche und eines
kriegerischen Adels band. In einer Zeit, in der private Kriege der
Habgier an der Tagesordnung waren,\footnote{Huizinga, ebenda, S. 21.}
hingen die Ausübung von Macht und das Überleben des Einzelnen von der
Bereitschaft anderer ab, ihre Versprechen zum Militärdienst unter Zwang
zu erfüllen. Es war offensichtlich entscheidend, dass diese Versprechen
verlässlich waren.

\subsection{Vor der Nationalität}\label{vor-der-nationalituxe4t}

Anders als heute spielte das Konzept der Nationalität im Mittelalter bei
der Begründung von Souveränität keine oder nur eine geringe Rolle.
Monarchen, aber auch einige Kirchenfürsten und mächtige Herren besaßen
Territorien aus privatem Recht. In einer Weise, die keine moderne
Entsprechung hat, konnten diese Herren Gebiete verkaufen oder
verschenken oder neue Gebiete durch Übertragung oder Heirat sowie durch
Eroberung erwerben. Man kann sich heute kaum vorstellen, dass die
Vereinigten Staaten unter die Souveränität eines nicht
englischsprachigen portugiesischen Präsidenten fallen, weil er zufällig
die Tochter des früheren amerikanischen Präsidenten geheiratet hat. Doch
etwas Ähnliches war im mittelalterlichen Europa Gang und Gäbe. Die Macht
wurde durch Vererbung weitergegeben. Städte und Länder wechselten die
Herrscher, so wie Antiquitäten den Besitzer wechseln. In vielen Fällen
stammten die Herrscher nicht aus den Regionen, in denen ihr Besitz lag.
Manchmal beherrschten sie die Landessprache nicht oder sprachen sie
schlecht und mit starkem Akzent. Doch für die persönlichen
Verpflichtungen machte es kaum einen Unterschied, ob ein Spanier König
von Athen oder ein Österreicher König von Spanien war.

\subsection{Körperschaftliche
Souveränität}\label{kuxf6rperschaftliche-souveruxe4nituxe4t}

Die Souveränität wurde auch von religiösen Körperschaften wie den
Tempelrittern, den Johannitern und dem Deutschen Orden ausgeübt. Diese
hybriden Institutionen haben keine modernen Entsprechungen. Sie
verbanden religiöse, soziale, gerichtliche und finanzielle Aktivitäten
mit der Souveränität über lokale Gebiete.{[}\^{}106{]} Sie übten zwar
eine territoriale Gerichtsbarkeit aus, waren aber fast das Gegenteil der
heutigen Regierungen, da die Nationalität keine Rolle bei der
Mobilisierung ihrer Unterstützung oder ihrem Regierungssystem spielte.
Die Mitglieder und Amtsträger dieser Orden stammten aus allen Teilen des
christlichen Europas, der so genannten „Christenheit``.

Niemand hielt es für angemessen oder notwendig, dass die Regierenden aus
der lokalen Bevölkerung stammten. Die Mobilisierung von Unterstützung im
zersplitterten mittelalterlichen Regierungssystem hing nicht wie in der
Neuzeit von einer nationalen Identität oder einer Verpflichtung
gegenüber dem Staat ab, sondern von persönlicher Loyalität und
gewohnheitsmäßigen Bindungen, die als eine Frage der persönlichen Ehre
aufrechterhalten werden mussten. Diese Eide konnte jeder schwören, egal
woher er kam, sofern er aufgrund seines Standes als würdig erachtet
wurde.

\subsection{Das Gelübde}\label{das-geluxfcbde}

Ritterliche Gelübde banden Menschen aneinander und wurden auf die Ehre
der Beteiligten geschworen. Wie Huizinga schrieb, „legten die Menschen
bei der Ablegung eines Gelübdes einige Entbehrungen auf sich, die sie
anspornten, die Handlungen, zu denen sie sich verpflichtet hatten, zu
vollenden``.{[}\^{}107{]} Der Einhaltung von Gelübden wurde so viel
Bedeutung beigemessen, dass die Menschen häufig den Tod riskierten oder
schwerwiegende Folgen in Kauf nahmen, um ihre Gelübde nicht zu brechen.
Die Eide selbst verpflichteten die Menschen oft zu Handlungen, die Ihnen
und den meisten Lesern dieses Buches wahrscheinlich lächerlich vorkommen
würden, weil sie eine Frage der Ehre sind.

So schworen die Ritter des Sterns beispielsweise einen Eid, sich niemals
„mehr als vier Morgen vom Schlachtfeld zurückzuziehen, wodurch bald
darauf mehr als neunzig von ihnen ihr Leben verloren``.{[}\^{}108{]} Das
Verbot, sich auch nur taktisch zurückzuziehen, ist als militärische
Strategie irrational. Aber es war ein allgemeines Gebot der ritterlichen
Gelübde. Vor der Schlacht von Agincourt ordnete der englische König an,
dass die Ritter auf Patrouille ihre Rüstung ablegen sollten, da es mit
ihrer Ehre unvereinbar gewesen wäre, sich in ihrer Rüstung aus den
feindlichen Linien zurückzuziehen. So kam es, dass sich der König selbst
verirrte und an dem Dorf vorbeikam, in dem die Vorhut seines Heeres
übernachtet hatte. Da er eine Rüstung trug, verbot ihm seine ritterliche
Ehre, einfach umzukehren, als er seinen Fehler entdeckte, und zum Dorf
zurückzukehren. Er verbrachte die Nacht in einer exponierten Lage.

So albern dieses Beispiel auch erscheinen mag, König Heinrich hat sich
wahrscheinlich nicht verkalkuliert, als er dachte, dass er durch den
Rückzug seine Ehre mehr riskiert hätte, als wenn er hinter den
feindlichen Linien schlief und damit ein demoralisierendes Beispiel für
seine gesamte Armee gegeben hätte.

Die Geschichte des Mittelalters ist voll von Beispielen prominenter
Personen, die Gelübde erfüllten, die uns lächerlich erscheinen würden.
In vielen Fällen standen die vorgeschlagenen Handlungen in keinem
objektiven Zusammenhang mit irgendeinem Nutzen, sondern waren nur eine
anschauliche Demonstration der Bedeutung, die die Betreffenden dem
Gelübde selbst beimaßen. Zu den üblichen Gelübden gehörten: ein Auge
geschlossen zu halten, nur im Stehen zu essen und zu trinken und ein
selbst auferlegter Krüppel zu werden, indem man sich in eine
Ein-Mann-Kettenbande begibt. Es war ein weit verbreiteter Brauch,
schmerzhafte Fußfesseln zu tragen. Wenn Sie heute jemanden mit einem
schweren Fußeisen auf der Straße strampeln sähen, würden Sie
wahrscheinlich annehmen, dass er geisteskrank ist, und nicht, dass er
ein Mann von großer Tugendhaftigkeit ist. Doch im Kontext der
Ritterlichkeit war das freiwillige Anlegen eines solchen Geräts ein
Ehrenabzeichen. Und es gab viele ähnliche Bräuche, die heute ebenso
lächerlich erscheinen würden. Wie Huizinga beschreibt, legten viele ein
Gelübde ab, „am Samstag nicht in einem Bett zu schlafen, am Freitag
keine tierische Nahrung zu sich zu nehmen, etc. Ein Akt der Askese jagt
den anderen: ein Adliger verspricht, keine Rüstung zu tragen, an einem
Tag in der Woche keinen Wein zu trinken, nicht in einem Bett zu
schlafen, sich nicht zu den Mahlzeiten zu setzen oder das Cilicium zu
tragen``.\footnote{Ebenda, S. 90.}

Die Fastenzeit ist eine stark abgemilderte Version dieses selbst
auferlegten Unbehagens.

Viele Liebhaber von Gelübden bildeten Orden, die ihren Mitgliedern
besonders schwere Entbehrungen als Ehrenprüfungen auferlegten. Der Orden
der Clalois und Galoises zum Beispiel kleidete sich im Sommer mit
„Pelzen und pelzgefütterten Hauben und zündete ein Feuer im Herd an,
während sie im Winter nur einen einfachen Mantel ohne Pelz tragen
durften, weder Mantel, noch Hut, noch Handschuhe, und nur sehr leichte
Bettwäsche hatten.`` Wie Huizinga berichtet, „ist es nicht
verwunderlich, dass viele Mitglieder an der Kälte starben.`` \footnote{Ebenda,
  S. 87.}

\begin{quote}
„Die mittelalterliche Selbstgeißelung war eine grausame Folter, die sich
die Menschen selbst zufügten, in der Hoffnung, einen richtenden und
strafenden Gott dazu zu bewegen, seine Rute wegzulegen, ihre Sünden zu
vergeben und ihnen die größeren Strafen zu ersparen, die ihnen sonst in
dieser und der nächsten Welt drohten.`` \footnote{Norman Cohn, \emph{The
  Pursuit of the Millennium: Revolutionary Millenarians and Mystical
  Anarchists of the Middle Ages}, überarbeitete und erweiterte Ausgabe
  (Oxford: Oxford University Press, 1970), S. 127.} - Norman Cohn
\end{quote}

\subsection{Geißelung, damals und
heute}\label{geiuxdfelung-damals-und-heute}

Vom Gelübde, das Gefahren und Entbehrungen mit sich brachte, war es nur
ein kleiner Schritt zu Prüfungen, Pilgerfahrten, Kasteiungen,
Unannehmlichkeiten und sogar absichtlich selbst zugefügten Verletzungen.
Dies konnte im Mittelalter als äußerst nützlich und lobenswert angesehen
werden. Sie waren Gesten der Ernsthaftigkeit, mit der Gelübde abgelegt
wurden, eine Logik, die auch heute noch bei der Aufnahme von
Studentenverbindungen nicht völlig fremd ist.

Im Sommer zu schwitzen, im Winter zu frieren oder barfuß durch den
Schnee zu pilgern, war relativ harmlos im Vergleich zu der „grimmigen
Folter`` der Selbstgeißelung. Dies war eine speziell mittelalterliche
Form der Buße, die fast zeitgleich mit dem Beginn des Feudalismus
aufkam. Sie wurde erstmals „von Einsiedlern in den Klostergemeinschaften
von Camaldoli und Fonte Avellana zu Beginn des elften Jahrhunderts
übernommen.`` \footnote{Ebenda.}

Anstatt bei kaltem Wetter nur barfuß zu laufen, organisierten die
Geißler Prozessionen, bei denen sie Tag und Nacht von einer Stadt zur
nächsten zogen. „Und jedes Mal, wenn sie in eine Stadt kamen, stellten
sie sich in Gruppen vor der Kirche auf und peitschten sich stundenlang
aus.`` \footnote{Ebenda, S. 128.}

Wir glauben, dass Menschen in der Zukunft, die auf die Ära des
Nationalstaates zurückblicken, einige der Unternehmungen, die im
zwanzigsten Jahrhundert im Namen der Staatsbürgerschaft unternommen
wurden, ebenso lächerlich finden werden wie wir die Selbstgeißelung. Aus
der Sicht der Informationsgesellschaft wird das Spektakel von Soldaten
in der Neuzeit, die um die halbe Welt reisen, um aus Loyalität zum
Nationalstaat den Tod in Kauf zu nehmen, als grotesk und lächerlich
empfunden werden. Es wird nicht viel anders erscheinen als einige der
außergewöhnlichen und übertriebenen Riten des Rittertums, wie z.B. das
Herumlaufen in Fußfesseln, auf das ansonsten vernünftige Menschen
während der Feudalzeit stolz waren.

\subsection{Die Ritterlichkeit weicht der
Staatsbürgerschaft}\label{die-ritterlichkeit-weicht-der-staatsbuxfcrgerschaft}

Das Rittertum verblasste und wurde durch die Staatsbürgerschaft ersetzt,
als die megapolitischen Bedingungen sich änderten, und der militärische
Zweck des Gelübdes gegenüber seinem Herrn war antiquiert. In der Welt
der Schießpulverwaffen und der industriellen Armeen bestanden ganz
andere Beziehungen zwischen den Menschen, die kämpften, und ihren
Befehlshabern. Das Bürgertum entstand zu einer Zeit, als die Erträge aus
der Gewalttätigkeit hoch waren und weiter stiegen, und der Staat
verfügte über weitaus größere Ressourcen als die sozialen Einheiten, die
im Mittelalter Krieg führten. Aufgrund seiner großen Macht und seines
Reichtums konnte der Nationalstaat direkt mit der Masse der einfachen
Soldaten verhandeln, die in seiner Uniform kämpften.

Solche Geschäfte erwiesen sich für den Staat als weitaus billiger und
weniger mühsam als der Versuch, durch Verhandlungen mit mächtigen
Fürsten und lokalen Würdenträgern militärische Kräfte aufzustellen, von
denen jeder in der Lage war, sich gegen Forderungen zu wehren, die
seinen Interessen zuwiderliefen, wie es kein einzelner Bürger des
Nationalstaates vermochte.

Aus Gründen, auf die wir später näher eingehen werden, hing die
Staatsbürgerschaft entscheidend von der Tatsache ab, dass kein
Individuum oder keine kleine Gruppe von Individuen megapolitisch in der
Lage war, unabhängig militärische Macht auszuüben. In dem Maße, wie die
Informationstechnologie die Logik des Kampfes verändert, wird sie die
Mythen der Staatsbürgerschaft ebenso sicher antiquieren, wie das
Schießpulver das mittelalterliche Rittertum antiquierte.

\subsection{Die Hell's Angels zu
Pferde}\label{die-hells-angels-zu-pferde}

Die Aristokraten der berittenen Krieger, die Westeuropa jahrhundertelang
beherrschte, waren kaum die Gentlemen, zu denen ihre Nachkommen wurden.
Sie waren rau und gewalttätig. Aus heutiger Sicht könnte man sie eher
als das mittelalterliche Äquivalent von Motorradclubs verstehen. Die
Sittenregeln und die vorgetäuschte Ritterlichkeit dienten eher dazu,
ihre Exzesse zu zügeln, als ihr tatsächliches Verhalten zu beschreiben.
Selbst eine enzyklopädische Darstellung der Regeln und Pflichten des
Rittertums hätte wenig oder nichts über die Grundlagen der Macht des
Adels verraten.

\subsection{Perfektion als Synonym für
Erschöpfung}\label{perfektion-als-synonym-fuxfcr-erschuxf6pfung}

Das Aufkommen effektiver Schießpulverwaffen am Ende des fünfzehnten
Jahrhunderts sorgte für eine gewaltige Explosion unter ihren Füßen -
gerade als die bewaffneten Ritter ihre Kunst wie nie zuvor
perfektioniert hatten. Zu diesem Zeitpunkt hatte die sorgfältige Zucht
endlich ein 1,60 m großes Schlachtpferd hervorgebracht, ein Pferd, das
die Statur hatte, einen berittenen Ritter in voller Rüstung bequem zu
tragen. Doch „Perfektion``, wie C. Northcote Parkinson scharfsinnig
bemerkte, „wird nur von Institutionen erreicht, die kurz vor dem
Zusammenbruch stehen.`` \footnote{C. Northcote Parkinson,
  \emph{Parkinson\textquotesingle s Law and Other Studies in
  Administration} (Boston: Houghton Mifflin, 1957), S. 60, zitiert in
  Tilly, S. 4.} Gerade als das neue Schlachtross perfektioniert war,
wurden neue Waffen eingesetzt, um Pferd und Ritter vom Schlachtfeld zu
vertreiben. Diese neuen Schießpulverwaffen konnten von einfachen Leuten
abgefeuert werden. Sie erforderten wenig Geschicklichkeit, waren aber
teuer in der Beschaffung. Durch ihre Verbreitung gewann der Handel
gegenüber der Landwirtschaft, die die Grundlage der Feudalwirtschaft
gewesen war, immer mehr an Bedeutung.

\subsection{Krieg in größerem
Maßstab}\label{krieg-in-gruxf6uxdferem-mauxdfstab}

Wie konnten die Schießpulverwaffen einen solchen Wandel herbeiführen?
Zum einen erhöhten sie das Ausmaß der Kämpfe, was bedeutete, dass das
Führen von Kriegen bald viel kostspieliger wurde als im Mittelalter. Vor
der Schießpulverrevolution wurden Kriege in der Regel von so kleinen
Gruppen geführt, dass sie über ein kleines und armes Gebiet geführt
werden konnten. Das Schießpulver verschaffte dem Kampf in größerem
Maßstab einen neuen Vorteil. Nur Herrscher mit Ansprüchen auf reiche
Untertanen konnten es sich leisten, unter den neuen Bedingungen wirksame
Streitkräfte aufzustellen. Diejenigen Herrscher, die dem Wachstum des
Handels am besten Rechnung trugen, in der Regel Monarchen, die sich mit
den städtischen Kaufleuten verbündeten, hatten auf dem Schlachtfeld
einen Wettbewerbsvorteil. Nach van Creveld „konnten sie, auch dank der
ihnen zur Verfügung stehenden überlegenen finanziellen Mittel, mehr
Kanonen als alle anderen kaufen und den Gegner in Stücke sprengen.``
\footnote{van Creveld, ebenda, S. 50.}

Auch wenn es noch Jahrhunderte dauern sollte, bis die volle Logik der
Schießpulverwaffen in den Bürgerarmeen der Französischen Revolution zum
Tragen kam, war die Einführung von Militäruniformen in der Renaissance
ein früher Hinweis auf die Umgestaltung der Kriegsführung durch das
Schießpulver. Die Uniformen symbolisieren auf treffende Weise die neuen
Beziehungen zwischen dem Krieger und dem Nationalstaat, die mit dem
Übergang vom Rittertum zum Bürgertum Hand in Hand gingen. Der neue
Nationalstaat würde mit seinen Bürgern eine „einheitliche`` Vereinbarung
treffen, im Gegensatz zu den speziellen, abweichenden Vereinbarungen,
die der Monarch oder der Papst mit einer langen Kette von Vasallen im
Feudalismus getroffen hatte. In dem alten System hatte jeder einen
anderen Platz in einer architektonischen Hierarchie. Jeder hatte einen
Vertrag, der so einzigartig war wie sein Wappen und die bunten Wimpel,
die er trug.

\subsection{Senkung der Opportunitätskosten von
Reichtum}\label{senkung-der-opportunituxe4tskosten-von-reichtum}

Die Schießpulverwaffen veränderten das Wesen der Gesellschaft noch auf
eine andere Weise radikal. Sie trennten die Ausübung von Macht von
körperlicher Kraft und senkten so die Opportunitätskosten der
Handelsaktivitäten. Reiche Kaufleute mussten sich nicht mehr auf ihre
eigene Geschicklichkeit und Stärke im Nahkampf oder auf Söldner mit
ungewisser Loyalität verlassen, um sich zu verteidigen. Sie konnten
darauf hoffen, von den neuen, größeren Armeen der großen Monarchen
verteidigt zu werden. William Playfair sagte über das Mittelalter:
„Während menschliche Kraft die Macht war, mit der Menschen im Falle von
Feindseligkeiten belästigt wurden, ... war es damals unmöglich, lange
Zeit gleichzeitig reich und mächtig zu sein.`` \footnote{Playfair,
  ebenda, S. 72.} Als das Schießpulver aufkam, war es unmöglich, mächtig
zu sein, ohne reich zu sein.

\subsection{Status und statisches
Verstehen}\label{status-und-statisches-verstehen}

Aus vielen der gleichen Gründe, aus denen die meisten Menschen heute
schlecht darauf vorbereitet sind, die neue Dynamik der
Informationsgesellschaft zu antizipieren, waren die führenden Denker der
mittelalterlichen Gesellschaft nicht in der Lage, den Aufstieg des
Handels, der eine so wichtige Rolle bei der Gestaltung der modernen Zeit
spielte, vorherzusehen oder zu verstehen. Die meisten Menschen vor fünf
Jahrhunderten betrachteten ihre sich verändernde Gesellschaft in
statischen Begriffen. Wie Huizinga sagte: „Nur sehr wenig Besitz ist im
modernen Sinne flüssig, während Macht noch nicht überwiegend mit Geld
verbunden ist; sie ist vielmehr noch der Person inhärent und hängt von
einer Art religiöser Ehrfurcht ab, die sie erweckt; sie macht sich durch
Prunk und Pracht oder eine zahlreiche Schar treuer Gefolgsleute
bemerkbar. Das feudale oder hierarchische Denken drückt die Idee der
Größe durch sichtbare Zeichen aus...`` \footnote{Huizinga, ebenda, S.
  26.} Da die Menschen im Spätmittelalter vor allem an ihren Status
dachten, waren sie nicht darauf vorbereitet, dass Kaufleute irgendetwas
Wichtiges zum Leben des Reiches beitragen könnten. Die Kaufleute waren
fast ausnahmslos Bürgerliche. Sie standen am unteren Ende der drei
Stände, unterhalb des Adels und des Klerus.

Selbst die scharfsinnigeren Denker der spätmittelalterlichen
Gesellschaft erkannten nicht die Bedeutung des Handels und anderer
Formen des Unternehmertums außerhalb der Landwirtschaft für die
Anhäufung von Reichtum. Für sie war die Armut eine apostolische Tugend.
Sie machten buchstäblich keinen Unterschied zwischen einem wohlhabenden
Bankier und einem Bettler. In Huizingas Worten: „Im dritten Stand wurde
im Prinzip weder zwischen reichen und armen Bürgern noch zwischen Stadt-
und Landbewohnern unterschieden``.\footnote{Ebenda, S. 57.} Weder Beruf
noch Reichtum spielten in ihrem System eine Rolle, sondern lediglich der
ritterliche Status.

Diese Blindheit gegenüber der wirtschaftlichen Dimension des Lebens
wurde von den Kirchenmännern verstärkt, die die ideologischen Hüter des
mittelalterlichen Lebens waren. Sie waren so weit davon entfernt, die
Bedeutung des Handels zu begreifen, dass ein weithin beachtetes
Reformprogramm aus dem 15. Jahrhundert vorschlug, alle Personen ohne
adligen Status sollten sich ausschließlich dem Handwerk oder der
Landarbeit widmen. Für den Handel war überhaupt keine Rolle
vorgesehen.\footnote{Ebenda.}

\begin{quote}
„Das Datum 1492, das üblicherweise verwendet wird, um die
mittelalterliche von der neuzeitlichen Geschichte zu trennen, eignet
sich so gut wie jeder andere Trennungspunkt, denn in der Perspektive der
Weltgeschichte symbolisiert die Reise von Kolumbus den Beginn einer
neuen Beziehung zwischen Westeuropa und dem Rest der Welt.`` \footnote{Frederic
  C. Lane, \emph{Venice: A Maritime Republic} (Baltimore: Johns Hopkins
  University Press, 1973), S. 275.} - Frederic C. Lane
\end{quote}

\section{DIE GEBURT DES
INDUSTRIEZEITALTERS}\label{die-geburt-des-industriezeitalters}

Viele der schärfsten Köpfe des fünfzehnten Jahrhunderts haben eine der
wichtigsten Entwicklungen der Geschichte, die vor ihren Augen begann,
völlig übersehen. Der Untergang des Feudalismus markiert den Beginn der
großen modernen Phase der westlichen Vorherrschaft. Es war eine Zeit, in
der die Rendite der Gewalt und die Größe der Unternehmen zunahmen. In
den letzten zweieinhalb Jahrhunderten hat die moderne Wirtschaft dem
Teil der Welt, der am meisten von ihr profitierte, einen beispiellosen
Anstieg des Lebensstandards beschert. Die Katalysatoren für diese
Veränderungen waren neue Technologien, von Schießpulverwaffen bis zur
Druckerpresse, die die Grenzen des Lebens auf eine Weise veränderten,
die nur wenige begreifen konnten.

Im letzten Jahrzehnt des fünfzehnten Jahrhunderts begannen Entdecker wie
Kolumbus gerade erst, den Zugang zu riesigen, unbekannten Kontinenten zu
eröffnen. Zum ersten Mal in der langen Geschichte der Menschheit wurde
die ganze Welt umsegelt. Galeonen, neue, hochmastige Improvisationen der
mediterranen Galeeren, umrundeten den Globus und kartographierten die
Passagen, die zu Handelswegen und Durchgangsstraßen für Krankheiten und
Eroberungen werden sollten. Konquistadoren, die ihre neuen Bronzekanonen
zu Wasser und zu Lande einsetzten, eröffneten neue Horizonte. Sie fanden
Reichtümer in Gold und Gewürzen, pflanzten die Saat für neue
Nutzpflanzen, von Tabak bis Kartoffeln, und steckten neue Weideflächen
für ihr Vieh ab.

\subsection{Die erste industrielle
Technologie}\label{die-erste-industrielle-technologie}

So wie die Kanone neue wirtschaftliche Horizonte eröffnete, eröffnete
die Druckerpresse neue geistige Horizonte. Sie war die erste Maschine
für die Massenproduktion, eine charakteristische Technologie, die den
Beginn der Industrialisierung markierte. Damit teilen wir die Ansicht
von Adam Smith in Der Wohlstand der Nationen, dass die industrielle
Revolution bereits lange vor seiner Niederschrift stattgefunden hat. Sie
war zwar noch nicht ausgereift, aber die Prinzipien der Massenproduktion
und des Fabriksystems waren bereits etabliert. Sein berühmtes Beispiel
der Stecknadelhersteller verdeutlicht dies. Smith erklärt, wie achtzehn
verschiedene Arbeitsgänge zur Herstellung von Stecknadeln eingesetzt
werden. Aufgrund der spezialisierten Technik und der Arbeitsteilung
konnte jeder Mitarbeiter an einem Tag 4.800 mal mehr Stecknadeln
herstellen als ein Einzelner.\footnote{Adam Smith, \emph{An Inquiry into
  the Nature and Causes of the Wealth of Nations} (Chicago: University
  of Chicago Press, 1976), S. 8-9.}

Smiths Beispiel unterstreicht die Tatsache, dass die industrielle
Revolution Jahrhunderte früher begann, als die Historiker üblicherweise
annehmen. Die meisten Lehrbücher datieren ihre Anfänge auf die Mitte des
achtzehnten Jahrhunderts. Das ist als Datum für die Startphase der
Verbesserung des Lebensstandards nicht unangemessen. Der eigentliche
megapolitische Übergang vom Feudalismus zum Industrialismus begann
jedoch schon viel früher, nämlich Ende des fünfzehnten Jahrhunderts.
Seine Auswirkungen zeigten sich fast unmittelbar in der Umwandlung der
herrschenden Institutionen, insbesondere in der Verfinsterung der
mittelalterlichen Kirche.

Die Historiker, die die Industrielle Revolution später ansetzen, messen
in Wirklichkeit etwas anderes, nämlich den Anstieg des Lebensstandards,
der auf die von Maschinen angetriebene Massenproduktion zurückzuführen
ist. Dadurch stieg der Wert ungelernter Arbeit und die Preise für eine
Vielzahl von Konsumgütern sanken. Die Tatsache, dass der Lebensstandard
in den verschiedenen Ländern zu unterschiedlichen Zeiten stark anstieg,
ist ein Hinweis darauf, dass etwas anderes gemessen wird als der
Übergang zur Megapolitik. Die Cambridge Economic History of Europe
spricht von „Industriellen Revolutionen`` im Plural und verbindet sie
ausdrücklich mit dem anhaltenden Wachstum der nationalen
Einkommen.\footnote{Siehe H. J. Habakkuk und M. Postan, Hrsg., \emph{The
  Cambridge Economic History of Europe}, vol.6, \emph{The Industrial
  Revolution and After: Incomes, Population and Technological Change}
  (Cambridge: Cambridge University Press, 1966).} In Japan und Russland
verzögerte sich dieser Einkommensanstieg bis zum Ende des 19.
Jahrhunderts. Der Anstieg des Lebensstandards und das nachhaltige
Wachstum des Volkseinkommens in anderen Teilen Asiens und einigen Teilen
Afrikas war ein Phänomen des 20. Jahrhunderts. In einigen Teilen Afrikas
ist ein nachhaltiges Wachstum bis heute ein Traum geblieben. Das
bedeutet jedoch nicht, dass diese Regionen nicht in der Moderne leben.

\subsection{Sinkende Einkommen im
Wandel}\label{sinkende-einkommen-im-wandel}

Das Einkommenswachstum ist nicht gleichbedeutend mit dem Beginn der
Industrialisierung. Der Übergang zur Industriegesellschaft war ein
megapolitisches Ereignis, das in den Einkommensstatistiken nicht direkt
messbar ist. In der Tat sanken die Realeinkommen der meisten Europäer in
den ersten beiden Jahrhunderten des Industriezeitalters. Sie begannen
erst nach Beginn des achtzehnten Jahrhunderts zu steigen und erreichten
erst um 1750 wieder das Niveau von 1250. Wir datieren den Beginn des
Industriezeitalters auf das Ende des fünfzehnten Jahrhunderts. Es waren
die industriellen Merkmale der frühmodernen Technologie, einschließlich
chemisch betriebener Waffen und Druckerpressen, die den Zusammenbruch
des Feudalismus auslösten.

\subsection{Senkung der Kosten des
Wissens}\label{senkung-der-kosten-des-wissens}

Die Fähigkeit zur Massenproduktion von Büchern war für die
mittelalterlichen Institutionen unglaublich subversiv, so wie sich die
Mikrotechnologie für den modernen Nationalstaat als subversiv erweisen
wird. Der Druck untergrub rasch das Monopol der Kirche auf das Wort
Gottes, während er gleichzeitig einen neuen Markt für Ketzerei schuf.
Ideen, die der geschlossenen Feudalgesellschaft zuwiderliefen,
verbreiteten sich rasch, und bis zum letzten Jahrzehnt des fünfzehnten
Jahrhunderts wurden 10 Millionen Bücher veröffentlicht. Da die Kirche
versuchte, den Buchdruck zu unterdrücken, wurden die meisten neuen
Bücher in den Gebieten Europas veröffentlicht, in denen der Einfluss der
etablierten Autorität am schwächsten war. Dies könnte eine enge Analogie
zu den heutigen Versuchen der US-Regierung sein, die
Verschlüsselungstechnologie zu unterdrücken. Die Kirche stellte fest,
dass die Zensur die Verbreitung subversiver Technologien nicht
unterdrückte, sondern lediglich dafür sorgte, dass sie in ihrer
subversivsten Form eingesetzt wurden.

\subsection{Der Verfall der Klöster}\label{der-verfall-der-kluxf6ster}

Viele scheinbar harmlose Verwendungen der Druckerpresse waren aufgrund
ihres Inhalts subversiv. Allein die Verbreitung des Wissens über die
Vermögen, die unerschrockene Abenteurer und Kaufleute verdienen konnten,
war ein starkes Lösungsmittel, das die Fesseln der feudalen
Verpflichtungen auflöste. Die Verlockung neuer Märkte sowie die
Notwendigkeit und Möglichkeit, Armeen und Flotten in großem Umfang zu
finanzieren, verliehen dem Geld einen Wert, der ihm in den feudalen
Jahrhunderten gefehlt hatte. Diese neuen Investitionsmöglichkeiten,
verstärkt durch mächtige Waffen, die den Ertrag der Gewalttätigkeit
erhöhten, machten es für den Grundherrn im Hinterland oder den Kaufmann
in der Stadt immer kostspieliger, sein Kapital der Kirche zu spenden. So
destabilisierte die Schaffung von Investitionsmöglichkeiten außerhalb
des Grundbesitzes die Institutionen des Feudalismus und untergrub seine
Ideologie.

Eine weitere subversive Folge des Buchdrucks war die drastische Senkung
der Kosten für die Vervielfältigung von Informationen. Ein wesentlicher
Grund dafür, dass die Alphabetisierung und der wirtschaftliche
Fortschritt im Mittelalter so gering waren, waren die hohen Kosten für
die Vervielfältigung von Manuskripten von Hand. Wie wir gesehen haben,
war die Vervielfältigung von Büchern und Manuskripten in den
Benediktinerklöstern eine der wichtigsten produktiven Aufgaben, die die
Kirche nach dem Fall Roms übernahm. Dies war ein äußerst kostspieliges
Unterfangen. Eine der dramatischsten Folgen des Buchdrucks war die
Entwertung der Skriptorien, in denen die Mönche Tag für Tag, Monat für
Monat arbeiteten, um Manuskripte zu erstellen, die mit Hilfe von
Druckerpressen innerhalb von Stunden vervielfältigt werden konnten. Die
neue Technologie machte das benediktinische Skriptorium zu einem
veralteten und kostspieligen Mittel der Wissensvervielfältigung. Dadurch
verloren die religiösen Orden und die Kirche, die die Schreiber
unterstützten, an wirtschaftlicher Bedeutung.

Die Massenproduktion von Büchern beendete das Monopol der Kirche auf die
Heilige Schrift wie auch auf andere Formen der Information. Die größere
Verfügbarkeit von Büchern senkte die Kosten für die Alphabetisierung und
vervielfachte so die Zahl der Denker, die in der Lage waren, ihre eigene
Meinung zu wichtigen Themen, insbesondere zu theologischen Themen, zu
äußern. Wie der Theologiehistoriker Euan Cameron es ausdrückt, legte
„eine Reihe von Meilensteinen im Verlagswesen`` in den ersten beiden
Jahrzehnten des sechzehnten Jahrhunderts den Grundstein für die
Anwendung der „modernen Textkritik auf die Heilige
Schrift``.{[}\^{}123{]} Dies „bedrohte das Monopol`` der Kirche, „indem
es korrupte Lesarten von Texten in Frage stellte, die zur Stützung
traditioneller Dogmen verwendet worden waren.`` {[}\^{}124{]} Dieses
neue Wissen förderte die Entstehung konkurrierender protestantischer
Sekten, die versuchten, ihre eigenen Interpretationen der Bibel zu
formulieren. Die Massenproduktion von Büchern senkte die Kosten der
Ketzerei und verschaffte den Häretikern ein großes Publikum von Lesern.

Auch das Verlagswesen trug zur Zerstörung des mittelalterlichen
Weltbildes bei. Die größere Verfügbarkeit und die niedrigeren Kosten für
Informationen führten zu einer Abkehr von einem Weltbild, das eher durch
Symbolik als durch kausale Zusammenhänge verbunden war. „Das symbolische
Weltbild zeichnet sich durch tadellose Ordnung, architektonische
Struktur und hierarchische Unterordnung aus. Denn jede symbolische
Verbindung impliziert einen Unterschied im Rang oder in der
Heiligkeit.\ldots{} Die Walnuss bedeutet Christus; der süße Kern ist
seine göttliche Natur, die grüne und breiige äußere Schale ist seine
Menschlichkeit, die hölzerne Schale dazwischen ist das Kreuz. So erheben
alle Dinge die Gedanken auf das Ewige...`` \footnote{Huizinga, ebenda,
  S. 198.}

Eine symbolische Denkweise passte nicht nur zu einer hierarchischen
Gesellschaftsstruktur, sondern auch zum Analphabetismus. Ideen, die
durch Symbole in Holzschnitten vermittelt wurden, waren für eine
ungebildete Bevölkerung zugänglich. Im Gegensatz dazu führte das
Aufkommen des Buchdrucks in der Neuzeit dazu, dass eine gebildete
Bevölkerung kausale Zusammenhänge mit Hilfe der wissenschaftlichen
Methode entwickeln konnte.

\section{EINE PARALLELE ZU HEUTE}\label{eine-parallele-zu-heute}

Die mittelalterliche Gesellschaft, die in der Mitte des fünfzehnten
Jahrhunderts scheinbar so stabil und sicher in ihrem Glauben war,
wandelte sich rasch. Ihre wichtigste Institution, die Kirche, sah ihre
Monopolstellung in Frage gestellt und erschüttert. Die Autorität, die
jahrhundertelang unangefochten war, wurde plötzlich in Frage gestellt.
Überzeugungen und Loyalitäten, die heiliger waren als die, die heute
jeden Bürger an einen Nationalstaat binden, wurden innerhalb weniger
Jahre überdacht und aufgegeben, und das alles aufgrund einer
technologischen Revolution, die im letzten Jahrzehnt des 15.
Jahrhunderts ihren Lauf nahm.

Wir glauben, dass sich ein ähnlich dramatischer Wandel wie vor
fünfhundert Jahren wiederholen wird. Die Informationsrevolution wird das
Machtmonopol des Nationalstaates ebenso sicher zerstören wie die
Schießpulverrevolution das Monopol der Kirche. Es besteht eine
auffällige Analogie zwischen der Situation am Ende des fünfzehnten
Jahrhunderts, als das Leben durch und durch von der organisierten
Religion durchdrungen war, und der heutigen Situation, in der die Welt
von der Politik durchdrungen ist. Die Kirche von damals und der
Nationalstaat von heute sind beides Beispiele für Institutionen, die
sich zu einem altersschwachen Extrem entwickelt haben. Wie die
spätmittelalterliche Kirche ist auch der Nationalstaat am Ende des
zwanzigsten Jahrhunderts eine hoch verschuldete Institution, die ihren
Lebensunterhalt nicht mehr bestreiten kann. Seine Aktivitäten werden
immer irrelevanter und sind sogar kontraproduktiv für den Wohlstand
derjenigen, die vor nicht allzu langer Zeit noch zu seinen treuesten
Anhängern zählten.

\subsection{„Verarmt, habgierig und
verschwenderisch``}\label{verarmt-habgierig-und-verschwenderisch}

Genauso wie der Staat heute einen schlechten Gegenwert für das
eingenommene Geld bietet, tat dies die Kirche am Ende des fünfzehnten
Jahrhunderts. Der Kirchenhistoriker Euan Cameron drückt es so aus: „Eine
verarmte lokale Priesterschaft schien für das Geld, das sie verlangte,
einen schlechten Dienst zu leisten; ein Großteil der Abgaben
‚verschwand' in geschlossenen Klöstern oder in den geheimnisvollen
Bereichen der höheren Bildung oder der Verwaltung. Trotz der üppigen
Schenkungen an einige Bereiche der Kirche gelang es der Institution als
Ganzes, gleichzeitig verarmt, habgierig und verschwenderisch zu
erscheinen.`` \footnote{Cameron, ebenda, S. 26-27.} Die Parallele zur
Regierung des späten zwanzigsten Jahrhunderts lässt sich nur schwer
leugnen.

Im späten fünfzehnten Jahrhundert wuchsen die religiösen Observanzen wie
heute die Programme in den Wohlfahrtsstaaten. Nicht nur, dass sich
besondere Segnungen endlos vervielfachten, ebenso wie der Vorrat an
Heiligen und Heiligengebeinen, es gab auch jedes Jahr mehr Kirchen, mehr
Klöster, mehr Konvente, mehr Ordensgemeinschaften, mehr Beichtväter
(ansässige Hauspriester), mehr Predigerstellen, mehr Domkapitel, mehr
gestiftete Kantoreien, mehr Reliquienkulte, mehr religiöse
Bruderschaften, mehr religiöse Feste und neue heilige Tage. Die
Gottesdienste wurden länger. Gebete und Hymnen wurden immer
komplizierter. Nacheinander tauchten neue Bettelorden auf, die um
Almosen baten. Das Ergebnis war eine institutionelle Überlastung, die
derjenigen ähnelte, die heute stark politisierte Gesellschaften
kennzeichnet.

Religiöse Feste und Feiertage wuchsen auf allen Seiten. Die
Gottesdienste wurden zahlreicher, mit besonderen Festen zu Ehren der
sieben Schmerzen Mariens, ihrer Schwestern und aller Heiligen aus dem
Stammbaum Jesu.{[}\^{}127{]} Für die Gläubigen wurde die Erfüllung ihrer
religiösen Pflichten immer kostspieliger und beschwerlicher, ähnlich wie
die Kosten für die Einhaltung des Gesetzes heute gestiegen sind.

\subsection{Die Zahlungslasten der
Unschuldigen}\label{die-zahlungslasten-der-unschuldigen}

Damals wie heute trugen die Produktiven eine wachsende Last der
Einkommensumverteilung. Diese Kosten stiegen stärker als von den
Verantwortlichen erkannt, weil sich die Nutzung des Kapitals verschob.
Der relative Vorteil des Landbesitzes gegenüber dem Geldkapital nahm ab.
Dennoch dachte man im Mittelalter weiterhin in Begriffen einer
statusgebundenen Gesellschaft, in der die soziale Stellung dadurch
bestimmt wurde, wer man war, und nicht durch die Fähigkeit, das Kapital
effektiv einzusetzen. Die steigenden Opportunitätskosten, die mit der
Inszenierung übertriebener religiöser Observanzen verbunden waren,
wurden kaum oder gar nicht berücksichtigt. Diese Kosten trafen vor allem
die ehrgeizigeren und fleißigeren Bauern, Bürger und Freibauern, die
mehr als der Adel darauf angewiesen waren, ihr Kapital nutzbringend
einzusetzen. Sie mussten einen unverhältnismäßig großen Teil der Kosten
für die Ausstattung der Tische bei den endlosen Festen und Feiertagen
sowie für den Unterhalt einer verschwenderischen Kirchenbürokratie
aufbringen.

\subsection{Kontraproduktive
Regulierung}\label{kontraproduktive-regulierung}

Am Ende des fünfzehnten Jahrhunderts kontrollierte die Kirche weitgehend
die Regelungsbefugnisse, die seither von den Regierungen übernommen
wurden. Die Kirche beherrschte wichtige Rechtsbereiche, indem sie
Urkunden ausstellte, Ehen registrierte, Testamente unterzeichnete,
Gewerbe zuließ, Grundstücke überschrieb und die Bedingungen für den
Handel festlegte. Die Einzelheiten des Lebens wurden durch das
Kirchenrecht fast ebenso gründlich geregelt wie heute durch die
Bürokratie, und zwar mit demselben Ziel. So wie die politische
Regulierung heute von Verwirrungen und Widersprüchen durchsetzt ist, so
war es auch das Kirchenrecht vor fünfhundert Jahren. Diese Vorschriften
unterdrückten und verkomplizierten den Handel oft auf eine Art und
Weise, die erkennen ließ, dass die Erleichterung der Produktivität nicht
im Sinne der Regulierer war.

So war es beispielsweise ein ganzes Jahr lang verboten, Geschäfte an dem
Wochentag zu tätigen, auf den der letzte achtundzwanzigste Dezember
fiel. Fiel dieser Tag auf einen Dienstag, so durften dienstags keine
Geschäfte getätigt werden, da dies ein obligatorischer Ausdruck der
Frömmigkeit zu Ehren der Schlachtung der Unschuldigen war. In den
Jahren, in denen der 28. Dezember auf einen anderen Tag als einen
Sonntag fiel, behinderte diese Vorschrift viele Arten des Handels und
erhöhte die Kosten, indem sie Transaktionen verzögerte oder gänzlich
verhinderte.

\subsection{Monopolpreisbildung}\label{monopolpreisbildung}

Das Kirchenrecht wurde auch zur Stärkung der Monopolpreise eingeführt.
Die Kirche erzielte beträchtliche Einnahmen aus dem Verkauf von Alaun,
der auf ihren Grundstücken in Tolfa, Italien, abgebaut wurde. Als einige
ihrer Kunden in der Textilindustrie den billigeren, aus der Türkei
importierten Alaun bevorzugten, versuchte der Vatikan, seine
Monopolpreise durch das Kirchenrecht aufrechtzuerhalten, indem er die
Verwendung des billigeren Alauns für sündhaft erklärte. Händler, die
darauf bestanden, das billigere türkische Produkt zu kaufen, wurden
exkommuniziert. Das berühmte Verbot, am Freitag Fleisch zu essen,
entsprang demselben Geist. Die Kirche war nicht nur der größte feudale
Grundbesitzer, sondern besaß auch bedeutende Fischereien. Die
Kirchenväter entdeckten eine theologische Notwendigkeit für die Frommen,
Fisch zu essen, was nicht zuletzt die Nachfrage nach ihrem Produkt zu
einer Zeit sicherte, in der die Transport- und Hygienebedingungen vom
Fischkonsum abrieten.

Wie der heutige Nationalstaat regulierte die spätmittelalterliche Kirche
nicht nur bestimmte Wirtschaftszweige, um ihre eigenen Interessen direkt
zu unterstützen, sondern sie nutzte ihre Regulierungsbefugnisse auch, um
sich auf andere Weise Einnahmen zu verschaffen. Die Kleriker gaben sich
besondere Mühe, Vorschriften und Edikte zu erlassen, die nur schwer zu
befolgen waren. So wurde beispielsweise der Begriff Inzest sehr weit
gefasst, so dass selbst entfernte Cousins und Personen, die nur durch
Heirat miteinander verwandt waren, eine Sondergenehmigung der Kirche
benötigten, um zu heiraten. Da dies in vielen kleinen europäischen
Dörfern vor dem Zeitalter des modernen Reisens fast jeden betraf, wurde
der Verkauf von Befreiungen für inzestuöse Ehen zu einer florierenden
Einnahmequelle der Kirche. Sogar der Sex in der Ehe selbst war durch
kirchliche Vorschriften stark eingeschränkt. Sexuelle Beziehungen
zwischen Ehepartnern waren sonntags, mittwochs und freitags sowie in den
vierzig Tagen vor Ostern und Weihnachten verboten. Außerdem mussten sich
die Eheleute vor dem Empfang der Kommunion drei Tage lang des
Geschlechtsverkehrs enthalten. Mit anderen Worten: Ehepaaren war es
verboten, an mindestens 55 Prozent der Tage im Jahr ohne Ablass Sex zu
haben. In The Bishop\textquotesingle s Brothels (Die Bordelle des
Bischofs) legt der Historiker E. J. Burford nahe, dass diese
„idiotischen`` Vorschriften für die Ehe dazu beitrugen, das Wachstum der
mittelalterlichen Prostitution anzukurbeln, wovon die Kirche mächtig
profitierte.{[}\^{}128{]} Burford berichtet, dass der Bischof von
Winchester viele Jahrhunderte lang der Direktor der Londoner
Bankside-Bordelle in Southwark war. Außerdem war der kirchliche Profit
aus der Prostitution keineswegs nur eine lokale englische Angelegenheit:

\begin{quote}
Papst Sixtus IV. (ca. 1471), der sich angeblich bei einer seiner
zahlreichen Mätressen mit Syphilis angesteckt hatte, war der erste
Papst, der Lizenzen für Prostituierte ausstellte und eine Steuer auf
deren Einkünfte erhob, wodurch sich die päpstlichen Einnahmen
beträchtlich vermehrten. In der Tat finanzierte die römische Kurie den
Bau des Petersdoms teilweise durch diese Steuer und den Verkauf von
Lizenzen. Sein Nachfolger, Papst Leo X., soll durch den Verkauf von
Lizenzen etwa zweiundzwanzigtausend Golddukaten eingenommen haben,
viermal so viel wie durch den Verkauf von Ablassbriefen in
Deutschland.\footnote{Ebenda, S. 102.}
\end{quote}

Selbst die berühmte Zölibatsregel, die den Priestern auferlegt wurde,
war eine lukrative Einnahmequelle für die mittelalterliche Kirche. Wie
Burford berichtet, verhängte die Kirche „eine Gaunerei, die als
cullagium bekannt ist``, eine Gebühr, die „konkubinären Priestern``
auferlegt wurde.\footnote{Ebenda.} Dies erwies sich als so lukrativ,
dass sie von den Bischöfen in Frankreich und Deutschland einheitlich
allen Priestern auferlegt wurde, obwohl das Laterankonzil 1215 „diesen
schändlichen Handel, durch den solche Prälaten regelmäßig die Erlaubnis
zur Sünde verkaufen`` angeprangert hatte.\footnote{Ebenda, S. 103.} Es
handelte sich lediglich um einen von vielen lukrativen Märkten für den
Verkauf von Lizenzen zur Verletzung des kanonischen Rechts und der
kirchlichen Vorschriften, ein Handel, der von der gleichen Logik
motiviert war, die habgierige Politiker dazu treibt, willkürliche
Regulierungsbefugnisse über den Handel anzustreben.

\subsection{Ablässe}\label{abluxe4sse}

Die Befugnis, willkürlich zu regulieren, ist auch die Befugnis, eine
Befreiung von dem Schaden zu verkaufen, den solche Regulierungen
anrichten können. Die Kirche verkaufte Genehmigungen oder „Ablässe``,
die alles erlaubten, von der Befreiung von geringfügigen Belastungen des
Handels bis zur Erlaubnis, in der Fastenzeit Milchprodukte zu essen.
Diese „Ablässe`` wurden nicht nur zu hohen Preisen an den Adel und die
reichen Bürger verkauft. Sie wurden auch als Lotteriepreise verpackt,
ähnlich wie die staatlichen Lotterien von heute, um die Pfennige der
Armen anzulocken.\footnote{Huizinga, ebenda, S. 151.} Der Handel mit
Ablassbriefen nahm zu, als die Ausgaben der Kirche ihre Einnahmen
überstiegen. Daraus schlossen viele auf das Offensichtliche, dass die
institutionelle Kirche ihre Macht in erster Linie dazu nutzte, Einnahmen
zu erzielen. Ein zeitgenössischer Kritiker drückte es so aus: „Das
Kirchenrecht wurde einzig und allein zu dem Zweck eingeführt, viel Geld
zu verdienen; wer ein Christ sein will, muss sich von seinen
Bestimmungen freikaufen.`` \footnote{Cameron, ebenda, S. 31.}

\subsection{Bürokratische Überlast}\label{buxfcrokratische-uxfcberlast}

Die Kosten für die Unterstützung der institutionalisierten Religion
hatten am Ende des fünfzehnten Jahrhunderts ein historisches Extrem
erreicht, ähnlich wie die Kosten für die Unterstützung der Regierung
heute ein altersschwaches Extrem erreicht haben. Je mehr das Leben von
der Religion gesättigt war, desto teurer und bürokratischer wurde die
Kirche. In Camerons Worten: „Es war viel einfacher, Menschen zu finden,
die die enorm gestiegene Zahl der kirchlichen Ämter am Ende des
Mittelalters besetzen konnten, als Geld zu finden, um sie zu bezahlen.``
\footnote{Ebenda, S. 24.} So wie bankrotte Regierungen heute auf
kontraproduktive Weise nach Einnahmen suchen, tat dies die Kirche vor
fünfhundert Jahren. In der Tat bedienten sich die Kirchenmänner einiger
der gleichen räuberischen Tricks, die die Politiker heute beherrschen.

Die mittelalterliche Kirche vor fünfhundert Jahren verbrauchte ebenso
wie der heutige Nationalstaat mehr Ressourcen der Gesellschaft als je
zuvor und jemals wieder. Die Kirche schien damals, wie der Staat heute,
selbst mit Rekordeinnahmen nicht in der Lage zu sein zu funktionieren
und sich selbst zu erhalten. So wie der Staat die spätindustriellen
Volkswirtschaften dominiert und in einigen westeuropäischen Ländern mehr
als die Hälfte aller Einnahmen ausgibt, so dominierte die Kirche die
spätfeudale Wirtschaft, verschlang Ressourcen und bremste das Wachstum.

\subsection{Defizit-Ausgaben im fünfzehnten
Jahrhundert}\label{defizit-ausgaben-im-fuxfcnfzehnten-jahrhundert}

Die Kirche griff zu allen erdenklichen Mitteln, um mehr Geld aus ihren
Einnahmen herauszupressen und ihre überbordende Bürokratie zu füttern.
Die Regionen, die direkt der Kirche unterstellt waren, mussten immer
höhere Steuern zahlen. In Provinzen und Königreichen, in denen die
Kirche keine direkte Besteuerungsbefugnis besaß, erließ der Vatikan
„Annaten``, eine Zahlung, die der örtliche Herrscher anstelle der
direkten Kirchensteuern zu leisten hatte.

Wie heute der Staat, so plünderte auch die Kirche ihre eigenen Kassen,
indem sie Gelder aus zweckgebundenen Wohltaten zur Deckung allgemeiner
Gemeinkosten abzweigte. Benefizien und käufliche religiöse Ämter wurden
offen verkauft, ebenso wie die Einnahmen aus dem Zehnten. In der Tat
wurden die Zinsen auf den Zehnten zum kirchlichen Äquivalent der
Anleihen, die von modernen Regierungen zur Finanzierung ihrer
chronischen Defizite ausgegeben wurden.

Während die Kirche als ideologische Verteidigerin des Feudalismus und
Kritikerin von Handel und Kapitalismus auftrat, bediente sie sich, wie
heute der Nationalstaat, jeder verfügbaren Marketingtechnik, um ihre
eigenen Einnahmen zu optimieren. Die Kirche betrieb ein florierendes
Geschäft mit dem Verkauf von Sakramentalien, darunter geweihte Kerzen,
Palmen, die am Palmsonntag gesegnet wurden, „Kräuter, die am Fest Mariä
Himmelfahrt gesegnet wurden, und vor allem die verschiedenen Arten von
Weihwasser.`` \footnote{Ebenda, S. 15.}

Ähnlich wie heutige Politiker, die ihren Wählern mit der Einschränkung
der Müllabfuhr und anderen Demütigungen drohen, wenn sie sich weigern,
höhere Steuern zu zahlen, neigten auch religiöse Autoritäten im
fünfzehnten Jahrhundert dazu, Gottesdienste zu unterbrechen, um
Gemeinden zur Zahlung willkürlicher Geldstrafen zu erpressen. Oft wurden
die Bußgelder für ein geringfügiges Vergehen einiger weniger Personen
verhängt, die nicht einmal Mitglieder der betreffenden Gemeinde gewesen
sein mussten. Im Jahr 1436 schloss Bischof Jacques Du Chatelier, „ein
sehr prahlerischer und habgieriger Mann``, die Kirche der Unschuldigen
in Paris für zweiundzwanzig Tage und unterbrach alle Gottesdienste, um
auf die Zahlung einer unvorstellbar hohen Geldstrafe durch zwei Bettler
zu warten. Die Männer hatten sich in der Kirche gestritten und einige
Blutstropfen vergossen, von denen der Bischof behauptete, sie hätten die
Kirche entweiht. Er erlaubte niemandem, die Kirche für Hochzeiten,
Beerdigungen oder die normalen Sakramente des Kalenders zu benutzen, bis
die Geldstrafe bezahlt war.\footnote{Huizinga, ebenda, S. 27.}

\begin{quote}
Die italienischen Stewes (um zu bringen dem Papst seinen Spaß)\\
zahlten gar zwanzigtausend Dukaten während eines Jahrs.\\
Außerdem geben sie einem Priester (in die Tasche hinein)\\
den Profit einer Hure, vielleicht auch von zweien oder dreien.\ldots{}\\
Mich dünkt, es muss bald eine Pause machen\\
wer es mit den Stewes derartig lässt krachen.\footnote{Burford, ebenda,
  S. 103.}

- Englische Ballade des fünfzehnten Jahrhunderts
\end{quote}

\subsection{Hass auf Kirchenführer}\label{hass-auf-kirchenfuxfchrer}

Kein Wunder, dass die allgemeine Meinung des späten fünfzehnten
Jahrhunderts den höheren und niederen Klerus verachtete, so wie die
allgemeine Meinung in hoch politisierten Gesellschaften heute die
Bürokratie und die Politiker verachtet. Johan Huizinga drückte es so
aus: „Hass ist das richtige Wort in diesem Zusammenhang, denn der Hass
war latent, aber allgemein und anhaltend. Das Volk wurde nicht müde, die
Laster des Klerus anzuklagen.`` \footnote{Huizinga, ebenda, S. 173.} Die
allgemeine Überzeugung, dass die Kirche „habgierig und
verschwenderisch`` sei, war auch deshalb richtig, weil sie stimmte. „Die
Weltlichkeit der höheren Ränge des Klerus und der Verfall der unteren
Klassen`` \footnote{Ebenda.} waren zu offensichtlich, um sie zu
übersehen. Vom Pfarrer bis zum Papst selbst scheint der Klerus korrupt
zu sein, wie es nur das Personal einer herrschenden Institution sein
kann.

Vor fünfhundert Jahren ließ Papst Alexander VI. sogar Giulio Andreotti
und Bill Clinton wie Vorbilder der Integrität erscheinen. Alexander VI.
war für seine wilden Partys berühmt. Als Kardinal in Siena veranstaltete
er eine berühmte Orgie, zu der nur „die schönsten jungen Frauen Sienas
eingeladen, ihre ‚Ehemänner, Väter und Brüder' aber ausgeschlossen
waren.`` \footnote{William Manchester, \emph{A World Lit Only by Fire:
  The Medieval Mind and the Renaissance} (Boston: Little, Brown, 1992),
  S. 75-76.} Die Orgie in Siena war berühmt, aber sie erwies sich später
als zahm im Vergleich zu denen, die Alexander veranstaltete, nachdem er
Papst wurde. Die vielleicht reißerischste dieser Orgien war das so
genannte Kastanienballett, bei dem die „fünfzig schönsten Huren Roms``
in einen Kopulationswettbewerb mit den Kirchenvätern und anderen
wichtigen Römern verwickelt waren. William Manchester beschreibt es so:
„Die Diener führten Buch über die Orgasmen jedes Mannes, denn der Papst
bewunderte die Potenz sehr... Nachdem alle erschöpft waren, verteilte
Seine Heiligkeit Preise - Mäntel, Stiefel, Mützen und feine seidene
Tuniken. Die Gewinner, so notierte der Tagebuchschreiber, waren
diejenigen, die am häufigsten mit den Kurtisanen geschlafen hatten.``
\footnote{Ebenda., S. 79.}

Alexander zeugte mindestens sieben, vielleicht sogar acht uneheliche
Kinder. Einer seiner angeblichen Söhne, Giovanni, war der so genannte
Infans Romanus, der von Alexanders unehelicher Tochter Lucrezia Borgia
geboren wurde, als diese achtzehn Jahre alt war. In einer geheimen
päpstlichen Bulle gab Alexander zu, Giovanni gezeugt zu haben. Wenn er
nicht der Vater war, war er mit Sicherheit der Großvater auf beiden
Seiten. Der Papst war in eine inzestuöse Dreierbeziehung mit Lucrezia
verwickelt, die auch die Geliebte von Juan, Herzog von Gandia,
Alexanders ältestem unehelichen Sohn, sowie die Geliebte eines anderen
unehelichen Sohnes, Kardinal Cesare Borgia, war. Cesare war der
Kirchenfürst, der Niccolo Machiavelli als Inspiration für Der Fürst
diente. Cesare war ein Mörder, ebenso wie der Papst, von dem bekannt
war, dass er mehrere Morde geplant hatte. Der eine oder andere von ihnen
wurde offenbar eifersüchtig auf Juan, dessen lebloser Körper am 15. Juni
l497 aus dem Tiber gefischt wurde.\footnote{Ebenda, S. 82-84.}

Die Führung der spätmittelalterlichen Kirche war ebenso korrupt wie die
Führung des heutigen Nationalstaates.

\begin{quote}
„Heute bin ich zweimal Vater geworden, mit dem Segen der Götter.``
\footnote{Huizinga, ebenda, S. 154.} - Rodolph Acricola, als er hörte,
dass seine Konkubine an dem Tag, an dem er zum Abt gewählt wurde, einen
Sohn zur Welt gebracht hatte.
\end{quote}

\section{HEUCHELEI}\label{heuchelei}

Unter einer „oberflächlichen Kruste der Frömmigkeit`` war die
spätmittelalterliche Gesellschaft bemerkenswert blasphemisch, respektlos
und ausschweifend. Die Kirchen waren die bevorzugten Treffpunkte junger
Männer und Frauen und häufige Treffpunkte von Prostituierten und
Verkäufern obszöner Bilder. Historiker berichten, dass „die
Respektlosigkeit der täglichen religiösen Praxis fast grenzenlos war.``
\footnote{Ebenda.} Chorsänger, die für die Seelen der Verstorbenen
angeheuert wurden, ersetzten in der Regel profane Worte in der Messe.
Vigilien und Prozessionen, die in der religiösen Praxis des Mittelalters
eine weitaus größere Rolle spielten als heute, wurden dennoch „durch
Schimpfworte, Spott und Trinkgelage entehrt.`` So sagte die führende
theologische Autorität des spätmittelalterlichen Europas, Denis der
Kartäuser.{[}\^{}145{]}

Auch wenn man einen solchen Bericht als das Gezeter eines steifen
Moralisten abtun könnte, so ist er doch nur einer von vielen Berichten,
die das gleiche Bild zeichnen. Es gibt reichlich Grund zu der Annahme,
dass das Unzüchtige und das Heilige im mittelalterlichen Leben häufig
eng miteinander verbunden waren. Pilgerfahrten zum Beispiel arteten so
oft in Krawall und Ausschweifungen aus, dass hochgesinnte Reformer
erfolglos dafür plädierten, sie zu unterbinden. Auch lokale religiöse
Prozessionen boten dem Mob regelmäßig Gelegenheit zu Vandalismus,
Plünderungen und allgemein zu allen möglichen betrunkenen Vergnügungen.
Selbst wenn die Menschen still saßen, um die Messe zu hören, war dies
häufig keine nüchterne Erfahrung. Vor allem in den Festnächten wurden in
der Kirche gewaltige Mengen Wein konsumiert. Aus Berichten des
Straßburger Konzils geht hervor, dass die „Betenden`` in der
St.~Adolphus-Nacht 1.100 Liter Wein tranken, den das Konzil zu Ehren des
Heiligen spendiert hatte.

Jean Gerson, ein bedeutender Theologe des 15. Jahrhunderts, berichtet,
dass „die heiligsten Feste, sogar die Weihnachtsnacht mit
Ausschweifungen, Kartenspielen, Fluchen und Gotteslästerung`` verbracht
wurden. Wenn das gemeine Volk wegen dieser Verfehlungen „ermahnt wird,
beruft es sich auf das Beispiel des Adels und des Klerus, die sich
ungestraft in gleicher Weise verhalten.`` \footnote{Ebenda.}

\subsection{Frömmigkeit und
Barmherzigkeit}\label{fruxf6mmigkeit-und-barmherzigkeit}

Die Frömmigkeit, die im späten Mittelalter die Sättigung der
Gesellschaft durch die organisierte Religion rationalisierte, diente
demselben Zweck wie die „Barmherzigkeit``, die heute die politische
Beherrschung des Lebens rechtfertigen soll. Der Verkauf von
Ablassbriefen, zur Befriedigung des Wunsches nach Frömmigkeit ohne
Moral, ist vergleichbar mit den verschwenderischen Ausgaben für die
Wohlfahrt, die den Anschein von Barmherzigkeit ohne Wohltätigkeit
erwecken sollen. Es war weitgehend unerheblich, ob die tatsächliche
Wirkung der empfangenen Praktiken darin bestand, den moralischen
Charakter zu verbessern oder Seelen zu retten, so wie es weitgehend
unerheblich ist, ob ein Wohlfahrtsprogramm tatsächlich das Leben der
Menschen verbessert, an die es gerichtet ist. „Frömmigkeit`` war ebenso
wie „Barmherzigkeit`` eine fast abergläubische Beschwörung.

In einer Zeit, in der kausale Zusammenhänge kaum verstanden wurden,
durchdrangen die Rituale und Sakramente der Kirche jede Phase des
Lebens. „. . . Eine Reise, eine Aufgabe, ein Besuch waren gleichermaßen
von tausend Formalitäten begleitet: Segnungen, Zeremonien, Formeln.``
\footnote{Ebenda., S. 9.} Auf Pergamentstücke geschriebene Gebete wurden
den Fieberkranken wie Halsketten umgehängt. Unterernährte Mädchen
drapierten ihre Haarsträhnen vor dem Bild des Heiligen Urban, um
weiteren Haarausfall zu verhindern. In Navarra zogen die Bauern in
Prozessionen hinter einem Bild des Heiligen Petrus her, um bei
Dürreperioden Regen zu erbitten.\footnote{Diese Beispiele religiöser
  Rituale stammen von Cameron, ebenda, S. 10-11.} Diese und andere
„unwirksame Techniken zur Linderung von Ängsten, wenn wirksame nicht zur
Verfügung standen``, wurden von den Menschen eifrig
angenommen.\footnote{Keith Thomas, \emph{Religion and the Decline of
  Magic} (London: Penguin, 1971), S. .800, zitiert in Cameron, ebenda,
  S. 10.}

\subsection{Zwei Unrechte für einen
Ritus}\label{zwei-unrechte-fuxfcr-einen-ritus}

Die Menschen waren so fest von den wundertätigen Eigenschaften der
Reliquien von Heiligen überzeugt, dass der Tod einer besonders frommen
Person häufig Anlass zu einem wahnsinnigen Ansturm auf die Teilung des
Leichnams war. Nachdem Thomas von Aquin im Kloster Fossanuova gestorben
war, enthaupteten die Mönche dort seinen Körper und kochten ihn, um in
den Besitz seiner Gebeine zu gelangen. Als die heilige Elisabeth von
Ungarn aufgebahrt war, „kam eine Schar von Verehrern und schnitt oder
riss Streifen aus dem Leinen, das ihr Gesicht umhüllte; sie schnitten
das Haar, die Nägel und sogar die Brustwarzen ab.`` \footnote{Huizinga,
  ebenda, S. 161.}

\subsection{Frömmigkeit ohne Tugend}\label{fruxf6mmigkeit-ohne-tugend}

Der mittelalterliche Geist betrachtete die Heiligen und ihre Reliquien
als Teil des Arsenals des Glaubens in einer Welt, die im Winter kälter,
nachts dunkler und angesichts von Krankheiten verzweifelter war, als
jeder Leser dieses Buches wahrscheinlich je erlebt hat. Stärker als in
der Neuzeit glaubten die Menschen im Mittelalter an die Existenz von
Dämonen, an ein aktives Eingreifen Gottes in die Welt und daran, dass
Gebet, Buße und Pilgerfahrten göttliche Gunst einbringen.

Einfach zu sagen, dass die Menschen an Gott glaubten, könnte weder die
Intensität ihres Bekenntnisses noch die scheinbare Leichtigkeit
vermitteln, mit der sich die mittelalterliche Frömmigkeit mit der Sünde
zu arrangieren schien. Der Glaube an die Wirksamkeit von Riten, Ritualen
und Sakramenten war so allgegenwärtig, dass er die Dringlichkeit eines
tugendhaften Verhaltens vielleicht zwangsläufig untergrub. Für jede
Sünde oder jeden geistigen Defekt gab es ein Heilmittel, eine Buße, die
die Schiefertafel reinigte, was zu einer „Heilsmathematik``
wurde.{[}\^{}151{]} Die Religion wurde so allgegenwärtig, dass ihre
Aufrichtigkeit zwangsläufig zu schwinden begann. Wie Huizinga es
formulierte: „Die Religion, die alle Beziehungen im Leben durchdringt,
bedeutet eine ständige Vermischung der Sphären des heiligen und des
profanen Denkens. Heilige Dinge werden zu gewöhnlich, um tief empfunden
zu werden.`` \footnote{Huizinga, ebenda, S. 148.} Und so war es auch.

\section{VERKLEINERUNG DER KIRCHE}\label{verkleinerung-der-kirche}

Am Ende des fünfzehnten Jahrhunderts war die Kirche nicht nur so korrupt
wie heute der Nationalstaat, sie war auch ein großer Hemmschuh für das
Wirtschaftswachstum. Die Kirche verschlang große Mengen an Kapital auf
unproduktive Weise, indem sie Lasten auferlegte, die die Leistung der
Gesellschaft begrenzten und den Handel unterdrückten. Diese Belastungen
waren ebenso zahlreich wie die, die der Nationalstaat heute auferlegt.
Wir wissen, was mit der organisierten Religion im Gefolge der
Schießpulverrevolution geschah: Sie schuf starke Anreize, die religiösen
Institutionen zu verkleinern und ihre Kosten zu senken. Als die
traditionelle Kirche sich weigerte, dies zu tun, ergriffen
protestantische Sekten die Gelegenheit, um zu konkurrieren. Dabei
setzten sie fast jedes erdenkliche Mittel ein, um die Kosten für ein
frommes Leben zu senken:

\begin{itemize}
\item
  Sie bauten sparsame neue Kirchen und entfernten manchmal die Altäre
  älterer Kirchen, um Kapital für andere Zwecke freizusetzen.
\item
  Sie revidierten die christliche Lehre in einer Weise, die die Kosten
  senkte, und betonten den Glauben gegenüber guten Taten als Schlüssel
  zur Erlösung.
\item
  Sie entwickelten eine neue, knappe Liturgie, reduzierten oder
  schafften Festtage ab und schafften zahlreiche Sakramente ab.
\item
  Sie schlossen Klöster und Nonnenklöster und hörten auf, Almosen an
  Bettelorden zu geben. Die Armut wurde von einer apostolischen Tugend
  zu einem unerwünschten und oft tadelnswerten sozialen
  Problem.\footnote{Für weitere Details zu den deutlichen Unterschieden
    zwischen den Perspektiven des fünfzehnten und sechzehnten
    Jahrhunderts in Bezug auf Armut, siehe Robert Jutte, \emph{Poverty
    and Deviance in Early Modern Europe} (Cambridge: Cambridge
    University Press, 1994), S. 15-17.}
\end{itemize}

Um zu verstehen, wie die Verkleinerung der Kirche die Produktivität
freigesetzt hat, muss man sich vor dem Brechen des Monopols der Kirche
die zahlreichen Hindernisse vor Augen führen, die sie dem Wachstum in
den Weg gelegt hat. Ähnlich wie heute der Nationalstaat erlegte die
Kirche am Ende des fünfzehnten Jahrhunderts eine unglaubliche Last an
überschüssigen Kosten auf.

\begin{enumerate}
\def\labelenumi{\arabic{enumi}.}
\item
  Direkte Kosten wie Zehnten, Steuern und Gebühren finanzierten die
  ausufernde kirchliche Bürokratie. Der Zehnte war auch in den
  protestantischen Kirchen üblich, die die mittelalterliche „Heilige
  Mutter Kirche`` ablösten, aber in städtischen Gebieten war er in der
  Regel nicht zu erheben. Das Ende des kirchlichen Monopols führte dazu,
  dass die Grenzsteuersätze in den Regionen mit dem am stärksten
  entwickelten Handel sanken.
\item
  Religiöse Doktrinen machten das Sparen schwierig. Der Erzbösewicht der
  mittelalterlichen Kirche war der „Geizhals``, der sein Gold auf Kosten
  seiner Seele sparte. Die Forderung an die Gläubigen, „gute Taten`` zu
  finanzieren, hatte kostspielige Beiträge an die Kirche zur Folge. Die
  Lehre von den „Erfüllungen`` verpflichtete die um ihr Seelenheil
  Besorgten, Gottesdienste oder „Totenmessen`` zu stiften, um das
  Fegefeuer zu vermeiden. Luther griff dies in der achten und
  dreizehnten seiner fünfundneunzig Thesen direkt an. Er schrieb, dass
  „die Sterbenden alle ihre Schulden durch ihren Tod bezahlen werden.``
  {[}\^{}154{]} Mit anderen Worten: Das Kapital des protestantischen
  Gläubigen stand zur Verfügung, um es an seine Erben weiterzugeben.
  Nach der protestantischen Lehre brauchte man keine Kantoreien zu
  stiften, um Messen zu wiederholen, in der Regel für dreißig Jahre, und
  manchmal, bei sehr wohlhabenden Personen, für immer.
\item
  Die Ideologie der mittelalterlichen Kirche förderte auch die Umleitung
  von Kapital in den Erwerb von Reliquien. Zahlreiche Reliquienkulte
  wurden mit großen Summen ausgestattet, um physische Gegenstände zu
  erwerben, die mit Christus oder verschiedenen Heiligen in Verbindung
  gebracht wurden. Die sehr wohlhabenden Menschen legten sogar
  persönliche Reliquiensammlungen an. Kurfürst Friedrich von Sachsen
  beispielsweise trug eine Sammlung von neunzehntausend Reliquien
  zusammen, von denen er einige 1493 auf einer Pilgerreise nach
  Jerusalem erwarb. In seiner Sammlung befanden sich unter anderem, wie
  er glaubte, „der Leib einer heiligen Unschuldigen, die Milch Marias
  und Stroh aus dem Stall der Geburt Christi``.{[}\^{}155{]} Vermutlich
  war die Rendite für das in diese Reliquien investierte Kapital gering.
  Die Verlagerung auf den Glauben und die Vorstellung von den
  Auserwählten verringerte die Bedeutung des Erwerbs von Gegenständen
  des christlichen Lebens, die als Amulette verwendet werden konnten,
  und ermutigte das Geld, produktivere Kanäle zu finden, die eine
  Rendite abwarfen, die der Monarch nutzen konnte.
\item
  Das Aufkommen der protestantischen Konfessionen brach die
  wirtschaftlichen Monopole der mittelalterlichen Kirche und führte zu
  einer erheblichen Schwächung der Regulierung. Wie wir gesehen haben,
  wurde das kanonische Recht häufig zur Unterstützung kirchlicher
  Monopole und wirtschaftlicher Interessen gebogen. Da die neuen
  Konfessionen weniger wirtschaftliche Interessen zu schützen und zu
  fördern hatten, führte ihre Version der religiösen Lehre tendenziell
  zu einem freieren System mit weniger Hemmungen für den Handel.
\item
  Die protestantische Revolution schaffte viele der Riten und Rituale
  der mittelalterlichen Kirche ab, die die Zeit der Gläubigen
  belasteten. Riten, Sakramente und heilige Tage waren so ausgearbeitet
  worden, dass sie im späten fünfzehnten Jahrhundert fast den gesamten
  Kalender ausfüllten. Diese zeremonielle Überfrachtung war eine
  logische Folge des Beharrens der Kirche darauf, „dass man Gebets- oder
  Gottesdiensthandlungen beliebig oft wiederholen und daraus Nutzen
  ziehen könne.`` \footnote{Ebenda., S. 11.} Das taten sie auch. Die
  Produktivität wurde durch längere und aufwändigere Gottesdienste,
  durch die Verpflichtung, wiederholte Gebete zur Buße zu sprechen, und
  durch die zunehmende Zahl von Heiligenfesten, an denen nicht
  gearbeitet werden durfte, belastet. Zahlreiche Vorschriften und
  Zeremonien unterbrachen den Tag und die Jahreszeiten und schränkten
  die für produktive Aufgaben zur Verfügung stehende Zeit erheblich ein.
  Dies mag den Rhythmus der mittelalterlichen Landwirtschaft, in der 90
  Prozent oder mehr der Bevölkerung tätig waren, kaum unterbrochen
  haben. Während der Jahreszeiten gab es viele Perioden, in denen die
  Feldarbeit nicht täglich erforderlich war. Die Ernteerträge schwankten
  unter mittelalterlichen Bedingungen wahrscheinlich mehr mit dem Wetter
  und unkontrollierbaren Schädlingsbefall als mit einer geringfügigen
  zusätzlichen Arbeit, die über das im Kirchenkalender vorgesehene
  Minimum hinausging.

  Das größere Problem der Produktivitätseinbußen betraf nicht so sehr
  die Landwirtschaft, sondern andere Bereiche. Die zeitlichen
  Anforderungen der Kirche waren weit weniger kompatibel mit dem
  Handwerk, der Industrie, dem Transportwesen, dem Handel oder anderen
  Unternehmen, bei denen Produktivität und Rentabilität entscheidend von
  der für die Aufgabe aufgewendeten Zeit bestimmt wurden.

  Es wird wohl kein Zufall sein, dass der große Übergang am Ende des 15.
  Jahrhunderts zu einer Zeit stattfand, in der die Grundstückspachten
  stiegen und die Reallöhne der Bauern sanken. Der zunehmende
  Bevölkerungsdruck hatte die Erträge des Gemeindelandes verringert, das
  oft in der Nähe von Flüssen und Bächen lag und von dem die Bauern für
  die Weidehaltung ihres Viehs und in einigen Fällen für Fisch und
  Brennholz abhängig waren. Die Verringerung des Lebensstandards setzte
  die Bauern zunehmend unter Druck, alternative Einkommensquellen zu
  finden. Infolgedessen „wandte sich ein immer größerer Teil der
  Landbevölkerung der Kleinproduktion für den Markt zu, vor allem der
  Textilherstellung, einem Prozess, der als ‚Auslagerung' oder
  ‚Protoindustrialisierung' bekannt ist``.\footnote{Ebenda, S. 5.} Die
  von der Kirche auferlegten zeremoniellen Zeitlasten standen den
  Bemühungen der ehrgeizigeren Bauern im Wege, ihr bäuerliches Einkommen
  durch handwerkliche Arbeit aufzubessern, denn sie verhinderten jede
  Umschichtung der Anstrengungen in neue wirtschaftliche Richtungen.

  Einer der wichtigsten Beiträge der protestantischen Sekten zur
  Produktivität war die Abschaffung von vierzig Festtagen. Dies sparte
  nicht nur die beträchtlichen Kosten für die Durchführung der Feste,
  einschließlich der Ausstattung der Dorftische mit Speisen und
  Getränken, sondern setzte auch viel wertvolle Zeit frei. Jeder, der
  die vierzig verbannten Festtage nicht mehr feierte, konnte seine
  jährliche Produktivität um dreihundert oder mehr Arbeitsstunden
  erhöhen. Kurz gesagt, die Abschaffung der zeremoniellen Überlastung in
  der mittelalterlichen Kirche machte den Weg frei für eine
  beträchtliche Steigerung der Produktion, einfach dadurch, dass Zeit
  frei wurde, die sonst für den Handel verloren gegangen wäre.
\item
  Durch den Bruch des kirchlichen Monopols wurden große Mengen an
  Vermögenswerten freigesetzt, die unter kirchlicher Verwaltung nur
  geringe Renditen abwarfen - eine Situation, die offensichtliche
  Parallelen zu den staatlichen Betrieben Ende des 20. Jahrhunderts
  aufweist. Die Kirche war der mit Abstand größte feudale Grundbesitzer.
  Ihr Zugriff auf den Grund und Boden entsprach dem des Staates in stark
  politisierten Gesellschaften - in einigen europäischen Ländern wie
  Böhmen überstieg er 50 Prozent des gesamten Grundbesitzes. Nach
  kanonischem Recht konnte ein Grundstück, das einmal in den Besitz der
  Kirche übergegangen war, nicht mehr veräußert werden. So wuchs der
  kirchliche Grundbesitz stetig an, da die Kirche immer mehr
  testamentarische Schenkungen von den Gläubigen erhielt, um
  verschiedene soziale Einrichtungen, Kantoreien und andere Aktivitäten
  zu finanzieren.

  Obwohl es schwierig ist, die relative Produktivität der kirchlichen
  Betriebe genau zu messen, muss sie am Ende des Mittelalters weitaus
  geringer gewesen sein als in der ersten Hälfte dieser Epoche. Im
  vierzehnten Jahrhundert hatte die zunehmende Betonung der Produktion
  für den Markt gegenüber der Subsistenzlandwirtschaft dazu geführt,
  dass die meisten weltlichen Grundherren von ungebildeten Vorstehern zu
  professionellen Managern wurden, um den Ertrag ihrer Höfe zu
  optimieren. Ihre Anreize führten wahrscheinlich dazu, dass sie den
  Ertrag der kirchlichen Güter, der theoretisch niemandem zum privaten
  Gewinn gereicht hätte, schnell übertrafen. Zweifellos bewirtschafteten
  einige der weltlicheren Fürstbischöfe ihre Güter auf eine Art und
  Weise, die sich von derjenigen der weltlichen Herren nicht
  unterschied. Doch die Produktivität anderer kirchlicher Besitztümer
  hätte sicherlich unter den Fehlern einer gleichgültigen Verwaltung
  durch eine riesige, weit verstreute Institution gelitten, deren
  Nachteile den Nachteilen des staatlichen und kommunalen Eigentums
  heute ähnlich gewesen wären. Es liegt auch auf der Hand, dass die
  Beschlagnahmung der Klöster Ressourcen freisetzte, die nach dem
  Aufkommen des Buchdrucks nicht mehr für die Reproduktion von Büchern
  und Manuskripten benötigt wurden.
\item
  Wie wir in The Great Reckoning beschrieben haben, reagierten einige
  protestantische Sekten sofort auf die Schießpulverrevolution, indem
  sie ihre Lehren in einer Weise änderten, die den Handel förderte, z.
  B. durch die Aufhebung des Verbots des Wuchers oder der Kreditvergabe
  gegen Zinsen. Die ideologische Opposition der mittelalterlichen Kirche
  gegen den Kapitalismus war ein Hemmschuh für das Wachstum. Die
  ideologische Hauptstoßrichtung der kirchlichen Lehren bestand darin,
  den Feudalismus zu stärken, an dem die Kirche als größter
  Feudalgrundbesitzer einen großen Anteil hatte. Bewusst oder unbewusst
  neigte die Kirche dazu, aus ihren eigenen wirtschaftlichen Interessen
  religiöse Tugenden zu machen, während sie gegen die Entwicklung der
  Industrie und des unabhängigen kommerziellen Reichtums kämpfte, die
  dazu bestimmt waren, das feudale System zu destabilisieren. Die
  Verbote von „Avancen`` zum Beispiel galten hauptsächlich für
  Handelsgeschäfte und nicht für feudale Abgaben, und niemals für den
  Verkauf von Ablässen. Die berüchtigten Versuche der Kirche, einen
  „gerechten Preis`` für Waren im Handel festzulegen, führten dazu, dass
  die wirtschaftliche Rentabilität derjenigen Produkte und
  Dienstleistungen, bei denen die Kirche nicht selbst als Produzent
  auftrat, unterdrückt wurde.

  Das Verbot des „Wuchers`` war ein deutliches Beispiel für den
  Widerstand der Kirche gegen kommerzielle Innovationen. Banken und
  Kredite waren für die Entwicklung größerer Handelsunternehmen von
  entscheidender Bedeutung. Indem die Kirche die Verfügbarkeit von
  Krediten einschränkte, bremste sie das Wachstum.
\item
  Auf subtilere Weise trug die Konzentration der neuen Konfessionen auf
  die Bibel als Text dazu bei, sowohl die Denkweise der
  mittelalterlichen Kirche als auch ihre Ideologie zu zerstören. Beide
  stellten Hindernisse für das Wachstum dar. Die kulturelle
  Programmierung des späten Mittelalters ermutigte die Menschen, die
  Welt in Begriffen der symbolischen Ähnlichkeit zu sehen und nicht in
  Ursache und Wirkung. Dies führte zu einem Kurzschluss des Denkens. Es
  wies auch weg von einer kaufmännischen Auffassung des Lebens. Das
  Denken in symbolischen Äquivalenzen lässt sich nicht ohne Weiteres in
  ein Denken in Marktwerten übertragen. „Die drei Stände stehen für die
  Eigenschaften der Jungfrau. Die sieben Kurfürsten des Reiches stehen
  für die Tugenden; die fünf Städte des Artois und des Hennegau, die
  1477 dem Haus Burgund treu blieben, sind die fünf klugen
  Jungfrauen\ldots{} Ebenso stehen die Schuhe für Sorgfalt und Fleiß,
  die Strümpfe für Ausdauer, das Strumpfband für Entschlossenheit usw.``
  \footnote{Huizinga, ebenda.} Wie dieses Beispiel des bedeutenden
  mittelalterlichen Historikers Johan Huizinga andeutet, wurde das
  Denken von Dogmen, starren Symbolen und Allegorien beherrscht, die
  jeden Aspekt des Lebens im Sinne einer hierarchischen Unterordnung
  miteinander verbanden. Jeder Beruf, jedes Teil, jede Farbe, jede Zahl,
  sogar jedes Element der Grammatik war in ein großes System religiöser
  Vorstellungen eingebunden.

  So wurden die alltäglichen Dinge des Lebens nicht im Hinblick auf ihre
  kausalen Zusammenhänge, sondern in Form von statischen Symbolen und
  Allegorien gedeutet. Manchmal wurden Tugenden und Laster
  personifiziert, und jedes Ding stand für etwas, das wiederum für etwas
  anderes stand, und zwar auf eine Art und Weise, die Ursache und
  Wirkung oft eher blockierte als klärte. Um die Dinge noch weiter zu
  verwirren, wurden Beziehungen oft willkürlich in Zahlensystemen
  verknüpft. Die Sieben spielte dabei eine besonders wichtige Rolle. Es
  gab die sieben Tugenden, die sieben Todsünden, die sieben Bitten des
  Vaterunsers, die sieben Gaben des Heiligen Geistes, die sieben Momente
  der Passion, die sieben Seligpreisungen und die sieben Sakramente,
  „dargestellt durch die sieben Tiere und gefolgt von den sieben
  Krankheiten.`` \footnote{Ebenda., S. 199.}
\end{enumerate}

\subsection{Journalismus im fünfzehnten
Jahrhundert}\label{journalismus-im-fuxfcnfzehnten-jahrhundert}

Eine Nachrichtensendung aus dem fünfzehnten Jahrhundert hätte, wenn sie
denn geschrieben worden wäre, keine der klassischen Fragen der
Tatsachenberichterstattung beantwortet, außer indirekt durch
allegorische Personifizierung. Nehmen wir diesen Bericht in einem
privaten Tagebuch über die Burgundermorde im Paris des fünfzehnten
Jahrhunderts:

\begin{quote}
Da erhob sich die Göttin der Zwietracht, die im Turm des bösen Rates
wohnte, und erweckte den Zorn, das wahnsinnige Weib, und die Begierde
und den Zorn und die Rache, und sie griffen zu den Waffen aller Art und
vertrieben die Vernunft, die Gerechtigkeit, das Gedenken an Gott und die
Mäßigung auf das schändlichste. Dann wütete der Wahnsinn, und Mord und
Totschlag töteten, schlugen nieder, töteten, massakrierten alles, was
sie in den Gefängnissen fanden. ... und Habgier steckte ihre Röcke in
den Gürtel zusammen mit Vergewaltigung, ihrer Tochter, und Diebstahl,
ihrem Sohn. ... Danach zogen die benannten Leute unter der Führung ihrer
Göttinnen, nämlich des Zorns, der Habgier und der Rache, die sie durch
alle öffentlichen Gefängnisse von Paris führten, usw.\footnote{Ebenda.,
  S. 203.}
\end{quote}

Die Abkehr vom mittelalterlichen Paradigma trug dazu bei, die Menschen
darauf vorzubereiten, in „modernen`` Begriffen über Ursache und Wirkung
zu denken, statt in Begriffen symbolischer Verknüpfungen und
allegorischer Personifikationen.

Man muss nicht behaupten, dass die Lehre und die Denkweise der
spätmittelalterlichen Kirche unaufrichtig waren, um zu erkennen, dass
sie sich eng an die Bedürfnisse des Agrarfeudalismus anlehnten und dem
Handel, geschweige denn der industriellen Entwicklung, nur wenig Raum
ließen. Vielmehr war es so, dass die Kirche als vorherrschende
Institution moralische, kulturelle und rechtliche Zwänge in einer Weise
formte, die eng an die Erfordernisse des Feudalismus angepasst war.
Genau aus diesem Grund waren sie für die Bedürfnisse der
Industriegesellschaft schlecht geeignet, so wie die moralischen,
kulturellen und rechtlichen Zwänge des modernen Nationalstaates für die
Erleichterung des Handels im Informationszeitalter schlecht geeignet
sind. Wir glauben, dass der Staat ebenso wie die Kirche revolutioniert
werden muss, um die Verwirklichung des neuen Potenzials zu ermöglichen.

Die protestantische Doktrin, dass der Himmel allein durch den Glauben
und ohne den Nutzen gestifteter Gebete für die Toten erreicht werden
kann, wurde als theologische Frage dargestellt. Doch es war eine
Theologie, die den wirtschaftlichen Gegebenheiten eines neuen Zeitalters
entsprach. Sie entsprach dem offensichtlichen Bedürfnis nach einem
kosteneffizienteren Weg zum Heil in einer Zeit, in der die
Opportunitätskosten, zusätzliches Kapital in die aufgeblähte kirchliche
Bürokratie zu stecken, plötzlich gestiegen waren. Die Menschen störten
sich weniger daran, ihr Geld der Kirche zu geben, wenn es keinen anderen
Ausweg gab. Aber als sie plötzlich die Chance sahen, das Hundertfache
ihres Kapitals für die Finanzierung einer Gewürzreise in den Osten zu
bekommen oder mit einer geringeren, aber immer noch vielversprechenden
Summe von 40 Prozent pro Jahr ein Bataillon für den König zu
finanzieren, suchten sie verständlicherweise die Gnade Gottes dort, wo
ihre eigenen Interessen lagen.

Viele Kaufleute und andere Bürgerliche wurden bald viel reicher, als es
ihre Vorfahren im Feudalismus gewesen waren. Die drastische Erhöhung des
Lebensstandards unter den Kaufleuten und kleinen Manufakturen in der
frühen Neuzeit war bei denjenigen, deren Einkommen und Lebensweise mit
dem Feudalismus zusammengebrochen waren, weitgehend unpopulär. Die
Schwächung des kirchlichen Monopols und die zunehmende politische Macht
der Reichen führten zu einem starken Rückgang der
Einkommensumverteilung. Die Bauern und städtischen Armen, die nicht
unmittelbar von dem neuen System profitierten, waren bitterlich neidisch
auf diejenigen, die profitierten. Huizinga beschrieb die vorherrschende
Haltung, was eine wichtige Parallele zur Informationsrevolution sein
könnte: „Der Hass auf die Reichen, insbesondere auf die Neureichen, die
damals sehr zahlreich waren, ist allgemein.`` \footnote{Ebenda., S. 27.}

Eine ebenso auffällige Parallele ergab sich aus einem enormen Anstieg
der Kriminalität. Der Zusammenbruch der alten Ordnung führt fast immer
zu einem Anstieg der Kriminalität, wenn nicht sogar zur völligen
Anarchie der feudalen Revolution, die wir im letzten Kapitel untersucht
haben. Am Ende des Mittelalters schnellte die Kriminalität ebenfalls in
die Höhe, als die alten Systeme der sozialen Kontrolle zusammenbrachen.
In Huizingas Worten: „Die Kriminalität wurde als eine Bedrohung für die
Ordnung und die Gesellschaft angesehen.`` \footnote{Ebenda., S. 22.} Sie
könnte in Zukunft ebenso bedrohlich sein.

Die moderne Welt entstand in einem Wirrwarr aus neuen Technologien,
neuen Ideen und dem Gestank von Schwarzpulver. Schießpulverwaffen und
eine verbesserte Schifffahrt destabilisierten die militärische Grundlage
des Feudalismus, während neue Kommunikationstechnologien dessen
Ideologie untergruben. Zu den Elementen, die durch die neue Technologie
des Drucks zum Vorschein kamen, gehörte auch die Korruption der Kirche,
deren Hierarchie wie auch deren Mitglieder ohnehin schon ein geringes
Ansehen genossen, in einer Gesellschaft, die paradoxerweise die Religion
in den Mittelpunkt von allem stellte. Es ist ein Paradoxon mit einer
offensichtlichen Parallele in der Ernüchterung über Politiker und
Bürokraten in einer Gesellschaft, die die Politik in den Mittelpunkt von
allem stellt.

Das Ende des fünfzehnten Jahrhunderts war eine Zeit der
Desillusionierung, der Verwirrung, des Pessimismus und der Verzweiflung.
Eine Zeit, die der heutigen sehr ähnlich ist.

\setsubtitle{Demokratie und Nationalismus als Ressourcenstrategien im Zeitalter der Gewalt}

\bookmarksetup{startatroot}

\chapter{DAS LEBEN UND DIE GESUNDHEIT DES
NATIONENSTAATES}\label{das-leben-und-die-gesundheit-des-nationenstaates}

\begin{quote}
„Der Erfolg im Krieg hängt vor allem davon ab, dass man genug Geld hat,
um alles zu beschaffen, was das Unternehmen braucht.`` \footnote{Zitiert
  in Tilly, ebenda, S. 84.} - Robert de Balsac, 1502
\end{quote}

\section{DIE TRÜMMER DER
GESCHICHTE}\label{die-truxfcmmer-der-geschichte}

Am 9. und 10. November 1989 übertrug das Fernsehen in die ganze Welt
Szenen, in denen euphorische Ostberliner die Berliner Mauer mit
Vorschlaghämmern einschlugen. Die jungen Unternehmer in der Menge
sammelten Mauerstücke ein, die später als Souvenir-Briefbeschwerer an
Kapitalisten in aller Welt vermarktet wurden. Mit diesen Relikten wurde
noch jahrelang ein reger Handel betrieben. Noch heute findet man
gelegentlich Anzeigen in kleinen Magazinen, in denen alte
DDR-Betonstücke zu Preisen angeboten werden, die sonst nur für
hochwertiges Silbererz gezahlt werden. Wir sind der Meinung, dass
diejenigen, die die Briefbeschwerer der Berliner Mauer gekauft haben, es
nicht eilig haben sollten, sie zu verkaufen. Sie sind ein Andenken an
etwas Größeres als den Zusammenbruch des Kommunismus. Wir glauben, dass
die Berliner Mauer zum wichtigsten historischen Trümmerhaufen wurde,
seit die Mauern von San Giovanni im Februar 1495 fast fünf Jahrhunderte
zuvor in die Luft gesprengt wurden.\footnote{Siehe John Keegan, \emph{A
  History of Warfare} (London: Hutchinson, 1993), S. 321.}

Die Zerstörung von San Giovanni durch den französischen König Karl VIII.
war die erste Explosion der Schießpulverrevolution. Sie markierte das
Ende der feudalen Phase der Geschichte und den Beginn der
Industrialisierung, wie wir bereits dargelegt haben. Der Fall der
Berliner Mauer markiert einen weiteren historischen Wendepunkt, den
Übergang vom Industriezeitalter zum neuen Informationszeitalter. Nie
zuvor hat es einen so großen symbolischen Triumph der Effizienz über die
Macht gegeben. Als die Mauern von San Giovanni fielen, war dies ein
deutlicher Beweis dafür, dass die wirtschaftliche Rentabilität von
Gewalt in der Welt drastisch gestiegen war. Der Fall der Berliner Mauer
besagt etwas anderes, nämlich dass die Rendite für Gewalt jetzt sinkt.
Dies ist etwas, das nur wenige erkannt haben, das aber dramatische
Folgen haben wird.

Aus Gründen, die wir in diesem Kapitel untersuchen, könnte sich die
Berliner Mauer als weitaus symbolträchtiger für die gesamte Ära des
industriellen Nationalstaates erweisen, als die Menschen in der Menge in
jener Nacht in Berlin oder die Millionen, die aus der Ferne zusahen,
verstanden haben. Die Berliner Mauer wurde zu einem ganz anderen Zweck
gebaut als die Mauern von San Giovanni - um die Menschen im Inneren an
der Flucht zu hindern - nicht um Räuber von außen am Eindringen zu
hindern. Allein diese Tatsache ist ein aufschlussreicher Indikator für
die Zunahme der Macht des Staates vom 15. bis zum 20. Jahrhundert. Und
das in mehr als einer Hinsicht.

Über Jahrhunderte hinweg machte der Nationalstaat alle nach außen
gerichteten Mauern überflüssig und unnötig. Das Zwangsmonopol des
Staates in den Gebieten, in denen er zuerst Fuß fasste, machte diese
sowohl im Inneren friedlicher als auch militärisch eindrucksvoller als
alle Souveränitäten, die die Welt zuvor gesehen hatte. Der Staat nutzte
die Ressourcen, die er aus einer weitgehend entwaffneten Bevölkerung
herausholte, um Kleinkriminelle auszumerzen. Der Nationalstaat wurde zum
erfolgreichsten Instrument der Geschichte für die Aneignung von
Ressourcen. Sein Erfolg beruhte auf seiner überlegenen Fähigkeit, den
Reichtum seiner Bürger auszuschöpfen.

\begin{quote}
„MTV ist mehr als nur ein Anbieter von Musikvideos und ein Werbeträger
für die Plattenindustrie. Es ist das erste wirklich globale Netzwerk,
das erste Netzwerk, das ein einziges Programm in praktisch jedem Land
der Welt ausstrahlt. Dabei schafft MTV für seine Zuschauer, Kinder und
junge Erwachsene, ein einheitliches Gefühl einer gemeinsamen globalen
Realität. Jüngste Untersuchungen haben ergeben, dass junge Menschen auf
der ganzen Welt mehr und mehr nicht nur gemeinsame Pop-Ikonen und einen
gemeinsamen Geschmack teilen, sondern auch gemeinsame Erwartungen an
ihre Karriere, gemeinsame Wertvorstellungen darüber, was im Leben
sinnvoll ist und wovor man sich fürchten muss, ein gemeinsames Gefühl
dafür, dass die Politik bei der Gestaltung ihrer Zukunft weniger wichtig
ist als ihre eigenen Fähigkeiten.`` \footnote{Jim Taylor und Watts
  Wacker, \emph{The 500-Year Delta: What Happens After What Comes Next},
  New York: HarperCollins, 1997, S. 38-39.} - Jim Taylor and Watts
Wacker, The 500-Year Delta: What Happens After What Comes Next
\end{quote}

\subsection{„Love It or Leave It`` (Es sei denn, du bist
reich)}\label{love-it-or-leave-it-es-sei-denn-du-bist-reich}

Noch bevor der Übergang vom Nationalstaat zu den neuen Souveränitäten
des Informationszeitalters vollzogen ist, werden viele Bewohner der
größten und mächtigsten westlichen Nationalstaaten, wie ihre Pendants in
Ost-Berlin im Jahr 1989, versuchen, ihren Weg nach draußen zu finden.
Für die Generationen, die vor dem Zweiten Weltkrieg oder zu Beginn des
Kalten Krieges aufgewachsen sind, ist das Überschreiten von Grenzen
traumatisch. Aber für die neuen Generationen, die sich aus einer
globaleren Perspektive heraus orientieren, ist das Verlassen des Landes,
in dem sie geboren wurden, nicht mehr so undenkbar wie für die Älteren,
denen die Ideologie des Nationalstaates tief in Fleisch und Blut
übergegangen ist. Jim Taylor und Watts Wacker berichten über die
verblüffenden Ergebnisse einer Massenbefragung von 20.000 Mittelschülern
auf fünf Kontinenten. In einer Stichprobe, die während des Schuljahres
1995/96 von der Brainwaves Group, einem New Yorker
Verbraucherforschungsunternehmen, durchgeführt wurde, stimmten neun von
zehn Schülern der Aussage zu, dass „es an mir liegt, das zu bekommen,
was ich vom Leben will``. Noch bemerkenswerter ist, dass „fast die
Hälfte der Teenager sagte, sie würden das Land ihrer Geburt verlassen,
um ihre Ziele zu erreichen`` \footnote{Ebenda, S. 39.}. Vielleicht hat
Bill Clinton, weil er auf die Einstellungen der MTV-Generation
eingestellt war, als erster Präsidentschaftskandidat, der auf MTV
Wahlkampf machte, versucht, es den Amerikanern zu erschweren, „das Land
ihrer Geburt zu verlassen, um ihre Ziele zu erreichen``. 1995, etwa zur
gleichen Zeit, als die Schüler ihre Absicht erklärten, die
Unabhängigkeit anzustreben, schlug der Präsident der Vereinigten Staaten
die Einführung einer Ausreisesteuer vor, einer „Berliner Mauer für
Kapital``, die von wohlhabenden Amerikanern die Zahlung eines
beträchtlichen Lösegelds verlangen würde, um auch nur mit einem Teil
ihres Geldes zu entkommen.

Clintons Lösegeld erinnert nicht nur an die Politik der späten DDR, ihre
Bürger wie Vermögenswerte zu behandeln, sondern auch an die zunehmend
drakonischen Maßnahmen, die ergriffen wurden, um die Finanzlage des im
Niedergang befindlichen Römischen Reiches zu stützen. Diese Passage aus
The Cambridge Ancient History erzählt die Geschichte:

\begin{quote}
So begann das erbitterte Bestreben des Staates, die Bevölkerung bis auf
den letzten Tropfen auszupressen. Da die wirtschaftlichen Ressourcen
nicht ausreichten, kämpften die Starken mit einer Gewalt und
Skrupellosigkeit, die der Herkunft der Machthaber entsprach, und mit
einem an Plünderungen gewöhnten Soldaten, um sich den größten Anteil zu
sichern. Die ganze Härte des Gesetzes wurde auf die Bevölkerung
losgelassen. Soldaten fungierten als Gerichtsvollzieher oder zogen als
Geheimpolizei durch das Land. Die Leidtragenden waren natürlich die
Besitzenden. An ihr Eigentum war relativ leicht heranzukommen, und im
Notfall waren sie die Klasse, von der am häufigsten und schnellsten
etwas erpresst werden konnte.\footnote{\emph{The Cambridge Ancient
  History}, ebenda, S. 263-64.}
\end{quote}

Wenn gescheiterte Systeme die Macht dazu haben, bürden sie denjenigen,
die zu entkommen versuchen, oft strafrechtliche Belastungen auf. Auch
hier zitieren wir The Cambridge Ancient History: „Während die besitzende
Klasse ihr Geld vergrub oder zwei Drittel ihres Besitzes opferte, um der
Obrigkeit zu entkommen, oder sogar so weit ging, ihren gesamten Besitz
aufzugeben, um sich von der Grundsteuer zu befreien, und die nicht
besitzende Klasse davonlief, antwortete der Staat mit zusätzlichem
Druck.``

Daran sollte man denken, wenn man vorausplant. Die Dämmerung von
Staatssystemen war in der Vergangenheit selten ein höflicher, geordneter
Prozess. In Kapitel 2 haben wir die üblen Angewohnheiten der römischen
Steuereintreiber erwähnt. Die große Zahl der deserti agri, der
verlassenen Bauernhöfe, in Westeuropa nach dem Zusammenbruch des
Römischen Reiches war nur ein kleiner Teil eines größeren Problems.
Tatsächlich waren die Abgaben in Gallien und in den Grenzgebieten, zu
denen das heutige Luxemburg und Deutschland gehörten, in der Regel
relativ gering. In der fruchtbarsten Region Roms, in Ägypten, wo die
Landwirtschaft aufgrund der Bewässerung produktiver war, stellte die
Flucht durch die Eigentümer ein noch größeres Problem dar. Die Frage, ob
ein Fluchtversuch unternommen werden sollte, das „ultimum refugium``,
wie es im Lateinischen hieß, wurde zur vorrangigen Frage für fast jeden,
der Eigentum besaß. Aufzeichnungen zeigen, dass „zu den üblichen Fragen,
die in Ägypten an ein Orakel gestellt wurden, drei Standardfragen
gehörten: ‚Werde ich zum Bettler?`, ‚Soll ich fliehen?' und ‚Wird meine
Flucht gestoppt werden?'\,`` \footnote{Cook et al., ebenda, S. 268.}

Clintons Vorschlag sagt: Ja. Es handelt sich um eine frühe Version eines
Fluchthindernisses, das in dem Maße, in dem die fiskalischen Ressourcen
des Nationalstaates schwinden, wahrscheinlich noch schwerwiegender
werden wird. Natürlich ist die erste US-Version einer Ausreisesperre
harmloser als Erich Honeckers Beton und Stacheldraht. Sie ist auch
preissensibler und belastet nur „Milliardäre`` mit einem zu
versteuernden Vermögen von über 600.000 Dollar. Dennoch wurde es mit
ähnlichen Argumenten gerechtfertigt, wie sie einst von Honecker zur
Verteidigung des berühmtesten öffentlichen Bauprojekts der späten
Deutschen Demokratischen Republik vorgebracht wurden. Honecker
behauptete, dass der ostdeutsche Staat eine erhebliche Investition in
potenzielle Flüchtlinge getätigt habe. Er wies darauf hin, dass die
freie Ausreise der Flüchtlinge zu einem wirtschaftlichen Nachteil für
den Staat führen würde, der ihre Anstrengungen in Ostdeutschland
benötige.

Wenn man die Prämisse akzeptiert, dass die Menschen Eigentum des Staates
sind oder sein sollten, war Honeckers Mauer sinnvoll. Berlin ohne Mauer
war für die Kommunisten ein Schlupfloch, so wie die Flucht vor der
amerikanischen Steuerhoheit für Clintons Finanzamt ein Schlupfloch war.
Clintons Argumente über die Flucht von Milliardären waren, abgesehen von
der üblichen Missachtung der Integrität von Zahlen durch einen
Politiker, ähnlich wie die von Honecker, aber etwas weniger logisch,
weil die US-Regierung in der Tat keine großen wirtschaftlichen
Investitionen in wohlhabende Bürger tätigt, die möglicherweise fliehen
wollen. Es geht nicht darum, dass sie auf Staatskosten ausgebildet
wurden und sich davonstehlen wollen, um woanders als Anwalt zu arbeiten.
Die überwältigende Mehrheit derjenigen, für die die Wegzugssteuer gelten
würde, hat ihren Reichtum durch eigene Anstrengungen trotz und nicht
wegen der US-Regierung geschaffen.

Da die obersten 1 Prozent der Steuerzahler heute 28,7 Prozent der
gesamten Einkommenssteuer in den Vereinigten Staaten zahlen, geht es
nicht darum, dass die Reichen eine echte Investition, die der Staat in
ihre Bildung oder ihren wirtschaftlichen Wohlstand getätigt hat, nicht
zurückzahlen können. Das Gegenteil ist der Fall. Diejenigen, die den
größten Teil der Rechnungen bezahlen, zahlen weit mehr als den Wert der
Leistungen, die sie erhalten. Mit einer durchschnittlichen jährlichen
Steuerzahlung von mehr als 125.000 Dollar kosten die Steuern das oberste
1 Prozent der amerikanischen Steuerzahler weit mehr, als sie heute
wissen. Angenommen, sie könnten über einen Zeitraum von 40 Jahren auch
nur eine 10-prozentige Rendite auf die zu viel gezahlten Steuern
erwirtschaften, so würde jede jährliche Steuerüberzahlung von 5.000
Dollar ihr Nettovermögen um 2,2 Millionen Dollar verringern. Bei einer
Rendite von 20 Prozent verringert sich das Nettovermögen um 44 Millionen
Dollar pro 5.000 Dollar Steuerüberschuss.

Mit dem Herannahen des neuen Jahrtausends werden die neuen
megapolitischen Bedingungen des Informationszeitalters immer deutlicher
machen, dass der aus dem Industriezeitalter übernommene Nationalstaat
eine räuberische Institution ist. Mit jedem Jahr, das vergeht, wird er
weniger ein Segen für den Wohlstand als vielmehr ein Hindernis sein, dem
der Einzelne entkommen will. Eine Flucht, die verzweifelte Regierungen
nur ungern zulassen werden. Die Stabilität und sogar das Überleben der
westlichen Wohlfahrtsstaaten hängt davon ab, ob sie in der Lage sind,
weiterhin einen riesigen Teil der weltweiten Gesamtproduktion für die
Umverteilung an eine Untergruppe von Wählern in den OECD-Ländern
abzuziehen. Dies setzt voraus, dass die Steuern, die den produktivsten
Bürgern der derzeit reichen Länder auferlegt werden, zu
supermonopolistischen Sätzen erhoben werden, die Hunderte oder sogar
Tausende Male höher sind als die tatsächlichen Kosten der
Dienstleistungen, die die Regierungen im Gegenzug erbringen.

\section{DAS LEBEN UND STERBEN DES
NATIONALSTAATES}\label{das-leben-und-sterben-des-nationalstaates}

Der Fall der Berliner Mauer war nicht nur ein sichtbares Symbol für den
Tod des Kommunismus. Er war eine Niederlage für das gesamte Weltsystem
der Nationalstaaten und ein Triumph der Effizienz und der Märkte. Der
Dreh- und Angelpunkt der Macht in der Geschichte hat sich verschoben.
Wir glauben, dass mit dem Fall der Berliner Mauer 1989 die Ära des
Nationalstaates zu Ende geht, eine eigentümliche zweihundertjährige
Phase der Geschichte, die mit der Französischen Revolution begann.
Staaten gibt es seit sechstausend Jahren. Doch vor dem neunzehnten
Jahrhundert machten sie nur einen kleinen Teil der weltweiten
Souveränitäten aus. Ihre Vorherrschaft begann und endete in der
Revolution. Die großen Ereignisse von 1789 brachten Europa auf den Weg
zu wirklich nationalen Regierungen. Die großen Ereignisse von 1989
markierten den Tod des Kommunismus und die Behauptung der Kontrolle der
Marktkräfte über die Macht der Massen. Diese beiden Revolutionen, die
genau zweihundert Jahre auseinander liegen, definieren die Ära, in der
der Nationalstaat im System der Großmächte vorherrschte. Die Großmächte
wiederum beherrschten die Welt, indem sie selbst in der entlegensten
Stammesenklave staatliche Systeme verbreiteten oder aufzwangen.

Der Siegeszug des Staates als wichtigstes Mittel zur Organisation von
Gewalt in der Welt war keine Frage der Ideologie. Er wurde durch die
kalte Logik der Gewalt notwendig. Es war, wie wir zu sagen pflegen, ein
megapolitisches Ereignis, das nicht so sehr durch die Wünsche von
Theoretikern und Staatsmännern oder gar durch die Manöver von Generälen
bestimmt wurde, sondern durch die verborgene Hebelwirkung der Gewalt,
die die Geschichte so bewegte, wie Archimedes davon träumte, die Welt zu
bewegen.

Staaten waren in den letzten zweihundert Jahren der Neuzeit die Norm.
Aber im weiteren Verlauf der Geschichte waren Staaten selten. Ihre
Lebensfähigkeit hing immer von außergewöhnlichen megapolitischen
Bedingungen ab. Vor der Neuzeit waren die meisten Staaten „orientalische
Despotien``, Agrargesellschaften in Wüsten, die für ihr Überleben auf
die Kontrolle von Bewässerungssystemen angewiesen waren. Selbst das
Römische Reich war durch seine Kontrolle über Ägypten und Nordafrika
indirekt eine hydraulische Gesellschaft. Aber nicht genug, um zu
überleben. Rom fehlte, wie den meisten vormodernen Staaten, letztlich
die Fähigkeit, die Einhaltung des Gewaltmonopols zu erzwingen, das die
Fähigkeit bietet, Menschen auszuhungern. Außerhalb Afrikas konnte der
römische Staat das Wasser für den Ackerbau nicht abstellen, indem er
ungehorsamen Menschen den Zugang zum Bewässerungssystem verwehrte.
Solche hydraulischen Systeme lieferten mehr Druckmittel für Gewalt als
jede andere megapolitische Konfiguration in der antiken Wirtschaft. Wer
auch immer das Wasser in diesen Gesellschaften kontrollierte, konnte
Beute in einer Größenordnung machen, die fast mit dem prozentualen
Anteil an der Gesamtproduktion vergleichbar ist, den moderne
Nationalstaaten absorbieren.\footnote{Für Weiteres zur Logik von
  hydraulischen Gesellschaften, siehe Karl A. Wittfogel, Oriental
  Despotism: A Comparative Study of Total Power (New Haven: Yale
  University Press, 1957).}

\subsection{Größe vor Effizienz}\label{gruxf6uxdfe-vor-effizienz}

Das Schießpulver ermöglichte es den Staaten, sich leichter über die
Grenzen der Reisfelder und trockenen Flusstäler hinaus auszudehnen. Die
Beschaffenheit der Schießpulverwaffen und der Charakter der
industriellen Wirtschaft führten zu erheblichen Größenvorteilen in der
Kriegsführung. Dies führte zu hohen und steigenden Renditen für Gewalt.
Der Historiker Charles Tilly drückte es so aus: „Staaten, die über die
größten Zwangsmittel verfügten, neigten dazu, Kriege zu gewinnen; die
Effizienz (das Verhältnis von Output zu Input) kam an zweiter Stelle
nach der Effektivität (Gesamtoutput)``.\footnote{Tilly, Ebenda, S.28.}
Da die Regierungen meist in großem Maßstab organisiert waren, brauchten
selbst die wenigen kleinen Souveränitäten, die überlebten, wie Monaco
oder Andorra, die Anerkennung der größeren Staaten, um ihre
Unabhängigkeit zu sichern. Nur große Regierungen, die über immer mehr
Ressourcen verfügten, konnten auf dem Schlachtfeld konkurrieren.

\subsection{Die große unbeantwortete
Frage}\label{die-grouxdfe-unbeantwortete-frage}

Dies bringt uns zu einem der großen ungelösten Rätsel der modernen
Geschichte: Warum der Kalte Krieg, der am Ende des Systems der
Großmächte stand, kommunistische Diktaturen gegen wohlfahrtsstaatliche
Demokratien als seine letzten Kontrahenten antreten ließ. Diese Frage
ist so wenig untersucht worden, dass es vielen tatsächlich plausibel
erschien, als ein Analyst des Außenministeriums, Francis Fukuyama, nach
dem Fall der Berliner Mauer „das Ende der Geschichte`` ausrief. Das
begeisterte Publikum, das seine Arbeit hervorrief, nahm zu viel als
selbstverständlich hin. Offenbar hatten sich weder der Autor noch viele
andere die Mühe gemacht, eine grundlegende Frage zu stellen: Welche
gemeinsamen Merkmale des Staatssozialismus und der wohlfahrtsstaatlichen
Demokratien führten dazu, dass sie die letzten Anwärter auf die
Weltherrschaft waren?

Dies ist ein wichtiges Thema. Schließlich sind in den letzten fünf
Jahrhunderten Dutzende von konkurrierenden Systemen der Souveränität
entstanden und wieder verschwunden, darunter absolute Monarchien,
Stammesenklaven, Fürstbistümer, direkte Herrschaft des Papstes,
Sultanate, Stadtstaaten und Täuferkolonien. Heute wären die meisten
Menschen überrascht, wenn sie wüssten, dass eine
Krankenhausverwaltungsgesellschaft mit ihren eigenen Streitkräften ein
Land jahrhundertelang regieren könnte. Und doch ist etwas Ähnliches
passiert. Nach 1228 herrschten die Deutschordensritter des
St.~Marienhospitals zu Jerusalem, die später mit den Rittern des
Schwertes von Livland vereinigt wurden, dreihundert Jahre lang über
Ostpreußen und verschiedene Gebiete in Osteuropa, darunter Teile von
Litauen und Polen. Dann kam die Schießpulverrevolution. Innerhalb
weniger Jahrzehnte wurden die Deutschordensritter als Herrscher all
ihrer Territorien vertrieben, und ihr Großmeister war militärisch nicht
mehr von Bedeutung als ein Schachmeister. Und warum? Warum sind so viele
andere Souveränitätssysteme zur Bedeutungslosigkeit geschrumpft, während
im großen Kampf um die Weltmacht am Ende des Industriezeitalters
Massendemokratien gegen staatssozialistische Systeme antraten?

\subsection{Uneingeschränkte
Kontrolle}\label{uneingeschruxe4nkte-kontrolle}

Wenn unsere Theorie der Megapolitik richtig ist, ist die Antwort
einfach. Es ist in etwa so, als würde man fragen, warum Sumoringer dazu
neigen, dick zu sein. Die Antwort ist, dass ein schlanker Sumoringer,
wie beeindruckend sein Verhältnis von Kraft zu Gewicht auch sein mag,
nicht mit einem anderen Ringer konkurrieren kann, der gigantisch ist.
Wie Tilly andeutet, geht es um die „Effektivität (Gesamtleistung)``,
nicht um die „Effizienz (Verhältnis von Leistung und Aufwand)``. In
einer zunehmend von Gewalt geprägten Welt waren die Systeme, die sich in
fünf Jahrhunderten des Wettbewerbs durchsetzten, notwendigerweise
diejenigen, die den größten Zugang zu den Ressourcen ermöglichten, die
für die Führung von Kriegen in großem Maßstab benötigt wurden.

Wie hat das funktioniert?

Im Fall des Kommunismus liegt die Antwort auf der Hand. Im Kommunismus
kontrollierten diejenigen, die den Staat kontrollierten, fast alles.
Wenn Sie während des Kalten Krieges Bürger der Sowjetunion gewesen
wären, hätte der KGB Ihnen Ihre Zahnbürste wegnehmen können, wenn er es
für seine Zwecke für nützlich gehalten hätte. Er hätte Ihnen auch Ihre
Zähne wegnehmen können. Glaubwürdigen Schätzungen zufolge, die seit der
Öffnung ehemaliger sowjetischer Archive im Jahr 1992 noch glaubwürdiger
geworden sind, haben die Geheimpolizei und andere Agenten des späten
Sowjetstaates in den 74 Jahren ihrer Herrschaft 50 Millionen Menschen
das Leben genommen. Das staatssozialistische System war in der Lage,
alles, was innerhalb seiner Grenzen existierte, für sein Militär zu
mobilisieren, ohne dass irgendjemand, der dort lebte, etwas dagegen
einzuwenden gehabt hatte.

Im Falle der westlichen Demokratien ist die Geschichte weniger
offensichtlich, was zum Teil daran liegt, dass wir daran gewöhnt sind,
die Demokratie im krassen Gegensatz zum Kommunismus zu sehen. Im
Hinblick auf das Industriezeitalter waren die beiden Systeme tatsächlich
große Gegensätze. Aber aus der Perspektive des Informationszeitalters
betrachtet, hatten die beiden Systeme mehr gemeinsam, als man vermuten
würde. Beide ermöglichten eine ungehinderte Kontrolle der Ressourcen
durch den Staat. Der Unterschied bestand darin, dass der demokratische
Wohlfahrtsstaat noch mehr Ressourcen in die Hände des Staates legte als
die staatssozialistischen Systeme.

Dies ist ein eindeutiges Beispiel für ein seltenes Phänomen: Weniger ist
mehr. Das staatssozialistische System beruhte auf der Doktrin, dass dem
Staat alles gehört. Der demokratische Wohlfahrtsstaat hingegen erhebt
bescheidenere Ansprüche und setzt damit bessere Anreize, um mehr
Leistung zu mobilisieren. Anstatt von Anfang an Anspruch auf alles zu
erheben, erlaubten die Regierungen im Westen dem Einzelnen, Eigentum zu
besitzen und Wohlstand anzuhäufen. Nachdem der Reichtum angehäuft worden
war, besteuerten die westlichen Nationalstaaten einen großen Teil des
Vermögens. Durch hohe Vermögens-, Einkommens- und Erbschaftssteuern
wurden dem demokratischen Wohlfahrtsstaat im Vergleich zu den
staatssozialistischen Systemen ungeheure Mengen an Ressourcen zur
Verfügung gestellt.

\subsection{Ineffizienz, wo es darauf
ankam}\label{ineffizienz-wo-es-darauf-ankam}

Im Vergleich zum Kommunismus war der Wohlfahrtsstaat tatsächlich ein
weitaus effizienteres System. Aber im Vergleich zu anderen Systemen zur
Anhäufung von Reichtum, wie etwa einer echten Laissez-faire-Enklave wie
Hongkong, war der Wohlfahrtsstaat ineffizient. Auch hier gilt: Weniger
war mehr. Es war genau diese Ineffizienz, die den Wohlfahrtsstaat unter
den megapolitischen Bedingungen des Industriezeitalters übermächtig
machte.

Wenn man begreift, warum das so ist, kommt man der Erkenntnis viel
näher, was der Fall der Berliner Mauer und der Untergang des Kommunismus
wirklich bedeuten. Der Fall der Berliner Mauer und der Untergang des
Kommunismus sind keine Garantie für den Siegeszug des demokratischen
Wohlfahrtsstaates, wie weithin angenommen wurde, sondern es ist eher so,
als ob ein zweieiiger Zwilling an Altersschwäche gestorben wäre. Die
gleiche megapolitische Revolution, die dem Kommunismus den Garaus
gemacht hat, wird wahrscheinlich auch die demokratischen
Wohlfahrtsstaaten, wie wir sie im zwanzigsten Jahrhundert kannten,
untergraben und zerstören.

\section{WER KONTROLLIERT DIE
REGIERUNG?}\label{wer-kontrolliert-die-regierung}

Der Schlüssel zu dieser unorthodoxen Schlussfolgerung liegt in der
Erkenntnis, wo die Kontrolle der demokratischen Regierung angesiedelt
ist. Das ist eine Frage, die nicht so einfach ist, wie es vielleicht
scheint. In der modernen Ära wurde die Frage, wer die Regierung
kontrolliert, fast immer als politische Frage gestellt. Es gab viele
Antworten darauf, aber fast immer ging es darum, die politische Partei,
Gruppe oder Fraktion zu identifizieren, die zu einem bestimmten
Zeitpunkt die Kontrolle über einen bestimmten Staat ausübte. Sie haben
von Regierungen gehört, die von Kapitalisten kontrolliert wurden. Von
Regierungen, die von der Arbeiterschaft kontrolliert wurden. Von
Regierungen, die von Katholiken und islamischen Fundamentalisten
kontrolliert wurden. Regierungen, die von Stammes- und Rassengruppen
kontrolliert wurden; Regierungen, die von Hutus und Regierungen, die von
Weißen kontrolliert wurden. Sie haben auch schon von Regierungen gehört,
die von Berufsgruppen wie Anwälten oder Bankern kontrolliert wurden. Sie
haben von Regierungen gehört, die von ländlichen Interessen, von
Großstadtmaschinerien und von Menschen, die in den Vorstädten leben,
kontrolliert werden. Und Sie haben sicherlich schon von Regierungen
gehört, die von politischen Parteien kontrolliert werden, von
Demokraten, Konservativen, Christdemokraten, Liberalen, Radikalen,
Republikanern und Sozialisten.

Aber Sie haben wahrscheinlich noch nicht viel von einer Regierung
gehört, die von ihren Kunden kontrolliert wird. Der
Wirtschaftshistoriker Frederic Lane hat in einigen seiner einleuchtenden
Aufsätze über die wirtschaftlichen Folgen von Gewalt die Grundlage für
ein neues Verständnis der Kontrolle der Regierung gelegt. Die
Betrachtung des Staates als wirtschaftliche Einheit, die Schutz
verkauft, veranlasste Lane, die Kontrolle des Staates in
wirtschaftlicher und nicht in politischer Hinsicht zu analysieren. Aus
dieser Sicht gibt es drei grundlegende Alternativen für die Kontrolle
der Regierung, von denen jede eine grundlegend andere Reihe von Anreizen
mit sich bringt: Eigentümer, Angestellte und Kunden.

\subsection{Eigentümer}\label{eigentuxfcmer}

In seltenen Fällen werden Regierungen auch heute noch von einem
Eigentümer kontrolliert, in der Regel einem erblichen Oberhaupt, dem das
Land praktisch gehört. Der Sultan von Brunei beispielsweise behandelt
die Regierung von Brunei in gewisser Weise wie sein Eigentum. Dies war
eher bei den Herren des Mittelalters üblich, die ihre Lehen wie Eigentum
behandelten, um ihre Einkünfte zu optimieren.

Lane beschrieb die Anreize der „Eigentümer des produzierenden
Unternehmens`` wie folgt:

\begin{quote}
Das Interesse an einer Gewinnmaximierung würde ihn dazu veranlassen,
unter Beibehaltung der Preise zu versuchen, seine Kosten zu senken. Wie
Heinrich VII. von England oder Ludwig XI. von Frankreich würde er
billige Tricks anwenden, zumindest so billige wie möglich, um seine
Legitimität zu bekräftigen, die innere Ordnung aufrechtzuerhalten und
die benachbarten Fürsten abzulenken, damit seine eigenen Militärausgaben
niedrig sind. Aus den gesenkten Kosten oder aus den erhöhten Abgaben,
die durch die Stabilität seines Monopols möglich wurden, oder aus einer
Kombination, häufte er einen Überschuss an...\footnote{Lane,
  \emph{Consequences of Organized Violence}, ebenda, S. 406.}
\end{quote}

Von Eigentümern kontrollierte Regierungen haben starke Anreize, die
Kosten für die Bereitstellung von Schutz oder die Monopolisierung von
Gewalt in einem bestimmten Gebiet zu senken. Solange ihre Herrschaft
jedoch sicher ist, haben sie wenig Anreiz, den Preis (die Steuer), den
sie von ihren Kunden verlangen, unter den Satz zu senken, der die
Einnahmen optimiert. Je höher der Preis ist, den ein Monopolist
verlangen kann, und je niedriger seine tatsächlichen Kosten sind, desto
größer ist der Gewinn, den er erzielt. Die ideale Steuerpolitik für eine
Regierung, die von ihren Eigentümern kontrolliert wird, wäre ein großer
Überschuss. Wenn Regierungen ihre Einnahmen hoch halten, aber ihre
Kosten senken können, hat dies große Auswirkungen auf die Nutzung der
Ressourcen. Arbeitskräfte und andere wertvolle Inputs, die andernfalls
für unnötig teuren Schutz verschwendet würden, stehen stattdessen für
Investitionen und andere Zwecke zur Verfügung. Je höher der Monarch
seinen Gewinn durch die Senkung der Kosten steigern kann, desto mehr
Ressourcen werden freigesetzt. Wenn diese Ressourcen für Investitionen
verwendet werden, sorgen sie für einen Wachstumsimpuls. Aber auch wenn
sie für den Prestigekonsum verwendet werden, tragen sie dazu bei, neue
Märkte zu schaffen und zu versorgen, die sonst nicht existieren würden,
wenn die Ressourcen für die Produktion von ineffizientem „Schutz``
verschwendet worden wären.

\subsection{Angestellte}\label{angestellte}

Die Anreize für Regierungen, die von ihren Angestellten kontrolliert
werden, sind leicht zu charakterisieren. Es sind ähnliche Anreize wie in
anderen von Arbeitnehmern kontrollierten Organisationen. In erster Linie
neigen von Arbeitnehmern geführte Organisationen dazu, jede Politik zu
bevorzugen, die die Beschäftigung erhöht, und Maßnahmen abzulehnen, die
Arbeitsplätze abbauen. Wie Lane es ausdrückt: „Als die Arbeitnehmer als
Ganzes die Kontrolle ausübten, hatten sie wenig Interesse daran, die für
den Schutz geforderten Beträge zu minimieren, und kein Interesse daran,
den großen Teil der Kosten zu minimieren, der auf die Arbeitskosten, auf
ihre eigenen Gehälter, entfällt. Eine Maximierung der Größe war auch
eher nach ihrem Geschmack.`` \footnote{Ebenda.} Eine Regierung, die von
ihren Mitarbeitern kontrolliert wird, hätte selten Anreize, entweder die
Kosten der Regierung oder den Preis für ihre Kunden zu senken. Wenn
jedoch die Bedingungen einen starken Preiswiderstand in Form von
Widerstand gegen höhere Steuern erzwingen, würden von Arbeitnehmern
kontrollierte Regierungen eher dazu neigen, ihre Einnahmen unter ihre
Ausgaben fallen zu lassen, als ihre Ausgaben zu kürzen. Mit anderen
Worten, ihre Anreize lassen vermuten, dass sie zu chronischen Defiziten
neigen könnten, was bei Regierungen, die von Eigentümern kontrolliert
werden, nicht der Fall wäre.

\subsection{Kunden}\label{kunden}

Gibt es Beispiele für Regierungen, die von ihren Kunden kontrolliert
werden? Ja. Das Beispiel der mittelalterlichen Handelsrepubliken wie
Venedig inspirierte Lane dazu, die Kontrolle der Regierung in
wirtschaftlicher Hinsicht zu analysieren. Dort kontrollierte eine Gruppe
von Großhändlern, die Schutz brauchten, jahrhundertelang effektiv die
Regierung. Sie waren wirklich Kunden für den Schutz, den die Regierung
bot, und nicht Eigentümer. Sie zahlten für die Dienstleistung. Sie
versuchten nicht, von ihrer Kontrolle des staatlichen Gewaltmonopols zu
profitieren. Falls doch, wurden sie von den anderen Kunden über lange
Zeiträume daran gehindert. Andere Beispiele für Regierungen, die von
ihren Kunden kontrolliert wurden, sind Demokratien und Republiken mit
begrenztem Wahlrecht, wie die antiken Demokratien oder die amerikanische
Republik in ihrer Gründungszeit. Damals durften nur diejenigen wählen,
die für die Regierung zahlten, etwa 10 Prozent der Bevölkerung.

Regierungen, die von ihren Kunden kontrolliert werden, haben ebenso wie
die Eigentümer Anreize, ihre Betriebskosten so weit wie möglich zu
senken.

Aber im Gegensatz zu Regierungen, die von Eigentümern oder Angestellten
kontrolliert werden, haben Regierungen, die tatsächlich von ihren Kunden
kontrolliert werden, Anreize, die Preise, die sie verlangen, niedrig zu
halten. Wo Kunden regieren, sind Regierungen schlank und im Allgemeinen
unauffällig, mit geringen Betriebskosten, minimaler Beschäftigung und
niedrigen Steuern. Ein Staat, der von seinen Kunden kontrolliert wird,
legt die Steuersätze nicht fest, um den Betrag zu optimieren, den der
Staat einnehmen kann, sondern um den Betrag zu optimieren, den die
Kunden einbehalten können.

Wie typische Unternehmen auf Wettbewerbsmärkten wäre auch ein Monopol,
das von seinen Kunden kontrolliert wird, gezwungen, sich um Effizienz zu
bemühen. Es wäre nicht in der Lage, einen Preis in Form von Steuern zu
verlangen, der die Kosten um mehr als eine knappe Marge übersteigt.

\section{DIE ROLLE DER DEMOKRATIE: WÄHLER ALS ARBEITNEHMER UND
KUNDEN}\label{die-rolle-der-demokratie-wuxe4hler-als-arbeitnehmer-und-kunden}

Lane behandelt die Demokratie auf herkömmliche Weise, indem er davon
ausgeht, dass sie gewaltausübende und gewaltproduzierende Unternehmen
„zunehmend unter die Kontrolle ihrer Kunden bringt``.\footnote{Ebenda.,
  S. 412.} Dies ist sicherlich die politisch korrekte Schlussfolgerung.
Aber ist sie wahr? Wir glauben nicht. Schauen Sie sich genau an, wie
moderne Demokratien funktionieren.

Zunächst mal weisen sie nur wenige Merkmale jener wettbewerbsfähigen
Branchen auf, in denen die Handelsbedingungen eindeutig von den Kunden
kontrolliert werden. Zum einen geben demokratische Regierungen in der
Regel nur einen Bruchteil ihrer Gesamtausgaben für den Schutzdienst aus,
der ihre Haupttätigkeit ist. In den Vereinigten Staaten beispielsweise
geben die Regierungen der Bundesstaaten und Kommunen nur 3,5 Prozent
ihrer Gesamtausgaben für die Bereitstellung von Polizei, Gerichten und
Gefängnissen aus. Rechnet man die Militärausgaben hinzu, so beträgt der
Anteil der Einnahmen, der für den Schutz aufgewendet wird, immer noch
nur etwa 10 Prozent. Ein weiterer aufschlussreicher Hinweis darauf, dass
die Massendemokratie nicht von ihren Kunden kontrolliert wird, ist die
Tatsache, dass die heutige politische Kultur, ein Erbe des
Industriezeitalters, es als empörend empfinden würde, wenn die Politik
in entscheidenden Fragen tatsächlich von den Interessen der Menschen
bestimmt würde, die die Rechnungen bezahlen. Stellen Sie sich den
Aufschrei vor, wenn ein amerikanischer Präsident oder ein britischer
Premierminister vorschlagen würde, die Gruppe der Bürger, die den
Großteil der Steuern zahlt, darüber entscheiden zu lassen, welche
Regierungsprogramme fortgeführt und welche Gruppen von Mitarbeitern
entlassen werden sollen. Dies würde die Erwartungen an die Arbeitsweise
der Regierung zutiefst verletzen, und zwar in einer Weise, wie es nicht
der Fall wäre, wenn man den Regierungsangestellten erlauben würde, zu
bestimmen, wessen Steuern erhöht werden sollen.

Wenn man jedoch darüber nachdenkt, dass der Kunde wirklich das Sagen
hat, wäre es empörend, wenn er nicht bekäme, was er will. Wenn Sie in
ein Geschäft gehen, um Möbel zu kaufen, und das Verkaufspersonal nimmt
Ihr Geld entgegen, ignoriert dann aber Ihre Wünsche und berät sich mit
anderen darüber, wie Sie Ihr Geld ausgeben sollen, wären Sie zu Recht
verärgert. Sie würden es weder für normal noch für vertretbar halten,
wenn die Angestellten des Geschäfts argumentieren würden, dass Sie die
Möbel gar nicht verdient hätten und sie stattdessen an jemanden
geliefert werden sollten, den sie für würdiger hielten. Die Tatsache,
dass so etwas im Umgang mit der Regierung vorkommt, zeigt, wie wenig
Kontrolle die „Kunden`` tatsächlich haben.

Die Kosten einer demokratischen Regierung sind in jeder Hinsicht außer
Kontrolle geraten, im Gegensatz zu der typischen Situation, in der die
Kundenpräferenzen die Anbieter zu Effizienz zwingen. Die meisten
Demokratien weisen chronische Defizite auf. Dies ist ein
finanzpolitisches Merkmal der Kontrolle durch die Beschäftigten.
Regierungen scheinen besonders resistent zu sein, wenn es darum geht,
die Kosten ihrer Tätigkeit zu senken. Eine fast allgemeingültige Klage
über die heutigen Regierungen weltweit ist, dass politische Programme,
wenn sie einmal eingeführt sind, nur sehr schwer wieder zurückgenommen
werden können. Einen Regierungsangestellten zu entlassen ist so gut wie
unmöglich. Tatsächlich besteht einer der Hauptvorteile der
Privatisierung ehemals staatlicher Funktionen darin, dass es unter
privater Kontrolle in der Regel viel einfacher ist, unnötige
Arbeitsplätze zu streichen. Von Großbritannien bis Argentinien war es
nicht ungewöhnlich, dass die neuen privaten Manager 50-95 Prozent der
ehemaligen Staatsbediensteten entlassen haben.

Denken Sie auch an die Grundlage, auf der die fiskalischen Bedingungen
für die staatliche Schutzleistung bepreist werden. Hinweise auf
wettbewerbliche Einflüsse auf die Steuersätze, nach denen staatliche
Leistungen bepreist werden, sucht man zumeist vergebens. Selbst die
gelegentlichen Debatten über Steuersenkungen, die in den letzten Jahren
den normalen politischen Diskurs unterbrochen haben, verraten, wie weit
der demokratische Staat normalerweise von einer Kontrolle durch seine
Kunden entfernt ist. Befürworter von Steuersenkungen haben manchmal
argumentiert, dass die Staatseinnahmen tatsächlich steigen würden, weil
die Steuersätze zuvor so hoch angesetzt waren, dass sie die
Wirtschaftstätigkeit behinderten.

Der Kompromiss, den sie normalerweise hervorheben wollten, war nicht der
Wettbewerb zwischen den Rechtsordnungen, sondern etwas viel
Erstaunlicheres. Sie argumentierten nicht, dass, weil die Steuersätze in
Hongkong nur 15 Prozent betragen, die Sätze in den Vereinigten Staaten
oder Deutschland nicht höher als 15 Prozent sein dürfen. Das Gegenteil
ist der Fall. In den Steuerdebatten wurde normalerweise davon
ausgegangen, dass der Steuerzahler nicht vor der Wahl steht, ob er seine
Geschäfte in einem Land oder in einem anderen Land tätigt, sondern ob er
seine Geschäfte zu Strafsteuersätzen tätigt oder Urlaub macht. Man hat
Ihnen gesagt, dass produktive Personen, die einer räuberischen
Besteuerung unterworfen sind, sich von ihren Briefkästen entfernen und
Golf spielen gehen würden, wenn ihre Steuerlast nicht gemildert würde.

Die Tatsache, dass ein solches Argument überhaupt vorgebracht werden
kann, zeigt, wie weit entfernt die von den demokratischen
Wohlfahrtsstaaten auferlegten Schutzkosten von einer
wettbewerbsorientierten Grundlage sind. Die Bedingungen der progressiven
Einkommensbesteuerung, die im Laufe des zwanzigsten Jahrhunderts in
jedem demokratischen Wohlfahrtsstaat aufkam, unterscheiden sich
dramatisch von den Preisbestimmungen, die von den Kunden bevorzugt
werden würden. Dies lässt sich leicht erkennen, wenn man die Besteuerung
zur Unterstützung eines monopolistischen Schutzes mit den Tarifen für
den Telefondienst vergleicht, der bis vor kurzem an den meisten Orten
ein Monopol war. Die Kunden würden aufschreien, wenn eine
Telefongesellschaft versuchen würde, Anrufe auf der gleichen Grundlage
wie die Einkommenssteuer zu berechnen. Stellen Sie sich vor, die
Telefongesellschaft schickt Ihnen eine Rechnung über 50.000 Dollar für
einen Anruf nach London, nur weil Sie während eines Gesprächs ein
Geschäft im Wert von 125.000 Dollar abgeschlossen haben. Weder Sie noch
irgendein anderer Kunde, der bei Verstand ist, würde diese Rechnung
bezahlen. Aber genau auf dieser Grundlage werden in jedem demokratischen
Wohlfahrtsstaat die Einkommenssteuern erhoben.

Wenn man genau darüber nachdenkt, unter welchen Bedingungen die
industriellen Demokratien funktioniert haben, ist es logischer, sie als
eine Form der Regierung zu betrachten, die von ihren Mitarbeitern
kontrolliert wird. Wenn man sich die Massendemokratie als eine von den
Arbeitnehmern kontrollierte Regierung vorstellt, erklärt sich auch die
Schwierigkeit, die Regierungspolitik zu ändern. Die Regierung scheint in
vielerlei Hinsicht zum Nutzen der Beschäftigten geführt zu werden. Zum
Beispiel scheinen die staatlichen Schulen in den meisten demokratischen
Ländern chronisch dysfunktional zu sein, ohne Chance auf die Behebung
ihrer Missstände. Wenn die Kunden wirklich das Sagen hätten, wäre es für
sie einfacher, neue politische Richtungen vorzugeben. Diejenigen, die
für eine demokratische Regierung zahlen, bestimmen nur selten die
Bedingungen für die Staatsausgaben. Stattdessen funktioniert der Staat
wie eine Genossenschaft, die sich der Kontrolle durch die Eigentümer
entzieht und wie ein natürliches Monopol funktioniert. Die Preise stehen
in keinem Verhältnis zu den Kosten. Die Qualität der Dienstleistungen
ist im Allgemeinen niedriger als in der Privatwirtschaft.
Kundenbeschwerden sind schwer zu beheben. Kurz gesagt, die
Massendemokratie führt zu einer Kontrolle der Regierung durch ihre
„Angestellten``.

Aber halt. Sie werden vielleicht sagen, dass es in den meisten Ländern
viel mehr Wähler gibt, als Personen, die auf der Gehaltsliste der
Regierung stehen. Wie ist es möglich, dass die Beschäftigten unter
solchen Bedingungen dominieren können? Der Wohlfahrtsstaat ist als
Antwort auf genau dieses Dilemma entstanden. Da es nicht genügend
Arbeitnehmer gab, um eine arbeitende Mehrheit zu bilden, wurde eine
wachsende Zahl von Wählern auf die Lohnliste gesetzt, um
Transferleistungen aller Art zu erhalten. So wurden die Empfänger von
Transferleistungen und Subventionen zu Pseudo-Staatsbediensteten, die
sich nicht mehr jeden Tag zur Arbeit melden mussten. Das war ein
Ergebnis, das von der megapolitischen Logik des Industriezeitalters
diktiert wurde.

Wenn das Ausmaß der Zwangsgewalt wichtiger ist als der effiziente
Einsatz von Ressourcen, wie es vor 1989 der Fall war, ist es für die
meisten Regierungen so gut wie unmöglich, von ihren Kunden kontrolliert
zu werden. Wie das Beispiel der späten Sowjetunion zeigt, war es bis vor
wenigen Jahren für Staaten möglich, große Macht in der Welt auszuüben
und gleichzeitig Ressourcen in großem Umfang zu verschwenden. Wenn die
Renditen für Gewalt hoch sind und steigen, bedeutet Größe mehr als
Effizienz. Größere Einheiten neigen dazu, sich gegenüber kleineren
durchzusetzen. Diejenigen Regierungen, die militärische Ressourcen
effektiver mobilisieren, auch wenn sie dafür viele von ihnen
verschwenden, setzen sich tendenziell gegenüber denjenigen durch, die
die Ressourcen effizienter nutzen.

Überlegen Sie, was das bedeutet. Es bedeutet unweigerlich, dass
Regierungen, die von ihren Kunden kontrolliert werden, sich nicht
durchsetzen und oft nicht überleben können, wenn Größe mehr bedeutet als
Effizienz. Unter solchen Bedingungen werden diejenigen Einheiten
militärisch am effektivsten sein, die die meisten Ressourcen für den
Krieg requirieren. Aber Regierungen, die wirklich von ihren Kunden, die
ihre Rechnungen bezahlen, kontrolliert werden, haben wahrscheinlich
keinen Freibrief, jedem in die Tasche zu greifen, um Ressourcen zu
beschaffen.

Die Kunden möchten normalerweise, dass die Preise, die sie für ein
Produkt oder eine Dienstleistung, einschließlich Schutz, zahlen, gesenkt
und unter Kontrolle gehalten werden. Hätten die westlichen Demokratien
während des Kalten Krieges unter der Kontrolle der Kunden gestanden,
hätte allein diese Tatsache sie militärisch zu schwächeren Konkurrenten
gemacht, da dies mit ziemlicher Sicherheit den Fluss von Ressourcen in
die Regierung eingeschränkt hätte. Denken Sie daran, dass dort, wo die
Kunden regieren, sowohl die Preise als auch die Kosten unter strenger
Kontrolle stehen sollten. Doch genau das ist nicht geschehen. Die
Wohlfahrtsstaaten waren während des Kalten Krieges die offensichtlichen
Gewinner des Ausgabenwettbewerbs. Kommentatoren aller Couleur nannten
als Grund für ihren Triumph die Fähigkeit, die Sowjetunion in den
Bankrott zu treiben.

Genau diese Tatsache verdeutlicht, wie die Ineffizienzen der Demokratie
sie in einer Zeit steigender Gewaltrenditen zur vorherrschenden
Megapolitik machten. Massive Militärausgaben mit all ihrer Verschwendung
stellen einen eindeutig suboptimalen Einsatz von Kapital zum privaten
Nutzen dar. Wir haben bereits angedeutet, dass Wohlfahrtsstaaten im
Vergleich zu staatssozialistischen Systemen zwar wirtschaftlich
effizient sind, dass sie aber bei der Schaffung von Wohlstand weit
weniger effizient sind als Laissez-faire-Enklaven wie Hongkong.
Ironischerweise war es gerade diese Ineffizienz des demokratischen
Wohlfahrtsstaates im Vergleich zu einem unbelasteten System der freien
Marktwirtschaft, die ihn unter den megapolitischen Bedingungen des
Industrialismus erfolgreich machte.

Wie konnte die von der Demokratie geförderte Ineffizienz zu einem Faktor
ihres Erfolgs im Zeitalter der Gewalt werden? Der Schlüssel zum
Entwirren dieses scheinbaren Paradoxons liegt in der Erkenntnis zweier
Punkte:

\begin{enumerate}
\def\labelenumi{\arabic{enumi}.}
\item
  Der Erfolg einer Souveränität in der Neuzeit lag nicht in der
  Schaffung von Reichtum, sondern in der Schaffung einer militärischen
  Macht, die in der Lage war, übermächtige Gewalt gegen jeden anderen
  Staat einzusetzen. Dazu brauchte man Geld, aber mit Geld allein konnte
  man keine Schlacht gewinnen. Die Herausforderung bestand nicht darin,
  ein System mit der effizientesten Wirtschaft oder der schnellsten
  Wachstumsrate zu schaffen, sondern ein System, das mehr Ressourcen
  abziehen und diese in das Militär lenken konnte. Militärausgaben sind
  naturgemäß ein Bereich, in dem der finanzielle Ertrag an sich gering
  oder gar nicht vorhanden ist.
\item
  Der einfachste Weg, um die Erlaubnis zu erhalten, Gelder in
  Aktivitäten zu investieren, die wenig oder keinen direkten
  finanziellen Ertrag bringen, wie z. B. Steuerzahlungen, ist, jemanden
  um Erlaubnis zu bitten, der nicht die Person ist, deren Geld man
  begehrt. Einer der Gründe, warum die Niederländer Manhattan für Perlen
  im Wert von dreiundzwanzig Dollar kaufen konnten, war, dass die
  Indianer, denen sie das Angebot unterbreiteten, nicht diejenigen
  waren, denen es eigentlich gehörte. Unter diesen Bedingungen ist es
  viel einfacher, ein „Ja`` zu bekommen, wie die Marketingleute sagen.
  Nehmen wir zum Beispiel an, wir als Autoren dieses Buches wollten,
  dass Sie nicht den Einbandpreis, sondern 40 Prozent Ihres
  Jahreseinkommens für ein Exemplar zahlen. Es wäre viel
  wahrscheinlicher, dass wir die Erlaubnis dazu bekämen, wenn wir jemand
  anderen als Sie fragen würden. Tatsächlich wären wir weitaus
  überzeugender, wenn wir uns stattdessen auf die Zustimmung mehrerer
  Personen verlassen könnten, die Sie nicht einmal kennen. Wir könnten
  eine Ad-hoc-Wahl durchführen, was H. L. Mencken mit weniger
  Übertreibung als er dachte als „eine fortgeschrittene Versteigerung
  gestohlener Waren`` bezeichnete. Und um das Beispiel realistischer zu
  machen, würden wir uns bereit erklären, einen Teil des Geldes, das wir
  von Ihnen gesammelt haben, mit diesen anonymen Umstehenden im
  Austausch für ihre Unterstützung zu teilen.
\end{enumerate}

Das ist die Rolle, die der moderne demokratische Wohlfahrtsstaat zu
erfüllen hat. Er war im Industriezeitalter ein unübertroffenes System,
weil er dort, wo es darauf ankam, sowohl effizient als auch ineffizient
war. Er kombinierte die Effizienz des Privateigentums und Anreize für
die Schaffung von Wohlstand mit einem Mechanismus, der einen im
Wesentlichen unkontrollierten Zugang zu diesem Wohlstand ermöglichte.
Die Demokratie hielt die Taschen der Wohlstandsproduzenten offen.
Militärisch war sie während der Hochphase steigender Gewaltrenditen in
der Welt gerade deshalb erfolgreich, weil sie es den Kunden erschwerte,
die von der Regierung erhobenen Steuern oder andere Möglichkeiten der
Finanzierung der Ausgaben für das Militär, wie etwa die Inflation,
wirksam zu beschränken.

\subsection{Warum die Kunden nicht dominieren
konnten}\label{warum-die-kunden-nicht-dominieren-konnten}

Diejenigen, die in der Neuzeit für den „Schutz`` zahlten, waren nicht in
der Lage, dem Souverän erfolgreich Ressourcen zu verweigern, selbst wenn
sie kollektiv handelten, da sie dies nur der Gefahr ausgesetzt hätte,
von anderen, möglicherweise feindlicheren Staaten überwältigt zu werden.
Dies war eine offensichtliche Überlegung während des Kalten Krieges. Die
Kunden bzw. Steuerzahler, die in den führenden westlichen
Industriestaaten einen unverhältnismäßig hohen Anteil an den Kosten des
Staates trugen, konnten sich nicht weigern, hohe Steuern zu zahlen. Das
Ergebnis wäre gewesen, dass sie sich der totalen Beschlagnahmung durch
die Sowjetunion oder eine andere aggressive Gruppe, die in der Lage war,
Gewalt zu organisieren, ausgesetzt hätten.

\subsection{Industrialisierung und
Demokratie}\label{industrialisierung-und-demokratie}

Längerfristig betrachtet könnte sich die Massendemokratie als
Anachronismus erweisen, der das Ende des Industriezeitalters nicht lange
überleben wird. Sicher ist, dass die Massendemokratie und der
Nationalstaat zusammen mit der Französischen Revolution am Ende des 18.
Jahrhunderts entstanden sind, wahrscheinlich als Reaktion auf einen
Anstieg des Realeinkommens. Um 1750 begannen die Einkommen in Westeuropa
erheblich zu steigen, was zum Teil auf das wärmere Wetter zurückzuführen
war. Dies fiel mit einer Periode technologischer Innovationen zusammen,
in der qualifizierte Arbeitsplätze von Handwerkern durch Maschinen
ersetzt wurden, die von ungelernten Arbeitern, sogar Frauen und Kindern,
bedient werden konnten. Diese neuen Industrieanlagen führten zu höheren
Einkommen für ungelernte Arbeitskräfte, wodurch die Einkommensverteilung
gleichmäßiger wurde.

Der entscheidende Auslöser der Revolution war vielleicht nicht, wie oft
angenommen wird, die perverse Idee, dass die Menschen dazu neigen, sich
aufzulehnen, wenn sich die Bedingungen verbessern. Wichtiger könnte die
Tatsache gewesen sein, dass es für den frühmodernen Staat endlich
praktisch wurde, die privaten Vermittler und mächtigen Magnaten, mit
denen er zuvor um Ressourcen gefeilscht hatte, zu umgehen und zu einem
System der „direkten Herrschaft`` überzugehen, in dem eine nationale
Regierung direkt mit den einzelnen Bürgern verhandelte, sie mit immer
höheren Steuersätzen belastete und einen schlecht bezahlten
Militärdienst als Gegenleistung für die Bereitstellung verschiedener
Leistungen forderte.\footnote{Tilly, ebenda, S. 96-126.}

Da die aufstrebende Mittelschicht bald über genügend Geld verfügte, das
besteuert werden konnte, war es für die Herrscher nicht mehr so wichtig
wie zuvor, mit mächtigen Grundbesitzern oder Großkaufleuten zu
verhandeln, die, wie der Historiker Charles Tilly schrieb, „in der Lage
waren, die Schaffung eines mächtigen Staates zu verhindern``, der „ihr
Vermögen beschlagnahmen und ihre Transaktionen einschränken würde.``
\footnote{Ebenda., S. 130.} Es ist leicht einzusehen, warum Regierungen
bei der Gewinnung von Ressourcen erfolgreicher waren, wenn sie mit
Millionen von Bürgern einzeln verhandelten als mit einer relativen
Handvoll von Herren, Herzögen, Grafen, Bischöfen, Vertragssöldnern,
freien Städten und anderen halbstaatlichen Einheiten, mit denen die
Herrscher europäischer Staaten vor der Mitte des 18. Jahrhunderts
verhandeln mussten.

Steigende Realeinkommen ermöglichten es den Regierungen, eine Strategie
zu verfolgen, mit der sie mehr Ressourcen unter ihre Kontrolle brachten.
Kleine Beträge, die von Millionen von Steuerzahlern eingenommen wurden,
konnten mehr Einnahmen bringen als größere Beträge, die von einigen
wenigen Mächtigen gezahlt wurden. Außerdem war es viel einfacher, mit
den Vielen umzugehen als mit den Wenigen, die im Allgemeinen nicht
bereit waren, ihr Geld zu verschenken, und die sich viel besser wehren
konnten.

Schließlich verfügte der typische Bauer, Kleinhändler oder Arbeiter im
Vergleich zum Staat selbst über verschwindend geringe Ressourcen. Es war
nicht einmal im Entferntesten möglich, dass der typische Privatmann in
Westeuropa am Vorabend der Französischen Revolution wirksam mit dem
Staat verhandeln konnte, um seinen Steuersatz zu senken, oder einen
wirksamen Widerstand gegen Regierungspläne und -maßnahmen leisten
konnte, die seine Interessen bedrohten. Doch genau das hatten mächtige
Privatmagnaten jahrhundertelang getan und würden es auch weiterhin tun.
Sie leisteten wirksamen Widerstand und verhandelten mit den Machthabern,
um deren Möglichkeiten zur Aneignung von Ressourcen zu beschränken.

\begin{quote}
„Der Kriegseintritt beschleunigte den Übergang von der indirekten zur
direkten Herrschaft. Fast jeder Staat, der in den Krieg zieht, stellt
fest, dass er den Aufwand nicht aus seinen angesammelten Reserven und
laufenden Einnahmen bezahlen kann. Fast alle kriegführenden Staaten
nehmen in großem Umfang Kredite auf, erhöhen die Steuern und
beschlagnahmen die Kampfmittel - einschließlich der Männer - von
widerstrebenden Bürgern, die eine andere Nutzung ihrer Ressourcen
vorgesehen hatten.`` \footnote{Ebenda., S. 110.} Charles Tilly
\end{quote}

Das Beispiel Polens in der Mitte des achtzehnten Jahrhunderts
veranschaulicht dies perfekt. Im Jahr 1760 umfasste die polnische
Nationalarmee 18.000 Soldaten. Im Vergleich zu den Armeen der
benachbarten Länder Österreich, Preußen und Russland, von denen das
kleinste über ein stehendes Heer von 100.000 Soldaten verfügte, war dies
eine magere Truppe. Tatsächlich war die polnische Nationalarmee im Jahr
1760 selbst im Vergleich zu anderen bewaffneten Einheiten innerhalb
Polens klein. Die kombinierten Streitkräfte des polnischen Adels
umfassten dreißigtausend Mann.\footnote{Dieses Beispiel ebenda, S. 139.}

Wäre der polnische König in der Lage gewesen, direkt mit Millionen
einzelner Polen in Kontakt zu treten und sie direkt zu besteuern,
anstatt sich darauf zu beschränken, die Mittel indirekt über die
Beiträge der mächtigen polnischen Magnaten zu erhalten, wäre die
polnische Zentralregierung zweifellos in der Lage gewesen, weitaus mehr
Einnahmen zu erzielen und somit eine größere Armee zu finanzieren.

Gegenüber einfachen Bürgern, die nicht in der Lage waren, gemeinsam mit
Millionen anderer einfacher Bürger zu handeln, sollten sich die
Zentralbehörden überall als unwiderstehlich mächtig erweisen. Aber der
polnische König hatte 1760 nicht die Möglichkeit, seine Bürger direkt zu
besteuern. Er musste über die Fürsten, wohlhabenden Kaufleute und
anderen Honoratioren verhandeln, die eine kleine, zusammenhängende
Gruppe bildeten. Sie konnten gemeinsam handeln und taten dies auch, um
den König daran zu hindern, sich ihre Ressourcen ohne ihre Zustimmung
anzueignen. Da der polnische Adel über weit mehr Truppen verfügte als er
selbst, konnte der König nicht darauf bestehen.

Wie sich herausstellte, war der militärische Nachteil, die Reichen und
Mächtigen bei der Beschaffung von Ressourcen nicht umgehen zu können, im
Zeitalter der Gewalt entscheidend. Innerhalb weniger Jahre hörte Polen
auf, als unabhängiges Land zu existieren. Es wurde von Österreich,
Preußen und Russland erobert, drei Ländern mit Armeen, die jeweils um
ein Vielfaches größer waren als die kleine polnische Streitmacht. In
jedem dieser Länder hatten die Herrscher Wege gefunden, die Fähigkeit
der wohlhabenden Kaufleute und des Adels zur Begrenzung der
Beschlagnahmung ihrer Ressourcen zu umgehen.

\subsection{Nach der französischen
Revolution}\label{nach-der-franzuxf6sischen-revolution}

Die Französische Revolution führte zu einer noch stärkeren Vergrößerung
der Armeen, eine Tatsache, die die Stärke der demokratischen Strategie
unter Beweis stellte, als die Rendite der Gewalt stieg. Der Handel, den
die Regierungen seit der Französischen Revolution eingingen, bestand
darin, ein noch nie dagewesenes Maß an Einmischung in das Leben des
Durchschnittsbürgers zu gewähren, und zwar im Gegenzug dafür, dass
dieser anstelle von Söldnern an Kriegen teilnahm und eine wachsende
Steuerlast aus seinem steigenden Einkommen zahlte.

Wie Tilly sagte:

\begin{quote}
Die Sphäre des Staates dehnte sich weit über seinen militärischen Kern
hinaus aus und seine Bürger begannen, ihn für ein sehr breites Spektrum
von Schutz, Rechtsprechung, Produktion und Verteilung in Anspruch zu
nehmen. In dem Maße, wie die nationalen Gesetzgebungen ihre eigenen
Bereiche weit über die Genehmigung von Steuern hinaus ausdehnten, wurden
sie zur Zielscheibe von Forderungen aller gut organisierten Gruppen,
deren Interessen der Staat berührte oder berühren konnte. Direkte
Herrschaft und nationale Massenpolitik wuchsen zusammen und verstärkten
sich gegenseitig.\footnote{Ebenda., S. 115.}
\end{quote}

Die gleiche Logik, die im achtzehnten Jahrhundert galt, blieb bis zum
Fall der Berliner Mauer 1989 gültig. Mit dem Fortschreiten des
Industriezeitalters stiegen die Einkommen für ungelernte Arbeit weiter
an, was die Massendemokratie zu einer noch effektiveren Methode zur
Optimierung der Ressourcengewinnung machte. Infolgedessen wuchs und
wuchs der Staat und erhöhte im Laufe des zwanzigsten Jahrhunderts in
einem durchschnittlichen Industrieland seinen Gesamtanspruch auf das
Jahreseinkommen um etwa 0,5 Prozent.

Im Industriezeitalter vor 1989 erwies sich die Demokratie gerade deshalb
als die militärisch effektivste Regierungsform, weil die Demokratie es
schwierig oder unmöglich machte, der Beschlagnahmung von Ressourcen
durch den Staat wirksame Grenzen zu setzen. Die großzügige Gewährung von
Wohlfahrtsleistungen für alle und jeden lud eine Mehrheit der Wähler
dazu ein, faktisch Angestellte der Regierung zu werden. Dies wurde zum
vorherrschenden politischen Merkmal aller führenden Industrieländer, da
sich die Wähler in ihrer Rolle als Kunden für den Schutzdienst in einer
schwachen Position befanden, um die Regierung wirksam zu kontrollieren.
Sie sahen sich nicht nur mit der aggressiven Bedrohung durch
kommunistische Systeme konfrontiert, die große Ressourcen für
militärische Zwecke produzieren konnten, da der Staat die gesamte
Wirtschaft kontrollierte. Eine echte Kontrolle des Staates durch die
Steuerzahler war auch aus einem anderen Grund nicht möglich.

Millionen von Durchschnittsbürgern können nicht wirksam
zusammenarbeiten, um ihre Interessen zu schützen. Da die Hindernisse für
ihre Zusammenarbeit hoch sind und der Gewinn für jeden Einzelnen bei
erfolgreicher Verteidigung der gemeinsamen Interessen der Gruppe minimal
ist, werden Millionen von Durchschnittsbürgern nicht so erfolgreich
sein, ihr Vermögen der Regierung vorzuenthalten, wie kleinere Gruppen
mit günstigeren Anreizen.

Unter sonst gleichen Bedingungen würde man daher erwarten, dass in einer
Massendemokratie ein höherer Anteil der Gesamtressourcen von der
Regierung requiriert wird, als in einer Oligarchie oder in einem System
fragmentierter Souveränität, in dem Magnaten militärische Macht ausübten
und ihre eigenen Armeen aufstellten, wie es überall im frühmodernen
Europa vor dem 18. Jahrhundert der Fall war.

Ein entscheidender, wenn auch selten untersuchter Grund für das Wachstum
der Demokratie in der westlichen Welt ist also die relative Bedeutung
der Verhandlungskosten in einer Zeit, in der die Rendite der Gewalt
stieg. Es war immer kostspieliger, Ressourcen von den Wenigen zu
beziehen als von den Vielen.

Eine relativ kleine, elitäre Gruppe von Reichen stellt eine kohärentere
und effektivere Einheit dar als eine große Masse von Bürgern. Die kleine
Gruppe hat stärkere Anreize, zusammenzuarbeiten. Sie wird ihre
Interessen fast zwangsläufig wirksamer schützen als eine
Massengruppierung.\footnote{Siehe Mancur Olson, \emph{The Logic of
  Collective Action} (Cambridge: Harvard University Press, 1965).} Und
selbst wenn die meisten Mitglieder der Gruppe sich entscheiden, bei
einer gemeinsamen Aktion nicht zu kooperieren, können einige wenige
Reiche in der Lage sein, genügend Ressourcen einzusetzen, um die Aufgabe
zu bewältigen.

Mit der demokratischen Entscheidungsfindung konnte der Nationalstaat
seine Macht über Millionen von Menschen, die nicht ohne weiteres
zusammenarbeiten konnten, um kollektiv in ihrem eigenen Interesse zu
handeln, viel umfassender ausüben, als gegenüber einer viel kleineren
Zahl, die die organisatorischen Schwierigkeiten bei der Verteidigung
ihrer konzentrierten Interessen leichter überwinden konnte. Die
Demokratie hatte den noch viel verlockenderen Vorteil, dass sie eine
legitimierende Entscheidungsregel schuf, die es dem Staat ermöglichte,
die Ressourcen der Wohlhabenden anzuzapfen, ohne direkt um ihre
Zustimmung verhandeln zu müssen. Kurzum, die Demokratie war als
Entscheidungsmechanismus gut an die megapolitischen Bedingungen des
Industriezeitalters angepasst. Sie ergänzte den Nationalstaat, weil sie
die Konzentration militärischer Macht in den Händen derjenigen
erleichterte, die ihn führten, und zwar zu einer Zeit, als das Ausmaß
der eingesetzten Kräfte wichtiger war als die Effizienz, mit der sie
mobilisiert wurden.

Ein entscheidender Beweis dafür war die Französische Revolution, die das
Ausmaß der militärischen Gewalt auf dem Schlachtfeld erhöhte. Danach
hatten andere konkurrierende Nationalstaaten kaum eine andere Wahl, als
sich einer ähnlichen Organisation anzunähern, wobei die Legitimität
letztlich an die demokratische Entscheidungsfindung gebunden war.
Zusammenfassend lässt sich sagen, dass der demokratische Nationalstaat
in den letzten zwei Jahrhunderten aus den folgenden, versteckten Gründen
erfolgreich war:

\begin{enumerate}
\def\labelenumi{\arabic{enumi}.}
\item
  Es gab steigende Renditen für Gewalt, so dass das Ausmaß der Gewalt
  wichtiger wurde als die Effizienz als Leitprinzip.
\item
  Die Einkommen stiegen so weit über das Existenzminimum, dass es dem
  Staat möglich wurde, große Mengen an Gesamtressourcen einzutreiben,
  ohne mit mächtigen Magnaten verhandeln zu müssen, die in der Lage
  waren, Widerstand zu leisten.
\item
  Die Demokratie erwies sich als hinreichend kompatibel mit dem
  Funktionieren freier Märkte, um die Schaffung von immer mehr Wohlstand
  zu ermöglichen.
\item
  Die Demokratie erleichterte die Beherrschung der Regierung durch ihre
  „Angestellten`` und stellte damit sicher, dass es schwierig sein
  würde, die Ausgaben, einschließlich der Militärausgaben, zu kürzen.
\item
  Die Demokratie als Entscheidungsregel erwies sich als wirksames
  Gegenmittel gegen die Fähigkeit der Wohlhabenden, gemeinsam zu
  handeln, um die Fähigkeit des Nationalstaates zu beschränken, Steuern
  zu erheben oder ihr Vermögen auf andere Weise vor Übergriffen zu
  schützen.
\end{enumerate}

Die Demokratie wurde zur militärisch siegreichen Strategie, weil sie es
dem Staat erleichterte, mehr Ressourcen in seine Hände zu bekommen. Im
Vergleich zu anderen Formen der Souveränität, deren Legitimität von
anderen Prinzipien abhing, wie z. B. der Feudalabgabe, dem göttlichen
Recht der Könige, der korporativen religiösen Pflicht oder den
freiwilligen Beiträgen der Reichen, wurde die Massendemokratie
militärisch am stärksten, weil sie der sicherste Weg war, um in einer
industriellen Wirtschaft Ressourcen zu sammeln.

\begin{quote}
„Die Nation als kulturell definierte Gemeinschaft ist der höchste
symbolische Wert der Moderne; sie wurde mit einem quasi sakralen
Charakter ausgestattet, der nur von der Religion übertroffen wird.
Tatsächlich leitet sich dieser quasi-sakrale Charakter von der Religion
ab. In der Praxis ist die Nation entweder zum modernen, säkularen Ersatz
für die Religion oder zu ihrem mächtigsten Verbündeten geworden. In der
modernen Zeit werden die von der Nation erzeugten Gemeinschaftsgefühle
hoch angesehen und als Grundlage für die Loyalität zur Gruppe gesucht...
Dass der moderne Staat oft der Nutznießer ist, sollte angesichts seiner
überragenden Macht kaum überraschen.`` \footnote{Josep R.
  Llobera,\emph{The God of Modernity: The Development of Nationalism in
  Western Europe} (Oxford: Berg Publishers, 1994), S. ix-x.} - Joseph R.
Llobera
\end{quote}

\subsection{Nationalismus}\label{nationalismus}

Ähnliches lässt sich über den Nationalismus sagen, der eine
Begleiterscheinung der Massendemokratie wurde. Staaten, die sich des
Nationalismus bedienen konnten, stellten fest, dass sie größere Armeen
zu geringeren Kosten mobilisieren konnten. Der Nationalismus war eine
Erfindung, die es einem Staat ermöglichte, seine militärische
Schlagkraft zu erhöhen. Wie die Politik selbst ist auch der
Nationalismus zumeist eine moderne Erfindung. Wie der Soziologe Joseph
Llobera in seinem reichhaltig dokumentierten Buch über den Aufstieg des
Nationalismus gezeigt hat, ist die Nation eine imaginäre Gemeinschaft,
die zu einem großen Teil als Mittel zur Mobilisierung staatlicher Macht,
während der Französischen Revolution, entstanden ist. Er drückt es so
aus: „Im modernen Sinne des Wortes gibt es das Nationalbewusstsein erst
seit der Französischen Revolution, seit der Zeit, als die
verfassungsgebende Versammlung 1789, das französische Volk mit der
französischen Nation gleichsetzte.`` \footnote{Ebenda., S. xiii.}

Der Nationalismus erleichterte die Mobilisierung von Macht und die
Kontrolle über eine große Zahl von Menschen. Nationalstaaten bildeten
sich, indem sie gemeinsame Merkmale, insbesondere die gesprochene
Sprache, hervorhoben und betonten. Dies erleichterte die Herrschaft ohne
die Einschaltung von Mittelsmännern. Es vereinfachte die Aufgaben der
Bürokratie. Erlasse, die nur in einer Sprache verkündet werden müssen,
können schneller und mit weniger Verwirrung versandt werden als solche,
die in ein Durcheinander von Sprachen übersetzt werden müssen. Der
Nationalismus senkte daher tendenziell die Kosten für die Kontrolle
größerer Gebiete. Vor dem Nationalismus benötigte der frühneuzeitliche
Staat die Hilfe von Herren, Herzögen, Grafen, Bischöfen, freien Städten
und anderen korporativen und ethnischen Vermittlern, von
Steuer-„Bauern`` bis hin zu militärischen Vertragshändlern und Söldnern,
um Einnahmen einzutreiben, Truppen aufzustellen und andere
Regierungsaufgaben zu erfüllen.

Der Nationalismus senkte auch entscheidend die Kosten für die
Mobilisierung von Militärpersonal, indem er die Identifikation der
Gruppe mit den Interessen des Staates förderte. Die Nutzung von
Gruppengefühlen für die Interessen des Staates war so vorteilhaft, dass
sich die meisten Staaten, selbst die angeblich internationalistische
Sowjetunion, dem Nationalismus als einer ergänzenden Ideologie
anschlossen.

Längerfristig gesehen ist der Nationalismus ebenso eine Anomalie wie der
Staat selbst. Wie der Historiker William McNeill dokumentiert hat, waren
polyethnische Souveränitäten in der Vergangenheit die Norm.\footnote{Siehe
  William McNeill, \emph{Polyethnicity and National Unity in World
  History} (Toronto: University of Toronto Press, 1986).}

In McNeills Worten: „Die Idee, dass eine Regierung rechtmäßig nur über
Bürger eines einzigen Ethnos herrschen sollte, begann sich in Westeuropa
gegen Ende des Mittelalters zu entwickeln.`` \footnote{Ebenda., S. 7.}
Eine frühe nationalistische Einheit war der Preußische Bund, der sich
1440 in Opposition zur Herrschaft des Deutschen Ordens bildete. Einige
der Merkmale des Ordens wurden bereits als polares Beispiel für eine
Souveränität im Gegensatz zum Nationalstaat hervorgehoben. Der Deutsche
Orden war eine Art Chartergesellschaft, deren Mitglieder fast alle nicht
aus Preußen stammten. Sein Hauptsitz wechselte zu verschiedenen Zeiten
von Bremen und Lübeck nach Jerusalem, Akkon, Venedig und schließlich
Marienberg an der Weichsel. Eine Zeit lang regierte sie den Bezirk
Burzenland in Siebenbürgen. Es ist nicht verwunderlich, dass eine
Souveränität, die einem Staat so unähnlich ist, zum Gegenstand eines der
ersten Versuche wurde, das Nationalgefühl als Faktor zur Organisation
von Macht zu mobilisieren. Ein Hinweis darauf, wie sehr sich der frühe
Nationalismus von späteren Spielarten unterschied, ist, dass die
deutschsprachigen Adligen des Preußischen Bundes den König von Polen
baten, Preußen unter polnische Herrschaft zu stellen, vor allem weil der
polnische König schon damals ein relativ schwacher Monarch war, von dem
man nicht erwartete, dass er mit der gleichen Strenge regierte wie der
Deutsche Orden.

Der Nationalismus kam in seinen frühen Formen kurz vor der
Schießpulverrevolution ins Spiel. Er entwickelte sich weiter, als sich
der frühneuzeitliche Staat entwickelte, und machte zur Zeit der
Französischen Revolution einen Quantensprung in seiner Bedeutung. Wir
sind der Meinung, dass der Nationalismus als Idee der Gewalt bereits auf
dem Rückzug ist. Seinen Höhepunkt erreichte er wahrscheinlich mit
Woodrow Wilsons Versuch, jede ethnische Gruppe in Europa am Ende des
Ersten Weltkriegs mit einem eigenen Staat auszustatten. Heute ist er
eine reaktionäre Kraft, die sich an Orten mit sinkenden Einkommen und
sinkenden Aussichten wie Serbien entzündet.

Wie wir später noch ausführen werden, erwarten wir, dass Nationalismus
ein Hauptthema von Personen mit geringer Qualifikation sein wird, die
sich nach Zwang sehnen, wenn der Wohlfahrtsstaat in den westlichen
Demokratien zusammenbricht. Noch haben Sie nichts gesehen. Für die
meisten Menschen im Westen waren die Folgen des Untergangs des
Kommunismus relativ harmlos. Sie haben einen Rückgang der
Militärausgaben, einen Einbruch der Aluminiumpreise und eine neue Quelle
von Eishockeyspielern für die NHL gesehen. Das sind die guten
Nachrichten. Das sind Nachrichten, die die meisten Menschen, die im
zwanzigsten Jahrhundert aufgewachsen sind, begrüßen könnten, vor allem,
wenn sie Eishockey-Fans sind. Die meisten Nachrichten, die sich als
weniger populär erweisen werden, stehen noch aus.

Mit dem Ende des Industriezeitalters fallen die megapolitischen
Bedingungen, die die Demokratie erfüllte, rasch weg. Daher ist es
zweifelhaft, dass die Massendemokratie und der Wohlfahrtsstaat unter den
neuen megapolitischen Bedingungen des Informationszeitalters lange
überleben werden.

\begin{quote}
„Der Kongress war kein Tempel der Demokratie, sondern ein Markt für den
Tausch von Gesetzen.`` - Alberto Fujimori, Präsident von Peru
\end{quote}

Zukünftige Historiker werden vielleicht berichten, dass wir bereits den
ersten postmodernen Staatsstreich erlebt haben - die bemerkenswerte
Abriegelung des Kongresses in Peru im Jahr 1993. Dies war nicht wirklich
ein Ereignis, das in den führenden Industriedemokratien viel
Aufmerksamkeit erregte. Aber es könnte sich im Laufe der Zeit als
bedeutsamer erweisen, als herkömmliche Analysten vermuten würden. Die
wenigen, die darüber nachgedacht haben, neigen dazu, es nur als eine
weitere Machtergreifung der Art zu sehen, die in der Geschichte
Lateinamerikas deprimierend vertraut geworden ist. Wir aber sehen darin
vielleicht den ersten Schritt zur Delegitimierung einer Regierungsform,
deren unmittelbare megapolitische Daseinsberechtigung mit dem Übergang
zum Informationszeitalter zu verschwinden beginnt. Fujimoris Schließung
des Kongresses ist ein Symptom für die endgültige Entwertung politischer
Versprechen. Ein ähnliches Schicksal könnte andere Parlamente erwarten,
wenn ihr Kredit aufgebraucht ist.

Der technologische Wandel, der den Industrialismus aushöhlt, hat in
vielen Ländern zu Regierungen geführt, die nicht mehr funktionieren.
Oder schlecht funktionieren. Insbesondere die Legislative scheint in
zunehmendem Maße dysfunktional zu sein. Sie erlassen Gesetze, die vor
fünfzig Jahren vielleicht nur dumm waren, heute aber gefährlich sind.
Dies zeigte sich auf spektakuläre Weise in Peru, wo die innere
Souveränität des Staates 1993 fast zusammengebrochen war.

\begin{quote}
„Überfälle, Entführungen, Vergewaltigungen und Morde gehen einher mit
einem zunehmend aggressiven Fahrverhalten und unsicheren Straßen. Die
Polizei hat allmählich die Kontrolle über die Situation verloren, und
einige ihrer Mitglieder sind in Skandale verwickelt und zu routinierten
Kriminellen geworden... Die Menschen haben sich allmählich daran
gewöhnt, außerhalb des Gesetzes zu leben. Diebstahl, illegale
Beschlagnahmungen und Fabrikübernahmen sind alltäglich geworden.``
\footnote{Hernando de Soto, \emph{The Other Path} (New York: Harper \&
  Row, 1989).} Hernando de Soto
\end{quote}

\subsection{Peru in Trümmern}\label{peru-in-truxfcmmern}

In gewisser Weise war Peru 1993 kein moderner Nationalstaat mehr. Es
hatte zwar noch eine Flagge und eine Armee, aber die meisten seiner
Institutionen lagen in Trümmern. Sogar die Gefängnisse waren von den
Insassen übernommen worden. Für diesen Zerfall gibt es eine Reihe von
Ursachen, aber die meisten Erklärungsversuche der Experten gehen am Kern
der Sache vorbei. Peru war ein frühes Opfer des technologischen Wandels,
der geschlossene Volkswirtschaften dysfunktional macht und die zentrale
Autorität überall untergräbt. Diese megapolitischen Spannungen werden
noch dadurch verstärkt, dass Entscheidungsinstitutionen wie der
peruanische Kongress durch perverse Anreize dazu verleitet werden, genau
die Probleme zu bündeln, die sie am dringendsten lösen müssen.

Die repräsentative Demokratie in Peru war wie ein Paar gezinkter Würfel.
Als Entscheidungsmechanismus zur Vergrößerung des Staates war sie
unübertroffen. Doch als die neuen Umstände die Übertragung von Macht
verlangten, machten die inhärenten Verzerrungen, die die Demokratie
unter den alten megapolitischen Bedingungen so nützlich gemacht hatten,
sie zunehmend dysfunktional. Die Gesetze, die der Kongress
verabschiedete, zerstörten rasch jede Grundlage von Wert und Respekt vor
dem Gesetz. Wie de Soto es in The Other Path ausdrückt: „Kleine
Interessengruppen bekämpfen sich untereinander, verursachen Konkurse und
ziehen Beamte hinein. Die Regierungen verteilen Privilegien. Das Gesetz
wird benutzt, um weit mehr zu geben und zu nehmen, als die Moral
erlaubt.`` \footnote{Ebenda.} Ein Kongress wie der peruanische, der ganz
und gar den Interessengruppen hörig ist, hat die moralische Statur einer
Hehlerbande, die gestohlene Waren versteigert. Er hat den freien Markt
illegal und damit das Gesetz lächerlich gemacht. Wie de Soto über die
Zeit vor Fujimori schreibt:

\begin{quote}
Eine völlige Umkehrung von Zweck und Mitteln hat das Leben der
peruanischen Gesellschaft auf den Kopf gestellt, so dass es Handlungen
gibt, die zwar offiziell kriminell sind, aber vom kollektiven
Bewusstsein nicht mehr verurteilt werden. Der Schmuggel ist ein Beispiel
dafür. Jeder, von der aristokratischen Dame bis zum einfachen Mann,
erwirbt Schmuggelware. Niemand hat dabei Skrupel, im Gegenteil, es wird
als eine Art Herausforderung an den individuellen Einfallsreichtum oder
als Rache gegen den Staat angesehen. Das Eindringen von Gewalt und
Kriminalität in den Alltag geht einher mit zunehmender Armut und
Entbehrung. Generell ist das reale Durchschnittseinkommen der Peruaner
in den letzten zehn Jahren stetig gesunken und liegt heute auf dem
Niveau von vor zwanzig Jahren. Auf allen Seiten türmen sich Berge von
Müll auf. Tag und Nacht belagern Legionen von Bettlern, Autowäschern und
Plünderern die Passanten und bitten um Geld. Geisteskranke tummeln sich
nackt auf den Straßen und stinken nach Urin. Kinder, alleinstehende
Mütter und Krüppel betteln an jeder Ecke um Almosen. Der traditionelle
Zentralismus unserer Gesellschaft hat sich eindeutig als unfähig
erwiesen, die vielfältigen Bedürfnisse eines Landes im Wandel zu
befriedigen.\footnote{Ebenda., S .6.}
\end{quote}

De Soto bezeichnete die Abkehr von der grotesken gesetzmäßigen
Wirtschaft zugunsten des Schwarzmarktes, die im Gange war, bevor
Fujimori den Kongress mit Vorhängeschlössern versperrte, als „eine
unsichtbare Revolution``.

Wir stehen den Vorteilen des freien Marktes positiv gegenüber, aber weit
weniger positiv dem Versprechen einer Gesellschaft, in der das Recht
genauso verkommen ist wie das Geld. Die Welt, die de Soto vor 1993 in
Peru beschrieb, war eine „Clockwork Orange``-Welt, in der übermäßig
zentralisierte und dysfunktionale staatliche Institutionen die
Zivilgesellschaft buchstäblich zerstörten.

Genau das wollte Fujimori ändern. Er hatte die Inflation gesenkt, indem
er die Druckerpressen abstellte. Es war ihm auch gelungen,
fünfzigtausend Staatsbedienstete zu entlassen und einige Subventionen zu
kürzen. Er hatte einen Anfang gemacht, um den Haushalt auszugleichen.
Sein Reformprogramm enthielt umfassende Pläne zur Schaffung freier
Märkte und zur Privatisierung der Industrie. Aber wie in der ehemaligen
Sowjetunion waren die meisten wichtigen Elemente von Fujimoris Reform
1993 noch nicht verabschiedet, einschließlich der ersten Runde der groß
angelegten Privatisierung von staatlichen Banken, Bergbauunternehmen und
Versorgungsbetrieben. Anstatt diese notwendigen Vorschläge umzusetzen,
versuchte der peruanische Kongress, ähnlich wie der russische Kongress,
der Jelzins Reformen in Moskau in Frage stellte, einen Rückzieher zu
machen. Ihr Plan: Wiederherstellung von Subventionen aus einer leeren
Staatskasse, Aufstockung der Gehaltslisten und Schutz aller
Besitzstände, insbesondere der Bürokratie - genau das, was man von einer
Regierung erwarten würde, die von ihren Angestellten kontrolliert wird.

Fujimori behauptete, der peruanische Kongress sei unentschlossen und
korrupt, eine Tatsache, der fast jeder zustimmte. Er behauptete weiter,
dass das Zaudern und die Korruption des Kongresses es unmöglich machten,
die kollabierende peruanische Wirtschaft zu reformieren oder einen
gewaltsamen Angriff der Narco-Terroristen und der nihilistischen Sendero
Luminoso (Leuchtender Pfad) Guerilla zu bekämpfen.

\subsection{Die 70-Prozent-Lösung}\label{die-70-prozent-luxf6sung}

Fujimori schloss also den Kongress, ein Akt, der darauf hätte hindeuten
können, dass er genauso autoritär war wie viele frühere
lateinamerikanische Führer. Wir waren jedoch der Meinung, und haben dies
damals auch gesagt, dass Fujimori ein grundlegendes Reformhindernis
richtig erkannt hatte. Die überschwänglichen Klagelieder amerikanischer
Redakteure und Beamten des Außenministeriums bezüglich des peruanischen
Kongresses wurden von der peruanischen Bevölkerung nicht geteilt.
Während die Nordamerikaner so taten, als sei der peruanische Kongress
die Inkarnation von Freiheit und Zivilisation, jubelte das peruanische
Volk. Die Popularität von Präsident Fujimori stieg auf über 70 Prozent,
als er den Kongress nach Hause schickte. Und er wurde später mit einem
Erdrutschsieg für eine zweite Amtszeit wiedergewählt. Die meisten Bürger
sahen ihre Legislative offenbar eher als Hindernis für ihr Wohlergehen
denn als Ausdruck ihrer Rechte. 1994 erreichte das reale
Wirtschaftswachstum in Peru 12,9 \%, das höchste der Welt.

\subsection{Die Deflation politischer
Versprechen}\label{die-deflation-politischer-versprechen}

Wir sahen in den Unruhen in Peru weniger einen Rückfall in die
Diktaturen der Vergangenheit als vielmehr eine erste Etappe einer
umfassenderen Übergangskrise. Es ist zu erwarten, dass es in vielen
Ländern zu Fehlregierungen kommen wird, wenn politische Versprechen
enttäuscht werden und den Regierungen das Geld ausgeht. Letztendlich
werden sich neue institutionelle Formen herausbilden müssen, die in der
Lage sind, die Freiheit unter den neuen technologischen Bedingungen zu
bewahren und gleichzeitig den gemeinsamen Interessen aller Bürger
Ausdruck und Leben zu verleihen.

Nur wenige haben begonnen, über die Unvereinbarkeit zwischen einigen der
Institutionen der industriellen Regierung und der Megapolitik der
postindustriellen Gesellschaft nachzudenken. Unabhängig davon, ob diese
Widersprüche ausdrücklich anerkannt werden oder nicht, werden ihre
Folgen nichtsdestotrotz immer deutlicher, wenn sich die Beispiele für
politisches Versagen in der ganzen Welt häufen. Die
Regierungsinstitutionen, die in der Neuzeit entstanden sind, spiegeln
die megapolitischen Bedingungen von vor einem oder mehreren
Jahrhunderten wider. Das Informationszeitalter wird neue Mechanismen der
Repräsentation erfordern, um chronische Funktionsstörungen und sogar den
sozialen Zusammenbruch zu vermeiden.

Als 1989 die Berliner Mauer fiel, bedeutete dies nicht nur das Ende des
Kalten Krieges, sondern war auch das äußere Zeichen für ein stilles
Erdbeben in den Grundfesten der Macht in der Welt. Es war das Ende einer
langen Periode der zunehmenden Rendite der Gewalt. Der Untergang des
Kommunismus, den wir 1987 in Blood in the Streets und noch früher in
unserem monatlichen Newsletter Strategic Investment prognostizierten,
war nicht nur die Ablehnung einer Ideologie. Er war das äußere Zeichen
für die wichtigste Entwicklung in der Geschichte der Gewalt in den
letzten fünf Jahrhunderten. Wenn unsere Analyse richtig ist, wird sich
die Organisation der Gesellschaft zwangsläufig ändern, um die
zunehmenden Ungleichgewichte bei der Anwendung von Gewalt
widerzuspiegeln. Die Grenzen, innerhalb derer die Zukunft liegen muss,
sind neu gezogen worden.

\setsubtitle{Der Triumph der Effizienz über die Macht}

\bookmarksetup{startatroot}

\chapter{DIE MEGAPOLITIK DES
INFORMATIONSZEITALTERS}\label{die-megapolitik-des-informationszeitalters}

\begin{quote}
„\ldots es ist die computergestützte Information, nicht die Arbeitskraft
oder die Massenproduktion, die zunehmend die US-Wirtschaft antreibt und
die in einer Welt, die für 500 Fernsehkanäle verdrahtet ist, Kriege
gewinnen wird. Die computergestützte Information existiert im Cyberspace
- die neue Dimension, die durch die endlose Reproduktion von
Computernetzwerken, Satelliten, Modems, Datenbanken und dem öffentlichen
Internet geschaffen wurde.`` \footnote{Neil Munro, \emph{The
  Pentagon\textquotesingle s New Nightmare: An Electronic Pearl Harbor,}
  Washington Post, 16. Juli 1995, S. C3.} - Neil Munro
\end{quote}

Am 30. Dezember 1936 besetzten Automobilarbeiter, die höhere Löhne
forderten, gewaltsam zwei der Hauptwerke von General Motors in Flint,
Michigan. Sie legten die Maschinen still, schalteten die Fließbänder ab
und machten es sich gemütlich. Die Arbeiter, die für den Betrieb der
Fabriken eingestellt worden waren, setzten sich in einem Streik
zusammen, der viele Wochen dauern sollte. Es war ein Drama, das von
gewalttätigen Ausschreitungen und den schwankenden Loyalitäten der
Polizei, der Miliz von Michigan und politischen Persönlichkeiten auf
allen Regierungsebenen geprägt war. Da die Gewerkschaft bei der
Durchsetzung ihrer Forderungen kaum Fortschritte machte, streikte sie am
1. Februar 1937 erneut.

Gewerkschaften übernahmen gewaltsam die Chevrolet-Fabrik von GM in
Flint. Indem sie die wichtigsten Fabriken von General Motors besetzten
und schlossen, lähmten die Arbeiter effektiv die Produktionskapazität
des Unternehmens. In den zehn Tagen nach der Besetzung der dritten
Fabrik produzierte GM in den USA nur 153 Autos.

Wir greifen diese Kurzmeldung von vor sechzig Jahren wieder auf, um die
Revolution in den megapolitischen Verhältnissen, die jetzt stattfindet,
in einen klareren Bezug zu setzen. Der GM-Sitzstreik ereignete sich
innerhalb der Lebenszeiten einiger Leser dieses Buches. Doch wir
glauben, dass Sitzstreiks im Informationszeitalter genauso veraltet sein
werden wie Sklaven, die mit riesigen Steinen durch die Wüste ziehen, um
Bestattungspyramiden für die Pharaonen zu errichten. Während
Gewerkschaften und ihre Einschüchterungstaktiken in der Industriezeit so
vertraut wurden, dass sie als unbestrittener Teil der sozialen
Landschaft gelten, waren sie auf spezielle megapolitische Bedingungen
angewiesen, die schnell verblassen. Auf der Autobahn der Informationen
wird es weder Chevrolets noch Gewerkschaften zum Bestreiken geben.

Die Schicksale von Regierungen werden denen ihrer Gegenstücke, den
Gewerkschaften, in den Niedergang folgen. Institutionalisierter Zwang,
der eine wesentliche Rolle in der Gesellschaft des zwanzigsten
Jahrhunderts spielte, wird nicht mehr möglich sein. Technologie führt zu
einer tiefgreifenden Veränderung in der Logik von Erpressung und Schutz.

\begin{quote}
„... es gibt keinen Privatbesitz, keine Herrschaft, kein Meins und Deins
als Unterscheidung; sondern nur, dass jedem das gehört, was er sich
erarbeiten kann, und nur so lange, wie er es behalten kann.``
\footnote{Thomas Hobbes, Leviathan, Kap. 13 \emph{The Natural Condition
  of Man as Concerning Their Felicity and Misery.}} - Thomas Hobbes
\end{quote}

\subsection{Erpressung und Schutz}\label{erpressung-und-schutz}

Im Laufe der Geschichte war Gewalt stets ein Dolch, der das Herz der
Wirtschaft bedrohte. Wie Thomas Schelling treffend sagte: „Die Macht zu
verletzen - Dinge zu zerstören, die jemand schätzt, Schmerz und Leid
zuzufügen - ist eine Art von Verhandlungsmacht, nicht leicht
einzusetzen, aber oft eingesetzt. In der Unterwelt ist es die Basis für
Erpressung, Raub und Entführung, in der Geschäftswelt für Boykott,
Streiks und Aussperrungen... Es ist oft die Basis für Disziplin, zivil
und militärisch; und die Götter benutzen es, um Disziplin
einzufordern.`` \footnote{Thomas Schelling, \emph{Arms and Influence}
  (New Haven: Yale University Press, 1966).} Die Fähigkeit einer
Regierung, zu besteuern, hängt von denselben Schwachstellen ab wie
private Erpressung und Raub. Obwohl wir es nicht in diesen Begriffen
wahrnehmen, bietet der Anteil der Vermögenswerte, die durch Kriminalität
und Regierung kontrolliert und zwangsweise ausgegeben werden, eine grobe
Messung des megapolitischen Gleichgewichts zwischen Erpressung und
Schutz. Wenn Technologie den Schutz von Vermögenswerten erschwerte, wäre
Kriminalität weit verbreitet, ebenso wie Gewerkschaftstätigkeiten. Unter
solchen Umständen würde der Schutz durch die Regierung daher einen
Aufschlag verlangen. Die Steuern wären hoch. Wo Steuern niedriger sind
und Lohnsätze am Arbeitsplatz durch Marktkräfte und nicht durch
politische Eingriffe oder Zwang bestimmt werden, hat die Technologie das
Gleichgewicht in Richtung Schutz gekippt.

Das technologische Ungleichgewicht zwischen Erpressung und Schutz
erreichte im letzten Drittel des zwanzigsten Jahrhunderts ein Extrem. In
einigen fortschrittlichen westlichen Gesellschaften wurden mehr als die
Hälfte der Ressourcen von den Regierungen in Anspruch genommen. Die
Einkommen eines großen Teils der Bevölkerung wurden entweder per Erlass
festgelegt oder unter dem Einfluss von Zwang bestimmt, zum Beispiel
durch Streiks und Gewaltandrohungen in anderen Formen. Der
Wohlfahrtsstaat und die Gewerkschaft waren beides technologische
Artefakte, die sich als Triumph der Macht über die Effizienz im
zwanzigsten Jahrhundert die Beute teilten. Sie hätten ohne die
Technologien, sowohl militärische als auch zivile, die den Ertrag von
Gewalt während des Industriezeitalters erhöhten, nicht existieren
können.

Die Fähigkeit, Vermögenswerte zu schaffen, hat schon immer eine gewisse
Anfälligkeit für Erpressung mit sich gebracht. Je größer die
geschaffenen oder besessenen Vermögenswerte, desto höher der zu zahlende
Preis, auf die eine oder andere Weise. Entweder man bezahlte jeden, der
die Hebelwirkung besaß, Gewalt für Erpressungszwecke einzusetzen, oder
man investierte in militärische Macht, die in der Lage war, jeglichen
Versuch einer Erpressung mit roher Gewalt zu vereiteln.

\begin{quote}
„Man soll nicht mehr von Frevel hören in deinem Lande noch von Schaden
oder Verderben in deinen Grenzen...`` - Jesaja 60:18
\end{quote}

\subsection{Die Mathematik des
Schutzes}\label{die-mathematik-des-schutzes}

Nun könnte die Klinge der Gewalt bald stumpf gemacht werden. Die
Informationstechnologie verspricht, das Gleichgewicht zwischen Schutz
und Erpressung dramatisch zu ändern und in vielen Fällen den Schutz von
Vermögenswerten zu erleichtern und die Erpressung zu erschweren. Die
Technologie des Informationszeitalters ermöglicht es, Vermögenswerte zu
schaffen, die vielen Formen von Zwang unzugänglich sind. Diese neue
Asymmetrie zwischen Schutz und Erpressung basiert auf einer
Grundwahrheit der Mathematik. Es ist einfacher zu multiplizieren als zu
teilen. Auch wenn diese Wahrheit grundlegend ist, waren ihre
weitreichenden Folgen vor dem Aufkommen von Mikroprozessoren
verschleiert. Hochgeschwindigkeitscomputer haben in den letzten zehn
Jahren viele Milliarden Mal mehr Berechnungen ermöglicht als in der
gesamten vorangegangenen Geschichte der Welt. Dieser Sprung in der
Berechnung hat es uns zum ersten Mal ermöglicht, einige der universellen
Eigenschaften der Komplexität zu erfassen. Was die Computer zeigen, ist,
dass komplexe Systeme nur von unten nach oben aufgebaut und verstanden
werden können. Primzahlen zu multiplizieren ist einfach. Aber die
Komplexität aufzuschlüsseln, indem man versucht, das Produkt großer
Primzahlen zu zerlegen, ist so gut wie unmöglich. Kevin Kelly, der
Herausgeber von Wired, bringt es so auf den Punkt: „Mehrere Primzahlen
in ein größeres Produkt zu multiplizieren, ist einfach; jedes
Grundschulkind kann das. Aber die Supercomputer der Welt ersticken beim
Versuch, ein Produkt in seine einfachen Primzahlen zu zerlegen.``
\footnote{Kevin Kelly, \emph{Out of Control: The New Biology of
  Machines, Social Systems, and the Economic World} (Reading, Mass.:
  Addison\textasciitilde Wesley, 1995), S. 45-46.}

\subsection{Die Logik von komplexen
Systemen}\label{die-logik-von-komplexen-systemen}

Die Cyber-Ökonomie wird unvermeidlich durch diese tiefe mathematische
Wahrheit geformt. Sie hat bereits einen offensichtlichen Ausdruck in
leistungsstarken Verschlüsselungsalgorithmen. Wie wir später in diesem
Kapitel erkunden werden, ermöglichen diese Algorithmen die Schaffung
eines neuen, geschützten Bereichs des Cyber-Handels, in dem der Hebel
der Gewalt stark reduziert sein wird. Das Gleichgewicht zwischen
Erpressung und Schutz wird sich drastisch in Richtung Schutz
verschieben. Dies wird das Aufkommen einer Wirtschaft erleichtern, die
mehr auf spontanen Anpassungsmechanismen beruht und weniger auf
bewusster Entscheidungsfindung und Ressourcenzuweisung durch Bürokratie.
Das neue System, in dem der Schutz im Vordergrund steht, wird sich stark
von dem unterscheiden, was aus der Dominanz von Zwang in der
industriellen Periode hervorging.

\subsection{Befehls- und Kontrollsysteme sind
primitiv}\label{befehls--und-kontrollsysteme-sind-primitiv}

In \emph{The Great Reckoning} haben wir geschrieben, dass der Computer
uns ermöglicht, die bisher unsichtbare Komplexität zu „sehen``, die in
einer Vielzahl von Systemen inhärent ist.\footnote{Siehe Kapitel 8 von
  \emph{The Great Reckoning}: Lineare Erwartungen in einer nichtlinearen
  Welt: Wie das Teleskop uns zum Rechnen führte; wie der Computer uns
  helfen kann zu sehen.} Nicht nur erweitert die fortgeschrittene
Rechenfähigkeit unser Verständnis über die Dynamik komplexer Systeme,
sie ermöglicht uns auch, diese Komplexitäten auf produktive Weise zu
nutzen. In gewisser Weise ist dies nicht einmal eine Wahl, sondern eine
Unvermeidlichkeit, wenn die Wirtschaft über die unflexible Phase der
zentralen Kontrolle hinaus fortgeschritten sein soll. Ein solches
System, das auf linearen Beziehungen basiert, ist grundsätzlich
primitiv. Die staatliche Aneignung von Ressourcen lenkt diese
zwangsläufig von hochwertigen, komplexen Nutzungen hin zu primitiven,
geringwertigen Nutzungen. Es ist ein Prozess, der durch die gleiche
mathematische Asymmetrie begrenzt ist, die das Aufspüren des Produkts
großer Primzahlen verhindert. Das Teilen der Beute kann nie etwas
anderes als primitiv sein.

\subsection{Alles wird komplexer}\label{alles-wird-komplexer}

Überall wo man im Universum hinschaut, sieht man Systeme, die mit ihrer
Evolution zunehmend komplexer werden. Das gilt in der Astrophysik. Das
gilt in einer Pfütze. Lässt man Regenwasser an einer niedrigen Stelle
stehen, wird es komplexer. Fortgeschrittene Systeme jeder Art sind
komplexe adaptive Systeme ohne eine leitende Autorität. Jedes komplexe
System in der Natur, wovon die Marktwirtschaft die offensichtlichste
soziale Ausdrucksform ist, beruht auf verteilten Fähigkeiten. Systeme,
die unter der breitesten Palette von Bedingungen am effektivsten
arbeiten, sind für ihre Widerstandsfähigkeit auf spontane Ordnung
angewiesen, die neuartige Möglichkeiten aufnimmt. Das Leben selbst ist
ein solches komplexes System. Milliarden potenzieller Kombinationen von
Genen ergeben ein einzelnes menschliches Individuum. Die Auswahl unter
ihnen würde jede Bürokratie überfordern.

Vor fünfundzwanzig Jahren wäre das nur eine Intuition gewesen. Heute ist
es nachweisbar. Je näher uns Computer dem Verständnis der Mathematik
künstlichen Lebens bringen, desto besser verstehen wir die Mathematik
des realen Lebens, die der biologischen Komplexität. Diese Geheimnisse
der Komplexität, durch Informationstechnologie genutzt, ermöglichen es,
Volkswirtschaften in komplexere Formen umzugestalten. Das Internet und
das World Wide Web haben bereits Eigenschaften eines organischen Systems
angenommen, wie Kevin Kelly in „Out of Control: The New Biology of
Machines, Social Systems, and the Economic World``
vorschlägt.\footnote{Ebenda, S. 2-4.} In seinen Worten ist die Natur
„eine Ideenfabrik. Lebenswichtige, postindustrielle Paradigmen sind in
jedem Ameisenhügel im Dschungel verborgen... Die Großhandelsübertragung
von Biologie in Maschinen sollte uns mit Ehrfurcht erfüllen. Wenn die
Einheit von Geborenem und Gemachtem vollständig ist, werden unsere
Fabrikationen lernen, sich anpassen, sich selbst heilen und sich
entwickeln. Dies ist eine Macht, von der wir bisher kaum geträumt
haben.`` \footnote{Ebenda, S. 4.}

Tatsächlich dürften die Folgen der „großflächigen Übertragung von
Biologie in Maschinen`` weitreichend sein. Es hat schon immer eine
starke Tendenz gegeben, dass soziale Systeme die Eigenschaften der
vorherrschenden Technologie nachbilden. Dies ist etwas, das Marx richtig
erkannt hat. Gigantische Fabriken fielen mit dem Zeitalter der großen
Regierungen zusammen. Mikroprozessoren machen Institutionen kleiner.
Wenn unsere Analyse zutrifft, wird die Technologie des
Informationszeitalters letztendlich eine Wirtschaft schaffen, die besser
geeignet ist, die Vorteile von Komplexität zu nutzen.

Die megapolitischen Dimensionen einer solchen Veränderung sind so wenig
verstanden, dass selbst die meisten von denen, die ihre mathematische
Bedeutung erkannt haben, dies auf eine anachronistische Weise getan
haben. Es fällt einfach schwer, vollständig zu begreifen und zu
verinnerlichen, dass der technologische Wandel in den nächsten Jahren
wahrscheinlich die meisten politischen Formen und Konzepte der modernen
Welt überholen wird. Zum Beispiel schrieb der verstorbene Physiker Heinz
Pagels in seinem weitsichtigen Buch \emph{Die Träume der Vernunft}: „Ich
bin davon überzeugt, dass die Nationen und Menschen, die die neue
Wissenschaft der Komplexität beherrschen, die wirtschaftlichen,
kulturellen und politischen Supermächte des nächsten Jahrhunderts
werden.`` \footnote{Heinz Pagels, \emph{The Dreams of Reason} (New York:
  Bantam Books, 1989), zitiert in Roger Lewin, \emph{Complexity: Life at
  the Edge of Chaos} (New York: Macmillan, I992), S. 10.} Es ist eine
beeindruckende Prognose. Aber wir glauben, dass sie zwangsläufig falsch
sein muss, nicht weil sie falsch wahrgenommen wird, sondern gerade weil
sie sich als richtiger herausstellen wird, als Dr.~Pagels zu äußern
wagte. Gesellschaften, die sich selbst so umgestalten, dass sie
komplexere adaptive Systeme werden, werden in der Tat prosperieren. Aber
wenn sie es tun, werden sie wahrscheinlich keine Nationen und erst recht
keine „politischen Supermächte`` mehr sein. Die wahrscheinlich
unmittelbaren Nutznießer der erhöhten Komplexität sozialer Systeme
werden vielmehr die souveränen Einzelpersonen des neuen Jahrtausends
sein.

So wie Pagels Vorhersage dasteht, ist es, als ob ein Schamane einer
Jagdgruppe von vor fünfhundert Generationen seinen Männern erzählt,
während sie um das Lagerfeuer kauern: „Ich bin überzeugt, dass die erste
Jagdgruppe, die die neue Wissenschaft des bewässerten Pflanzens
beherrscht, mehr Freizeit für Geschichtenerzählen haben wird, als sogar
jene Kerle am See, die die großen Fische fangen.`` So richtig er auch
über die Bedeutung der Komplexität lag, Pagels hat die grundlegendste
Tatsache von allen übersehen. Wenn sich die Logik der Gewalt ändert,
verändert sich die Gesellschaft.

\section{DIE LOGIK DER GEWALT}\label{die-logik-der-gewalt}

Um zu verstehen, wie und warum, ist es notwendig, sich auf mehrere
Aspekte der Megapolitik zu konzentrieren, die selten in Betracht gezogen
werden. Dies sind Themen, die der Historiker Frederic C. Lane untersucht
hat, dessen Arbeit über Gewalt und die wirtschaftliche Bedeutung des
Krieges an anderer Stelle in diesem Band besprochen wird. Als Lane in
der Mitte dieses Jahrhunderts schrieb, war die Informationsgesellschaft
noch nirgends in Sicht. Unter den gegebenen Umständen könnte er durchaus
angenommen haben, dass der Wettbewerb um Gewaltanwendung in der Welt,
mit dem Erscheinen des Nationalstaats seine letzte Phase erreicht hatte.
In seinen Werken gibt es keinen Hinweis darauf, dass er die
Mikroverarbeitung vorhergesehen hat oder glaubte, dass es technologisch
machbar wäre, Vermögenswerte im Cyberspace zu schaffen, einem Bereich
ohne physische Existenz. Lane hatte nichts über die Auswirkungen der
Möglichkeit zu sagen, dass große Massen an Handel nahezu immun gegen die
Hebelwirkung der Gewalt gemacht werden könnten.

Obwohl Lane die derzeit ablaufenden technologischen Revolutionen nicht
vorhersehen konnte, waren seine Erkenntnisse über die verschiedenen
Stufen der Monopolisierung von Gewalt in der Vergangenheit so klar, dass
sie offensichtliche Anwendung auf die aufkommende Informationsrevolution
finden. Lanes Untersuchung der gewalttätigen mittelalterlichen Welt
lenkte seine Aufmerksamkeit auf Themen, die konventionelle Ökonomen und
Historiker tendenziell vernachlässigen. Er erkannte, dass die
Organisation und Kontrolle von Gewalt eine große Rolle dabei spielt,
welche Verwendung von knappen Ressourcen gemacht wird.\footnote{Lane,
  \emph{Economic Consequences of Organized Violence,} ebenda, S. 402.}
Lane erkannte auch, dass die Produktion von Gewalt in der Regel nicht
als Teil des wirtschaftlichen Outputs betrachtet wird, die Kontrolle von
Gewalt jedoch für die Wirtschaft von entscheidender Bedeutung ist. Die
Hauptaufgabe der Regierung besteht darin, Schutz vor Gewalt zu
gewährleisten. Wie er es ausdrückte:

„Jedes wirtschaftliche Unternehmen benötigt und bezahlt für Schutz.
Schutz vor der Zerstörung oder bewaffneten Übernahme seines Kapitals und
der gewaltsamen Störung seiner Arbeit. In hochorganisierten
Gesellschaften ist die Bereitstellung dieser Dienstleistung, des
Schutzes, eine der Funktionen einer besonderen Vereinigung oder eines
Unternehmens, das man Regierung nennt. Tatsächlich ist eine der
auffälligsten Eigenschaften von Regierungen ihr Versuch, Recht und
Ordnung zu schaffen, indem sie selbst Gewalt anwenden und durch
verschiedene Mittel die Anwendung von Gewalt durch andere
kontrollieren.`` \footnote{Frederic C. Lane, \emph{The Economic Meaning
  of War and Protection}, in Venice and History: The Collected Papers of
  Frederic C. Lane (Baltimore: The Johns Hopkins Press, 1966), S.
  383-384.}

Das ist ein Punkt, der offenbar zu grundlegend ist, um in Lehrbüchern zu
erscheinen oder Teil der staatsbürgerlichen Diskussion zu sein, die
vermutlich den Verlauf der Politik bestimmt. Aber er ist auch zu
grundlegend, um ignoriert zu werden, wenn man die sich entfaltende
Informationsrevolution verstehen will. Der Schutz von Leben und Eigentum
ist in der Tat ein entscheidendes Bedürfnis, das jede Gesellschaft, die
jemals existiert hat, geplagt hat. Wie man gewalttätige Aggression
abwehren kann, ist das zentrale Dilemma der Geschichte. Es kann nicht
leicht gelöst werden, trotz der Tatsache, dass Schutz auf mehr als eine
Weise bereitgestellt werden kann.

\subsection{Das Ende einer Ära}\label{das-ende-einer-uxe4ra}

Während wir schreiben, beginnen die megapolitischen Folgen des
Informationszeitalters gerade erst spürbar zu werden. Der
wirtschaftliche Wandel der letzten Jahrzehnte hat dazu geführt, dass
Informations- und Rechenleistung das primäre Gut ist, weit vor
Maschinenkraft. Es findet eine Verlagerung statt von der Fabrik zum
Arbeitsplatz, von der Massenproduktion zu kleineren Teams oder sogar zu
Einzelpersonen, die alleine arbeiten. Mit der Abnahme hoch organisierter
Großunternehmen sinkt auch das Potential für Sabotage und Erpressungen
am Arbeitsplatz. Kleinere Betriebseinheiten sind zudem deutlich schwerer
von Gewerkschaften zu organisieren.

Mikrotechnologie ermöglicht es Unternehmen, kleiner und mobiler zu sein.
Viele handeln mit Dienstleistungen oder Produkten, die kaum natürliche
Ressourcen enthalten. Prinzipiell könnten diese Geschäfte fast überall
auf dem Planeten betrieben werden. Sie sind nicht an einen bestimmten
Ort gebunden, wie eine Mine oder ein Hafen. Daher werden sie im Laufe
der Zeit weit weniger anfällig dafür sein, entweder von Gewerkschaften
oder von Politikern besteuert zu werden. Eine alte chinesische
Volksweisheit besagt: „Von den sechsunddreißig Möglichkeiten, sich aus
Schwierigkeiten herauszuwinden, ist die beste - zu gehen.`` \footnote{Shi
  Mai\textquotesingle an und Lao Guanzhong, \emph{Outlaws of the Marsh},
  trans. Sidney Shapiro (Bloomington: Indiana University Press, 1981),
  S. 12.}

Im Informationszeitalter wird diese östliche Weisheit leicht anwendbar
sein. Wenn Betriebe aufgrund übermäßiger Anforderungen an einem Ort
unbequem werden, wird es wesentlich leichter sein, umzuziehen.
Tatsächlich wird es im Informationszeitalter, wie wir weiter unten
ausführen, möglich sein, virtuelle Unternehmen zu gründen, deren Domizil
in jeder Rechtsordnung vollständig vom aktuellen Markt abhängt. Über
Nacht zunehmende Erpressungsversuche, sei es durch Regierungen oder
andere Organisationen, könnten dazu führen, dass die Aktivitäten und
Vermögenswerte des virtuellen Unternehmens mit Lichtgeschwindigkeit aus
dem Rechtsgebiet fliehen.

Die zunehmende Integration von Mikrotechnologie in industrielle Prozesse
bedeutet, dass selbst solche Firmen, die noch mit groß angelegten,
hergestellten Produkten handeln, nicht mehr so anfällig für Gewalt sind,
wie sie es einmal waren. Ein Beispiel, das diesen Punkt veranschaulicht,
ist der Zusammenbruch des langwierigen Streiks der eingeschränkten
Autoarbeitergewerkschaft gegen Caterpillar, der in den letzten Tagen des
Jahres 1995 nach fast zwei Jahren abgebrochen wurde. Im Gegensatz zu den
Montagelinien der 1930er Jahre beschäftigt das heutige Caterpillar-Werk
weit mehr qualifizierte Arbeiter. Durch ausländischen Wettbewerb unter
Druck gesetzt, lagerte Caterpillar einen Großteil seiner gering
qualifizierten Arbeit aus, schloss ineffiziente Anlagen und investierte
fast 2 Milliarden Dollar in die Automatisierung von Maschinenwerkzeugen
durch Computer und die Installation von Montagerobotern. Sogar der
Streik selbst trug dazu bei, arbeitssparende Effizienzsteigerungen
anzuspornen. Das Unternehmen behauptet nun, 2000 Mitarbeiter weniger zu
benötigen als zu Beginn des Streiks.\footnote{George E Will,
  \emph{Farewell to Welfare States}, Washington Post, 17. Dezember 1995,
  S. C7.}

Die Megapolitik des Produktionsprozesses hat sich drastischer verändert,
als es die meisten Menschen realisieren. Diese Veränderung ist noch
nicht klar ersichtlich, teilweise weil es immer eine Verzögerung
zwischen einer Revolution in den megapolitischen Bedingungen und den
unvermeidbaren, darauffolgenden institutionellen Veränderungen gibt.
Darüber hinaus bedeutet die rasche Evolution der
Mikroprozessortechnologie, dass jetzt Produkte am Horizont auftauchen,
deren megapolitische Konsequenzen sogar vor ihrer Existenz antizipiert
werden können. Sie werden eine völlig neue Welt schaffen.

\section{AUSBEUTUNG DER KAPITALISTEN DURCH DIE
ARBEITER}\label{ausbeutung-der-kapitalisten-durch-die-arbeiter}

Der Charakter der Technologie im größten Teil des zwanzigsten
Jahrhunderts machte die gewaltsame Besetzung einer Fabrik, oder einen
Sitzstreik, zu einer herausfordernden Taktik, die Eigentümer oder
Manager entkräften mussten. Wie Historiker Robert S. McElvaine es
ausdrückte, machte es ein Sitzstreik „schwierig für Arbeitgeber, den
Streik zu brechen, ohne das Gleiche mit ihrer eigenen Ausrüstung zu
tun.`` \footnote{Robert S. MeElvaine, \emph{The Great Depression:
  America, 1929-1941} (New York: Times Books, 1984), S. 292.} In der Tat
hielten die Arbeiter das Kapital der Eigentümer physisch als Geisel. Aus
den unten erörterten Gründen waren größere Industrieunternehmen
leichtere Ziele für Gewerkschaften, als kleinere Firmen. Im Jahr 1937
war General Motors vielleicht das führende Industrieunternehmen der
Welt. Seine Fabriken waren unter den größten und kostspieligsten
Ansammlungen von Maschinen, die je zusammengestellt wurden, mit vielen
Tausenden von Arbeitern. Jede Stunde, jeder Tag, den die GM-Werke
gezwungen waren, untätig zu sitzen, kostete das Unternehmen ein kleines
Vermögen. Ein Streik, der über Wochen hinweg ungelöst blieb, wie im
Winter 1936-37, bedeutete rasch anschwellende Verluste.

\subsection{Die Herausforderung von Angebot und
Nachfrage}\label{die-herausforderung-von-angebot-und-nachfrage}

Als GM nach der Beschlagnahmung seiner dritten Fabrik nicht mehr in der
Lage war, Autos zu produzieren, kapitulierte das Unternehmen bald vor
der Gewerkschaft. Dies war mitnichten eine wirtschaftliche Entscheidung,
die auf dem Angebot und der Nachfrage nach Arbeitskräften basierte. Weit
gefehlt. Als General Motors den Forderungen der Gewerkschaft nachkam,
waren neun Millionen Menschen in den Vereinigten Staaten arbeitslos, 14
Prozent der Arbeitskräfte. Die meisten der Arbeitssuchenden hätten gerne
einen Arbeitsplatz bei GM angenommen. Sie hatten sicherlich die
Fähigkeiten, um die Arbeitsplätze in der Montagelinie zu besetzen,
obwohl man dies aus den meisten zeitgenössischen Berichten nicht
herauslesen kann. Ein feines Benehmen überschattete die direkte Analyse
der Arbeitsbeziehungen während der industriellen Periode. Eine ihrer
Vortäuschungen war die Idee, dass Fabrikarbeitsplätze, insbesondere in
der Mitte des zwanzigsten Jahrhunderts, qualifizierte Arbeitsplätze
waren. Das war nicht wahr. Die meisten Fabrikarbeitsplätze hätten von
fast jedem erledigt werden können, der in der Lage war, pünktlich zu
erscheinen. Sie erforderten wenig oder gar keine Ausbildung, nicht
einmal die Fähigkeit zu lesen oder zu schreiben. Noch in den 1980er
Jahren konnten große Teile der Belegschaft von General Motors entweder
nicht lesen, nicht rechnen oder beides. Bis in die 1990er Jahre erhielt
der typische Montagelinienarbeiter bei GM nur einen Tag Einarbeitung,
bevor er seinen Platz in der Montagelinie einnahm. Ein Job, den man an
einem einzigen Tag erlernen kann, ist keine qualifizierte Arbeit.

Und doch konnten die Fabrikarbeiter von GM im Jahr 1937, unqualifizierte
und qualifizierte Arbeiter gleichermaßen, ihren Arbeitgebern eine
Gehaltserhöhung abringen. Ihr Erfolg hatte viel mehr mit den Dynamiken
der Gewalt als mit dem Angebot und der Nachfrage nach Arbeitskräften zu
tun. Im März 1937, dem Monat nach der Beilegung der GM-Konfrontation,
gab es in den Vereinigten Staaten 17 weitere Sitzstreiks. Die meisten
waren erfolgreich. Ähnliche Episoden ereigneten sich in jedem
industrialisierten Land. Die Arbeiter besetzten einfach die Fabriken und
erpressten die Eigentümer. Es war eine Taktik von großer Einfachheit,
und in den meisten Fällen war sie für die Teilnehmer profitabel und
unterhaltsam. Ein Sitzstreiker schrieb: „Ich habe eine tolle Zeit, etwas
Neues, etwas anderes, jede Menge Essen und Musik.`` \footnote{Ebenda, S.
  293.}

Der Sitzstreik bei General Motors von 1936-37 und die anderen
gewaltsamen Fabrikbesetzungen dieser Zeit waren Beispiele für ein
Phänomen, das wir in \emph{Blood in the Streets} als „die Ausbeutung der
Kapitalisten durch die Arbeiter`` bezeichneten. Dies war nicht die
Sichtweise, die Pete Seeger in seinen traurigen Liedern zum Ausdruck
brachte. Aber solange man keine Karriere als Volksliedsänger in einem
Arbeiterquartier plant, ist das Wichtige nicht die allseits beliebte
Interpretation, sondern die zugrunde liegende Realität. Wo immer man in
der Geschichte hinschaut, gibt es im Allgemeinen eine Schicht der
Rationalisierung und des Wunschdenkens, die die wahren megapolitischen
Grundlagen einer systematischen Erpressung verschleiert. Wenn man die
Rationalisierungen für bare Münze nimmt, ist es unwahrscheinlich, dass
man verstehen wird, was wirklich vor sich geht.

\section{DIE ENTSCHLÜSSELUNG DER LOGIK DER
ERPRESSUNG}\label{die-entschluxfcsselung-der-logik-der-erpressung}

Um die megapolitischen Implikationen der aktuellen Verschiebung hin zum
Informationszeitalter zu erkennen, muss man den Schein beiseiteschieben
und sich auf die eigentliche Logik der Gewalt in der Gesellschaft
konzentrieren. Dies ist vergleichbar mit dem Schälen einer überreifen
Zwiebel. Es könnte Ihnen Tränen in die Augen treiben, aber schauen Sie
nicht weg! Wir untersuchen zuerst die Logik der Erpressung am
Arbeitsplatz und erweitern dann die Analyse auf breitere Themen, die die
Erstellung und den Schutz von Vermögenswerten sowie die Natur der
modernen Regierung betreffen. In größerem Maße als sich die meisten
Menschen vorstellen können, war der Wohlstand der Regierung, wie auch
der Gewerkschaften, direkt mit der verfügbaren Hebelwirkung für
Erpressung verbunden. Diese Hebelwirkung war im neunzehnten Jahrhundert
viel geringer als im zwanzigsten. Im nächsten Jahrtausend wird sie
nahezu auf den Nullpunkt fallen.

Die gesamte Logik von Regierungen und die Art der Macht wurden durch die
Mikroverarbeitung transformiert. Dies mag auf den ersten Blick
übertrieben erscheinen. Aber schauen Sie genau hin. Der Erfolg der
Regierungen ging im zwanzigsten Jahrhundert Hand in Hand mit dem Erfolg
der Gewerkschaften. Vor diesem Jahrhundert beanspruchten die meisten
Regierungen weit weniger Ressourcen als die militanten
Wohlfahrtsstaaten, an die wir uns gewöhnt haben. Ebenso waren
Gewerkschaften vor diesem Jahrhundert kleine oder unwichtige Faktoren im
Wirtschaftsleben. Die Fähigkeit der Arbeiter, ihre Arbeitgeber zur
Zahlung von über dem Marktwert liegenden Löhnen zu zwingen, beruhte auf
denselben megapolitischen Bedingungen, die es den Regierungen erlaubten,
40 Prozent oder mehr der Wirtschaftsleistung in Form von Steuern zu
erheben.

\subsection{Arbeitsplatz-Erpressung vor dem zwanzigsten
Jahrhundert}\label{arbeitsplatz-erpressung-vor-dem-zwanzigsten-jahrhundert}

Der Aufstieg und Fall der Gewerkschaftserpressung der Kapitalisten lässt
sich gut durch die sich verändernde Megapolitik des Produktionsprozesses
erklären. Im Jahr 1776, als Adam Smith \emph{Der Wohlstand der Nationen}
veröffentlichte, waren die Bedingungen für Erpressung am Arbeitsplatz so
ungünstig, dass „Zusammenschlüsse`` von Arbeitern, „um den Preis ihrer
Arbeit zu erhöhen``, selten haltbar waren. Die meisten Produktionsfirmen
waren winzig und familiengeführt. Industrielle Aktivitäten im größeren
Maßstab begannen gerade erst zu entstehen. Dies schloss Möglichkeiten
für Gewalt nicht aus, gab ihnen jedoch wenig Hebelwirkung. Tatsächlich
wurden Gewerkschaften während der Zeit Smiths und weit ins 19.
Jahrhundert hinein in Großbritannien, den USA und anderen Ländern nach
dem Common Law allgemein als illegale Zusammenschlüsse angesehen. Adam
Smith beschrieb versuchte Streiks mit diesen Worten: „Ihre üblichen
Vorwände sind manchmal die hohen Preise für Lebensmittel, manchmal der
große Gewinn, den ihre Meister mit ihrer Arbeit machen.\ldots{} Sie
greifen immer zum lautesten Geschrei und manchmal auch zu schockierender
Gewalt und Empörung.`` \footnote{Smith, ebenda, S. 75.} Dennoch ziehen
die Arbeiter „sehr selten einen Vorteil aus diesen turbulenten
Zusammenschlüssen``, abgesehen von „der Bestrafung oder dem Ruin des
Anführers.`` \footnote{Ebenda, S.76.}

Im neunzehnten Jahrhundert wuchsen die Größenvorteile in der Industrie
und die Unternehmensgröße. Dennoch arbeiteten die meisten Menschen
weiterhin als Bauern oder Kleinunternehmer für sich selbst, und
Bemühungen von Gewerkschaftsorganisationen, wie sie von Adam Smith
beschrieben wurden, endeten „im Allgemeinen ohne Ergebnis``.\footnote{Ebenda.}
Die rechtliche und politische Stellung von Gewerkschaften änderte sich
erst, als die Größe der Unternehmen zunahm. Die ersten Gewerkschaften,
die sich erfolgreich organisierten, waren Handwerksgewerkschaften von
hochqualifizierten Arbeitern, die sich normalerweise ohne umfangreiche
Gewalt organisierten. Sie neigten dazu, Lohnerhöhungen zu vereinbaren,
die den Grenzkosten ihrer Ersetzung entsprachen. Bei Gewerkschaften für
ungelernte Arbeitskräfte sah das anders aus. Sie neigten dazu, den
Wechsel zu größeren Unternehmen auszunutzen, indem sie gerade diejenigen
Branchen für die Organisationsbemühungen auswählten, die besonders
anfällig für Zwang waren, entweder weil sie in größerem Maßstab
arbeiteten oder weil die Art der Betriebe ihre Eigentümer der physischen
Sabotage aussetzte. Dieses Muster wurde von Newcastle bis Argentinien
bestätigt.\footnote{Eine der ersten argentinischen Gewerkschaften, die
  sich organisierte, war die Eisenbahngewerkschaft im Jahr 1887. Siehe
  Carmelo Mesa-Lago, \emph{Social Security in Latin America: Pressure
  Groups, Stratification, and Inequality} (Pittsburgh: University of
  Pittsburgh Press, 1978), S. 161.}

Ein frühes Beispiel für gewalttätige Arbeiterbewegungen in den
Vereinigten Staaten war ein Angriff auf den Chesapeake und Ohio Kanal im
Jahr 1834. Im Gegensatz zu den meisten Unternehmen im frühen 19.
Jahrhundert war der C\&O Kanal kein abgegrenzter und leicht zu
schützender Betrieb. Ursprünglich sollte er sich über 550 Kilometer
erstrecken, mit einem Höhenunterschied von 900 Metern vom unteren
Potomac bis zum Norden Ohios.\footnote{Zu Einzelheiten über die Planung
  und den Bau des C\&O-Kanals, siehe Robert J. Brugger, Maryland: A
  Middle Temperament 1634-1980 (Baltimore: The Johns Hopkins Press,
  1990), S. 202-203 f.} Ein so großes Bauvorhaben war eine schwierige
Aufgabe, die nie ganz abgeschlossen wurde. Dennoch waren viele Arbeiter
damit beschäftigt, es zu versuchen. Einige von ihnen erkannten bald,
dass der Kanal leicht lahmgelegt werden konnte. Tatsächlich hätte der
Kanal ohne regelmäßige Wartung durch Bisamratten, die unter dem
Treidelpfad gruben, sabotiert werden können. Während des Betriebs
konnten die Schleusen und Rinnen des Kanals leicht durch unsachgemäße
Benutzung, Überschwemmungen durch heftige Regenfälle oder durch
unbeaufsichtigte Boote beschädigt werden. Es war einfach für Streikende,
den Wasserweg mit versunkenen Booten oder anderen Trümmern zu
blockieren. Anfang 1834 führte ein Aufstand unter konkurrierenden Banden
irischer Arbeiter am C\&O zu einem Versuch, diesen potenziellen
Sabotageakt durchzuführen und den Kanal zu übernehmen. Der Versuch
scheiterte jedoch und hinterließ fünf Tote, nachdem Präsident Andrew
Jackson Bundestruppen aus Ft. McHenry entsandt hatte, um die Arbeiter zu
vertreiben.

Bergwerke und Eisenbahnen boten ebenfalls frühe Ziele für
gewerkschaftliche Aktivitäten in Amerika. Ähnlich wie der
Chesapeake-und-Ohio-Kanal waren auch sie äußerst anfällig für Sabotage.
Bergwerke konnten zum Beispiel überflutet oder am Eingang blockiert
werden. Allein das Töten der Maultiere, die die Erzwägen aus den
unterirdischen Minen zogen, schuf eine schwierige und unangenehme
Situation für die Besitzer. Ähnlich erstreckten sich
Eisenbahngleisbetten über viele Meilen und konnten nur mit
Schwierigkeiten bewacht werden. Es war für Gewerkschaftsschläger relativ
einfach, Minen und Eisenbahnen anzugreifen und erheblichen
wirtschaftlichen Schaden anzurichten. Solche Angriffe waren während der
Versuche, effektive Gewerkschaften zu organisieren, an der Tagesordnung.
Diese Bemühungen waren allgemein am intensivsten während Perioden, in
denen die Reallohnentwicklung aufgrund von Deflation stieg. Wenn
Eigentümer versuchten, Nominallöhne anzupassen, löste dies oft Proteste
aus, die zu Gewalt führten. Solche Vorfälle waren in der Depression, die
auf den Panikzustand von 1873 folgte, weit verbreitet.

Im Dezember 1874 brach offener Krieg in den Anthrazitkohlefeldern von
Ost-Pennsylvania aus. Die Gewerkschaften organisierten eine gewalttätige
Streitmacht unter dem Deckmantel einer Geheimgesellschaft namens
„Ancient Order of Hibernians``. Auch bekannt als die „Molly Maguires``,
nach einer irischen Revolutionärin, war diese Gruppe dafür bekannt, die
Kohlefelder zu terrorisieren und jene Bergleute zu behindern, die
arbeiten wollten. Ihre Mitglieder wurden wegen Sabotage und Zerstörung
von Eigentum, unverblümtem Mord und Attentaten verurteilt.\footnote{Irving
  J. Sloan, \emph{Our Violent Past: An American Chronicle} (New York:
  Random House, 1970), S. 177.}

Es gab auch wiederkehrende Gewalt unter den Eisenbahnangestellten. Zum
Beispiel kam es im Juli 1877 zu schwerwiegenden Aufständen mit dem Ziel,
das Eigentum sowohl der Pennsylvania als auch der Baltimore \& Ohio
Eisenbahnen zu zerstören. Die Arbeiter übernahmen die Kontrolle über
Weichen, rissen Gleise heraus, versiegelten Güterbahnhöfe, machten
Lokomotiven unbrauchbar, sabotierten und plünderten Züge und
Schlimmeres. In Pittsburgh wurden Rundhäuser der Pennsylvania Eisenbahn
in Brand gesetzt, während sich Hunderte von Menschen darin befanden.
Dutzende wurden getötet, zweitausend Eisenbahnwaggons wurden verbrannt
und geplündert und die Werkstatt wurde zerstört, zusammen mit einem
Getreidesilo und 125 Lokomotiven. Bundestruppen griffen ein, um die
Ordnung wiederherzustellen.\footnote{Zu Einzelheiten über die Gewalt bei
  den Eisenbahnstreiks von 1877, siehe ebenda, und Brugger, ebenda, S.
  341-344.}

Obwohl diese frühen Streiks von Sozialisten und Gewerkschaftsaktivisten
wohlwollend betrachtet wurden, konnten sie wenig öffentliche
Unterstützung finden. Trotz der inhärenten Verwundbarkeit von Industrien
wie Bergwerken und Eisenbahnen waren die allgemeinen megapolitischen
Bedingungen noch nicht günstig genug für die Ausbeutung der Kapitalisten
durch die Arbeiter. Der Umfang der Unternehmen war zu gering, um
systematische Erpressung zu ermöglichen. Obwohl es gefährdete Branchen
gab, stellten sie einen zu geringen Anteil der Bevölkerung ein, als dass
die Vorteile der Zwangsmaßnahmen gegen Arbeitgeber breit geteilt werden
konnten. Ohne eine solche Unterstützung waren sie untragbar, da die
Eigentümer sich auf den Schutz der Regierung verlassen konnten. Während
Gewerkschaften manchmal versuchten, lokale Beamte durch Einschüchterung
daran zu hindern, Unterlassungsanordnungen durchzusetzen, waren auch
diese Bemühungen selten erfolgreich. Selbst die gewalttätigsten Streiks
wurden in der Regel innerhalb von Tagen oder Wochen durch militärische
Mittel unterdrückt.

\subsection{Erpressung leicht gemacht}\label{erpressung-leicht-gemacht}

Eine Lehre für das Informationszeitalter ist die Tatsache, dass die
Versuche der Gewerkschaften, Löhne oberhalb des marktüblichen Niveaus zu
erreichen, selten erfolgreich waren, wenn die Unternehmen klein waren.
Selbst Geschäftsbereiche, die eindeutig anfällig für Sabotage waren, wie
Kanäle, Eisenbahnen, Straßenbahnen und Bergwerke, konnten nicht leicht
kontrolliert werden. Das liegt nicht daran, dass die Gewerkschaften vor
Gewalt zurückgeschreckt hätten. Ganz im Gegenteil. Gewalt wurde
reichlich eingesetzt, manchmal gegen prominente Einzelpersonen. Zum
Beispiel wurde in einem Fall, der in der amerikanischen
Gewerkschaftsbewegung als „Rache der Bergarbeiter`` gefeiert wurde,
Gouverneur Frank Steunenberg aus Idaho, der einen von Bergarbeitern
initiierten Versuch, die Grundstücke von Coeur d'Alene zu blockieren,
abgelehnt hatte, durch einen von der Gewerkschaft angeheuerten
Auftragsmörder getötet.\footnote{Sloan, ebenda, S. 202. Siehe außerdem
  S. S. Boynton, \emph{Miners\textquotesingle{} Vengeance,} Overland
  Monthly, vol.22 (1893), S. 303-307.} Doch sogar Mord und Morddrohungen
reichten meist nicht aus, um eine Anerkennung der Gewerkschaften zu
erzielen, bevor im zwanzigsten Jahrhundert großangelegte Fabriken und
Massenproduktionsunternehmen auftauchten.

Um zu verstehen, warum sich die Umstände der Gewerkschaften im
zwanzigsten Jahrhundert so veränderten, muss man die Eigenschaften der
Produktionstechnologie betrachten. Mit dem schnellen Anstieg der
Beschäftigung in Fabriken für Arbeiter mit geringer Ausbildung in den
frühen Jahrzehnten des zwanzigsten Jahrhunderts, veränderte sich
definitiv etwas. Diese Veränderung machte Unternehmen, die an der Spitze
der Wirtschaft standen, besonders anfällig für Erpressung. Tatsächlich
luden die physischen Eigenschaften der industriellen Technologie
Arbeitnehmer fast dazu ein, Zwang einzusetzen, um die Kapitalisten unter
Druck zu setzen. Betrachten Sie Folgendes:

\begin{enumerate}
\def\labelenumi{\arabic{enumi}.}
\tightlist
\item
  \emph{Die meisten Industrieprodukte hatten einen hohen Anteil an
  natürlichen Ressourcen.} Dies führte dazu, die Produktion an eine
  begrenzte Anzahl von Standorten zu binden, fast so, wie Minen dort
  angesiedelt sein müssen, wo die Erzrohstoffe liegen. Fabriken, die in
  der Nähe von Verkehrszentren mit bequemem Zugang zu Zulieferern und
  Rohstoffen platziert wurden, hatten bedeutende Betriebsvorteile. Dies
  erleichterte es Zwangsorganisationen wie Regierungen und
  Gewerkschaften, einige dieser Vorteile für sich zu nutzen.
\item
  \emph{Steigende Skaleneffekte führten zu sehr großen Unternehmen.}
  Fabriken zu Beginn des neunzehnten Jahrhunderts waren relativ klein.
  Aber als die Skaleneffekte mit der Einführung der Fließbandproduktion
  im zwanzigsten Jahrhundert zunahmen, stiegen Größe und Kosten der
  Anlagen, die im Vordergrund des Produktionsprozesses standen, rasch
  an. Dies machte sie in vielerlei Hinsicht zu einfacheren Zielen. Zum
  Beispiel gehen signifikante Skaleneffekte oft Hand in Hand mit langen
  Produktzyklen. Lange Produktzyklen sorgen für stabilere Märkte. Dies
  wiederum lädt dazu ein, Unternehmen räuberisch ins Visier zu nehmen,
  da es impliziert, dass es langfristige Vorteile zu erlangen gibt.
\item
  \emph{Die Anzahl der Wettbewerber in führenden Branchen fiel stark.}
  Es war nicht ungewöhnlich in der industriellen Ära, dass nur eine
  Handvoll Firmen um Milliarden-Dollar-Märkte konkurrierten. Dies trug
  dazu bei, diese Firmen zum Ziel für Gewerkschaftserpressung zu machen.
  Es ist viel einfacher, fünf Firmen anzugreifen als fünftausend. Die
  hohe Konzentration der Industrie war selbst ein Faktor, der Erpressung
  begünstigte. Dieser Vorteil verstärkte sich selbst, weil die Firmen,
  die zu Monopollohnzahlungen gezwungen wurden, wahrscheinlich nicht mit
  harter Konkurrenz von anderen Firmen konfrontiert wurden, die nicht
  ebenfalls mit über dem Marktdurchschnitt liegenden Arbeitskosten
  belastet waren. Gewerkschaften konnten daher einen erheblichen Teil
  der Gewinne solcher Unternehmen absorbieren, ohne sie sofort in den
  Bankrott zu treiben. Offensichtlich hätten Arbeitnehmer, wenn
  Arbeitgeber routinemäßig pleite gegangen wären, sobald sie gezwungen
  wurden, über dem Marktdurchschnitt liegende Löhne zu zahlen, wenig
  davon gehabt, sie dazu zu zwingen.
\item
  \emph{Die Kapitalanforderungen für freie Investitionen stiegen in
  Übereinstimmung mit der Unternehmensgröße.} Dies erhöhte nicht nur die
  Verletzlichkeit des Kapitals und vergrößerte die Kosten von
  Werkschließungen; es machte es auch immer unwahrscheinlicher, dass ein
  modernes Werk von einer einzelnen Person oder Familie besessen werden
  konnte, außer durch Erbschaft von jemandem, der das Geschäft in
  kleinerem Maßstab gestartet hatte. Um die enormen Kosten für die
  Ausrüstung und den Betrieb einer großen Fabrik zu finanzieren, musste
  das Vermögen von Hunderten oder Tausenden von Menschen auf den
  Kapitalmärkten gebündelt werden. Dies erschwerte es den zerstreuten
  und fast anonymen Eigentümern zunehmend, ihr Eigentum zu verteidigen.
  Sie hatten kaum eine andere Wahl, als sich auf professionelle Manager
  zu verlassen, die selten mehr als eine kaum messbare Menge der
  ausstehenden Aktien des Unternehmens hielten. Die Abhängigkeit von
  untergeordneten Managern schwächte den Widerstand der Firmen gegen
  Erpressung. Den Managern fehlten starke Anreize, Leib und Leben zu
  riskieren, um das Eigentum der Firma zu schützen. Ihre Bemühungen
  entsprachen selten der Art von Militanz, die häufig bei Eigentümern
  von Spirituosenläden und anderen kleinen Geschäften zu beobachten ist,
  wenn ihr Eigentum bedroht wird.
\item
  \emph{Die Zunahme der Firmengröße bedeutete auch, dass mehr Menschen
  insgesamt in weniger Unternehmen beschäftigt waren als je zuvor.} In
  einigen Fällen fanden zehntausende von Arbeitnehmern Arbeit in einem
  einzigen Unternehmen. In militärischer Hinsicht waren die
  Firmeninhaber und -leiter stark in der Unterzahl im Vergleich zu den
  Beschäftigten in untergeordneten Positionen. Verhältnisse von dreißig
  zu eins oder weniger waren üblich. Dieser Nachteil stieg mit der
  Unternehmensgröße, da große Mengen von Arbeitnehmern, die
  zusammenkamen, Gewalt auf anonyme Weise leichter einsetzen konnten.
  Unter solchen Bedingungen ist es unwahrscheinlich, dass die Arbeiter
  irgendeinen bedeutsamen Kontakt oder Beziehungen zu den Eigentümern
  der Fabrik gehabt hätten. Der anonyme Charakter dieser Beziehungen
  erleichterte es den Arbeitern zweifellos, die Bedeutung der
  Eigentumsrechte der Eigentümer zu ignorieren.
\item
  \emph{Die Beschäftigung einer großen Anzahl von Menschen in einer
  kleinen Anzahl von Unternehmen war ein weit verbreitetes soziales
  Phänomen.} Dies verstärkte die Megapolitikvorteile der Gewerkschaften
  im Vergleich zum 19. Jahrhundert in Amerika, als die meisten Menschen
  selbstständig oder in kleinen Firmen arbeiteten. 1940 arbeiteten 6
  Prozent der amerikanischen Arbeiterschaft als Angestellte.\footnote{Benjamin
    Schwartz, \emph{American Inequality: Its History and Scary Future,}
    New York Times, 19. Dezember 1995, S. A25.} Als Folge davon breitete
  sich die Unterstützung für die Erpressung zur Lohnerhöhung unter einer
  großen Anzahl von Menschen aus, die glaubten, dass sie davon
  profitieren könnten. Dies wurde durch eine Studie von 1938-39 mit
  1.700 Menschen in Akron, Ohio, zum Firmeneigentum illustriert. Die
  Umfrage ergab, dass 68 Prozent der CIO-Gummiarbeiter sehr wenig oder
  gar kein Verständnis für das Konzept des Firmeneigentums hatten,
  während nur ein Prozent in die Kategorie „starke Befürworter von
  Firmeneigentumsrechten`` eingeordnet wurden.\footnote{MeElvaine,
    ebenda, S. 293.} Andererseits fiel kein einziger Geschäftsmann,
  nicht einmal ein kleiner Eigentümer, in die gleiche Kategorie der
  „starken Opposition gegen das Firmeneigentum; 94 Prozent erhielten
  Bewertungen im Bereich der extrem hohen Unterstützung für die
  Eigentumsrechte``.\footnote{Ebenda.}
\item
  \emph{Die Fließbandtechnologie war von Natur aus sequenziell.} Die
  Tatsache, dass der gesamte Produktionsprozess von der Bewegung und
  Montage von Teilen in einer festen Reihenfolge abhängig war, schuf
  zusätzliche Anfälligkeiten für Störungen. Tatsächlich war das
  Fließband wie eine Eisenbahn innerhalb der Fabrikmauern. Wenn die
  Strecke blockiert werden oder die Verfügbarkeit eines einzelnen Teils
  unterbrochen werden konnte, wurde der gesamte Produktionsprozess zum
  Stillstand gebracht.
\item
  \emph{Die Fließbandtechnologie standardisierte die Arbeit.} Dies
  reduzierte die Schwankungen im Output für Personen mit
  unterschiedlichen Fähigkeiten, die mit denselben Werkzeugen
  arbeiteten. Ein wesentliches Ziel der Fabrikplanung bestand darin, ein
  System zu schaffen, in dem ein Genie und ein Idiot in
  aufeinanderfolgenden Schichten am Fließband dasselbe Produkt
  herstellen würden. Was man als „dumme`` Maschinen bezeichnen könnte,
  wurde so konzipiert, dass sie nur eine Art von Output liefern konnte.
  Dies machte es selbst für den Käufer eines Cadillacs überflüssig, sich
  nach der Identität der Fließbandarbeiter zu erkundigen, die sein
  Fahrzeug herstellten. Alle Produkte sollten gleich sein, unabhängig
  von den Unterschieden in Fähigkeiten und Intelligenz zwischen den
  Arbeitern, die sie hergestellt hatten.
\end{enumerate}

Die Tatsache, dass ungelernte Arbeiter an der Montagelinie dasselbe
Produkt herstellen konnten wie qualifiziertere Personen, trug zur
egalitären Agenda bei, indem sie den Eindruck erweckte, dass die
wirtschaftlichen Beiträge von allen gleich seien. Unternehmerisches
Können und geistige Anstrengung schienen weniger wichtig. Der Zauber der
modernen Produktion schien in den Maschinen selbst zu liegen. Auch wenn
sie tatsächlich nicht von jedem entworfen werden können, schienen sie
dennoch für fast jeden intellektuell zugänglich zu sein. Dies verlieh
der Fiktion mehr Plausibilität, dass ungelernte Arbeit von
Fabrikbesitzern „ausgebeutet`` werde, die man ohne Verlust für
irgendeine andere Partei, abgesehen von ihnen selbst, aus der Gleichung
entfernen könnte. „Wir haben gelernt, dass wir die Fabrik übernehmen
können``, wie es ein Streikender bei GM formulierte. „Wir wussten schon,
wie man sie betreibt. Wenn General Motors nicht aufpasst, werden wir
eins und eins zusammenzählen.`` \footnote{Ebenda.}

Die Merkmale der industriellen Technologie führten einheitlich zur
Gründung von Gewerkschaften, um die Anfälligkeit für Erpressungen
auszunutzen, sowie zu größeren Regierungen, die sich an den hohen
Steuern bereicherten, die große Industrieanlagen zahlen mussten. Dies
passierte nicht nur ein oder zwei Mal, sondern überall dort, wo
groß-industrielle Produktion Einzug hielt. Immer wieder entstanden
Gewerkschaften, die Gewalt anwendeten, um deutlich über das Marktniveau
hinausgehende Löhne zu erzielen. Sie waren dazu in der Lage, weil
industrielle Fabriken in der Regel teuer, aufwändig, unbeweglich und
kostspielig waren. Sie konnten kaum versteckt werden. Sie konnten nicht
verlegt werden. Jeder Moment, in dem sie nicht in Betrieb waren,
bedeutete, dass ihre enormen Kosten nicht amortisiert wurden.

All dies machte sie zu leichten Zielen für erzwungene Abzockereien, eine
Tatsache, die in der Geschichte der Gewerkschaften viel offensichtlicher
ist, als die vorherrschende Ideologie des zwanzigsten Jahrhunderts es
glauben lässt. Der bekannte Ökonom Henry Simons formulierte das Problem
im Jahr 1944:

„Arbeitnehmerorganisationen ohne große Macht zur Zwangsausübung und
Einschüchterung sind eine unrealistische Abstraktion. Gewerkschaften
haben jetzt solche Macht; sie haben sie immer gehabt und werden sie
immer haben, solange sie in ihrer derzeitigen Form bestehen. Wenn die
Macht klein oder unsicher ist, muss sie offen und umfassend ausgeübt
werden; wenn sie groß und unangefochten ist, wird sie wie die Macht der
Regierung, selbstbewusst gehalten, respektvoll angesehen und selten
auffällig gezeigt.`` \footnote{Henry C. Simons, \emph{Some Reflections
  on Syndicalism}, Journal of Political Economy März 1944, S. 22.}

So präzise Simons Analyse auch war, er lag dennoch bei einem
entscheidenden Punkt falsch.

Er nahm an, dass Gewerkschaften „immer über das verfügen werden``, was
er als „große Macht zu Zwang und Einschüchterung`` beschrieb.
Tatsächlich schwinden die Gewerkschaften jedoch nicht nur in den USA und
Großbritannien, sondern auch in anderen ausgereiften
Industriegesellschaften. Der Grund für ihr Verschwinden, den Simons
übersehen hat und den selbst viele Gewerkschaftsorganisatoren nicht
verstehen, liegt darin, dass der Übergang zu einer
Informationsgesellschaft die megapolitischen Bedingungen auf
entscheidende Weise verändert hat und die Sicherheit von Eigentum stark
erhöht hat. Die Mikrotechnologie hat bereits begonnen, die Erpressung,
die den Wohlfahrtsstaat unterstützt, zu untergraben, da sie selbst im
kommerziellen Bereich völlig andere Anreize schafft als die industrielle
Periode.

\begin{enumerate}
\def\labelenumi{\arabic{enumi}.}
\item
  \emph{Die Informationstechnologie hat einen vernachlässigbaren Anteil
  an natürlichen Ressourcen.} Sie vermittelt wenige oder gar keine
  inhärenten Standortvorteile. Die meisten Informationstechnologien sind
  hochgradig mobil. Da sie unabhängig vom Ort funktionieren können,
  erhöht die Informationstechnologie die Mobilität von Ideen, Personen
  und Kapital. General Motors kann seine drei Montagelinien in Flint,
  Michigan, nicht einfach einpacken und wegfliegen. Ein
  Softwareunternehmen kann das. Die Eigentümer können ihre Algorithmen
  auf tragbare Computer herunterladen und mit dem nächsten Flugzeug
  abhauen. Solche Unternehmen haben auch einen zusätzlichen Anreiz, hohe
  Steuern oder Gewerkschaftsforderungen nach Monopollöhnen zu entkommen.
  Kleinere Firmen neigen dazu, mehr Wettbewerber zu haben. Wenn Sie
  Dutzende oder sogar Hunderte von Wettbewerbern haben, die Ihre Kunden
  locken, können Sie es sich nicht leisten, Politikern oder Ihren
  Mitarbeitern viel mehr zu bezahlen, als sie tatsächlich wert sind.
  Wenn Sie alleine versuchen würden, dies zu tun, wären Ihre Kosten
  höher als die Ihrer Wettbewerber und Sie würden pleite gehen. Das
  Fehlen von signifikanten Betriebsvorteilen an einem bestimmten Ort
  bedeutet, dass Zwangsorganisationen, wie Regierungen und
  Gewerkschaften, unweigerlich weniger Hebelwirkung haben werden, um
  einige dieser Vorteile für sich selbst auszubeuten.
\item
  \emph{Die Informationstechnologie senkt die Unternehmensgröße.} Dies
  führt zu kleineren Firmen und damit zu einer größeren Anzahl von
  Konkurrenten. Der erhöhte Wettbewerb reduziert das Potenzial für
  Erpressung, indem die Anzahl der Ziele erhöht wird, die physisch
  kontrolliert werden müssen, um Löhne oder Steuersätze über
  Wettbewerbsniveaus anzuheben. Der drastische Rückgang der
  durchschnittlichen Unternehmensgröße, der durch die
  Informationstechnologie ermöglicht wurde, hat bereits die Anzahl der
  Personen reduziert, die in untergeordneten Positionen beschäftigt
  sind. In den USA beispielsweise lassen weit verbreitete Schätzungen
  vermuten, dass 1996 bis zu 30 Millionen Personen alleine in ihren
  eigenen Unternehmen gearbeitet haben. Offensichtlich neigen diese 30
  Millionen Menschen kaum dazu gegen sich selbst zu streiken. Es ist nur
  unwesentlich weniger plausibel, dass die zusätzlichen Millionen, die
  in kleinen Unternehmen mit einer Handvoll Angestellten arbeiten,
  versuchen würden, ihre Arbeitgeber zur Zahlung von über dem
  Marktdurchschnitt liegenden Löhnen zu erpressen. Im
  Informationszeitalter wird den Arbeitnehmern, die ihre Löhne durch
  Erpressung erhöhen wollen, der militärische Vorteil der
  überwältigenden Zahl fehlen, der sie in der Fabrik zu einer
  furchteinflößenden Macht gemacht hat. Je weniger Personen in einem
  Unternehmen beschäftigt sind, desto geringer sind die Möglichkeiten
  für anonyme Gewalt. Aus diesem einen Grund allein würden zehntausend
  Arbeiter, die auf fünfhundert Unternehmen verteilt sind, eine
  geringere Gefahr für das Eigentum dieser Unternehmen darstellen als
  zehntausend Arbeiter in einem einzigen Unternehmen, selbst wenn das
  Verhältnis von Arbeitnehmern zu Eigentümern/Managern genau gleich
  wäre.
\item
  \emph{Die sinkende Betriebsgröße bedeutet auch, dass Bemühungen um
  überdurchschnittliche Löhne weniger wahrscheinlich breite
  gesellschaftliche Unterstützung finden, wie dies in der Industriezeit
  der Fall war.} Gewerkschaften, die versuchen, Arbeitgeber abzuzocken,
  sind viel eher in der Situation der Kanalarbeiter,
  Eisenbahnangestellten und Bergarbeiter des neunzehnten Jahrhunderts.
  Selbst wenn einige Firmen mit großen Betriebsgrößen als Überbleibsel
  aus dem Industriezeitalter verbleiben, tun sie dies in einem Kontext
  von weit verbreiteter Beschäftigung in kleinen Firmen. Die Überzahl
  von kleinen Firmen und Kleinbetrieben lässt auf eine größere
  gesellschaftliche Unterstützung für Eigentumsrechte schließen, selbst
  wenn der Wunsch, Einkommen umzuverteilen, unverändert bleibt.
\item
  \emph{Die Informationstechnologie senkt die Kapitalkosten, was auch
  dazu neigt, den Wettbewerb zu erhöhen, indem sie Unternehmertum
  fördert und mehr Menschen die Möglichkeit gibt, unabhängig zu
  arbeiten.} Geringere Kapitalanforderungen reduzieren nicht nur die
  Barrieren für den Markteinstieg; sie reduzieren auch die
  „Ausstiegsbarrieren``. Mit anderen Worten, sie implizieren, dass
  Firmen wahrscheinlich weniger Vermögen im Vergleich zu ihrem Einkommen
  haben und daher weniger die Fähigkeit haben, Verluste zu verkraften.
  Sie werden nicht nur weniger Kredite bei den Banken aufnehmen müssen;
  Unternehmen im Informationszeitalter werden wahrscheinlich auch
  weniger Sachwerte besitzen, die sie erfassen müssen.
\item
  \emph{Die Informationstechnologie verkürzt den Produktlebenszyklus.}
  Dies führt zu einer schnelleren Produktobsoleszenz. Auch dies tendiert
  dazu, etwaige Gewinne, die durch das Erpressen von über dem
  Marktniveau liegenden Löhnen erzielt werden könnten, kurzlebig zu
  machen. In stark wettbewerbsorientierten Märkten können überhöhte
  Löhne direkt zu einem raschen Arbeitsplatzverlust und sogar zum
  Bankrott des Unternehmens führen. Nach vorübergehend höheren Löhnen zu
  greifen, auf Kosten der Sicherheit des eigenen Arbeitsplatzes, ist
  vergleichbar mit dem Verbrennen der eigenen Möbel, um das Haus um
  einige Grad wärmer zu machen.
\item
  \emph{Informationstechnologie ist nicht sequenziell, sondern simultan
  und verstreut.} Anders als die Fließbandarbeit kann
  Informationstechnologie mehrere Prozesse gleichzeitig verarbeiten. Sie
  verteilt Arbeitsvorgänge auf Netzwerken, ermöglicht Redundanz und
  Substitution zwischen Arbeitsstationen, die sich zu Tausenden oder
  sogar zu Millionen an jedem Ort der Erde befinden können. Bei einer
  zunehmenden Anzahl von Aktivitäten können Menschen zusammenarbeiten,
  ohne jemals physischen Kontakt miteinander zu haben. Mit
  fortschrittlicher virtueller Realität und Videokonferenzen wird die
  Tendenz zur Verbreitung von Funktionen und Telearbeit beschleunigen.
  Das entspricht dem Äquivalent des Informationszeitalters von
  „Outsourcing``, das die Macht der mittelalterlichen Zünfte brach. Die
  Tatsache, dass immer weniger Menschen zusammen in verrauchten Fabriken
  arbeiten, nimmt nicht nur einen wichtigen Vorteil weg, den Arbeiter
  bei der Erpressung der Kapitalisten genossen haben, sondern macht es
  auch immer schwieriger, die Art von Erpressung, die am Arbeitsplatz
  akzeptabel war, von Ausbeutung zu unterscheiden. Bisher durften nur
  Personen, die zusammengearbeitet und von einer Firma an einem
  gemeinsamen Ort angestellt wurden, Gewalt anwenden, um ihre Einkommen
  zu erhöhen. Wenn der „Arbeitsplatz`` nicht als zentrale Lage existiert
  und die meisten Funktionen an Subunternehmer und Telearbeiter
  ausgelagert sind, wird es kaum etwas geben, um ihre Bemühungen, Geld
  von ihren Kunden oder „Arbeitgebern`` zu erpressen, von einer
  systematischen Ausbeutung zu unterscheiden.

  Ist zum Beispiel ein Telearbeiter, der zusätzliches Geld verlangt
  unter der Drohung, die Computer des Unternehmens mit einem Virus zu
  infizieren, ein streikender Arbeitnehmer? Oder ist er ein
  Internet-Gangster?

  Ob er das eine oder andere ist, macht keinen Unterschied. Die Reaktion
  der betroffenen Unternehmen wird wahrscheinlich in jedem Fall ähnlich
  sein. Technische Lösungen gegen Informationssabotage, wie verbesserte
  Verschlüsselung und Netzwerksicherheit, die die Gefahr eines externen
  Hackers beantworten sollten, könnten auch die Möglichkeiten des
  unzufriedenen Mitarbeiters oder Subunternehmers, Schäden bei Parteien
  zu verursachen, mit denen er regelmäßig oder sporadisch handelt,
  unwesentlich machen. Natürlich könnte man vorschlagen, dass der
  Arbeiter oder Telearbeiter immer ins Büro zurückkehren und dort einen
  traditionelleren Streik durchführen könnte. Aber selbst das könnte im
  Informationszeitalter nicht so einfach sein, wie es scheinen mag. Die
  Möglichkeit der Informationstechnologie, räumliche Begrenzungen zu
  überwinden und wirtschaftliche Funktionen zu verteilen, bedeutet, dass
  zum ersten Mal Arbeitgeber und Arbeitnehmer nicht einmal im gleichen
  Rechtssystem ansässig sein müssen. Hier sprechen wir nicht nur über
  die Unterschiede zwischen den Stadtteilen Mayfair und Peckham, sondern
  auch über Arbeitgeber in Bermuda und Telearbeiter in Neu-Delhi.

  Wenn sich die Inder von den Berichten über die großen GM-Streiks von
  1936-37 anstecken ließen und beschließen, auf die Bermudas zu reisen,
  um Streikposten zu stellen, könnten sie bei ihrer Ankunft überhaupt
  kein Büro vorfinden. Chiat/Day, eine große Werbefirma, hat bereits
  begonnen, ihre Hauptverwaltung zu zerlegen. Ihre Mitarbeiter oder
  Subunternehmer bleiben in Kontakt durch Anrufweiterleitung und das
  Internet. Wenn es notwendig wird, Talentteams zu treffen, um die
  Arbeit an Projekten zu koordinieren, mieten sie Hoteltagungsräume.
  Wenn das Projekt vorbei ist, checken sie aus. Die Tatsache, dass die
  Mikroverarbeitung dazu beiträgt, den Produktionsprozess von der
  starren Abfolge des Fließbandes zu befreien und zu zerstreuen,
  verringert den Einfluss, den Zwangsinstitutionen wie Gewerkschaften
  und Regierungen früher hatten, erheblich. Wäre das Fließband wie eine
  Eisenbahn innerhalb der Fabrikmauern, die leicht durch einen
  Sitzstreik erobert werden könnte, so ist der Cyberspace ein
  unbegrenzter Bereich ohne physische Existenz. Er kann nicht mit Gewalt
  besetzt oder erpresst werden. Die Position der Arbeitnehmer, die
  Gewalt als Hebel nutzen möchten, um ein höheres Einkommen zu
  erpressen, wird im Informationszeitalter viel schwächer sein, als es
  bei den Sitzstreikern bei General Motors im Jahre 1936-37 der Fall
  war.
\item
  \emph{Mikroprozessoren individualisieren die Arbeit, Industrietechnik
  standardisiert die Arbeit.} Wer dieselben Werkzeuge benutzt,
  produziert dasselbe Ergebnis. Die Mikrotechnologie hat begonnen,
  „dumme`` Maschinen durch intelligentere Technologie zu ersetzen, die
  eine stark variable Leistung ermöglicht. Die erhöhte Variabilität der
  Ergebnisse für Personen, die dieselben Werkzeuge benutzen, hat
  tiefgreifende Auswirkungen, die wir in den kommenden Kapiteln
  untersuchen werden. Unter den wichtigeren Aspekten ist die Tatsache,
  dass bei variablem Output auch die Einkommen variieren. Der größte
  Teil des Werts in Bereichen, in denen die Fähigkeiten variieren, wird
  tendenziell von einer kleinen Anzahl von Personen geschaffen. Dies ist
  ein häufiges Merkmal der wettbewerbsintensivsten Märkte. Das ist
  beispielsweise im Sport offensichtlich. Millionen von jungen Menschen
  weltweit spielen auf verschiedenen Ebenen Fußball. Aber 99 Prozent des
  Geldes, das ausgegeben wird, um Fußballspiele anzusehen, werden
  gezahlt, um die Leistungen eines winzigen Bruchteils der Gesamtzahl
  der Spieler zu sehen. Gleiches gilt für die Welt der Schauspielerei,
  die voller angehender Schauspieler und Schauspielerinnen ist. Dennoch
  werden nur wenige Stars. Ebenso werden jährlich zehntausende von
  Büchern veröffentlicht. Aber das meiste Geld aus Tantiemen wird an
  eine kleine Anzahl von Bestseller-Autoren gezahlt, die ihre Leser
  wirklich unterhalten können. Unglücklicherweise gehören wir nicht zu
  ihnen.

  Ein weiteres Hindernis für Erpressung ist die Tatsache, dass die
  Leistung der Personen, die dieselben Geräte benutzen, sehr
  unterschiedlich ist. Sie verursacht ein großes Verhandlungsproblem
  darüber, wie der Gewinn geteilt werden sollte. Wo ein relativ kleiner
  Teil der an einer bestimmten Tätigkeit teilnehmenden Personen den
  größten Wert schafft, ist es mathematisch nahezu unmöglich, dass sie
  durch ein erzwungenes Ergebnis, das den Durchschnitt der Einkommen
  darstellt, bessergestellt sein könnten. Ein Softwareprogrammierer
  könnte einen Algorithmus zur Steuerung eines Roboters entwickeln,
  dessen Wert in die Millionen geht. Ein anderer, der mit identischer
  Ausstattung arbeitet, könnte ein Programm schreiben, das nichts wert
  ist. Der produktivere Programmierer ist nicht eher bereit, sein
  Einkommen an das seines Kollegen zu binden, als dass Tom Clancy
  zustimmt, seine Buchtantiemen mit unseren zu mitteln.

  Schon in der Anfangsphase der Informationsrevolution ist es viel
  offensichtlicher als noch 1975, dass Fähigkeiten und geistige
  Fertigkeiten entscheidende Variablen für die Wirtschaftsleistung sind.
  Damit ist die einst stolze Rationalisierung für die Erpressung der
  Kapitalisten durch die Arbeiter, die während der Industrieperiode
  vorherrschte, bereits hinfällig geworden. Die Fantasie, dass
  ungelernte Arbeit tatsächlich den Wert schafft, der scheinbar in einem
  unverhältnismäßigen Anteil von den Kapitalisten und Unternehmern
  eingesackt wird, ist bereits ein Anachronismus. Im Falle der
  Informationstechnologie ist das nicht einmal mehr eine plausible
  Fiktion. Wenn sich ein Programmierer hinsetzt, um einen Code zu
  schreiben, besteht eine direkte Verbindung zwischen seinen Fähigkeiten
  und seinem Produkt, als dass man sich über die Verantwortlichkeiten im
  Unklaren sein könnte. Es ist unbestreitbar offensichtlich, dass ein
  Analphabet oder Halbanalphabet keinen Computer programmieren könnte.
  Es ist daher ebenso offensichtlich, dass jeglicher Wert der von
  anderen Leuten erstellten Programme nicht von ihm gestohlen worden
  sein kann. Deshalb hört man jetzige Schreie über „Ausbeutung`` von
  Arbeitern hauptsächlich unter Hausmeistern.

  Die Informationstechnologie macht deutlich, dass das Problem, dem
  Personen mit geringer Qualifikation gegenüberstehen, nicht darin
  besteht, dass ihre Produktionskapazitäten unfair ausgenutzt werden;
  sondern vielmehr die Befürchtung, dass sie möglicherweise nicht in der
  Lage sind, einen echten wirtschaftlichen Beitrag zu leisten. Wie Kevin
  Kelly in \emph{Out Of Control} vorschlägt, könnte der „Emporkömmling``
  eines Auto-Unternehmens des Informationszeitalters das Produkt von
  „einem Dutzend Menschen`` sein, die die meisten ihrer Teile auslagern
  und dennoch Autos produzieren, die sorgfältiger auf die Wünsche des
  Käufers zugeschnitten sind, als alles, was bisher aus Detroit oder
  Tokio zu sehen war: „Autos, jedes einzelne auf den Kunden
  zugeschnitten, werden von einem Netzwerk von Kunden bestellt und
  verschickt, sobald sie fertig sind. Formen für die Karosserie des
  Autos werden schnell von computergesteuerten Lasern geformt und mit
  Designs versehen, die durch Kundenreaktionen und Zielmarketing
  entstanden sind. Eine flexible Reihe von Robotern montiert die Autos.
  Die Roboterreparatur und -verbesserung wird an ein Roboterunternehmen
  ausgelagert.`` \footnote{Kelly, op. cit., S. 191-192.}
\end{enumerate}

\subsection{„Werkzeuge mit einer
Stimme``}\label{werkzeuge-mit-einer-stimme}

Immer mehr kann unqualifizierte Arbeit von automatisierten Maschinen,
Robotern und Computersystemen, wie digitalen Assistenten, erledigt
werden. Als Aristoteles Sklaven als „Werkzeuge mit einer Stimme``
bezeichnete, sprach er von Menschen. In nicht allzu ferner Zukunft
werden „Werkzeuge mit einer Stimme``, wie die Dschinns aus dem Märchen,
in der Lage sein zu sprechen und Anweisungen zu befolgen, und sogar
komplexe Aufgaben zu bewältigen. Die rasch zunehmende Rechenleistung hat
bereits eine Reihe von primitiven Anwendungen der Spracherkennung
hervorgebracht, wie Freisprechtelefone und Computer, die mathematische
Berechnungen nach verbalen Anweisungen durchführen. Computer, die
Sprache in Text umwandeln, wurden bereits Ende 1996 auf den Markt
gebracht, während wir dies schreiben. Da die Fähigkeiten zur
Mustererkennung verbessert werden, werden Computer, die mit
Sprachsynthesizern vernetzt sind, über Netzwerke zahlreiche Funktionen
durchführen, die früher von Menschen übernommen wurden, die als
Telefonoperator, Sekretäre, Reiseagenten, Verwaltungsassistenten,
Schachmeister, Schadensregulierer, Komponisten, Aktienhändler,
Cyberkrieg-Spezialisten, Waffenanalysten oder sogar gewiefte
Verführerinnen tätig waren, die die Anrufe von 0190-Nummern beantwortet
haben.

Michael Mauldin von der Carnegie-Mellon-Universität hat eine künstliche
Persönlichkeit namens Julia programmiert, die in der Lage ist, fast
jeden, mit dem sie im Internet spricht, zu täuschen. Presseberichten
zufolge ist Julia eine „schlagfertige Dame, die ihr Leben in einem
Rollenspiel im Internet verbringt. Sie ist schlau, witzig und liebt es
zu flirten. Darüber hinaus ist sie eine Eishockey-Expertin und kann im
Handumdrehen den perfekten sarkastischen Kommentar abgeben. Julia ist
jedoch keine Frau. Sie ist ein Bot, eine künstliche Intelligenz, die nur
im Äther des Internets existiert.`` \footnote{Gayle Ni. Hanson, \emph{A
  Riveting Account of \textquotesingle Life\textquotesingle{} in
  Postmodernist Cyberspace}, Washington Times, 24. Dezember 1995, S. B7.}
Der erstaunliche Fortschritt, der bereits in der Programmierung von
künstlicher Intelligenz und digitalen Dienern gemacht wurde, lässt kaum
Zweifel daran, dass noch viele praktische Anwendungen zu erwarten sind.
Dies hat bedeutende megapolitische Konsequenzen.

\subsection{Das Individuum als
Ensemble}\label{das-individuum-als-ensemble}

Die Entwicklung von „Werkzeugen mit einer Stimme`` für vielfältige
Anwendungen schafft die Möglichkeit, dass sich das Individuum auf
mehrere gleichzeitige Aktivitäten verteilt. Das Individuum wird nicht
mehr einzigartig sein, sondern potenziell ein Ensemble aus Dutzenden
oder vielleicht sogar Tausenden von Aktivitäten, die durch intelligente
Agenten durchgeführt werden. Dies wird nicht nur die Produktivität der
talentiertesten Individuen erheblich steigern; es macht das souveräne
Individuum potenziell auch militärisch weitaus schlagkräftiger, als es
das Individuum je zuvor war.

Nicht nur wird ein Individuum offensichtlich seine Aktivitäten durch den
Einsatz einer im Grunde unbegrenzten Anzahl von intelligenten Agenten
vervielfachen können. Er oder sie wird sogar nach dem Tod handeln
können. Zum ersten Mal wird ein Individuum in der Lage sein,
komplizierte Aufgaben fortzusetzen, auch wenn es biologisch tot ist. Es
wird nicht mehr möglich sein, dass entweder ein Kriegsgegner oder ein
Verbrecher die Fähigkeit eines Individuums zur Vergeltung völlig
auslöschen kann, indem er es tötet. Dies ist eine der revolutionärsten
Innovationen in der Logik der Gewalt in der gesamten Geschichte.

\subsection{Einsichten für das
Informationszeitalter}\label{einsichten-fuxfcr-das-informationszeitalter-1}

Die größten Veränderungen im Leben betreffen Variablen, die niemand
beobachtet. Anders ausgedrückt, wir nehmen Variablen für
selbstverständlich hin, die über Jahrhunderte oder sogar Hunderte von
Generationen nur wenig geschwankt haben. Für den größten Teil der
Geschichte, wenn nicht sogar der menschlichen Existenz, hat das
Gleichgewicht zwischen Schutz und Erpressung innerhalb einer schmalen
Marge geschwankt, wobei die Erpressung stets die Oberhand behielt. Jetzt
steht eine Veränderung bevor. Die Informationstechnologie bereitet den
Boden für eine grundlegende Verschiebung in den Faktoren, die die Kosten
und Belohnungen von Gewaltanwendung bestimmen. Die Tatsache, dass
intelligente Agenten verfügbar sein werden, um Gewaltanwendung zu
untersuchen und vielleicht in der einen oder anderen Weise gegen
diejenigen, die Gewalt initiieren, zu reagieren, ist nur ein Hinweis auf
diese neue Aussicht auf Schutz. Vor 25 Jahren wäre die folgende Aussage
nicht mehr als das Geschwafel eines Exzentrikers gewesen: „Wenn du mich
tötest, werde ich das Geld von deinen Bankkonten fegen und es an
Wohltätigkeitsorganisationen in Nepal spenden.`` Nach der
Jahrtausendwende mag das nicht mehr der Fall sein. Ob es sich als
praktische Bedrohung erweisen würde, würde von Faktoren wie Zeit und Ort
bestimmt. Aber selbst wenn die Konten des potenziellen Übeltäters
undurchdringbar wären, gäbe es sicherlich andere kostspielige
Unannehmlichkeiten, die eine Armee von intelligenten Agenten als
Vergeltung für ein Verbrechen verhängen könnte. Denken Sie mal darüber
nach.

\subsection{Neue Alternativen von
Schutz}\label{neue-alternativen-von-schutz}

Das ist lediglich einer von vielen Wegen, den Schutz zu verbessern, die
durch die Technologie des Informationszeitalters ermöglicht werden. Die
meisten dieser Wege tendieren dazu, jenes nahezu monopolistische Anrecht
auf Schutz und Erpressung, das Regierungen in den letzten zwei
Jahrhunderten genossen haben, zu untergraben. Auch ohne den neuen
technologischen Schnickschnack gab es stets Alternativen für Schutz, die
nicht alle dazu tendiert haben von der Regierung monopolisiert zu
werden.

Eine Person, die sich bedroht fühlt, könnte einfach davonlaufen. Als die
Welt noch jung war und die Horizonte offen, wurde die Option der Flucht
häufig genutzt. Wenn Menschen sich um Verluste durch Diebstahl oder
Vandalismus sorgen, können sie sich dazu entscheiden,
Versicherungspolicen abzuschließen, um solche Risiken abzusichern.
Verwünschungen und Zaubersprüche, obwohl schwache Formen des Schutzes,
haben auch Leben gerettet und Diebstahlhandlungen abgewehrt. Sie
funktionieren manchmal in Gesellschaften, in denen die Räuber
abergläubisch sind. Wertgegenstände können auch geschützt werden, indem
man sie verbirgt. Dies ist manchmal eine effektive Methode, wenn sie
angewendet werden kann. Vermögenswerte können vergraben, mit Schlössern
gesichert, hinter hohen Mauern platziert und mit Sirenen und
elektronischen Überwachungsgeräten ausgestattet werden. Das Verstecken
von Personen und Eigentum war jedoch nicht immer praktikabel.

Trotz aller Vielfalt an Schutzmaßnahmen, die historisch eingesetzt
wurden, dominiert eine Methode alle anderen - die Fähigkeit, Gewalt mit
Gewalt zu übertreffen, eine größere Kraft aufzubringen, um jeden zu
überrumpeln, der Sie angreift oder Ihr Eigentum stiehlt. Die Frage ist,
an wen man sich für einen solchen Dienst wenden kann und wie man
jemanden motivieren kann, sein Leben und seine Gliedmaßen zu riskieren,
um einen bei der Bekämpfung von Schlägern zu helfen, die Gewalt gegen
einen initiieren könnten. Manchmal haben nahe Verwandte den Hilferuf
beantwortet. Manchmal haben Stammes- und Clan-basierte Gruppen als
inoffizielle Polizei gedient, die auf Gewalt gegen ihre Mitglieder mit
Blutfehden reagierten. Manchmal wurden Söldner oder private Wachen zur
Abwehr von Angriffen eingesetzt, aber nicht immer auf so nützliche
Weise, wie man es sich wünschen würde. Die neuen, intelligenten Agenten
des Informationszeitalters, auch wenn ihre Aktivitäten größtenteils auf
den Cyberspace beschränkt sein werden, bieten eine neue Alternative.
Ihre Loyalitäten, anders als die der Söldner, privaten Wächter und sogar
entfernten Cousins, sind unbestreitbar.

\subsection{Die Paradoxien der Macht}\label{die-paradoxien-der-macht}

Der Einsatz von Gewalt zur Abwehr von Gewalt ist voller Paradoxien.
Unter den bisher existierenden Bedingungen hätte jede Gruppe oder
Organisation, die man zur erfolgreichen Verteidigung seines Lebens und
Vermögens vor Angriffen anwerben konnte, notwendigerweise auch die
Fähigkeit gehabt, einem beides zu entziehen. Das ist ein Nachteil, für
den es keine einfache Lösung gibt. Normalerweise könnte man den
Wettbewerb dazu nutzen, Anbieter eines wirtschaftlichen Dienstes davon
abzuhalten, die Wünsche seiner Kunden zu ignorieren. Aber wo Gewalt eine
Rolle spielt, hat direkter Wettbewerb oft perverse Auswirkungen. In der
Vergangenheit hat dies in der Regel zu erhöhter Gewalt geführt. Wenn
zwei potentielle Schutzorganisationen ihre Streitkräfte aussenden, um
einander zu verhaften, gleicht das Ergebnis mehr einem Bürgerkrieg als
Schutz. Wenn man Schutz vor Gewalt sucht, möchte man normalerweise nicht
die Gewalt steigern, sondern sie unterdrücken. Und zwar unter
Bedingungen, die es nicht zulassen, dass die Kunden, die in erster Linie
für den Schutzdienst zahlen, ausgeplündert werden.

\begin{quote}
„\ldots während der Zeit, in der Menschen ohne eine gemeinsame Macht
leben, die sie alle in Ehrfurcht hält, befinden sie sich in dem Zustand,
den man Krieg nennt: und ein solcher Krieg bedeutet jeder gegen jeden,
wobei die Menschen ohne irgendeine andere Sicherheit leben als die, die
ihre eigene Stärke und ihre eigenen Erfindungen ihnen geben können.`` -
Thomas Hobbes
\end{quote}

\subsection{Monopol und Anarchie}\label{monopol-und-anarchie}

Deshalb war Anarchie, oder „der Krieg von jedem gegen jeden``, wie
Hobbes es beschrieb, selten ein zufriedenstellender Zustand. Lokaler
Wettbewerb in der Anwendung von Gewalt hat normalerweise bedeutet,
höhere Kosten für Schutz zu zahlen und weniger davon zu genießen.
Gelegentlich haben freidenkende Marktenthusiasten vorgeschlagen, dass
allein Marktmechanismen ausreichen würden, um für die Polizeiüberwachung
von Eigentumsrechten und den Schutz des Lebens zu sorgen, ohne dass eine
Souveränität überhaupt erforderlich wäre.\footnote{Eine knappe
  Einführung in die wissenschaftliche Erforschung der Anarchie findet
  sich in Gordon Tullock, Hrsg., \emph{Explorations in the Theory of
  Anarchy} (Blacksburg, Va.: Virginia Polytechnic Institute and State
  University, 1972). Siehe außerdem Murray N. Rothbard, \emph{Power and
  Market, Government and the Economy} (Menlo Park, Calif., 1970); und
  Robert Nozick, \emph{Anarchy, State and Utopia} (New York: Basic
  Books, 1974).} Einige der Analysen waren elegant, aber die Tatsache
bleibt, dass die freie Marktversorgung von Polizei- und
Justizdienstleistungen unter den megapolitischen Bedingungen des
Industrialismus sich als nicht lebensfähig erwiesen hat. Nur primitive
Gesellschaften, in denen das Verhalten stark stereotypisiert ist und die
Bevölkerung klein und homogen ist, haben ohne Regierungen überleben
können, die den Dienst der lokalen Monopolisierung des Schutzes durch
Gewalt boten.

Beispiele für anarchische Gesellschaften über dem Niveau von
Jäger-und-Sammler-Stämmen sind rar und antik. Sie finden sich
ausschließlich in den einfachsten Volkswirtschaften isolierter
Regenwasser-Bauern. Die Kafiren im vorislamischen Afghanistan. Einige
irische Stämme im dunklen Zeitalter. Einige Indianerstämme in Brasilien,
Venezuela und Paraguay. Andere Ureinwohner in verstreuten Teilen der
Welt. Ihre Methoden, Schutz ohne Regierung zu organisieren, sind nur
Kennern extremer Fälle bekannt. Wenn Sie mehr über sie erfahren möchten,
zitieren wir in unseren Anmerkungen mehrere Bücher, die weitere Details
enthalten.\footnote{Siehe Pierre Clastres, \emph{Society Against the
  State: The Leader as Servant and the Humane Uses of Power Among the
  Indians of the Americas} (New York: Urizen Books, 1977); und Jones,
  ebenda.} Primitive Gruppen konnten ohne eine spezielle Organisation,
die sich auf Gewalt spezialisiert, funktionieren, da sie kleine,
geschlossene Gesellschaften waren. Und sie waren isoliert. Sie konnten
auf enge Verwandtschaftsbeziehungen zurückgreifen, um sich gegen die
meisten gewaltsamen Bedrohungen in geringem Maße zu verteidigen, die die
einzigen waren, mit denen sie wahrscheinlich konfrontiert wurden. Als
sie auf größere Bedrohungen stießen, die von Staaten organisiert wurden,
wurden sie überwältigt und einer von außenstehenden Gruppen
monopolisierten Herrschaft unterworfen. Dies geschah immer wieder.
Überall dort, wo sich Gesellschaften gebildet haben, die über Banden und
Stämme hinausgingen, insbesondere dort, wo Handelswege verschiedene
Völker miteinander in Kontakt brachten, sind immer wieder Spezialisten
für Gewalt aufgetaucht, um den Überschuss zu plündern, den friedliche
Menschen produzieren konnten. Als technologische Bedingungen die Erträge
aus Gewalt erhöhten, besiegelten sie das Schicksal von Gesellschaften,
die nicht darauf vorbereitet waren, viele Ressourcen in die
Kriegsführung zu bewegen.

\begin{quote}
„Welche Fürsten leisteten den Polizeidienst? Welche waren
Gewaltverbrecher oder sogar Plünderer? Ein Plünderer konnte tatsächlich
zum Polizeichef werden, sobald er seine ‚Beute' regulierte, sie an die
Zahlungsfähigkeit anpasste, sein Revier gegen andere Plünderer
verteidigte und sein territoriales Monopol lange genug aufrechterhielt,
damit es durch Gewohnheit legitimiert wurde.`` \footnote{Lane,
  \emph{Economic Consequences of Organized Violence}, ebenda, S. 403.} -
Frederic C. Lane
\end{quote}

\subsection{Regierungen als Verkäufer von
Schutz}\label{regierungen-als-verkuxe4ufer-von-schutz}

Wie wir bereits mehrfach festgestellt haben, besteht die hauptsächliche
wirtschaftliche Funktion der Regierung aus der Perspektive derjenigen,
die die Steuern zahlen, darin, den Schutz von Leben und Eigentum zu
gewährleisten. Dennoch agiert die Regierung oft wie die organisierte
Kriminalität, indem sie Ressourcen von Menschen innerhalb ihres
Einflussbereichs als Tribut oder Beute extrahiert. Die Regierung ist
nicht nur ein Schutzdienst; sie ist auch eine organisierte
Schutzgelderpressung. Während die Regierung Schutz vor Gewalt von
anderen bietet, berechnet sie den Kunden auch für den Schutz vor
Schäden, die sie ansonsten selbst verursachen würde. Die erste Aktion
ist eine wirtschaftliche Dienstleistung. Die zweite ist eine
Schutzgelderpressung. In der Praxis mag die Unterscheidung zwischen den
beiden Formen des „Schutzes`` schwer zu erkennen sein. Regierungen, so
hat Charles Tilly betont, könnte man vielleicht am besten als „unsere
größten Beispiele für organisierte Kriminalität`` verstehen.\footnote{Charles
  Tilly, \emph{War Making and State Making as Organized Crime}, in Peter
  B. Evans, Dietrich Rue Schemeyer, and Theda Skoepol, Bringing the
  State Back In (Cambridge: Cambridge University Press, 1985), S. 169.}

Die Aktivitäten selbst der besten Regierung beinhalteten meist eine
Mischung aus dem wirtschaftlichen Dienstleistungsschutz und Erpressung.
Historisch gesehen konnten beide Verfolgungen optimiert werden, wenn die
Regierung ein nahezu Monopol auf Zwang innerhalb der Gebiete durchsetzen
konnte, in denen sie tätig war. In Fällen, in denen eine einzige
bewaffnete Gruppe die Vorherrschaft in der Anwendung von Gewalt erlangen
konnte, war die Qualität des Schutzdienstes, den sie bieten konnte,
normalerweise weit überlegen gegenüber dem, was von einer der mehreren
konkurrierenden Schutzagenturen erzielt werden konnte, die um das
gleiche Territorium kämpfte.

\subsection{Ein natürliches Monopol auf
Land}\label{ein-natuxfcrliches-monopol-auf-land}

Das Erreichen eines lokalen Monopols auf Zwang ermöglichte einer
Regierung nicht nur, ihre potentiellen Kunden effektiver vor Gewalttaten
von anderen zu schützen, sondern senkte auch die Betriebskosten der
Regierung erheblich. Wie Lane es ausdrückte: „Die Gewalt anwendende,
Gewalt kontrollierende Industrie war ein natürliches Monopol, zumindest
auf dem Land. Innerhalb territorialer Grenzen konnte der von ihr
erbrachte Dienst viel kostengünstiger von einem Monopol produziert
werden.`` \footnote{Ebenda.} „So ermöglichte ein Monopol auf die
Ausübung von Gewalt innerhalb eines zusammenhängenden Territoriums es
einem schutzproduzierenden Unternehmen, sein Produkt zu verbessern und
seine Kosten zu senken.`` \footnote{Lane, \emph{Economic Consequences of
  Organized Violence}, ebenda, S. 402.} Eine solche
Regierungsorganisation konnte mehr Schutz mit weniger Ausgaben anbieten,
wenn sie nicht ständig militärische Maßnahmen ergreifen musste, um
konkurrierende Gruppen abzuwehren, die Schutzgelder von ihren Kunden zu
erpressen versuchten.

Die Aussicht, dass die Informationstechnologie helfen könnte, die
Annahme, dass Souveränität auf einem territorialen Monopol basieren
muss, zu „lockern``, hat bereits die Aufmerksamkeit von politischen
Theoretikern erregt. Dies ist das Thema von \emph{Beyond Sovereignty:
Territory and Political Economy in the Twenty-First Century} von David
J. Elkins. Elkins bestätigt unsere These, dass Monopolregierungen dazu
bestimmt sind, genauso entbündelt zu werden wie religiöse Monopole in
den Jahren nach 1500. Er schreibt: „Wir nahmen früher an, dass
Religionen ihr eigenes Territorium oder ‚Revier' haben sollten. Als die
Nationen die Weltreligionen als souveräne Schiedsrichter über Leben und
Tod ablösten, wich die ‚Kompaktheit' und die ‚Begrenztheit' der Religion
der uns heute vertrauten Vermischung von Gläubigen in ein und demselben
Gebiet. Stattdessen lehnen wir die Vermischung von Nationen oder
Provinzen ab, obwohl ich glaube, dass diese Annahme gerade im Begriff
ist zusammenzubrechen.`` \footnote{David J. Elkins, \emph{Beyond
  Sovereignty: Territory and Political Economy in the Twenty-First
  Century}. Toronto: University of Toronto Press, 1995, S. 13-14.} Er
argumentiert weiter: Im Einklang mit unserer Ansicht, dass territoriale
Souveränitätsmonopole abgebaut werden können, ohne dass Anarchie
entsteht, wie das beispielsweise die Teilung der Souveränität zwischen
nationalen und provinziellen Regierungen in einem föderalen System wie
Kanada und der gemeinschaftlichen Regierung mit gemeinsamer
französischer und britischer Souveränität, die einige Pazifikinseln für
einen Großteil dieses Jahrhunderts prägte, zeigen. Somit können
territoriale Souveränitätsmonopole, während sie selten gewaltsam
entbündelt worden sind, durch Vereinbarungen entbündelt werden. Laut
Elkins, und wir stimmen zu, ist „die territoriale Nation ein Bündel oder
Korb, in den andere Aspekte unseres Lebens passen. Es ähnelt dem
wirtschaftlichen Konzept eines ‚Warenkorbs' -- man kann Artikel nicht
einfach einzeln nehmen, sondern muss sie kollektiv nehmen. In einem
Restaurant kann man ‚à la carte' bestellen; aber was unsere Identitäten
anbelangt, müssen wir nehmen, was die Nationen gebündelt haben, was
‚table d'hôte' entspricht. ... Regierung à la carte wird den Bürgern im
einundzwanzigsten Jahrhundert natürlich erscheinen.`` \footnote{Ebenda,
  S. 29.} Es gibt keine Entwicklung, die dramatischer zur Zerlegung der
Souveränität und zum Aufstieg der Regierung à la carte beitragen wird
als das Auftauchen einer Cyberwirtschaft, die physische Grenzen
vollständig überwindet.

\begin{quote}
„Während die Frequenzen steigen und die Wellenlängen sinken, verbessert
sich die digitale Leistung exponentiell. Die Bandbreite erhöht sich, der
Energieverbrauch schrumpft, die Antennengröße schrumpft, die
Interferenzen brechen zusammen, die Fehlerquoten sinken.`` - George
Gilder
\end{quote}

\section[DAS GESETZ DES TELEKOSMOS HEBT DIE GESETZE DER NATIONEN
AUF]{\texorpdfstring{DAS GESETZ DES TELEKOSMOS HEBT DIE GESETZE DER
NATIONEN
AUF\footnote{Der Begriff Telekosmos bezieht sich in diesem Kontext auf
  das Buch \emph{Telecosm: The World After Bandwidth Abundance} von
  George Gilder. Der Begriff hat sich im deutschen Sprachraum jedoch
  nicht auf die hier intendierte Weise durchgesetzt. Anm. d.~Übers.}}{DAS GESETZ DES TELEKOSMOS HEBT DIE GESETZE DER NATIONEN AUF}}\label{das-gesetz-des-telekosmos-hebt-die-gesetze-der-nationen-auf507}

Wir sind nicht die Einzigen, die der Meinung sind, dass die Bandbreite
(oder die Belastbarkeit der Kommunikationsmedien) dazu bestimmt ist, den
Territorialstaat zu übertrumpfen. Jim Taylor und Watts Wacker, die
Autoren von \emph{The 500-year Delta: What Happens After What Comes
Next}, formulieren ihr Argument nicht so wie wir, aber sie sehen klar,
dass „Zugang Globalismus schafft und der Globalismus die politischen
Systeme stört, indem er das Konzept der Grenzen obsolet macht. Wenn
Grenzen verschwinden, wird das Konzept der Besteuerung, das Regierungen
stützt, immer brüchiger... In dem Maße, wie die Grenzen verschwinden,
zerfällt das Konzept des Anspruchs - der Glaube, dass man, weil man an
einem bestimmten Ort geboren wurde, Anspruch auf die wirtschaftlichen
Vorteile hat, die mit diesem Ort verbunden sind - und in dem Maße, wie
er zerfällt, zerfallen auch die Vorzüge der Nationalität. Und während
all dies geschieht, werden die Ideale, die der Nationalität zugrunde
liegen - Patriotismus, Demokratie, Staat, Assimilation,
Vereinheitlichung, verantwortungsvolle Beteiligung, was auch immer das
in der jeweiligen Nation bedeuten mag - auf den Müllhaufen der
Geschichte verbannt.`` \footnote{Jim Taylor und Watts Wacker, \emph{The
  500-Year Delta: What Happens After What Comes Next}. New York:
  HarperCollins, 1997, S. 40.} Ohne es ausdrücklich zu sagen, spüren
auch sie offenbar, dass sich die Geschichte auf die Befreiung des
souveränen Individuums zubewegt. Wie sie es ausdrücken: „Am Horizont
wartet eine viel reinere Form des Individualismus als die Demokratie,
wie wir sie jetzt verstehen, es zulässt.`` \footnote{Ebenda, S. 67.} Wie
wird dies geschehen? Taylor und Wacker sehen eine starke Dynamik am
Werk:

Es ist eine einfache Tatsache, dass das größere Gefühl des Patriotismus
- die Liebe zur Nation, das Gefühl der kindlichen Pflicht ihr gegenüber
- keine noch länger besonders nützliche Veranlagung ist.\ldots{} Bürger,
die sich in der globalen Gesellschaft wohlfühlen, werden sich global
identifizieren. Sie werden politische, gesellschaftliche und
wirtschaftliche Entscheidungen treffen, die nicht auf nationaler
Identität basieren, sondern darauf, wie sich diese Entscheidungen auf
sie selbst und auf Gleichgesinnte in der ganzen Welt beziehen.\ldots{}
Nationen und Unternehmen, die erfolgreich sind, werden sich entsprechend
organisieren. Sie werden die Freiheit zu wissen, zu gehen, zu tun und zu
sein maximieren. Nationen und Unternehmen, die das nicht tun, die
weiterhin Rückzugsgefechte auf der Grundlage von Nostalgie führen,
werden verkümmern.\footnote{Ebenda, S. 41-42.}

Die Entwertung der physischen Grenzen durch die jährliche Verdreifachung
der Bandbreite und das geometrische Wachstum des Internets und des World
Wide Web wird den Prozess der Abschaffung der Regierungen beschleunigen.
Wenn sich die jährliche Verdreifachung der Bandbreite bis zum Jahr 2012
fortsetzt, würde dies ein milliardenfaches Wachstum der Bandbreite seit
1993 bedeuten, seit George Gilder zum ersten Mal andeutete, dass die
Bandbreite noch schneller wachsen würde als die Kapazität der
Mikroprozessoren. Sollte dies der Fall sein - wovon wir angesichts der
jüngsten Durchbrüche in der integrierten Optik ausgehen - würde die
dadurch entstehende Fülle an Kommunikationsmöglichkeiten zu einer
fantastischen Zunahme des Cyberhandels führen. Mit dem
Wellenmultiplexverfahren kann ein einziger Faserstrang, der so dünn wie
ein menschliches Haar ist, eine Billion Bits pro Sekunde
übertragen.\footnote{George Gilder, \emph{Fiber Keeps Its Promise: Get
  Ready. Bandwidth Will Triple Each Year for the Next 25, Creating
  Trillions in New Wealth.} Forbes ASAP, 7. April 1997.} Mit anderen
Worten: Ein einziges Glasfaserkabel könnte fünfundzwanzigmal mehr Bits
aufnehmen als die Gesamtlast aller Kommunikationsnetze der Welt
zusammen. Die Erweiterungsmöglichkeiten sind verblüffend. Wenn so viel
Kommunikationskapazität freigesetzt wird, wird sehr viel mehr Geld für
Kommunikation ausgegeben werden, weil sie so billig ist. Und so
etablierte Medien wie Telefon und Fernsehen werden zu Anachronismen. Das
World Wide Web wird jeden Computer mit einer reichhaltigeren Mischung
von Signalen versorgen, als die Verbraucher es heute mit Kabelfernsehen
erleben. Die Revolution der Bandbreite wird die Menschen mehr und mehr
in die grenzenlose virtuelle Welt der Online-Communities und des
Cyberhandels ziehen, eine Welt, die eine so hohe grafische Dichte
aufweist, dass sie zum „Metaverse`` wird, der Art von alternativer
Cyberspace-Realität, die sich der Science-Fiction-Autor Neal Stephenson
vorgestellt hat. Stephensons „Metaverse`` ist eine virtuelle
Gemeinschaft mit eigenen Gesetzen, Fürsten und Bösewichten.\footnote{Siehe
  Neal Stephenson, \emph{Snow Crash}. New York: Bantam Books, 1993.} Da
immer mehr wirtschaftliche Aktivitäten in den Cyberspace verlagert
werden, wird der Wert der staatlichen Monopolmacht innerhalb der Grenzen
schrumpfen, so dass die Staaten einen wachsenden Anreiz haben, ihre
Souveränität aufzuteilen und zu fragmentieren.

So wie Nationalstaaten heute Anreize haben, Freihäfen, Freihandelszonen
und Zona Francas einzurichten, so werden sie auch Anreize haben, ihre
Souveränität zu verpachten. Wir haben bereits die weit fortgeschrittenen
Verhandlungen zwischen dem neunhundert Jahre alten Souveränen Orden der
Johanniter von Jerusalem, Rhodos und Malta, besser bekannt als
Malteserritter, und der Republik Malta über die Rückgabe der
Souveränität über Fort St.~Angelo an den Orden diskutiert. Wir erwarten,
dass diese Verhandlungen erfolgreich abgeschlossen werden. Andere werden
folgen. Einige Nationalstaaten werden die Souveränität über kleine
Enklaven und abgelegene Gebiete an völlig neue Gruppierungen und
virtuelle Gemeinschaften abtreten. Es ist in der Tat nicht
unwahrscheinlich, dass kommerzielle Unternehmen wie Sicherheitsfirmen
und Hotelketten um die Souveränität über kleine Gebiete bieten.
Wackenhut, Pinkerton und Argenbright könnten in Zukunft hybride
Rentnergemeinschaften und steuerfreie Zonen in attraktiven Gegenden der
Welt anbieten. Religiöse Organisationen wie die Malteserritter, die
jedoch alle denkbaren Konfessionen vertreten, werden auf ihre Weise
versuchen, in bestimmten abgelegenen Winkeln der Erde den Himmel zu
verwirklichen. Selbst wohlhabende Einzelpersonen und Familien werden
ihre eigenen Grundstücke besitzen, auf denen sie eine begrenzte
Souveränität ausüben, ihre eigenen Briefmarken und Pässe herausgeben und
eine Website unterhalten.

\section{MONOPOL UND RAUB}\label{monopol-und-raub}

Beachten Sie, dass die Anreize, Souveränität gegen eine Gebühr zu teilen
oder zu verpachten, ganz andere sind als die, denen Herrscher historisch
ausgesetzt waren, die mit ihrem lokalen Monopol auf Zwang militärischer
Konkurrenz ausgesetzt waren. Verpachtete Souveränität ist nicht
destabilisierender als die Errichtung einer Freihandelszone. Im
Gegensatz dazu wirkt sich der militärische Wettbewerb um Macht, wie er
von kämpfenden Warlords und Guerillabanden ausgeübt wird, direkt darauf
aus, ob die angehende Regierung stärkere Anreize hat, die Menschen in
ihrer Reichweite zu beschützen oder auszuplündern. Wo rivalisierende
Gruppen miteinander ringen und in einem unausgewogenen Gleichgewicht
manövrieren, steigt der Anreiz, räuberische Gewalt anzuwenden.
Plünderung wird attraktiver. Weil die Macht weniger stabil ist und das
lokale Monopol auf Zwang weniger sicher ist, schrumpfen die
Zeithorizonte derjenigen, die in der Lage sind, Gewalt anzuwenden. Der
„König des Berges`` steht möglicherweise auf einem derart rutschigen
Hang, dass er nicht erwarten kann, lange genug zu überleben, um seinen
Anteil an den erheblichen Gewinnen, die letztendlich aus der Eindämmung
von Gewalt resultieren, zu realisieren. Wenn dies der Fall ist, gibt es
wenig, was diejenigen, die das, was als Regierung gilt, befehligen,
davon abzuhalten, ihre Macht zum Terrorisieren und Plündern der
Gesellschaft einzusetzen.

Die Logik der Gewalt besagt also, dass je mehr konkurrierende bewaffnete
Gruppen in einem Gebiet aktiv sind, desto höher ist die
Wahrscheinlichkeit, dass sie zu räuberischer Gewalt greifen. Ohne eine
einzige überwältigende Macht, die freie Gewaltanwendung unterdrückt,
neigt diese dazu, sich auszubreiten, und viele Errungenschaften der
ökonomischen und sozialen Zusammenarbeit gehen dabei verloren.

Der Schaden, der entstehen kann, wenn Gewalt in einem Zustand der
Anarchie freien Lauf gelassen wird, wird durch das Schicksal Chinas
unter den Kriegsherren in den 1920er Jahren veranschaulicht. Es ist eine
Geschichte, die wir in \emph{The Great Reckoning} nacherzählen. Die
konkurrierenden Kriegsherren richteten großen Schaden in Gebieten an, in
denen es keine einzige, überwältigende Macht gab, um sie in Schach zu
halten. Ähnliche Geschichten, die einen ähnlichen Punkt illustrieren,
wurden der Welt in lebendiger Farbe durch CNN-Nachrichtenteams, die die
Straßen von Mogadischu, Somalia, durchstreiften, übertragen. Die
bewaffneten Truppen der somalischen Kriegsherren, die „Technicals``
genannt, brachten Anarchie in dieses traurige Land, bevor die
Vereinigten Staaten eine massive militärische Intervention einleiteten,
um sie einzudämmen. Als die allmächtige Macht der US-Streitkräfte
zurückgezogen wurde, holten die Technicals ihre Waffen wieder hervor,
und die Anarchie setzte sich fort. Ein Bericht in der Washington Post
bemerkte:

\begin{quote}
Pickup-Trucks, die mit Flugabwehrkanonen bestückt sind, pflügen einmal
mehr die staubigen, von Trümmern übersäten Straßen. Auch sind die
prahlerischen jungen Männer in T-Shirts und mit Kalaschnikow-Gewehren
über ihren Schultern zurück, die an provisorischen Straßensperren Geld
von vorbeifahrenden Autos und Bussen erpressen. Ein Stadtviertel, das
von einer Miliz kontrolliert wird, ist so schwer bewaffnet, dass die
Einheimischen es „Bosnien-Herzegowina`` nennen. Das Reisen durch die
gefährlichen Straßen dieser Stadt erinnert deutlich an 1992, als der
chaotische Krieg unter rivalisierenden Milizen Somalia in Anarchie und
eine Hungersnot stürzte, was eine US-geführte militärische Intervention
auslöste. Um Mogadischu zu durchqueren, müssen Reisende heutzutage
erneut eine Ladung bewaffneter Gangster anheuern, in der Hoffnung, dass
diese für rund hundert Dollar am Tag Schutz bieten, plus Freizeit für
das Mittagessen.\footnote{Keith B. Richburg, \emph{Two Years After U.S.
  Landing in Somalia, It\textquotesingle s Back to Chaos}, Washington
  Post, 4. Dezember 1994, S. Al.}
\end{quote}

Die Beispiele von Somalia, Ruanda und weiteren, die Sie bald im
Fernsehen sehen werden, liefern einen farbenfrohen Beweis dafür, dass
gewaltsame Konkurrenz um die Gebietskontrolle nicht die gleichen
unmittelbaren wirtschaftlichen Gewinne bringen wie andere
Wettbewerbsformen. Im Gegenteil, die umherziehenden Banditen und
Plünderer, die in der Anarchie konkurrieren, haben nicht einmal die
schwachen Anreize zum Schutz produktiver Tätigkeiten, die manchmal
selbst die schwere Hand von Diktatoren erleichtern, wenn ihre Herrschaft
gesichert ist.

\begin{quote}
„Die Gesellschaft dessen, was wir das moderne Zeitalter nennen, ist vor
allem im Westen durch ein gewisses Maß an Monopolisierung geprägt. Der
freie Gebrauch von Militärwaffen wird dem Einzelnen verweigert und einer
zentralen Autorität jeglicher Art vorbehalten, ebenso wie die
Besteuerung des Eigentums oder Einkommens der Einzelnen in den Händen
einer zentralen sozialen Autorität konzentriert ist. Die finanziellen
Mittel, die so in diese zentrale Autorität fließen, erhalten ihr Monopol
auf militärische Gewalt aufrecht, während diese wiederum das Monopol auf
Besteuerung aufrechterhält. Keines hat in irgendeinem Sinn Vorrang vor
dem anderen; sie sind zwei Seiten desselben Monopols. Wenn eines
verschwindet, folgt das andere automatisch; die Monopolregel kann
manchmal stärker auf der einen Seite als auf der anderen erschüttert
werden`` \footnote{Zitiert in Tilly, \emph{Coercion, Capital and
  European States}, ebenda, S. 85.} - Norbert Elias
\end{quote}

\section{DIE ENTWICKLUNG DES
SCHUTZES}\label{die-entwicklung-des-schutzes}

Lane entwickelte ein Argument, das wir für unsere Zwecke in der
Vorstellung, wie das Informationszeitalter sich entfalten könnte,
zweckentfremdet haben. Er argumentierte, dass die Geschichte der
westlichen Wirtschaften seit dem dunklen Zeitalter in Bezug auf vier
Stufen von Wettbewerb und Monopol in der Organisation von Gewalt
interpretiert werden kann. Obwohl Lane größtenteils über die
megapolitischen Faktoren schweigt, die wir als Einflussfaktoren auf die
Größe der Regierungsoperationen identifizieren, stimmt seine Erforschung
der Ökonomie der Gewalt eng mit dem Argument überein, das wir in
\emph{Blood in the Streets} und \emph{The Great Reckoning} und an
anderer Stelle in diesem Band dargelegt haben.

Wir haben bereits einige der megapolitischen Faktoren analysiert, die
eine Rolle in der Evolution der westlichen Gesellschaft nach dem Fall
Roms gespielt haben. Lane hat sich ebenfalls mit diesem Zeitraum
beschäftigt und sich auf die wirtschaftlichen Konsequenzen dieses
Wettbewerbs um die Monopolisierung der Gewalt konzentriert. Er erkannte
vier wichtige Stadien in der Funktionsweise der Wirtschaften im Laufe
der letzten tausend Jahre, wobei jede einen unterschiedlichen Abschnitt
in der Organisation der Gewalt beinhaltete.\footnote{Beachten Sie, dass
  Lanes vier Stadien des Wettbewerbs und der Monopolisierung im Gebrauch
  von Gewalt sich von den vier Stadien in der Organisation des
  Wirtschaftslebens unterscheiden, die wir identifizieren, nämlich
  Sammeln und Jagen, Landwirtschaft, Industrialisierung und das
  Informationszeitalter.}

\subsection{Aus dem dunklen Zeitalter}\label{aus-dem-dunklen-zeitalter}

Die erste Stufe ist die von „Anarchie und Raub``, die die feudale
Revolution von vor tausend Jahren geprägt hat. Während Lane die Daten
für keine seiner zusammenfassenden Perioden angibt, setzt die Mathematik
die Grenze seiner ersten Periode recht klar, und seine Beschreibung der
Phase der „Anarchie und Raub`` scheint den Zuständen während des
Übergangs von der Dunklen Ära zu entsprechen, wenn der Einsatz von
Gewalt „sogar auf dem Land hochgradig wettbewerbsfähig`` war.\footnote{Lane,
  \emph{Economic Consequences of Organized Violence}, ebenda, S. 411.}
Er erklärt nicht warum, aber wenn Gewalt „hochgradig wettbewerbsfähig``
ist, bedeutet dies normalerweise, dass es erhebliche Hindernisse für die
Ausübung von Macht auf irgendeine Distanz gibt. In militärischen
Begriffen dominiert die Verteidigung über den Angriff.

Aus Gründen, die wir in Kapitel 3 erklärt haben, fiel diese Phase von
„Anarchie und Raub`` mit einem Rückgang der Produktivität in der
Landwirtschaft aufgrund nachteiliger klimatischer Veränderungen
zusammen. Da die Technologie zu der Zeit nur wenige wirksame
Skaleneffekte zur Sicherung eines Gewaltmonopols bot, war der Wettbewerb
zwischen den angehenden Herrschern weit verbreitet. Die wirtschaftliche
Aktivität wurde erstickt.

Die Schwäche der Wirtschaft verschärfte das Problem, eine stabile
Ordnung zu etablieren. Die Errichtung eines lokalen Gewaltmonopols
verursachte zu hohe Kosten im militärischen Bereich, im Vergleich zum
geringen Wert des Wirtschaftsumsatzes. Ohne die Fähigkeit, ein wirksames
Monopol über ein wirtschaftlich tragfähiges Gebiet durchzusetzen,
terrorisierten und plünderten die bewaffneten Ritter zu Pferd, während
sie ihren Kunden kaum „Schutz`` boten.

\subsection{Feudalismus}\label{feudalismus}

„Die zweite Phase beginnt dort, wo kleine regionale oder provinzielle
Monopole etabliert werden. Die landwirtschaftliche Produktion steigt an
und der größte Teil des Überschusses wird von den kürzlich etablierten
Monopolisten der Gewalt eingezogen.`` \footnote{Ebenda.} Trotzdem ist
der Überschuss während dieser zweiten Phase, die wir mit dem frühen
Mittelalter identifizieren, relativ karg. Das Wirtschaftswachstum wird
durch das Fehlen von Größenvorteilen bei der Organisation von Gewalt
gebremst, was die militärischen Kosten für die Durchsetzung lokaler
Monopole hoch hält. Aber während die Kosten hoch bleiben, steigt der
Preis, den Minisouveränitäten für Schutz verlangen können, da die
wirtschaftliche Aktivität sich ausdehnt, wenn die Anarchie eingedämmt
wird.

Während einer späten Phase der zweiten Stufe locken viele Tributnehmer
Kunden durch Sonderangebote für landwirtschaftliche und kommerzielle
Unternehmen an. Sie bieten Schutz zu niedrigen Preisen für diejenigen
an, die neue Länder in Kultur bringen werden, und spezielle
Polizeidienste zur Förderung des Handels, wie den von den Grafen von
Champagne für Händler, die zu ihren Messen kommen.`` \footnote{Ebenda.}
Mit anderen Worten, als sie in der Lage waren, eine ausreichende
Kontrolle über das Territorium zu verhandeln, taten lokale Kriegsherren
das, was lokale Händler tun, wenn sie ihren Marktanteil erhöhen müssen:
sie rabattierten ihre Dienstleistungen, um Kunden anzuziehen. Die
Kriegsherren nutzten später die zusätzlichen Ressourcen aus der
zusätzlichen wirtschaftlichen Aktivität, um ihre Kontrolle über größere
Gebiete zu festigen. Einmal fest etabliert, begannen sie, mehr von den
Vorteilen des Monopols zu genießen. Ihre militärischen Kosten für
Polizeiarbeit neigten dazu zu fallen und sie konnten auch den Preis, den
sie verlangten, erhöhen, ohne sich Sorgen zu machen, dass dies ihre
Dienstleistung für Kunden weniger attraktiv machte.

In dieser komplizierten Phase der westlichen Geschichte nehmen
diejenigen, die Gewalt ausüben, die mittelalterlichen Herren und
Monarchen, den größten Teil des Überschusses über das Existenzminimum
ein. Es gibt nur wenige Händler. Die erfolgreichsten sind diejenigen,
die am besten in der Lage sind, die Steuern, Gebühren und anderen
Kosten, die von denen verlangt werden, die Geld für „Schutzdienste``
fordern, zu umgehen oder zu minimieren.

\subsection{Die frühe Neuzeit}\label{die-fruxfche-neuzeit}

Eine dritte Phase wird erreicht, wenn die Kaufleute und Landbesitzer,
die nicht auch auf Gewalt spezialisiert sind, „mehr vom ökonomischen
Überschuss bekommen als Lehensträger und Monarchen ... In dieser dritten
Phase erhalten die Unternehmen, die sich auf Gewalt spezialisiert haben,
weniger vom Überschuss als die Unternehmen, die Schutz vor den
Regierungen kaufen.`` \footnote{Ebenda, S. 412.} Da erfolgreiche
Kaufleute eher dazu neigen, ihre Gewinne neu zu investieren als sie zu
verbrauchen, führte der höhere Gewinn der Kaufleute in dieser
geschichtlichen Phase zu einem sich selbst verstärkenden Wachstum.

\subsection{Das Zeitalter der
Fabriken}\label{das-zeitalter-der-fabriken}

Lane identifiziert den Übergang von der dritten zur vierten Phase mit
dem Aufkommen von technologischen und industriellen Innovationen als
wichtigere Faktoren für den Gewinn als die Senkung der Schutzkosten.
Dabei scheint Lane sich auf die Zeit seit 1750 zu beziehen. Von diesem
Zeitpunkt an begann der Charakter der Technologie, eine eindeutig
dominierende Rolle für den Wohlstand der Regionen zu spielen. Um ein
extremes Beispiel zu nennen, selbst in Gebieten, in denen es überhaupt
keine Regierung gab - wie in einigen Teilen Neuseelands vor 1840 - war
es unwahrscheinlich, dass sie einfach deshalb sehr wohlhabend wurden,
weil sie keine Steuern zahlten. Zu diesem Zeitpunkt in der Geschichte
waren Innovationen in der Industrietechnologie wichtiger für die
Erzielung von Gewinnen als jede Einsparung, die durch Senkung der
Schutzkosten erzielt werden konnte, sogar bis auf null. Als die Größe
der Regierung zunahm, wurden die ursprünglich von Regierungen
entwickelten Kredit- und Finanzierungsmechanismen zur Beschaffung von
Ressourcen für militärische Operationen auch für die Finanzierung von
größeren Unternehmungen verfügbar.

Obwohl Lane es nicht direkt ausspricht, hat die Konzentration von
technologischen Vorteilen an einem bestimmten Ort den Wettbewerb
zwischen den Rechtsgebieten reduziert und es „Unternehmen, die sich auf
den Einsatz von Gewalt spezialisiert haben``, oder Regierungen,
ermöglicht, höhere Preise zu verlangen. Wenn es große technologische
Lücken zwischen den Wettbewerbern in einer Jurisdiktion und einer
anderen gibt, wie es während des Industriezeitalters der Fall war,
neigen Unternehmer in den Jurisdiktionen mit der besten Technologie
dazu, mehr Geld zu verdienen, auch wenn sie höhere Steuern und andere
Kosten an ihre Regierungen zahlen müssen.

\subsection{Raub mit einem Lächeln}\label{raub-mit-einem-luxe4cheln}

Regierungen im Industriezeitalter genossen ein erfreuliches Monopol, das
sie ausnutzen konnten. Ihre tatsächlichen Kosten für die Gewährleistung
des Lebens- und Körperschutzes waren im Vergleich zu den Preisen
(Steuern), die sie verlangten, verschwindend gering. Doch sie befanden
sich tatsächlich in einem Bereich, in dem der Wettbewerb so verdreht
war, dass sie sich wesentlich stärker auf das Geschäft des Raubes statt
auf das des Schutzes konzentrieren konnten und diese Tatsache nahezu
unbemerkt blieb. Es war ein seltener Moment in der Geschichte.

Die Nachteile der Anarchie unter den megapolitischen Bedingungen des
Industrialismus machten einen Wettbewerb in Schutzdiensten innerhalb
desselben Territoriums technologisch ungeeignet. Die einzige
Möglichkeit, unter diesen Bedingungen effektiven Schutz zu erreichen,
bestand darin, die größere Fähigkeit zur Anwendung von Gewalt zu
besitzen. Daher gab es wenig zu gewinnen, indem man versuchte, diesen
Teil der eigenen Steuern, der, in den Worten von Lane, „als Zahlung für
die erbrachte Dienstleistung`` ging, besser von „einem anderen Teil, bei
der die Versuchung nahe liegt, sie Raub zu nennen`` zu
unterscheiden.\footnote{Ebenda, S. 403.} Die Unterscheidung war
sicherlich real genug. Aber da man in jedem Fall die Steuern zahlen
musste, hatte es wenig Nutzen, diese vollständig zu entwickeln, außer
einer morbiden Neugierde nachzugeben. Wie Lane sagte, egal welcher Teil
der Steuern als Plünderung bezeichnet wurde, sie waren ein Preis, den
man zahlen musste „um größere Verluste zu vermeiden``.\footnote{Ebenda,
  S. 404.}

\subsection{Der Aufstieg der Einkommen im
Industriezeitalter}\label{der-aufstieg-der-einkommen-im-industriezeitalter}

Ein Teil des Grundes, warum dieses Dilemma während der letzten zwei
Jahrhunderte der Vorherrschaft des Nationalstaates erträglich war, lag
in der Tatsache, dass die Einkommen dramatisch stiegen, insbesondere in
den Jurisdiktionen, in denen die industrielle Entwicklung hauptsächlich
stattfand. Diejenigen, die die OECD-Regierungen leiteten, nahmen Jahr
für Jahr einen höheren Prozentsatz der Einkommen ein. Aber die Zunahme
der Plünderung wurde dennoch von weit größerem Wohlstand begleitet und
von einer größeren Ungleichheit des Wohlstands im Vergleich zum Rest der
Welt. Unter solchen Bedingungen waren Einwände gegen die Steuerwelle
unvermeidlich marginal und unzureichend, um die Ereignisse von ihrem
logischen Verlauf abzulenken. Tatsächlich hing das militärische
Überleben eines Industriestaates, wie in den vorherigen Kapiteln
dargelegt, weitgehend davon ab, dass seinen Ansprüchen auf die
Ressourcen seiner Bürger keine wirksamen Grenzen gesetzt werden konnten.

In jedem Industriestaat verliefen die Politiken mehr oder weniger in
dieselbe Richtung. Auf dem Höhepunkt des Industrialismus nach dem
Zweiten Weltkrieg erreichte der Grenzsteuersatz 90 Prozent oder mehr.
Dies war eine weitaus aggressivere Behauptung des Rechts des Staates,
Ressourcen zu extrahieren, als selbst die orientalischen Despoten der
frühen hydraulischen Zivilisationen zu machen neigten. Doch die
industrielle Form der Plünderung folgte ihrer eigenen Logik. Ein großer
Teil davon wurde von der Charakteristik der industriellen Technologie in
der ersten Hälfte des zwanzigsten Jahrhunderts bestimmt, die wir bereits
zuvor beschrieben haben.

Diese Technologie machte es praktisch unvermeidbar, dass der Staat einen
großen Anteil des Einkommens beschlagnahmte und umverteilte, wobei ein
Großteil der Last dieses Raubzugs auf eine kleine Gruppe von
Kapitalisten fiel. Die meisten industriellen Prozesse waren stark
abhängig von natürlichen Ressourcen und daher an die Orte gebunden, an
denen diese Ressourcen vorhanden waren. Ein Stahlwerk, eine Mine oder
ein Hafen konnten nur unter enormen Kosten oder gar nicht verlegt
werden. Solche Einrichtungen waren daher stationäre Ziele, die leicht
besteuert werden konnten. Grundstücks-, Unternehmens- und
Entnahmebesteuerung stiegen im Laufe dieses Jahrhunderts stark an. Auch
die Einkommensteuern stiegen, zunächst auf die Kapitalisten, aber
schließlich auch auf die Arbeiter selbst. Die Einführung großflächiger
industrieller Beschäftigung machte eine breit angelegte Einkommensteuer
praktisch, da Gehälter direkt an der Quelle gepfändet werden konnten,
wobei die Steuerbehörden die Einnahmen in Zusammenarbeit mit den
Buchhaltungsabteilungen der Industrieunternehmen koordinierten. Heute
nehmen wir dies als gegeben hin, aber die Einziehung einer
Einkommensteuer am Fabriktor war eine weitaus einfachere Aufgabe als
sich über das ganze Land zu verteilen, um einen Teil der Gewinne von
Millionen von unabhängigen Handwerkern und Bauern einzutreiben.

Kurz gesagt, die industrielle Technologie machte die Besteuerung
routinierter, berechenbarer und weniger gefährlich als die Besteuerung
in vielen früheren Epochen. Dennoch wurde dadurch ein höherer
Prozentsatz der gesellschaftlichen Ressourcen entzogen als durch jede
andere Form der Souveränität zuvor.

\subsection{Was schützen wir?}\label{was-schuxfctzen-wir}

Die Tatsache, dass Gesellschaften reicher werden konnten, während der
Gesamtprozentsatz des Einkommens, der durch Steuern absorbiert wurde,
erheblich anstieg, wirft eine Frage über den Charakter des Schutzes auf,
den Regierungen den industriellen Ökonomien boten. Was genau haben sie
beschützt? Unsere Antwort: Hauptsächlich industrielle Anlagen mit hohen
Kapitalkosten und bedeutender Anfälligkeit für Angriffe. Die Präsenz von
Großindustrieunternehmen wäre in einer ungeordneten Umgebung mit mehr
wettbewerblicher Gewalt nicht möglich gewesen, selbst wenn das Ergebnis
des Wettbewerbs eine Verringerung des allgemeinen Anteils am
Gesamtertrag gewesen wäre, den die Regierung einnimmt.

Aus diesem Grund sind kapitalintensive Betriebe in den amerikanischen
Slums sowie in Dritte-Welt-Ländern, in denen adhoc-Gewalt an der
Tagesordnung ist, unwirtschaftlich. Die industrielle Gesellschaft als
Ganzes konnte voranschreiten, weil eine bestimmte Art von Ordnung
etabliert und aufrechterhalten wurde. Unternehmen waren regelmäßigen,
vorhersehbaren „Abschöpfungen`` ausgesetzt, statt unregelmäßiger Gewalt.

Selbst auf dem Höhepunkt der Industrialisierung war es immer eine
Übertreibung, davon zu sprechen, dass eine Regierung ein
„Gewaltmonopol`` anwendet. Alle Regierungen versuchen ein solches
Monopol aufrechtzuerhalten, aber wie wir gesehen haben, stellten die
Mitarbeiter von Industrieunternehmen oft fest, dass sie in der Lage
waren, Gewalt gegen ihre Arbeitgeber anzuwenden. Solange die breite
Öffentlichkeit überhaupt Zugang zu Waffen hat, oder eine chaotische
Menge die physische Fähigkeit hat, einen Bus umzukippen oder Steine auf
die Polizei zu werfen, monopolisieren diejenigen, die die Regierung
kontrollieren, die Gewalt nicht völlig. Sie kontrollieren lediglich die
vorherrschende Gewalt, die in einem Maße dominant ist, dass es für die
meisten Menschen unter den gegebenen Bedingungen unwirtschaftlich wird,
mit ihnen zu konkurrieren.

\begin{quote}
„Eine netzbasierte Regierung, kann nur mit Zustimmung der Regierten
operieren. Daher muss jede Internetregierung ihren Bürgern echte
Vorteile bieten, wenn sie will, dass diese bleiben. Diese Vorteile
müssen nicht nur persönliche Güter oder Dienstleistungen sein, sondern
es können auch die breiteren Vorteile eines regulativen Regimes sein:
Ein sauberer, transparenter Marktplatz mit definierten Regeln und
Konsequenzen oder eine betreute Gemeinschaft, in der Kinder den
Menschen, denen sie begegnen, vertrauen können und die Privatsphäre des
Einzelnen geschützt ist.`` \footnote{Esther Dyson, \emph{Release 2.1: A
  Design for Living in the Digital Age}. New York: Broadway Books, 1998,
  S. 131.} - Esther Dyson
\end{quote}

\subsection{Das Informationszeitalter}\label{das-informationszeitalter}

Das Informationszeitalter bringt eine fünfte Phase in der Entwicklung
des Wettbewerbs bei der Verwendung von Gewalt im Westen hervor - eine
Phase, die von Lane nicht vorhergesehen wurde. Diese fünfte Phase
beinhaltet den Wettbewerb im Cyberspace, einer Arena, die nicht von
einem „Gewalt anwendenden Unternehmen`` monopolisiert werden kann. Sie
kann nicht monopolisiert werden, weil sie kein Territorium ist.

Obwohl Lanes Argumentation die herkömmlichen Nachkriegsannahmen über die
Unvermeidlichkeit des Nationalstaats einbezieht, erkannte er einen
Punkt, der möglicherweise zukünftig wichtiger für das Verständnis der
Zukunft ist, als es vor vierzig oder fünfzig Jahren den Anschein hatte.
Es ist die Tatsache, dass Regierungen nie stabile Monopole der Gewalt
auf dem offenen Meer etabliert haben. Denken Sie einmal darüber nach.
Kein Regierungsgesetz hat dort jemals ausschließlich Geltung gefunden.
Dies ist von äußerster Wichtigkeit, um zu verstehen, wie die
Organisation von Gewalt und Schutz sich entwickeln wird, während die
Wirtschaft sich in den Cyberspace verlagert, der überhaupt keine
physische Existenz hat. Aus den gleichen Gründen, die Lane wegen der
Beobachtung angeführt hat, dass keine Regierung jemals in der Lage
gewesen ist, Gewalt auf dem Meer zu monopolisieren, ist es noch weniger
wahrscheinlich, dass eine Regierung ein unendliches Reich ohne physische
Grenzen erfolgreich monopolisieren könnte.

\section{WETTBEWERB OHNE ANARCHIE}\label{wettbewerb-ohne-anarchie}

In der Vergangenheit, wenn Bedingungen es schwierig machten, dass eine
einzige gewalttätige Einheit ein Monopol errichtete, waren die
Ergebnisse Anarchie und Raub. Das Informationszeitalter hat jedoch die
technologischen Bedingungen, unter denen Gewalt organisiert wird, auf
tiefergreifende Weise verändert. Anders als in der Vergangenheit, als
die Unfähigkeit, den Schutz in einer Region zu monopolisieren, höhere
Militärkosten und geringere wirtschaftliche Erträge bedeutet, weist die
Tatsache, dass Regierungen den Cyberspace nicht monopolisieren können,
auf geringere Militärkosten und höhere wirtschaftliche Erträge hin. Denn
die Informationstechnologie schafft eine neue Dimension beim Schutz. Zum
ersten Mal in der Geschichte ermöglicht die Informationstechnologie die
Schaffung und den Schutz von Vermögenswerten, die vollständig außerhalb
des Hoheitsbereichs jedes einzelnen staatlichen Gewaltmonopols liegen.

\begin{quote}
„Länder, in denen die Einheiten der politischen Macht und Verwaltung
vielfältig sind und denen eine zentrale, stabile und unangefochtene
überwachende Instanz für Rechtsprechung und Macht fehlt, müssen ihre
eigenen funktionierenden Lösungen zur Bewältigung der Probleme finden,
die solche Grenzen hervorrufen.`` \footnote{Rees Davies, \emph{Frontier
  Arrangements in Fragmented Societies: Ireland and Wales}, in Robert
  Bartlett und Angus MacKay, eds., Medieval Frontier Societies (Oxford:
  Oxford University Press, 1992), S. 80.} - Rees Davies
\end{quote}

\subsection{Die Analogie zur Grenze}\label{die-analogie-zur-grenze}

Der Cyberspace ist in gewissem Sinne das Äquivalent zu einer
technologisch geschützten Handelszone, wie sie in Grenzgebieten während
des Mittelalters existierte. In der Vergangenheit, als die Macht der
Herren und Könige schwach war und sich die Ansprüche eines oder mehrerer
an einer Grenze überschnitten, existierte so etwas wie eine
Wettbewerbsregierung. Ein Blick darauf, wie die Markregionen
funktionierten, könnte Einblicke geben, wie Gesetze der Mark oder etwas
Ähnliches in den Cyberspace einwandern könnten.

Andorra überlebt als eine Art versteinerte Markregion zwischen
Frankreich und Spanien, ein Relikt von geopolitischen Bedingungen, die
es für beide Königreiche erschwert haben, das andere in diesem kalten
und nahezu unzugänglichen Gebiet von 500 Quadratkilometern in den
Pyrenäen zu dominieren. Im Jahr 1278 wurde eine Vereinbarung getroffen,
die die Oberhoheit über Andorra zwischen lokalen französischen und
spanischen Feudalherren, dem französischen Grafen von Foix und dem
spanischen Bischof von Urgel, aufteilte. Jeder von ihnen ernannte einen
von zwei „Viquiers``, die die minimale Gewalt der Regierung in Andorra
nur spärlich ausübten, hauptsächlich durch das Kommando über die winzige
andorranische Miliz, die heute eine Polizeitruppe ist. Die Rolle des
Grafen wurde vor langer Zeit durch die Geschichte überholt. Die
französische Regierung vertritt ihn heute aus Paris. Zu ihren Aufgaben
gehört es, die Hälfte des jährlichen Tributs zu akzeptieren, den Andorra
zahlt, ein Betrag, der weniger als eine Monatsmiete in einer
heruntergekommenen Wohnung beträgt. Der Bischof von Urgel erhält
weiterhin seinen Anteil am Tribut, so wie es seine Vorgänger im
Mittelalter taten.

Wie der geteilte Tribut andeutet, hat es in Andorra statt nur einer zwei
Quellen der „Aufsichtsgerichtsbarkeit und Macht`` gegeben. Berufungen
von andorranischen Zivilklagen wurden traditionell entweder beim
Bischofskolleg von Urgel oder beim Kassationshof in Paris eingereicht.

Eine Folge der unsicheren Position von Andorra war, dass fast keine
Gesetze erlassen wurden. Andorra hat seit mehr als siebenhundert Jahren
eine winzige Regierung und keine Steuern. Heute macht es das zu einem
immer attraktiveren Steuerparadies. Doch bis vor einer Generation war
Andorra für seine Armut bekannt. Einst dicht bewaldet, wurde es im Laufe
der Jahrhunderte von den Bewohnern entwaldet, die in den bitteren
Wintern versuchten, sich warm zu halten. Der gesamte Ort ist jedes Jahr
von November bis April eingeschneit. Selbst im Sommer ist Andorra so
kalt, dass Getreide nur an den südlichen Ausläufern wächst. Wenn unsere
Beschreibung es unattraktiv erscheinen lässt, haben Sie gerade das
Geheimnis seines Erfolges entdeckt. Andorra überlebte als feudale
Enklave im Zeitalter der Nationalstaaten, weil es abgelegen und
bettelarm war.

Einst gab es zahlreiche mittelalterliche Grenz- oder „Mark``-Regionen,
in denen Souveränitäten verschmolzen. Diese gewalttätigen Grenzgebiete
bestanden für Jahrzehnte oder manchmal sogar Jahrhunderte in den
Randgebieten Europas. Die meisten waren arm. Wie wir bereits erwähnt
haben, gab es Mark-Territorien zwischen den keltischen und englischen
Kontrollgebieten in Irland, zwischen Wales und England, Schottland und
England, Italien und Frankreich, Frankreich und Spanien, Deutschland und
den slawischen Grenzgebieten Mitteleuropas und zwischen den christlichen
Königreichen Spaniens und dem islamischen Königreich Granada. Genau wie
Andorra entwickelten diese Markgebiete eigene institutionelle und
rechtliche Formen, die wir im nächsten Jahrtausend wahrscheinlich
wiedersehen werden.

Aufgrund der Wettbewerbsposition beider Autoritäten, von denen beide
schwach waren, würden Herrscher manchmal sogar unter ihren Untertanen
Freiwillige anwerben, um in Markregionen zu siedeln, um die Reichweite
ihrer Autorität zu erhöhen. Fast selbstverständlich wurden die
Untertanen mit der Befreiung von Steuern in die Mark gelockt. Angesichts
der geringen Margen, auf denen sie konkurrierten, würde es schwieriger
für seine Anhänger werden, über die Runden zu kommen, wenn eine der
Autoritäten in einer Mark versuchen würde, Steuern zu erheben. Ebenso
würde er jedem einen Grund geben, sich seinem Konkurrenten
anzuschließen. Daher hatten die Bewohner einer Mark in der Regel die
Wahl, zu entscheiden, wessen Gesetze sie befolgen sollten. Diese Wahl
basierte auf der Schwäche der konkurrierenden Autoritäten; es handelte
sich nicht um eine ideologische Geste.

Dennoch traten praktische Schwierigkeiten auf, die gelöst werden
mussten. Im Feudalsystem standen Grundbesitzer, die Eigentum auf beiden
Seiten einer nominellen Grenze besaßen, vor einem ernsthaften Konflikt
der Pflichten. Beispielsweise könnte ein Herr an der Grenze zwischen
Schottland und England, der in beiden Königreichen Besitztümer besaß,
theoretisch im Kriegsfall beiden militärischen Dienst schulden. Um diese
widersprüchlichen Verpflichtungen zu lösen, konnten fast alle in der
feudalen Hierarchie durch einen rechtlichen Prozess namens ‚Bekenntnis'
wählen, wessen Gesetze sie befolgen wollten.

Die Informationstechnologie wird äquivalente Möglichkeiten für einen
wettbewerbsorientierten Standortwechsel wirtschaftlicher Aktivitäten
schaffen, allerdings mit wichtigen Unterschieden. Einer davon ist, dass
der Cyberspace, im Gegensatz zu mittelalterlichen Grenzgesellschaften,
aller Voraussicht nach das reichste aller Wirtschaftsgebiete sein wird.
Es wird daher eher eine wachsende als eine rückläufige Grenze
darstellen. Nur wenige Menschen in den Kernregionen der
mittelalterlichen Gesellschaft hätten sich ohne starke Anreize, oft auch
religiöser Natur, an die Grenzen bewegen wollen, da diese Regionen in
der Regel gewalttätig und arm waren. Daher zogen sie keine Ressourcen
aus der Kontrolle der Behörden ab. Der Cyberspace wird das aber tun.

Zweitens wird die neue Grenze kein Duopol sein, das zu Kollusionen
zwischen den beiden Behörden einlädt, um über ihre Grenzansprüche
Kompromisse zu finden. Solche Kompromisse neigten dazu, während der
mittelalterlichen Periode aus zwei Gründen nicht effektiv zu sein: Es
gab häufig deutliche kulturelle Unterschiede zwischen den
rivalisierenden Behörden; und wichtiger noch, ihnen fehlte die physische
Kapazität, eine ausgehandelte Lösung durchzusetzen, da sie nicht über
ausreichende militärische Präsenz vor Ort verfügten. In der Ära des
Nationalstaates, als nationale Behörden tatsächlich ausreichende
militärische Macht ausübten, um Lösungen durchzusetzen, verschwanden die
meisten Markregionen und unscharfen Grenzen. Die Grenzfestlegung wurde
zur Norm. Das ist eine stabile Lösung, wenn Gewalt-Duopolisten vor der
Aufgabe stehen, ihre Autorität über angrenzende Regionen zu teilen. Aber
der Wettbewerb um die Ansiedlung von Transaktionen in der
Cyberwirtschaft wird nicht zwischen zwei Behörden stattfinden, sondern
zwischen Hunderten von Behörden auf der ganzen Welt. Für die
territorialen Staaten wird es nahezu unmöglich sein, ein effektives
Kartell zu schaffen, um die Steuersätze hoch zu halten. Das wird aus
demselben Grund wahr sein, weil Kollusionen zur Erzielung von
Monopolpreisen auf Märkten mit Hunderten von Wettbewerbern nicht
funktionieren.

Als Beweis betrachten Sie den Schritt der Seychellen, einem kleinen Land
im Indischen Ozean, ein neues Investitionsgesetz zu erlassen, das
US-Regierungsbeamte als „Welcome Criminals``-Gesetz bezeichnen. Nach
diesem Gesetz erhält jeder, der 10 Millionen Dollar in den Seychellen
investiert, nicht nur einen garantierten Schutz vor der Auslieferung,
sondern auch einen diplomatischen Pass. Entgegen den Behauptungen der
US-Regierung sind die beabsichtigten Begünstigten jedoch nicht
Drogenhändler, die ohnehin im Allgemeinen unter dem Schutz wichtigerer
Regierungen stehen, sondern unabhängige Unternehmer, die politisch
inkorrekt geworden sind. Der erste potenzielle Begünstigte des
Seychellen-Gesetzes ist ein weißer Südafrikaner, der reich wurde, indem
er die Wirtschaftssanktionen gegen das ehemalige Apartheid-Regime
umging. Jetzt steht er vor der Gefahr wirtschaftlicher Vergeltung durch
die neue südafrikanische Regierung und ist bereit, den Seychellen für
Schutz zu zahlen.\footnote{Siehe Thomas W. Lippman, \emph{Seychelles
  Offers Investors Safe Haven for \$10 Million}, Washington Post, 31.
  December 1995, S. A27.}

Unabhängig vom Verdienst jedes einzelnen Falles zeigt das Beispiel,
warum Versuche von Regierungen, ein Schutzkartell auf dem Boden
aufrechtzuerhalten, zum Scheitern verurteilt sind.

Im Gegensatz zur mittelalterlichen Grenze, bei der der Wettbewerb nur
zwischen zwei Autoritäten stattfand, wird die Grenze im Cyber-Handel
zwischen hunderten von Zuständigkeitsbereichen liegen, wobei die Zahl
wahrscheinlich rasch in die Tausende steigen wird.

Im Zeitalter der virtuellen Unternehmen werden Einzelpersonen ihre
einkommensgenerierenden Aktivitäten in der Rechtsordnung ansiedeln, die
die beste Dienstleistung zu den niedrigsten Kosten bietet. Mit anderen
Worten, Souveränität wird kommerzialisiert werden. Im Gegensatz zu
mittelalterlichen Grenzgesellschaften, die in den meisten Fällen verarmt
und gewalttätig waren, wird der Cyberspace weder das eine noch das
andere sein. Der Wettbewerb, zu dem die Informationstechnologie die
Regierungen antreibt, ist kein militärischer, sondern ein Wettbewerb in
Bezug auf Qualität und Preis eines wirtschaftlichen Dienstes - echter
Schutz. Kurz gesagt, Regierungen werden verpflichtet sein, den Kunden
das zu geben, was sie wollen.

\subsection{Die verminderte Nützlichkeit von
Gewalt}\label{die-verminderte-nuxfctzlichkeit-von-gewalt}

Das heißt natürlich nicht, dass Regierungen darauf verzichten werden,
Gewalt einzusetzen. Ganz im Gegenteil. Vielmehr sagen wir, dass Gewalt
einen Großteil ihres Einflusses verliert. Eine mögliche Reaktion der
Regierungen wäre die verstärkte Anwendung von Gewalt auf lokaler Ebene,
um die abnehmende globale Bedeutung der Gewalt zu kompensieren. Was auch
immer Regierungen tun, sie werden nicht in der Lage sein, den Cyberspace
auf dieselbe Art und Weise mit Gewalt zu durchdringen, wie sie die
Territorien, die sie in der modernen Welt monopolisiert haben, mit
Gewalt durchdrungen haben. Egal wie viele Regierungen versuchen, in den
Cyberspace einzudringen, sie werden in diesem Bereich nicht mächtiger
oder leistungsfähiger sein als jeder andere.

Ironischerweise würden die Versuche der Nationalstaaten,
„Informationskriege`` zu führen, um den Cyberspace zu dominieren oder
den Zugang dazu zu verhindern, wahrscheinlich nur ihren eigenen
Untergang beschleunigen. Die Tendenz zur Auflösung großer Systeme ist
bereits stark durch das Wegfallen von Skaleneffekten und die steigenden
Kosten der Zusammenhaltung von fragmentierenden sozialen Gruppen. Die
Ironie der Informationskriege besteht darin, dass sie den spröden
Systemen, die aus dem Industriezeitalter übriggeblieben sind,
wahrscheinlich mehr Schock zufügen könnten als der aufkommenden
Informationswirtschaft selbst.

Solange die essenzielle Informationstechnologie funktioniert, könnte der
Cyber-Handel parallel zu den Kämpfen der Informationskriege ablaufen,
auf eine Weise, die in einem territorialen Krieg nie möglich wäre. Man
könnte sich nicht vorstellen, dass Millionen von kommerziellen
Transaktionen an der Front eines der Kriege des zwanzigsten Jahrhunderts
stattfinden. Aber virtuelle Kriege könnten die Kapazität des Cyberspace
für die Durchführung vielfältiger Aktivitäten nicht erschöpfen. Und da
es keine virtuelle Realität gibt, gäbe es wenig Gefahr der physischen
Nähe und praktisch gar keine, von einem explodierenden virtuellen
Schrapnell getroffen zu werden.

\subsection{Die Verwundbarkeit von großen
Systemen}\label{die-verwundbarkeit-von-grouxdfen-systemen}

Die Gefahren des Informationskrieges sind hauptsächlich Gefahren für
groß angelegte Industriesysteme, die unter zentraler Steuerung und
Kontrolle betrieben werden. Die militärischen Autoritäten in den
Vereinigten Staaten und anderen führenden Nationalstaaten planen und
befürchten Akte der Informationssabotage, die schwerwiegende Folgen für
die Abschaltung großer Systeme haben könnten. Ein Akt des Cyberkrieges
könnte eine Telefonumschaltstation abschalten, die Flugverkehrskontrolle
stören oder ein Pumpsystem sabotieren, das den Wasserfluss zu einer
Stadt regelt. Ein programmiertes Virus könnte sogar konventionelle oder
nukleare Generatoren abschalten und Teile des Stromnetzes lahmlegen.
Sogenannte Logikbomben könnten eine Vielzahl von Informationen
durcheinanderbringen, am empfindlichsten in zentralen Kontrollsystemen,
die vulnerable, groß angelegte Systeme aus der Industriezeit betreiben.
Kurz gesagt, ohne eine massive und umfassende Zerstörung aller
Informationstechnologie, die die Weltwirtschaft buchstäblich zum
Stillstand bringen würde, würden Cyberhandel und virtuelle Realität
jenseits der Kapazität jeder Regierung bleiben, sie zu ersticken oder
gar zu monopolisieren.

Sogar einer der deutlichsten Nachteile der Informationstechnologie, die
offensichtliche Anfälligkeit von Informationsspeichersystemen für
Verfall und Zerstörung, wurde durch neue Archivierungstechnologie
weitgehend gelöst. Ein neues System namens „High-Density Read-Only
Memory`` oder ``HD-ROM`` verwendet eine Ion-Mühle, ähnlich wie die in
computerunterstützten Fertigungssystemen, um Archive in einem Vakuum zu
erstellen. Die Speicherkapazität beträgt nun bis zu 25.000 Megabyte pro
Quadratzoll. Im Gegensatz zu früheren Systemen, die anfällig für
vorzeitigen Verfall und Störungen durch Schock waren, versprechen die in
HD-ROM gespeicherten Daten, für die Dauer vorhanden zu sein. Einer der
Entwickler von HD-ROM, Bruce Lamartine, sagt: „Es ist quasi
unempfindlich gegen die Verheerungen der Zeit, thermische und
mechanische Schocks oder die elektromagnetischen Felder, die für andere
Speichermedien so zerstörend sind.`` \footnote{Siehe \emph{ROM of Ages},
  Wired, Januar 1996, S. 52.} Selbst die Detonation einer Bombe durch
Kernwaffenterroristen würde nicht unbedingt wichtige Informationen
durcheinanderbringen oder zerstören, wie die Codes für digitales Geld,
auf die das reibungslose Funktionieren einer Cyberspace-Wirtschaft
angewiesen sein wird.

\begin{quote}
„Moderne Armeen sind so abhängig von Informationen, dass es möglich ist,
sie blind und taub zu machen, um einen Sieg zu erzielen, ohne im
herkömmlichen Sinne zu kämpfen.`` \footnote{Zitiert in James Adams,
  \emph{Dawn of the Cyber Soldiers}, The Sunday Times (London), 15.
  Oktober 1995, S. 3-5.} - Oberst Alan Campen, U.S.A.F. (im Ruhestand)
\end{quote}

\section{ÜBERMACHTBEFUGNISSE DER VIRTUELLEN
KRIEGSFÜHRUNG}\label{uxfcbermachtbefugnisse-der-virtuellen-kriegsfuxfchrung}

Die Annahmen des Nationalstaats im Krieg werden immer weniger sinnvoll,
je mehr die Bedeutung von Informationen in der Kriegsführung zunimmt. Da
der Cyberspace keine physische Existenz hat, haben Dimensionen, wie wir
sie in der physischen Welt kennen, keine dominierende Bedeutung. Es ist
unwichtig, wie viele Programmierer an der Festlegung einer
Befehlssequenz beteiligt waren. Alles, was zählt, ist, ob das Programm
funktioniert. Der souveräne Einzelne kann in der Cyberwelt ebenso viel
zählen wie ein Nationalstaat mit einem Sitz in der UN, einer eigenen
Flagge und einer Armee im Einsatz. In rein wirtschaftlicher Hinsicht
verfügen bereits einige souveräne Individuen über investierbare
Einkommen in Höhe von jährlich mehreren Hundert Millionen, Summen, die
die verfügbare Ausgabenmacht einiger bankrotter Nationalstaaten
übertreffen. Aber das ist noch nicht alles. Im Hinblick auf virtuelle
Kriegsführung durch Informationsmanipulation können einige Individuen
genauso groß oder größer als viele der Staaten der Welt erscheinen. Ein
bizarres Genie, das mit digitalen Dienern arbeitet, könnte theoretisch
die gleiche Wirkung in einem Cyberkrieg erzielen wie ein Nationalstaat.
Bill Gates könnte das sicherlich.

In diesem Sinne ist das Zeitalter des souveränen Individuums nicht nur
ein Slogan. Ein Hacker oder eine kleine Gruppe von Mathematikern, ganz
zu schweigen von einer Firma wie Microsoft oder fast jeder
Softwarefirma, könnte theoretisch alle oder einige der Dinge tun, die
die Cyberkrieg-Arbeitsgruppe des Pentagon in petto hat. Es gibt Hunderte
von Firmen im Silicon Valley und anderswo, die bereits über größere
Kapazitäten verfügen, einen Cyberkrieg zu führen, als 90 Prozent der
existierenden Nationalstaaten.

Die Annahme, dass die Regierungen weiterhin das Leben am Boden
monopolisieren werden, während alternative Schutzmöglichkeiten auf allen
Seiten geöffnet werden, ist ein Anachronismus. Ein weitaus
wahrscheinlicheres Szenario ist, dass die Nationen umstrukturiert werden
müssen, um ihre Anfälligkeit für Computerviren, logische Bomben,
infizierte Drähte und Falltürprogramme zu verringern, die von der
US-amerikanischen National Security Agency oder auch nur einem
jugendlichen Hacker überwacht werden könnten.

Die megapolitische Logik des Cyberspace legt nahe, dass zentrale Befehl-
und Kontrollsysteme, die derzeit die großen Infrastrukturen der Welt
dominieren, durch multizentrische Sicherheitsmodelle mit verteilten
Fähigkeiten ersetzt werden müssen, damit sie nicht leicht von einem
Computervirus erfasst oder blockiert werden können. Neue Softwaretypen,
bekannt als agorische offene Systeme, werden die Befehls- und
Kontrollsoftware, die aus dem industriellen Zeitalter übernommen wurde,
ersetzen. Diese ältere Software verteilte Rechenkapazitäten nach starren
Prioritäten, ähnlich wie die zentralen Planer bei Gosplan in der
ehemaligen Sowjetunion Güter nach starren Regeln auf Güterwagen
verteilen. Die neuen Systeme werden durch Algorithmen gesteuert, die
Marktmechanismen nachahmen, um Ressourcen effizienter durch einen
internen Bieterprozess zu verteilen, der die Wettbewerbsprozesse im
Gehirn nachahmt. Anstelle von riesigen Computermonopolen, die wichtige
Befehls- und Kontrollfunktionen ausführen, werden diese im neuen
Jahrtausend dezentralisiert.

Es gibt kein besseres Beispiel für die Widerstandsfähigkeit verteilter
Netzwerke im Vergleich zu Befehls- und Kontrollsystemen als das, was von
digitalem Equipment in seinem Forschungslabor in Palo Alto gegeben
wurde. Ein Ingenieur öffnete die Tür zu einem Schrank, in dem das
firmeneigene Computernetzwerk untergebracht war. Wie Kevin Kelly
berichtet, hat der Ingenieur dann dramatisch ein Kabel aus dem
Netzwerkgerät gerissen. „Das Netzwerk leitete die Datenströme um die
Störstelle herum und blieb dabei völlig unbeeindruckt.`` \footnote{Kelly,
  ebenda, S. 19.}

Das Informationszeitalter wird nicht nur den Wettbewerb ohne Anarchie im
Cyberspace erleichtern; es wird zwangsläufig zur Neugestaltung wichtiger
Systeme führen, die aus dem Industriezeitalter übriggeblieben sind. Eine
solche Umgestaltung ist essentiell, um sie weniger anfällig für Unheil
zu machen, das von überall und von jedem kommen könnte. Genauso wie das
Industriezeitalter unweigerlich zur Umgestaltung von Institutionen
führte, die aus dem Mittelalter übrig waren, wie Schulen und
Universitäten, so werden auch die verbliebenen Institutionen des
Industriezeitalters wahrscheinlich in verkleinerten Formen
weiterentwickelt, die der Logik der Mikrotechnologie entsprechen.

Der Schutzbedarf vor Banditen auf der „Autobahn der Informationen`` wird
eine weit verbreitete Annahme von Verschlüsselungsalgorithmen mit
öffentlichem und privatem Schlüssel erfordern. Diese ermöglichen bereits
jedem einzelnen Nutzer eines Computers, jede Nachricht sicherer zu
kodieren, als das Pentagon seine Startcodes nur eine Generation zuvor
hätte versiegeln können. Diese kraftvollen, unknackbaren
Verschlüsselungsformen sind notwendig, um finanzielle Transaktionen vor
Hackern und Dieben zu schützen.

Sie werden auch aus einem weiteren Grund notwendig sein. Private
Finanzinstitute und Zentralbanken werden unknackbare
Verschlüsselungsalgorithmen einführen, wenn sie erkennen, dass die
US-Regierung - und sie ist vielleicht nicht die einzige - in der Lage
ist, aktuelle Banksysteme und Computersysteme zu durchdringen und
buchstäblich ein Land in den Bankrott zu treiben oder das Bankkonto von
nahezu jedem Menschen überall zu leeren. Es gibt keinen technologischen
Grund, warum irgendein Individuum oder irgendein Land seine finanziellen
Einlagen oder Transaktionen dem Erbarmen der US-amerikanischen National
Security Agency oder den Nachfolgern des KGB oder irgendeiner ähnlichen
Organisation, legal oder illegal, ausliefern sollte.

Verschlüsselungsalgorithmen, die selbst von Regierungen nicht geknackt
werden können, sind keine Hirngespinste. Sie sind bereits als Shareware
im Internet verfügbar. Sobald Satellitensysteme in niedriger Umlaufbahn
voll funktionsfähig sind, wird es Einzelpersonen mittels
fortschrittlicher Computer und Antennen, die nicht größer sind als die
von Mobiltelefonen, möglich sein, weltweit zu kommunizieren, ohne
überhaupt mit dem Telefonsystem verbunden sein zu müssen. Es wird
genauso unmöglich sein für eine Regierung, den Cyberspace zu
monopolisieren, einem Bereich ohne physische Existenz, wie es für
mittelalterliche Ritter unmöglich gewesen wäre, Transaktionen im
industriellen Zeitalter auf einem schweren Schlachtross zu
kontrollieren.

\subsection{Schutz durch
Geheimhaltung}\label{schutz-durch-geheimhaltung}

Informationsgesellschaften werden riesige Ressourcen außerhalb des
Raubzugs stellen. Wenn der Cyberspace zunehmend Finanztransaktionen und
andere Formen des Handels beherbergt, werden die dort eingesetzten
Ressourcen mehr oder weniger immun gegen gewöhnliche Erpressungen und
Diebstähle sein. Daher werden Räuber nicht in der Lage sein, so große
Anteile von Ressourcen zu akkumulieren, wie sie es heute tun und wie sie
es während des größten Teils des zwanzigsten Jahrhunderts getan haben.

Unweigerlich wird daher der staatliche Schutz eines großen Teils des
Weltvermögens überflüssig werden. Die Regierung wird nicht besser in der
Lage sein, ein Bankguthaben im Cyberspace zu schützen, als Sie es sind.
Da die Regierung weniger notwendig sein wird, wird ihr relativer Preis
aus diesem alleinigen Grund wahrscheinlich fallen. Es gibt noch andere
Gründe.

Mit einem großen und wachsenden Anteil an Finanztransaktionen, die im
Cyberspace im neuen Jahrtausend stattfinden, werden Einzelpersonen die
Wahl haben, in welcher Rechtsordnung sie diese durchführen wollen. Dies
wird einen intensiven Wettbewerb schaffen, um die Dienstleistungen der
Regierung (die von ihr erhobenen Steuern) auf nicht-monopolistischer
Basis zu bepreisen. Dies ist revolutionär. Wie George Melloan im Wall
Street Journal argumentierte, ist die Institution, die den Kräften des
globalen Wettbewerbs am erfolgreichsten widerstanden hat, der
Wohlfahrtsstaat. „Eine Studie von Forschern der Wharton School und der
Australian National University diskutierte die Kräfte, die auf
Einkommensüberweisungen einwirken. Geoffrey Garrett und Deborah Mitchell
kamen zu dem Schluss, dass ‚es so gut wie keine Beweise dafür gibt, dass
eine erhöhte Marktintegration nach unten Druck auf ihre grundlegendsten
Wohlfahrtsprogramme ausgeübt hat.' Im Gegenteil, sie schreiben,
‚Regierungen haben unveränderlich auf eine erhöhte Integration in
internationale Märkte reagiert, indem sie Einkommensübertragungen
erhöhten.'\,`` \footnote{George Melloan, \emph{Welfare State Reform Is
  Mostly Mythological}, The Wall Street Journal. 14. Oktober 1996, S.
  A19.} Das Aufkommen der Cyberwirtschaft wird den Wohlfahrtsstaat
endlich einem echten Wettbewerb aussetzen. Es wird die Natur der
Souveränitäten verändern und Volkswirtschaften umgestalten, da das
Gleichgewicht zwischen Schutz und Erpressung stärker auf die Seite des
Schutzes schwingt, als es jemals zuvor der Fall war.

\setsubtitle{Das Aufkommen der Cyberökonomie}

\bookmarksetup{startatroot}

\chapter{DIE ÜBERWINDUNG DES
RÄUMLICHEN}\label{die-uxfcberwindung-des-ruxe4umlichen}

\begin{quote}
„Das eigentliche Problem ist die Kontrolle. Das Internet ist zu weit
verbreitet, um einfach so von einer einzelnen Regierung dominiert zu
werden. Durch die Schaffung einer reibungslosen globalen
Wirtschaftszone, die sowohl anti-souverän als auch unregulierbar ist,
stellt das Internet die bloße Idee eines Nationalstaates in Frage.``
\footnote{Perry Barlow, \emph{Thinking Locally, Acting Globally}, Time,
  15. Januar 1996, S. 57.} - John Perry Barlow
\end{quote}

Die Informationsautobahn ist zu einer der bekanntesten Metaphern aus den
Anfangstagen des digitalen Zeitalters geworden. Sie ist nicht nur
aufgrund ihrer Allgegenwärtigkeit bemerkenswert, sondern auch für das
weit verbreitete Missverständnis, das sie sich über die Cyberwirtschaft
offenbart. Eine Autobahn ist schließlich eine industrielle Version eines
Fußweges, ein Netzwerk für den physischen Transit von Menschen und
Gütern. Die Informationswirtschaft ist nicht wie eine Autobahn, eine
Eisenbahn oder eine Pipeline. Sie transportiert oder befördert keine
Informationen von einem Punkt zum anderen, so wie die Trans-Canada
Autobahn schwere Lastwagen von Alberta nach New Brunswick transportiert.
Was die Welt als „Informationsautobahn`` bezeichnet, ist nicht nur ein
Transitweg. Es ist das Ziel.

Der Cyberspace überwindet Räumlichkeit. Er beinhaltet nichts Geringeres
als das sofortige Teilen von Daten überall und nirgendwo gleichzeitig.
Die entstehende Informationswirtschaft basiert auf den Verbindungen, die
Millionen von Nutzern mit Millionen von Computern verknüpfen und immer
wieder neu verknüpfen. Ihr Wesen liegt in den neuen Möglichkeiten, die
aus diesen Verbindungen entstehen. Wie John Perry Barlow es ausdrückte:
„Was das Netz bietet, ist das Versprechen eines neuen sozialen Raums,
der global und anti-souverän ist und in dem jeder, egal wo, dem Rest der
Menschheit ohne Angst sagen kann, was er oder sie glaubt. Diese neuen
Medien sind eine Vorahnung der intellektuellen und wirtschaftlichen
Freiheit, die alle autoritären Mächte der Welt auslöschen könnte.``
\footnote{Ebenda.}

Der Cyberspace, ähnlich dem imaginären Reich der Götter Homers, ist ein
Bereich abseits der vertrauten irdischen Welt von Landwirtschaft und
Fabriken. Doch seine Auswirkungen werden nicht imaginär, sondern real
sein. In weit größerem Ausmaß, als viele derzeit verstehen, wird das
sofortige Teilen von Informationen wie ein Lösungsmittel wirken, das
große Institutionen auflöst. Es wird nicht nur die Logik der Gewalt
verändern, wie wir bereits untersucht haben; es wird auch die
Informations- und Transaktionskosten, die bestimmen, wie sich
Unternehmen organisieren und wie die Wirtschaft funktioniert, radikal
verändern. Wir erwarten, dass die Mikroverarbeitung die wirtschaftliche
Organisation der Welt verändern wird.

\begin{quote}
„Es ist heute mehr denn je in der Geschichte der Welt möglich, dass ein
Unternehmen von überall aus agieren und Ressourcen aus der ganzen Welt
nutzen kann, um ein Produkt herzustellen, das überall verkauft werden
kann.`` - Milton Friedman
\end{quote}

\section{DIE TYRANNEI DES ORTES}\label{die-tyrannei-des-ortes}

Die Tatsache, dass der erste Versuch des verblassenden industriellen
Zeitalters, die Informationsökonomie zu konzipieren, darin besteht, sie
sich in Form eines riesigen öffentlichen Bauprojekts vorzustellen,
zeigt, wie sehr unser Denken in den Paradigmen der Vergangenheit
verhaftet ist. Es ist eher so, als würden wir Bauern am Ende des
achtzehnten Jahrhunderts eine Fabrik als „einen Bauernhof mit einem
Dach`` beschreiben hören. Doch die Metapher der „Super-Autobahn`` ist
noch aufschlussreicher. Sie verrät auch, inwieweit wir der Tyrannei des
Ortes ausgeliefert sind. Selbst wenn die Technologie uns ermöglicht, die
Räumlichkeit zu überwinden, bekommt das Instrument unserer Befreiung
einen Spitznamen, der es als Weg von Ort zu Ort beschreibt. Wie Lachse,
die durch ihren Heiminstinkt geprägt sind, ist unser Bewusstsein immer
noch tief von Vorstellungen der Räumlichkeit gezeichnet.

Die gesamte Geschichte hindurch waren Volkswirtschaften an ein lokales
geografisches Gebiet gebunden. Die meisten Menschen, die vor dem
zwanzigsten Jahrhundert lebten, verbrachten ihre Tage de facto wie
Gefangene unter Hausarrest, selten mehr als ein paar Tage Fußmarsch von
dem Ort entfernt, an dem sie geboren wurden. Eine Reise über eine
beliebige Distanz war das Werk von Generationen. Nur gelegentlich löste
irgendeine Krise - Krieg, Pest, eine unerwünschte klimatische
Veränderung - eine breite Migration aus. Um Menschen dazu zu bringen,
ein elendes Dorf zu verlassen, bedurfte es etwas Spektakulärem und
Dringendem. Nichts Geringeres konnte die Menschen dazu bewegen, ihre
Habseligkeiten zusammenzupacken und auf der Suche nach einem besseren
Leben abzuwandern.

Bis vor kurzem wurden diejenigen, die außerhalb ihres eigenen Gebiets
nach Möglichkeiten suchten, oft berühmt. Nehmen wir zum Beispiel Marco
Polo, der immer noch dafür bekannt ist, dass er den eurasischen
Kontinent bereist hat, um den Hof des großen Khan zu besuchen. Er war zu
seiner Zeit die Ausnahme. Wenige andere Reisetagebücher der Vormoderne
sind erhalten geblieben. Unter den meistgelesenen sticht
\emph{Mandevilles Reisen}, geschrieben auf Französisch im Jahr 1357,
hervor. Interessant ist, dass es wahrscheinlich von jemandem verfasst
wurde, der Europa nie verlassen hat. Mandeville vermittelt bezaubernde
und oft fantastische Details über das Leben in aller Welt,
einschließlich der Behauptung, dass viele Äthiopier nur einen Fuß haben:
„...der Fuß ist so groß, dass er den ganzen Körper vor der Sonne
abschirmt, wenn sie sich ausruhen wollen.`` \footnote{M. C. Seymour,
  ed., Mandeville: \emph{Travels} (Oxford: Oxford University Press,
  1968), S. 122.} Offensichtlich wussten nur wenige von Mandevilles
Zeitgenossen, die seine beliebte Geschichte lasen, dass sein
äthiopischer „Bigfoot`` nicht existierte.

Erst mit Beginn des modernen Zeitalters, durch die Entdeckungsreisen am
Ende des 15. Jahrhunderts, gab es dauerhafte Kontakte zwischen den
Kontinenten. Wagemutige Kapitäne wie Christoph Kolumbus und Vasco da
Gama, die sich daran machten, den Gewürzhandel zu erobern, waren
außergewöhnlich genug, um für den größten Teil von fünf Jahrhunderten in
jedem gebildeten Haushalt in Erinnerung zu bleiben.

Von der Entstehung der Landwirtschaft bis in die jüngsten Generationen
war das Leben durch seine Unbeweglichkeit geprägt. Dies ist heute fast
vergessen, insbesondere in den europäischen Siedlungskolonien der „Neuen
Welt``, wo Bewegungen flüssiger sind und jeder dazu neigt, seine
Perspektive aus dem Blickwinkel eines Einwanderers zu betrachten. Ein
wesentliches Thema im Grundschulunterricht in Nordamerika ist, dass die
Kolonisten aus Europa kamen, um Freiheit und Chancen zu suchen, was der
Wahrheit entspricht. Was jedoch selten erzählt wird, ist, wie
widerwillig die meisten Menschen die Reise angetreten haben, selbst wenn
sie zu Hause Armut gegenüberstanden. Die wenigen, die auswanderten,
erlitten nach heutigen Begriffen unvorstellbare Strapazen, um sich zu
etablieren. Nur die unternehmungslustigsten oder verzweifeltsten Armen
wagten es. Mitte des 17. Jahrhunderts revoltierten Insassen in
Bridewell, dem berüchtigten Zuchthaus in London, um „ihren Unwillen zum
Ausdruck zu bringen, nach Virginia zu gehen``.\footnote{R. C. Johnson,
  \emph{The Transportation of Vagrant Children from London to Virginia,
  1618-1622}, in H. S. Reinmuth, ed., Early Stuart Studies (Minneapolis:
  University of Minnesota Press, 1970), S. \textasciitilde43-44, zitiert
  aus Jutte, ebenda, S. 168.} Im Jahr 1720 kam es zu Tumulten auf den
Straßen von Paris, um Obdachlose, Diebe und Mörder zu befreien, die zur
Deportation nach Louisiana vorgesehen waren.

\subsection{Enge Horizonte}\label{enge-horizonte}

Die physischen Schwierigkeiten der Kommunikation und des Transports,
ergänzt in den meisten Zeiten und Orten durch begrenzte
Sprachkenntnisse, machten den Fokus menschlichen Handelns eng und lokal.
Noch im frühen zwanzigsten Jahrhundert war es üblich, chinesische Dörfer
zu finden, die nur fünf Meilen voneinander entfernt lagen und trotz
Küstennähe gegenseitig unverständliche Dialekte sprachen. Die lokale
Organisation fast aller Wirtschaften legte eine Strafe für schmale
Märkte und verpasste Chancen auf. Die Faktorkosten wurden aufgrund
begrenzten Wettbewerbs hoch gehalten. Der Zugang zu spezialisierten
Fähigkeiten war minimal. Mit Einkommen, so niedrig, dass sie am Rand des
Verderbens kratzten, und ohne Zugang zu externem Kapital oder
effizienten Versicherungsmärkten, waren Kleinbauern in großen Teilen der
Welt in Armut gefangen. Wir haben einige der Schwierigkeiten diskutiert,
mit denen Bauern aufgrund des eingeschränkten Dorflebens konfrontiert
waren. Selbst jetzt, während wir schreiben, kämpfen mindestens eine
Milliarde Menschen, meist in Asien und Afrika, um mit weniger als einem
Dollar pro Tag zu überleben.

\section{„JEDE POLITIK IST LOKAL``}\label{jede-politik-ist-lokal}

In einem größeren Ausmaß als allgemein realisiert wird, hat die
Unbeweglichkeit von Menschen und ihren Vermögenswerten die Art und Weise
geprägt, wie wir die Welt sehen. Selbst diejenigen, die am meisten
bereit zu sein scheinen, zuzustimmen, dass die Erde am Ende des
zwanzigsten Jahrhunderts ein kleiner Ort ist, denken weiterhin in
Begriffen, die von antiquierten Konzepten der Industriepolitik
eingeschränkt sind. Dies wird durch einen Slogan unterstrichen, der in
den 1980er Jahren unter Umweltschützern populär wurde: „Denke global,
handle lokal``. Es ist eine Aufforderung, die die Logik der Politik
widerspiegelt, eine Logik, die schon immer auf lokalen Machtvorteilen
basierte.

Die lokale Denkweise wurde durch die Megapolitik aller vergangenen
Gesellschaften bestimmt. Alle topographischen Merkmale, die als Barriere
oder Förderer der Ausübung von Macht dienen, sind lokal. Jeder Fluss,
jeder Berg, jede Insel ist lokal. Das Klima ist lokal. Temperatur,
Regenfall und Anbaubedingungen variieren, wenn man auf und über einen
Berg geht. Jeder Mikroorganismus, der sich ausbreitet, tut dies an einem
bestimmten und nicht an einem beliebigen anderen Ort.

Kein Wunder also, dass die Tyrannei des Ortes unsere Vorstellungen davon
durchdringt, wie eine Gesellschaft zu organisieren ist und wie sie
funktionieren muss. Die Machtvorteile, die der einen oder anderen Gruppe
ein lokales Monopol auf Gewalt eingeräumt haben, sind immer irgendwo
entstanden und verblassten entlang der megapolitischen Ränder, an denen
die Grenzen gezogen werden. Das ist der Grund, warum es noch nie eine
Weltregierung gegeben hat.

Die Bedeutung des Ortes für die Ausübung von Macht wurde selten explizit
thematisiert, aber einige Befürworter der erzwungenen Umverteilung der
Früchte menschlichen Handelns begannen schon in den 1930er Jahren, den
sinkenden Einfluss des Ortes zu spüren. Sie erkannten in der modernen
Verkehrstechnologie eine Teilung des sozialen Raums zwischen den
Gutverdienern und den Armen. Diese Furcht wurde von John Dos Passos in
„The Big Money`` erfasst: „Der ‚Vagabund' sitzt am Straßenrand, pleite
und hungrig. Über ihm fliegt ein Transkontinentalflugzeug voller
gutbezahlter Führungskräfte. Die Oberschicht hat die Luft, die
Unterschicht die Straße erobert: Zwischen ihnen gibt es keine Verbindung
mehr, es sind zwei Nationen.`` \footnote{John Dos Passos, \emph{The Big
  Money} (New York: Harcourt, Brace \& Co., 1936).} Das ist eine andere
Art zu sagen, dass verbesserte Verkehrsmittel die Möglichkeit zur
Erpressung einfach dadurch verringern, dass sie die Auswahl an Orten
erhöhten, an denen erfolgreiche Personen sein können. Sicherlich war der
Vagabund auf der Straße unten in keiner Position, um von denen über ihm
fliegenden Personen eine milde Gabe zu verlangen. Die Tendenzen, die Dos
Passos vor sechzig Jahren beobachtet hat, haben sich nur noch stärker
ausgeprägt.

\subsection{Massenverkehrsmittel}\label{massenverkehrsmittel}

Im Jahr 1995 überquerten täglich eine Million Menschen irgendwo auf der
Welt Grenzen. Dies stellt eine erstaunliche Veränderung gegenüber der
Vergangenheit dar. Vor dem zwanzigsten Jahrhundert war das Reisen so
selten, dass die meisten Grenzen einfach als Grenzen und nicht als
Hindernisse für den Transit angesehen wurden. Pässe waren unbekannt. Die
Entwicklung von Ozeandampfern, Zügen und anderen verbesserten
Transportmitteln hat die Bewegung dramatisch erhöht. Doch diese Bewegung
wurde stärker von Staaten reguliert, deren Macht durch die gleichen
Verbesserungen beim Transport und der Kommunikation erhöht wurde, die
das zivile Reisen billiger und einfacher machten. Die Einführung von
Filmen und vor allem des Fernsehens hat ebenfalls viel dazu beigetragen,
den Horizont zu erweitern und Reisen und Einwanderung anzuregen. Aber
bis jetzt sind die grundlegenden Annahmen der sozialen und
wirtschaftlichen Organisation in der Räumlichkeit verankert geblieben.

\begin{quote}
„... um dieses Versagen der Nerven zu vermeiden, für den die Geschichte
so gnadenlos bestraft. Wir müssen den Mut haben, alle technischen
Extrapolationen bis zu ihrem logischen Schluss zu verfolgen.``
\footnote{Clarke, ebenda, S. 29.} - Arthur C. Clarke
\end{quote}

\section{DER FEHLER VON MINIMALEN
ERWARTUNGEN}\label{der-fehler-von-minimalen-erwartungen}

Die geographische Fessel der Vorstellungskraft ist immer noch so eng,
dass einige Experten, die das Internet im Jahr 1995 untersuchten, zu dem
Schluss gekommen sind, dass es kaum kommerzielles Potenzial hat und kaum
Bedeutung, außer als elektronisches Medium für Chats und eine
Vertriebsstelle für Pornographie. Die vielen Zweifler an der
wirtschaftlichen Bedeutung des Cyberspace sind die Ewiggestrigen des
Informationszeitalters. Ihre Selbstzufriedenheit steht der des
britischen Establishments gegenüber, das sich in den 1930er Jahren mit
dem Niedergang des Reiches konfrontiert sah. Immer wenn sich Eliten
bedroht sehen, ist ihre erste Reaktion Verleugnung. Dies zeigt sich in
der frommen Hoffnung, dass das Internet nie viel bedeuten wird, manchmal
sogar von Autoritäten unterstützt, die es eigentlich besser wissen
sollten. Wir haben zuvor auf die Arbeit von David Kline und Daniel
Burstein, \emph{Road Warriors: Dreams and Nightmares Along the
Information Highway}, verwiesen. Ihre Abwertung des ökonomischen
Potenzials des Netzes ist ein weiterer Beweis dafür, dass technisches
Wissen nicht gleichbedeutend mit dem Verständnis über die Konsequenzen
der Technologie ist.\footnote{Zitiert aus Kline und Burstein, ebenda, S.
  \textasciitilde05.}

Selbst die technisch versiertesten Beobachter der Vergangenheit haben
oft die Auswirkungen neuer Technologien nicht erkannt. Ein im Jahr 1878
einberufener britischer Parlamentsausschuss, der die Perspektiven für
Thomas Edisons Glühbirne betrachten sollte, bezeichnete Edisons Ideen
als „gut genug für unsere transatlantischen Freunde, ... aber unwürdig
der Aufmerksamkeit von praktischen oder wissenschaftlichen
Männern``.\footnote{Clarke, ebenda, S. 20.} Thomas Edison selbst war ein
Mann großer Visionen, aber er dachte, dass der von ihm erfundene
Phonograph hauptsächlich von Geschäftsleuten zum Diktieren genutzt
würde. Kurz bevor die Gebrüder Wright bewiesen, dass Flugzeuge fliegen
können, demonstrierte der renommierte amerikanische Astronom Simon
Newcomb, warum es unmöglich ist, dass Flugzeuge, die schwerer sind als
Luft, fliegen können. Er folgerte: „Die Demonstration, dass keine
mögliche Kombination bekannter Substanzen, bekannter Formen von
Maschinen und bekannter Formen von Kraft in einer praktischen Maschine
vereint werden kann, mit der Menschen lange Strecken durch die Luft
fliegen können, scheint dem Autor so vollständig zu sein, wie es für die
Demonstration einer physikalischen Tatsache möglich ist``.\footnote{Ebenda.}
Kurz nachdem Flugzeuge zu fliegen begannen, erklärte ein weiterer
renommierter Astronom, William H. Pickering der Öffentlichkeit, warum
kommerzielle Reisen niemals möglich sein würden: „Die breite
Öffentlichkeit stellt sich oft gigantische Flugmaschinen vor, die über
den Atlantik rasen und unzählige Passagiere transportieren, analog zu
unseren modernen Dampfschiffen. ... Es ist klar, dass wir mit unseren
derzeitigen Geräten, weder mit unseren Lokomotiven noch mit unseren
Autos, in Bezug auf Geschwindigkeit konkurrieren können``.\footnote{Ebenda,
  S.21.} Wir haben zuvor bereits an eine andere extrem ungenaue
Prophezeiung über das Potenzial einer neuen Technologie erinnert - die
Prognose des Autoherstellers Mercedes zu Beginn des 20. Jahrhunderts,
dass es weltweit nie mehr als eine Million Autos geben würde. Sie
wussten mehr über Autos als fast jeder andere, aber sie hätten sich in
Bezug auf die Auswirkung von Autos auf die Gesellschaft nicht stärker
irren können.

Angesichts dieser Tradition ahnungsloser Missverständnisse ist es kaum
überraschend, dass viele Beobachter nur langsam die wichtigsten
Implikationen der neuen Informationstechnologie erfassen - die Tatsache,
dass sie die Tyrannei der Räumlichkeit überwindet. Die neue Technologie
schafft zum ersten Mal eine unendliche, nicht irdische Sphäre für
wirtschaftliche Aktivitäten. Sie eröffnet die Möglichkeit, die neuen
Grenzen der Cyber-Ökonomie zu erkunden, „global zu denken und global zu
handeln``. Dieses Kapitel erklärt, warum.

\section{JENSEITS DER RÄUMLICHKEIT}\label{jenseits-der-ruxe4umlichkeit}

Das wiederkehrende Thema der Unterschätzung technologischer
Entwicklungen ist weiterhin präsent, diesmal mit Schwerpunkt auf
Informationstechnologie. Die rapide Verarbeitung und Nutzung von
Informationen ersetzt physische Produkte zunehmend als wichtigste
Einnahmequelle. Die Informationstechnologie entkoppelt das Potenzial,
Einkommen zu erzielen, von einer Residenz an einem bestimmten
geographischen Ort. Da ein immer größerer Teil des Produkt- und
Dienstleistungswertes durch hinzugefügte Ideen und Wissen geschaffen
wird, wird ein immer kleinerer Anteil des Mehrwerts in lokalen
Jurisdiktionen erfasst werden können. Da Ideen überall formuliert und
weltweit mit Lichtgeschwindigkeit übertragen werden können, wird die
Informationswirtschaft sich zwangsläufig drastisch von der Wirtschaft
des Fabrikzeitalters unterscheiden.

Wir würden den Kritikern zugestehen, dass eine Aufzählung der Aufgaben,
die man 1996 durch das Internet hätte erledigen können, banal erscheinen
mag. Schließlich ist es nicht besonders revolutionär, einen Artikel über
Gartenarbeit im Netz zu lesen oder aus der Ferne eine Kiste Wein zu
kaufen. Allerdings sollte das Potenzial der Cyberökonomie nicht
ausschließlich nach ihren frühen Anfängen beurteilt werden, genau so
wenig wie das Potenzial des Automobils, die Gesellschaft zu verändern,
nach dem beurteilt werden kann, was man 1900 sehen konnte. Wir gehen
davon aus, dass die Cyberökonomie sich durch mehrere Stufen entwickeln
wird.

\begin{enumerate}
\def\labelenumi{\arabic{enumi}.}
\item
  Die primitivsten Manifestationen des Informationszeitalters nutzen das
  Netz einfach als Informationsmedium, um ansonsten gewöhnliche
  industrielle Transaktionen zu erleichtern. In diesem Stadium ist das
  Netz nicht mehr als ein exotisches Liefersystem für Kataloge. Virtual
  Vineyards zum Beispiel, einer der ersten Cyberhändler, verkauft
  einfach Wein von einer Seite im World Wide Web. Solche Transaktionen
  sind noch nicht direkt subversiv gegenüber den alten Institutionen.
  Sie nutzen Industriewährungen und finden innerhalb erkennbarer
  Zuständigkeiten statt. Eine solche Nutzung des Internets hat wenig
  megapolitischen Einfluss.
\item
  Ein mittleres Stadium des Internet-Handels wird die
  Informationstechnologie auf eine Weise einsetzen, die im
  Industriezeitalter unmöglich gewesen wären, wie etwa in der
  Fernbuchhaltung oder bei medizinischen Fern-Diagnosen. Weitere
  Beispiele für diese neuen Anwendungen der erweiterten Rechenleistung
  werden unten ausgeführt. Die zweite Stufe des Netz-Handels wird noch
  innerhalb des alten institutionellen Rahmens funktionieren, nationale
  Währungen verwenden und sich der Zuständigkeit der Nationalstaaten
  unterwerfen. Die Händler, die das Netz für den Verkauf nutzen, werden
  es noch nicht zur Verbuchung ihrer Gewinne nutzen, sondern nur um
  Umsätze zu erzielen. Diese auf Internettransaktionen basierenden
  Gewinne werden weiterhin besteuert werden.
\item
  Ein fortgeschritteneres Stadium wird den Übergang zu echter
  Cyberökonomie kennzeichnen. Transaktionen werden nicht nur über das
  Netz stattfinden, sondern auch die Zuständigkeit von Nationalstaaten
  überwinden. Die Bezahlung wird in Cyberwährung erfolgen. Gewinne
  werden in Cyber-Banken verbucht. Investitionen werden bei
  Cyber-Brokern gemacht. Viele Transaktionen werden nicht besteuert
  werden. In diesem Stadium wird die Cyberökonomie beginnen, erhebliche
  megapolitische Konsequenzen zu haben, die wir bereits skizziert haben.
  Die Macht von Regierungen über traditionelle wirtschaftliche Bereiche
  wird durch die neue Logik des Netzes transformiert werden.
  Exterritoriale Regulierungsgewalt wird zusammenbrechen. Die
  Zuständigkeiten werden sich auflösen. Die Struktur von Unternehmen
  wird sich verändern und ebenso die Art und Weise der Arbeit und der
  Beschäftigung. Dieser Überblick über die Phasen der
  Informationsrevolution ist nur der kargste Entwurf dessen, was das
  weitreichendste wirtschaftliche Transformationsereignis sein könnte.
\end{enumerate}

\section{DIE GLOBALISIERUNG DES
HANDELS}\label{die-globalisierung-des-handels}

Im Informationszeitalter werden die meisten aktuellen rechtlichen
Vorteile durch Technologie rasch erodieren. Neue Arten von Vorteilen
werden entstehen. Die sinkenden Kommunikationskosten haben bereits den
Bedarf an Nähe als notwendige Bedingung für Geschäftstätigkeiten
reduziert. 1946 war es technisch möglich, dass ein Investor in London
einen Auftrag an einen Makler in New York stellte. Aber nur die größte
und überzeugendste Transaktion hätte dies gerechtfertigt: Ein
dreiminütiges Telefongespräch zwischen New York und London kostete 650
Dollar. Heute kostet es 0,91 Dollar. Der Preis für ein
interkontinentales Telefonat ist in einem halben Jahrhundert um mehr als
99 Prozent gesunken.

\subsection{Konvergente Kommunikation}\label{konvergente-kommunikation}

Bald könnte der Unterschied zwischen einem interkontinentalen Chat und
einem lokalen Anruf minimal sein. Ebenso könnten die Unterschiede
zwischen Ihrem Telefon, Ihrem Computer und Ihrem Fernseher gering sein.
Alle werden interaktive Kommunikationsgeräte sein, die eher anhand
ergonomischer statt funktioneller Merkmale unterschieden werden können.
Sie werden in der Lage sein, ein Gespräch über das Internet mit
Mikrofonen und Lautsprechern auf Ihrem PC zu führen. Oder einen Film
anzusehen. Sie werden in der Lage sein, Ihrem Fernseher zu antworten und
über das Netzwerk, das von den Unterhaltungsmedien des Fernsehens
bereitgestellt wird, riesige Datenmengen zu übertragen. Da die
industrielle Unterscheidung zwischen verschiedenen Kommunikationsformen
abbricht und die Kosten sinken, werden immer mehr Dienste Ihnen nach
Nutzungszeit berechnen und nicht nach dem Ziel Ihrer Nachrichten.
Gespräche oder Datenübertragungen an jedem Ort der Welt werden kaum mehr
kosten als ein lokaler Anruf in den meisten Ländern im Jahr 1985.

\subsection{Internet ohne Kabel}\label{internet-ohne-kabel}

Satelliten in einer niedrigen Umlaufbahn und andere drahtlose
Technologien werden Daten direkt an einen Piepser in Ihrer Tasche, einen
tragbaren Computer oder eine Arbeitsstation übertragen, ohne dass ein
lokales Telefon- oder Fernsehkabelsystem erforderlich ist. Kurz gesagt,
das Internet wird kabellos sein. Die ersten Schritte in diese Richtung
dürften wegen der relativ langsamen Datenübertragungsgeschwindigkeit in
den frühen drahtlosen Medien und den Schwierigkeiten, schwache Signale
von Teilnehmergeräten zu „hören``, von denen einige mobil und
batteriebetrieben sein werden, zögerlich sein. Nichtsdestotrotz werden
diese technischen Probleme angegangen und gelöst werden.

\subsection{Geschäft ohne Grenzen}\label{geschuxe4ft-ohne-grenzen}

Die fortgesetzte Steigerung der Rechenleistung wird zu einer
verbesserten Kompressionstechnologie führen, die den Datenfluss
beschleunigt. Die weit verbreitete Nutzung bestehender
Verschlüsselungsalgorithmen zwischen öffentlichen und privaten
Schlüsseln wird es Anbietern, wie zum Beispiel Satellitensystemen,
ermöglichen, die Abrechnungsfunktion direkt in den Service zu
integrieren, was die Kosten senkt. Gleichzeitig mit diesem Dienst werden
die Anbieter in der Lage sein, die auf den PCs gespeicherten Konten zu
belasten, ähnlich wie France Telecom die in den Pariser Telefonzellen
verwendeten „Smart Cards`` belastet.

\subsection{Das Telefon wird zur Bank}\label{das-telefon-wird-zur-bank}

Der Unterschied ist, dass Sie in naher Zukunft in der Lage sein werden,
mit allen Arten von Transaktionen Guthaben auf Ihrem Konto anzusammeln
und Ihre Telefonbox mit sich zu führen. Ihr PC wird die Filiale Ihrer
Bank und der globalen Geldvermittlungsstelle sein, ebenso wie das
Pendant zum Pariser Kiosk, an dem Sie Ihre anonyme Telefonkarte kaufen.
Und wie die 150 „Smart Card``-Bezahltelefone, die bei gewaltsamer
Öffnung mit einem Brecheisen für Diebe nutzlos sind, könnte Ihr Computer
nur von jemandem ausgeraubt werden, der in der Lage ist, einen
hochentwickelten Computercode zu knacken oder zu manipulieren. Das würde
viele Grobmotoriker, die bloß mit einer Brechstange umgehen können,
ausschließen. Bei richtiger Verschlüsselung könnte nichts in Ihrem
Computer entschlüsselt oder missbraucht werden.

Bis zur Jahrtausendwende werden Sie in der Lage sein, Geschäfte fast
überall nördlich der Antarktis abzuwickeln. Überall dort, wo ein
verkabeltes oder digitales Mobiltelefon verfügbar ist. Überall dort, wo
interaktive Kabel-Fernsehsysteme genutzt werden. Überall dort, wo ein
Satellit über unseren Köpfen schwebt oder andere drahtlose
Übertragungssysteme installiert sind. Sie werden in der Lage sein, Daten
zu übermitteln, zu sprechen und sich mittels virtueller Realität über
Grenzen und Gegebenheiten hinweg zu bewegen. Telefonnummern, die den Ort
des Anrufers durch Vorwahlen kennzeichnen, werden wahrscheinlich von
universellen Zugangsnummern abgelöst, die die Person, mit der Sie
kommunizieren möchten, überall auf dem Planeten erreichen können.

\subsection{Chinesisch verstehen}\label{chinesisch-verstehen}

Sie werden nicht nur in der Lage sein zu sprechen und ein Fax zu senden.
Mit der Zeit könnten Sie einen mehrjährigen Lernprozess verkürzen und
mit einem Vorarbeiter in Shanghai auf Chinesisch sprechen. Es wird dann
nicht mehr so wichtig sein, dass Sie seine Sprache oder seinen Dialekt
nicht sprechen. Seine Worte mögen Chinesisch sein, aber Sie werden
hören, wie sie etwa ins Deutsche übersetzt werden. Er wird Ihre
Unterhaltung auf Chinesisch hören. Mit der Zeit wird die Fähigkeit,
sofortige Übersetzungen zu nutzen, den Wettbewerb in Regionen, in denen
Sprachen und Redewendungen bisher bedeutende Hindernisse waren, deutlich
erhöhen. Wenn das passiert, wird es wenig oder gar keine Rolle mehr
spielen, ob die chinesische Regierung sich wünscht, dass der Anruf
getätigt wird oder nicht.

\subsection{Individuell angepasste
Medien}\label{individuell-angepasste-medien}

Während die Welt immer näher zusammenrückt, haben Sie mehr Möglichkeiten
als jemals zuvor, Ihren speziellen Platz darin zu gestalten. Selbst die
Informationen, die Sie regelmäßig aus den Medien erhalten, werden
Informationen Ihrer Wahl sein. Die Massenmedien werden sich zu
individualisierten Medien wandeln. Wenn Sie vor allem an Schach
interessiert sind oder ein begeisterter Katzenliebhaber sind, können Sie
Ihre Abendnachrichtensendung so programmieren, dass sie für Sie wichtige
Informationen über Katzen oder Schach enthält. Sie werden nicht länger
auf die Gnade von Kai Gniffke oder der Tagesschau angewiesen sein, um
die Nachrichten zu sehen, die Sie sehen möchten. Sie werden in der Lage
sein, Nachrichten auszuwählen, die nach Ihren Anweisungen
zusammengestellt und bearbeitet werden.

\subsection{Von Massen- zu
Individualfertigung}\label{von-massen--zu-individualfertigung}

Wenn die Nachrichtenlage ruhig ist, können Sie auf einen virtuellen
Katalog im World Wide Web zugreifen. Wenn Sie eine Hose sehen, die Ihnen
einigermaßen gefällt, können Sie beim Bestellen die Weite des
Hosenbeinabschlusses anpassen. Sie wird von Robotern in Malaysia
individuell zugeschnitten, die auf Ihrem Computer eingescannte und über
das Netz übertragene Fotos verwenden und sie an Ihren Körper anpassen.

\subsection{Cyber-Makler}\label{cyber-makler}

Sie werden in der Lage sein, Cyber-Geld für Investitionen sowie zur
Bezahlung von Dienstleistungen und Produkten zu nutzen. Wenn Sie in
einer Rechtsordnung wie den Vereinigten Staaten leben, die Ihre
Investmentoptionen stark reguliert, können Sie Ihre Aktivitäten in einer
Rechtsordnung ansiedeln, die Ihnen die Freiheit gibt, eine vollständige
Palette von Investmentoptionen zu verfolgen. Es spielt keine Rolle, ob
Sie in Cleveland oder Belo Horizonte leben, Sie können Ihr
Investmentgeschäft in Bermuda, auf den Cayman-Inseln, in Rio de Janeiro
oder Buenos Aires abwickeln. Wo auch immer Sie sich befinden, der
Gebrauch digitaler Ressourcen wird sich mit der Entwicklung der
Cyber-Wirtschaft ausweiten. Sie werden in der Lage sein, Expertensysteme
zu nutzen, um Ihre Investitionen auszuwählen, und Cyber-Buchhalter und
-Buchprüfer, um den Fortschritt Ihres Portfolios in Echtzeit zu
überwachen.

\subsection{Virtuelle Kultur}\label{virtuelle-kultur}

Wenn Sie nicht gerade Ihre Gewinn- und Verlustdaten prüfen, können Sie
einen virtuellen Besuch im Louvre machen. Für Ihre Reise müssen Sie
möglicherweise eine Lizenzgebühr, die dem Drittel eines Cents
entspricht, an Bill Gates oder eine andere Person mit gleicher
Voraussicht zahlen, die die Rechte zur virtuellen Besichtigung des
Museums erworben hat. Während Sie sich fragen, ob die Mona Lisa Probleme
mit ihren Zähnen hatte, könnte Ihr Computer die Übersetzung von S. I.
Hsiungs „The Romance of the Western Chamber`` herunterladen. Zu einer
von Ihnen ausgewählten Zeit wird Ihr persönliches Kommunikationssystem
den Text vorlesen wie ein mittelalterlicher Barde.
Multitasking-Programme ermöglichen es Ihnen, viele Funktionen
gleichzeitig auszuführen.

\subsection{Rechtssysteme im Netz
einkaufen}\label{rechtssysteme-im-netz-einkaufen}

Wenn Sie sich von Ihrer Dosis an Klassikern inspirieren lassen, können
Sie ein virtuelles Unternehmen gründen, um dramatische Inszenierungen
berühmter Literatur zu vermarkten, die auf dreidimensionalen
Netzhautbildschirmen zu sehen sind. Anstatt in die Luft projiziert zu
werden, werden die Bilder direkt auf die Netzhäute der Zuschauer mit
niedrigenergetischen Lasern projiziert, die fünfzigtausend Mal pro
Sekunde fluktuieren. Diese Technologie, die bereits von MicroVision in
Seattle, Washington, entwickelt wird, wird vielen Personen, die
rechtlich blind sind, das Sehen ermöglichen. Bevor Sie das Projekt in
Angriff nehmen, könnten Sie Ihren digitalen Assistenten beauftragen, die
aktuellen Vertragsangebote zum Schutz von Produktionsstätten in
Malaysia, China, Peru, Brasilien und der Tschechischen Republik zu
prüfen. Wenn Sie einen Standort ausgewählt haben, können Sie dank der
St.~Georges Trust Company Ihre Firma innerhalb von einer Stunde auf den
Bahamas gründen lassen. Ihre Anweisungen werden alle flüssigen
Vermögenswerte der Firma auf ein Cyberkonto in einer Cyberbank
übertragen, die gleichzeitig in Neufundland, den Cayman-Inseln, Uruguay,
Argentinien und Liechtenstein ansässig ist. Wenn eines dieser
Rechtssysteme versucht, Ihnen die Betriebserlaubnis zu entziehen oder
die Vermögenswerte der Einleger zu beschlagnahmen, werden die
Vermögenswerte automatisch mit Lichtgeschwindigkeit in ein anderes
Rechtssystem übertragen.

\section{QUALITATIVE FORTSCHRITTE}\label{qualitative-fortschritte}

Viele der Transaktionen, die Sie bald im Cyberspace durchführen können,
wären im Industriezeitalter unmöglich gewesen, und das nicht nur, weil
sie eine Sprachbarriere überwinden. Den digitalen Assistenten zu
beauftragen, nicht übersetzte Artikel in ungarischen wissenschaftlichen
Journals zu finden, ist qualitativ anders als mit einem Bibliothekar zu
sprechen. Einem Oxford-Tutorium aus einer Entfernung von fünftausend
Meilen beizuwohnen ist nicht dasselbe wie das Tutorium zu besuchen, wenn
man nur sechs Meilen von Carfax entfernt schläft. Und das Spielen am
Roulette-Rad im Hotel de Paris, Monte Carlo, ist eine neue Erfahrung,
wenn man es auf einer Party in Punta del Este, Uruguay, durch Virtual
Reality tun kann.

\subsection{Ein Cyberbesuch beim
Cyberarzt}\label{ein-cyberbesuch-beim-cyberarzt}

Innerhalb kürzester Zeit, schneller als viele Experten es derzeit für
möglich halten, werden Tätigkeiten in die Cyberwirtschaft übergehen, die
Technologien auf neuartige Weise kombinieren, um die Tyrannei des Ortes
und die veralteten Institutionen der Industriewirtschaft zu überwinden.
Wenn Sie Magenschmerzen haben, können Sie schon bald eine digitale
Ärztin konsultieren, ein Expertensystem mit enzyklopädischem Wissen über
Symptome, Krankheiten und Gegenmittel. Es wird auf Ihre
Krankengeschichte in verschlüsselter Form zugreifen, fragen, ob Ihre
Schmerzen nach dem Essen oder vor den Mahlzeiten auftreten. Ob sie
stechend oder dumpf, anhaltend oder episodisch sind. Welche Fragen auch
immer Ärzte stellen, der digitale Arzt wird sie stellen. Er kann
feststellen, dass Sie zu viel Wein trinken, oder nicht genug. Sie
könnten an einen Cyberspezialisten überwiesen werden. Wenn Sie eine
Operation benötigen, könnte ein Cyberchirurg in Bermuda die Operation
mit der Hilfe von spezialisierten Geräten, die Mikroeinschnitte
durchführen, aus der Ferne durchführen.

\subsection{Lebenswichtige
Informationsverarbeitung}\label{lebenswichtige-informationsverarbeitung}

Das mag alles wie Science-Fiction klingen. Aber viele der Komponenten
der Cyberchirurgie sind bereits vorhanden. Andere werden bereits
funktionsfähig sein, während Sie dieses Buch lesen. General Electric hat
eine neue Magnetresonanztherapiemaschine (MRT) in fünfzehn
Krankenhäusern auf der ganzen Welt eingeführt. Es wird erwartet, dass
die Maschine eine dreijährige Forschungs- und Entwicklungsphase
durchläuft, aber danach wird sie sich wahrscheinlich rapide ausbreiten
und zur Norm für viele Arten von Operationen werden. Es ist ein
Beispiel, aber ein gutes, dafür wie die Technologie die Gesellschaft
verändert.

Die meisten von uns sind mit Magnetresonanztomographie (MRT) -Maschinen
vertraut, bei denen Magnetresonanzverfahren verwendet werden, um Ärzten
weiche Gewebebilder zu diagnostischen Zwecken zur Verfügung zu stellen.
Sie liefern bessere Bilder von weichem Gewebe als Röntgen oder
Ultraschall und sind zu einem wesentlichen Teil moderner
Diagnosetechniken geworden, insbesondere bei Krebserkrankungen. Sie
haben jedoch derzeit zwei wesentliche Einschränkungen. Das Röhrensystem
ermöglicht keinen freien Zugang zum Patienten; die Maschinen sind von
begrenzter Leistung.

\subsection{Cyberchirurgie}\label{cyberchirurgie}

General Electric hat Magnetresonanzmaschinen so überarbeitet, dass sie
sowohl für die Behandlung als auch für die Diagnose eingesetzt werden
können. Die Leistung wurde um das Fünffache erhöht. Die Röhre wurde im
Grunde genommen in zwei Teile geschnitten, sodass der Patient nun
zwischen zwei donutförmigen Einheiten liegt und nicht vollständig
eingeschlossen wird. Anstatt ein Bild aufzunehmen und dann in Anbetracht
dieses Bildes eine Operation durchzuführen, wird der Chirurg das, was er
tut, sehen können, während er es tut. Die MRT wird mit nicht-invasiver
oder weniger invasiver Chirurgie unter Einsatz von Mikrotechniken
kombiniert werden. Anstatt große Einschnitte mit dem Skalpell zu machen,
wird der Chirurg Mikroeinschnitte mit Sonden vornehmen und sehen können,
was die Sonden während der Operation offenbaren. Er führt die Operation
aus dem Bild heraus durch, anstatt direkt in den Körper zu schauen. Im
Prinzip werden die Sonden aus der Ferne bedienbar sein. Sie werden in
der Lage sein, Tumore mit Laser- oder Kryo-Heiz- oder Gefriergeräten von
großer Präzision zu zerstören.

Dies wird Operationen ermöglichen, die heute noch unmöglich sind,
insbesondere in der Neurochirurgie, wo Tumore oft sehr nahe an
lebenswichtigen Gehirnregionen liegen. Es wird auch wiederholte
Operationen erlauben, bei denen das Trauma der traditionellen
chirurgischen Operation nicht ohne inakzeptable Schäden wiederholt
werden kann. Einige Forscher sind der Meinung, dass das Skalpell für
Weichteilchirurgie bis zum Jahr 2010 als ein veraltetes Relikt angesehen
werden könnte. Eine Menge Angst und viele der Nachbeben werden aus der
Chirurgie genommen, wenn das zutrifft. Offensichtlich ist dies eine sehr
gute Nachricht für den Patienten. Operationen, die jetzt Stunden dauern
und denen Tage oder Wochen im Krankenhaus folgen, werden nur eine halbe
Stunde dauern und vielleicht überhaupt keine Krankenhausaufnahme
erfordern. Es wäre möglich, dass sich der Chirurg und der Patient nie im
selben Raum befinden. Aber was wird das für Krankenhäuser und Chirurgen
bedeuten?

\subsection{Weniger Mikrochirurgen machen
mehr}\label{weniger-mikrochirurgen-machen-mehr}

Es wird eine Revolution in der Chirurgie geben. In der Ausbildung hat
ein Drittel der jungen Chirurgen nicht die notwendigen Fertigkeiten für
die Mikrochirurgie erworben. Ein Drittel schafft es gerade so, und ein
Drittel ist hervorragend. Ähnliche Verhältnisse finden sich in
Umschulungskursen für ältere Chirurgen. Weniger Chirurgen werden in der
Lage sein, mehr Operationen in kürzerer Zeit durchzuführen. Es ist
wahrscheinlich, dass Krankenkassen und Personen, die eine Operation
suchen, auf Erfolgsstatistiken für jeden Chirurg bestehen werden, die
recht unterschiedlich sein können. Patienten werden Chirurgen
bevorzugen, die die besten Ergebnisse liefern, insbesondere wenn ihre
Zustände lebensbedrohlich sind. In einigen Fällen können die besten
Chirurgen sogar aus der Ferne operieren. Sie könnten die gesamte
Operation aus einem anderen Rechtssystem heraus durchführen, in der die
Steuern niedriger sind und Gerichte überhöhte Schadensersatzansprüche
für Fehlverhalten nicht anerkennen.

\subsection{Digitale Anwälte}\label{digitale-anwuxe4lte}

Bevor er der Durchführung einer Operation zustimmt, wird der erfahrene
Chirurg wahrscheinlich auf einen digitalen Rechtsanwalt zurückgreifen,
um einen sofortigen Vertrag zu entwerfen, der die Haftung auf der
Grundlage der Größe und Eigenschaften des in den Bildern der
Magnetresonanzmaschine dargestellten Tumors spezifiziert und begrenzt.
Digitale Anwälte werden Informationserfassungssysteme sein, die die
Auswahl der Vertragsbestimmungen automatisieren und künstliche
Intelligenzverfahren wie neuronale Netzwerke nutzen, um private Verträge
an transnationale rechtliche Bedingungen anzupassen. Teilnehmer an den
meisten hochwertigen oder wichtigen Transaktionen werden nicht nur nach
geeigneten Partnern suchen, mit denen sie Geschäfte abwickeln können;
sie werden auch nach einem geeigneten Domizil für ihre Transaktionen
suchen.

\subsection{Notfallberatung}\label{notfallberatung}

Fahren wir mit dem Beispiel der Cyberchirurgie fort. Die Technologie des
Informationszeitalters wird einen besonderen Wert auf die höchsten
Fähigkeiten in der Chirurgie legen, wie es fast in jedem Bestreben der
Fall sein wird. Patienten sind bereit gewesen, einen solchen Mehrwert so
lange zu bezahlen, wie es Messer gegeben hat. Doch die Grenzen der
Informationen und die Schwierigkeit, im Notfall an einem konkreten Ort
nach Chirurgen zu suchen, haben den Markt für Chirurgie sehr
unvollkommen gemacht. Er wird im Informationszeitalter weniger
unvollkommen sein. Ein Patient, der innerhalb von 24 Stunden, oder
vielleicht sogar 45 Minuten, eine Operation braucht, könnte digitale
Assistenten beauftragen, die besten zehn Chirurgen weltweit zu finden,
die bereit sind, solch eine Aufgabe aus der Ferne durchzuführen, ihre
Erfolgsraten in ähnlichen Fällen zu überprüfen, und Angebote für ihren
speziellen Fall von entsprechenden digitalen Dienern einzuholen. All
dies könnte in wenigen Augenblicken abgewickelt werden. Als Folge dessen
werden die bevorzugten 10 Prozent der Chirurgen einen sehr viel größeren
Anteil am globalen Markt für Chirurgie haben. Die MRT-Maschine sowie
Mikrochirurgietechniken werden den Mehrwert ihrer Arbeit erhöhen.
Chirurgen mit geringeren Fähigkeiten werden sich auf lokale Restmärkte
konzentrieren.

Dieses Beispiel von Leben und Tod hilft, einige der revolutionären
Folgen der Befreiung von Volkswirtschaften von der Tyrannei der
Räumlichkeit zu verdeutlichen. Jemand könnte einwenden, dass die
MRT-Maschine von General Electric nicht dafür gedacht war, aus der Ferne
eingesetzt zu werden. Vielleicht, aber das verfehlt den Punkt. Diese
oder ähnliche Geräte werden es bald sein. Wenn Operationen besser von
Chirurgen durchgeführt werden, die auf einen Bildschirm schauen, anstatt
den Patienten direkt zu betrachten, wird es weniger wichtig sein, wo
sich der Chirurg und sein Bildschirm befinden. Eine immer größere Anzahl
von Dienstleistungen ist dazu bestimmt, neu konfiguriert zu werden, um
die Tatsache zu berücksichtigen, dass Informationstechnologie es
Personen auf der ganzen Welt ermöglicht, sogar in einer so heiklen
Angelegenheit wie einer Operation, miteinander zu interagieren. In
Aktivitäten, die weniger präzise Ausrüstungen erfordern und bei denen
das Risiko eines Misserfolgs geringer ist, wird die Cyberökonomie noch
schneller florieren.

\begin{quote}
„Die Finanzpolitik des Wohlfahrtsstaates erfordert, dass es für die
Eigentümer von Vermögen keine Möglichkeit gibt, sich selbst zu
schützen.`` - Alan Greenspan
\end{quote}

\section{DIE ABSCHWÄCHUNG DES
ZWANGS}\label{die-abschwuxe4chung-des-zwangs}

In nahezu jedem Wettbewerbsbereich, einschließlich des Großteils der
weltweiten, billionenschweren Anlageaktivitäten, wird die Verlagerung
von Transaktionen in den Cyberspace durch einen nahezu hydraulischen
Druck angetrieben -- dem Bestreben, räuberischen Besteuerungen zu
entgehen, einschließlich der Steuer, die die Inflation auf jeden ausübt,
der sein Vermögen in einer nationalen Währung hält.

\subsection{Dem Schutzgeldsystem
entfliehen}\label{dem-schutzgeldsystem-entfliehen}

Sie müssen nicht lange über die Megapolitik des Informationszeitalters
nachdenken, um zu erkennen, dass räuberische Steuern und Inflation, wie
sie die reichsten Industrieländer ihren Bürgern mit einer
Selbstverständlichkeit auferlegen, in den neuen Grenzen des Cyberspace
lächerlich wettbewerbsunfähig sein werden. Kurz nach der
Jahrtausendwende wird jeder, der Einkommenssteuern zu den derzeit
aufgelegten Raten zahlt, dies aus freien Stücken tun. Wie Frederic C.
Lane anmerkte, zeigt die Geschichte, dass „an den Grenzen und auf hoher
See, wo niemand ein dauerhaftes Monopol auf die Anwendung von Gewalt
hat, Händler die Zahlung von Abgaben vermieden, die so hoch waren, dass
Schutz durch andere Mittel billiger zu bekommen war.`` \footnote{Lane,
  \emph{Economic Consequences of Organized Violence}, ebenda, S. 404.}

Die Cyberwirtschaft bietet genau eine solche Alternative. Keine
Regierung wird in der Lage sein, sie zu monopolisieren. Und die darin
enthaltenen Informationstechnologien werden einen günstigeren und
effektiveren Schutz für Finanzvermögen bieten, als es die meisten
Regierungen jemals anbieten konnten.

\subsection{Die schwarze Magie des
Zinseszinses}\label{die-schwarze-magie-des-zinseszinses}

Bedenken Sie, dass jede jährliche Steuerzahlung von 5.000 Dollar, die
über vierzig Jahre geleistet wird, Ihr Nettovermögen um 2,2 Millionen
Dollar reduziert, vorausgesetzt, Sie könnten bloß eine 10-prozentige
Rendite auf Ihr Kapital erzielen. Bei einer 20-prozentigen Rendite
steigt der kumulierte Verlust auf etwa 44 Millionen Dollar. Für
besserverdienende Personen in einem Hochsteuerland sind die kumulierten
Verluste durch übermäßige Besteuerung im Laufe eines Lebens gewaltig.
Die meisten verlieren mehr, als sie je besessen haben.

Das scheint unmöglich zu sein, aber die Mathematik ist eindeutig. Sie
können dies selbst mit einem Taschenrechner überprüfen. Die obersten 1
Prozent der Steuerzahler in den Vereinigten Staaten zahlen im
Durchschnitt jährlich mehr als 125.000 Dollar an
Bundes-Einkommensteuern. Für einen Bruchteil dieses Betrags, nämlich
45.000 Dollar pro Jahr, könnte man in der Schweiz im Rahmen eines
privaten Steuerabkommens leben und Recht und Ordnung genießen, die von
dem wohl ehrlichsten Polizei- und Justizsystem der Welt gewährleistet
werden. Aus dieser Perspektive könnten die zusätzlich über dieses
großzügige Niveau hinaus gezahlten 80.000 Dollar an Einkommensteuer
durchaus als Tribut oder Raub betrachtet werden. 45.000 Dollar sind
sicherlich eine erhebliche Zahlung zur Aufrechterhaltung von Recht und
Ordnung, wenn man bedenkt, dass polizeilicher Schutz als öffentliches
Gut gedacht ist. Theoretisch können öffentliche Güter zu Grenzkosten von
Null auf zusätzliche Nutzer ausgeweitet werden. Die Schweizer freuen
sich, wenn Sie eine ausgehandelte feste Steuer von 45.000 Dollar (50.000
Schweizer Franken) pro Jahr zahlen, denn sie machen einen jährlichen
Gewinn von 45.000 Dollar mit jedem Millionär, der sich anmeldet.

Im Vergleich zur Schweizer Alternative würden die lebenslang erlittenen
Verluste für das Zahlen von Bundessteuern in den USA für einen Investor,
der im Durchschnitt eine Rendite von 20 Prozent erzielen könnte, 705
Millionen Dollar betragen. Aber denken Sie daran, dass wir dabei von
einer jährlichen Steuerzahlung von 45.000 Dollar ausgehen. Im Vergleich
zu einem Steuerparadies wie Bermuda, wo die Einkommenssteuer null ist,
würden die lebenslang erlittenen Verluste für das Zahlen von Steuern
nach amerikanischen Sätzen etwa 1,1 Milliarden Dollar betragen.

Möglicherweise wenden Sie ein, dass eine jährliche Rendite von 20
Prozent eine hohe Rendite ist. Zweifellos hätten Sie recht. Aber
angesichts des überragenden Wachstums in Asien in den letzten
Jahrzehnten haben viele Anleger weltweit das und mehr bereits erreicht.
Die durchschnittliche jährliche Rendite bei Immobilien in Hongkong
beträgt seit 1950 über 20 Prozent. Sogar einige Volkswirtschaften, die
weniger für ihr Wachstum bekannt sind, haben einfache Möglichkeiten für
hohe Gewinne geboten. Sie hätten in den letzten drei Jahrzehnten mit
Einlagen in US-Dollar in paraguayischen Banken eine durchschnittliche
reale Rendite von mehr als 30 Prozent pro Jahr einstreichen können. Hohe
Anlagerenditen sind an einigen Orten einfacher zu realisieren als an
anderen, aber geschickte Anleger können sicherlich Gewinne von 20
Prozent oder mehr in guten Jahren erzielen, auch wenn sie nicht konstant
die Leistungen von George Soros oder Warren Buffet erreichen.

Natürlich sind die Opportunitätskosten, die durch räuberische
Einkommens- und Kapitalertragssteuern auferlegt werden, umso größer, je
höher die Rendite ist, die Sie auf Ihr Kapital erzielen könnten. Aber
die Schlussfolgerung, dass der Verlust enorm ist, in der Tat sogar
größer als die gesamte Vermögenssumme, die Sie jemals ansammeln könnten,
hängt nicht davon ab, dass Sie astronomische Renditen erreichen können.
Einige Investmentfonds, die in den USA tätig sind, haben über einen
Zeitraum von mehr als einem halben Jahrhundert durchschnittlich
jährliche Gewinne von über 10 Prozent verzeichnet. Wenn Sie nicht besser
abschneiden und zu den obersten 1 \% der amerikanischen
Einkommensbezieher gehören, dann verringert sich Ihr Nettowert um mehr
als 33 Millionen Dollar allein durch die Einkommenssteuer, bei der Sie
jährlich über 45.000 Dollar zahlen. Im Vergleich zu einem Land ohne
Einkommensteuer beträgt der Verlust 55 Millionen Dollar.

\subsection{55 Dollar statt 55 Millionen
Dollar}\label{dollar-statt-55-millionen-dollar}

Wenn die Gewinnmaximierungsannahmen der Ökonomen korrekt sind, wie wir
allgemein annehmen, dann ist eine der sichersten Vorhersagen, die man
machen kann, dass die meisten Menschen handeln würden, um 55 Millionen
Dollar zu retten, wenn sie könnten. Das ist unsere Vorhersage. Wenn die
schwarze Magie des Zinseszinses in den Köpfen der erfolgreichen Leute in
Hochsteuerländern klarer wird, werden sie ausgiebig in verschiedenen
Gerichtsbarkeiten einkaufen, so wie sie jetzt nach Autos suchen oder
Versicherungstarife vergleichen. Wenn Sie daran zweifeln, halten Sie
doch einfach mal ein paar Leute an, die zufällig auf den Straßen von New
York oder Toronto herumlaufen und fragen sie, ob sie für 55 Millionen
Dollar nach Bermuda ziehen würden. Die Frage beantwortet sich von
selbst. Das Dilemma, das sie aufwirft, erinnert an das, was Mark Twain
sich vorstellte, als er entschied, ob er lieber nackt mit Lillian
Russell oder voll bekleidet mit General Grant übernachten würde. Er
zögerte nicht lange. Die Einwohner ausgewachsener Wohlfahrtsstaaten,
insbesondere der Vereinigten Staaten, sind vielleicht langsamer, aber
nur, weil sie sich der Wahl, die sie haben, noch nicht bewusst sind. Zu
gegebener Zeit werden sie sich dessen bewusst werden. Sie und jeder, der
von dem Wunsch motiviert ist, ein besseres Leben zu führen, wird die
Anziehungskraft verspüren, die Verluste zu reduzieren, die Sie durch
räuberische Besteuerung erleiden. Sie müssen lediglich Ihre
Transaktionen im Cyberspace durchführen. Dies wird natürlich in vielen
Rechtssystemen illegal sein. Aber alte Gesetze haben neuen Technologien
selten etwas entgegenzusetzen. In den 1980er Jahren war es in den
Vereinigten Staaten illegal, eine Faxnachricht zu senden. Die U.S. Post
betrachtete Faxe als Post der ersten Klasse, auf die die U.S. Post ein
altes Monopol beanspruchte. Eine entsprechende Verordnung wurde
erlassen, die die Anforderung wiederholte, dass alle Faxsendungen zur
nächsten Poststelle geleitet werden, um mit der regulären Post geliefert
zu werden. Milliarden von Faxnachrichten später ist immer noch unklar,
ob diesem Gesetz jemals jemand nachgekommen ist. Wenn ja, dann war die
Einhaltung nur von kurzer Dauer. Die Vorteile, in der aufkommenden
Cyberwirtschaft zu arbeiten, sind noch überzeugender als das Umgehen der
Post beim Senden eines Faxes.

Die weit verbreitete Einführung von Technologien zur Verschlüsselung mit
öffentlichem und privatem Schlüssel wird es bald ermöglichen, viele
wirtschaftliche Aktivitäten überall nach Belieben durchzuführen. Wie
James Bennet, Technologie-Redakteur von Strategic Investment,
geschrieben hat:

\begin{quote}
Die Durchsetzung von Gesetzen, insbesondere Steuergesetzen, ist stark
von der Überwachung von Kommunikation und Transaktionen abhängig. Wenn
die nächsten logischen Schritte unternommen worden sind und
Offshore-Bankstandorte die Dienste von Kommunikationen in starker
RSA-Verschlüsselung mittels elektronischer Post anbieten, die
Kontonummern aus öffentlichen Schlüsselsystemen verwendet, werden
Finanztransaktionen fast unmöglich zu überwachen sein, egal ob bei der
Bank oder in den Kommunikationen. Selbst wenn die Steuerbehörden einen
Maulwurf in der Offshore-Bank platzieren würden oder die Bankdaten
ausrauben würden, könnten sie die Einleger nicht
identifizieren.\footnote{James Bennet, \emph{The Information Revolution
  and the Demise of the Income Tax}, Strategic Investment, November
  1994, S. 11-12.}
\end{quote}

In einem Umfang, der noch nie zuvor möglich war, werden Einzelpersonen
bestimmen können, wo sie ihre wirtschaftlichen Aktivitäten ansiedeln und
wie viel Einkommenssteuer sie vorziehen zu zahlen. Viele Transaktionen
im Informationszeitalter werden überhaupt nicht in irgendeiner
territorialen Souveränität angesiedelt sein müssen. Diejenigen, die es
tun, werden zunehmend ihren Weg zu Orten wie Bermuda, den Cayman-Inseln,
Uruguay oder ähnlichen Rechtssystemen finden, die keine Einkommensteuern
oder andere kostspielige Belastungen auf den Handel erheben.

\subsection{Vom Monopol zur
Konkurrenz}\label{vom-monopol-zur-konkurrenz}

Regierungen haben sich daran gewöhnt, „Schutzdienste`` aufzuerlegen,
die, in den Worten von Frederic C. Lane, „von schlechter Qualität und
skandalös überteuert`` \footnote{Lane, \emph{Economic Consequences of
  Organized Violence}, ebenda, S. 404.} sind. Dieser Brauch, weit mehr
zu berechnen, als die Dienstleistungen der Regierung tatsächlich wert
sind, entwickelte sich über Jahrhunderte des Monopols. Steuern wurden
rücksichtslos für jeden erhöht, der fähig schien zu zahlen - gerade weil
Regierungen ein Monopol oder Quasi-Monopol auf Zwang hatten. Diese
Tradition des Monopols wird auf eine tiefgehende Weise mit den neuen
megapolitischen Möglichkeiten des Cyberhandels kollidieren.

Die Verschlüsselung wird es einfach machen, Transaktionen im Cyberspace
zu schützen. Die Kosten für ein effektives
Verschlüsselungssoftwareprogramm wie PGP sind geringer als die
Provision, die ein Full-Service-Broker für den Handel von hundert Aktien
erhebt. Dennoch wird es fast jede Transaktion unsichtbar und
unangreifbar für Regierungen und Diebe für viele Jahre in der Zukunft
machen. Die neue Technologie des Informationszeitalters wird
Cybervermögen zu verschwindend geringen Kosten effektiv schützen. Für 55
Dollar statt 55 Millionen Dollar werden Teilnehmer in der Cyberökonomie
einen besseren und echteren Schutz ihrer Vermögenswerte genießen, als
sie es in der Industrieära oder zu irgendeiner früheren Zeit in der
Geschichte genossen haben. Leicht verwendbare
Verschlüsselungsalgorithmen und die Fähigkeit, zwischen terrestrischen
Domizilen für Transaktionen zu wählen, werden einen effektiven Schutz
gegen die größte Quelle von Plünderungen bieten: die Nationalstaaten
selbst.

Das bedeutet jedoch nicht, dass territoriale Regierungen vollständig
ausmanövriert werden. Sie werden immer noch in der Lage sein,
Schwachstellen zum persönlichen Schaden auszunutzen, um Kopfsteuern zu
erheben oder vielleicht sogar wohlhabende Einzelpersonen zu erpressen.
Sie könnten auch die Erhebung von Verbrauchssteuern durchsetzen. Der
Schutz, die wichtigste staatliche Dienstleistung, wird jedoch auf eine
nahezu wettbewerbsfähige Grundlage gestellt. Weniger von den Kosten, die
produktive Menschen für ihren Schutz zahlen, können von den politischen
Behörden beschlagnahmt und umverteilt werden. Technologische
Innovationen werden einen großen und wachsenden Teil des weltweiten
Reichtums außerhalb der Reichweite von Regierungen platzieren. Dies
verringert die Risiken des Handels und senkt, in den Worten der
Historikerin Janet Abu-Lughod, „den Anteil aller Kosten``, der
andernfalls „für Transitzölle, Tribut oder einfache Erpressung``
aufgewendet werden müsste.\footnote{Abu-Lughod, ebenda, S. 177.}

Es kam in der Geschichte selten vor, dass Regierungen wirklich durch
Wettbewerb eingeschränkt waren. In den wenigen Zeiten, in denen so etwas
passiert ist, waren die Regierungen schwach und die Technologie zwischen
Rechtssystemen ähnlich. Wie Lane vorschlug, ist der Hauptfaktor, der die
Rentabilität unter solchen Bedingungen beeinflusst, der Unterschied, in
den Schutzkosten, die verschiedene Unternehmer zahlen müssen. Der
mittelalterliche Kaufmann, der zwanzig verschiedene Zollgebühren zahlen
musste, um seine Waren zum Markt zu bringen, konnte nicht mit einem
Kaufmann konkurrieren, der nur vier Zollgebühren zahlen musste, um die
gleichen Waren zum Kunden zu liefern. Ähnliche Bedingungen werden mit
dem Informationszeitalter zurückkehren. Die Rentabilität wird erneut
nicht so sehr vom technologischen Vorsprung abhängen, sondern davon, ob
es Ihnen gelingt, die Kosten, die Sie für Ihren Schutz zahlen müssen, zu
minimieren.

Diese neue ökonomische Dynamik steht in direktem Gegensatz zu den
Wunschvorstellungen von Regierungen aus der industriellen Ära,
Monopolpreise für ihre Schutzdienstleistungen durchzusetzen. Aber ob es
uns gefällt oder nicht, das alte System wird in der neuen
Wettbewerbsumgebung des Informationszeitalters nicht mehr lebensfähig
sein. Jede Regierung, die darauf besteht, ihre Bürger mit hohen Steuern
zu belasten, die Wettbewerber nicht zahlen, wird lediglich
sicherstellen, dass Gewinne und Reichtum anderswo hingelangen. Daher
wird das Versagen der reifen Wohlfahrtsstaaten, die Steuern langfristig
zu senken, sich selbst korrigieren. Regierungen, die zu stark besteuern,
machen das Wohnen innerhalb ihres Einflussbereichs zu einem
Insolvenzantrag.

\begin{quote}
„\ldots wie der König nach seinem Ermessen Geld aus beliebigem Material
und in beliebiger Form herstellen und den Standard dafür festlegen kann,
so kann er sein Geld in Substanz und Prägung ändern, den Wert davon
erhöhen oder mindern, oder es vollständig entwerten und annullieren``
\footnote{Zitiert aus Henry Mark Holzer,
  \emph{Governments\textquotesingle{} Money Monopoly} (New York: Books
  in Focus, 1981), S. 4.} - Aus einer englischen Gerichtsentscheidung,
1604
\end{quote}

\section{DAS ENDE DER SEIGNIORAGE}\label{das-ende-der-seigniorage}

Regierungen werden nicht nur ihre Macht verlieren, viele Arten von
Einkommen und Vermögen zu besteuern; sie sind auch dazu bestimmt, ihr
Machtmonopol über Geld zu verlieren. In der Vergangenheit waren
megapolitische Übergänge mit Veränderungen in der Eigenschaft von Geld
verbunden.

\begin{itemize}
\tightlist
\item
  Die Einführung des Münzwesens trug dazu bei, den fünfhundertjährigen
  Expansionszyklus der antiken Wirtschaft einzuleiten, der mit der
  Geburt Christi und den niedrigsten Zinssätzen vor der Neuzeit seinen
  Höhepunkt erreichte.
\item
  Der Beginn des Dunklen Zeitalters fiel mit der faktischen Schließung
  der Münzstätten zusammen. Römische Münzen zirkulierten zwar weiterhin,
  aber die Geldmenge ging zusammen mit dem Handel in einer sich selbst
  verstärkenden Abwärtsspirale zurück.
\item
  Die feudale Revolution fiel mit der Wiedereinführung von Geld, Münzen,
  Wechseln und anderen Instrumenten zur Abwicklung von Handelsgeschäften
  zusammen. Insbesondere der Anstieg der europäischen Silberproduktion
  durch die neuen Minen am Rammelsberg in Deutschland ermöglichte eine
  Zunahme des Münzumlaufs, der den Handel ankurbelte.
\item
  Die größte Revolution des Geldes vor dem Informationszeitalter kam mit
  dem Aufkommen der Industrialisierung. In der Schießpulverrevolution
  festigte der frühneuzeitliche Staat seine Macht. In dem Maße, wie
  seine Kontrolle zunahm, machte der Staat seine Macht über das Geld
  geltend und stützte sich dabei in hohem Maße auf die charakteristische
  Technologie der Industrialisierung, die Druckerpresse. Das erste
  Instrument der Massenproduktion, die Druckerpresse, wurde von den
  Regierungen in der Neuzeit häufig zur Massenproduktion von Papiergeld
  eingesetzt.
\end{itemize}

Papiergeld ist ein eindeutig industrielles Produkt. Vor der Erfindung
der Druckerpresse wäre es unpraktisch gewesen, Quittungen oder
Zertifikate zu duplizieren, die zu Papiergeld wurden. Mönche in
Skriptorien hätten ihre Zeit verschwendet, Fünfzig-Pfund-Noten zu
zeichnen. Papiergeld hat auch erheblich zur Macht des Staates
beigetragen, nicht nur durch Gewinne aus der Abwertung der Währung,
sondern auch dadurch, dass der Staat Einfluss darauf hatte, wer Reichtum
ansammeln konnte. Wie Abu-Lughod es formulierte: „Wenn Papiergeld, das
vom Staat gedeckt wird, zur anerkannten Währung wird, wird es schwierig,
Kapital entgegen oder unabhängig von den staatlichen Maschinerien
anzuhäufen.`` \footnote{Abu-Lughod, ebenda., S. 332.}

\section{CYBERGELD}\label{cybergeld}

Nun bedeutet das Aufkommen des Informationszeitalters eine weitere
Revolution in der Eigenschaft des Geldes. Mit dem Beginn des
Cyberhandels führt der Weg unweigerlich zum Cybergeld. Diese neue Art
des Geldes wird das bisherige Kräfteverhältnis neu ausrichten, indem es
die Fähigkeit der Nationen der Welt reduziert, darüber zu bestimmen, wer
zum souveränen Individuum wird. Ein entscheidender Teil dieser
Veränderung wird sich durch den Einfluss der Informationstechnologie
vollziehen, indem sie Vermögensinhaber vor einer Enteignung durch
Inflation schützt. Bald werden Sie fast jede Transaktion über das
Internet oder das World Wide Web zur gleichen Zeit, in der sie die
Transaktion tätigen, mit Cybergeld bezahlen.

Diese neue digitale Form von Geld ist dazu bestimmt, eine entscheidende
Rolle im Cyberhandel zu spielen. Es wird aus verschlüsselten Sequenzen
von mehreren hundertstelligen Primzahlen bestehen. Einzigartig, anonym
und überprüfbar, wird dieses Geld die umfangreichsten Transaktionen
ermöglichen. Es wird außerdem in den kleinsten Bruchteil des Wertes
teilbar sein. Durch das Drücken einer Taste kann es in einem
billionenschweren Großhandelsmarkt ohne Grenzen gehandelt werden.

\subsection{Die Wahl des Geldes}\label{die-wahl-des-geldes}

Unweigerlich wird dieses neue Cybergeld entnationalisiert werden. Wenn
souveräne Individuen in einem Bereich ohne physische Realität über
Grenzen hinweg handeln können, werden sie es nicht mehr erdulden müssen,
dass Regierungen den Wert ihres Geldes durch Inflation mindern. Warum
sollten sie auch? Die Kontrolle über das Geld wird von den
Machtzentralen auf den globalen Marktplatz verlagert. Jede Einzelperson
oder Firma mit Zugang zum Cyberspace wird in der Lage sein, problemlos
aus einer Währung auszusteigen, die Gefahr läuft, an Wert zu verlieren.
Anders als heute wird es keine Notwendigkeit geben, in gesetzlichen
Zahlungsmitteln zu handeln. Tatsächlich ist es wahrscheinlich, dass bei
Transaktionen, die den Globus umspannen, mindestens eine Partei in einer
Währung handeln wird, die für sie kein gesetzliches Zahlungsmittel ist.

\subsection{Die Reduzierung der Nachteile des
Tauschhandels}\label{die-reduzierung-der-nachteile-des-tauschhandels}

Sie werden in der Lage sein, in der Cyberökonomie in jedem Medium Ihrer
Wahl zu handeln. Wie der verstorbene, mit dem Nobelpreis ausgezeichnete
Wirtschaftswissenschaftler F. A. von Hayek argumentierte, gibt es
„keinen eindeutigen Unterschied zwischen Geld und Nicht-Geld``. Er
schrieb: „Obwohl wir für gewöhnlich annehmen, dass es eine klare
Unterscheidung zwischen dem gibt, was Geld ist und was nicht - und das
Gesetz im Allgemeinen versucht, eine solche Unterscheidung zu machen -
gibt es keinen so klaren Unterschied, was die kausalen Auswirkungen von
monetären Ereignissen betrifft. Was wir stattdessen vielmehr vorfinden,
ist ein Kontinuum, in dem Objekte unterschiedlicher Liquidität oder mit
Werten, die unabhängig voneinander schwanken können, in dem Maße
ineinander übergehen, wie sie als Geld funktionieren.`` \footnote{Friedrich
  A. von Hayek, \emph{The Denationalization of Money} (London: Institute
  of Economic Affairs, 1976), S. 47.} Digitales Geld in globalen
Computernetzwerken wird jedes Objekt auf Hayeks Kontinuum der Liquidität
flüssiger machen - mit Ausnahme von staatlichen Papieren. Eine Folge
davon wird sein, dass der Tauschhandel viel praktischer wird. Eine
wachsende Anzahl von Objekten und Dienstleistungen wird in spezifischen
Angeboten für andere Objekte und Dienstleistungen angeboten. Diese
potenziellen Transaktionen werden weltweit im Netz beworben, was ihre
Liquidität um ein Vielfaches erhöht.

Eines der Hauptprobleme des Tauschhandels war immer die Schwierigkeit,
eine Person mit einem bestimmten Bedarf mit einer anderen zu verbinden,
die genau dieses Angebot hatte und genau das erwerben wollte, was die
erste Partei zum Tausch anbot. Der primitive Tauschhandel stolperte über
die entmutigende Unwahrscheinlichkeit, genau zwei Parteien in einem
lokalen Markt zu finden, die einen Austausch wünschten. Bargeld überwand
die Beschränkungen des Tauschhandels, und seine Vorteile bleiben in den
meisten Transaktionen überzeugend. Aber enorme Zunahmen der
Rechenleistung und die Globalisierung des Handels im Cyberspace
verringern auch die Nachteile des Tauschhandels. Die Chancen, jemanden
mit genau entsprechenden Wünschen zu finden, steigen dramatisch, wenn
Sie sofort weltweit suchen können, anstatt sich nur auf diejenigen zu
beschränken, die Sie lokal treffen könnten.

\subsection{Unfälschbar}\label{unfuxe4lschbar}

Obwohl Papiergeld zweifellos weiterhin als verbliebenes Tauschmedium für
die Armen und diejenigen, die mit Computern nicht umgehen können, im
Umlauf bleiben wird, wird das Geld für Transaktionen von hohem Wert
privatisiert. Cyber-Geld wird nicht mehr nur in nationalen Einheiten,
wie das Papiergeld der Industriezeit, bemessen sein. Wahrscheinlich wird
es nach Gramm oder Unzen Gold definiert, so fein teilbar wie das Gold
selbst. Oder es könnte auf Basis anderer realer Wertspeicher definiert
werden. Selbst dort, wo unterschiedliche Preismaßstäbe verwendet werden
oder bestimmte Transaktionen weiterhin in nationalen Währungen bemessen
werden, wird das Cyber-Geld den Verbrauchern weitaus besser dienen als
staatliches Geld es jemals konnte. Die rasch voranschreitende
Rechenkapazität wird die Schwierigkeiten, Preise an verschiedene
Tauschmittel anzupassen, auf ein Minimum reduzieren. Jede Transaktion
beinhaltet die Übertragung verschlüsselter mehrerer hundertstelliger
Primzahlenfolgen. Im Gegensatz zu den von Regierungen im Zeitalter des
Goldstandards ausgegebenen Papiergeldscheinen, die nach Belieben
dupliziert werden konnten, wird der neue digitale Goldstandard oder
seine Tauschwertäquivalente fast unmöglich zu fälschen sein. Der
zugrundeliegende mathematische Beweis dafür lautet, dass es nahezu
unmöglich ist, das Produkt aus mehreren hundertstelligen Primzahlen zu
entwirren. Alle Quittungen werden nachweislich einzigartig sein.

Die Namen traditioneller Währungen wie „Pfund`` und „Peso`` spiegeln die
Tatsache wider, dass sie ursprünglich als Maßeinheiten für das Gewicht
bestimmter Mengen von Edelmetallen entstanden sind. Das Pfund Sterling
war einmal ein Pfund Sterlingsilber. Papiergeld im Westen begann als
Lager- oder Safedepot-Scheine für Mengen von Edelmetallen. Regierungen,
die diese Scheine ausgaben, stellten bald fest, dass sie deutlich mehr
davon drucken konnten, als sie tatsächlich aus ihrem Gold- und
Silberbestand einlösen konnten. Das war einfach. Keine Person, die ein
Gold- oder Silberzertifikat besaß, konnte irgendeine Information über
den tatsächlichen Vorrat an Edelmetallen aus ihrer Quittung entnehmen.
Abgesehen von den Seriennummern sahen alle Scheine gleich aus, eine
Tatsache, die sowohl Fälschern als auch Politikern und Bankern gefiel,
die von der Inflation der Geldmenge profitieren wollten.

Cyberwährungen werden auf diese Weise praktisch unmöglich zu fälschen
sein, ob offiziell oder inoffiziell. Die Überprüfbarkeit der digitalen
Belege schließt dieses klassische Mittel zur Enteignung von Vermögen
durch Inflation aus. Das neue digitale Geld des Informationszeitalters
wird die Kontrolle über das Tauschmedium wieder den Eigentümern von
Vermögen zurückgeben, die es bewahren möchten, anstatt es den
Nationalstaaten zu überlassen, die es verschwinden lassen möchten.

\subsection{Die Transaktionskosten von ``freier``
Währung}\label{die-transaktionskosten-von-freier-wuxe4hrung}

Die Nutzung dieser neuen Cyber-Währung wird Sie weitgehend von der Macht
des Staates befreien. Früher haben wir die trostlose Bilanz der
nationalstaatlichen Welt in Bezug auf die Werterhaltung ihrer Währungen
über das letzte halbe Jahrhundert erwähnt. Seit dem Zweiten Weltkrieg
hat keine Währung weniger Wert durch Inflation verloren als die Deutsche
Mark. Und dennoch verschwand zwischen dem 1. Januar 1949 und Ende Juni
1995 sogar 71 Prozent ihres Wertes. Die Weltreservewährung in dieser
Zeit, der US-Dollar, verlor 84 Prozent seines Wertes.\footnote{Siehe
  Kapitel 1, Anmerkung 6.} Dies ist ein Maß für den Reichtum, den
Regierungen durch Ausnutzung ihrer territorialen Monopole auf
gesetzliche Zahlungsmittel enteigneten.

Beachten Sie, dass es keine intrinsische Notwendigkeit gibt, dass eine
Währung abwertet oder die nominellen Lebenshaltungskosten jedes Jahr
steigen. Im Gegenteil. Die technische Herausforderung, die Kaufkraft von
Ersparnissen aufrechtzuerhalten, ist trivial. Dies wird deutlich, wenn
Sie sich einfach die langfristige Kaufkraft von Gold ansehen.

Zwischen dem 1. Januar 1949 und Ende Juni 1995 verlor die beste der
nationalisierten Währungen fast drei Viertel ihres Wertes, während die
Kaufkraft von Gold tatsächlich stieg. Wie Professor Roy W. Jastrom in
seinem Buch \emph{The Golden Constant} dokumentiert hat, hat Gold seine
Kaufkraft mit geringfügigen Schwankungen so lange aufrechterhalten, wie
zuverlässige Preisaufzeichnungen verfügbar sind, im Falle Englands bis
1560.

Nationalwährungen, die an Gold gebunden sind, haben ihre Kaufkraft auch
dann bewahrt, wenn militärische Notwendigkeiten nicht erdrückend waren.
Der Wert des britischen Pfunds Sterling stieg im relativ friedlichen 19.
Jahrhundert eher an, obwohl es nur lose mit Gold verbunden war. Die
neuen megapolitischen Bedingungen des Informationszeitalters machen
jedoch nicht eine schwache Verbindung, wie den Goldstandard, sondern
eine starke Verbindung möglich, die zum ersten Mal durch erheblich
verbesserte Informations- und Rechenressourcen in den Händen der
Verbraucher gestärkt wird.

\begin{quote}
„Die Bedrohung durch den schnellen Verlust ihres gesamten Geschäfts,
wenn sie die Erwartungen nicht erfüllen (und wie jede
Regierungsorganisation sicherlich die Gelegenheit missbrauchen wird, mit
Rohstoffpreisen zu spielen!), würde einen viel stärkeren Schutz bieten
als jeder, der gegen ein Regierungsmonopol entwickelt werden wird``
\footnote{Hayek, ebenda, S. 40.} - Friedrich A. Von Hayek
\end{quote}

\subsection{Die Privatisierung von
Geld}\label{die-privatisierung-von-geld}

Friedrich von Hayek argumentierte 1976, dass der Einsatz von
wettbewerbsfähigen, privaten Währungen die Inflation beseitigen
würde.\footnote{Ebenda.} Ohne Anforderungen an gesetzliche
Zahlungsmittel, die die Annahme einer im Wert steigenden Währung
innerhalb eines Rechtssystems erzwingen, argumentierte Hayek, würde der
Markt den privaten Emittenten von Währungen auferlegen, den Wert ihres
Tauschmediums zu erhalten. Jeder Emittent einer privaten Währung, der es
nicht schafft, seinen Wert zu erhalten, würde bald seine Kunden
verlieren. Die Entwicklung von verschlüsseltem Cybercash wird Hayeks
Logik anschaulich zum Leben erwecken.

Die Theorie des „freien Bankwesens``, wie es genannt wird, ist nicht
bloß ein hypothetisches akademisches Gedankenspiel. Private,
konkurrierende Währungen zirkulierten in Schottland von Anfang des
achtzehnten Jahrhunderts bis 1844. Während dieser Zeit gab es in
Schottland keine Zentralbank. Es gab wenige Vorschriften oder
Beschränkungen für den Eintritt in das Bankgeschäft. Private Banken
nahmen Einlagen entgegen und gaben ihre eigenen privaten Währungen aus,
die durch Goldbarren gedeckt waren. Wie Professor Lawrence White
dokumentiert hat, hat dieses System gut funktioniert. Es war stabiler,
mit weniger Inflation als das stärker regulierte und politisierte System
des Bankwesens und des Geldes, das in England während der gleichen Zeit
angewendet wurde.\footnote{Siehe Lawrence White, \emph{Free Banking in
  Britain} (London: Institute of Economic Affairs, 1995).} Michael
Prowse von der Financial Times fasste die Erfahrungen Schottlands mit
dem freien Bankwesen zusammen: „Es gab wenig Betrug. Es gab keine
Anzeichen für eine Überausgabe von Banknoten. Die Banken hielten in der
Regel weder übermäßige noch unzureichende Reserven. Bank Runs waren
selten und uferten nicht aus. Die freien Banken genossen das Ansehen der
Bürger und bildeten eine solide Grundlage für ein Wirtschaftswachstum,
das das in England für die meiste Zeit dieser Periode übertraf.``
\footnote{Michael Prowse, \emph{Bring Back Gold}, Financial Times, 5.
  Februar 1996, S. 12.}

Was unter den technologischen Bedingungen des achtzehnten und
neunzehnten Jahrhunderts gut funktioniert hat, wird mit der Technologie
des einundzwanzigsten Jahrhunderts sogar noch besser funktionieren. Sie
werden bald in der Lage sein, mit digitalem Geld von einer privaten
Firma zu handeln, das ähnlich ausgegeben wird, wie American Express
Reiseschecks als Belege für Bargeld ausstellt. Eine Institution von
höherem Ansehen als jede Regierung, wie zum Beispiel ein führendes
Bergbauunternehmen oder die Swiss Bank Corporation, könnte
verschlüsselte Quittungen für Goldmengen oder sogar für einzigartige
Goldbarren erstellen, die durch molekulare Signaturen identifiziert und
möglicherweise sogar mit Hologrammen beschriftet werden. Diese
Quittungen werden dann als Geld gehandelt, mit fast keiner Möglichkeit,
dass sie gefälscht oder inflationiert werden können.

Das neue digitale Gold wird viele der praktischen Probleme, die die
direkte Nutzung von Gold als Geld in der Vergangenheit gehemmt haben,
überwinden. Es wird nicht länger unpraktisch, umständlich oder
gefährlich sein, mit großen Summen von Gold zu handeln. Digitale
Quittungen werden nicht zu schwer zum Tragen sein. Tatsächlich wird ihre
einzige physische Existenz in aufwendigen Mustern von Computercode
bestehen. Ebenso wird es nicht schwierig sein, digitale Quittungen in so
kleine Einheiten zu unterteilen, dass sie selbst für Mikro-Käufe genutzt
werden können. Ein winziges Stück physisches Gold, klein genug, um damit
einen Kaugummi zu bezahlen, würde bald verloren gehen oder mit einem
Stück verwechselt werden, das klein genug ist, um zwei Kaugummis zu
bezahlen. Aber für den Computer wird es einfach sein, diese Stückelungen
von digitalem Geld zu unterscheiden, egal ob es die Größe eines
Streifenhörnchens oder die eines Nashorns hat.

Die Fähigkeit von digitalem Geld, Mikrozahlungen zu ermöglichen, wird
die Entstehung von neuen Geschäftsarten begünstigen, die bisher nicht
existieren konnten, die sich auf die Organisation der Verteilung von
Informationen mit geringem Wert spezialisieren. Die Anbieter dieser
Informationen werden nun durch Direkt-Abbuchungsgebührensysteme
entschädigt, die die bisher abschreckenden Transaktionskosten
überwinden. Wenn die Kosten für die Rechnungsstellung den Wert einer
Transaktion überschreiten, wird sie wahrscheinlich nicht stattfinden.
Die Nutzung von Cyber-Geld erleichtert sehr kostengünstige gleichzeitige
Abrechnungen, bei denen Konten mit deren Nutzung belastet werden. Wir
zitierten ein solches Beispiel oben, indem wir uns vorstellten, dass Sie
möglicherweise eine Lizenzgebühr in Höhe von einem Drittel eines Pennys
an Bill Gates, oder wer auch immer die virtuellen Realitätsrechte für
eine Tour durch den Louvre besitzt, zahlen könnten. Dies kann auf
tausend Arten geschehen. Die virtuelle Realität wird nahezu unbegrenzte
Lizenzierungsmöglichkeiten schaffen, die dennoch nur
Mikro-Lizenzgebühren einfordern. Eines Tages werden Sie in der Lage
sein, das dritte Spiel der World Series von 1969 erneut anzusehen und
Mikrolizenzen an die Spieler zu zahlen, deren Bilder verwendet werden,
um Ihre virtuelle Realität real erscheinen zu lassen.

\section{INFLATION BESEITIGEN}\label{inflation-beseitigen}

Ungeachtet solcher Möglichkeiten dürfte die bedeutendste Auswirkung des
neuen digitalen Geldes das Ende der Inflation und die Entschuldung des
Finanzsystems sein. Die wirtschaftlichen Implikationen sind
tiefgreifend. Der Anstieg der Inflation im zwanzigsten Jahrhundert, wie
wir es in \emph{Blood in the Streets} und \emph{The Great Reckoning}
dargelegt haben, war eng mit der weltweiten Machtverteilung verbunden.
Steigende Erträge durch Gewaltanwendung erforderten drastisch höhere
Militärausgaben, die wiederum immer aggressivere Bemühungen zur
Enteignung von Reichtum notwendig machten. Regierungen entdeckten, dass
sie effektiv eine jährliche Vermögenssteuer auf alle erheben konnten,
die Guthaben in ihrer nationalen Währung hielten. Diese jährliche
Vermögenssteuer auf Währungsinhaber konnte auch als Transaktionsgebühr
gesehen werden, die den Nutzern der Währung erlaubte, ihren Reichtum in
einer von den Emittenten bereitgestellten bequemen Form zu
halten.\footnote{Die Inflation hatte in der Industriezeit, als die
  Preise und Löhne nach unten hin unflexibel waren, eine weitere
  Verlockung. Eine mäßige Inflation erhöhte die Produktion, indem sie
  die Reallöhne und Preise senkte.}

Es mag ungewöhnlich sein, die Inflation als eine Transaktionsgebühr für
die Bequemlichkeit des Geldbesitzes zu betrachten, aber denken Sie genau
darüber nach. Während des Industriezeitalters haben wir uns so sehr
daran gewöhnt, die Bereitstellung von Währung als Dienstleistung zu
sehen, für die man nicht direkt bezahlt, dass es leicht war zu
vergessen, dass die Ausgeber der Dollars, Pesos, Pfund und Franken,
nämlich die Regierungen, von uns verlangt haben, dass wir dafür
bezahlen, und zwar teuer -- durch Inflation. Die Rate dieser
inflatorischen Transaktionsgebühr auf Währung variierte im letzten
halben Jahrhundert von einem Tiefststand von 2,7 Prozent jährlich für
die Deutsche Mark bis hin zu Raten, die gefährlich nahe an 100 Prozent
lagen. So wurden beispielsweise zwischen 1960 und 1991, als Präsident
Menem die argentinische Währungsreform einleitete, durch die Inflation
siebzehn Nullen aus den verschiedenen Versionen der argentinischen
Währung gestrichen. Wenn 1960 der gesamte Reichtum der Welt in
argentinische Pesos umgewandelt und vergraben worden wäre, wäre es 1991
nicht mehr die Mühe wert gewesen, diesen wieder auszugraben.

Argentiniens Beispiel ist ein führender Indikator für das nächste
Jahrtausend. Währung wird nicht mehr inflationierbar sein, weil andere
Staaten nicht mehr damit davonkommen, ebenso wie Argentinien es nicht
mehr kann. Inflation hatte in der Industriezeit, als die Preise und
Löhne nach unten hin unflexibel waren, eine weitere Verlockung. Eine
mäßige Inflation erhöht die Produktion durch Senkung der Reallöhne, und
die Preise könnten durch eine aus anderen Ländern importierte
Kreditverknappung beeinträchtigt werden. Privates Geld wird aufgrund des
Wettbewerbsdrucks nicht inflationierbar sein.

Die Beseitigung der Inflation wird die getarnten Profite wegnehmen, die
die Inflation bisher denjenigen zukommen ließ, die die monopolistischen
Ausgeber der Währung waren. Falls alle getarnten Gewinne aus der Ausgabe
von Geld verschwinden würden, wäre eine neue Zahlungsmethode notwendig,
um die Währungsausgeber direkt zu entschädigen. Die Nutzung des neuen
Geldsystems wird daher wahrscheinlich explizitere Transaktionskosten
beinhalten, vielleicht eine Gebühr in der Größenordnung von 1 Prozent
pro Jahr. Das ist ein kleiner Preis im Vergleich zu der jährlichen
Inflationsstrafe von 2,7 bis 99 Prozent, die von Nationalstaaten
verhängt wird. Dies gilt umso mehr, da die Wahrscheinlichkeit besteht,
dass die Gesamtpreise in Zukunft sinken werden, wenn Monopole abgebaut
werden und der Wettbewerb weltweit zunimmt.

\subsection{Vertragliche Hebelwirkung}\label{vertragliche-hebelwirkung}

Das Aufkommen digitaler Geldarten wird nicht nur ein für alle Mal die
Inflation besiegen; es wird auch den Hebel in den Bankensystemen der
Welt reduzieren. Die Fähigkeit von Menschen auf der ganzen Welt,
regulatorische Behörden zu umgehen und ihre Gelder direkt über das
Internet zu verschieben, ist eine völlig beispiellose Folge der
Globalisierung der Märkte. Es wird jenseits der Macht jeder Regierung
liegen, dies zu regulieren. Wenn Regierungen nicht mehr in der Lage
sind, Währungen durch Gelddrucken zu entwerten oder die Sparer zu
betrügen, indem sie die Kreditvergabe über die konzerneigenen
Bankensysteme nach Belieben ausweiten, verlieren sie einen großen Teil
ihrer indirekten Fähigkeit, Ressourcen zu beschlagnahmen.

\subsection{Höhere Zinssätze}\label{huxf6here-zinssuxe4tze}

Dies wird ein offensichtliches Dilemma für die meisten westlichen
Regierungen schaffen. Sie werden mit starken Einbußen bei den
Steuereinnahmen und der faktischen Abschaffung der Hebelwirkung im
Geldsystem konfrontiert sein. Gleichzeitig werden sie die ungedeckten
Verpflichtungen und die aufgeblähten Erwartungen für Sozialausgaben, die
aus der Industrieära geerbt wurden, beibehalten. Das erwartete Resultat
ist eine intensive fiskale Krise mit vielen unangenehmen sozialen
Nebenwirkungen, die wir in den späteren Kapiteln berücksichtigen werden.
Die wirtschaftliche Konsequenz dieser Übergangskrise wird wahrscheinlich
einen einmaligen Anstieg der realen Zinssätze beinhalten. Die Schuldner
werden in Bedrängnis geraten, da langfristige Verbindlichkeiten, die
unter dem alten System eingegangen wurden, abgewickelt werden und
Vorzugskredite versiegen.

\subsection{Durch den Wettbewerb
verändert}\label{durch-den-wettbewerb-veruxe4ndert}

Regierungen, die mit ernsthafter Konkurrenz zu ihren Währungsmonopolen
konfrontiert sind, werden wahrscheinlich versuchen, die
kostenpflichtigen Cyberwährungen zu unterbieten, indem sie Kredite
straffen und Sparern höhere tatsächliche Renditen auf Kontoguthaben in
nationalen Währungen bieten. Einige Regierungen könnten sogar danach
streben, Gold als weiteres Mittel zur Bewältigung des Wettbewerbs von
privaten Währungen wieder zu monetarisieren. Sie könnten zu dem Schluss
kommen, dass sie höhere Seigniorage-Gewinne von einem locker
kontrollierten Goldstandard des 19. Jahrhunderts erzielen könnten, als
wenn sie es zulassen würden, dass ihre nationale Währung vollständig
durch kommerzielles Cyber-Geld ersetzt wird. Aber nicht alle Regierungen
werden auf die gleiche Weise reagieren. Diejenigen, die in Regionen
leben, in denen die Computernutzung und die Beteiligung am Netz gering
sind, können sich in der Anfangsphase der Cyberwirtschaft für eine
altmodische Hyperinflation entscheiden. Das wird es diesen Regierungen
nicht ermöglichen, sich die Geldbestände der Reichen zu sichern, aber es
wird Ressourcen von denen abzwacken, die wenig Vermögen oder Zugang zur
Cyberwirtschaft haben. Regierungen, die solche Taktiken anwenden,
könnten sich nichtsdestotrotz international Geld in der Cyberwährung
leihen.

Andere Regierungen könnten sich den Chancen anpassen, die durch die
Informationswirtschaft entstehen, und lokale Transaktionen in
Cyberwährung erleichtern. Diejenigen Rechtsgebiete, die die Gültigkeit
digitaler Signaturen als erste anerkennen und die Zwangsvollstreckung
bei Nichtbezahlung von Cyberschulden vor Ort ermöglichen, werden von
einem unverhältnismäßig starken Anstieg der langfristigen Kapitalvergabe
profitieren. Natürlich wäre in Gebieten, in denen lokale Gerichte
Strafen verhängen oder Schuldnern einen Zahlungsausfall ohne rechtliche
Konsequenzen gestatten, kein Cyber-Geld für langfristige Kredite
verfügbar.

\subsection{Ertragslücke}\label{ertragsluxfccke}

Die Kombination aus Kreditkrisen, wettbewerbsbedingten Anpassungen durch
nationale Währungsbehörden und anfänglichen Übergangshürden bei der
Vergabe von Cyberwährungskrediten wird in den Anfangsstadien der
Informationswirtschaft zu einer Renditelücke führen. Cyber-Geld wird
niedrigere Zinssätze als nationale Währungen zahlen und wahrscheinlich
auch explizite Transaktionskosten tragen. Diese offensichtlichen
Nachteile des Haltens von Guthaben in digitaler Währung werden jedoch
durch einen verbesserten Schutz vor Verlusten aus Raubsteuern und
Inflation ausgeglichen. Da es wahrscheinlich an Gold gebunden ist, wird
Cyber-Geld auch von der Aufwertung des Goldes profitieren. Der Goldpreis
wird wahrscheinlich deutlich steigen, unabhängig davon, welche der
alternativen Regierungspolitiken dominiert. Warum? Der reale Goldpreis
steigt praktisch immer in einer Deflation. Eine Deflation spiegelt
schließlich einen Liquiditätsmangel wider. Und Gold ist die ultimative
Form der Liquidität.

\subsection{Die Deflation des
Industriezeitalters}\label{die-deflation-des-industriezeitalters}

Höhere Realzinsen überall werden die Liquidierung von kostenintensiven,
unproduktiven Aktivitäten anspornen und vorübergehend den Konsum
reduzieren. Wir haben die Logik des Konjunkturzyklus und seine Auflösung
in \emph{Blood in the Streets} und \emph{The Great Reckoning}
untersucht, daher werden wir diese Argumente hier nicht erneut
durchgehen. Es genügt an dieser Stelle zu erwähnen, dass sich das
deflationäre Umfeld noch einige Zeit hinziehen könnte, mit größeren
Konsequenzen für die industriellen Hochlohnländer Nordamerikas und
Westeuropas als für die Niedriglohnländer Asiens und Lateinamerikas.

\section{LANGFRISTIG NIEDRIGERE
ZINSSÄTZE}\label{langfristig-niedrigere-zinssuxe4tze}

Während die frühen Auswirkungen des Aufkommens der Cyberwirtschaft
wahrscheinlich höhere Zinsen beinhalten werden, wird die langfristige
Konsequenz genau das Gegenteil sein. Die nach Steuern abgezogenen
Erträge für die Sparer werden deutlich steigen, sobald die Ressourcen
dem Zugriff der Regierungen entkommen. Dramatische Verbesserungen bei
der Effizienz der Ressourcennutzung und die Befreiung von Kapital, um
weltweit die höchsten Renditen zu finden, sollten schnell für den
Verlust an Produktion entschädigen, der zu Anfang in der Übergangskrise
eintritt.

\subsection{Anlegerkontrolle über das
Kapital}\label{anlegerkontrolle-uxfcber-das-kapital}

Konventionelle Denker, die unser Argument bis hierhin betrachten, würden
zu dem Schluss kommen, dass der Zusammenbruch der Einkommensumverteilung
in den führenden Nationenstaaten die Welt in einen wirtschaftlichen
Kollaps stürzen würde. Glauben Sie das nicht! Wir bestreiten nicht die
Tatsache, dass eine Übergangskrise wahrscheinlich ist. Aber die Ansicht,
dass der Staat die Funktionsweise der Wirtschaft durch massive
Umverteilung von Ressourcen verbessert, ist ein Anachronismus, ein
Glaubensgrundsatz, der etwa dem weit verbreiteten Aberglauben am Ende
des Mittelalters entspricht, dass Fasten und Selbstkasteiung einer
Gemeinschaft zugutekamen. Es sollte nicht vergessen werden, dass
Regierungen Ressourcen in großem Stil verschwenden. Verschwenderischer
Umgang mit Ressourcen macht arm. Eine dramatische Verbesserung der
Effizienz der Ressourcennutzung wird eintreten, wenn die Einnahmen, die
historisch von Regierungen vereinnahmt wurden, stattdessen von Personen
mit echtem Talent kontrolliert werden.

Zehn und schließlich Hunderte von Milliarden Dollar werden von
Hunderttausenden und letztendlich Millionen von souveränen
Einzelpersonen kontrolliert werden. Es ist wahrscheinlich, dass diese
neuen Verwalter des Weltvermögens weitaus besser als Politiker in der
Lage sein werden, Ressourcen zu nutzen und Investitionen einzusetzen.
Zum ersten Mal in der Geschichte werden es aufgrund der megapolitischen
Bedingungen die fähigsten Investoren und Unternehmer sein, die am Ende
die Kontrolle über das Kapital haben, und nicht mehr die Spezialisten
für Gewalt. Es ist nicht unplausibel zu erwarten, dass die Renditen
dieser dezentralisierten, marktgesteuerten Investitionen doppelt oder
sogar dreimal so hoch sein könnten wie die geringen Erträge aus den
politisch gesteuerten Haushaltszuweisungen der Ära des Nationalstaates.
Es war in den letzten Jahrzehnten des zwanzigsten Jahrhunderts nicht
unüblich, in jedem Land Beispiele für staatliche Investitionen zu
finden, die deutlich negativ waren. In der überarbeiteten Version von
\emph{The Great Reckoning} vom November 1992 zitierten wir offizielle
russische Statistiken, die darauf hindeuteten, dass die gesamte
russische Wirtschaft „nur 30 Milliarden Dollar wert war, weniger als ein
Drittel des Wertes des Rohstoffeinsatzes. Daher würde sich der Output
der russischen Wirtschaft mehr als verdreifachen, wenn die inländische
Fertigungs- und Dienstleistungswirtschaft vollständig stillgelegt wäre.
Statt Wert hinzuzufügen, wird er abgezogen.`` \footnote{Davidson und
  Rees-Mogg, ebenda, S. 203.}

Zugegebenermaßen ist das Beispiel Russlands nach dem Zusammenbruch des
Kommunismus ein extremes, aber es gibt genügend Beweise dafür, dass eine
Verringerung der staatlichen Kontrolle über Ressourcen dazu neigt, die
wirtschaftliche Effizienz zu verbessern. Wachstumsraten, die vom
Economist zitiert werden, legen nahe, dass wirtschaftliche Freiheit
stark mit wirtschaftlichem Wachstum korreliert, wobei die schnellsten
Wachstumsraten in den freiesten Ländern zu verzeichnen sind. Die
Cyberökonomie des Informationszeitalters wird freier sein als jeder
andere kommerzielle Bereich in der Geschichte. Es ist daher vernünftig
zu erwarten, dass die Cyberökonomie schnell zur wichtigsten neuen
Wirtschaft des neuen Jahrtausends werden wird. Ihr Erfolg wird neue
Teilnehmer aus aller Welt anziehen, so wie die breite Verwendung von
Faxgeräten das Fernkopieren auch für Nichtnutzer immer attraktiver
machte. Aber noch wichtiger ist, dass die Freiheit von räuberischer
Gewalt es der Cyberökonomie ermöglichen wird, mit weit höheren
zusammengesetzten Wachstumsraten zu wachsen als herkömmliche
Volkswirtschaften, die von Nationalstaaten dominiert werden.

Dies ist vielleicht der wichtigste Punkt bei der Vorwegnahme der
wirtschaftlichen Auswirkungen des wahrscheinlichen Zusammenbruchs der
monopolistischen Steuer- und Inflationsfähigkeit des Staates. Von
Übergangsschwierigkeiten, die Jahrzehnte dauern könnten, mal abgesehen,
dürften die langfristigen Perspektiven für die Weltwirtschaft sehr
positiv sein. Wenn die Umstände es den Menschen erlauben, die
Schutzkosten zu senken und die Tributzahlungen an diejenigen, die die
organisierte Gewalt kontrollieren, zu minimieren, wächst die Wirtschaft
in der Regel dramatisch. Wie Lane sagte: „Ich möchte behaupten, dass der
wichtigste Einzelfaktor in den meisten Wachstumsperioden, wenn überhaupt
ein Faktor am wichtigsten war, die Verringerung des Anteils der für
Krieg und Polizei aufgewendeten Mittel war.`` \footnote{Lane,
  \emph{Economic Consequences of Organized Violence}, ebenda, S. 413.}:

Es könnten erhebliche Effizienzsteigerungen erzielt werden, wenn weniger
Ressourcen für Raub und das Leben von der Beute des Raubes aufgewendet
würden. Würde der Preis der Sicherheit auf eine Wettbewerbsbasis
gestellt, bei der lokale Monopole auf der Grundlage von Preis und
Qualität um Kunden konkurrieren, könnten potenziell riesige
Effizienzgewinne möglich sein. Das zu erwartende Ergebnis wären weit
geringere Steuersätze und weniger Verlust von Ressourcen und Aufwand in
der politischen Aktivität, die nicht mehr die gewaltigen zuvor erzielten
Gewinne einbringen würde.

Würden Wähler bereitwillig auf politische Vorteile verzichten, an die
sie sich gewöhnt haben? Das ist eine Frage, die wir an anderer Stelle
ausführlich diskutieren. Aber eine einfache Antwort wäre, dass sie
vielleicht keine Wahl haben. Niemand demonstriert jetzt gegen
Regenwetter oder Dürre, egal wie wirtschaftlich schädlich oder
unangenehm es sein mag. Niemand, egal wieviel kriminelle Energie er
besitzt, hält einen Bettler als Geisel und fordert unter Todesdrohung
ein hohes Lösegeld. Wenn es für Politiker unmöglich wird, Ressourcen zur
Umverteilung zu erlangen, könnte die Öffentlichkeit rational reagieren
und die Politik vergessen, genauso wie gut meinende Menschen aufgehört
haben, Bußprozessionen zu organisieren, als das Mittelalter zu Ende
ging.

\setsubtitle{Die Revolution der Einkommensfähigkeit in einer Welt ohne Arbeitsplätze}

\bookmarksetup{startatroot}

\chapter{DAS ENDE DER EGALITÄREN
ÖKONOMIE}\label{das-ende-der-egalituxe4ren-uxf6konomie}

\begin{quote}
„Gott lässt sich nicht spotten: Denn was der Mensch sät, das wird er
ernten.`` - Galater 6:7
\end{quote}

Große Veränderungen in den vorherrschenden Produktions- oder
Verteidigungsformen verändern die Struktur der Gesellschaft und das
Verhältnis von Reichtum und Macht unterschiedlicher Gruppen. Das
Informationszeitalter bedeutet mehr als nur den wachsenden Einsatz
leistungsstarker Computer. Es bedeutet eine Revolution in Lebensstilen,
Institutionen und der Verteilung von Ressourcen. Da die Rolle von
verdeckter Gewalt bei der Kontrolle von Ressourcen stark abnehmen wird,
wird sich eine neue Zusammensetzung von Reichtum entwickeln, ohne die
zwangsvermittelnde Rolle der Regierung, die das 20. Jahrhundert prägte.
Weil der Standort in der Informationsgesellschaft viel weniger bedeuten
wird, wird die Bedeutung aller Organisationen, die eher innerhalb als
jenseits geografischer Grenzen operieren, in der Zukunft abnehmen.
Politiker, Gewerkschaften, regulierte Berufe, Lobbyisten und Regierungen
als solche werden, weniger wichtig sein. Da Vetternwirtschaft und
Handelshemmnisse, die von Regierungen erzwungen wurden, weniger wertvoll
sein werden, werden weniger Ressourcen verschwendet, um Lobbyarbeit zu
fördern oder zu bekämpfen.

Diejenigen, die Zwang und lokale Vorteile genutzt haben, um Einkommen
umzuverteilen, sind dazu bestimmt, einen Großteil ihrer Macht zu
verlieren. Dies wird die Herrschaft über die Ressourcen verändern.
Privat generiertes Vermögen, das bisher vom Nationalstaat beansprucht
wurde, wird stattdessen von denen einbehalten, die es verdienen. Es
werden immer mehr Vermögen in die Hände der fähigsten Unternehmer und
Risikokapitalgeber weltweit gelangen. Globalisierung, zusammen mit
anderen Merkmalen der Informationswirtschaft, wird dazu tendieren, das
Einkommen der talentiertesten Personen in jedem Bereich zu erhöhen. Da
der Grenzwert, der durch herausragende Leistungen erzeugt wird, so enorm
sein wird, wird die Verteilung der Verdienstmöglichkeiten in der
gesamten Weltwirtschaft ähnlich aussehen wie jetzt in Leistungsberufen
wie der Leichtathletik und der Oper.

\section{EINE GRÖSSENORDNUNG JENSEITS DES
PARETO-PRINZIPS}\label{eine-gruxf6ssenordnung-jenseits-des-pareto-prinzips}

Das Pareto-Prinzip besagt, dass 80 Prozent des Nutzens von 20 Prozent
der Betroffenen stammt oder auf sie zurückzuführen sind. Dies mag
ungefähr zutreffen, obwohl auffällig ist, dass ein Prozent der
Bevölkerung der Vereinigten Staaten 28,7 Prozent der Einkommenssteuer
zahlt. Dies legt nahe, dass Gesellschaften mit dem Fortschreiten ins
Informationszeitalter eine noch schiefere Verteilung von Einkommen und
Fähigkeiten erleben werden, als Vilfredo Pareto sie am Ende des letzten
Jahrhunderts beobachtet hat. Die Menschen sind es durchaus gewohnt,
große Ungleichheiten im Reichtum zu haben. Im Jahr 1828 sollen 4 Prozent
der New Yorker 62 Prozent des gesamten Reichtums der Stadt besessen
haben. Bis 1845 besaßen die obersten 4 Prozent etwa 81 Prozent des
gesamten Unternehmens- und Nicht-Unternehmensvermögens in New York City.
Im weiteren Sinne besaßen die obersten 10 Prozent der Bevölkerung etwa
40 Prozent des Reichtums in den gesamten Vereinigten Staaten im Jahr
1860. Den Aufzeichnungen zufolge besaßen die reichsten 12 Prozent dann
etwa 86 Prozent des amerikanischen Vermögens im Jahr 1890.\footnote{Benjamin
  Schwarz, \emph{\textasciitilde American Inequality: Its History and
  Scary Future}, New York Times, 19. Dezember 1995, S. A25.}

Die Zahlen von 1890 liegen nahe an dem, was Pareto im Sinn hatte. Sie
weichen hauptsächlich von seinem Verhältnis von 80:20 Prozent ab, weil
gegen Ende des neunzehnten Jahrhunderts ein massiver Zustrom mittelloser
Einwanderer Amerika erreichte. Der Anteil der Einwanderer am
Gesamtvermögen war vernachlässigbar; daher machte ihre Ankunft die
gesamten Vermögensbestände ungleicher. Tatsächlich handelt es sich
hierbei um eine eindrucksvolle Veranschaulichung der Tatsache, dass ein
echter Aufstieg der Chancen fast zwangsläufig mindestens zu einem
kurzzeitigen Anstieg der Ungleichheit führen muss. Bis 1890 machten
Einwanderer etwa 15 Prozent der gesamten Bevölkerung der USA aus, in
einigen der nordöstlichen Staaten, in denen ein Großteil des Einkommens
und des Vermögens generiert wurde, betrug ihr Anteil aber mehr als 40
Prozent.\footnote{Adna Ferrin Weber, \emph{The Growth of Cities in the
  Nineteenth Century} (New York: Macmillan, 1899; Neuauflage von Cornell
  University Press, 1963), S. 249.} Unter Berücksichtigung des Anstiegs
der Einwanderung passte das Amerika des späten neunzehnten Jahrhunderts
fast genauso gut in Paretos Formel wie die Schweiz des späten
neunzehnten Jahrhunderts, in der er lebte.

Das Informationszeitalter hat bereits die Verteilung des Reichtums
verändert, insbesondere in den Vereinigten Staaten, und gehört zu den
Gründen für die Bitterkeit der modernen amerikanischen Politik, die wir
im nächsten Kapitel näher untersuchen werden. Das Informationszeitalter
erfordert für den wirtschaftlichen Erfolg einen recht hohen Standard an
Alphabetisierung und Rechenfähigkeiten. Eine umfangreiche Untersuchung
des US-Bildungsministeriums, „Erwachsenenbildungsstand in Amerika``, hat
gezeigt, dass bis zu 90 Millionen Amerikaner von über fünfzehn Jahren
unzureichende Kompetenzen besitzen. Oder, in der bildhafteren
Charakterisierung des amerikanischen Auswanderers Bill Bryson
ausgedrückt: „Sie sind so dumm wie Schweinesabber.`` \footnote{Bill
  Bryson, \emph{The Lost Continent} (New York: Harper Perennial, 1989),
  S. 72.} Insbesondere wurden 90 Millionen erwachsene Amerikaner als
unfähig eingestuft, einen Brief zu schreiben, einen Busfahrplan zu
verstehen oder zu addieren und zu subtrahieren, selbst mit Hilfe eines
Taschenrechners. Diejenigen, die keinen gewöhnlichen Busfahrplan
verstehen können, werden wahrscheinlich nicht viel von der
Informationsautobahn mitbekommen. Dieses Drittel der Amerikaner, die
sich nicht auf die elektronische Informationswelt vorbereitet haben,
bilden eine wütende Unterschicht. An der Spitze der Gesellschaft steht
eine kleine Gruppe, vielleicht 5 Prozent, von hochgebildeten
Informationsarbeitern oder Kapitalbesitzern, die im
Informationszeitalter das Äquivalent der Landaristokratie des
Mittelalters bilden - mit dem entscheidenden Unterschied, dass die Elite
des Informationszeitalters Spezialisten in Produktion, nicht in Gewalt
sind.

\subsection{Die Megapolitik der
Innovation}\label{die-megapolitik-der-innovation}

Aus nicht eindeutig nachvollziehbaren Gründen haben die meisten
Soziologen des zwanzigsten Jahrhunderts angenommen, dass technologischer
Fortschritt tendenziell zu immer egalitäreren Gesellschaften führen
würde. Dies war jedoch vor etwa 1750 nicht der Fall. Zu dieser Zeit
begann die innovative neue Industrietechnologie Arbeitsmöglichkeiten für
Ungelernte zu eröffnen und die Unternehmensgröße zu erhöhen. Die neue
Technologie der Fabrik erhöhte nicht nur das tatsächliche Einkommen der
Armen ohne ihr Zutun; sie steigerte auch die Macht politischer Systeme,
die dadurch nicht nur besser Einkommen umverteilen konnten, sondern auch
widerstandsfähiger gegen Unruhen wurden. Aus einer langfristigen
Perspektive heraus gibt es keinen inhärenten Grund anzunehmen, dass
Technologie immer tendenziell dazu neigt, Unterschiede in menschlichem
Talent und Motivation zu verschleiern, anstatt sie zu betonen. Einige
Technologien waren verhältnismäßig egalitär und erforderten Beiträge
vieler unabhängiger Arbeiter mit annähernd gleicher Nützlichkeit; andere
legten Macht oder Reichtum in die Hände weniger Meister, während die
meisten Menschen kaum mehr als Leibeigene waren. Geschichte und
Technologie haben unterschiedliche Nationen in unterschiedlicher Weise
geprägt. Das Fabrikzeitalter produzierte die eine Form, und das
Informationszeitalter produziert eine andere, weniger gewalttätige, und
daher elitärere und weniger egalitäre als die, die es ersetzt.

\section{AMMONS RÜBE}\label{ammons-ruxfcbe}

Im späten neunzehnten Jahrhundert begann eine Reihe von Ökonomen, unter
denen William Stanley Jevons in England der renommierteste war, mit der
Entwicklung der mathematischen Wirtschaftswissenschaften. Einer der
Ersten, der die Wahrscheinlichkeitstheorie auf eine wichtige soziale
Frage anwandte, war der deutsche Wirtschaftswissenschaftler Otto Ammon,
dessen Arbeit erstmals 1899 von Carlos C. Closson ins Englische
übersetzt wurde. Dies geschah in einem Artikel im \emph{Journal of
Political Economy} mit dem Titel „Some Social Applications of the
Doctrine of Probability.`` \footnote{Dieser Artikel erschien in der 4.
  Auflage von Adrian Darnell\textquotesingle s collection, \emph{Early
  Mathematical Economists}, 6 vols. (London: Pickering \& Chatto, 1991).}
Man könnte annehmen, dass ein solcher Artikel heute nur noch von
antiquarischem Interesse ist. Tatsächlich behandelt er jedoch ein
ökonomisches Problem, das wieder in den Vordergrund tritt, und zwar auf
eine immer noch inspirierende Art und Weise.

Otto Ammon interessierte sich für die Verteilung von Fähigkeiten in der
Gesellschaft und deren Beziehung zur Verteilung von Einkommen und
Status. Als Ausgangspunkt nahm er das wahrscheinliche Auftreten von
Gesamtpunkten aus vier Würfeln, jeder mit sechs Seiten. Von 1.296
möglichen Würfen treten einige Summen häufiger auf als andere.

\emph{Die Summe von 24 Punkten wird einmal auftreten. Die Summe von 23
Punkten wird 4 mal auftreten. Die Summe von 22 Punkten wird 10 mal
auftreten. Die Summe von 21 Punkten wird 20 mal auftreten. Die Summe von
20 Punkten wird 35 mal auftreten. Die Summe von 19 Punkten wird 56 mal
auftreten. Die Summe von 18 Punkten wird 80 mal auftreten. Die Summe von
17 Punkten wird 104 mal auftreten. Die Summe von 16 Punkten wird 125 mal
auftreten. Die Summe von 15 Punkten wird 140 mal auftreten. Die Summe
von 14 Punkten wird 146 mal auftreten. Die Summe von 13 Punkten wird 140
mal auftreten. Die Summe von 12 Punkten wird 125 mal auftreten. Die
Summe von 11 Punkten wird 104 mal auftreten. Die Summe von 10 Punkten
wird 80 mal auftreten. Die Summe von 9 Punkten wird 56 mal auftreten.
Die Summe von 8 Punkten wird 35 mal auftreten. Die Summe von 7 Punkten
wird 20 mal auftreten. Die Summe von 6 Punkten wird 10 mal auftreten.
Die Summe von 5 Punkten wird 4 mal auftreten. Die Summe von 4 Punkten
wird einmal auftreten.}

Es wird sofort deutlich, dass hohe und niedrige Werte beide
vergleichsweise selten sind. Es gibt zwei mögliche Summen, aber die
oberen vier davon treten nur fünfunddreißig Mal auf. Die mittlere Gruppe
von sieben Werten kann voraussichtlich 884 Mal auftreten; das mittlere
Drittel der möglichen Werte ist das Ergebnis in mehr als zwei Dritteln
aller Würfe. Dies führt zur charakteristischen Häufung in der Mitte in
der Wahrscheinlichkeitstheorie.

Otto Ammon argumentierte, dass diese zufällige Verteilung von
Würfelwürfen der Verteilung menschlicher Fähigkeiten entsprach. Er
schrieb dies, bevor die Entwicklung von Intelligenztests und IQs
stattfand, und stützte sich auf die früheren Arbeiten über Intelligenz
von Francis Galton. Ammon war nicht der Meinung, dass sozialer Nutzen
oder Erfolg im Leben einfach von der Intelligenz abhing. Er listete
„drei Gruppen von geistigen Eigenschaften auf, die weitgehend darüber
entscheiden, welchen Platz ein Mann im Leben einnehmen wird``. Diese
lauteten:

\begin{enumerate}
\def\labelenumi{\arabic{enumi}.}
\item
  \emph{Geistige Eigenschaften}; hierzu zählte ich alles, was zur
  rationalen Seite der menschlichen Fähigkeiten gehört, wie schnelle
  Auffassungsgabe, Gedächtnis, Urteilsvermögen, Erfindungsgeist und
  alles andere, was zu diesem Gebiet gehört.
\item
  \emph{Moralische Eigenschaften}; nämlich Selbstkontrolle,
  Willenskraft, Fleiß, Ausdauer, Mäßigung, Beachtung der familiären
  Verpflichtungen, Ehrlichkeit und dergleichen.
\item
  \emph{Wirtschaftliche Merkmale}; wie unternehmerische Fähigkeiten,
  organisatorisches Talent, technisches Können, Vorsicht, kluge
  Kalkulation, Weitblick, Sparsamkeit und so weiter.
\end{enumerate}

Zu diesen geistigen Merkmalen fügte er hinzu:

\begin{enumerate}
\def\labelenumi{\arabic{enumi}.}
\setcounter{enumi}{3}
\tightlist
\item
  \emph{Körperliche Eigenschaften}; Arbeitskraft, Ausdauer, die
  Fähigkeit, Anstrengungen zu ertragen und Erregungen jeder Art zu
  widerstehen, Vitalität, Gesundheit, usw.
\end{enumerate}

In Otto Ammons Sichtweise entsprach die wahrscheinliche Verteilung
dieser Qualitäten von Intelligenz, Charakter, Talent und Körper den
Punktzahlen im Würfelspiel. Er ging sogar noch weiter und argumentierte,
dass es tatsächlich viel mehr als vier Variablen gab und dass diese in
mehr als sechs Stufen variierten. Würde man anstelle von vier Würfeln
acht werfen, dann gäbe es nicht weniger als 1.679.616 mögliche Würfe,
aber man könnte erwarten, dass die höchste Punktzahl, achtundvierzig,
nur einmal auftritt. Der Mann oder die Frau, die in all den Faktoren,
die den Platz im Leben bestimmen, sehr hoch abschneiden, sind viel
seltener als die Wahrscheinlichkeit, vier Sechser zu werfen, nahelegen
würde; vielleicht so selten wie das Werfen von acht Sechsern. Dennoch
merkt Ammon an, dass eine Mischung aus hohen und niedrigen Punktzahlen
in diesen menschlichen Qualitäten „Personen mit unausgeglichenen,
unharmonischen Gaben`` hervorbringen kann, die „trotz einiger brillanter
Eigenschaften die Prüfungen des Lebens nicht erfolgreich bestehen
können.``

\begin{quote}
„Wie ein einsamer Berggipfel, oder eher wie die Turmspitze einer
Kathedrale, erheben sich die Männer von hohem Talent und Genius über die
breite Masse der Mittelmäßigkeit ... Die Anzahl der hochbegabten ist in
jedem Fall so klein, dass es unmöglich ist, dass ‚viele' solcher in den
niederen Klassen durch die Unvollständigkeit der sozialen Institutionen
zurückgehalten wurden.`` - Otto Ammon
\end{quote}

\subsection{Eigenschaften und
Einkommen}\label{eigenschaften-und-einkommen}

Dann wendete sich Ammon der Einkommensverteilung zu. Natürlich waren die
Statistiken der 1890er Jahre weit weniger hinreichend als sie es heute
wären, aber die deutsche Bürokratie war bereits gut entwickelt, und Otto
Ammon fand in Sachsen, Preußen, Baden und anderen deutschen Staaten
Einkommenskurven, die er sowohl mit seiner wahrgenommenen Verteilung
menschlicher Fähigkeiten als auch mit den Wahrscheinlichkeiten beim
Würfeln gleichsetzte. Er fand ähnliche Zahlen in Charles Booths
\emph{Life and Labour of the People of London} (1892). Tatsächlich sieht
Booths soziale Verteilung ziemlich so aus, wie man es von Ammons
Wahrscheinlichkeitstheorie erwarten würde. Booth stellte in London fest,
dass 25 Prozent arm oder sehr arm waren, 51,5 Prozent ein komfortables
Leben führten und 15 Prozent wohlhabend oder reich waren; wenn man die
zwei niedrigsten Kategorien von Booth nimmt, ergeben sie 9,5 Prozent.
Vor den Wohlfahrtsstaaten des 20. Jahrhunderts war es üblich, von den am
schlechtesten Gestellten als das „versunkene Zehntel`` zu
sprechen.\footnote{Siehe z.B. Weber, ebenda, S. 2.} Die zwei höchsten
Kategorien von Booth summierten sich auf 7 Prozent.

Aus all dem zog Otto Ammon eine Reihe von interessanten
Schlussfolgerungen. Er war der Ansicht, dass die Fähigkeiten der
Menschen, im weitesten Sinne definiert, ihren Platz in der Gesellschaft
und ihr Einkommen bestimmten. Er glaubte, dass hohe Fähigkeiten auf
natürliche Weise dazu führen, dass Menschen im Einkommen und in der
sozialen Position aufsteigen. „Wie ein einsamer Berggipfel, oder eher
wie die Turmspitze einer Kathedrale, erheben sich die Männer von hohem
Talent und Genius über die breite Masse der Mittelmäßigkeit...`` Er
glaubte auch, dass die „wahre Form der sogenannten sozialen Pyramide,
die einer eher flachen Zwiebel oder einer Rübe hat.`` Die Rübe hat einen
schmalen Stiel oben und eine schmale Wurzel unten. Eine solche soziale
Rübe ist als Metapher einer sozialen Pyramide vorzuziehen, denn wie die
moderne Industriegesellschaft hat, sie ihre Masse in der Mitte, während
die Pyramide ihre Masse unten hat.

\subsection{Die Form der Rübe}\label{die-form-der-ruxfcbe}

Moderne Industriegesellschaften sind tatsächlich alle wie Rüben, mit
einer kleinen wohlhabenden und oberen Berufsklasse an der Spitze, einer
größeren Mittelschicht und einer Minderheit als arme Klasse am unteren
Ende. Im Verhältnis zur Mittelschicht sind beide Extreme klein. Im
modernen London, wenn nicht gar in Washington, gibt es sicherlich mehr
Millionäre als Obdachlose.

All das ist faszinierend, doch das unmittelbare Interesse an Ammons
Arbeit liegt in der bedeutenden langfristigen Veränderung, die wir in
den finanziellen und politischen Beziehungen zwischen der Spitze und der
Mitte erleben. Die Fähigkeiten, die im Fabrikzeitalter, das nun zu Ende
geht, benötigt wurden, unterscheiden sich zweifellos von denen, die das
Informationszeitalter erfordert. Die meisten Menschen konnten die
Fähigkeiten erlernen, die erforderlich waren, um die Maschinen der Mitte
des 20. Jahrhunderts zu bedienen, aber diese Arbeitsplätze wurden nun
durch intelligente Maschinen ersetzt, die sich im Grunde genommen selbst
steuern. Ein ganzes Arbeitsfeld mit geringen und mittleren Fertigkeiten
ist bereits verschwunden. Wenn wir richtig liegen, ist dies ein
Vorläufer für das Verschwinden des größten Teils der Beschäftigung und
die Neugestaltung der Arbeit am Effektivmarkt.

\begin{quote}
„Es ist eine Tatsache, die offiziell, aber leise anerkannt wurde, dass
die meisten der arbeitslosen Jugendlichen überhaupt keine
Qualifikationen haben...``\footnote{Clive Jenkins und Barrie Sherman,
  \emph{The Collapse of Work} (London: Methuen, 1979), S. 103.} - Clive
Jenkins und Barrie Sherman
\end{quote}

\section{WENIGER MENSCHEN WERDEN MEHR ARBEIT
LEISTEN}\label{weniger-menschen-werden-mehr-arbeit-leisten}

Wir können die einfache Vier-Würfel-Verteilung der menschlichen
Fähigkeiten nehmen und annehmen, dass die Menschen im Fabrikzeitalter
mit einem Satz von 4 x 2 oder mehr punkten könnten. Das würde bedeuten,
dass über 95 Prozent der Bevölkerung über dem liegen, was Charles Booth
als „die niedrigste Grenze der positiven sozialen Nützlichkeit``
bezeichnete. Tatsächlich wurden in den 1940er und 1950er Jahren 3
Prozent als Vollbeschäftigungsstandard festgesetzt. Nehmen wir an, dass
im Informationszeitalter die erforderliche Punktzahl auf 4 x 3 gestiegen
ist und das erforderliche Minimum von 8 auf 12 gestiegen ist. Das würde
bedeuten, dass fast 24 Prozent unter dieser Grenze der „sozialen
Nützlichkeit`` liegen würden.

Etwas Ähnliches würde am oberen Ende der Skala geschehen. Im
Fabrikzeitalter war das erforderliche Niveau hoher Fähigkeiten
vielleicht bei 4 x 4; nehmen wir an, dass es im Informationszeitalter
auf 4 x 5 gestiegen ist. In diesem Fall würde der Anteil der Menschen,
die für die Spitzenarbeitsplätze qualifiziert sind, die auch am besten
bezahlt werden, von 34 Prozent auf 5 Prozent fallen.

Diese Zahlen sind rein hypothetisch. Natürlich wissen wir nicht, wie die
Verschiebung in den Anforderungen an die Fähigkeiten sein wird - oder
bereits war - aber es hat sicherlich einen Anstieg gegeben. Aufgrund der
Form der Rübe würde ein relativ bescheidener Anstieg der
Mindestanforderungen an die Fähigkeiten eine große Anzahl von Menschen
außerhalb einer bedeutenden wirtschaftlichen Rolle stellen. Ebenso würde
ein recht kleiner Anstieg in den höheren Fähigkeitsanforderungen die
Anzahl der für die höheren Arbeitsplätze qualifizierten Personen sehr
dramatisch reduzieren. Eine Verschiebung findet statt: Wir wissen nur
noch nicht, wie groß sie sein wird.

Es gibt tatsächlich keinen Mangel an sozialen und politischen Beweisen
dafür, dass dieser Wandel in allen fortgeschrittenen
Industriegesellschaften stattfindet, dass das Tempo dafür zunimmt, und
dass die Bewegung bereits weit fortgeschritten ist. Die Belohnungen für
seltene Fähigkeiten haben zugenommen und nehmen weiter zu. Dies wurde
von konventionellen Denkern mit Unbehagen festgestellt. Betrachten Sie
zum Beispiel \emph{The Winner-Take-All Society} von Robert H. Frank und
Philip J. Cook.\footnote{Robert H. Frank and Philip J. Cook, \emph{The
  Winner-Take-All Society} (New York: The Free Press, 1995).} Es
dokumentiert die wachsende Tendenz, dass die talentiertesten
Konkurrenten in vielen Bereichen in den Vereinigten Staaten sehr hohe
Einkommen erzielen. Gleichzeitig sinken die Möglichkeiten für mittlere
Fähigkeiten; eine beträchtliche Anzahl von gering qualifizierten
Personen fällt jetzt aus dem Bereich heraus, der mit einem komfortablen
Leben belohnt wird, obwohl sie vielleicht noch einen Platz in kleineren
Dienstleistungen finden.

Wenn das Informationszeitalter sowohl an der Spitze als auch am unteren
Ende höhere Fähigkeiten verlangt, werden alle außer den obersten 5
Prozent relativ benachteiligt sein, aber die obersten 5 Prozent werden
enorm profitieren. Sie werden sowohl einen höheren Anteil des Einkommens
verdienen als auch einen größeren Anteil von dem, was sie verdienen,
behalten. Gleichzeitig werden sie einen größeren Teil der weltweiten
Arbeit leisten als je zuvor. Viele werden sich als souveräne Individuen
herausbilden. Im Informationszeitalter wird die Einkommensverteilung
eher aussehen wie 1750 als wie 1950.

Gesellschaften, die auf Einkommensgleichheit und hohen Konsum für
Menschen mit geringen oder bescheidenen Fähigkeiten konditioniert
wurden, werden Demotivation und Unsicherheit gegenüberstehen. Da die
Volkswirtschaften von immer mehr Ländern die Informationstechnologie
tiefer integrieren, werden sie das Aufkommen einer mehr oder weniger
unbeschäftigbaren Unterschicht erleben - ein Phänomen, das in
Nordamerika bereits deutlich sichtbar ist. Genau das passiert gerade.
Dies wird zu einer Reaktion von nationalistischer, technikfeindlicher
Voreingenommenheit führen, wie wir im nächsten Kapitel detailliert
ausarbeiten werden.

Es könnte sich herausstellen, dass das Zeitalter der Fabriken eine
einzigartige Phase war, in der halb-dumme Maschinen eine äußerst
profitable Nische für ungelernte Menschen hinterließen. Nun, da die
Maschinen in der Lage sind, für sich selbst zu sorgen, verschüttet das
Informationszeitalter seine Gaben auf die oberen 5 Prozent von Otto
Ammons Rübe. Schon für die oberen 10 Prozent, die sogenannte kognitive
Elite, sah das Informationszeitalter vielversprechender aus. Doch am
bestmöglichen wird es für die oberen 10 Prozent der oberen 10 Prozent
sein, die kognitive Doppelspitze. Im feudalen Zeitalter waren hundert
halb-qualifizierte Bauern nötig, um einen hoch-qualifizierten
Kriegsherren (oder Ritter) zu Pferde zu unterstützen. Die souveränen
Individuen der Informationswirtschaft werden keine Kriegsherren sein,
sondern Meister spezialisierter Fähigkeiten, einschließlich
Unternehmertum und Investitionen. Dennoch scheint das feudale Verhältnis
von hundert zu eins zurückzukehren. Zum Besseren oder Schlechteren
scheinen die Gesellschaften des 21. Jahrhunderts ungleicher zu werden
als die, in denen wir im 20. Jahrhundert gelebt haben.

\section{DIE MEISTEN MENSCHEN WERDEN VOM ENDE DER POLITIK
PROFITIEREN}\label{die-meisten-menschen-werden-vom-ende-der-politik-profitieren}

Es ist unwahrscheinlich, dass die egalitäre Wirtschaft und die Nationen,
die sie unterstützt, ohne eine Krise verschwinden können. Obwohl eine
Krise per Definition nur eine kurze Zeit andauern kann, stellen wir uns
dennoch vor, dass das Trauma des Endes der Nationen über Jahre hinweg
nachhallen könnte. Ohne dieses Trauma zu ignorieren, dessen Ausmaße wir
später genauer erforschen, dürfen wir nicht vergessen, dass in vielen
Teilen der Welt der Übergang zur Informationswirtschaft zu einem Anstieg
der Produktion führen wird, mit höheren Einkommen allgemein. Tatsächlich
steigen in jenen Bereichen, die nie vollständig von den Vorteilen der
Industrialisierung profitiert haben, aber nun dem freien Markt
offenstehen, die Einkommen aller Bevölkerungsschichten oder werden
steigen.

Die Abschaffung des Zwanges als Merkmal des Wirtschaftslebens wird es
den Produzenten ermöglichen, Vermögenswerte zu behalten, die bisher
beschlagnahmt und umverteilt wurden. Umverteilung bedeutete in der
Regel, dass Vermögenswerte zu weniger wertvollen Zwecken eingesetzt
wurden und so die Produktivität des Kapitals reduziert wurde. Vermögen,
welches überproportional von Personen genommen wurde, die besonders
investitionsfähig waren, wurden von Politikern an weniger Geschickte
umverteilt. In den meisten Fällen wurde das umverteilte Einkommen in
untergeordnete wirtschaftliche Aktivitäten investiert. Die Auswirkungen
der Befreiung von systematischem Zwang werden zwischen den
Rechtssystemen stark variieren. Diese Einfrierung von Ressourcen wird
Wohlfahrtsstaaten in den Bankrott treiben und die Nachteile großer
Skaleneffekte verstärken, die große Regierungen und alle durch sie
subventionierten Einrichtungen untergraben. Andererseits wird der
Übergang zur Cyberwirtschaft die wirtschaftlichen Nachteile verringern,
die Menschen in Regionen erfahren, die traditionell unter der
Unfähigkeit gelitten haben, in großem Maßstab zu organisieren.

\begin{quote}
„Wenn die Welt wie ein großer Markt agiert, wird jeder Arbeitnehmer mit
jeder Person überall auf der Welt konkurrieren, der in der Lage ist, die
gleiche Arbeit zu verrichten. Es gibt viele von ihnen und viele von
ihnen sind hungrig.`` \footnote{Clay Chandler,
  \emph{Buchanan\textquotesingle s Success Frightens Business},
  Washington Post, 22. Februar 1996, S. D12.} - Andrew S. Grove,
Präsident, Intel Corp.
\end{quote}

\section{DIE VERSCHIEBUNG GEOGRAPHISCHER
VORTEILE}\label{die-verschiebung-geographischer-vorteile}

Da es keine steigenden Renditen durch Gewalt mehr geben wird, wird es
keinen Vorteil mehr darstellen, unter einer Regierung zu leben, die sie
einfordern kann. Einst kompetente Regierungen werden nicht länger
Freunde der Vermögensanhäufung sein, sondern deren Feinde. Hohe Steuern,
lästige regulatorische Kosten und ehrgeizige Pläne zur
Einkommensumverteilung werden die Gebiete unter ihrer Kontrolle zu
unattraktiven Orten für Geschäftstätigkeiten machen.

Diejenigen, die in Regionen leben, die während der industriellen Periode
arm oder unterentwickelt geblieben sind, haben am meisten von der
Befreiung der Volkswirtschaften von den Fesseln der Geographie zu
gewinnen. Dies steht im Gegensatz zu dem, was man oft hört. Die
Hauptkontroverse rund um den Aufstieg der Informationswirtschaft und den
Vormarsch des souveränen Individuums konzentriert sich auf die angeblich
negativen Auswirkungen auf die „Fairness``, die sich aus dem Ende der
Politik ergibt. Es ist sicherlich wahr, dass die Einführung der globalen
Informationswirtschaft den Plänen zur Einkommensumverteilung in großem
Maßstab den Todesstoß versetzen wird. Die Hauptnutznießer der
Einkommensumverteilung im Industriezeitalter waren die Bewohner von
wohlhabenden Gebieten, deren Konsumniveau zwanzigmal höher war als der
weltweite Durchschnitt. Einkommensumverteilung hat nur in den
OECD-Ländern bemerkbare Auswirkungen auf die Einkommen unqualifizierter
Personen gehabt.

Die größten Einkommensungleichheiten wurden \emph{innerhalb} von
Rechtssystemen beobachtet. Einkommensumverteilung hat wenig dazu
beigetragen, diese Ungleichheiten zu beseitigen. In der Tat sind wir der
Meinung, dass ausländische Hilfe und internationale
Entwicklungsprogramme den paradoxen Effekt hatten, die realen Einkommen
armer Menschen in armen Ländern zu senken, indem sie inkompetente
Regierungen subventionierten. Dieses Problem analysieren wir genauer,
wenn wir die Auswirkungen der Informationsrevolution auf die Moral
betrachten.

\subsection{Ein Jahrhundert steigender
Einkommensungleichheit}\label{ein-jahrhundert-steigender-einkommensungleichheit}

Während des industriellen Zeitalters war der Faktor, der am meisten dazu
beitrug, das lebenslange Einkommen einer normalen Person zu bestimmen,
die politische Zuständigkeit, in der sie sich zufällig befand. Im
Gegensatz zum gängigen Eindruck in reichen Volkswirtschaften heute,
stieg die Einkommensungleichheit während der industriellen Periode
rapide an. Eine von der Weltbank zitierte Schätzung legt nahe, dass das
durchschnittliche Pro-Kopf-Einkommen in den reichsten Ländern von elfmal
so hoch wie in den ärmsten Ländern im Jahr 1870 auf fünfzigmal so hoch
im Jahr 1985 anstieg.\footnote{Stephanie Flanders und Martin Wolfe,
  \emph{Haunted by the Trade Spectre}, Financial Times, 24. Juli 1995,
  S. 11. Sie zitieren den jüngsten Bericht der Weltentwicklung der
  Weltbank über Arbeiter in einer integrierten Weltwirtschaft.} Während
die Ungleichheit auf weltweiter Ebene dramatisch zunahm, schien dies für
den Bruchteil der Weltbevölkerung, der in den wohlhabenden
Industrieländern lebte, oft andersherum. Die Einkommensungleichheit
stieg eher innerhalb der Rechtssysteme als zwischen ihnen.

Aus von uns bereits untersuchten Gründen hat der Charakter der
industriellen Technologie selbst dazu beigetragen, dass die
Einkommenslücken in den Zuständigkeitsbereichen, in denen halbwegs
kompetente Regierungen die Ausübung von Macht in großem Maßstab
beherrschten, geringer wurden. Als die Gewinne aus der Gewalt stiegen,
was sie während des Industriezeitalters taten, wurden Regierungen, die
in großem Maßstab operierten, in der Regel von ihren Angestellten
kontrolliert. Dies machte es effektiv unmöglich, Kontrollen über die
Ansprüche aufzuerlegen, die diese Regierungen auf Ressourcen erhoben.
Ihre uneingeschränkte Kontrolle über Ressourcen verschaffte ihnen einen
bedeutenden militärischen Vorteil, solange das Ausmaß der Macht
gegenüber der Effizienz, mit der sie eingesetzt wurde, überwog. Ein
keineswegs unbedeutender Nebeneffekt von Regierungen, die von ihren
Mitarbeitern kontrolliert wurden, war eine starke Beschleunigung der
Einkommensumverteilung. Fast jede Gesellschaft hat eine bestimmte
Regelung zur Einkommensumverteilung, zumindest vorübergehend in
außergewöhnlichen Umständen. Eine genaue Untersuchung der Geschichte der
Bedürftigenhilfe zeigt jedoch, dass „Sozialleistungen`` tendenziell
großzügiger sind, wenn die Armut minimal ist. Es ist wahrscheinlicher,
dass die Einkommensumverteilung eingeschränkt wird, wenn die Einkommen
einer großen Zahl von Menschen sinken. Die Bedingungen in den
wohlhabenden Industriegesellschaften in der zweiten Hälfte des 20.
Jahrhunderts waren nahezu perfekt für die Umverteilung des Einkommens.
Dies führte zu deutlich höheren Belohnungen für ungelernte Arbeit
innerhalb dieser begünstigten Zuständigkeitsgebiete. Schließlich bot es
sogar hohe Konsumniveaus für diejenigen, die überhaupt nicht arbeiteten.

\subsection{Das Paradoxon des industriellen
Reichtums}\label{das-paradoxon-des-industriellen-reichtums}

Die Ironie besteht darin, dass auch in diesen Rechtssystemen mehr
Menschen zu Wohlstand kamen. Dieses scheinbare Paradoxon ergibt Sinn,
wenn man die Dynamik der Megapolitik versteht, die wir in den
vorangegangenen Kapiteln untersucht haben. Führende Sektoren der
Industriewirtschaft erforderten die Aufrechterhaltung von Ordnung in
großem Maßstab, um optimal zu funktionieren. Dies machte sie besonders
anfällig für Erpressung durch Gewerkschaften und Regierungen, die darauf
bedacht waren, die Anzahl der Personen unter ihrer Kontrolle zu
maximieren. Dennoch erstickte die weit verbreitete Umverteilung von
Einkommen die Fähigkeit der Industriewirtschaft zu funktionieren nicht
völlig. Wer das Glück hatte, während der Hochphase der
Industrialisierung in Westeuropa, den ehemaligen britischen
Siedlungskolonien oder Japan geboren zu werden, war daher wahrscheinlich
viel reicher als eine Person mit vergleichbaren Fähigkeiten in
Südamerika, Osteuropa, der späten Sowjetunion, Afrika und den
asiatischen Gebieten. Der positive Einfluss der Informationstechnologie
wird dazu beitragen, viele der Hindernisse für die Entwicklung zu
überwinden, die während eines Großteils der modernen Zeit der Mehrheit
der Weltbevölkerung den Genuss eines Großteils der Vorteile freier
Märkte verwehrten.

\begin{quote}
„Die einheimischen Eigenschaften von armen Ländern sind auffallend
ungeeignet für effektive Großorganisationen, besonders für solche, die
(wie Regierungen) über ein großes geografisches Gebiet operieren
müssen.`` \footnote{Siehe Mancur Olson, \emph{Diseconomies of Scale and
  Development}, Cato Journal, vol.7, no.1 (Spring/Summer 1987).} -
Mancur Olson
\end{quote}

\section{SKALENNACHTEILE UND VERZÖGERTES
WACHSTUM}\label{skalennachteile-und-verzuxf6gertes-wachstum}

Wie Mancur Olson bewiesen hat, war die Rückständigkeit im zwanzigsten
Jahrhundert nicht auf einen Mangel an Kapital oder spezialisierten
Fähigkeiten an sich zurückzuführen. In seinem 1987 veröffentlichten
Aufsatz „Diseconomies of Scale and Development``, welches zwei Jahre vor
dem Fall der Berliner Mauer erschien, schrieb Olson: „Wenn Kapital
tatsächlich in den armen Ländern knapp gewesen wäre, hätte die
‚Grenzproduktivität' und somit die Rentabilität der Nutzung größer sein
müssen als in den wohlhabenden Ländern. Die niedrigen Wachstumsraten
vieler Länder, die nicht unerhebliche Beträge an ausländischer Hilfe
erhalten haben, und die geringe Produktivität einiger moderner Fabriken,
die in armen Ländern gebaut wurden, haben die Glaubwürdigkeit der
Erklärung der Unterentwicklung durch „Kapitalknappheit`` weiter
verringert``.\footnote{Ebenda} Dies muss zutreffen. Hätte ein Mangel an
Kapital oder Fähigkeiten die Hauptbeeinträchtigung dargestellt, wären
die in armen Regionen erzielten Erträge höher gewesen als in
entwickelten Ländern. Sowohl qualifiziertes Personal als auch Kapital
wären in diese Regionen geströmt, bis die Erträge sich ausgeglichen
hätten. Tatsächlich war meist das Gegenteil der Fall. Es gab eine
erhebliche Auswanderung von gebildeten Menschen aus rückständigen
Gebieten. Und die wenigen Glücklichen, die es schafften, an solchen
Orten Kapital anzuhäufen, exportierten es so schnell wie möglich in die
Schweiz und andere fortgeschrittene Länder.

\subsection{Bessere Regierungen können nicht importiert
werden}\label{bessere-regierungen-kuxf6nnen-nicht-importiert-werden}

Olson argumentiert, und wir stimmen zu, dass das wahre Hindernis für die
Entwicklung in rückständigen Ländern der eine Produktionsfaktor war, der
nicht einfach aus dem Ausland geliehen oder importiert werden konnte,
nämlich die Regierung. Dieses Problem verschärfte sich im Laufe des
zwanzigsten Jahrhunderts. 1900 waren Großbritannien und Frankreich
zusammen mit einigen anderen europäischen Ländern in das Geschäft
involviert, kompetente Regierungen in Regionen zu exportieren, in denen
die einheimischen Mächte nicht in der Lage waren, effektiv im großen
Maßstab zu funktionieren. Aber die sich verändernden megapolitischen
Bedingungen im zwanzigsten Jahrhundert erhöhten die Kosten und
verringerten den Ertrag dieser Aktivität. Kolonialismus, oder
Imperialismus, wie er weniger liebevoll genannt wurde, hörte auf, ein
lohnendes Geschäft zu sein. Technologische Verschiebungen erhöhten die
Kosten für die Ausdehnung der Macht vom Zentrum zur Peripherie und
senkten die militärischen Kosten für einen effektiven Widerstand.
Daraufhin zogen sich die imperialistischen Mächte zurück oder hielten
sich nur noch in winzigen Enklaven wie Bermuda oder den Cayman-Inseln
auf.

\begin{quote}
„Wenn der postkoloniale Nationalstaat zu einer Fessel des Fortschritts
geworden war, wie bis zum Ende der 1980er Jahre immer mehr Kritiker in
Afrika zuzustimmen schienen, dann dürfte der Hauptgrund dafür kaum zu
bezweifeln sein. Der Staat war nicht befreiend und schützend für seine
Bürger, ungeachtet dessen, was seine Propaganda behauptete; im
Gegenteil, seine brutale Wirkung war eher einschränkend und
ausbeuterisch, oder er versäumte es schlichtweg, in irgendeinem sozialen
Sinne überhaupt zu funktionieren.`` \footnote{Basil Davidson, \emph{The
  Black Mans\textquotesingle{} Burden: Africa and the Curse of the
  Nation State} (New York: Times Books, 1992), S. 290.} - Basil Davidson
\end{quote}

Die einheimischen Regierungen, die in den nicht von Europäern
besiedelten Ländern an die Stelle der Kolonialherrschaft traten,
rekrutierten ihre Führer und Verwalter aus Bevölkerungsgruppen, die
wenig Erfahrung oder Fähigkeiten in der Führung von Großunternehmen
hatten. In vielen Fällen, insbesondere in Afrika, wurde die von den
abziehenden Kolonialmächten überlassene Infrastruktur schnell
geplündert, zerstört oder dem Verfall überlassen. Telefonleitungen
wurden von Plünderern abgerissen und zu Armbändern verarbeitet. Straßen
wurden nicht mehr instandgehalten. Die Eisenbahnlinien wurden
unbrauchbar, da die Gleisbetten zerfielen und die Lokomotiven ausfielen.
In Zaire war die von den Belgiern aufwendig errichtete
Verkehrsinfrastruktur bis 1990 nahezu völlig verschwunden. Nur ein paar
altersschwache Flussboote waren noch in Betrieb, von denen eines vom
Diktator als eine Art schwimmender Palast übernommen wurde.

Unzuverlässige Kommunikation und Transport spiegeln die Inkompetenz von
rückständigen Nationalstaaten bei der Aufrechterhaltung der Ordnung
wider. Sie hielten die Preise hoch und die Chancen für den Großteil der
Weltbevölkerung minimal. Wie Olson hervorhebt:

\begin{quote}
Erstens zwingen schlechte Transport- und Kommunikationsmöglichkeiten ein
Unternehmen dazu, sich hauptsächlich auf lokale Produktionsfaktoren zu
stützen. Wenn sich der Umfang eines Unternehmens erhöht, muss es weiter
ausholen, um Produktionsfaktoren zu erhalten, und je schlechter die
Transport- und Kommunikationssysteme sind, desto schneller steigen diese
Kosten mit der steigenden Produktion. Der zweite und wichtigere Grund,
warum schlechte Transport- und Kommunikationssysteme effektive
Großunternehmen behindern, ist, dass sie die Koordinierung solcher
Unternehmen erheblich erschweren.\footnote{Olson, ebenda.}
\end{quote}

\subsection{Die Erleichterung der Last schlechter
Regierungen}\label{die-erleichterung-der-last-schlechter-regierungen}

Die ehrgeizigen Armen der Welt, mehr als alle anderen, könnten davon
profitieren, wenn die Informationstechnologie die Fähigkeit, Einkommen
zu erzielen, von dem Ort, an dem man lebt, entkoppelt. Neue
Technologien, wie das digitale Mobiltelefon, ermöglichen Kommunikation,
die unabhängig von der Fähigkeit der lokalen Polizei ist, jeden
Telefonmast eines Rechtssystems vor Kupferdieben zu schützen. Wenn
drahtlose Fax- und Internetverbindungen verfügbar werden, spielt es
keine so große Rolle mehr, ob extrem arme Postangestellte Briefe
stehlen, nur um die Briefmarke zu klauen.

In vielen Fällen ersetzen effektive Kommunikationssysteme sogar die
Notwendigkeit des physischen Transports von Gütern und Dienstleistungen.
Bessere Kommunikation und eine stark gestiegene Rechenleistung machen
nicht nur die Koordination komplexer Aktivitäten billiger und
effektiver; sie verringern auch die Skaleneffekte und zersetzen große
Organisationen. Diese Entwicklungen verringern allgemein die Nachteile,
unter denen Personen in rückständigen Ländern durch das Leben unter
inkompetenten Regierungen gelitten haben. Die Informationsrevolution
wird es viel weniger wichtigmachen, ob Regierungen fähig sind, kompetent
zu funktionieren. Daher wird es für Personen, die in traditionell armen
Ländern leben, einfacher sein, die Hürden zu überwinden, die ihre
Regierungen bisher dem Wirtschaftswachstum in den Weg gestellt haben.

\subsection{Gleiche Chancen im
Informationszeitalter}\label{gleiche-chancen-im-informationszeitalter}

Im Informationszeitalter werden bekannte Standortvorteile schnell durch
Technologie verändert. Die Ertragskapazität für Menschen mit ähnlichen
Fähigkeiten wird weitaus gleichmäßiger, unabhängig davon, in welcher
Rechtsordnung sie leben. Dies hat bereits angefangen. Da Institutionen,
die Macht und lokale Vorteile zur Umverteilung von Einkommen verwendet
haben, an Einfluss verlieren, wird die Einkommensungleichheit innerhalb
von Rechtsordnungen steigen. Der globale Wettbewerb wird auch dazu
neigen, das Einkommen der hochtalentierten Personen in jeder Branche zu
erhöhen, egal wo sie leben, ähnlich wie es jetzt im professionellen
Sport der Fall ist. Der Grenzwert, der durch überlegene Leistung auf
einem globalen Markt erzeugt wird, wird enorm sein.

Während die öffentliche Debatte sich auf die wachsende „Ungleichheit``
in den OECD-Ländern konzentrieren wird, werden Individuen überall eine
weitaus gleichberechtigtere Chance genießen. Sie werden nicht länger in
einem Rechtssystem leben müssen, das in großem Maßstab gut funktioniert,
um erfolgreich zu sein. Angeborene Fähigkeiten und die Bereitschaft, sie
zu entwickeln, werden auf einem gleichberechtigteren Spielfeld gemessen
werden als je zuvor. Rechtliche Vorteile, die während der industriellen
Periode zu einer wachsenden Ungleichheit zwischen reichen und armen
Volkswirtschaften geführt haben, werden sich dramatisch ändern.

\subsection{Höhere Renditen in ärmeren
Gebieten}\label{huxf6here-renditen-in-uxe4rmeren-gebieten}

Die Hindernisse, die Regierungen in ärmeren Regionen dem reibungslosen
Funktionieren freier Märkte in den Weg legen, werden erheblich
verringert, wenn die Cyberwirtschaft in Kraft tritt. Infolgedessen
werden Kapital und Fähigkeiten, die knapp sind, tatsächlich höhere
Renditen in vielen derzeit ärmeren Gebieten erzielen, genau wie die
Entwicklungstheoretiker der 1950er Jahre postulierten. Und sowohl das
Kapital als auch die Fähigkeiten werden viel leichter importierbar sein.
Die aufstrebenden Volkswirtschaften werden in ihrem Produktionsprozess
nicht mehr so stark auf lokale Faktoren angewiesen sein wie im
industriellen Zeitalter. Ihre verbesserte Fähigkeit, Kapital und
Expertise aus der Ferne zu beziehen, wird zu höheren Wachstumsraten
führen. Dies wird passieren, unabhängig davon, ob inkompetente
Regierungen ehrlicher werden oder besser in der Lage sind,
Eigentumsrechte zu schützen. Da sie keine Macht über den Cyberspace
haben, werden schlechte Regierungen einfach nicht mehr in der Lage sein,
die Menschen in ihrem Rechtsraum daran zu hindern, von der
wirtschaftlichen Freiheit zu profitieren.

\subsection{Positive Verstärkung}\label{positive-verstuxe4rkung}

In der neuen Cyberwirtschaft wird die fast vollständige Übertragbarkeit
der Informationstechnologie das Ansammeln vieler in der Industriezeit
entstandener Standortvorteile untersagen. Erhöhter Wettbewerb zwischen
immer mehr Rechtsgebieten wird auf neuen Arten von lokalem Vorteil
beruhen. Souveränität wird kommerziell statt ausbeuterisch sein.
Regierungen werden durch die Kraft des Wettbewerbs verpflichtet sein,
Richtlinien zu erlassen, die denjenigen ihrer Kunden entgegenkommen, die
den größten Beitrag zum wirtschaftlichen Wohlergehen leisten, anstatt
denen, die wenig oder negative wirtschaftliche Beiträge leisten.

Dies wird eine enorme Veränderung gegenüber dem üblichen Vorgehen des
zwanzigsten Jahrhunderts darstellen. Die Ideologie des Nationalstaates
war, dass das Leben auf positive Weise reguliert werden kann und sollte,
indem unerwünschte Ergebnisse subventioniert und erwünschte bestraft
werden. Arm zu sein ist unerwünscht; daher wurden die Armen
subventioniert. Reich zu werden ist wünschenswert; deshalb wurden
Strafsteuern auf die Reichen verhängt, um das Leben „fairer`` zu
gestalten.

Da diese gesamte politische Herangehensweise in einer megapolitischen
Grundlage verwurzelt war, die jeglichem Einspruch trotzte, spielte es
kaum eine Rolle, was die perversen Folgen der Subventionierung von
Fehlfunktionen waren. Auch die Fähigkeiten, die harte Arbeit und der
Einfallsreichtum, mit denen der umverteilte Reichtum erwirtschaftet
wurde, wurden kaum berücksichtigt. Die Ergebnisse wurden im Sinne von
Ansprüchen gemessen. Die politische Auffassung des zwanzigsten
Jahrhunderts ging davon aus, dass die Ergebnisse nur dann „fair`` wären,
wenn sie gleich sind.

\subsection{Das Neue Paradigma}\label{das-neue-paradigma}

Die neuen megapolitischen Bedingungen des 21. Jahrhunderts ermöglichen
es, dass Marktergebnisse in Bereichen reguliert werden, die zuvor von
der Politik dominiert wurden. Das Marktmodell geht davon aus, dass
Ergebnisse besser reguliert werden können, wenn erwünschte Ergebnisse
belohnt und unerwünschte bestraft werden. Arm zu sein, ist unerwünscht,
und reich zu werden, ist wünschenswert. Daher sollten Anreize die
Schaffung von Reichtum belohnen und Menschen dazu ermutigen, für die
Ressourcen, die sie verbrauchen, zu bezahlen. Das Leben ist „fairer``,
wenn Menschen mehr von dem behalten dürfen, was sie verdienen.

Dies ist eine Ansicht, die im neuen Jahrtausend häufiger zu hören sein
wird als im nun endenden Jahrhundert. Darüber hinaus wird sie noch
eindringlicher sein, weil sie megapolitisch begründet ist. Kapital im
Informationszeitalter wird mit jedem Augenblick mobiler. Die
Möglichkeit, ein hohes Einkommen zu erzielen, ist nicht mehr an den
Wohnsitz von bestimmten Orten gebunden, wie es einst der Fall war, als
der größte Teil des Reichtums durch die Nutzung natürlicher Ressourcen
entstand. Mit jedem Tag, der vergeht, wird es für Menschen, die
hochmobile Informationstechnologie nutzen, einfacher, Vermögenswerte zu
schaffen, die weit weniger der Hebelwirkung von Gewalt ausgesetzt sind
als jede frühere Form von Reichtum.

Willkürliche politische Regulierungen, die Kosten verursachen, ohne
gleichzeitig marktfördernde Vorteile zu generieren, werden bald nicht
mehr realisierbar sein. Starke Wettbewerbskräfte neigen dazu, die Preise
für Güter, Dienstleistungen, Arbeitskräfte und Kapital auf der ganzen
Welt anzugleichen. Regierungen werden weniger Spielraum haben,
willkürliche Politiken durchzusetzen, als sie es gewohnt sind. Jede
Regierung, die versucht, strengere Regulierungen für eine Tätigkeit
aufzuerlegen als andere Souveränitäten, wird diese Tätigkeit einfach
fernhalten. In einigen Fällen wird das Vertreiben unerwünschter
Aktivitäten den Markt natürlich erfreuen und diese Rechtsräume umso
beliebter und wohlhabender machen. In diesem Sinne können bestimmte
Regulierungen mit den Hausregeln verglichen werden, die von den
Besitzern einer Hotelkette durchgesetzt werden. Wenn sie es Menschen
verbieten, barfuß zu gehen oder in der Lobby zu rauchen, werden sie
zweifellos einige Kunden verlieren. Aber das Vertreiben dieser Kunden
wird der Zuständigkeit insgesamt vielleicht nicht einmal Kunden oder
Einnahmen kosten. Gut beschuhte Nichtraucher könnten genau deshalb mehr
bezahlen, weil barfuß rauchende Menschen ausgeschlossen werden. Genauso
können Regulierungen, die es teuer oder unmöglich machen, eine
Verarbeitungsanlage in einer bestimmten Zuständigkeit zu betreiben, die
Verarbeitung anderswo ohne Einnahmeverlust für die Gesamtzuständigkeit
verlagern.

Diese Beispiele zeigen, wie Regulierungen in seltenen Fällen eher einen
positiven als negativen Marktwert haben können, insbesondere in einer
Welt mit rasant wachsender Anzahl an Gerichtsbarkeiten. Regeln, die hohe
Standards in Bezug auf öffentliche Gesundheit, saubere Luft und sauberes
Wasser bewahren, werden an vielen Orten sehr geschätzt. Das Gleiche gilt
für andere, manchmal exotischere Regulierungen und Vereinbarungen, wie
sie vielleicht von Immobilienentwicklern oder Hotels auferlegt werden,
die bestimmte Marktsegmente bedienen.

\subsection{Keine Zölle im
Cyberspace}\label{keine-zuxf6lle-im-cyberspace}

Wir gehen davon aus, dass die Kommerzialisierung der Souveränität rasch
zur Dezentralisierung vieler großer territorialer Souveränitäten führen
wird. Die Tatsache, dass die Informationstechnologie nicht den
Grenzkontrollen unterworfen werden kann, die immer noch den Handel mit
Industrie- und Agrarprodukten behindern können, hat wichtige
Auswirkungen. Sie bedeutet, dass Protektionismus mit der Zeit immer
weniger wirksam sein wird, da der Handel mit Informationen physische
Produkte bei der Generierung von Wohlstand verdrängen wird. Es bedeutet
auch, dass kleinere Regionen immer weniger von der Aufrechterhaltung
umfangreicher politischer Zuständigkeiten abhängig sein werden, um den
Zugang zu Märkten zu gewährleisten, auf denen sie Einkommen erzielen
können.

Die Informationstechnologie setzt Menschen, die in früher geschützten
Dienstleistungssektoren arbeiten, dem ausländischen Wettbewerb aus. Wenn
eine Firma in Toronto vor zwanzig Jahren einen Buchhalter einstellen
wollte, musste diese Person physisch in Toronto oder in einer
nahegelegenen Gemeinde wohnen, die für den Pendelverkehr erreichbar war.
Im Zeitalter der Information könnte ein Buchhalter in Budapest oder
Bangalore, Indien, die Arbeit erledigen und alles benötigte Material in
verschlüsselter Form über das Internet herunterladen. Sofortige
Kommunikation durch Satellitenverbindungen erreicht jeden Teil der Welt
per Modem und Fax in nur einem Moment. Jemand, der Aktienanalysten
braucht, könnte statt eines Wall Street Analysten, siebenundzwanzig
davon zum gleichen Preis in Indien einstellen. Da sich die
Informationstechnologie alle achtzehn Monate um ein Vielfaches
verbessert (Moores Gesetz), werden immer mehr
Dienstleistungssektor-Angestellte einem Preiswettbewerb ausgesetzt, der
jenseits der Fähigkeit von Politikern liegt, ihn zu verhindern. Dieser
Wettbewerb wird schließlich für die akademischen Berufe ebenso gelten
wie für die Buchhalter. Digitale Anwälte und Cyber-Ärzte werden in der
Informationswirtschaft vermehrt auftreten.

\subsection{Sterbehilfe für
Nationalstaaten}\label{sterbehilfe-fuxfcr-nationalstaaten}

Mit den wirtschaftlichen Vorteilen, die ehemals innerhalb der Grenzen
von Nationalstaaten festgehalten wurden, die nun zerfallen, werden die
Nationalstaaten letztendlich unter ihren schweren Verpflichtungen
zusammenbrechen. Aber die Tatsache, dass alle Nationalstaaten auf der
Todesliste stehen, bedeutet nicht, dass sie alle zum gleichen Zeitpunkt
sterben werden. Weit davon entfernt. Der Dezentralisierungsdruck wird in
der Regel in großen politischen Einheiten am stärksten sein, in denen
die Einkommen des größten Teils der Bevölkerung stagnieren oder sinken.
Jurisdiktionen in Lateinamerika und Asien, in denen das
Pro-Kopf-Einkommen schnell steigt, können Generationen lang bestehen
bleiben, oder bis die lebenslangen Einkommensperspektiven dort denen in
den ehemals reichen Industrieländern entsprechen. An diesem Punkt wird
es keine einfachen Kostensubstitutionen mehr zu gewinnen geben, und die
Politik des Wachstums wird herausfordernder werden.

Wir vermuten ebenfalls, dass Nationalstaaten mit einer einzigen großen
Metropole länger bestehen bleiben werden als solche mit mehreren großen
Städten, was mehrere Interessenszentren mit ihren verschiedenen
Umgebungen impliziert.

Ein weiterer Anreiz zur Dezentralisierung wird die hohe Verschuldung der
Zentralregierung sein. Drei der wohlhabendsten Industrieländer mit der
höchsten relativen Verschuldung - Kanada, Belgien und Italien - sind
nicht zufällig Nationen mit fortgeschrittenen Separatistenbewegungen.
Alle drei Länder haben unter chronischen Haushaltsdefiziten gelitten und
haben jetzt nationale Schulden, die über 100 Prozent des
Bruttoinlandsprodukts hinausgehen. Da die Staatsverschuldungen in jedem
Land gestiegen sind, hat die Anziehungskraft der Separatistenbewegungen
ebenfalls zugenommen. In Italien hat sich die Lega Nord als eine
dynamische und beliebte regionale politische Bewegung entwickelt. Ihre
Plattform basiert auf einer einfachen mathematischen Beobachtung:
Norditalien, oder „Padanien``, wäre reicher als die Schweiz, wenn nicht
große Teile seines Einkommens abgezogen würden, um Rom und den ärmeren
Süden zu subventionieren. Die Lega Nord schlägt eine offensichtliche
Lösung vor: Sie will sich von Italien abspalten und so einigen der
schlimmsten Konsequenzen des Zinseszinses entkommen. Ebenso in Belgien,
wo die Staatsverschuldung über 130 Prozent des BIP liegt, zanken die
Flamen und Wallonen wie ein zerstrittenes Paar vor einer Scheidung. Eine
wachsende Minderheit unter den Flamen argumentiert, dass sie die
Wallonen unfair subventionieren, und dass eine Aufteilung Belgiens in
zwei Teile ihre wirtschaftliche Situation verbessern könnte.

Kanadas Fall unterscheidet sich in dem Detail, dass Französisch-Kanada,
die Hauptregion, die jetzt für Separatismus eintritt, historisch gesehen
von Englisch-Kanada subventioniert wurde. Aber mit der steigenden
Bundesschuld und dem wachsenden Defizit wird in Quebec allmählich die
Erkenntnis geweckt, dass diese Form der Einkommensumverteilung abnehmen
wird. Der Bloc Quebecois kokettiert deshalb mit einem Reiz, den er vor
einem Jahrzehnt nicht hatte - dem Versprechen, das nach Steuerabzug
verbleibende Einkommen durch Abschaffung der kanadischen
Bundesteuerzahlung zu erhöhen. Separatistenführer schlagen ebenfalls
vor, dass Quebec Kanada verlassen sollte, ohne einen anteiligen Teil der
Bundesschuld zu übernehmen.

Englischsprachige Kanadier sträuben sich gegen dieses Argument und
neigen dazu, es abzulehnen, weil sie sich der umfangreichen Transfers,
die im Laufe der Jahre nach Quebec geflossen sind, durchaus bewusst
sind. Nichtsdestotrotz ist die Anziehungskraft des Parti Quebecois
stark, und es scheint nur eine Frage der Zeit zu sein, bis ein
Referendum zur Sezession Kanada auflöst. Ein ähnliches Schicksal
erwartet andere Nationalstaaten, wenn ihre finanziellen Verhältnisse
sich verschlechtern.

Ein weiterer Faktor, der Kanadas langfristiges Überleben bedroht, ist
die Tatsache, dass es ein dünn besiedeltes Land mit weitläufiger
Infrastruktur aus der Industrieära ist, die gewartet werden muss. Der
Übergang ins Informationszeitalter entwertet unweigerlich die physische
Infrastruktur. Da Heimarbeiter die Fabrikarbeiter und Büroangestellten
ersetzen, wird es weniger wichtig, ob Autobahnen und andere Verkehrswege
wieder aufgebaut und gut gewartet werden. Angesichts der knapper
werdenden finanziellen Ressourcen werden sich immer mehr Gruppierungen
im kanadischen Leben, auf die im 18. Jahrhundert von Adam Smith
befürwortete exklusive Finanzierung von öffentlichen Gütern
zurückziehen. Er schrieb in \emph{Der Wohlstand der Nationen}:

\begin{quote}
Würden die Straßen von London auf Kosten der{[}\^{}nationalen{]}
Staatskassen beleuchtet und gepflastert, ist es dann wahrscheinlich,
dass sie so gut beleuchtet und gepflastert wären, wie sie es derzeit
sind, oder sogar zu so geringen Kosten? Die Ausgaben würden, anstatt
durch eine lokale Steuer auf die Bewohner jeder einzelnen Straße,
Gemeinde oder Bezirks in London erhoben, in diesem Fall aus den
allgemeinen Staatskassen finanziert und folglich von allen Einwohnern
des Königreichs besteuert werden, von denen der größte Teil keinen
Nutzen aus der Beleuchtung und Pflasterung der Londoner Straßen
zieht.\footnote{Adam Smith, \emph{The Wealth of Nations}, S. 724. Dieser
  Punkt wurde durch ein Argument von Edwin G. West vorgeschlagen in
  \emph{Adam Smith and Modern Economics} (Aldershot, England: Edward
  Elgar Publishing, 1990), S. 88-89.}
\end{quote}

Man ersetze London mit Toronto und man befindet sich in einer Gleichung,
die vielen Menschen in Alberta und British Columbia durch den Kopf gehen
wird. Die Logik der Dezentralisierung wird sich als ansteckend erweisen.

Wenn Kanada zerfällt, wird dies zu einer deutlichen Zunahme der
sezessionistischen Aktivität im Pazifischen Nordwesten der USA führen.
Die Bewohner von Alaska, Washington, Oregon, Idaho und Montana würden
sich im Wettbewerb mit Alberta und British Columbia als unabhängige
Souveränitäten eindeutig benachteiligt sehen.

\section{NACH DEM NATIONALSTAAT}\label{nach-dem-nationalstaat}

Anstatt von Nationalstaaten wird man zunächst kleinere
Verwaltungsbezirke auf Provinzebene sehen und letztendlich kleinere
Souveränitäten, Enklaven verschiedener Art, ähnlich mittelalterlicher
Stadtstaaten, umgeben von ihren Einzugsgebieten. So ungewöhnlich es
Menschen erscheinen mag, denen die Bedeutung der Politik eingebläut
wurde, wird die Politik dieser neuen Mini-Staaten in vielen Fällen mehr
durch unternehmerische Positionierung als durch politisches Machtgeringe
bestimmt. Diese neuen, fragmentierten Souveränitäten werden
unterschiedliche Geschmäcker bedienen, genau wie Hotels und Restaurants,
indem sie spezifische Vorschriften in ihren öffentlichen Bereichen
durchsetzen, die den Marktsegmenten entsprechen, aus denen sie ihre
Kunden ziehen. Das bedeutet natürlich nicht, dass es keine spezifischen
Probleme gibt, die aus der Organisation des Schutzes auf einer
nomadischen Grundlage entstehen. Diese werden wir im nächsten Kapitel
behandeln.

\begin{quote}
„Stadtluft macht frei`` - Mittelalterliches Sprichwort
\end{quote}

\subsection{Keine Bürger des
Schutzherren}\label{keine-buxfcrger-des-schutzherren}

Trotz dieser Schwierigkeiten findet der menschliche Erfindungsreichtum
in der Regel einen Weg, um Institutionen zu schaffen, die
gewinnbringende Möglichkeiten nutzen, selbst wenn die Nachfrage von
Personen ausgeht, die nur wenig zahlen können. Wenn die potenziellen
Kunden zu den reichsten Menschen auf der Welt gehören, sollte dieser
Trend umso hervorstechender sein. Die Möglichkeit, abzuwandern oder „mit
den Füßen abzustimmen`` besteht, wenn veraltete Produkte, Organisationen
oder sogar Regierungen ihren Reiz verlieren und kaum Aussichten auf
sofortige Verbesserung bieten. Nehmen Sie zum Beispiel das Wachstum
mittelalterlicher Städte, die als sichere Zufluchtsorte für Leibeigene
dienten, die vor der feudalen Unterdrückung flüchteten. Ihre Rolle
könnte vergleichbar sein mit der Rolle neuer Rechtssysteme, die die
bevorstehende Abkehr von Nationalstaaten auffangen. Die Aufnahme von
Ausländern, die vor einem Herrn flüchteten, als „Bürger des
Schutzherren``, widersetzte sich den damals geltenden Konventionen des
Feudalrechts und der bischöflichen Autorität. Aber es war dennoch eine
generell erfolgreiche Alternative für diejenigen, die sie nutzten und
trug maßgeblich dazu bei, den Griff des Feudalismus zu lösen. Wie der
mittelalterliche Historiker Fritz Rorig es ausdrückte, wäre der
Leibeigene eines weltlichen Herrn „nach einem Jahr und einem Tag ein
freier Bürger der Stadt``.\footnote{Fritz Rorig, \emph{The Medieval
  Town} (Berkeley: University of California Press, 1967), S. 28.} Es ist
vernünftig zu erwarten, dass neue institutionelle Zufluchtsorte durch
„neue rechtliche Prinzipien`` entstehen, um staatlichen Bürgern einen
fiskalischen Zufluchtsort zu bieten, genau wie die mittelalterliche
Stadt den feudalen Untertanen, die in ihrem Schatten lebten, Zuflucht
bot.

Der Ökonom Albert O. Hirschman, der sich in \emph{Exit, Voice, and
Loyalty}, erstmals 1969 veröffentlicht, mit den theoretischen Feinheiten
des „Abstimmens mit den Füßen`` befasste, prophezeite, dass
technologische Fortschritte die Wahrscheinlichkeit eines Ausstiegs als
Strategie zur Bewältigung von Staaten im Niedergang erhöhen würden. Er
schrieb: „Nur wenn Länder aufgrund des Fortschritts in der Kommunikation
und der allgemeinen Modernisierung einander zu ähneln beginnen, wird die
Gefahr eines vorzeitigen und übermäßigen Ausstiegs drohen
...``.\footnote{Albert O. Hirschman, \emph{Exit, Voice, and Loyalty}
  (Cambridge: Harvard University Press, 1969), S. 81.} Genau das
passiert gerade. Die Informationstechnologie verringert rasch viele der
Unterschiede zwischen den Jurisdiktionen und macht den Ausstieg zu einer
deutlich attraktiveren Option. Selbstverständlich versteht man unter
„vorzeitigen und übermäßigen Ausstiegen`` in Hirschmans Sprachgebrauch
die Sicht dessen, was für den verlassenen Staat optimal ist. Zweifellos
glaubten die Herren im mittelalterlichen Europa, dass sie unter
„vorzeitigen und übermäßigen Ausstiegen`` ihrer Leibeigenen in Städte
litten, wo diese ihre Freiheit erlangten.

Um zu unserem früheren Beispiel zurückzukehren: Es ist nicht so weit
hergeholt, wie es scheinen mag, dass es eine Reihe von Mini-Staaten
geben wird, die Exilanten, die vor den sterbenden Nationalstaaten
fliehen, Zuflucht bieten. Diese Souveränitäten konkurrieren um die
Bedingungen des Exils. Einige, vielleicht an der Westküste von
Nordamerika, könnten Menschen ansprechen, die nicht rauchen und
intolerant gegenüber Passivrauch von denen sind, die rauchen.
Offensichtlich würden solche Regime bei Rauchern nicht beliebt sein.
Regeln, die ihre Gewohnheit verbieten, werden für viele Raucher als
willkürliche Bestrafung erscheinen.

Im industriellen Zeitalter der Massenpolitik wurden solche
Meinungsverschiedenheiten in politischen Kampagnen ausgetragen, die
letztlich die eine oder andere Gruppe zwangen, sich den Wünschen der
Mächtigeren zu beugen. Es ist jedoch keineswegs zwingend, dass
Auseinandersetzungen über sich gegenseitig ausschließende Entscheidungen
so gelöst werden, dass die Präferenzen großer Teile der Bevölkerung
unterdrückt werden müssen.

Einige Menschen mögen gern Gänseleberpastete, andere Hot Dogs und wieder
andere Sojaquark. Normalerweise müssen sie nicht über ihre
Ernährungsvorlieben diskutieren, da ihre kulinarischen Entscheidungen
nicht miteinander verknüpft sind. Niemand zwingt alle, das gleiche Essen
zu konsumieren. Unter den megapolitischen Bedingungen jedoch, zwangen
die Regierungen in der industriellen Ära viele Arten von kollektiven und
sogar privaten Gütern zur gemeinsamen Nutzung. Warum? Weil es große
wirtschaftliche Vorteile zu erlangen gab, indem man in großem Maßstab
agierte. Es war daher unpraktisch, weite Gerichtsbarkeiten in Enklaven
aufzuteilen, in denen jeder, auch bei wichtigen Entscheidungen, seinen
eigenen Weg gehen konnte. Der von Adam Smith vertretene ausschließende
Ansatz bei der Bereitstellung öffentlicher Güter lässt sich viel
leichter verwirklichen, wenn sich die Zahl der Gerichtsbarkeiten
verzehnfacht oder gar verhundertfacht. Im Informationszeitalter werden
die wachsenden Souveränitäten vielmehr kleine Enklaven sein, als
kontinentale Imperien. Einige könnten nordamerikanische Indianerstämme
sein, die eine Steuerhoheit über ihre Reservate beanspruchen, so wie sie
heute das Recht beanspruchen, Spielkasinos zu betreiben oder unter
Missachtung von Grenzwerten zu fischen.

Da die Informationstechnologie viele der Nachteile der Dezentralisierung
von Handelsräumen beseitigt, wird es für die neuen Souveränitäten
praktisch sein, mehr nach den Prinzipien von Clubs oder
Interessengruppen zu arbeiten als nach denen, die die territorialen
Nationalstaaten regierten. So wie es nicht unerlässlich ist, dass jeder
potenzielle Kunde den gleichen Geschmack für Kleidung hat oder die
gleichen Fernsehprogramme ansieht, wird es weniger wichtig sein, als es
scheinen mag, dass jeder mit den Gemeinsamkeitspunkten übereinstimmt,
die den Führungsstil der fragmentierten Souveränitäten definieren.

Weit verstreute Geschmacksrichtungen werden in weit divergierenden
Stilen fragmentierter Souveränität resultieren, ganz so wie es zunehmend
breitere Auswahlmöglichkeiten im Kleidungsstil oder bei Fernsehsendungen
gibt. Einige Kleinststaaten können sogar wie Hotelgruppen in
Franchisesystemen zusammengeschlossen sein oder gemeinsam operieren, um
Vorteile bei Polizeifunktionen und anderen Restdienstleistungen der
Regierung zu erzielen. Diejenigen, die saubere Straßen mögen und es
verabscheuen, Kaugummi unter Tischplatten zu finden, werden Singapur
ansprechend finden. Fans von Beavis und Butthead wahrscheinlich nicht.
Diejenigen, die ein wildes Nachtleben mögen, werden Macao oder Panama,
oder einen ähnlichen Ort bevorzugen. Kunden, die sich mit den Sitten
eines Rechtssystems unwohl fühlen, werden sich in anderen willkommen
fühlen. Während Salt Lake City möglicherweise rauchfrei sein mag, wird
der neue Stadtstaat in Havanna, vielleicht umbenannt in Monte Cristo,
wahrscheinlich in eine Wolke aus Zigarrenrauch gehüllt sein.

\begin{quote}
„Das bedeutet, dass alle Monopole, Hierarchien, Pyramiden und
Machtstrukturen der Industriegesellschaft diesem ständigen Druck,
Intelligenz an die Ränder aller Netzwerke zu verteilen, weichen werden.
Vor allem aber wird das Mooresche Gesetz die wichtigste Konzentration,
die wichtigste physische Ansammlung von Macht im heutigen Amerika zu
Fall bringen: die Großstadt - die große Ansammlung von Industriestädten,
die jetzt von Lebenserhaltungssystemen über Wasser gehalten wird - mit
360 Milliarden direkten Subventionen von uns allen jedes Jahr.
Großstädte sind übriggebliebenes Gepäck aus dem Industriezeitalter.``
\footnote{Tom Peters und George Gilder, \emph{City vs.~Country: Tom
  Peters \& George Gilder Debate the Impact of Technology on Location},
  Forbes, Februar 1995.} - George Gilder
\end{quote}

Eine eigenartige Ironie der Wiederkehr von Mikro-Souveränitäten oder
„Stadtstaaten`` ist es, dass es mit der Entleerung vieler Städte
zusammenfallen könnte. Die Großstadt war weitgehend ein Artefakt des
Industriezeitalters im Westen. Sie entstand zusammen mit dem
Fabriksystem, um Skaleneffekte bei der Produktion von Produkten mit
hohem natürlichen Ressourceninhalt zu nutzen.

Als das 19. Jahrhundert begann, galten Städte mit mehr als 100.000
Einwohnern als riesig und außerhalb von Asien, wo die
Bevölkerungsstatistiken zweifelhaft waren, gab es keine Städte mit mehr
als einer Million Menschen. Die größte Stadt in den Vereinigten Staaten
im Jahr 1800 war Philadelphia, mit einer Bevölkerung von 69.403. New
York zählte gerade einmal 60.489. Baltimore war mit 26.114 Einwohnern
die drittgrößte Stadt in Amerika.\footnote{Weber, ebenda, S. 21.} Die
meisten der künftigen großen Metropolen Europas hatten
Bevölkerungszahlen, die nach heutigen Standards gering sind. London, mit
einer Bevölkerung von 864.845, war wahrscheinlich die größte Stadt der
Welt. Paris, mit 547.756, war die einzige andere Stadt in Europa mit
mehr als einer halben Million Einwohner im Jahr 1801.\footnote{Ebenda,
  S. 46 for London, S. 73 for Paris.} Lissabons Bevölkerung betrug
350.000.\footnote{Ebenda, S. 120.} Wien hatte eine Bevölkerung von
252.000.\footnote{Ebenda, S. 95.} Berlin hatte 1819 knapp 200.000
erreicht.\footnote{Ebenda, S. 84.} Madrid zählte 156.670
Einwohner.\footnote{Ebenda, S. 119.} Brüssel hatte 1802 eine Bevölkerung
von 66.297. Budapest zählte gerade einmal 61.000.\footnote{Ebenda, S.
  101.}

Die Versuchung ist groß, das Wachstum von Großstädten als direkte Folge
des Bevölkerungswachstums zu betrachten. Aber dies ist nicht unbedingt
der Fall. Jeder Mensch auf der Erde könnte nach Texas gepackt werden,
wobei jede Familie in ihrem eigenen freistehenden Haus mit Garten lebt,
und es wäre immer noch etwas von Texas übrig. Wie Adna Weber in der
klassischen Studie „Das Wachstum von Städten im 19. Jahrhundert``
argumentierte, erklärt das Bevölkerungswachstum allein nicht, warum
Menschen in städtischen Umgebungen leben, anstatt auf dem Land verteilt
zu sein. 1890 hatte Bengalen in etwa die gleiche Bevölkerungsdichte wie
England. Doch die städtische Bevölkerung von Bengalen betrug nur 4,8
Prozent, während es in England 61,7 Prozent waren.\footnote{Ebenda, S.
  5.}

Historisch gesehen waren Städte durch Mauern vom Umland abgeschottet, um
Plünderer und die unteren Klassen fernzuhalten. Das Wachstum der
industriellen Beschäftigung im neunzehnten und zwanzigsten Jahrhundert
führte zur Entstehung von Großstädten. Heute ist die Großstadt höchst
anfällig für Zusammenbrüche, da der Industrialismus zu verblassen
beginnt. Das perfekte Beispiel dafür ist Detroit, die führende
Industriestadt der Mitte des zwanzigsten Jahrhunderts. Einst floss ein
großer Anteil der industriellen Weltproduktion durch Detroit. Jetzt ist
es eine leere Hülle, geplagt von Kriminalität und Chaos. In vielen
Blocks der Innenstadt von Detroit wurden ein oder mehrere verfallene
Gebäude niedergebrannt oder abgerissen, so dass der Eindruck entsteht,
dass die Stadt eine Reihe von Bombenangriffen aus dem Zweiten Weltkrieg
überlebt hat.

Detroit steht als Mahnmal dafür, dass viele Industriestädte nicht mehr
lebensfähig sind. Sie werden zerfallen, da Informationen und Ideen an
Wert gewinnen, im Vergleich zur traditionellen Produktion aus
natürlichen Ressourcen. Vielerorts ist die Großstadt bereits zu groß
geworden, um ihr eigenes Gewicht zu tragen. Um eine Metropole
funktionstüchtig zu halten, müssen eine erhebliche Anzahl von
Unterstützungssystemen effektiv im großen Maßstab funktionieren. Das
enge Zusammenleben von Millionen von Menschen impliziert eine enorme
Zunahme der Verwundbarkeit für Kriminalität, Sabotage und willkürliche
Gewalt. Während der industriellen Ära wurde der Preis für den Schutz vor
diesen Risiken durch die Wirtschaftlichkeit der Hochskalenproduktion
ausgeglichen.

Im Informationszeitalter werden nur Städte, die ihre Unterhaltskosten
durch ein hohes Lebensqualitätsniveau ausgleichen können,
überlebensfähig bleiben. Personen in der Ferne werden nicht mehr
gezwungen sein, sie zu subventionieren. Ein guter Indikator für die
Lebensfähigkeit von Städten ist, ob die Menschen im Zentrum der Stadt
reicher sind als die an ihren Außenrändern. Buenos Aires, London und
Paris werden noch lange einladende Orte zum Leben und Arbeiten sein,
lange nachdem das letzte gute Restaurant in South Bend, Louisville und
Philadelphia geschlossen hat.

\subsection{Länderstaaten}\label{luxe4nderstaaten}

Einige Stadtstaaten könnten sich lediglich als Enklaven erweisen, ohne
angeschlossene Städte. Vielleicht wäre es besser, sie als Dorfstaaten
oder Länderstaaten zu betrachten.

Natürliche Ressourcen werden auch unterschiedlich bewertet. Wenn man
überall Geschäfte machen kann, entscheidet man sich vielleicht dafür,
Geschäfte an einem schönen Ort zu machen, an dem man durchatmen kann,
ohne zu viel krebserzeugende Verschmutzung einzuatmen.
Kommunikationstechnologien, die Sprachhindernisse minimieren, machen es
immer einfacher, fast überall zu leben, wo die Umwelt attraktiv ist.
Dünn besiedelte Regionen mit gemäßigtem Klima und einem großen Vorrat an
fruchtbarem Land pro Kopf, wie Neuseeland und Argentinien, werden
ebenfalls einen vergleichbaren Vorteil genießen, weil sie hohe Standards
in der öffentlichen Gesundheit aufweisen und kostengünstige Produzenten
von Lebensmitteln und erneuerbaren Produkten sind. Solche Produkte
werden von der steigenden Nachfrage profitieren, wenn der Lebensstandard
von Milliarden Menschen in Ostasien und Lateinamerika steigt.

\subsection{Das Inäquivalenz-Theorem}\label{das-inuxe4quivalenz-theorem}

Viele Annahmen von Ökonomen über menschliches Verhalten sind in der
Tyrannei des Ortes verwurzelt. Ein deutliches Beispiel ist Ricardos
„Äquivalenztheorem``, das besagt, dass die Bürger eines Landes, das
große Defizite aufweist, ihre persönlichen Erwartungen in Erwartung
höherer Steuersätze anpassen werden, die in der Zukunft zur Tilgung der
Schulden erforderlich sind. In diesem Sinne gibt es eine „Äquivalenz``
zwischen der Finanzierung von Ausgaben durch Besteuerung und durch
Verschuldung. Zumindest gab es eine solche Äquivalenz zu Beginn des 19.
Jahrhunderts, als Ricardo seine Schriften verfasste. Im
Informationszeitalter wird der rationale Mensch jedoch nicht auf die
Aussicht höherer Steuern zur Finanzierung von Defiziten reagieren, indem
er seine Sparquote erhöht; er wird seinen Wohnsitz verlagern oder seine
Transaktionen anderswo tätigen. Aus demselben Grund, aus dem Produzenten
unter Lieferanten nach den geringsten Kosten suchen, wird jeder noch
stärker motiviert sein, alternative Schutzanbieter zu suchen. Die
Vorteile, dies zu tun, werden die zu erwartenden Margen durch den
Wechsel zu einem neuen Lieferanten von Kunststoffrohren bei Weitem
übertreffen. Das zu erwartende Ergebnis ist, dass souveräne Individuen
und andere rationale Menschen Gerichtsbarkeiten mit großen unbeglichenen
Verbindlichkeiten verlassen werden.

Günstige Regierungen, die wenige Schulden haben und ihren Kunden geringe
Kosten auferlegen, werden im Informationszeitalter die bevorzugten
Wohnorte für die Schaffung von Wohlstand sein. Dies bedeutet, dass die
Aussichten für Geschäfte in Gebieten, in denen die Verschuldung niedrig
ist und die Regierungen bereits umstrukturiert wurden, wie in
Neuseeland, Argentinien, Chile, Peru, Singapur und anderen Teilen Asiens
und Lateinamerikas, viel attraktiver sind. Diese Gebiete werden auch
bessere Plattformen für Geschäfte sein als die unreformierten,
kostenintensiven Volkswirtschaften in Nordamerika und Westeuropa.

\subsection{Die Erosion lokaler
Preisabweichungen}\label{die-erosion-lokaler-preisabweichungen}

Die erheblich reduzierten Informationskosten werden die meisten lokalen
Preisvorteile überflüssig machen. Käufer werden nicht nur in der Lage
sein, eine immense Anzahl von Verkaufsstellen zu durchsuchen, um die
niedrigsten Preise für handelbare Waren zu finden; sie werden auch
Dienstleistungen aus der Ferne nutzen können, um über juristische
Grenzen hinweg einzukaufen. Dies wird es den Menschen erheblich
erleichtern, Eigenschaften von schwer zu analysierenden Produkten wie
Versicherungen zu vergleichen. Und sie werden Handelsbeschränkungen
umgehen, die durch lokale Lizenzierungsverfahren auferlegt werden. Daher
werden Gewinnmargen wahrscheinlich in jedem Bereich sinken, in dem
lokale Preisunterschiede durch zusätzliche Informationen und Wettbewerb
reduziert werden können.

\section{NEUE ORGANISATORISCHE
ANFORDERUNGEN}\label{neue-organisatorische-anforderungen}

Die Cyberwirtschaft wird sich in der Art und Weise, wie ihre Teilnehmer
interagieren, erheblich von der Industriewirtschaft unterscheiden.
Informationstechnologie wird viele der langfristigen organisatorischen
Vorteile von Unternehmen, die aus hohen Transaktions- und
Informationskosten entstehen, beseitigen. Das Informationszeitalter wird
das Zeitalter der „virtuellen Unternehmen`` sein.

Viele Analysten, die über weitaus mehr Kenntnisse in der
Informationstechnologie verfügen als wir, haben völlig verfehlt zu
erkennen, dass sie dazu bestimmt ist, die Logik der
Wirtschaftsorganisation zu transformieren. Die neue Technologie
überwindet nicht nur Grenzen und Barrieren; sie revolutioniert auch die
„internen`` Kosten der Berechnung. Sogar die wenigen Unternehmen, die
nicht der erhöhten grenzüberschreitenden Konkurrenz aufgrund der
verbesserten Informations- und Kommunikationstechnologie ausgesetzt sein
werden, werden neuen organisatorischen Anforderungen ausgesetzt sein.
Rasch fallende Informations- und Transaktionskosten werden die
Größenvorteile entscheidend senken, wodurch viele der Anreize, die
während der industriellen Periode zu langfristigen Unternehmen und
Karriere-Einstellungen geführt haben, hinfällig werden.

\subsection{Warum Unternehmen?}\label{warum-unternehmen}

Die klassischen Ökonomen wie Adam Smith waren fast stumm zur Frage der
Unternehmensgröße. Sie sprachen nicht darüber, was die optimale Größe
von Unternehmen beeinflusst, warum Unternehmen bestimmte Formen
annehmen, oder sogar, warum Unternehmen überhaupt existieren. Warum
stellen Unternehmer Mitarbeiter ein, anstatt jede zu erledigende Aufgabe
unter unabhängigen Auftragnehmern auf dem Auktionsmarkt auszuschreiben?
Der Nobelpreisträger Ronald Coase leitete eine neue Denkrichtung in den
Wirtschaftswissenschaften ein, indem er einige dieser wichtigen Fragen
stellte. Die Antworten, die er half zu formulieren, deuten auf die
revolutionären Folgen der Informationstechnologie für die Struktur des
Geschäfts hin. Coase argumentierte, dass Unternehmen ein effizienter Weg
sind, um Informationsdefizite und hohe Transaktionskosten zu
überwinden.\footnote{Siehe Ronald Coase, \emph{The Nature of the Firm},
  Neuauflage in Louis Putterman und Randall S. Kroszner, eds., \emph{The
  Economic Nature of the Firm: A Reader} 2nd ed.~(Cambridge: Cambridge
  University Press, 1996), S. 89-104.}

\subsection{Informations- und
Transaktionskosten}\label{informations--und-transaktionskosten}

Um zu verstehen, warum, betrachten Sie die Hindernisse, denen Sie
gegenübergestanden hätten, wenn Sie versucht hätten, eine industrielle
Montagelinie ohne, ein einziges Unternehmen zu betreiben, das die
Aktivitäten koordiniert. Im Prinzip hätte ein Automobil produziert
werden können, ohne dass die Produktion unter der Aufsicht eines
einzigen Unternehmens zentralisiert würde. Der Ökonom Oliver Williamson,
mit Coase zusammen, ist ein weiterer Pionier in der Entwicklung der
Theorie des Unternehmens. Williamson definierte sechs verschiedene
Betriebs- und Kontrollmethoden. Darunter die „Entrepreneurmethode``, „in
der jede Arbeitsstation von einem Spezialisten besessen und betrieben
wird.`` \footnote{Zitiert von West, ebenda, S. 58; siehe außerdem Oliver
  E. Williamson, \emph{The Organization of Work: A Comparative
  Insititutional Assessment}, Journal of Economic Behaviour and
  Organisation, vol.1, no.1.} Eine weitere ist die, die Williamson die
„verbundenen Arbeitsstationen`` nennt, bei denen „ein Zwischenprodukt
von jedem Arbeiter über die Stufen hinweg transferiert wird.``
\footnote{Zitiert von West, ebenda, S. 59; siehe außerdem Williamson,
  ebenda.} Es gibt keinen physischen Grund, warum tausende von
Mitarbeitern nicht durch eine Schar unabhängiger Auftragnehmer ersetzt
hätte werden können, die jeweils Raum auf dem Fabrikboden mieten, Teile
ersteigern und anbieten, die Achse zu montieren oder die Kotflügel an
das Chassis zu schweißen. Dennoch würde man vergeblich nach einem
Beispiel für eine industrielle Automobilfabrik suchen, die von
unabhängigen Auftragnehmern organisiert und betrieben wird.

\subsection{Koordinationsprobleme}\label{koordinationsprobleme}

Der Betrieb einer Industrieanlage ohne den Vorteil der Koordinierung
durch ein einziges Unternehmen hätte die meisten der durch den Betrieb
in großem Maßstab zu erzielenden Einsparungen zunichte gemacht. Die
massiven Transaktionsprobleme bei der Koordinierung eines
Flickenteppichs von Kleinunternehmen hätten das Fließband praktisch zum
Erliegen gebracht. Ein solches System hätte unaufhörliche Verhandlungen
zwischen den einzelnen Auftragnehmern erforderlich gemacht. Anstatt sich
auf die Produktion zu konzentrieren, hätte die Vielzahl von
Auftragnehmern oder Unternehmern ihre Zeit und Aufmerksamkeit darauf
verwenden müssen, die Preise für die Komponenten festzulegen und die
Bedingungen ihrer eigenen, sich ständig ändernden Interaktionen
auszuarbeiten. Allein die Überwachung der Produktion wäre ein
schwieriges Problem gewesen.

\subsection{Die Befugnis zu handeln}\label{die-befugnis-zu-handeln}

Bei einer derartigen Anzahl unabhängiger Organisationen, die sich um den
Zusammenbau eines Autos bemühen, wäre die Entwicklung und Überarbeitung
der Modelle ein Alptraum gewesen. Man braucht sich nur vorzustellen, wie
schwierig es für den Konstrukteur gewesen wäre, hunderte von
unabhängigen Auftragnehmern von den Änderungen zu überzeugen, die für
die Einführung eines neuen Modells erforderlich sind. In der Praxis wäre
beinahe einstimmiges Einverständnis erforderlich gewesen. Jeder, der
Widerstand leistet oder irgendeine Änderung in der Spezifikation des
Produkts ablehnt, hätte entweder die Verbesserung des Modells effektiv
verhindert oder die Kosten für dessen Einführung erhöht, was die Gewinne
durch den Betrieb in großem Maßstab weiter gefährdet hätte.

\subsection{Unnötige Verhandlungen}\label{unnuxf6tige-verhandlungen}

Eine Montagelinie, die von unabhängigen Vertragspartnern gemietet (oder
getrennt erworben) worden wäre, hätte zahlreichen Risiken ausgesetzt
sein können, die durch den Betrieb innerhalb eines einzigen Unternehmens
vermieden wurden. Tod, Krankheit oder finanzieller Misserfolg einzelner
Vertragspartner hätte bei Betrieben, die die Zusammenarbeit von
tausenden von Menschen erfordert, um ein einzelnes Produkt unter einem
Dach herzustellen, niemals zum Erfolg geführt. Der Auktionsmarkt wäre
sicherlich in der Lage gewesen, diese Auftragnehmer zu ersetzen. Aber
bei jeder Nachfolge wäre eine Verhandlungslösung erforderlich gewesen,
wie die Übernahme des bisherigen Betreibers durch seinen Nachfolger.
Außerdem wäre eine Vereinbarung über die Übernahme der Miete für die
Werkshalle und vielleicht ein neuer Mietvertrag für die Schweißmaschine
oder die Presse zum Ausstanzen der Rücklichtfassungen erforderlich
gewesen. All dies wäre sehr kompliziert gewesen.

\subsection{Anreizfallen}\label{anreizfallen}

Ein weiteres kritisches Problem einer Produktionslinie unabhängiger
Vertragspartner unter den Bedingungen des industriellen Zeitalters war,
dass die Kapitalanforderungen für die einzelnen Vertragspartner stark
variieren würden. Eine Kunststoffform, die benötigt wird, um zum
Beispiel einen Schalter im Armaturenbrett herzustellen, könnte relativ
kostengünstig sein, während die Ausrüstung zum Gießen eines Motorblocks
oder zum Ausstanzen des Blechs an einem Kotflügel Millionen kostet. Der
hohe Ressourcengehalt und die sequenzielle Natur der Fließbandproduktion
machten Probleme aufgrund hoher Kapitalkosten unvermeidlich, aus
Gründen, die im letzten Kapitel analysiert wurden. Vertragspartner mit
kapitalintensiven Aufgaben wären im Wesentlichen abhängig von der
Zusammenarbeit anderer, um ihre Investitionen amortisieren zu können.
Die Fähigkeit der Auftragnehmer mit höherem Kapitalbedarf, Geld zu
beschaffen und gewinnbringend zu arbeiten, hätte davon abgehangen, dass
sie sich die Mitarbeit vieler anderer Teilnehmer am Prozess sichern,
deren Kapitalkosten weitaus geringer waren. In vielen Fällen hätten sie
diese nicht bekommen.

Es hätte einen erheblichen Anreiz für die Kleinen gegeben, die Großen
auszunutzen. Jene, die weniger Geld benötigten, um ihren speziellen
Funktionsteil an der Fertigungsstraße zu betreiben, hätten davon
profitieren können, in entscheidenden Momenten nicht zu kooperieren. Wie
streikende Arbeiter hätten sie unter diesem oder jenem Vorwand die
Fertigungsstraße stilllegen können, wodurch sie selbst nur geringe
Kosten tragen, aber denen mit größeren Kapitalinvestitionen viel Leid
zufügen würden. Der Produktionsprozess hätte ständigen Manipulationen
unterlegen, wobei kleinere Vertragspartner diejenigen mit höheren
Kapitalkosten durch ihre Fähigkeit, die Produktion zu behindern,
erpressen könnten. Das Manövrieren kleinerer Auftragnehmer, um
Nebenleistungen von den großen Unternehmen zu erpressen, hätte die
Effizienz des Systems reduziert.

\subsection{Die Unternehmenslösung}\label{die-unternehmensluxf6sung}

Kurz gesagt, viele der wirtschaftlichen Vorteile, die während des
Industriezeitalters durch den Betrieb einer Großproduktion erzielt
werden konnten, wären verloren gegangen, wenn die Produktion auf eine
Vielzahl von Einzelunternehmern aufgeteilt worden wäre. Das einzelne
große Unternehmen war trotz seiner sonstigen Einschränkungen ein
effizienter Weg, diese Nachteile zu überwinden. Große Unternehmen waren
bürokratisch. Aber in gewissem Maße waren Bürokratie und Hierarchie
genau das, was während des Industriezeitalters benötigt wurde.
Verwaltungs- und Managementteams überwachten und koordinierten die
Produktion. Eine Unmenge mittlerer Manager leitete Befehle weiter die
Hierarchie hinab und andere Informationen zurück die Befehlskette
hinauf. Zudem bot die Unternehmensbürokratie buchhalterische
Kontrollmechanismen und minimierte Agentur-Probleme, bei denen
Mitarbeiter nicht im besten Interesse des Unternehmens handeln, das sie
beschäftigt. Um anspruchsvolle Buchführung unter den Bedingungen des
Industriezeitalters zu erreichen, war die Arbeit vieler Menschen
erforderlich. Die Aufrechterhaltung einer solchen Verwaltungsbürokratie
war teuer. Sie musste bezahlt werden, unabhängig davon, ob die
Produktion aktiv war oder nicht. Da diese Verwalter über entscheidendes
Wissen zur Betriebsführung verfügten, wurden sie in der Regel über dem
üblichen Marktpreis für ihre Fähigkeiten vergütet.

\subsection{„Organisatorischer
Müßiggang``}\label{organisatorischer-muxfcuxdfiggang}

Die große Anzahl professioneller Manager und Verwalter hatten auch den
Nachteil, dazu zu neigen, die Firma „zu kapern`` und sie in ihren
eigenen Interessen, statt denen der Aktionäre zu betreiben. In der
industriellen Ära war es beispielsweise nicht ungewöhnlich, Firmen zu
finden, die verschwenderisch viel für Büromöbel, Clubmitgliedschaften
und andere Vergünstigungen zahlten, die zwar von der Geschäftsführung
genossen wurden, die aber möglicherweise keine direkte Rendite für die
Investoren erbrachten. In einem komplizierten Geschäft war es unmöglich,
von außen zu überwachen, welche Gemeinkosten wesentlich und welche
Ausgaben Luxus für die Mitarbeiter waren. Es war auch schwierig zu
verhindern, dass manchmal ein beträchtlicher Teil der Firmenmitarbeiter
ihre Aufgaben vernachlässigte. Die Tatsache, dass es technologisch
schwierig war, die Leistung zu überwachen, machte ein großes mittleres
Management erforderlich und gleichzeitig war es schwierig, die
Überwacher zu überwachen. All diese Bedingungen trugen zu dem bei, was
als „organizational slack`` bekannt wurde, ein Begriff, der 1963 von
Richard Cyert und James March in \emph{A Behavioral Theory of the Firm}
geprägt wurde.\footnote{Richard Cyert und James March, \emph{A
  Behavioral Theory of the Firm} (Englewood Cliffs, N.J.: Prentice-Hall,
  1983).} Eine sorgfältige Prüfung ergab, dass zahlreiche reale
Unternehmen ihr Potenzial bei weitem nicht ausgeschöpft haben.

\begin{quote}
„Ob du Ergebnisse erzielst oder nicht, die Bezahlung ist die gleiche.\\
Ob du hart arbeitest oder nicht, die Bezahlung ist die gleiche.\\
Ob du dich kümmerst oder nicht, die Bezahlung bleibt die gleiche.``
\footnote{Chris Dray\textasciitilde{} \emph{Civil Servants Lead Lives of
  Quiet Collusion}, Globe and Mail, 2. Februar 1996, S. A14.} - Chris
Dray
\end{quote}

\subsection{„Nicht mein Problem``}\label{nicht-mein-problem}

Als Unternehmen, das auf Dauer angelegt ist, hat das große
Industrieunternehmen den bereits erwähnten Nachteil, dass es von den
Gewerkschaften ausgeplündert werden kann. Es teilte auch einige Merkmale
der Bürokratie, die in noch übertriebenerer Form in Regierungsbüros
vorhanden sind. Anweisungen wurden von oben herab erteilt. Aufgaben
waren stereotyp und abgegrenzt. Diese Aufgaben wurden oft starr
definiert. Grenzen entstanden zwischen den Jobkategorien, ähnlich denen,
die von den Kartellen zur Regulierung der akademischen Berufe
durchgesetzt wurden. Es wäre vielen während des Industriezeitalters
genauso seltsam erschienen, von einem Buchhalter zu erwarten, dass er
eine durchgebrannte Glühbirne in einer Lampe auf seinem Schreibtisch
wechselt, wie einen Anwalt zur Behandlung einer Grippe zu rufen. Von den
Mitarbeitern wurde weder erwartet, die abgesteckten Grenzen zwischen
starr definierten Funktionen zu überschreiten, noch war es ihnen in den
meisten Fällen überhaupt gestattet.

„Nicht mein Problem`` war ein weit verbreitetes Motto, das den
„organisatorischen Müßiggang`` des Industriezeitalters unterstrich.
Jeder hatte eine genau definierte Rolle mit stereotypen Aufgaben, die
niemand überschreiten durfte, auch wenn dies die Produktivität
gesteigert hätte. Jeder Angestellte in der Unternehmensbürokratie wurde
nach „Qualifikation`` eingestellt, von der angenommen wurde, dass sie
die Leistung in seiner spezifischen Funktion voraussagen könnte. Mit
wenigen Ausnahmen wurde jeder nach einer Arbeitsplatz-Klassifikation
bezahlt, mit mehr oder weniger gleichem Gehalt im gesamten Unternehmen.
Da die spezifische Leistung in den administrativen Hierarchien der
Großkonzerne oft nicht gemessen wurde, ähnlich wie in staatlichen
Bürokratien, wurde die Arbeit in einem gemütlichen Tempo erledigt.
Während das Unternehmen also die Skaleneffekte der Massenproduktion
nutzte, tat es dies auf Kosten anderer Ineffizienzen.

\begin{quote}
„Auf einem Markt macht man etwas nicht, weil es einem gesagt wird oder
weil es auf Seite 30 des strategischen Plans steht. Ein Markt kennt
keine Arbeitsgrenzen... Es gibt keine Befehle, keine Übersetzung von
Signalen von oben, niemand teilt die Arbeit in Pakete auf. Auf einem
Markt hat man Kunden und die Beziehung zwischen einem Anbieter und einem
Kunden ist grundsätzlich nicht organisatorisch, denn sie besteht
zwischen zwei unabhängigen Einheiten.`` \footnote{William Bridges,
  \emph{Jobshift: How to Prosper in a Workplace Without Jobs} (Reading,
  Mass.: Addison-Wesley, 1994), S. 62, 64.} William Bridges
\end{quote}

\subsection{Neue Gebote}\label{neue-gebote}

Die neuen megapolitischen Bedingungen des Informationszeitalters werden
die Logik der Unternehmensorganisation erheblich verändern. Ein Teil
davon ist offensichtlich. Wenn die Informationstechnologie irgendetwas
tut, dann senkt sie die Kosten für die Verarbeitung, Berechnung und
Analyse von Informationen dramatisch. Eine Auswirkung solcher
Technologie besteht darin, die Notwendigkeit der Einstellung großer
Mengen von mittleren Führungskräften zur Überwachung von
Produktionsprozessen zu reduzieren. Tatsächlich ersetzen in vielen
Fällen automatisierte Werkzeugmaschinen die Arbeiter, die stündlich
entlohnt werden. Und dort, wo der Produktionsprozess weiterhin von
Menschenhand geführt wird, wurde der Kontroll- und Koordinierungsprozess
größtenteils automatisiert. Anlagen, die mit Mikroprozessoren
ausgestattet sind, können den Fortschritt der Montagelinie viel
effektiver überwachen als Manager es jemals konnten. Die neue Ausrüstung
kann nicht nur die Geschwindigkeit und Genauigkeit messen, mit der
Menschen arbeiten, sie kann auch automatisch Konten führen und
Komponenten nachbestellen, sobald diese aus dem Inventar entnommen
werden. Selbst die kleinsten Betriebe können sich nun
Finanzkontrollprogramme leisten, die ihre Finanzen mit größerer
Geschwindigkeit und Raffinesse buchen, als selbst die größten
Unternehmen dies vor einigen Jahrzehnten durch ihre
Produktionshierarchien hätten erreichen können.

Die Tatsache, dass Informationstechnologie eine dezentrale, nicht
sequentielle Ausgabe von Produkten mit reduziertem Naturressourceninhalt
ermöglicht, verringert drastisch die Anfälligkeit für Spielchen und
Erpressung, wie wir bereits untersucht haben. Dies sind jedoch nicht die
einzigen Eigenschaften von Informationstechnologie, die sie immer
attraktiver für die Auslagerung von Funktionen machen, die früher von
Mitarbeitern erledigt wurden. Die Kapitalkosten sind geringer. Die
Produktionszyklen sind kürzer. Die unabhängigen Auftragnehmer selbst,
einschließlich der Ein-Personen-Firmen, verfügen über deutlich
ausgefeiltere Informationsnetzwerke. Bald werden sie in der Lage sein,
sich auf eine Vielzahl von digitalen Dienern zu verlassen, um eine
breite Palette von Bürofunktionen zu erfüllen, vom Beantworten des
Telefons bis hin zu Sekretariatsdiensten. Digitale Diener werden
Sekretäre, Werbeagenten, Reisebüros, Bankangestellte und Bürokraten
sein.

\subsection{Das Verschwinden guter
Arbeitsplätze}\label{das-verschwinden-guter-arbeitspluxe4tze}

In zunehmendem Maße werden Individuen, die in der Lage sind, einen
bedeutenden wirtschaftlichen Wert zu schaffen, den größten Teil des von
ihnen geschaffenen Werts für sich selbst behalten können.
Unterstützungspersonal, das bisher einen großen Teil, der von den
Haupterzeugern in einem Unternehmen erzeugten Einnahmen absorbiert hat,
wird durch kostengünstige automatisierte Agenten und Informationssysteme
ersetzt. Dies deutet darauf hin, dass eine Organisation in der Lage sein
wird, sich einer höheren Qualität des Dienstes zu versichern, indem sie
diesen vertraglich auslagert, anstatt die Funktion innerhalb des
Unternehmens zu halten, wo es deutlich schwieriger sein wird, Individuen
für eine gut ausgeführte Aufgabe zu belohnen. Ein virtuelles Unternehmen
wird den größten Teil des „organisatorischen Müßiggangs`` beseitigen,
indem sie die Organisation eliminiert.

„Gute Arbeitsplätze`` werden der Vergangenheit angehören. Ein „guter
Arbeitsplatz``, wie es der Wirtschaftswissenschaftler von Princeton,
Orly Ashenfelter, ausdrückte, „ist ein Arbeitsplatz, der mehr zahlt als
man wert ist.`` \footnote{Siehe Al Ehrbar, \emph{`Re-Engineering' Gives
  Firms New Efficiency, Workers the Pink Slip}, Wall Street Journal, 22.
  Juli 1992, S. A14, zitiert durch Bridges, ebenda, S. 39.} Im
Industriezeitalter existierten viele „gute Arbeitsplätze`` aufgrund
hoher Informations- und Transaktionskosten. Unternehmen wuchsen und
internalisierten eine größere Bandbreite an Funktionen, weil sie so
Skaleneffekte realisieren konnten. Die Aufblähung der Unternehmen wurde
auch durch Steuergesetze gefördert. Die hohen Steuern, die in den
späteren Stadien des Industriezeitalters vorherrschten, überhöhten
künstlich die Vorteile der Gründung eines langlebigen Unternehmens und
der Beschäftigung von Festangestellten. In den meisten Ländern haben
Steuergesetze und Vorschriften die Kosten für Gründung und Auflösung von
Firmen auf Projektbasis erheblich erhöht. Außerdem zwingen sie die
Unternehmer dazu, selbständige Auftragnehmer als Arbeitnehmer zu
betrachten. Gesetzliche Eingriffe haben das Angebot an „guten
Arbeitsplätzen`` vorübergehend weiter erhöht, indem sie es kostspielig
und schwierig machten, einen Arbeitnehmer zu entlassen, auch wenn er nur
wenig zur Produktivität des Unternehmens beiträgt.

Unvermeidlich und logisch sicherte der Charakter von
Unternehmensorganisationen im industriellen Zeitalter, dass die
hochqualifizierten und talentierten Menschen, die einen
überproportionalen Anteil der Wertschöpfung in einer Organisation
lieferten, proportional weniger bezahlt wurden als ihr Beitrag wert war.
Dies wird sich im Informationszeitalter ändern.

Die Mikroprozessor-Revolution erhöht drastisch die Verfügbarkeit von
Informationen und senkt die Transaktionskosten. Dies führt zu einer
Veränderung der Struktur von Unternehmen. Anstelle einer permanenten
Bürokratie werden Aktivitäten um Projekte organisiert, ähnlich wie dies
bereits bei Filmunternehmen der Fall ist. Die meisten der früheren
„internen`` Funktionen des Unternehmens werden an unabhängige
Auftragnehmer ausgelagert. Die industriellen Arbeitnehmer, die „gute
Arbeitsplätze`` hatten, aber wenig beitrugen und sich auf ihre Kollegen
verließen, um für sie „einzuspringen``, werden sich bald auf dem
Effektivmarkt um Verträge bemühen müssen. Gleiches gilt für viele
loyale, fleißige Mitarbeiter. „Gute Arbeitsplätze`` werden zum
Anachronismus, weil Arbeitsplätze im Allgemeinen anachronistisch werden.

Im Extremfall der großen japanischen Unternehmen erwarteten die
Mitarbeiter, einen Arbeitsplatz auf Lebenszeit zu haben. Selbst wenn sie
keine produktive Aufgabe zu erfüllen hatten, wurden sie behalten,
manchmal nur, um „an einem leeren Schreibtisch in einer Fabrikecke`` zu
sitzen. Nun wird aber auch in Japan die aufgeblähte Büroarbeiterschaft
reduziert. Die Schlagzeile einer Geschichte in der International Herald
Tribune erzählte die Geschichte: „Abschied ist ein bitterer Schmerz:
Japans Kultur der Anstellung auf Lebenszeit geht schmerzhaft zu Ende.``
\footnote{Sheryl WuDunn, \emph{Parting Is Such Sour Sorrow:
  Japan\textquotesingle s Job-for-Life Culture Painfully Expires},
  International Herald Tribune, 13. Juni 1996, S. 13.}

Im postindustriellen Zeitalter wird Arbeit aus Aufgaben bestehen, die
man erledigt, nicht als etwas, das man „hat``. Vor der industriellen Ära
war eine dauerhafte Anstellung praktisch unbekannt. Wie William Bridges
es ausdrückte: „Vor 1800 - und in vielen Fällen noch lange danach -
bezog sich \emph{Job} immer auf eine bestimmte Aufgabe oder
Unternehmung, nie auf eine Rolle oder Position in einer Organisation.
... Zwischen 1700 und 1890 findet das Oxford English Dictionary viele
Verwendungen von Begriffen wie \emph{Job-Kutscher, Job-Arzt und
Job-Gärtner} - alle beziehen sich auf Menschen, die einmalig angestellt
wurden. Jobarbeit (ein weiterer häufiger Begriff) war gelegentliche
Arbeit, keine regelmäßige Beschäftigung.`` \footnote{Bridges, ebenda, S.
  31-32.} Im Informationszeitalter werden die meisten Aufgaben, die
früher zur Senkung der Informations- und Transaktionskosten in den
Unternehmen angesiedelt waren, wieder auf den Effektivmarkt verlagert.
Die ``Just-in-time``-Bestandsführung und das Outsourcing sind beide dank
der Informationstechnologie praktisch. Es sind Schritte in Richtung Ende
des Arbeitsplatzes. Schon jetzt haben große Unternehmen wie AT\&T alle
dauerhaften Arbeitsplätze abgeschafft. Positionen in diesem großen
Unternehmen bestehen jetzt aus Zeitarbeit. In Bridges Worten: „Die
Beschäftigung wird wieder vorübergehend und situativ, und Kategorien
verlieren ihre Grenzen.`` \footnote{Ebenda, S. 58.} In der neuen
Cyberwirtschaft werden „unabhängige Auftragnehmer`` über Kontinente
hinweg telearbeiten, um sich auf dem Äquivalent der Informationsära zur
Montagelinie zusammenzuschließen.

\subsection{Hollywood übernimmt}\label{hollywood-uxfcbernimmt}

Das Geschäftsmodell der neuen Informationswirtschaft könnte eine
Filmproduktionsfirma sein. Solche Unternehmen können sehr ausgefeilt
sein, mit Budgets in Höhe von Hunderten Millionen Dollar. Obwohl sie oft
große Operationen sind, sind sie auch temporärer Natur. Eine Filmfirma,
die einen Film für 100 Millionen Dollar produziert, kann sich für ein
Jahr zusammenschließen und sich dann auflösen. Während die Leute, die an
der Produktion arbeiten, talentiert sind, besteht keinerlei
Erwartungshaltung, dass die Arbeit an diesem Projekt gleichbedeutend mit
einem „Arbeitsplatz`` ist. Wenn das Projekt vorbei ist, werden die
Beleuchtungstechniker, Kameramänner, Toningenieure und
Garderobenspezialisten ihrer eigenen Wege gehen. Sie könnten bei einem
anderen Projekt wiedervereint sein, oder auch nicht.

Mit fallenden Skaleneffekten und gleichzeitig sinkenden
Kapitalanforderungen für viele informationsintensive Aktivitäten wird es
einen starken Anreiz für Unternehmen geben, sich aufzulösen.
Geschäftstätigkeiten werden eher ad hoc und temporärer Natur sein.
Unternehmen werden tendenziell kurzlebiger sein. Virtuelle Unternehmen,
die Talente für spezifische Zwecke zusammenbringen, werden effizienter
sein als langjährige Firmen. Da die Verschlüsselung immer verbreiteter
wird und die Besteuerung von Kapital durch Wettbewerb gedrängt wird,
werden künstliche Skaleneffekte, die die Existenz von „dauerhaften``
Firmen aufrechterhalten, weichen. Dies wird passieren, ob die Steuern
nun schnell oder langsam gesenkt werden. Wenn sie schnell gesenkt
werden, werden die künstlichen Kosten für Projektarbeit schneller
verschwinden. Wenn sie langsamer gesenkt werden, wird die Hauptlast der
anachronistisch hohen Steuern auf bestehende Unternehmen fallen, während
neue Unternehmen als virtuelle Konzerne agieren, was ihnen besser
ermöglicht, kostspielige Lasten zu vermeiden, die vom sterbenden
Nationalstaat auferlegt wurden.

Während besondere Fähigkeiten und Talente in der Informationswirtschaft
wichtiger denn je sein werden, werden die meisten künstlichen Grenzen
zwischen den Berufen verschwinden. Fortgeschrittene Informations- und
Speichertechnologien werden die Geschäftsgeheimnisse, und
Fachinformationen von Berufen im Bereich Jura, Medizin und Buchhaltung
für jedermann zugänglich sein. Der wirtschaftliche Wert des
Auswendiglernens als Fähigkeit wird sinken, während die Bedeutung der
Synthese und der kreativen Anwendung von Informationen steigen wird.

Die vollen Auswirkungen dieser Veränderungen werden durch veraltete
Regulierung verlangsamt. Aber auf längere Sicht wird die Macht der
Regierungen, die Cyberwirtschaft zu regulieren, bis hin zum Nullpunkt
schwinden. Jede künstliche Regulierung von Berufsmonopolen, die die
Kosten in die Höhe treibt, ohne dass der Nutzen auf dem Markt bewertet
wird, wird letztlich ignoriert werden.

Es gibt weitere Implikationen der Verschiebung hin zu einer
Informationswirtschaft:

\begin{itemize}
\tightlist
\item
  Lokale Vorschriften, die höhere Kosten verursachen, werden auf eine
  Marktbasis umgestellt.
\item
  Der Wettbewerb zwischen den einzelnen Ländern um die Ansiedlung von
  Tätigkeiten mit hoher Wertschöpfung, die im Prinzip überall
  angesiedelt werden könnten, wird sich verschärfen. Kein Knotenpunkt
  ist zwangsläufig attraktiver als ein anderer.
\item
  Die Geschäftsbeziehungen werden sich zunehmend auf „Vertrauenskreise``
  stützen. Aufgrund der Verschlüsselung, die es Individuen ermöglicht,
  unbemerkt zu stehlen, wird Ehrlichkeit eine hoch geschätzte
  Eigenschaft von Geschäftspartnern sein.
\item
  Patent- und Urheberrechtsregelungen werden sich aufgrund des leichten
  Zugangs zu bestimmten Informationen ändern.
\item
  Schutz wird zunehmend technologisch und nicht mehr juristisch sein.
  Die unteren Klassen werden abgeschottet. Der Übergang zu „Gated
  Communities`` ist nahezu unvermeidlich. Die Ausgrenzung von
  Unruhestiftern ist ein wirksames und traditionelles Mittel zur
  Minimierung krimineller Gewalt in Zeiten schwacher zentraler
  Autorität.
\item
  Massengüter werden wie im Mittelalter stark besteuert und lokal
  verschickt, während Luxusgüter gering besteuert und über große
  Entfernungen verschickt werden.\footnote{Abu-Lughod, ebenda, S. 186.}
\item
  Die polizeilichen Aufgaben werden zunehmend von privaten Wächtern
  übernommen, die mit Handelsvereinigungen verbunden sind.
\item
  In der Übergangsphase könnten private Unternehmen gegenüber
  börsennotierten Firmen im Vorteil sein, da private Unternehmen einen
  größeren Spielraum haben, um den von den Regierungen auferlegten
  Kosten zu entgehen.
\item
  Die Beschäftigung auf Lebenszeit wird verschwinden, da
  „Arbeitsplätze`` zunehmend zu Aufgaben oder „Zeitarbeit`` werden und
  nicht mehr zu Positionen innerhalb einer Organisation.
\item
  Die Kontrolle über die wirtschaftlichen Ressourcen wird sich vom Staat
  auf Personen mit überdurchschnittlichen Fähigkeiten und Intelligenz
  verlagern, da es immer einfacher wird, Wohlstand zu schaffen, indem
  man den Produkten Wissen hinzufügt.
\item
  Viele Angehörige erlernter Berufe werden durch interaktive
  Informationsbeschaffungssysteme verdrängt werden.
\item
  Personen mit geringerer Intelligenz werden neue Überlebensstrategien
  entwickeln, die eine stärkere Konzentration auf die Entwicklung von
  Freizeitbeschäftigungen, sportlichen Fähigkeiten und Kriminalität
  sowie den Dienst an der wachsenden Zahl souveräner Individuen
  beinhalten, da die Einkommensungleichheit innerhalb der
  Gerichtsbarkeiten zunimmt.
\end{itemize}

Politische Systeme, die sich in einer Zeit entwickelt haben, in der die
Gewaltzunahme im Vordergrund stand, müssen schmerzhafte Anpassungen
durchlaufen. Da Effizienz gegenüber der Macht eines Systems immer
wichtiger wird, werden kleine, effiziente Souveränitäten, die ihren
Bürgern mehr Schutz zu geringeren Kosten bieten, immer nachhaltiger.

Wie im Mittelalter gibt es auch heute wieder zunehmende Skalenvorteile
bei der Organisation von Gewalt. Dies spiegelt sich bereits in der
steigenden Anzahl souveräner Einheiten seit dem Fall des Kommunismus
wider. Wir erwarten, dass die Anzahl der Souveränitäten in der Welt sich
schnell vervielfacht, wenn sich die Logik des Informationszeitalters
durch die Erfahrung bestätigt.

Die Macht wird erneut auf kleinster Ebene ausgeübt. Enklaven und
Provinzen könnten sogar feststellen, dass sie im Vergleich zu
länderübergreifenden Nationen erhebliche Vorteile haben, indem sie ihren
„Kunden`` wettbewerbsfähige Konditionen für Souveränitätsleistungen
anbieten. Dies wird sich stark von der schnell sterbenden modernen Ära
unterscheiden, in der keine Einheit überleben konnte, es sei denn, sie
könnte eine militärische Macht kontrollieren, die ausreicht, um ein
Königreich zu beherrschen. In der Vergangenheit, als es noch
Größenvorteile bei der Machtausübung gab, kontrollierten diejenigen, die
am meisten vom Schutz profitierten, wie die wohlhabenden Kaufleute in
den spätmittelalterlichen Stadtstaaten, die Regierung. Aus unserer Sicht
können Sie schon mal anfangen, wieder nach dergleichen zu suchen. Die
Reduzierung von Raublasten und eine effizientere Ressourcenverteilung
sollten in Gebieten, in denen die Kunden Kontrolle über die lokalen
Souveränitäten ausüben, zu einem schnellen Wachstum führen.

Wie wir im Folgenden untersuchen werden, wird die Frage, ob diese
Entwicklungen gegen den Widerstand von Legionen von Verlierern
fortgesetzt werden können oder sollten, zu den wichtigsten Kontroversen
des Informationszeitalters gehören.

\setsubtitle{}

\bookmarksetup{startatroot}

\chapter{NATIONALISMUS, REAKTIONISMUS UND DIE NEUEN
LUDDITEN}\label{nationalismus-reaktionismus-und-die-neuen-ludditen}

\begin{quote}
„Nationalismus ist, selbstverständlich, intrinsisch absurd. Warum sollte
das Glück oder Unglück der Geburt als Amerikaner, Albaner, Schotte oder
Bewohner der Fidschi-Inseln Loyalitäten aufzwingen, die ein
individuelles Leben dominieren und eine Gesellschaft so strukturieren,
dass sie formal in Konflikt mit anderen steht? In der Vergangenheit
existierten lokale Loyalitäten gegenüber Orten, Clans und Stämmen,
Verpflichtungen gegenüber Herrschern oder Grundherren, dynastische oder
territoriale Kriege, aber primäre Loyalitäten lagen bei der Religion,
Gott oder dem Gott-König, möglicherweise beim Kaiser einer Zivilisation
als solcher. Es gab keine Nation. Es gab die Bindung an das Vaterland,
das Land der Vorfahren, oder den Patriotismus, aber vor den modernen
Zeiten von Nationalismus zu sprechen, ist anachronistisch.`` \footnote{William
  Pfaff, \emph{The Wrath of Nations: Civilization and the Furies of
  Nationalism} (New York: Simon \& Schuster, 1993), S. 17.} - William
Pfaff
\end{quote}

Die Aussage, dass die „Welt kleiner wird``, ist eine informative
Redewendung, die von so angesehenen Autoritäten wie der IBM-Werbeagentur
bekräftigt wird. Ihre multikulturellen Werbespots für das Internet mit
dem Titel „Lösungen für einen kleinen Planeten`` erinnern Sportfans, die
es vielleicht selbst nicht erkennen, daran, dass sich die Bedingungen
für die Beziehungen zwischen Menschen in weit entlegenen Gebieten durch
die Technologie verändert haben. Wir verweisen auf den renommierten
Historiker William McNeill in Bezug auf eine nützliche Fußnote zu den
Auswirkungen. Er schreibt: „Die fortschreitende Intensivierung der
Kommunikation und des Transports begünstigt nicht mehr die nationale
Konsolidierung, sondern bewirkt das Gegenteil, da ihre Reichweite die
bestehenden politischen und ethnischen Grenzen überschreitet.\footnote{William
  H. McNeill, \emph{Reasserting the Polyethnic Norm}, in John Hutchinson
  und Anthony D. Smith, eds., \emph{Nationalism} (Oxford: Oxford
  University Press, 1994), S. 300.} In dem Maße, wie die Welt „kleiner``
wird und sich die Kommunikationsmittel verbessern, werden die zufälligen
und „intrinsisch absurden`` Ansprüche der Nationen und des Nationalismus
zwangsläufig schwächer.

\section{DIE GROSSE TRANSFORMATION}\label{die-grosse-transformation}

Das Problem mit dieser vernünftigen Erwartung besteht darin, dass die
gesamte bisherige Geschichte darauf hindeutet, dass sie nicht auf
vernünftige Weise erfüllt werden kann. Der damit einhergehende Übergang
wird eine Krise mit sich bringen. Er beinhaltet eine radikal neue
Denkweise, eine neue Vorstellung von Gemeinschaft, die über
Nationalismus und den Nationalstaat hinausgeht. Wie Michael Billig
hervorgehoben hat, sind „unsere Überzeugungen über die Nation und über
die Natürlichkeit der Zugehörigkeit zu einer Nation`` die „Produkte
eines bestimmten historischen Zeitalters``.\footnote{Michael Billig,
  \emph{Banal Nationalism} (London: Sage Publications, 1995), S. 16.}
Dieses Zeitalter, das moderne Zeitalter, könnte bereits vorbei sein.
Seine vorherrschenden Institutionen, die Nationalstaaten, bestehen zwar
noch weiter, aber sie überleben nur prekär auf einem erodierten
Fundament. Wenn auch dieser Vorhang fällt und die Nationalstaaten
zusammenbrechen, rechnen wir mit einer heftigen Reaktion, insbesondere
in den wohlhabenden Ländern, in denen die „Volkswirtschaft`` im 20.
Jahrhundert ein hohes Einkommen für ungelernte Arbeit brachte. Wir sind
der Meinung, dass die Veränderungen der megapolitischen Bedingungen
durch das Aufkommen der Informationstechnologie zu radikalen
institutionellen Veränderungen führen werden. Die These dieses Buches
ist, dass die geballte Macht des Nationalstaates dazu bestimmt ist,
privatisiert und kommerzialisiert zu werden. Wie bei allen wirklich
radikalen institutionellen Veränderungen wird die Privatisierung und
Kommerzialisierung der Souveränität eine Revolution des „gesunden
Menschenverstandes`` nach sich ziehen, bezogen darauf, wie die Welt
wahrgenommen wird. Solche Veränderungen geschehen selten auf eine
graduelle, lineare Art und Weise.

Ganz im Gegenteil. Tatsächlich ist dies aus Gründen, die wir in
\emph{The Great Reckoning} untersucht haben, praktisch ausgeschlossen.
Wir rechnen damit, dass das Informationszeitalter zu Verwerfungen führt
- zu deutlichen Brüchen mit den Institutionen und dem Bewusstsein der
Vergangenheit. Folgendes wird man beobachten können, während dieser
Prozess vonstattengeht:

\begin{enumerate}
\def\labelenumi{\arabic{enumi}.}
\tightlist
\item
  Veränderungen in der Wirtschaftsorganisation auf eine Art, wie sie in
  den vorhergehenden Kapiteln aufgrund der Auswirkungen der
  Mikroverarbeitung beschrieben wurden.
\item
  Ein mehr oder weniger rascher Rückgang der Bedeutung aller
  Organisationen, die innerhalb und nicht über geografische Grenzen
  hinweg agieren. Regierungen, Gewerkschaften, lizenzierte Berufe und
  Lobbyisten werden im Informationszeitalter weniger wichtig sein, als
  sie es im Industriezeitalter waren. Da Vetternwirtschaft und
  Handelsbeschränkungen von Regierungen weniger nützlich sein werden,
  werden weniger Ressourcen für Lobbyarbeit verschwendet.\footnote{Gordon
    Tullock, \emph{Rent-Seeking} (Aldershot, England: Edward Elgar,
    1993).}
\item
  Die weitläufige Anerkennung, dass der Nationalstaat veraltet ist,
  führt zu weit verbreiteten Sezessionsbewegungen in vielen Teilen der
  Welt.
\item
  Ein Rückgang des Status und der Macht traditioneller Eliten sowie ein
  Rückgang des Respekts gegenüber den Symbolen und Glaubenssätzen, die
  den Nationalstaat rechtfertigen.
\item
  Eine intensive und sogar gewalttätige nationalistische Reaktion ist
  vor allem unter jenen Personen zu beobachten, die ihren Status, ihr
  Einkommen und ihre Macht verlieren, wenn das, was sie als ihr
  „normales Leben`` betrachten, durch politische Dezentralisierung und
  neue Marktvereinbarungen gestört wird. Zu Merkmalen dieser Reaktion
  gehören:

  \begin{enumerate}
  \def\labelenumii{\alph{enumii})}
  \tightlist
  \item
    Misstrauen gegenüber und Widerstand gegen Globalisierung,
    Freihandel, „ausländisches`` Eigentum und Eindringen in die lokale
    Wirtschaft;
  \item
    Feindseligkeit gegenüber Einwanderung, insbesondere von Gruppen, die
    sich sichtbar von der bisherigen nationalen Gruppe unterscheiden;
  \item
    Weitverbreiteter Hass auf die Informationselite, die Reichen und die
    gut Ausgebildeten sowie Klagen über Kapitalflucht und verschwindende
    Arbeitsplätze;
  \item
    Extreme Maßnahmen von Nationalisten, die die Sezession von
    Individuen und Regionen von geschwächten Nationalstaaten verhindern
    wollen, einschließlich des Rückgriffs auf Kriege und „ethnische
    Säuberungen``, die die nationalistische Identifikation mit dem Staat
    verstärken und die Ansprüche des Staates auf Menschen und ihre
    Ressourcen rationalisieren.
  \end{enumerate}
\item
  Da es offensichtlich sein wird, dass Informationstechnologien es
  souveränen Individuen erleichtern, der Staatsgewalt zu entfliehen,
  wird die Reaktion auf den Zusammenbruch von organisiertem Zwang auch
  einen neo-luddistischen Angriff auf diese neuen Technologien und
  diejenigen, die sie nutzen, beinhalten.
\item
  Die nationalistisch-luddistische Reaktion wird nicht in allen Regionen
  und Bevölkerungsgruppen gleich sein:

  \begin{enumerate}
  \def\labelenumii{\alph{enumii})}
  \tightlist
  \item
    Die Reaktion wird in schnell wachsenden Volkswirtschaften, in denen
    das Pro-Kopf-Einkommen während des Industriezeitalters niedrig war
    und in denen die Vertiefung der Märkte die Einkommen aller
    Qualifikationsgruppen anhebt, weniger stark sein.
  \item
    Die reaktionären Tendenzen werden in den derzeit reichen Ländern am
    stärksten zu spüren sein, insbesondere in Gemeinschaften mit einem
    hohen Anteil an wert- und qualifikationsarmen Menschen, die früher
    ein hohes Einkommen genossen haben.\footnote{Der enge Zusammenhang
      zwischen Fähigkeiten und Werten und damit dem wirtschaftlichen
      Erfolg wird von Lawrence E. Harrison in \emph{Who Prospers? How
      Cultural Values Shape Economic and Political Success} (New York:
      Basic Books, 1992) beschrieben}
  \item
    Vom Unabomber einmal abgesehen, werden die Neo-Ludditen die meisten
    ihrer Anhänger unter den Menschen in den unteren zwei Dritteln der
    Einkommenskapazität in den führenden Nationalstaaten finden.
  \item
    Die nationalistische und luddistische Reaktion wird jedoch nicht
    unter den Ärmsten am stärksten sein, sondern unter den mittelmäßig
    qualifizierten Personen, den Durchschnittlichen mit
    Abschlusszeugnissen, die während des Industriezeitalters erwachsen
    wurden und nun mit einer Abwärtsmobilität konfrontiert sind.
  \end{enumerate}
\item
  Da neue megapolitische Bedingungen zusammen mit neuen, zusätzlichen
  Ideologien und Moralvorstellungen zu einem neuen Bewusstsein für
  Identität führen, werden die alten Bedingungen des Nationalismus ihren
  Reiz verlieren.
\item
  Die nationalistische Reaktion wird in den ersten Jahrzehnten des neuen
  Jahrtausends ihren Höhepunkt erreichen und dann verblassen, wenn sich
  die Effizienz fragmentierter Souveränitäten, als überlegen gegenüber
  der gebündelten Macht des Nationalstaates erweist. Wir vermuten, dass
  das angeborene Mobbing durch Nationalstaaten gegen alternative
  Gerichtsbarkeiten, veranschaulicht durch die russische Invasion in
  Tschetschenien, dazu neigen wird, Nationen und nationalistischen
  Fanatikern die Sympathie der neuen Generationen zu entziehen, die
  unter den megapolitischen Bedingungen des Informationszeitalters zur
  Reife kommen.
\item
  Der Nationalstaat wird letztendlich in einer Finanzkrise
  zusammenbrechen. Systemische Krisen entstehen typischerweise, wenn
  scheiternde Institutionen mit steigenden Ausgaben und fallenden
  Einnahmen konfrontiert sind - eine Situation, die die führenden
  Nationalstaaten wahrscheinlich treffen wird, wenn Rentenleistungen und
  medizinische Ausgaben Anfang des einundzwanzigsten Jahrhunderts stark
  ansteigen. Während wir dies schreiben, sind sowohl das Vereinigte
  Königreich als auch die Vereinigten Staaten mit mehreren Billionen
  durch ungedeckte Pensionsverpflichtungen belastet (auf der
  Pro-Kopf-Ebene sieht es ähnlich aus), die höchstwahrscheinlich keines
  der beiden Länder in den Griff bekommen wird. Andere führende
  Nationalstaaten stehen vor ähnlich ruinösen Belastungen.
\end{enumerate}

\section{PARALLELEN ZUR RENAISSANCE}\label{parallelen-zur-renaissance}

Wir haben zuvor Gründe aufgezeigt, warum wir glauben, dass der
Zusammenbruch des Nanny-Staates Konsequenzen haben wird, die stark jenen
ähneln, die mit dem Zusammenbruch des institutionellen Monopols der
Heiligen Mutterkirche vor fünf Jahrhunderten verbunden waren. Nicht
unähnlich dem Nationalstaat heute, hatte die Kirche damals
jahrhundertelang eine unangefochtene Vormachtstellung. In mancherlei
Hinsicht war die Kirche sogar fester etabliert als es der Staat
fünfhundert Jahre später wurde. Die Kirche behauptete seit langem, als
„die universelle Autorität an der Spitze der christlichen Gesellschaft``
zu handeln.\footnote{John B. Morrall, \emph{Political Thought in
  Medieval Times} (New York: Harper, 1958), S. 48.} Das ist die
Charakterisierung des mittelalterlichen intellektuellen Historikers John
B. Morrall. Doch obwohl vor der technischen Revolution der 1490er Jahre
nur wenige Europäer den Anspruch der Kirche auf Vorherrschaft in
Christentum bestritten hätten, überlebte die Kirche in ihrer
traditionellen Rolle kaum eine weitere Generation.

\subsection{Die Privatisierung des
Gewissens}\label{die-privatisierung-des-gewissens}

Bis Anfang der 1520er Jahre hatten Millionen von guten Europäern die
universelle Autorität der katholischen Kirche abgelehnt, eine Ketzerei,
die nur wenige Jahrzehnte zuvor mit Folter und Tod bestraft worden war.
In der Tat waren viele mittelalterliche europäische Kathedralen und
Kirchen mit belehrenden Schnitzereien von Ketzern geschmückt, denen
Dämonen die Zunge herausgerissen hatten.\footnote{Beispiel: Die Fassade
  der Kathedrale in Angoulême, Frankreich.}

Die Botschaft dieser Folterungen hat sicherlich viele Analphabeten
beeindruckt, die die Opfer nur aufgrund ihrer Strafe als Ketzer erkannt
haben könnten. Die Ikonografie war eindeutig: Ketzer waren jene, deren
Zungen verstümmelt wurden. So hart diese Strafe auch war, sie war nur
das Vorspiel für die ultimative Strafe für Ketzerei: Tod auf dem
Scheiterhaufen. Zum Entsetzen der Kirche war die Lektion jedoch nicht
abschreckend genug. Das Aufkommen der Druckpresse erhöhte die Menge an
ketzerischen Argumenten so dramatisch, dass selbst die Erwartung einer
grausamen Bestrafung potenzielle Ketzer nicht mehr abschreckte.
Tatsächlich zahlten nicht wenige unglückliche Pioniere der
Religionsfreiheit in der frühen Neuzeit Europas für ihre Behauptungen
der spirituellen Unabhängigkeit, indem ihre Zungen herausgeschnitten
wurden. Andere wurden auf dem Scheiterhaufen verbrannt. Die reaktionären
Agenten der Inquisition verbrannten buchstäblich Menschen für
Äußerungen, die wir als gewöhnliche Gewissensbekundungen ansehen würden.

Alles in allem kostete die Reformation und die Reaktion, die sie
auslöste, Millionen Menschen das Leben. Allein die Toten der
Schlachtfelder in der zweiten Hälfte des Dreißigjährigen Krieges
beliefen sich auf 1.151.000.\footnote{Karen A. Rasler und William R.
  Thompson, \emph{War and State Making: The Shaping of the Global
  Powers. Studies in International Conflict}, Ausgabe 2 (Boston: Unwin
  Hyman, 1989), S. 13.} Viele weitere starben an Hunger, Krankheiten,
oder durch die Hand der Inquisition und anderer Behörden. Bei weitem
nicht alle Gewalttaten wurden von katholischen Behörden verübt. Im Tower
von London wurden die Knochen von mehr als tausend führenden englischen
Katholiken entdeckt, die von König Heinrich VIII. brutal ermordet worden
sein sollen. Einige, einschließlich Sir Thomas More und Bischof St.~John
Fisher, wurden öffentlich hingerichtet, weil sie sich weigerten, den
alten Glauben aufzugeben.\footnote{Julian Large, \emph{Bishop Died for
  Standing Firm Against Henry VIII}, \emph{Daily Telegraph}, 16. Juni
  1996, S. 2.} Auf der anderen Seiten verbrannte Queen Mary, die
katholische Tochter von König Heinrich VIII., die von einer durch ihren
Vater geerbten Syphilis geplagt war, in den letzten beiden Jahren ihrer
Regentschaft dreihundert protestantische Ketzer auf dem Scheiterhaufen.

Das war der Preis, den Personen unterschiedlicher Überzeugungen zahlten,
als sie ihre religiösen Überzeugungen behaupteten und das lange
verweigerte Recht einforderten, die Kirche ihrer Wahl zu unterstützen.
Aus unserer Sicht am Ende des zwanzigsten Jahrhunderts lagen diese
Ausdrücke des persönlichen Glaubens wohl innerhalb des Bereichs, der
durch die Religionsfreiheit und die Meinungsfreiheit geschützt werden
sollte. Aber im frühen 16. Jahrhundert gab es weder Religionsfreiheit
noch Meinungsfreiheit. Die Behörden der damaligen Zeit leiteten ihre
Orientierung noch immer aus der abnehmenden mittelalterlichen
Weltanschauung ab. In ihren Augen standen Ausdrücke individueller
Autonomie in Opposition zur Autorität, insbesondere der plentitudo
potestatis (Fülle der Gewalt) des Papstes und waren schlicht empörend
und eindeutig subversiv. Wie der theologische Historiker Euan Cameron
sagte, haben religiöse Reformer wie Martin Luther Ansichten vertreten,
die „eine bewusste und entscheidende Trennung von der institutionellen
und spirituellen Kontinuität der alten Kirche`` bedeuteten.\footnote{Cameron,
  ebenda, S.97.}

\subsection{Ketzerei und Hochverrat}\label{ketzerei-und-hochverrat}

In diesem Sinne erwarten wir einen „vorsätzlichen und entscheidenden
Bruch`` mit der institutionellen und ideologischen Kontinuität des
Nationalstaats. Bis zum Ende des ersten Viertels des nächsten
Jahrhunderts werden Millionen von rechtschaffenen Individuen das
säkulare Äquivalent der Ketzerei des sechzehnten Jahrhunderts begangen
haben - eine schwache Form von Verrat. Sie werden ihre Treue zum
schwächelnden Nationalstaat widerrufen, um ihre eigene Souveränität zu
behaupten, ihr Recht, nicht ihre Bischöfe oder ihre Gotteshäuser,
sondern ihre Form der Regierung als Kunden zu wählen. Die Privatisierung
von Souveränität wird der Privatisierung des Gewissens vor fünf
Jahrhunderten ähneln. Beides bezeichnet den massenhaften Treuebruch von
ehemaligen Unterstützern dominanter Institutionen. Wie Albert O.
Hirschman, ein Experte für „Responses to Decline in Firms, Organizations
and States``, geschrieben hat, ist diese Art des Ausstiegs schwierig,
weil „Ausstieg oft als kriminell gebrandmarkt wurde, weil er als
Desertieren, Überlaufen und Verrat gebrandmarkt wurde.`` \footnote{Hirschman,
  ebenda, S.17.}

Souveräne Individuen werden nicht länger nur das hinnehmen, was der
Staat ihnen als Humankapital zuweist. Millionen werden die
Verpflichtungen der Staatsbürgerschaft ablegen, um Kunden für die
nützlichen Dienstleistungen zu werden, die Regierungen anbieten.
Tatsächlich werden sie parallele Institutionen schaffen und
unterstützen, die die meisten mit der Staatsbürgerschaft verbundenen
Dienstleistungen auf rein kommerzieller Basis stellen. Während des
größten Teils des zwanzigsten Jahrhunderts wurden produktive Menschen
vom Staat wie Vermögenswerte behandelt, ähnlich wie der Milchbauer seine
Kühe behandelt. Sie wurden immer stärker ausgepresst. Jetzt werden den
Kühen Flügel wachsen.

\subsection{Abkehr von der
Staatsbürgerschaft}\label{abkehr-von-der-staatsbuxfcrgerschaft}

Genauso, wie neue megapolitische Bedingungen im sechzehnten Jahrhundert
das Monopol der Kirche untergraben haben, erwarten wir, dass die
Megapolitik des Informationszeitalters letztendlich die Bedingungen der
Regierungsführung im einundzwanzigsten Jahrhundert diktieren wird, so
ungeheuerlich ihre neuen Bedingungen auch denen erscheinen mögen, die
die Werte der modernen Politik als ihre eigenen betrachten. Die
Entwicklung vom Status eines „Bürgers``, zu dem eines „Kunden``
beinhaltet einen Verrat an der Vergangenheit, der so scharf ist, wie der
Übergang von der Ritterlichkeit zur Staatsbürgerschaft in der frühen
Neuzeit. Der Abfall der Informationselite von der Staatsbürgerschaft
wird einen ähnlichen Anreiz haben wie der, der vor fünfhundert Jahren
Millionen von Europäern dazu veranlasste, die Unfehlbarkeit des Papstes
abzulehnen.

Falls der Vergleich mit der Reformation nicht überzeugend ist, könnte
dies teilweise darauf zurückzuführen sein, dass es heute nicht sofort
ersichtlich ist, dass die Abwendung von religiösen Institutionen jemals
die gleiche Bedeutung hatte, die Hochverrat im zwanzigsten Jahrhundert
erlangte. Außer in einigen islamischen Ländern, ist Ketzerei am Ende des
zwanzigsten Jahrhunderts ein geistliches Vergehen, das nicht
verheerender für den Ruf eines Individuums ist, als ein Verkehrsticket
für zu schnelles Fahren in einer Tempo-30-Zone.\footnote{Für
  zeitgenössische Belege hierfür siehe Bruce Bawer, \emph{Who's on
  Trial, the Heretic or the Church?} New York Times Magazine, 7. April
  1996, S.36f.} Tatsächlich ist es in Europa und Nordamerika nicht
ungewöhnlich, Geistliche und sogar Bischöfe zu finden, die nicht an Gott
glauben oder entscheidende Glaubensgrundsätze, die sie vertreten,
leugnen. Heute müsste Ketzerei nahezu offensichtliche Teufelsanbetung
sein, um aufzufallen. In den meisten westlichen Ländern sind religiöse
Lehren so schlecht gestaltet und schlampig gehalten, dass nur wenige
Menschen überhaupt in der Lage sind, die theologischen Punkte
identifizieren zu können, die in der Vergangenheit im Mittelpunkt der
Kontroversen um die Ketzerei standen.\footnote{Malcolm Lambert,
  \emph{Medieval Heresy} 2. Ausgabe (Oxford: Blackwell, 1992).} Dies
spiegelt die allgemeine Abkehr von der Religion wider.

In gewissem Maße haben religiöse Führer tatsächlich dazu beigetragen,
die Ernsthaftigkeit gegenüber spirituellen Themen im späten zwanzigsten
Jahrhundert zu untergraben, indem sie ihre Energien von spirituellen
Anliegen ablenkten, um Lobbyisten und soziale Agitatoren zu werden. Von
der Macht angezogen, wie lose Metallspäne von einem Magneten, verwenden
sie einen Großteil ihrer Aktivitäten darauf, politische Führer unter
Druck zu setzen, um eine Politik der Umverteilung zu übernehmen, die für
die nationalistische Vereinbarung entscheidend sind. Man denke an die
lautstarken Bemühungen der katholischen Kirche in Argentinien, die
Regierung von Präsident Carlos Menem unter Druck zu setzen, um
Wirtschaftsreformen auszusetzen, zugunsten herkömmlicher Geldinflation
und keynesianischer Fiskalpolitik. Ähnliche Beschwerden wurden von
religiösen Führern gegen Bemühungen, aufgeblähte Haushalte in Neuseeland
und vielen anderen Ländern umzustrukturieren, vorgebracht. Katholische
Bischöfe haben vehement gegen die Reform des Sozialsystems in den
Vereinigten Staaten lobbyiert.

\subsection{Eine fiskalische
Inquisition?}\label{eine-fiskalische-inquisition}

Einfach ausgedrückt, konzentrieren sich die religiösen Führer von heute
mit ihrer schwindenden moralischen Autorität mehr auf die weltliche
Erlösung und die Beeinflussung des Staates als auf das Seelenheil.
Angesichts dieser Tatsache kann erwartet werden, dass sie als Komplizen
an der Reaktion gegen die kommende säkulare Reformation teilnehmen
werden. Da der Nationalstaat in Frage gestellt wird und zu wanken
beginnt, wird er nicht mehr in der Lage sein, die Versprechen
materieller Vorteile zu erfüllen, die für die breite Unterstützung
entscheidend sind. Der de facto Vertrag, der zur Zeit der französischen
Revolution eingegangen wurde, wird keinen Bestand mehr haben. Der Staat
wird nicht mehr in der Lage sein, seinen Bürgern kostengünstige oder
kostenlose Schulbildung, geschweige denn medizinische Versorgung,
Arbeitslosenversicherung und Renten im Austausch für ansonsten schlecht
bezahlten Militärdienst zu garantieren. Obwohl die sich ändernden
Anforderungen der Kriegsführung den Regierungen ermöglichen werden, sich
selbst und die unter ihrer Herrschaft stehenden Territorien ohne
gigantische Armeen zu verteidigen, wird dies die Regierungen kaum vor
der Kritik für die Auflösung dessen bewahren, was inzwischen ein
anachronistischer Vertrag geworden ist.

Tatsächlich werden, sobald die neue megapolitische Logik greift, die
Auswirkungen bei den Verlierern in der neuen Informationswirtschaft auf
enormen Unmut stoßen. Es ist daher so gut wie sicher, dass viele
religiöse Führer, zusammen mit den Hauptnutznießern der Staatsausgaben,
an der Spitze einer nostalgischen Reaktion stehen werden, die die
Ansprüche des Nationalismus wieder geltend machen wollen. Sie werden
behaupten, dass kein Amerikaner, Franzose, Kanadier oder Angehöriger
einer anderen Nationalität - hier Lücke füllen - hungrig ins Bett gehen
sollte. Selbst Länder wie Neuseeland, die an der Spitze der Reform
stehen und unverhältnismäßig von einem „marktfreundlichen Globalismus``
profitieren könnten, werden von reaktionären Verlierern geplagt werden.
Sie werden versuchen, den Kapital- und Personenverkehr über Grenzen
hinweg zu behindern. Und sie werden dort nicht haltmachen. Demagogen wie
Winston Peters, der Anführer der New Zealand First Party, sind zu faul,
um auf originelle Weise darüber nachzudenken, wie die neue Welt
funktionieren wird. Aber zu gegebener Zeit werden Winston und seine
Leute auf die Logik der Informationswirtschaft aufmerksam gemacht. Sie
werden versuchen, die Verbreitung von Computern, Robotik,
Telekommunikation, Verschlüsselung und anderen Informationstechnologien,
die die Verdrängung von Arbeitskräften in fast jedem Sektor der
Weltwirtschaft bewirken, zu stoppen. Wo auch immer Sie hinschauen gibt
es Politiker, die liebend gern die Aussichten auf langfristigen
Wohlstand zunichtemachen würden, nur um zu verhindern, dass Individuen
ihre Unabhängigkeit von der Politik erklären.

\subsection{20/20 Vision}\label{vision}

Bis 2020, beziehungsweise grob fünf Jahrhunderte nachdem Martin Luther
seine 95 subversiven Thesen an die Kirchentür in Wittenberg nagelte,
wird die Wahrnehmung der Kosten-Nutzen-Relationen der Staatsbürgerschaft
eine ähnlich subversive Klärung durchlaufen haben. Die Vorstellung des
Nationalstaates wird unter Personen mit Fähigkeiten und Reichtum, den
souveränen Individuen der Zukunft, das politische Äquivalent einer
Laser-Operation durchlaufen. Sie werden die perfekte Wahrnehmung haben.
Im zwanzigsten Jahrhundert, wie auch während des gesamten modernen
Zeitalters, machten anhaltend hohe Erträge aus Gewalt große Regierungen
zu einer lohnenden Angelegenheit. Die Entschlossenheit einer geballten
Macht mobilisierte die Loyalität der Reichen und Ehrgeizigen zu den
OECD-Nationalstaaten, trotz der räuberischen Steuern, die auf Einkommen
und Kapital erhoben wurden. Politiker konnten Grenzsteuersätze von
nahezu oder sogar mehr als 90 Prozent in jedem OECD-Land in dem
Jahrzehnt unmittelbar nach dem Zweiten Weltkrieg durchsetzen.

Wie wir untersucht haben, hatten die Reichen kaum eine andere Wahl, als
solchen Zumutungen zuzustimmen. Gegebenheiten zwangen sie dazu, sich zum
Schutz auf Regierungen zu verlassen, die Gewalt im großen Ausmaß
beherrschen konnten. Es spielte kaum eine Rolle, abgesehen vielleicht
von britischen Polizeibeamten, die die Möglichkeit einer Versetzung nach
Hongkong hatten, dass die OECD-Regierungen Monopolsteuern erhoben.
Jeder, der ein hohes Einkommenspotenzial hatte und während des
Industriezeitalters die Möglichkeit genießen wollte, an wirtschaftlichen
Innovationen teilzuhaben, hatte normalerweise kaum andere Möglichkeiten,
als in einer Hochsteuerwirtschaft zu leben. Dies bedeutete, eine
Steuerlast zu schultern, die in keinem Verhältnis zu den erhaltenen
Dienstleistungen stand.

\subsection{Die Mathematik der
Politik}\label{die-mathematik-der-politik}

Der amerikanische Vizepräsident des 19. Jahrhunderts, John C. Calhoun,
skizzierte geschickt die Mathematik der modernen Politik. Calhouns
Formel teilt die gesamte Bevölkerung des Nationalstaates in zwei Klassen
auf: Steuerzahler, die mehr zu den Kosten staatlicher Dienstleistungen
beitragen, als sie verbrauchen, und Steuerkonsumenten, die vom Staat
Leistungen erhalten, die ihren Beitrag zu den Kosten übersteigen. Mit
einigen auffälligen Ausnahmen waren die meisten Unternehmer der OECD zu
einem überwältigenden Ausmaß Nettosteuerzahler, als das zwanzigste
Jahrhundert zu Ende ging. Zum Beispiel trugen 1996 die obersten 1
Prozent der britischen Steuerzahler 17 Prozent der gesamten
Einkommensteuerlast. Sie zahlten 30 Prozent mehr als die unteren 50
Prozent der Verdiener, die gerade einmal 13 Prozent zu den
Einkommensteuerzahlungen beitrugen. In den USA war die Last für die
Reichen noch größer, mit den oberen 1 Prozent, die 1994 28 Prozent der
gesamten Einkommensteuereinnahmen zahlten.\footnote{David Smith,
  \emph{What Clarke Could Learn from Reagan}, \emph{The Sunday Times}
  (London), 16. Juni 1996, S. 6.} Nicht nur mussten die Reichen für
Dienstleistungen zahlen, die, wie uns Frederic C. Lane in Erinnerung
ruft, „von schlechter Qualität und unverschämt überteuert`` waren. Ihre
Zahlungen waren oft nicht proportional zu überhaupt irgendeinem
Dienst.\footnote{Lane, \emph{Economic Consequences of Organized
  Violence}, S. 404.} Die Leistungen, für die die Top-Steuerzahler
zahlten, gingen oft komplett an andere. In den meisten Fällen waren die
Reichen froh, die Regierungsdienstleistungen zu unterkonsumieren, die
normalerweise von geringer Qualität waren. Regierungsbüros in fast jedem
Land waren berüchtigt ineffizient, hauptsächlich weil sie dazu neigten,
von Mitarbeitern gesteuert zu werden, die keinen Anreiz zur
Produktivitätssteigerung hatten. Nach praktisch jedem Maßstab zahlten
die größten Steuerzahler während der industriellen Ära viele Male mehr
für Regierungsdienstleistungen, als sie auf einem
wettbewerbsorientierten Markt wert gewesen wären.

Dies blieb kaum unbeachtet. Unglücklicherweise war jedoch die
Erkenntnis, dass Zahlungen an die Regierung für Schutz nach Lanes Worten
„nach idealen Standards verschwenderisch`` waren, in der Mitte des
zwanzigsten Jahrhunderts selten eine einklagbare Erkenntnis. Es war eher
einfach ein zu akzeptierender Mangel, „eine der verschiedenen Arten von
Verschwendung, die in die soziale Organisation eingebaut sind.``
\footnote{Ebenda}

Die Alternative für die Unzufriedenen bestand nicht darin, zum Beispiel
von Großbritannien nach Frankreich oder von den Vereinigten Staaten nach
Kanada zu ziehen. Außer in seltenen Fällen hätte das wenig genützt. Alle
führenden Nationalstaaten litten unter dem gleichen Nachteil. Sie alle
übernahmen mehr oder weniger beschlagnahmende Steuerregime. Um einen
wesentlichen Anstieg der Autonomie zu erreichen, musste man die
Kernländer Europas und Nordamerikas vollständig verlassen und sich in
Richtung Peripherie bewegen. Die Steuerlast war in Teilen Asiens,
Südamerikas und auf verschiedenen abgelegenen Inseln bedeutend
niedriger. Aber für das Entkommen vor räuberischer Besteuerung musste
man in der Regel einen Preis bezahlen, einen Verlust an wirtschaftlichen
Möglichkeiten und oft einen Rückgang des Lebensstandards. Wie wir
erforscht haben, waren in den Bedingungen des Industriezeitalters die
wirtschaftlichen Möglichkeiten begrenzt und die Lebensstandards waren
unterdurchschnittlich in den meisten Gebietskörperschaften außerhalb der
Kernindustriestaaten, die sich an beschlagnahmender Besteuerung
beteiligten.

Betrachten Sie die kommunistischen Systeme als Paradigma. Zusammen mit
vielen Regimen der Dritten Welt haben sie in der Regel keine hohen
Einkommensteuern - oder überhaupt keine - erhoben.\footnote{Kuba führte
  erst 1996 als Notstandsgesetz eine Einkommenssteuer ein, um auf die
  wirtschaftliche Depression zu reagieren, die nach dem Ende der durch
  den Zusammenbruch des Kommunismus in Europa bedingten Subventionen
  einsetzte.} Nichtsdestotrotz suchten während der drei Viertel des
Jahrhunderts, in dem die Sowjetunion existierte, nur wenige, wenn
überhaupt, Unternehmer dort Steuerzuflucht. Obwohl die sowjetischen
Einkommensteuersätze nicht hoch waren, boten sie keinen Vorteil, weil
die Sowjets darauf bestanden, Eigentumsrechte nicht anzuerkennen. Dies
stellte eine noch größere Belastung dar als die Besteuerung. Die
kommunistischen Systeme machten es nahezu unmöglich, ein Unternehmen zu
organisieren und ernsthaft Geld zu verdienen. Im Grunde genommen
konfiszierte der kommunistische Staat das Einkommen vor der Steuer.

Darüber hinaus wäre jeder, der aus irgendeinem exzentrischen Grund
bereits über ein sicheres Einkommen verfügte und sich entschieden hätte,
in Moskau oder Havanna zu leben, kaum in der Lage gewesen, mit Geld
einen angemessenen Lebensstandard zu kaufen. Abgesehen von Zugang zu
guten Zigarren, Kaviar, exzellenten Orchestern und dem Ballett bot das
Leben in den ehemaligen kommunistischen Systemen nur wenige
Konsumvergnügen. Die meisten der wenigen guten Dinge im Leben waren
nicht erhältlich oder wurden streng nach politischem Einfluss
rationiert, anstatt offen ausgetauscht zu werden. Unter dem Risiko, das
Stereotyp von Kritikern des postmodernen Lebens zu bestätigen, die „die
Bedeutung des Konsums in der postmodernen Erfahrung`` betonen, hat der
weltweit steigende Standard von Gütern und Dienstleistungen seit dem
Fall des Kommunismus sicherlich den Wettbewerb zwischen den
Gerichtsbarkeiten belebt und so dazu beigetragen, die Bindungen an
Nation und Ort zu schwächen.\footnote{Ni. Featherstone, \emph{Consumer
  Culture and Postmodernism} (London: Sage, 1991) und J. F. Sherry,
  \emph{Postmodern Alternative: The Interpretative Turn in Consumer
  Research}, in T. Robertson und H. Kassarjian, eds., \emph{Handbook of
  Consumer Research} (Englewood Cliffs, N.J.: Prentice-Hall, 1991),
  diskutiert in Billig, ebenda.}

Unter dem alten Regime waren die Verbraucherauswahlmöglichkeiten so
begrenzt, dass sogar Castro selbst es schwer gefunden hätte, eine
Packung ordentliche Zahnseide zu bekommen, falls er Cohiba-Fragmente von
seinen Zähnen entfernen wollte. Bis vor kurzem konnten selbst die
Reichen in vielen Teilen der Welt nicht den Lebensstandard genießen, der
unter der Mittelschicht in Westeuropa oder Nordamerika gängig war.
Angesichts dieser traurigen Situation fühlten sich die meisten Personen
mit herausragendem Talent dazu veranlasst, den nationalistischen Handel
während des Industriezeitalters zu akzeptieren. Sie blieben darauf
angewiesen und zahlten horrende Steuern für den zweifelhaften Schutz,
den der jeweilige Nationalstaat anbot, der auf dem Territorium, auf dem
sie geboren wurden, Gewalt monopolisierte.

\begin{quote}
„Das Paradies ist verriegelt und der Cherub hinter uns; wir müssen die
Reise um die Welt machen und sehen, ob es vielleicht von hinten irgendwo
wieder offen ist.\emph{``} - Heinrich von Kleist
\end{quote}

Der Fall des Kommunismus beseitigte einen „Eisernen Vorhang``, der das
Reisen beeinträchtigt und die Globalisierung des Handels effektiv
blockiert hatte, wodurch die Welt künstlich „groß`` gehalten wurde. Das
Flugzeug, in Kombination mit den Informationstechnologien, die den
Kommunismus untergruben, erhöhte den Wettbewerb um hochwertige
Reisedollars. Die Parade von Bankern, die ein- und ausgingen, auch in
den entlegensten Provinzen, war ein enormer Anreiz für den Wohn- und
Gastronomiestandard weltweit. Damit beziehen wir uns nicht auf die
Verbreitung von McDonald\textquotesingle s Hamburgern und Kentucky Fried
Chicken Filialen, sogar in ehemals so abschreckenden Orten wie Moskau
und Bukarest. Weniger beachtet, aber wichtiger, war die Ausbreitung von
führenden Hotelketten und hochwertigen Restaurants, die Grand Cru
anstelle von Wodka und Cola servieren. Dank dieser Transformation kann
jetzt jeder, der es sich leisten kann, fast überall auf dem Planeten
einen hohen materiellen Lebensstandard genießen. Es gibt kaum noch ein
Land, in dem es nicht ein erstklassiges Hotel und mindestens ein
Restaurant gibt, das einen Michelin-Inspektor interessieren würde.

Wie Hirschman vor einem Vierteljahrhundert vorausahnte, hat der
technologische Fortschritt die Anziehungskraft des Ausstiegs als Lösung
für unzufriedenstellende Dienstleistungen und Preisgestaltungen
erheblich erhöht. Er schrieb: „Die Loyalität zu seinem Heimatland ist
etwas, auf das wir verzichten könnten.\ldots{} Erst wenn sich die Länder
aufgrund der Fortschritte in der Kommunikation und der allgemeinen
Modernisierung einander angleichen, besteht die Gefahr der verfrühten
und übermäßigen Abwanderung, wofür der ‚Brain Drain' ein aktuelles
Beispiel ist.`` \footnote{Hirschman, ebenda, S. 81.} Beachten Sie, wie
wir in Kapitel 8 darauf hingewiesen haben, dass Hirschmans Standard der
„verfrühten und übermäßigen Abwanderung`` aus der Perspektive des
verlassenen Nationalstaates gesehen wird, nicht aus der Perspektive des
Individuums, das ein besseres Leben sucht.

Nichtsdestotrotz bleibt seine Schlussfolgerung, dass Gemeinsamkeiten
zwischen den Ländern die Attraktivität der Abwanderung und des Ausstiegs
erhöhen, unanfechtbar. Die Tatsache, dass es nun einfacher ist, überall
gut zu leben, macht das Leben dort attraktiv, wo die Kosten am
geringsten sind. Noch wichtiger als die Möglichkeit, fast überall gut zu
leben, ist jedoch die Tatsache, dass man nun überall ein hohes Einkommen
erzielen kann. Es ist nicht mehr notwendig, in einem Gebiet mit hohen
Kosten zu wohnen, um genügend Vermögen anzusammeln, um, wie Lord Keynes
empfahl, „weise, angenehm und gut`` zu leben. Aus Gründen, die wir
bereits erörtert haben, verändert die Mikrotechnologie die grundlegende
megapolitische Basis, auf der der Nationalstaat ruht. Im
Informationszeitalter wird eine neue Cyberwirtschaft entstehen, die über
die Kapazität jeder Regierung hinausgeht. Zum ersten Mal wird die
Technologie es dem Einzelnen ermöglichen, Vermögen in einem Bereich
anzuhäufen, der sich nicht so leicht den Anforderungen des
systematischen Zwanges beugt.

Die neue Gesellschaft und damit auch die neue Kultur wird einerseits
durch das definiert, was Maschinen besser können als Menschen, durch
Automatisierung, die eine steigende Anzahl von niedrigqualifizierten
Aufgaben obsolet macht. Andererseits wird sie durch die Macht bestimmt,
die die Informationstechnologie Menschen gibt, die tatsächlich das
Talent haben, sie zu nutzen. Eine solche Gesellschaft wird größere
Spannungen zwischen einer kleinen Klasse, die man als
Informationsaristokratie bezeichnen könnte, und einer wachsenden
Unterschicht haben, die man als Informationsarme bezeichnen könnte. Ein
Unterschied zwischen ihnen wird sein, dass die Informationsarmen
entweder geographisch gebunden sind oder wenig Nutzen aus einem Umzug
ziehen. Die Informationsaristokratie hingegen, wie wir anderswo
diskutieren, wird extrem mobil sein, da sie in der Lage sein wird, an
jedem beliebigen Ort, der für sie attraktiv ist, Geld zu verdienen,
genau wie es beliebte Schriftsteller schon immer konnten. Robert Louis
Stevenson konnte vor hundert Jahren auf einer Insel im Pazifik seinen
Lebensunterhalt verdienen; jetzt können alle Mitglieder der
Informationsaristokratie dasselbe tun.

\subsection{Marktwettbewerb zwischen
Rechtssystemen}\label{marktwettbewerb-zwischen-rechtssystemen}

Da die Informationstechnologie die Tyrannei des Ortes überwindet, wird
sie automatisch Rechtssysteme überall einem de facto globalen Wettbewerb
hinsichtlich Qualität und Preis aussetzen. Mit anderen Worten,
Regierungen, die lokale territoriale Monopole ausüben, wie die meisten
anderen Einheiten, werden endlich echtem Markt-Wettbewerb unterzogen,
basierend darauf, wie gut sie ihre Kunden bedienen. Dies wird bald
unumgänglich deutlich machen, dass die alte Logik, die in der
industriellen Ära Hochkostenregime begünstigte, sich umgekehrt hat.
Führende Nationalstaaten mit ihren räuberischen, umverteilenden
Steuerregimen und strengen Vorschriften werden nicht mehr bevorzugte
Jurisdiktionen sein. Nüchtern betrachtet bieten sie schlechten Schutz
und geminderte wirtschaftliche Möglichkeiten zu Monopolpreisen. In den
kommenden Jahren könnten sie sozial unfreundlicher und gewalttätiger
sein als Regionen in Asien und Lateinamerika, wo die Einkommen
traditionell ungleicher sind. Die führenden Sozialstaaten werden ihre
talentiertesten Bürger durch Abwanderung verlieren.

\subsection{Die bevorstehende „extranationale``
Ära}\label{die-bevorstehende-extranationale-uxe4ra}

Mit der Entwicklung des Zeitalters des „souveränen Individuums`` werden
viele der fähigsten Menschen aufhören, sich als Teil einer Nation zu
sehen, als „Briten``, „Amerikaner`` oder „Kanadier``. Ein neues
„transnationales`` oder „extranationales`` Verständnis der Welt und eine
neue Art, den eigenen Platz darin zu identifizieren, wird im neuen
Jahrtausend entdeckt. Wie bereits angedeutet, war ein früher Beleg für
diese neue Denkweise die Feststellung, dass fast die Hälfte der
jugendlichen MTV-Zuschauer damit rechnet, das Land ihrer Geburt zu
verlassen, um das Leben zu erreichen, das sie sich für sich wünschen.
Diese neue Gleichung der Identität wird, anders als die Nationalität,
nicht ein Produkt der systematischen Zwangsläufigkeit sein, die im
zwanzigsten Jahrhundert die Nationalstaaten und das Staatensystem
universell gemacht hat.

Die bloße Tatsache, dass globale Entwicklungen allgemein als
„international`` bezeichnet werden, zeigt, wie tief das nationalistische
Paradigma in unsere Auffassung der Welt eingedrungen ist. Nach zwei
Jahrhunderten Indoktrination von Mysterien der „internationalen
Beziehungen`` und des „internationalen Rechts`` ist es leicht zu
übersehen, dass „Internationalität`` kein langjähriges westliches
Konzept ist. Tatsächlich wurde das Wort \emph{international} von Jeremy
Bentham im Jahr 1789 erfunden. Er verwendete es zum ersten Mal in seinem
Buch \emph{Eine Einführung in die Prinzipien der Moral und
Gesetzgebung}. Bentham schrieb: „Das Wort international, das muss man
zugeben, ist ein neues Wort, obwohl man hofft, dass es ausreichend
analog und verständlich ist.`` Das Wort setzte sich durch, aber nicht
bloß in dem engen Sinn, den Bentham beabsichtigte. „International``
wurde zu einem schwammigen Synonym für alles, was auf dem Globus
geschieht.

Das Internationale Zeitalter begann im Jahr 1789, demselben Jahr wie die
Französische Revolution. Es dauerte zwei Jahrhunderte, bis 1989, als der
Aufstand gegen den Kommunismus in Europa begann. Wir glauben, dass diese
zweite Revolution das Ende des Internationalen Zeitalters markierte, und
nicht nur, weil die diskreditierte kommunistische Hymne „Die
Internationale`` war. Die Befehlswirtschaft mit staatlichem Eigentum war
der kühnste Ausdruck des Nationalstaats. Die enge Beziehung zwischen
Staatsmacht und Nationalismus spiegelte sich in der Sprache wider. Das
aggressivste Verb des modernen Zeitalters war „verstaatlichen``, was
bedeutet, unter staatliches Eigentum und Kontrolle zu bringen. Es war
ein Wort, das in den meisten Teilen der Welt während des Internationalen
Zeitalters leicht von Demagogen in den Mund genommen wurde. Jetzt ist es
Teil des Vokabulars der Vergangenheit. Die Verstaatlichung ist
anachronistisch geworden, gerade weil die Staatsmacht anachronistisch
geworden ist.

Im Zwielicht der modernen Ära wurde die konzentrierte Macht des Staates
durch das Zusammenspiel von technologischer Innovation und Marktkräften
untergraben. Nun steht die nächste Etappe im Triumph des Marktes kurz
bevor. Es werden nicht nur einzelne Nationalstaaten beginnen, sich
aufzulösen, sondern unserer Ansicht nach ist sogar der Club der
Nationalstaaten, die Vereinten Nationen, zum Bankrott verurteilt. Wir
wären nicht überrascht, wenn die UNO schon bald nach der
Jahrtausendwende aufgelöst werden würde.

Wenn „international`` eine Aktie wäre, wäre jetzt die Zeit zu verkaufen.
Dieses Konzept wird im neuen Jahrtausend wahrscheinlich durch neue
Formen ersetzt oder zumindest auf seinen ursprünglichen Sinn
eingegrenzt, und das aus gutem Grund. Die gesamte Welt wird nicht länger
von einem System interagierender souveräner Nationen dominiert.
Beziehungen werden die neuen „extranationalen`` Formen annehmen, die
durch die wachsende Bedeutung von Mikrojurisdiktionen und souveränen
Individuen vorgegeben werden. Ein Konflikt zwischen einer Enklave an der
Küste von Labrador und einem souveränen Individuum wird nicht mehr als
„internationaler`` Konflikt bezeichnet werden. Er wird extranational
sein.

In dem neuen, kommenden Zeitalter werden Gemeinschaften und Loyalitäten
nicht mehr territorial begrenzt sein. Die Identifikation wird eher auf
echte Gemeinsamkeiten, gemeinsame Überzeugungen, gemeinsame Interessen
und gemeinsame Gene ausgerichtet sein, als auf die
Scheinverwandtschaften, die bei den Nationalisten so im Vordergrund
stehen. Der Schutz wird auf neue Weisen organisiert werden, die nicht
mit einem Sextanten, einer Lotstange oder anderen frühmodernen
Vermessungsinstrumenten erfasst werden können, die territoriale Grenzen
markieren.

\section{ERFUNDENE GEMEINSCHAFTEN UND
TRADITIONEN}\label{erfundene-gemeinschaften-und-traditionen}

Die Vorstellung, dass Menschen sich naturgemäß in einer „erfundenen``
Gemeinschaft, einer Nation, zusammenschließen müssen, wird im nächsten
Jahrhundert von der kosmopolitischen Elite wahrscheinlich als
exzentrisch und unvernünftig angesehen werden, so wie es durch den
größten Teil der menschlichen Existenz der Fall gewesen wäre. Für den
Nationalstaat gibt es, wie der Soziologe Anthony Giddens schrieb,
„keinen Präzedenzfall in der Geschichte.`` \footnote{Anthony Giddens,
  \emph{Social Theory and Modern Sociology} (Cambridge: Polity Press,
  1987), S.166, zitiert in Billig, ebenda.} Michael Billig, eine
Autorität auf dem Gebiet des Nationalismus, verstärkte diesen Punkt:

\begin{quote}
Zu früheren Zeiten hielten die Menschen die Vorstellungen von Sprache
und Dialekt, geschweige denn von Territorium und Souveränität, die heute
so alltäglich sind und uns so materiell real erscheinen, nicht fest.
Diese Vorstellungen sind so tief in unserem gegenwärtigen gesunden
Menschenverstand verankert, dass es leicht ist zu vergessen, dass sie
erfundene Dauerhaftigkeiten sind. Die mittelalterlichen Schuhmacher in
den Werkstätten von Montaillou oder San Mateo könnten uns jetzt, mit dem
Abstand von 700 Jahren, als engstirnige, abergläubische Figuren
erscheinen. Aber sie hätten unsere Vorstellungen über Sprache und Nation
als seltsam mystisch empfunden; sie wären verwirrt, warum diese Mystik
eine Frage von Leben und Tod sein soll.\footnote{Billig, ebenda, S. 36.}
\end{quote}

Wir vermuten, dass denkende Menschen in der extranationalen Zukunft
gleichermaßen verwirrt sein werden. Wie Benedict Anderson es
formulierte, sind Nationen „ausgedachte Gemeinschaften``.\footnote{Benedict
  Anderson, \emph{Imagined Communities} (London: Verso, 1983), zitiert
  von Billig, ebenda, S. 10.} Dies bedeutet jedoch nicht, dass das, was
ausgedacht wird, unbedingt trivial ist. Wie Dr.~Johnson beobachtete,
würde ein Mann ohne die Macht der Vorstellungskraft genauso gerne „mit
einem Zimmermädchen wie mit einer Herzogin schlafen``. Dennoch mag es
für diejenigen, die im zwanzigsten Jahrhundert aufgewachsen sind, so
scheinen, als ob „Nationen`` eine so unvermeidliche Organisationseinheit
sind, dass es schwierig zu begreifen ist, dass sie „ausgedacht`` und
nicht natürlich sind. Um zu verstehen, wie unterschiedlich die Zukunft
von der Welt, die wir kennen sein kann, ist es notwendig zu sehen, wie
der Nationalismus dem „gesunden Menschenverstand`` des
Industriezeitalters aufgezwungen wurde.

Es ist leicht zu übersehen, in welchem Maße die „nationale
Gemeinschaft`` durch eine kontinuierliche Investition von
Vorstellungskraft geformt wird. Es gibt keine objektiven Kriterien, um
präzise zu definieren, welche Gruppe eine „Nation`` sein sollte und
welche nicht. Auch gibt es streng genommen keine „natürlichen Grenzen``,
wie die angesehenen Historiker Owen Lattimore und C. R. Whittaker
gezeigt haben. „Eine wichtige imperiale Grenze``, sagte Lattimore und
schrieb über das kaiserliche China, „ist nicht nur eine Linie, die
geographische Regionen und menschliche Gesellschaften teilt. Sie
repräsentiert auch das optimale Wachstum einer bestimmten
Gesellschaft.`` \footnote{Owen Lattimore, \emph{Inner Asian Frontiers of
  China} (New York: Beacon Press, 1960), S.60. Zitiert von Ronald
  Findlay, \emph{Towards a Model of Territorial Expansion and the Limits
  of Empire}, in Michelle R. Garfinkel und Stergios Skaperdas, eds.,
  \emph{The Political Economy of Conflict and Appropriation} (Cambridge:
  Cambridge University Press, 1996), S. 54.} Oder, wie der
Wirtschaftswissenschaftler der Columbia University, Ronald Findlay, es
ausdrückte: „Soweit sie in der Wirtschaft überhaupt berücksichtigt
werden, werden die Grenzen eines gegebenen Wirtschaftssystems oder
‚Landes' in der Regel als gegeben betrachtet, zusammen mit der
Bevölkerung, die innerhalb dieser Grenzen lebt. Doch es ist
offensichtlich, dass diese Grenzen, egal wie sehr sie im internationalen
Recht auch heiliggesprochen werden, irgendwann von rivalisierenden
Anspruchstellern angefochten wurden und letztlich durch das
Gleichgewicht der wirtschaftlichen und militärischen Macht zwischen den
streitenden Parteien bestimmt wurden.`` \footnote{Findlay, ebenda, S.
  41.}

Jemand mit allen verfügbaren Daten über die Hälfte der Nationalstaaten
der Welt und einer Sammlung von feinen Satellitenkarten wäre nicht in
der Lage vorherzusagen, wo die Grenzen der anderen Nationen liegen. Es
gibt auch keine wissenschaftliche Methode, um biologisch oder
linguistisch die Mitglieder einer Nationalität von denen einer anderen
zu unterscheiden. Kein Autopsieverfahren, egal wie fortgeschritten,
könnte nach einem Flugzeugabsturz genetisch die Überreste von
Amerikanern, Kanadiern und Sudanesen unterscheiden. Die Grenzen zwischen
Staaten und Nationalitäten sind nicht natürlich, wie die Grenzen
zwischen Spezies oder der physischen Unterschiede zwischen Tierrassen.
Sie sind vielmehr Artefakte von vergangenen und laufenden Bemühungen,
Macht auszuüben.

\begin{quote}
„Eine Sprache ist ein Dialekt mit einer Armee und einer Marine`` - Mario
Pei
\end{quote}

\section{SPRACHEN ALS AUSDRUCK VON
MACHT}\label{sprachen-als-ausdruck-von-macht}

Überraschenderweise lässt sich das Gleiche auch auf Sprachen anwenden.
Nach Jahrhunderten der Vorherrschaft des Nationalstaats mag der Gedanke,
dass die „Sprache`` keine objektive Grundlage zur Unterscheidung
zwischen Völkern bildet, als unüberlegt oder sogar absurd erscheinen.
Aber sehen Sie genauer hin. Die Geschichte der modernen Sprachen
offenbart deutlich, inwiefern sie geformt wurden, um die nationale
Identifikation zu verstärken. Westliche „Sprachen``, wie wir sie heute
verstehen und sprechen, haben sich nicht natürlich zu ihren
gegenwärtigen Formen entwickelt. Sie sind auch nicht objektiv von
„Dialekten`` zu unterscheiden. In der modernen Welt möchte niemand einen
„Dialekt`` sprechen. Fast jeder zieht es vor, dass seine Muttersprache
als echte „Sprache`` betrachtet wird.

\begin{quote}
„Lassen Sie niemanden behaupten, das Wort sei in solchen Momenten von
wenig Nutzen. Wort und Handlung sind eins. Die kraftvolle, energische
Bestätigung, die die Herzen beruhigt, erschafft Taten - das, was gesagt
wird, wird produziert. Hier ist die Handlung der Diener des Wortes, sie
folgt ihm unterwürfig nach, wie am ersten Tag der Welt: Er sagte und die
Welt war.`` -Michelet, August 1792
\end{quote}

\subsection{„Wort und Handlung sind zusammen
eins``}\label{wort-und-handlung-sind-zusammen-eins}

Vor der Französischen Revolution hatte beispielsweise die in
Südfrankreich gesprochene Version des gemischten Lateins, la langue d'oc
oder Okzitanisch, mehr mit der in Katalonien in Nordspanien gesprochenen
Volkssprache gemein als mit la langue d'oil, der Pariser Sprache, die
zur Grundlage von „Französisch`` wurde. Tatsächlich war die „Erklärung
der Rechte des Menschen und des Bürgers``, die im Pariser Stil verfasst
war, für die Mehrheit der Menschen innerhalb der heutigen Grenzen
Frankreichs unverständlich.\footnote{Billig, ebenda, S. 25.} Eine der
Herausforderungen, vor denen die französischen Revolutionäre standen,
war die Berechnung, wie sie ihre Flugblätter und Erlasse in die Dialekte
unzähliger Dörfer übersetzen konnten, die sich nur vage miteinander
verständigen konnten.

Die Menschen, die in dem Gebiet lebten, das später zu „Frankreich``
wurde, sprachen auf unterschiedliche Weisen. Diese verschiedenen
Sprachformen wurden bewusst zu einer offiziellen Sprache
zusammengeführt, und dies war eine gezielte politische Entscheidung.
Seit dem Edikt von Villers-Cotterets, das 1539 von Franz I.\footnote{Anderson,
  ebenda, S. 93.} erlassen wurde, war geschriebenes Französisch die
offizielle Sprache der Gerichte. Doch das bedeutete nicht, dass es
allgemein verständlich war. Genauso wenig war das „Rechtsfranzösisch``
in England nach 1200 allgemein verständlich, als es zur offiziellen
Sprache der Gerichte wurde. Beide waren eher „Verwaltungsdialekte`` und
keine standardisierten Sprachen, die im gesamten Territorium gesprochen
und verstanden wurden.

Die französischen Revolutionäre wollten etwas Umfassenderes erschaffen,
eine nationale Sprache. Der Historiker Janis Langins merkt in \emph{The
Social History of Language} an, dass „eine einflussreiche Meinungsgruppe
unter den Revolutionären glaubte, dass der Triumph der Revolution und
die Verbreitung der Aufklärung durch einen bewussten Versuch, ein
Standardfranzösisch im Territorium der Republik zu etablieren, gefördert
würde.`` \footnote{Janis Langins, \emph{Words and Institutions During
  the French Revolution: The Case of
  \textquotesingle Revolutionary\textquotesingle{} Scientific and
  Technical Education}, in Peter Burke und Roy Porter, \emph{The Social
  History of Language} (Cambridge: Cambridge University Press, 1987), S.
  137.}

Zu dieser „bewussten Bemühung`` gehörte auch, dass man sich über die
Verwendung einzelner Wörter den Kopf zerbrach. Man denke nur an das
Beispiel des Adjektivs „revolutionär``, das Marabou erstmals 1789
verwendete. Nach einer Phase der „ziemlich breiten und undifferenzierten
Verwendung``, wie Langins es ausdrückt, „folgte während des Terrors eine
Phase der Unterdrückung und Vergessenheit, die mehrere Jahrzehnte
andauerte... Am 12. Juni 1795 entschied die Konvention, die Sprache
ebenso wie die von unseren ehemaligen Tyrannen (d.h. den besiegten
Robespierristen) geschaffenen Institutionen zu reformieren, und ersetzte
das Wort ‚revolutionär' in offiziellen Bezeichnungen.`` \footnote{Ebenda,
  S. 140, 142.} Diese Tradition der Sprachplanung überlebt in der
pingeligen Aufnahme der französischen Behörden von Wörtern wie
„weekend``, die ihren Weg ins Französische aus dem Englischen gefunden
haben.

Vor zwei Jahrhunderten diskriminierten die nationalen
Sprachkonstrukteure in Frankreich jedoch nicht nur Wörter aus Übersee;
sie standen vor der weitaus größeren Aufgabe, lokale Sprachvarianten
innerhalb des Territoriums der Republik auszurotten. Diese Übung
beschränkte sich nicht nur auf die Unterdrückung der langue d'oc. Das
damals an der Riviera gesprochene „Französisch`` stand dem
„Italienisch`` der Ostküste näher als dem Pariser Französisch.
Gleichzeitig hätte man die Sprache des Elsass eventuell als eine Form
des Deutschen bezeichnen können, das selbst zahlreiche lokale Varianten
hatte. Das Baskische wurde in den Pyrenäen gesprochen. Wie das
Bretonische, das entlang der Nordwestküste Frankreichs gesprochen wurde,
hatte das Baskische wenig gemeinsam mit irgendeinem der „Volksdialekte``
des Lateinischen, die die Basis des „Französischen`` bildeten. Es gab
auch eine erhebliche Anzahl flämischer Sprechender im Nordosten. „Die
Pariser Sprechweise``, wie uns Michael Billig erinnert, „wurde nicht
durch spontane Marktprozesse verbreitet, sondern gesetzlich und
kulturell als ‚Französisch' aufgezwungen.`` \footnote{Billig, ebenda, S.
  27.}

Was in Frankreich zutraf, galt auch anderswo beim Aufbau von
Nationalstaaten. Sprachen wurden oft von Armeen getragen und von
Kolonialmächten aufgezwungen. Zum Beispiel wurde die Landkarte Afrikas
nach der Unabhängigkeit gemäß den Gebieten definiert, in denen die
Verwaltungssprachen der europäischen Mächte dominierten. Lokale Dialekte
wurden selten in Schulen unterrichtet. Die Unterscheidungen zwischen
anerkannten „Sprachen``, die dazu neigten, „Nationen`` zu definieren,
selbst Nationen mit willkürlichen kolonialen Grenzen, und „Dialekten``,
die dies nicht taten, waren in hohem Maße politisch.

Kurzum, die Durchsetzung einer „Nationalsprache`` war Teil eines
weltweit angewandten Verfahrens zur Stärkung der Staatsmacht. Die
Förderung oder Verpflichtung aller innerhalb des Territoriums, in dem
der Staat das Gewaltmonopol innehatte, die „Muttersprache`` zu sprechen,
brachte bedeutende Vorteile für die Ausübung von Macht mit sich.

\subsection{Die militärische Dimension der sprachlichen
Einheitlichkeit}\label{die-milituxe4rische-dimension-der-sprachlichen-einheitlichkeit}

In einer Welt, in der die Gewaltbereitschaft stieg, brachte die
Einführung einer Nationalsprache militärische Vorteile mit sich. Eine
Nationalsprache war fast eine Voraussetzung für die Konsolidierung der
Zentralmacht in Nationalstaaten. Zentrale Behörden, die ihre Bürger dazu
ermutigten, die gleiche Sprache zu sprechen, waren besser in der Lage,
die militärische Macht lokaler Magnaten zu schwächen. Die
Standardisierung der Sprache nach der Französischen Revolution
ermöglichte die preiswerteste und effektivste Form moderner
militärischer Macht - die nationalen Wehrpflichtarmeen. Eine gemeinsame
Sprache ermöglichte es den Truppen aus allen Regionen der „Nation``,
fließend miteinander zu kommunizieren. Dies war eine Voraussetzung,
bevor massenweise Wehrpflichtarmeen unabhängige Bataillone verdrängen
konnten, die nicht von den zentralen Behörden, sondern von mächtigen
lokalen Magnaten aufgestellt und kontrolliert wurden.

Vor der Französischen Revolution, wie wir in Kapitel 5 besprochen haben,
wurden Truppen von lokalen Machthabern aufgestellt und befehligt, die
möglicherweise auf Kampfaufforderungen aus Paris oder einer anderen
Hauptstadt reagierten. In jedem Fall wurde ihre Haltung nach
sorgfältigen Verhandlungen festgelegt. Wie Charles Tilly anmerkt, bot
die „Fähigkeit, Unterstützung zu gewähren oder zu verweigern... große
Verhandlungsmacht``.\footnote{Tilly, \emph{Coercion, Capital, and
  European States}, S. 22.} Darüber hinaus war ein weiterer Nachteil
unabhängiger Militäreinheiten aus Sicht der zentralen Behörden die
Fähigkeit, sich den Bemühungen der Regierung zu widersetzen, die
einheimischen Ressourcen zu beschlagnahmen. Es war offensichtlich eine
schwierige Herausforderung für die zentralen Behörden, ob König oder
Revolutionskonvent, Steuern zu erheben oder anderweitig Ressourcen von
lokalen Machthabern zu entziehen, die private Armeen befehligten, die in
der Lage waren, diese Vermögenswerte zu verteidigen.

„Nationalarmeen`` stärkten besonders die Macht der nationalen Regierung,
ihren Willen im ganzen Gebiet durchzusetzen. Die Einführung einer
Nationalsprache spielte eine entscheidende Rolle bei der Bildung von
Nationalarmeen. Bevor sich Nationalarmeen formierten und effektiv
funktionieren konnten, war es offensichtlich nützlich, dass ihre
verschiedenen Mitglieder fließend kommunizieren konnten.

Es war daher ein militärischer Vorteil, wenn jeder innerhalb einer
Zuständigkeit Befehle und Anweisungen verstehen sowie bestimmte
Informationen entlang der bürokratischen Befehlskette zurückvermitteln
konnte. Die französischen Revolutionäre demonstrierten den Wert dieser
Fähigkeit fast sofort. Neben dem Betrieb eines Äquivalents einer
Sprachschule richteten sie auch spezielle, monatliche „Intensivkurse``
ein, in denen laut Langins „Hunderte von Studenten aus ganz Frankreich
in den Techniken zur Herstellung von Schießpulver und Kanonen geschult
wurden.`` \footnote{Langins, ebenda, S. 143.}

Der militärische Vorteil des französischen Ansatzes zeigte sich durch
ihre Erfolge in der napoleonischen Periode sowie durch gegensätzliche
Beispiele von dem, was Regimen passierte, die im Krieg nicht auf die
Mobilisierungsvorteile einer gemeinsamen Sprache bauen konnten. Einer
von vielen Faktoren, die zu den katastrophalen Niederlagen und der
Demoralisierung der russischen Streitkräfte in den frühen Tagen des
Ersten Weltkriegs beitrugen, war die Tatsache, dass das aristokratische
Offizierskorps des Zaren in der Regel auf Deutsch kommunizierte (die
andere Hofsprache der Romanows war Französisch), eine Sprache, die die
einfachen Soldaten und umso mehr die Bevölkerung, nicht verstanden.

Dies weist auf einen weiteren wichtigen militärischen Vorteil einer
gemeinsamen Sprache hin. Es verringert die motivationalen Hürden zur
Kriegsführung. Propaganda ist nutzlos, wenn sie unverständlich ist. In
dieser Hinsicht waren auch die französischen Revolutionäre gut auf die
Möglichkeiten abgestimmt. Ihre „dominante Idee``, so Langins, war „der
Wille des Volkes. Sie mussten sich daher mit dem Volkswillen
identifizieren, indem sie ihn in seiner eigenen spezifischen Sprache
ausdrückten.`` \footnote{Ebenda, S. 139.} Vor 1789 war die
wechselseitige Unverständlichkeit unter den „Staatsbürgern`` ein
Nachteil bei der Darstellung des „Willens des Volkes`` und somit eine
Kontrolle über die Ausübung von Macht auf nationaler Ebene. In vielerlei
Hinsicht standen mehrsprachige Staaten und Imperien während des
Industriezeitalters höheren Hindernissen gegenüber, sich für den Krieg
zu mobilisieren.

Im Zweifelsfall wurden sie daher tendenziell von Nationalstaaten
verdrängt, die ihre Bürger besser zum Kampf motivieren und Ressourcen
für den Krieg mobilisieren konnten. Dies wird durch die nationale
Konsolidierung veranschaulicht, wie beispielsweise die Erfindung von
Frankreich und den Franzosen am Ende des 18. Jahrhunderts. Es wird auch
durch Fälle nationalistischer Auflösung verdeutlicht, wie den
Zusammenbruch des österreichisch-ungarischen Reiches nach dem Ersten
Weltkrieg. Die neuen Nationalstaaten, die im Zuge des Habsburgerreiches
- Österreich, Ungarn, die Tschechoslowakei und Jugoslawien - entstanden
sind, waren, wie Keynes sagte, „unvollständig und unreif``. Doch ihre
Forderungen nach der Bildung unabhängiger Nationalstaaten, die sich um
nationale Identitäten gruppierten, die zumindest teilweise durch Sprache
definiert waren, überzeugten Woodrow Wilson und andere führende
Politiker der Alliierten, die den Vertrag von Versailles ausarbeiteten.

Die Aufteilung Mitteleuropas nach dem Ersten Weltkrieg veranschaulicht,
wie zweischneidig die Sprache beim Staatsaufbau wurde. Wenn die
Gewalttätigkeit zunahm, erleichterte eine gemeinsame Sprache die
Ausübung von Macht und die Konsolidierung von Zuständigkeitsbereichen.
Wenn jedoch die Anreize zur Konsolidierung schwächer waren, neigten die
von Minderheiten gebildeten Fraktionen, die sich um Sprachstreitigkeiten
rankten, ebenfalls dazu, mehrsprachige Staaten zu zersplittern. Der
Anstieg des separatistischen Empfindens in den Städten des
österreichisch-ungarischen Reiches im 19. Jahrhundert folgte auf
Epidemien, die die deutschsprachige Bevölkerung dezimierten. Prag war zu
Beginn des 19. Jahrhunderts eine deutschsprachige Stadt. Wie andere
Städte wuchs sie im Laufe des Jahrhunderts durch Migration rasant,
hauptsächlich durch die Assimilation von landlosen,
tschechischsprachigen Bauern vom Land. Die Neuankömmlinge mussten
anfangs Deutsch lernen, um sich zurechtzufinden, was sie auch taten.
Doch als Mitte des Jahrhunderts Hungersnöte und Krankheiten viele
deutschsprachige Stadtbewohner dahinrafften, wurden sie von
tschechischsprachigen Bauern ersetzt. Plötzlich gab es so viele
tschechischsprachige Menschen, dass es für die neuen Bewohner nicht mehr
nötig war, Deutsch zu lernen. So wurde Prag zu einer
tschechischsprachigen Stadt und dem Zentrum des tschechischen
Nationalismus.

Zeitgenössische Sezessionsbewegungen bilden sich in mehrsprachigen
Ländern nun häufig um Sprachkonflikte herum. Das ist offensichtlich in
Belgien und Kanada der Fall, zwei Nationen, die, wie wir zuvor
bemerkten, wahrscheinlich zu den ersten in der OECD gehören werden, die
sich im neuen Jahrtausend auflösen. Wenige Regierungen können die
repressiven Maßnahmen zur Durchsetzung der Spracheinhaltung übertreffen,
die von der Parti Quebecois in Quebec eingeführt wurden.\footnote{Rheal
  Seguin, \emph{PQ Ready to Harden Laws on Language: English Signs Face
  Ban in Quebec}, \emph{Globe and Mail}, 29. August 1996, S. A1.}
Überraschenderweise spielten auch Sprachbeschwerden bei der Initiierung
der frühen Aktivitäten der nördlichen Sezessionisten in Italien eine
Rolle, das ebenfalls mit der Auflösung konfrontiert ist. In den frühen
1980er Jahren erklärte die damals so genannte Lombarden-Liga das
Lombardische „zu einer von der italienischen getrennten Sprache``.
Billig merkt an: „Hätte das Programm der Liga während der frühen 1980er
Jahre Erfolg gehabt, und hätte sich die Lombardei von Italien losgesagt
und ihre eigenen Staatsgrenzen etabliert, könnte man vorhersagen: immer
mehr hätte das Lombardische als verschieden vom Italienischen anerkannt
werden müssen.`` \footnote{Billig, ebenda, S. 35.} Das ist keine
willkürliche Behauptung. Es spiegelt wider, was in ähnlichen Fällen
passiert ist. Zum Beispiel starteten norwegische Nationalisten, nachdem
Norwegen 1905 unabhängig wurde, einen gezielten Versuch, Merkmale der
„Norwegischen Sprache`` zu identifizieren, die sich von Dänisch und
Schwedisch unterscheiden. Im gleichen Sinne änderten Aktivisten, die ein
unabhängiges Weißrussland befürworteten, Verkehrsschilder in
„Weißrussisch``, konnten jedoch offenbar nicht verdeutlichen, dass
Weißrussisch eine eigenständige Sprache und kein russischer Dialekt ist.

Jetzt, da die militärischen Zwänge, die eine einheitliche Sprache
begünstigen, weitgehend überholt sind, erwarten wir, dass die nationalen
Sprachen verschwinden werden, aber nicht ohne Kampf. Es ist zu erwarten,
dass das altbewährte Sprichwort „Krieg ist die Gesundheitskur des
Staates`` ausführlich getestet werden wird. Während der Nationalstaat an
Bedeutung verliert, werden Demagogen und Reaktionäre Kriege und
Konflikte schüren, ähnlich den ethnischen und stammesgebundenen
Auseinandersetzungen, die das ehemalige Jugoslawien und zahlreiche
Gebiete in Afrika, von Burundi bis Somalia, erschüttert haben. Konflikte
werden sich als praktisch erweisen, da sie jenen, die versuchen, den
Trend zur Kommerzialisierung der Souveränität aufzuhalten, Vorwände
liefern. Kriege werden die Bemühungen um die Aufrechterhaltung
strengerer Steuersysteme und die Verhängung härterer Strafen für die
Flucht vor den Pflichten und Lasten der Staatsbürgerschaft erleichtern.
Kriege werden die „Wir gegen sie``-Dimension des Nationalismus stützen.
Den Befürwortern des systematischen Zwangs wird die kommerzielle
Souveränität, die dem Einzelnen eine Auswahl von Souveränitätsdiensten
auf der Grundlage von Preis und Qualität ermöglicht, nicht weniger als
eine Sünde erscheinen, als die Behauptung des Rechts des Einzelnen,
gegen die Urteile des Papstes während der Reformation Einspruch zu
erheben und seinen eigenen Weg zum Heil zu wählen.

Die Parallele wird durch die Tatsache unterstrichen, dass sowohl der
Buchdruck am Ende des fünfzehnten Jahrhunderts als auch die neue
Informationstechnologie am Ende des zwanzigsten Jahrhunderts okkultes
Wissen auf befreiende Weise für Einzelpersonen zugänglich gemacht haben.
Der Buchdruck brachte die Heilige Schrift und andere heilige Texte
direkt in die Reichweite von Individuen, die zuvor auf Priester und die
Kirchenhierarchie angewiesen waren, um das Wort Gottes zu
interpretieren. Die neue Informationstechnologie bringt Informationen
über Handel, Investitionen und aktuelle Ereignisse in die Reichweite
jedes Individuums mit einem Computeranschluss. Informationen, die zuvor
nur für Personen an der Spitze von Regierungs- und
Unternehmenshierarchien verfügbar waren.

\begin{quote}
„Die Entwicklung des Druck- und Verlagswesens ermöglichte ein neues
nationales Bewusstsein und förderte den Aufstieg der modernen
Nationalstaaten.`` \footnote{Jack Weatherford, \emph{Savages and
  Civilization: Who Will Survive?} (New York: Fawcett Columbine, 1994),
  S. 143.} - Jack Weatherford
\end{quote}

\subsection{Rock and Roll im
Cyberspace}\label{rock-and-roll-im-cyberspace}

Geben Sie sich keinen Illusionen hin: Das Aufkommen des Internets und
des World Wide Webs wird den Nationalismus ebenso zerstören wie das
Aufkommen des Schießpulvers und der Druckerpresse den Nationalismus
begünstigt hat. Globale Computerverbindungen werden Latein als
universelle Sprache nicht zurückbringen, aber sie werden dazu beitragen,
den Handel von lokalen Dialekten, wie dem Französischen in Quebec, in
die neue globale Sprache des Internets und des World Wide Web zu
verlagern - die Sprache, die Otis Redding und Tina Turner der Welt
beigebracht haben, die Sprache des Rock and Roll, Englisch.

Diese neuen Medien werden den Nationalismus untergraben, indem sie neue
Zugehörigkeiten schaffen, die geografische Grenzen überflüssig machen.
Sie werden ein weit verstreutes Publikum ansprechen, das sich überall
dort bildet, wo sich gebildete Menschen gerade aufhalten. Diese neuen,
nicht-territorialen Bindungen werden gedeihen und dabei helfen, einen
neuen Fokus für „Patriotismus`` zu schaffen. Besser gesagt werden sie
neue „In-Groups`` bilden, mit denen sich der Einzelne identifizieren
kann, ohne unbedingt seine wirtschaftliche Rationalität zu opfern. Die
Geschichte der Juden in den letzten zweitausend Jahren zeigt, dass dies
langfristig und unter feindlichen lokalen Bedingungen möglich ist. Wie
der zu Beginn dieses Kapitels zitierte Kommentar von William Pfaff
nahelegt, ist es historisch falsch zu denken, dass die Loyalität zum
Vaterland notwendigerweise Loyalität zu einer Institution ähnlich einem
Nationalstaat bedeutet. Geoffrey Parker und Lesley M. Smith
verdeutlichen dies in \emph{The General Crisis of the Seventeenth
Century}, indem sie zeigen, dass scheinbare Beispiele für frühmodernen
Nationalismus oft Beispiele von Patrioten sind, die eine viel engere
Definition des Vaterlandes verteidigen - oft gegen die Eingriffe eines
Staates. Sie schreiben: „Allzu oft stellt sich bei genauer Betrachtung
heraus, dass eine vermeintliche Zugehörigkeit zu einer nationalen
Gemeinschaft gar keine ist. Das Vaterland selbst ist mindestens genauso
wahrscheinlich eine Heimatstadt oder Provinz wie die ganze Nation.``
\footnote{Geoffrey Parker und Lesley M. Smith, \emph{The General Crisis
  of the Seventeenth Century} (London: Routledge \& Kegan Paul, 1985),
  S. 122.}

Wie Jack Weatherford in seinem Buch \emph{Savages and Civilization}
anschaulich erklärt, hatte der Aufstieg des Buchdrucks, der ersten
Massenproduktionstechnologie, dramatische Auswirkungen auf die
Entstehung der Politik, die Loyalität gegenüber einem umfassenderen
Nationalstaat fordert. Um das Jahr 1500 gab es Druckpressen an 236 Orten
in Europa, „die zusammen etwa 20 Millionen Bücher gedruckt hatten.``
\footnote{Weatherford, ebenda, S. 144.} Gutenbergs erstes gedrucktes
Buch war eine Ausgabe der Bibel auf Latein. Es folgten Ausgaben anderer
populärer mittelalterlicher Bücher auf Latein. Wie Weatherford erklärt,
ging der Druck in eine Richtung, die den frühen Erwartungen widersprach,
dass die leicht verfügbaren Texte die Verwendung von Latein und sogar
Griechisch ausweiten würden. Im Gegenteil. Es gab zwei wichtige Gründe,
warum die Druckpresse die Verwendung von Latein nicht verstärkte.
Erstens war die Druckpresse eine Massenproduktionstechnologie. Wie
Benedict Anderson hervorhebt: „Während handschriftliches Wissen knapp
und esoterisch war, lebte das gedruckte Wissen von Reproduzierbarkeit
und Verbreitung.`` \footnote{Anderson, ebenda, S. 90.} Sehr wenige
Europäer waren im Jahr 1500 mehrsprachig. Das bedeutete, dass das
Publikum für Werke in Latein kein Massenpublikum war. Die überwiegende
Mehrheit, die nur eine Sprache sprach, bildete einen viel größeren Markt
potenzieller Leser. Darüber hinaus galt, was für die Leser zutraf, noch
mehr für die Autoren. Verleger brauchten Produkte zum Verkaufen.

Da es nur wenige zeitgenössische Autoren des 15. oder 16. Jahrhunderts
gab, die zufriedenstellende neue Werke in lateinischer Sprache verfassen
konnten, wurden die Verleger durch die Notwendigkeit des Marktes dazu
gebracht, Werke in der Volkssprache zu veröffentlichen. Der Buchdruck
trug somit dazu bei, Europa in sprachliche Untergruppen zu unterteilen.
Dies wurde nicht nur durch die Veröffentlichung neuer Werke gefördert,
die die Identität neuer Sprachen wie Spanisch und Italienisch
begründeten, sondern auch durch die Einführung charakteristischer
Schrifttypen wie Roman, Italic und der schweren gotischen Schrift, die
im deutschen Verlagswesen bis weit ins 20. Jahrhundert üblich war. Das
neue Verlagswesen in der Landessprache, was Anderson als
„Druckkapitalismus`` beschreibt, war äußerst erfolgreich. Besonders
hervorzuheben ist, dass die Druckerpresse der Ketzerei den
entscheidenden Schub gab, den wir auch von der Entnationalisierung des
Einzelnen durch das Internet erwarten. Insbesondere wurde Luther dadurch
als „der erste Bestsellerautor bekannt. Oder anders ausgedrückt, der
erste Autor, der seine neuen Bücher auf der Grundlage seines Namens
‚verkaufen' konnte.`` \footnote{Ebenda, S.91.} Erstaunlicherweise
machten Luthers Werke „nicht weniger als ein Drittel aller auf Deutsch
verkauften Bücher zwischen 1518 und 1525`` aus.\footnote{Ebenda.}

In vielerlei Hinsicht wird die neue Technologie des
Informationszeitalters einem Teil der megapolitischen Auswirkungen der
Technologie des fünfzehnten Jahrhunderts, der Druckerpresse, die den
Aufstieg von Nationalstaaten anregte und unterstützte, entgegenwirken.
Das World Wide Web schafft einen kommerziellen Raum mit einer globalen
Sprache, dem Englischen. Es wird schließlich durch
Simultanübersetzungssoftware verstärkt, die fast jeden effektiv
mehrsprachig macht und dazu beiträgt, Sprache und Vorstellungskraft zu
entnationalisieren. So wie die Technologie der Druckerpresse die
Zugehörigkeit zur dominierenden Institution des Mittelalters, der
Heiligen Mutter Kirche, untergraben hat, so erwarten wir, dass die neue
Kommunikationstechnologie des Informationszeitalters die Autorität des
Nanny-Staates untergraben wird. Mit der Zeit wird fast jeder Bereich
mehrsprachig werden. Lokale Dialekte werden an Bedeutung gewinnen.
Zentrale Propaganda wird weitgehend an Zusammenhalt verlieren, da
Migranten und Sprecher von Minderheitensprachen mutiger werden, sich der
Assimilation in die Nation zu widersetzen.

\section{MILITÄRISCHER MYSTIZISMUS}\label{milituxe4rischer-mystizismus}

Nationen sind weit davon entfernt, objektive Gemeinschaften zu sein, in
demselben Sinne, in dem beispielsweise
„Jäger-und-Sammler-Gemeinschaften`` objektiv sind, sondern sie werden
aus einer Mystik heraus erdacht, die von einem untergegangenen
militärischen Zwang inspiriert ist, nämlich dem Anspruch, jede in einem
Territorium lebende Person durch ein Identitätsgefühl zu verbinden, das
wichtiger erscheinen soll als das eigene Leben. Wie Kantorowicz
bemerkte, ist es kein Zufall, dass „zu einem bestimmten Zeitpunkt in der
Geschichte der Staat im Abstrakten oder der Staat als Unternehmen als
mystischer Körper erschien und dass der Tod für diesen neuen mystischen
Körper gleichwertig mit dem Tod eines Kreuzfahrers für die Sache Gottes
war!{}`` \footnote{E. H. Kantorowicz, zitiert von Llobera, ebenda, S.
  83.} In diesem Sinne kann der Nationalstaat als mystisches Konstrukt
verstanden werden. Doch wie Billig bemerkt, ist der Nationalismus „eine
banale Mystik, die so banal ist, dass alles Mystische schon lange
verdampft zu sein scheint.`` Sie „bindet ‚uns' an die Heimat - diesen
besonderen Ort, der mehr als nur ein Ort ist, mehr als ein bloßes
geophysikalisches Gebiet. In all dem wird die Heimat heimelig gemacht,
als unantastbar und, sollte der Anlass entstehen, den Preis des Opfers
wert. Und vor allem die Männer erhalten ihre speziellen, lustvollen
Erinnerungen an die Möglichkeiten des Opfers.`` \footnote{Billig,
  ebenda, S. 175.}

Die phantasievolle Verbindung zwischen der Nation und der Heimat wird
von Nationalisten bei jeder Gelegenheit hervorgehoben. Wie Billig
vorschlägt, wird die Nation als „heimischer Raum, gemütlich innerhalb
seiner Grenzen, sicher vor der gefährlichen Außenwelt in den
Vorstellungen präsentiert. Und ‚wir', die Nation innerhalb der Heimat,
können ‚uns' so leicht als eine Art Familie vorstellen.`` \footnote{Ebenda,
  S. 109.} Die Klischees des Nationalismus, die unermüdlich und
routinemäßig wiederholt werden, beinhalten viele alltägliche Metaphern
von Verwandtschaft und Identität. Sie assoziieren die Nation mit dem
Gefühl des Individuums für „inklusive Fitness``, ein starkes Motiv für
Altruismus und Aufopferung.

\begin{quote}
„Dass es altruistischen Opfermut bei sozialen Insekten, anderen Tieren
und Menschen gibt, impliziert, dass die Maximierung des Eigeninteresses
nicht ausschließlich in Bezug auf die Wünsche und Bedürfnisse eines
individuellen Organismus definiert werden kann. Tatsächlich hat die
Präsenz von Altruismus, insbesondere gegenüber Verwandten, in den
biologischen Wissenschaften eine vollständige Neubewertung der
traditionellen Vorstellungen vom Überleben des Stärksten erfordert. Dies
hat zu der wachsenden Überzeugung geführt, dass die natürliche Selektion
letztendlich nicht auf der Ebene des Individuums operiert.`` \footnote{Shaw
  und Wong, ebenda, S. 26-27.} - R. Paul Shaw und Yuwa Wong
\end{quote}

\section{NATIONALISMUS UND INKLUSIVE
FITNESS}\label{nationalismus-und-inklusive-fitness}

Unser Hauptaugenmerk in diesem Buch liegt auf objektiven
„megapolitischen`` Faktoren, die die Kosten und Belohnungen menschlicher
Entscheidungen verändern. Die zugrunde liegende Annahme, auf der die
Vorhersagekraft der Analyse beruht, ist, dass Individuen Belohnungen
suchen und Kosten vermeiden wollen. Dies ist eine wesentliche Wahrheit
dessen, was Charles Darwin „die Wirtschaft der Natur`` nannte. Aber es
ist nicht die ganze Wahrheit. Einfache Belohnungsoptimierung erklärt
nicht alles im Leben. Sie beleuchtet jedoch zwei der drei Hauptformen
menschlicher Gesellschaften, die Pierre van den Berghe als „Reziprozität
und Zwang`` \footnote{Pierre Van Den Berghe, \emph{A Socio-Biological
  Perspective}, in Hutchinson und Smith, eds., Nationalism, S. 97.}
identifizierte. Unter „Reziprozität`` versteht van den Berghe
„Kooperation zum gegenseitigen Nutzen``.\footnote{Ebenda.} Die
komplexesten und weitreichendsten Beispiele für Reziprozität sind
Marktinteraktionen: Handeln, Kaufen, Verkaufen, Produzieren und andere
wirtschaftliche Aktivitäten. „Zwang ist der einseitige Nutzen von Kraft,
das heißt für Zwecke von artenspezifischem Parasitismus oder Räuberei.``
\footnote{Ebenda.} Wie wir in diesem Band und zwei früheren Büchern
untersucht haben, glauben wir, dass Zwang ein entscheidendes Element in
der menschlichen Gesellschaft ist, ein größeres, als gewöhnlich erkannt
wird. Zwang hilft, die Sicherheit von Eigentum zu bestimmen und begrenzt
die Fähigkeit von Einzelpersonen, in gegenseitig vorteilhafte
Zusammenarbeit einzutreten. Zwang liegt jeder Politik zugrunde. Das
dritte Element in van den Berghes Typologie der menschlichen
Gesellschaft ist die „Verwandtenselektion``, das kooperative Verhalten,
das Tiere mit ihren Verwandten zeigen. Verwandtenselektion, die weiter
unten ausführlicher beschrieben wird, ist auch ein entscheidendes
Merkmal der „Wirtschaft der Natur``.

Wie Jack Hirshleifer geschrieben hat: „Die Wiederbelebung der
Darwin'schen Selektionstheorie in Bezug auf Probleme des sozialen
Verhaltens, die mittlerweile als Soziobiologie bekannt ist``, hat „einen
ausgesprochen wirtschaftlichen Aspekt.`` Und weiter:

\begin{quote}
Die Soziobiologie versucht in dem gesamten Spektrum des Lebens die
allgemeinen Gesetze zu finden, die die vielfältigen Formen des
Zusammenlebens von Organismen bestimmen. Zum Beispiel: Warum beobachten
wir manchmal Sex und Familien, manchmal Sex ohne Familien, manchmal
weder Sex noch Familien? Warum leben manche Tiere in Schwärmen, während
andere Einzelgänger bleiben? Warum beobachten wir manchmal innerhalb von
Gruppen hierarchische Dominanzmuster und manchmal nicht? Warum teilen
Organismen einiger Arten Territorien auf und andere nicht? Was bestimmt
die Selbstlosigkeit der sozialen Insekten und warum ist dieses Muster in
der Natur so selten? Wann sehen wir eine friedliche
Ressourcenallokation, wann eine gewalttätige? Dies sind Fragen, die in
erkennbarer Weise ökonomisch gestellt und beantwortet werden.
Soziobiologen fragen, was die Netto-Vorteile der beobachteten
Verhaltensmuster für die sie anzeigenden Organismen sind und welche
Mechanismen dafür sorgen, dass diese Muster in sozialen Gleichgewichten
bestehen bleiben. Es ist vielleicht diese Behauptung der Kontinuität
zwischen dem wirtschaftlichen und dem biologischen Verhalten (von einem
Kritiker als „genetischer Kapitalismus`` bezeichnet), die die Abneigung
einiger Ideologen gegenüber der Soziobiologie erklärt. ...\footnote{Jack
  Hirshleifer, Economic Behaviour in Adversitv (Chicago: University of
  Chicago Press, 1987), S. 170.}
\end{quote}

Wir führen Soziobiologie in unsere Analyse des Nationalismus ein, weil
sie uns Aspekte der menschlichen Natur aufzeigt, die eine systematische
Zwangsdurchsetzung begünstigen. Wir stimmen mit dem Naturwissenschaftler
Cohn Tudge, dem Autor von \emph{The Time Before History}, überein, dass
wir, bevor wir die heutige Welt verstehen können - geschweige denn einen
Blick auf die zukünftige werfen können - das Vorwort zur Geschichte
verstehen müssen. Das bedeutet, dass wir uns „im großen Maßstab der Zeit
betrachten`` müssen.\footnote{Cohn Tudge, The Time Before History: 5
  Million Years of Human Impact (New York: Scribners, 1996), S. 17.}
Tudge erinnert uns daran, „dass unter der Oberfläche unseres Lebens viel
tiefergehende und stärkere Kräfte am Werk sind, die letztendlich uns
alle und alle unsere Mitgeschöpfe beeinflussen...\footnote{Ebenda, S.
  17-18.} Wir vermuten, dass zu diesen „tiefergehenden und stärkeren
Kräften`` auch eine genetisch bedingte Motivationskomponente gehört, die
dem Nationalismus zugrunde liegt. Wie Hirshleifer unter Berufung auf
Adam Smith und R.H. Coase anmerkt, „sind menschliche Wünsche
letztendlich anpassende Reaktionen, die durch die biologische Natur und
Situation des Menschen auf der Erde geformt sind``.\footnote{Hirshleifer,
  ebenda, S. 172.} Dies tritt bei den offensichtlich biologischen
Anspielungen in den meisten Diskussionen über Nationalismus in den
Vordergrund. Selbst in den Vereinigten Staaten, einer auffallend
multiethnischen Nation, wird die Regierung in familiären Begriffen als
„Uncle Sam`` personifiziert.

\subsection{Die biologische Vererbung}\label{die-biologische-vererbung}

Kurz gesagt sind die menschliche Natur, die Entstehung der Arten und
ihre Entwicklung durch natürliche Selektion, Elemente, die bei dem
Verständnis der fortlaufenden Evolution der menschlichen Gesellschaft
berücksichtigt werden sollten. Im vorliegenden Fall geht es um die
wahrscheinliche menschliche Reaktion auf neue Umstände, die durch die
Informationstechnologie hervorgerufen werden. Insbesondere konzentrieren
wir uns auf die Reaktion des Aufkommens der Cyberwirtschaft und ihren
vielen Folgen, einschließlich des Aufkommens einer wirtschaftlichen
Ungleichheit, die stärker ist als alles bisher Gesehene. Schlüssel zu
zumindest einigen der erwarteten Reaktionen liegen in unserem
genetischen Erbe.

Wenn eine neue Spezies entsteht, wirft sie nicht die gesamte DNA ab, die
sie in ihrer vorherigen Form trug, sondern ergänzt sie. Der gesamte
Unterschied zwischen einem Menschen und einem Schimpansen liegt in
weniger als 2 Prozent der DNA der jeweiligen Spezies; etwas über 98
Prozent ihrer DNA ist beiden gemein, und ein Teil davon kann bis zu sehr
primitiven frühen Organismen zurückverfolgt werden, weit unten in der
historischen Kette der Entwicklung.

\section{GENETISCHE TRÄGHEIT}\label{genetische-truxe4gheit}

Menschliche Kulturen enthalten ebenso universelle Elemente, von denen
einige tatsächlich von vorhumanen Vorfahren vererbt werden. Wie wir
Nahrung suchen, wie wir uns paaren, wie wir Familien bilden, wie wir uns
gegenüber fremden Gruppen verhalten, wie wir uns verteidigen - all dies
sind komplexe Mischungen aus Instinkt und Kultur mit sehr primitiven
Wurzeln. Sie sind jedoch alle in der Lage, moderne Anpassungen
vorzunehmen, wie beispielsweise die, die den Nationalstaat in der
modernen Ära geprägt haben. Wenn wir Kulturen auf diese Weise
betrachten, werden wir sie als parallel zur genetischen Entwicklung
sehen. Die drei großen Unterschiede sind, dass Kulturen durch die
Informationskette zwischen Menschen, nicht durch die genetische Kette
zwischen den Generationen übertragen werden; sie können bis zu einem
gewissen Grad, vielleicht weniger, als wir denken, durch bewusste
intelligente Handlungen verändert werden; sie ändern sich mit der
vorherrschenden Umgebung von Kosten und Belohnungen, die viel schneller
mutiert als die genetische Veränderung. Physisch sind wir unseren
Vorfahren von vor dreißigtausend Jahren sehr ähnlich; kulturell haben
wir uns ziemlich weit von ihnen entfernt.

\subsection{Evolutionsmodelle}\label{evolutionsmodelle}

Es gibt zwei biologische Modelle für die Art und Weise, wie Arten sich
entwickeln. Die naturwissenschaftliche Orthodoxie ist neo-darwinistisch.
Zufällige genetische Veränderungen führen zu unterschiedlichen
physischen Formen. Die meisten dieser Formen bringen keinen
Überlebensvorteil, wie zum Beispiel die Albino-Amsel und sie tendieren
dazu auszusterben. Eine geringe Anzahl von ihnen ist vorteilhaft für das
Überleben und verbreitet sich durch die Gattung. Es gibt noch viele
Schwierigkeiten in dieser Theorie, die von Wissenschaftlern im nächsten
Jahrhundert gelöst werden könnten, aber die Zufälligkeit und das
Überleben vorteilhafter Anpassungen sind die derzeitige
naturwissenschaftliche Orthodoxie und haben eine gewisse
Erklärungskraft. Die Alternative ist eine Variante der Theorie des
frühen zwanzigsten Jahrhunderts des französischen Philosophen Henri
Bergson, der glaubte, dass die Natur über einen nicht-zufälligen
kreativen Zweck verfügt, eine intelligente Kraft, die nach Lösungen
sucht. Dieses Konzept findet sich wieder in den Werken zeitgenössischer
Autoritäten wie David Layzer und Stephen Jay Gould, die betont haben,
dass genetische Variation nicht einfach zufällig ist, sondern deutliche
Neigungen zeigt.\footnote{Siehe Stephen Jay Gould, \emph{Evolutionary
  Biology of Constraints}, Daedalus, Frühling 1980, und David Layzer,
  \emph{Altruism and Natural Selection}, Journal of Social and
  Biological Structures (1978), zitiert von Howard Margolis,
  Selfishness, Altruism and Rationality (Chicago: University of Chicago
  Press, 1984).} Dies ist kein Schöpfungsglaube im strengen biblischen
Sinne, aber es vermeidet viele der Probleme des orthodoxen Darwinismus.

„Der große theoretische Beitrag der Soziobiologie besteht darin, das
Konzept der Fitness auf das der ‚inklusiven Fitness' zu erweitern.
Tatsächlich kann ein Tier seine Gene direkt durch seine eigene
Fortpflanzung oder indirekt durch die Fortpflanzung von Verwandten, mit
denen es spezifische Genanteile teilt, duplizieren. Daher können Tiere
erwartungsgemäß kooperativ handeln und so die Fitness des anderen
verbessern, insofern sie genetisch verwandt sind. Dies wird als
Verwandtenselektion bezeichnet. Kurz gesagt, Tiere sind nepotistisch,
d.h. sie bevorzugen Verwandte gegenüber Nicht-Verwandten und nahe
Verwandte gegenüber entfernten Verwandten. Dies kann bewusst passieren,
wie bei Menschen, oder häufiger unbewusst.`` \footnote{Van Den Berghe,
  ebenda, S. 96.} - Pierre van den Berghe

\section{GENETISCH BEEINFLUSSTE
MOTIVATIONSFATOREN}\label{genetisch-beeinflusste-motivationsfatoren}

Die biologische Perspektive auf menschliches Verhalten wurde durch die
Einführung des Konzepts der „inklusiven Fitness`` im Jahr 1963 von W. D.
Hamilton in „Die Evolution des altruistischen Verhaltens`` erweitert.
Hamilton erkannte, dass Menschen zwar grundsätzlich auf
eigenorientiertes Verhalten ausgerichtet sind, sie aber auch
gelegentlich altruistische Handlungen oder Selbstopferungen unternehmen,
die keinen offensichtlichen Vorteil im Sinne des Lebens des Individuums
bieten. Hamilton versuchte diese scheinbaren Widersprüche zu versöhnen,
indem er postulierte, dass die grundlegende Maximierungseinheit nicht
der einzelne Organismus, sondern das Gen ist. Individuen einer jeden Art
streben danach, nicht nur ihr eigenes persönliches Wohlbefinden zu
maximieren, sondern auch das, was Hamilton ihre „inklusive Fitness``
nannte. Dabei argumentierte er, dass „inklusive Fitness`` nicht nur das
persönliche Überleben im darwinistischen Sinne beinhaltet, sondern auch
die verbesserte Fortpflanzung und das Überleben naher Verwandter, die
die gleichen Gene teilen.\footnote{See W. D. Hamilton, \emph{The
  Evolution of Altruistic Behavior}, American Naturalist, 1963, S.
  346-54.} Hamiltons These der „inklusiven Fitness`` hilft, viele
ansonsten merkwürdige Eigenschaften menschlicher Gesellschaften zu
erhellen, einschließlich Aspekte der Politik in Nationalstaaten.

\subsection{Altruismus: Fehlbezeichnung oder fossile
Verwandtenauswahl?}\label{altruismus-fehlbezeichnung-oder-fossile-verwandtenauswahl}

Gemäß van den Berghe ist „Altruismus`` also meist auf Verwandte
ausgerichtet, insbesondere auf nahe Verwandte und ist in der Tat ein
ungenauer Begriff. Es stellt den ultimativen genetischen Egoismus dar.
Es ist lediglich der blinde Ausdruck der Maximierung der inklusiven
Fitness.`` \footnote{Van Den Berghe, ebenda, S. 96.} Allerdings bedeutet
dies nicht, dass es keinen Altruismus außerhalb der engen genetischen
Beziehung gibt, die Hamilton und van den Berghe erwähnen. Die
Unsicherheiten, die durch die Tatsache verursacht werden, dass Menschen
sich sexuell reproduzieren und nicht durch asexuelle Klonung,
garantieren nahezu, dass eine Neigung zur „Maximierung der inklusiven
Fitness`` eine Menge „Altruismus`` fördern würde, der den Allelen
außerhalb des „egoistischen Gens`` nützt. Zunächst einmal besteht immer
die Möglichkeit, dass einige Personen, die Hilfe leisten, dies in der
irrigen Annahme tun, dass sie nahen Verwandten helfen. Der Vater, der
eine Opferhandlung für seinen Nachwuchs auf sich nimmt, ist vielleicht
in Wirklichkeit nicht der Vater, glaubt jedoch, dass er es
ist.\footnote{Die gleiche Logik gilt natürlich auch für den Sohn oder
  die Tochter, die Opfer bringen für diejenigen, die sie für ihre
  Geschwister halten, obwohl sie es nicht sind.} Dies ist nicht nur ein
Thema für Seifenopern, es veranschaulicht auch das ursprüngliche Rätsel,
dass das Überleben der „egoistischen Gene`` wahrscheinlich erleichtert
wird, wenn jeder scheinbare Vater sich so verhält, als sei er
tatsächlich der Vater, auch wenn die Möglichkeit besteht, dass er es
nicht ist.

Allerdings, wie Hirshleifer betont, wenn sie im richtigen Licht
betrachtet werden, handelt es sich bei vielen der Paradoxien des
„Altruismus`` um semantische Verwirrungen, die Menschen oft verwirren
oder dazu verleiten, den Kontext des Wettbewerbs aus den Augen zu
verlieren, in dem „Hilfe`` einen Überlebensvorteil bieten könnte: „‚Wenn
eine altruistische Strategieentscheidung in Konkurrenz mit
Nicht-Altruismus überleben soll, muss der Altruismus mehr zur
Selbsterhaltung beitragen als der Nicht-Altruismus, und daher kann er
eigentlich kein Altruismus sein.' All diese Verwirrungen könnten
vermieden werden, wenn wir den Begriff ‚Altruismus' fallen lassen und
stattdessen fragen: Was sind die Bestimmungsfaktoren für das völlig
objektive Phänomen, das man Hilfe nennen könnte?{}`` \footnote{Hirshleifer,
  ebenda, S. 179.}

Diese Frage ist vielleicht am interessantesten im Falle der
„Verwandtschaftshilfe``. Hamiltons Grundformel der inklusiven Fitness
beinhaltete eine biologische Kosten-Nutzen-Analyse, in der ein
Individuum oder „das Gen, das helfendes Verhalten kontrolliert``, das
Überleben einer identischen Kopie von sich selbst genauso hoch bewertet
wie sein eigenes Überleben. Daher ist die Bereitschaft, Hilfe zu
leisten, geschweige denn Opfer zu bringen, abhängig von der
Wahrscheinlichkeit, dass ein anderes Individuum ein identisches Gen hat.
„Insbesondere führt ein Gen zur Verwandtschaftshilfe und instruiert
einen Menschen (bei sonst gleichen Bedingungen), sein Leben zu geben,
wenn er dadurch zwei Geschwister, vier Halbgeschwister, acht Cousinen
und Cousins usw. retten kann.`` \footnote{Ebenda.}

\section{WAHRSCHEINLICHKEITSPROBLEME DER INKLUSIVEN
FITNESS}\label{wahrscheinlichkeitsprobleme-der-inklusiven-fitness}

Während dieses biologische Prinzip im Grundsatz klar erscheint, verbirgt
eine genauere Untersuchung eine Reihe von Schwierigkeiten. Zum Beispiel
bedeutet die Tatsache, dass Geschwister oder Kinder vielleicht zu 50
Prozent mit einem identischen Gen übereinstimmen, rein logisch gesehen,
nicht, dass dieses Gen tatsächlich in ihnen zum Ausdruck kommt. Jeder
Mensch trägt zwei Sätze jedes Genes, eines vom Vater, eines von der
Mutter. Aber das bedeutet natürlich, dass nur die Hälfte der von einem
Elternteil getragenen Gene zwangsläufig beim Nachkommen vorhanden sind.
Darüber hinaus besteht immer das Risiko einer Mutation bei der
Fortpflanzung, was, auch wenn es unwahrscheinlich sein mag, die
Gewissheit einer genetischen Kosten-Nutzen-Analyse verringert. Nimmt man
also die Metapher vom „Gen als Optimierer`` ernst, so ist der Fall des
Vaters, der nicht der Stammvater ist, nur das deutlichste Beispiel für
ein umfassenderes Problem. Wenn es tatsächlich das Überleben des
„egoistischen Gens`` ist, das durch das Opfern für nahe Verwandte
optimiert wird, dann kann jede Möglichkeit, die dazu führt, dass die
identische Kopie des „egoistischen Gens`` durch ein anderes Allel
ersetzt wird, als einer der komplizierten Tricks angesehen werden, die
Mutter Natur sich selbst spielt.

\subsection{Unsichere Folgen}\label{unsichere-folgen}

Der auf Verwandte ausgerichtete Altruismus geht daher mit Problemen
einher. Es besteht nicht nur für das „egoistische Gen`` das
Wahrscheinlichkeitsproblem, dass scheinbare Verwandte des Wirts
tatsächlich vielleicht keine identischen Kopien desselben teilen. Es
gibt auch die Schwierigkeit, unter unsicheren Bedingungen festzustellen,
ob eine gegebene altruistische Handlung in der Tat hauptsächlich
Verwandten zugutekommt oder anderen. (Eine altruistische Handlung, die
hauptsächlich anderen zugutekommt, könnte tatsächlich die inklusive
Fitness des egoistischen Gens schädigen, indem sie die Aussichten
verringert, dass es in nachfolgenden Populationen vertreten ist.)
Betrachten wir ein furchtbares Beispiel, das durch die Nachrichten
inspiriert wurde, während wir dies schreiben. Nehmen wir an, ein
Elternteil in Dunblane, Schottland, erfährt kurzfristig, dass ein
bewaffneter Wahnsinniger beabsichtigt, Amok an einer örtlichen Schule zu
laufen. Durch sofortiges Handeln könnte er oder sie den heldenhaften,
aber möglicherweise zum Scheitern verurteilten Versuch unternehmen, sich
dem Wahnsinnigen entgegenzustellen und so möglicherweise seine oder ihre
Kinder in der Schule zu retten.

Vielleicht aber auch nicht.

Sogar ein gnadenloser Wahnsinniger, der darauf aus ist, jedes Kind auf
dem Planeten zu töten, wäre in dem Schaden, den er anrichten könnte,
dahingehend eingeschränkt, dass ihm die Munition ausgeht oder er von
anderen überwältigt wird. Hätte sich das opferbereite Elternteil
entschieden, nicht einzugreifen, hätten seine Kinder mit hoher
Wahrscheinlichkeit ohnehin überlebt, wie die meisten anderen Kinder in
der Schule auch. Der ganze Schaden, den eine mutige Opferhandlung
verhindert hätte, wäre wahrscheinlich auf die Kinder anderer gefallen.
Daher hätte der betreffende Vater oder die Mutter tatsächlich seine oder
ihre „inklusive Fitness`` verringert, indem er oder sie ihr Leben vor
allem für die Kinder anderer riskiert hätte. Indem er alle seine Kinder
eines ihrer Elternteile beraubt, hätte er diese Kinder wahrscheinlich in
einer schlechteren Position im darwinistischen Kampf zurückgelassen.

Obwohl dies zugegebenermaßen ein übertriebenes Beispiel ist, ist es doch
realistisch. Es spiegelt die Tatsache wider, dass es unzählige
Situationen im Leben gibt, in denen große oder kleine Hilfsaktionen
positive Effekte haben. In vielen Fällen können die direkten
Begünstigten solcher Aktionen nicht leicht auf nahe Verwandte isoliert
werden. Ironischerweise könnte dies, wie wir unten betrachten, Teil des
Überlebensvorteils sein, der es denen mit weniger diskriminierenden
Hilfsgenen ermöglichte, all die Jahrtausende von Unannehmlichkeiten bis
heute zu überleben.

\subsection{Altruismus und genetische
Trägheit}\label{altruismus-und-genetische-truxe4gheit}

Wenn wir, wie wir glauben, der These des „egoistischen Gens`` als genaue
Näherung dessen, was menschliches Handeln motiviert, zustimmen, wäre es
zu einfach anzunehmen, dass das helfende oder opferbereite Verhalten,
das sie hervorbringt, nur im engen und ausschließlichen Interesse der
tatsächlichen Verwandten handeln könnte. Unvollständiges Wissen macht
die Unterscheidung von Verwandten unter bestimmten Umständen zu einer
unsicheren Kunst. Und selbst wenn die Verwandten bekannt wären, könnte
die tatsächliche Darstellung eines gegebenen „egoistischen Gens`` in der
Verwandtenpopulation nicht mehr als eine Frage der Wahrscheinlichkeiten
festgestellt werden. Bis vor kurzem war es unmöglich, tatsächliche
genetische Marker unter Individuen zu unterscheiden. Und wir sind noch
ein gutes Stück davon entfernt, praktisch unterscheiden zu können,
welche nahen Verwandten tatsächlich welches „egoistische Gen``
ausdrücken, das sein Überleben optimiert. Darüber hinaus besteht die
größere Schwierigkeit, Vorteile auf Verwandte und nicht auf andere zu
begrenzen.

Darüber hinaus ist es aus Erfahrung offensichtlich, dass Menschen
manchmal ihre „Fürsorgeinstinkte`` zum Wohle von Nicht-Verwandten
einsetzen, wenn geeignete Verwandte nicht zur Verfügung stehen. Das
klarste Beispiel dafür ist das Verhalten von Eltern gegenüber
adoptierten Kindern oder sogar das Verhalten bestimmter Personen, in der
Regel Kinderlose, gegenüber ihren Haustieren. Es ist nicht unerhört,
dass solche Individuen ernsthafte Verletzungen und sogar den Tod
riskieren, um Katzen zu retten, die auf einem Baum gefangen sind.
Sicherlich sterben jedes Jahr eine nicht unerhebliche Anzahl von
Menschen bei Haushaltsunfällen, die in gewisser Weise von Haustieren
verursacht werden, die sich in Gefahr befinden. Was für Haustiere gilt,
gilt umso mehr für adoptierte Kinder. Es ist sicherlich nicht
übertrieben zu sagen, dass Eltern von adoptierten Kindern sie oft so
behandeln, als ob sie Verwandte wären, was dem Konzept der
„Verwandtenselektion`` eine weitere Bedeutung verleiht.

Solche Fälle diskreditieren die Theorie des „egoistischen Gens`` nicht
so sehr, wie manche Kritiker es sich wünschen würden. Im Gegenteil. Wir
sehen Beispiele von Menschen, die sich verhalten, als ob sie sich für
nahe Verwandte opfern würden, um ihre eigene inklusive Fitness
voranzutreiben, als Fälle von „genetischer Trägheit``. Mit anderen
Worten, sie spiegeln die Tatsache wider, die Howard Margolis in
\emph{Selfishness, Altruism and Rationality} bemerkt hat, nämlich dass
sich die „menschliche Gesellschaft schneller änderte`` als die
menschliche Genetik. Menschen verhalten sich daher weiterhin „weitgehend
als ob sie in einer kleinen Jäger-und-Sammler-Gruppe leben würden.``
\footnote{Margolis, ebenda, S. 32.} Ein entscheidendes Merkmal einer
solchen Gruppe war, wie Van Den Berghe es formulierte, dass

\begin{quote}
sie kleine, inzestuöse Populationen von einigen hundert Individuen
waren. ... Die Mitglieder des Stammes, obwohl in kleinere
Verwandtschaftsgruppen unterteilt, betrachteten sich als ein einziges
Volk, isoliert gegenüber der Außenwelt und verkettet durch ein Netz aus
Verwandtschaft und Heirat, wodurch der Stamm tatsächlich zu einer
Superfamilie wurde. Eine hohe Rate an Inzest stellte sicher, dass die
meisten Ehepartner auch Verwandte waren.`` \footnote{Van Den Berghe,
  ebenda, S. 98.}
\end{quote}

Kurz gesagt waren ethnische Gruppen für die gesamte menschliche Existenz
vor dem Aufkommen der Landwirtschaft „inzestuöse Superfamilien``.
Angesichts dieser früheren Identität zwischen Familie und In-Group,
könnte es eine genetisch beeinflusste Tendenz geben, die In-Group als
Verwandtschaft zu behandeln. Es ist leicht vorstellbar, dass solch ein
Verhalten in der Vergangenheit einen Überlebenswert haben könnte, wenn
jedes Mitglied der „inzestuösen Superfamilie`` miteinander verwandt war.
Wie Margolis vorschlägt, ist es leicht vorstellbar, dass „für solch
kleine Gruppen von eng verwandten Jägern und Sammlern, inklusiver
Egoismus (abgesehen von jeder Aussicht auf Reziprozität oder Rache)
alleine bereits eine Maßnahme von Engagement für das Gruppeninteresse
unterstützen würde. Man könnte dann argumentieren, dass eine gewisse
Tendenz zur gruppeninteressierten Motivation als eine Art von fossilem
Verwandten-Altruismus fortbesteht.`` \footnote{Margolis, ebenda, S. 32.}
Mit anderen Worten: Da wir die genetische Ausstattung von Jägern und
Sammlern beibehalten haben, spiegelt unser Verhalten gegenüber internen
Gruppen die Art von „Altruismus`` wider, von der man erwarten würde,
dass sie den Überlebenserfolg von internen Gruppen, die aus
„Inzucht-Superfamilien`` bestehen, optimiert.

Vermutlich, so spekuliert Margolis, könnte diese Tendenz zum
gruppeninteressierten Verhalten, die aus dem „fossilen
Verwandten-Altruismus`` oder der genetischen Trägheit hervorgeht, zum
Überleben des Homo sapiens beigetragen haben, „während andere
menschenähnliche Arten ausstarben.`` \footnote{Ebenda.}

\subsection{Epigenese}\label{epigenese}

Wir betrachten das „Als ob`` Verhalten als ein hervorragendes Beispiel
für „Epigenese``, oder die Tendenz von genetisch bedingten
motivationalen Faktoren, den Menschen von Natur aus dazu zu verleiten,
bestimmte Entscheidungen über andere zu bevorzugen. Mit anderen Worten,
der menschliche Geist ist keine tabula rasa, also keine leere Tafel,
sondern eine Festplatte mit vorkonfigurierten Schaltkreisen, die
bestimmte Reaktionen leichter erlernbar und attraktiver machen als
andere. Somit besagt die Annahme, dass der Geist geneigt ist, in
Kategorien einer Außengruppe, die Feindseligkeit erregt, und einer
Innengruppe, zu der man eine große Zuneigung oder Loyalität empfindet,
die normalerweise für Verwandte reserviert ist, zu denken.\footnote{Shaw
  and Wong, ebenda, S. 68-74\textasciitilde{}}

Diese epigenetische Neigung, sich innerhalb einer Gruppe zu verhalten,
als würde sie aus nahen Verwandten bestehen, schafft eine Anfälligkeit
für Manipulationen, die häufig von Nationalisten ausgenutzt wurde, um
opferbereite Unterstützung für den Staat zu erzeugen. In diesem Sinne
ist es kein Zufall, dass nationalistische Propaganda überall in eine
Sprache der Verwandtschaft verpackt wird.

\begin{quote}
„Mit der Stimme ihrer Kanonen, die Alarm schlagen, fordert das schöne
Frankreich seine Kinder auf, sich zu erheben. Die Soldaten um uns herum
bewaffnen sich. Auf, auf, es ist unsere Mutter, die ruft.`` \footnote{Quoted
  by Shaw und Wong, ebenda, S. 91.} -- Choral der französischen Soldaten
\end{quote}

\subsection{Falsche Verwandtschaft}\label{falsche-verwandtschaft}

Betrachten Sie die starke Tendenz von Politikern überall, den Staat in
Begriffen zu beschreiben, die aus der Verwandtschaft entlehnt sind. Die
Nation ist „unser Vaterland`` oder „unser Mutterland``. Ihre Bürger sind
„wir``, „Familienmitglieder``, unsere „Brüder und
Schwestern``.\footnote{See Billig, ebenda, S. 71.} Die Tatsache, dass so
kulturell unterschiedliche Staaten wie Frankreich, China und Ägypten
solche Vergleiche verwenden, ist unserer Ansicht nach kein rhetorischer
Zufall, sondern ein Paradebeispiel für „Epigenese`` oder die Tendenz von
genetisch beeinflussten Motivationsfaktoren, Menschen dazu zu
veranlassen, bestimmte Entscheidungen zu bevorzugen.

Wie funktioniert diese Epigenese? Der Identifikationsmechanismus, der
verwendet wird, um emotionale Treue zur Nation zu fördern, nutzt
verschiedene Mittel, die in der primitiven Vergangenheit als Beweise für
Verwandtschaft gedient hätten, um „die individuellen Anliegen bezüglich
der inklusiven Fitness mit den Interessen des Staates zu
verbinden``.\footnote{Shaw and Wong, ebenda, S. 106.} Zum Beispiel
konzentrieren sich Shaw und Wong auf fünf Identifikationsmittel, die von
modernen Nationen verwendet werden, um ihre Bevölkerungen gegen
Außenstehende zu mobilisieren. Diese lauten:

\begin{quote}
1. Eine gemeinsame Sprache\\
2. ein gemeinsames Heimatland\\
3. ähnliche phänotypische Merkmale\\
4. ein gemeinsames religiöses Erbe und\\
5. der Glaube an eine gemeinsame Abstammung.\footnote{Ebenda.}
\end{quote}

Solche Merkmale hätten natürlich in der primitiven Vergangenheit die
ethnische Kerngruppe ausgezeichnet. Ein großer Teil der Anziehungskraft
des Nationalismus lässt sich auf die Art und Weise zurückführen, wie
diese Identifikationsmechanismen angenommen und in die Sprache der
Verwandtschaft eingebettet wurden, wie am obigen Beispiel des
französischen Soldatenlieds illustriert. Solche
Mobilisierungsmechanismen, welche den Staat als das „Vaterland`` oder
„Mutterland`` bezeichnen, sind weltweit üblich, denn sie funktionieren.

\subsection{Genetische Buchführung}\label{genetische-buchfuxfchrung}

Der imaginäre Charakter dieser Verwandtschaftsbeziehungen in Bezug auf
den Staat zeigt sich darin, dass sie keine der Variabilitätsgrade
aufweisen, die die tatsächliche Verwandtschaft kennzeichnen. Selbst in
Großfamilien, wo jeder mit jedem verwandt ist, sind nicht alle im
gleichen Grad miteinander verwandt. Eltern und Geschwister sind die
engsten Verwandten, Großeltern und Cousins und Cousinen sind weniger eng
miteinander verwandt, und entfernte Cousins und Cousinen sind so weit
voneinander entfernt, dass die Wahrscheinlichkeit, dass sie ein
bestimmtes Gen gemeinsam haben, kaum größer ist als bei völlig Fremden.
Ehemänner und Ehefrauen sind im Allgemeinen nicht mehr so eng verwandt,
wie es in der Steinzeit der Fall war. In jedem Fall ist tatsächliche
Verwandtschaft in mathematischen Begriffen als der
„Verwandtschaftskoeffizient`` definierbar, den Hamilton als Maß für
genetische Überschneidungen berechnet hat.\footnote{Siehe Hamilton,
  ebenda, und W. D. Hamilton, \emph{The Genetical Evolution of Social
  Behavior, land!}, Theoretical Biology vol.7, S. 1-16, 17-52.}

Im Gegensatz dazu wird die nationale „Familie`` als vollkommen und
flexibel mit den territorialen Ausmaßen des Staates verstanden.
Nationalität breitet sich gleichmäßig, wie eine Flüssigkeit, in jede
Ritze innerhalb der streng definierten Grenzen aus. Benedict Anderson
schreibt: „In der modernen Auffassung ist die Staatssouveränität
vollständig, flach und gleichmäßig über jeden Quadratzentimeter eines
rechtlich abgegrenzten Territoriums wirksam.`` \footnote{Anderson,
  ebenda.} Und natürlich ist bei Opfern für den Staat der Koeffizient
der imaginären Verwandtschaft immer eins.

Diese Identifikation von inklusiver Fitness mit dem Nationalstaat ist
interessant, weil sie Aufschluss darüber geben könnte, inwieweit die
Menschen die Veränderungen des neuen Jahrtausends begrüßen oder ihnen
widerstehen. Wie wir zuvor untersucht haben, waren vor dem
Informationszeitalter sämtliche Arten von Gesellschaften territorial
orientiert. Sie bildeten sich entweder um das Heimatterritorium der
ethnischen Kerngruppe, oder sie nutzten, wie der Nationalstaat, die
gleichen Motive der Gruppensolidarität, um Kräfte zur Verteidigung eines
lokalen Territoriums gegen Außenstehende zu mobilisieren. In jedem Fall
war es der Fremde außerhalb des unmittelbaren Territoriums, der als
Feind gefürchtet wurde. Angesichts der Annahmen von Verwandtenselektion
in der urzeitlichen Vergangenheit ergibt dies Sinn. Als die Menschheit
in ihrer aktuellen genetischen Form auftauchte, waren die Mitglieder des
Stammes enge Verwandte. Sie waren Mitglieder einer ethnischen
Kerngruppe, der „Inzucht-Superfamilie``.

Darüber hinaus gab es angesichts der Erfordernisse der
Verwandtenselektion einen praktischen wirtschaftlichen Grund für das
Individuum, den Wohlstand und das Überleben der unmittelbaren Verwandten
mit dem seines Stammes oder seiner Superfamilie zu identifizieren. Ein
Mitglied eines Jäger-und-Sammler-Stammes war wirklich abhängig von dem
Erfolg des ganzen Stammes für seinen Wohlstand. Es gab kein unabhängiges
Eigentum und es gab keine Möglichkeit, dass ein Individuum oder eine
Familie realistisch gehofft haben könnte, abgetrennt vom Stamm zu
überleben und zu gedeihen. Dies verknüpfte das Eigeninteresse des
Individuums stark mit dem der Gruppe. In den Worten von Hirshleifer: „In
dem Maße, wie die Mitglieder einer Gruppe ein gemeinsames Schicksal oder
Ergebnis teilen, wird das Helfen des anderen zur Selbsthilfe.``
\footnote{Hirshleifer, ebenda, S. 188.}

\begin{quote}
„Offensichtlich betrachtet der primitive Mann - und die Lovedu können
als Vertreter von hunderten ähnlicher Völker betrachtet werden - eine
Gesellschaft, in der in jedem Augenblick jeder genau gleichgestellt ist,
als Norm.`` - Helmut Schoeck
\end{quote}

\subsection{Neue Umstände, alte Gene}\label{neue-umstuxe4nde-alte-gene}

Nun ermöglicht die Mikrotechnologie die Schaffung von sehr
unterschiedlichen Bedingungen im Vergleich zu denen, für die wir
genetisch durch die Gegebenheiten der Steinzeit veranlagt waren. Die
Informationstechnologie schafft ökonomische Ungleichheiten in einem
Ausmaß, das weit über das erreichbare Spektrum unserer Vorfahren in der
paradiesisch egalitären Steinzeit hinausgeht. Die
Informationstechnologie schafft auch supraterritoriale Vermögenswerte,
die dazu beitragen, die Verkörperung der In-Group, den Nationalstaat, zu
untergraben. Ironischerweise werden diese neuen Cyber-Vermögenswerte
wahrscheinlich gerade deshalb von höherem Wert sein, weil sie in
größerer Entfernung von zu Hause angesiedelt sind. Dies gilt umso mehr,
wenn es zu einer Gegenreaktion kommt, wie wir sie gegen die
wirtschaftliche Ungleichheit infolge der zunehmenden Verbreitung der
Informationstechnologie in den reichen Industrieländern erwarten. Genau
diese Tatsache würde dazu tendieren, Vermögenswerte, die auf großer
Distanz gehalten werden, wertvoller zu machen. Sie wären nicht nur
weniger dem Neid ausgesetzt, sondern würden wahrscheinlich auch
außerhalb der Reichweite der räuberischsten Gruppe liegen, mit der ein
Individuum zurechtkommen muss - seinem eigenen Nationalstaat.

\subsection{Die Ungleichgewichte von Natur und
Nationalismus}\label{die-ungleichgewichte-von-natur-und-nationalismus}

Es ist vielleicht ein Zeichen für die Bedeutung der Epigenese bei der
Bildung von Einstellungen, dass die Ironie der Identifikation mit der
eigenen Gruppe in Bezug auf den modernen Nationalstaat so wenig beachtet
wurde. Die Logik von Gewalt in der modernen Zeit neigt dazu, den Impuls
zu verwirren, der die Tendenz, die Fitness zunächst mit der eigenen
Gruppe zu identifizieren, hervorrief. Warum? Weil die Identifizierung
der „inklusiven Fitness`` des Individuums mit einer nationalen In-Group,
statt das Überleben und den Wohlstand der nahen Verwandten in einer
feindlichen Welt zu fördern, den Wert jeglichen Opfers, das das
Individuum hätte erbringen können, auf ein Niveau der
Bedeutungslosigkeit für seine Verwandten verwässerte. Der typische
moderne Nationalstaat war einfach zu groß, um einen statistisch
bedeutsamen „Verwandtschaftskoeffizienten`` zwischen dem Einzelnen und
anderen Bürgern der Nation zu ermöglichen, die Anspruch auf ihn erheben.
Nicht nur, dass der Anteil der nahen Verwandten innerhalb der In-Group
stark abgenommen hat, von fast völliger Einheit in der Steinzeit auf
eine bloße chemische Spur im zwanzigsten Jahrhundert; der
``Verwandtschaftskoeffizient`` zwischen dem einzelnen Bürger und dem
Rest der Nation dürfte in den meisten Fällen nicht wesentlich höher
gewesen sein als bei der gesamten Menschheit. Eine In- Group mit zig
Millionen oder sogar hunderten von Millionen (oder im Falle der
Chinesen, mehr als einer Milliarde Mitgliedern) wurde so gigantisch,
dass der Effekt der inklusiven Fitness jeglichen Opfers oder Nutzens auf
die Größenordnung eines Tropfens im Ozean verdünnte. Streng logisch
betrachtet kann der moderne Nationalist daher im Gegensatz zum Jäger und
Sammler der Steinzeit nicht erwarten, dass irgendeine Geste der
Aufopferung oder Hilfe für seine „In-Group`` die Überlebenschancen
seiner Familie in sinnvoller Weise verbessert.

Ungeachtet dessen, dass nationale Ökonomien die grundlegenden Einheiten
wurden, in denen das Wohlergehen in der modernen Ära gemessen wurde,
stellte die größte Hürde für den Erfolg des talentierten Individuums und
somit auch der seiner Verwandten, die im Namen der Nation, also der
In-Group selbst, auferlegten Lasten dar. Zumindest stimmte dies für
diejenigen, die hauptsächlich in reziproker anstatt in erzwungener
Geselligkeit tätig waren, um erneut auf van den Berghes Kategorien
menschlichen Verhaltens zu verweisen.\footnote{Van Den Berghe, ebenda,
  S. 97.}

Die Logik des Nationalstaats legt nahe, dass der endgültige Preis von
Staatsbürgerschaft Selbstaufopferung und Tod ist. Wie Jane Bethke
Elshtain bemerkte, indoktrinieren Nationalstaaten ihre Bürger mehr zum
Opfer als zur Aggression: „Der junge Mann zieht nicht so sehr in den
Krieg, um zu töten, sondern um zu sterben, um seinen eigenen Körper für
den des großen Körpers, den Volkskörper, zu opfern.`` \footnote{J. B.
  Elshtain, \emph{Sovereignty, Identity, Sacrifice}, in M. Ringrove und
  A. J. Lerner, eds., \emph{Reimaging the Nation} (Buckingham, England:
  Open University Press, 1993), herausgestellt von Billig, ebenda.} Der
Opfertrieb ist nicht weniger aktiv, wenn es um den Steuerzahler geht.
Steuern zu zahlen, wie auch Waffen zu tragen, ist eine Pflicht, und kein
Austausch, bei dem man Geld aufgibt, um ein Produkt oder eine
Dienstleistung von gleichem oder größerem Wert zu erhalten. Das wird in
der Alltagssprache anerkannt. Die Leute sprechen von einer
„Steuerlast``, während sie nicht von einer „Lebensmittel-Last`` beim
Einkaufen von Essen, oder der „Auto-Last`` beim Kauf eines Fahrzeugs,
oder einer „Urlaubs-Last`` beim Reisen sprechen, gerade weil
kommerzielle Käufe in der Regel faire Austauschgeschäfte sind. Ansonsten
würden die Käufer sie nicht tätigen.

In dieser Hinsicht zeigt der Nationalismus, wie die Epigenese die Logik
der darwinschen „Ökonomie der Natur`` umkehren kann. Der Nationalstaat
ermöglichte systematische Raubzüge auf territorialer Basis. Im Gegensatz
zu der Situation, der Jäger und Sammler in der Steinzeit
gegenüberstanden, war der Hauptparasit und Räuber des Individuums am
Ende des zwanzigsten Jahrhunderts wahrscheinlich nicht der
„Außenseiter``, der ausländische Feind, sondern eher die vermeintliche
Verkörperung der „In-Group``, der lokale Nationalstaat selbst. Daher ist
der Hauptvorteil, den das Aufkommen von Vermögenswerten bietet, die im
Informationszeitalter die Territorialität überschreiten, genau die
Tatsache, dass solche Vermögenswerte außerhalb der Reichweite der
systematischen Zwangsanwendung gesetzt werden können, die vom lokalen
Nationalstaat mobilisiert wird, in dessen Gebiet das potentielle
souveräne Individuum ansässig ist.

Wenn unsere Sichtweise korrekt ist, wird die Mikrotechnologie es
technisch möglich machen, dass Einzelpersonen weitgehend den Belastungen
der untergeordneten Bürgerschaft entkommen können. Sie werden in der
neuen „Virtuellen Stadt`` übernationale Souveräne über sich selbst sein,
keine Untertanen, die sich vertraglich oder privatrechtlich
verpflichten, und zwar auf eine Art und Weise, die eher an das
vormoderne Europa erinnert, wo sich Kaufleute durch Handelsverträge und
Chartas „vor willkürlichen Beschlagnahmungen ihres Eigentums`` und der
„Befreiung vom Grundherrschaftsrecht`` schützen konnten.\footnote{See
  Abu-Lughod, ebenda, S. 90.} In der Cyberkultur werden erfolgreiche
Personen von den Pflichten der Staatsbürgerschaft, die durch den Zufall
der Geburt entstehen, befreit. Sie werden nicht länger dazu neigen, sich
hauptsächlich als Briten oder Amerikaner zu verstehen. Sie werden
übernationale Bewohner der gesamten Welt sein, die zufällig in einer
oder mehreren ihrer Ortschaften leben.

\section{DIE CYBERWIRTSCHAFT UND UNSER GENETISCHES
ERBE}\label{die-cyberwirtschaft-und-unser-genetisches-erbe}

Der Knackpunkt ist jedoch, dass dieses technologische Wunder und das
damit einhergehende wirtschaftliche Wunder - das Entkommen aus der
Tyrannei des Ortes - von der Bereitschaft der Individuen abhängt, einen
Großteil ihres Vermögens und ihrer Zukunft Fremden anzuvertrauen. In
strenger genetischer Buchführung wären diese Fremden natürlich nicht
unbedingt weniger genetisch nahe als die meisten unserer „Mitbürger``,
auf die wir in den letzten Jahrhunderten angewiesen waren.

Die Frage ist, ob die perversen Ergebnisse der Gruppenharmonie im Falle
des Nationalstaates negative oder positive Indikatoren für die
Cyberwirtschaft sind. Werden die „Zurückgebliebenen``, die den Nutzen
einer erzwungenen Umverteilung verlieren könnten, den Tod des
Nationalstaates so behandeln, als wäre es ein Angriff auf ihre
Verwandten? Die ersten 25 Jahre des neuen Jahrtausends werden es uns
zeigen. Die emotionalen Reaktionen könnten komplex sein. Die Tatsache,
dass im zwanzigsten Jahrhundert 115 Millionen Menschen ihr Leben für
ihre Nationalstaaten opferten, ist ein drastischer Beweis für die Macht
der Epigenese.\footnote{Charles Tilly, \emph{Collective Violence in
  European Perspective}, in T. R\textasciitilde{} Gurr, ed.,
  \emph{Violence in America}, vol.2, Protest, Rebellion, Reform (Newbury
  Park, Calif.: Sage Publications, 1989), S. 93.} Es zeigt, dass viele
den Fortbestand ihrer Nationen für eine Frage von Leben und Tod hielten.
Die Frage ist, ob diese Haltung auch in einem neuen Zeitalter mit
anderen megapolitischen Erfordernissen Bestand haben wird.

Die Tatsache, dass genetisch beeinflusste Opferhandlungen im Namen des
Nationalstaates oft gegen den evolutionären Zweck der
Verwandtenselektion gewirkt haben, zeigt auch, dass Menschen
anpassungsfähig genug sind, um sich an viele Umstände anzupassen, für
die wir unter den Bedingungen der Steinzeit nicht genetisch programmiert
wurden. Wie Tudge bei der Beschreibung der „extremen Universalität`` des
Menschen ausführt: „Wir sind das tierische Äquivalent der
Turing-Maschine: das universelle Gerät, das für jede Aufgabe eingesetzt
werden kann.`` \footnote{Tudge, ebenda, S. 168.} Welche Tendenz wird in
der kommenden Übergangskrise zum Vorschein kommen? Wahrscheinlich beide.

Die Kommerzialisierung der Souveränität selbst hängt von der
Bereitschaft hunderttausender souveräner Individuen und vieler Millionen
anderer ab, ihre Vermögenswerte in die „Erste Bank im Nirgendwo`` zu
investieren, um sich vor direktem Zwang zu schützen. Diese Art von
Vertrauen hat offensichtlich kein Gegenstück in der Urzeit. Es gab
wenige Vermögenswerte in der Steinzeit. Die wenigen, die es gab, wurden
von einem Stamm, einer „inzuchtstarken Superfamilie``, kontrolliert, die
paranoid gegenüber Außenstehenden war. Und doch eröffnet sie, trotz der
evolutionären Neuheit der Cyberwirtschaft, den Menschen die Möglichkeit,
unsere ausgeprägteste genetische Vererbung auszudrücken - die
Intelligenz, die mit unseren überdimensionalen Gehirnen einhergeht.
Einige Mitglieder der Informationselite werden sicherlich klug genug
sein, eine gute Sache zu erkennen, wenn sie eine sehen.

Darüber hinaus sollte die Schaffung von Vermögenswerten, die weitgehend
vor Plünderungen geschützt sind, tatsächlich in einer praktischen Weise
dazu beitragen, die „inklusive Fitness`` von souveränen Individuen zu
erhöhen. Während die ökonomische Logik der Teilnahme an der
Cyberwirtschaft die Begründungen der Nationalstaaten auf den Kopf
stellt, ist sie überzeugend, insbesondere für Personen mit hohen
Fähigkeiten.

Um ihren Vorteil bei der Auswahl zwischen Rechtsgebieten zu optimieren,
müssen Individuen bereit sein, den Nationalstaat zu verlassen und ihren
persönlichen Schutz hauptsächlich durch marktgetriebene
Sicherheitsdienste in Gebieten zu gewährleisten, die weit entfernt von
ihrem Geburts- und Heimatort sein können. Dies impliziert einen
signifikanten Vorteil darin, mehrsprachig und kosmopolitisch in der
Kultur zu sein, anstatt nationalistischer Gesinnung. Und es lässt
weiterhin den Schluss zu, dass jeder, der ernsthaft das befreiende
Potenzial der Cyberwirtschaft für sich und seine Familie realisieren
will, anfangen sollte, sich in mehreren Rechtssystemen außerhalb
derjenigen willkommen zu fühlen, in der er während seiner
hauptsächlichen Karriere ansässig war. Für weitere Details sehen Sie
sich bitte unsere Diskussion über Strategien zur Erlangung von
Unabhängigkeit im Anhang an.

\subsection{Echte Gemeinsamkeiten}\label{echte-gemeinsamkeiten}

Ein neues extranationales Verständnis der Welt und eine neue Art der
Identifikation des eigenen Platzes darin könnten die Gewohnheiten
menschlicher Kulturen ändern, wenn nicht gar unsere angeborenen
Neigungen. Die neue extranationale Identität, die wir im neuen
Jahrtausend zu sehen erwarten, könnte es einfacher machen, sich an die
neue Welt anzupassen, als es wahrscheinlich erscheint. Im Gegensatz zur
Nationalität werden die neuen Identitäten nicht aus der systematischen
Zwangsmaßnahme hervorgehen, die die Nationalstaaten und das
Nationalstaatensystem im zwanzigsten Jahrhundert allgegenwärtig gemacht
hat. In der kommenden neuen Ära werden Gemeinschaften und Allianzen
nicht territorial begrenzt sein. Identifikation wird präziser auf echte
Gemeinsamkeiten, gemeinsame Interessen oder tatsächliche Verwandtschaft
abzielen, anstatt auf die falsche Gemeinsamkeit der Staatsbürgerschaft,
die in der konventionellen Politik so unablässig gefördert wird. Schutz
wird auf neue Weise organisiert werden, die keine Analogie zu einem
Vermessungsgerät hat, das territoriale Grenzen definiert. Vermögenswerte
werden immer mehr im Cyberspace statt an einem bestimmten Ort gehalten,
was neue Wettbewerbe zur Reduzierung der „Schutzkosten`` oder Steuern
erleichtern wird, die in den meisten territorialen Rechtssystemen
erhoben werden.

\begin{quote}
„Ehrgeizige Menschen verstehen daher, dass ein Nomadenleben der Preis
ist, um voranzukommen.`` \footnote{Christopher Lasch, \emph{The Revolt
  of the Elites and the Betrayal of Democracy} (New York: W W Norton \&
  Company, 1995), S. 5.} - Christopher Lasch
\end{quote}

\section{FLUCHT VOR DEM
NATIONALSTAAT}\label{flucht-vor-dem-nationalstaat}

Ungeachtet des festen Griffs, den der Nationalstaat als „In-Group`` auf
die moderne Vorstellungskraft gehabt hat, werden fähige Menschen, die
nicht den Nutzen einer Zugehörigkeit zu einer überaus teuren „imaginären
Gemeinschaft`` bereits jetzt schon bezweifeln, bald anfangen dies zu
tun. Tatsächlich haben die Anhänger des Nationalstaats bereits begonnen,
sich über die wachsende Entfremdung der kognitiven Eliten zu beklagen.
Der verstorbene Christopher Lasch hat in seinem Pamphlet \emph{The
Revolt of the Elites and the Betrayal of Democracy} diejenigen
angegriffen, „deren Lebensunterhalt nicht so sehr auf dem Besitz von
Eigentum, als vielmehr auf der Manipulation von Informationen beruht.``
\footnote{Ebenda, S. 34.} Lasch beklagt den extranationalen Charakter
der aufkommenden Informationswirtschaft. Er schreibt:

\begin{quote}
Die Märkte, in denen die neuen Eliten agieren, sind jetzt international
ausgerichtet. Ihr Erfolg hängt von Unternehmen ab, die über nationale
Grenzen hinweg operieren. Sie sind mehr mit dem reibungslosen
Funktionieren des gesamten Systems beschäftigt als mit irgendeinem
seiner Teile. Ihre Loyalitäten - falls der Begriff in diesem Kontext
nicht schon anachronistisch ist - sind eher international als regional,
national oder lokal. Sie haben mehr gemeinsam mit ihren Kollegen in
Brüssel oder Hongkong als mit der Masse der Amerikaner, die noch nicht
in das Netzwerk der globalen Kommunikation eingebunden sind.\footnote{Ebenda,
  S. 34-35.}
\end{quote}

Obwohl Lasch alles andere als ein leidenschaftsloser Beobachter war und
sein Porträt der Informationselite offensichtlich unvorteilhaft meinte,
ruht seine Verachtung für diejenigen, die sich von der Tyrannei des
Ortes befreien, auf der Wahrnehmung einiger der gleichen Entwicklungen,
die im Mittelpunkt dieses Buches stehen. Wenn wir Laschs Kritiken oder
die von Mickey Kaus (\emph{The End of Equality}), Michael Walzer
(\emph{Spheres of Justice}) oder Robert Reich (\emph{The Work of
Nations}) lesen, sehen wir Teile unserer Analyse bestätigt, auch wenn
sie unglücklicherweise oft von Autoren formuliert wurden, die den vielen
Folgen der Vertiefung der Märkte eher negativ gegenüberstehen, ganz zu
schweigen von der Entnationalisierung souveräner Individuen. Lasch
geißelt jene mit übernationalen Ambitionen, die „eine Mitgliedschaft in
der neuen Aristokratie der Gehirne begehren`` dafür, dass sie
„Beziehungen zum internationalen Markt für schnell zirkulierendes Geld,
Glamour, Mode und Popkultur pflegen.`` Er fährt fort:

\begin{quote}
Es ist fraglich, ob sie sich überhaupt als Amerikaner sehen.
Patriotismus rangiert sicherlich nicht sehr hoch in ihrer Hierarchie der
Tugenden. „Multikulturalismus`` hingegen passt perfekt zu ihnen und
beschwört das angenehme Bild eines globalen Basars herauf, auf dem
exotische Küchen, exotische Kleidungsstile, exotische Musik und
exotische Stammesbräuche wahllos genossen werden können, ohne Fragen zu
stellen und ohne Verpflichtungen einzugehen. Die neuen Eliten fühlen
sich nur auf Reisen zu Hause, auf dem Weg zu einer hochrangigen
Konferenz, zur großen Eröffnung einer neuen Filiale, zu einem
internationalen Filmfestival oder einem unentdeckten Urlaubsort. Ihre
Sicht auf die Welt ist im Grunde die eines Touristen - nicht gerade eine
Perspektive, die eine leidenschaftliche Hingabe zur Demokratie
fördert.\footnote{Ebenda, S. 6.}
\end{quote}

\subsection{Wirtschaftsnationalismus}\label{wirtschaftsnationalismus}

Hinter den Kritiken an den „Dauerreisenden``, die die virtuellen
Gemeinschaften des Informationszeitalters bilden, verbirgt sich die
Erkenntnis, dass für viele Mitglieder der Elite die Vorteile der
Vergänglichkeit bereits die Kosten übersteigen. Kritiker wie Lasch und
Walzer bestreiten nicht, dass eine klare Kosten-Nutzen-Analyse für
hochqualifizierte Personen die Staatsbürgerschaft für obsolet hält. Sie
tun auch nicht so, als ob die Zinseszins-Tabellen wirklich zeigen
würden, dass es eine bessere Rendite bringt, sein Geld in ein nationales
Sozialversicherungsprogramm zu pumpen, ganz zu schweigen von den
Einkommenssteuern, als private Investitionen. Im Gegenteil, sie
verstehen die Mathematik. Sie haben die Summen ihrer offensichtlichen
Schlussfolgerungen gesehen. Aber anstatt die subversive Logik der
ökonomischen Rationalität anzuerkennen, schrecken sie davor zurück und
werten es als „Verrat``, wenn die Informationselite die Tyrannei des
Ortes überwindet und „die Unaufgeklärten`` zurücklässt.\footnote{Ebenda,
  S. 21.}

Sozialdemokraten, so wie Pat Buchanan, sind wirtschaftliche
Nationalisten, die den Triumph der Märkte über die Politik ablehnen. Sie
verurteilen die „neue Aristokratie der Köpfe`` dafür, dass sie losgelöst
vom Ort sind und sich nicht leidenschaftlich für die Interessen der
Massen einsetzen. Obwohl sie die Entnationalisierung des Einzelnen als
solche nicht ausdrücklich anerkennen, protestieren sie gegen ihre frühen
Anzeichen und Manifestationen, was Walzer als „Imperialismus des
Marktes`` beschreibt, oder die Tendenz des Geldes, „Grenzen zu
überschreiten``, um Dinge zu kaufen, die, wie Lasch ausführt, „nicht zum
Verkauf stehen sollten``, wie etwa die Befreiung vom
Militärdienst.\footnote{Ebenda, S. 21.} Beachten Sie die reaktionäre
Anspielung auf die militärischen Forderungen des Nationalstaates als
heiligen Boden, den Geld und Märkte nicht betreten sollten.

Diese Kritik an der Informationselite deutet auf die Bedingungen einer
populären Reaktion gegen den Aufstieg der souveränen Individuen im
nächsten Jahrtausend hin. Mit der Verfügbarkeit neuer, marktgesteuerter
Schutzformen wird es für viele fähige Personen immer offensichtlicher,
dass die meisten vermeintlichen Vorteile der Staatsangehörigkeit
illusorisch sind. Dies wird nicht nur zu einer besseren Erfassung der
Opportunitätskosten der Staatsbürgerschaft führen, sondern auch neue
Wege angeblich „politische`` und sogar „ökonomische`` Fragen zu
formulieren. Zum ersten Mal wird „ein einzelner Unternehmer, der für
sich selbst handelt``, in der Lage sein, seine eigenen Schutzkosten zu
variieren, indem er sich zwischen verschiedenen Rechtsordnungen bewegt,
ohne darauf zu warten, dass dies durch „Gruppenentscheidungen und
Gruppenmaßnahmen`` geschieht, um Frederic C. Lanes Formulierung eines
alten Dilemmas zu zitieren.\footnote{Lane, \emph{The Economic Meaning of
  War}, in Venice and History. The Collected Papers of Frederic C. Lane,
  S. 385.}

Da der Preis, der für den Schutz gezahlt wird, „dem Prinzip der
Substitution`` unterliegt, wird dies die Mathematik des Zwangs
offenlegen und den Konflikt zwischen der neuen kosmopolitischen Elite
des Informationszeitalters und den „Informationsarmen``, dem Rest der
Bevölkerung, der größtenteils einsprachig ist und keine herausragenden
Problemlösungsfähigkeiten oder global vermarktbare Fähigkeiten besitzt,
verschärfen. Diese „Verlierer`` oder „Zurückgebliebenen``, wie Thomas L.
Friedman sie beschreibt, werden zweifellos weiterhin ihr Wohlergehen mit
dem politischen Leben bestehender Nationalstaaten
identifizieren.\footnote{See Thomas L. Friedman,
  \emph{Don\textquotesingle t Leave Globalization\textquotesingle s
  Losers Out of Mind}, International Herald Tribune, 18. Juli 1996, S.
  8.}

\section{DIE MEISTEN POLITISCHEN AGENDEN WERDEN REAKTIONÄR
SEIN}\label{die-meisten-politischen-agenden-werden-reaktionuxe4r-sein}

Die meisten von denen, die eine leidenschaftliche politische Agenda
verfolgen, ob sie nun nationalistisch, ökologisch oder sozialistisch
sind, werden sich dafür einsetzen, den wankenden Nationalstaat zu
verteidigen, während das 21. Jahrhundert beginnt. Im Laufe der Zeit wird
es immer offensichtlicher, dass das Überleben des Nationalstaats und das
nationalistische Empfinden Voraussetzungen sind, um einen Raum für
politischen Zwang zu bewahren. Wie Billig feststellt, ist Nationalismus
„die Bedingung für konventionelle (politische) Strategien, unabhängig
von der konkreten Politik.`` \footnote{Billig, ebenda, S. 99.} Daher
wird der nationalistische Inhalt in allen politischen Programmen in den
kommenden Jahren zunehmen wie der Bauch eines Vielfraßes. Umweltschützer
zum Beispiel werden sich weniger auf den Schutz von „Mutter Erde`` und
mehr auf den Schutz des „Heimatlandes`` konzentrieren. Aus Gründen, die
wir später erläutern, werden die Nation und die Staatsbürgerschaft
besonders jenen heilig sein, die Gleichheit besonders schätzen. Mehr als
sie vielleicht jetzt verstehen, werden sie mit Christopher Lasch
übereinstimmen, der Hannah Arendt folgte, als er verkündete: „Es ist die
Staatsbürgerschaft, die Gleichheit gewährt, nicht die Gleichheit, die
das Recht auf Staatsbürgerschaft schafft.`` \footnote{Lasch, ebenda, S.
  88.}

Die Privatisierung der Souveränität wird den Bonus auf Gleichheit des
industriellen Zeitalters abschwächen, indem sie die Bindung der Schöpfer
von Reichtum an Nation und Ort aufhebt. Die Staatsbürgerschaft wird
nicht länger als Mechanismus zur Durchsetzung der Einkommensumverteilung
auf Grundlage der Gleichheit der Stimme innerhalb eines begrenzten
Territoriums dienen. Die Folgen werden sein, dass die fortschrittliche
Geschichtsauffassung weitere Blessuren erleiden wird. Im Gegensatz zu
den Erwartungen der vermeintlich fortschrittlichen Menschen zu Beginn
des zwanzigsten Jahrhunderts wurde der freie Markt nicht durch die
Jahrzehnte zerstört, sondern blieb triumphierend. Die Marxisten
erwarteten, dass der Untergang des Kapitalismus, der nie eintrat, zur
Überwindung der Nationalstaaten und zur Entstehung eines universellen
Klassenbewusstseins der Arbeiter führen würde. Tatsächlich wird der
Staat untergehen, aber auf eine ganz andere Weise. Es wird nahezu das
Gegenteil ihrer Erwartung eintreten. Der Triumph des Kapitalismus wird
zur Entstehung eines neuen globalen oder extranationalen Bewusstseins
unter den Kapitalisten führen, von denen viele zu souveränen Individuen
werden. Weit davon entfernt, auf den Staat angewiesen zu sein, um die
Arbeiter zu disziplinieren, wie die Marxisten es sich vorgestellt
hatten, waren die fähigsten, wohlhabendsten Personen Nettoverlierer der
Handlungen des Nationalstaates. Es sind eindeutig sie, die am meisten zu
gewinnen haben, wenn sie den Nationalismus überwinden, während die
Märkte über den Zwang triumphieren.

Vielleicht nicht sofort, aber bald, sicherlich innerhalb einer
Generation, wird fast jeder aus der Informationselite seine
Einkommensaktivitäten in Niedrigsteuer- oder Nichtsteuerländern
ansiedeln. Während das Informationszeitalter den Globus verändert, wird
es eine unmissverständliche Lektion in Sachen Zinseszins erteilen.
Innerhalb von Jahren, geschweige denn Jahrzehnten, wird allgemein
anerkannt sein, dass fast jeder talentierte Mensch einen viel höheren
Nettowert anhäufen und ein besseres Leben genießen kann, indem er hoch
besteuerte Nationalstaaten verlässt. Wir haben bereits auf die enormen
Kosten hingewiesen, die die führenden Nationalstaaten aufbürden, aber da
dies der Kern eines wenig verstandenen Problems ist, lohnt es sich, die
Opportunitätskosten der Nationalität zu betonen.

\subsection{Opportunitätskosten}\label{opportunituxe4tskosten}

Weit davon entfernt, unter dem Verlust oder der Einschränkung von
Regierungsdienstleistungen zu leiden, die derzeit durch hohe Steuern
finanziert werden, wird die Informationselite auf beispiellose Weise
gedeihen. Einfach dadurch, dass sie der gegenwärtig übermäßigen
Steuerlast entkommen, werden sie einen enormen Spielraum zur
Verbesserung des materiellen Wohlbefindens ihrer Familien gewinnen. Wie
bereits angedeutet, verringert jeder jährlich gezahlte Steuerbetrag von
5.000 Dollar den lebenslangen Nettowert um 2,4 Millionen Dollar, während
man jährlich 10 Prozent aus den Investitionen verdienen könnte. Wenn man
jedoch 20 Prozent verdienen könnte, würde jede jährliche Steuerzahlung
von 5.000 Dollar einen über den Zeitraum von vierzig Jahren um 44
Millionen Dollar ärmer machen. Kumulativ würde einen die Zahlung von
5.000 Dollar pro Jahr daher mehr als eine Million Dollar pro Jahr
kosten. Mit dieser Rate würde eine jährliche Steuer von 250.000 Dollar
bald zu einem jährlichen Verlust von mehr als 50 Millionen Dollar
führen, oder 2,2 Milliarden Dollar in einem Leben. Und natürlich
bedeuten sporadisch höhere Einkünfte, und sei es auch nur für ein paar
Jahre, vor allem in jungen Jahren, einen noch erschreckenderen Verlust
an Vermögen durch räuberische Besteuerung.

Wir haben zu unserer eigenen Zufriedenheit gesehen, dass Renditen von
mehr als 20 Prozent möglich sind. Unsere Kollegen bei Lines Overseas
Management in Bermuda erzielten während der Jahre, in denen wir dieses
Buch schrieben, dreistellige Renditen, die im Durchschnitt 226 Prozent
pro Jahr betrugen. Ihre Erfahrung unterstreicht das, was die
Tabellenkalkulation nahelegt, nämlich dass für viele Spitzenverdiener
und Kapitalbesitzer die räuberische Besteuerung lebenslange
Kostenbelastungen verursacht, die einem großen Vermögen entspricht.

Eine Einzelperson mit hoher Verdienstmöglichkeit, die Steuern auf
Hongkong-Niveau zahlt, könnte am Ende tausendfach mehr Vermögen haben
als jemand mit der gleichen Vorsteuerleistung, der Steuern auf
nordamerikanischem oder europäischem Niveau zahlt. Wenn Sie Ihr Kapital
immer wieder dem Zugriff eines Hochsteuerlandes aussetzen, ist das so,
als würden Sie an einem Wettrennen teilnehmen und jemanden bei jedem
Schritt auf Sie schießen lassen. Wenn Sie das gleiche Rennen mit
angemessenem Schutz bestreiten könnten und ungeschoren davonkämen,
würden Sie natürlich viel weiter und schneller vorankommen.

Die souveränen Individuen der Zukunft werden die „vorübergehenden``
Neigungen, die Kritikern der Informationselite wie Christopher Lasch so
missfallen, zu ihrem Vorteil nutzen und sich die profitabelsten
Gerichtsbarkeiten zum Wohnen suchen. Obwohl dies der Logik des
Nationalismus zuwiderläuft, stimmt es mit einer überzeugenden
wirtschaftlichen Logik überein. Ein 10-prozentiger oder gar zehnfacher
Unterschied im Endergebnis motiviert gewinnmaximierende Individuen
häufig dazu, ihren Lebensstil und ihre Produktionstechniken sowie ihren
Wohnsitz zu ändern. Die Geschichte der westlichen Zivilisation ist ein
Zeugnis ständiger Veränderungen, bei denen Menschen und Wohlstand immer
wieder unter dem Antrieb sich wandelnder Megapolitik in neue Bereiche
voller Chancen gewandert sind. Ein tausendfacher Unterschied in den
Renditen würde dem stärksten Anreiz entsprechen, den es jemals gab, um
rationale Menschen in Bewegung zu versetzen. Oder anders ausgedrückt:
Die meisten Menschen, insbesondere diejenigen, die Thomas L. Friedman
als die „Verlierer und Zurückgebliebenen`` bezeichnet, würden, wenn sie
die Chance bekämen, jeden Nationalstaat liebend gerne für 50 Millionen
Dollar verlassen, ganz zu schweigen von den noch höheren Kosten, die die
Nationalstaaten den obersten 1 Prozent der Steuerzahler auferlegen. Der
Aufstieg der souveränen Individuen, die sich nach alternativen
Rechtssystemen umsehen, ist daher eine der sichersten Vorhersagen, die
man treffen kann.

\section{DIE VERMARKTUNG DER
SOUVERÄNITÄT}\label{die-vermarktung-der-souveruxe4nituxe4t}

Im Kosten-Nutzen-Verhältnis betrachtet, war die Staatsbürgerschaft
bereits ein schreckliches Geschäft, als das zwanzigste Jahrhundert zu
Ende ging. Dies wurde durch eine unfreiwillig komische parlamentarische
Forschungsnotiz mit dem Titel „Ist die Queen eine australische
Staatsbürgerin?{}`` hervorgehoben, die von Ian Ireland vom Australian
Parliamentary Research Service im August 1995 erstellt wurde.\footnote{Ian
  Ireland, \emph{Is the Queen an Australian Citizen?} Parliamentary
  Research Service, Australia, Nr.6, 28. August 1995.} Ireland
untersuchte das australische Staatsbürgerschaftsgesetz von 1948 und
überprüfte die vier Wege, auf denen man die australische
Staatsbürgerschaft erlangen kann. Diese sind ähnlich den Optionen für
die Staatsbürgerschaft in anderen führenden Nationalstaaten, nämlich:

\begin{itemize}
\tightlist
\item
  Staatsbürgerschaft durch Geburt
\item
  Staatsbürgerschaft durch Adoption
\item
  Staatsbürgerschaft durch Abstammung
\item
  Staatsbürgerschaft durch Gewährung
\end{itemize}

Das ist alles nichts Besonderes, abgesehen davon, dass es die
Aufmerksamkeit auf den Unterschied zwischen Souveränität und
Staatsbürgerschaft lenkt. Wie Ireland sagt: „Nach traditionellen
juristischen und politischen Konzepten ist der Monarch souverän und die
Menschen sind seine/ihre Untertanen. Untertanen sind durch Treue und
Unterwerfung an den Monarchen gebunden.`` In Kenntnisnahme der
offensichtlichen Tatsache, dass Königin Elisabeth II. souverän ist,
schlussfolgert er, dass „es ein Argument gibt, dass die Queen keine
australische Staatsbürgerin ist.`` \footnote{Ebenda, S. 2.}

Tatsächlich ist sie das nicht. Die Queen, möge sie lange leben, hat das
Glück, über den Status der Staatsbürgerschaft erhaben zu sein. Sie ist
souverän, die Souveränin über ihre Untertanen. Wie eine Handvoll anderer
Monarchen auf der Welt, ist die Königin von Geburt an souverän und hat
ihren Status als eine Angelegenheit von Bräuchen geerbt, die die
modernen Zeiten vorwegnehmen. Die Idee der Monarchie ist alt, sie geht
zurück auf die allerersten historischen Aufzeichnungen des menschlichen
Lebens. Die Länder, die ihre Monarchie beibehalten haben, verdanken ihre
Verfassung ihrer alten Geschichte, aber sie trägt immer noch dazu bei,
die Form ihrer Gesellschaft zu bestimmen, und zwar in Bezug auf das
Klassenprestige, wenn auch nicht in Bezug auf die politische Macht.
Postmoderne Individuen, die ohne den Vorsprung der Königin beginnen,
werden gezwungen sein, neue rechtliche Begründungen zu erfinden, auf
denen sie die faktische Souveränität gründen können, die ihnen die
Informationstechnologie an die Hand gibt.

Souveräne Individuen müssen sich auch mit den zersetzenden Folgen des
Neids auseinandersetzen - ein Problem, das Monarchen manchmal aufhält,
das aber intensiver bei Personen spürbar sein wird, die nicht
traditionell verehrt werden, sondern ihre eigene Souveränität
erschaffen. Wie Helmut Schoeck in seiner umfassenden Untersuchung,
\emph{Envy}, schrieb: „Dort, wo es nur einen König gibt, einen
Präsidenten der Vereinigten Staaten - mit anderen Worten, nur ein
Mitglied eines bestimmten Status - kann er mit relativer Straflosigkeit
ein Leben führen, das selbst in einem viel kleineren Maßstab Empörung in
der gleichen Gesellschaft hervorrufen würde, wenn es von erfolgreichen
Mitgliedern größerer beruflicher oder sozialer Gruppen übernommen
würde.`` \footnote{Schoeck, ebenda, S. 265.} Monarchen, als
Verkörperungen der Nation, genießen eine gewisse Immunität gegen Neid,
die nicht auf souveräne Individuen übertragen wird.

Die „Verlierer und Zurückgebliebenen`` in der Informationsgesellschaft
werden den Erfolg der Gewinner sicherlich mit Neid und Abneigung
betrachten, insbesondere weil die Vertiefung der Märkte andeutet, dass
dies zunehmend eine „Der Gewinner bekommt alles``-Welt sein wird. Immer
mehr basieren Belohnungen bereits auf relativen statt absoluten
Leistungen, wie es bei der industriellen Produktion der Fall war. Ein
Fabrikarbeiter wurde entweder auf Basis von Anwesenheitsstunden,
gemessen an der Stechuhr, bezahlt, oder nach einem Kriterium der
Ausbringung, wie Stückzahl, zusammengebaute Einheiten oder etwas
Ähnliches.\footnote{Für eine kritische Betrachtung der Vergütung in
  Abhängigkeit von der relativen Leistung, siehe Robert H. Frank und
  Philip J. Cook, \emph{The Winner-Take-All Society}, S. 24f.}
Standardisierte Bezahlungen waren möglich, da die Ausbringung für jeden,
der dieselben Werkzeuge verwendete, ähnlich war. Aber die Schaffung von
konzeptionellem Reichtum variiert, ähnlich wie künstlerische Leistungen,
dramatisch zwischen Personen, die dieselben Werkzeuge verwenden. In
dieser Hinsicht wird die gesamte Wirtschaft zunehmend wie eine Oper, bei
der die höchsten Belohnungen an die gehen, die die besten Stimmen haben,
und diejenigen, die falsch singen, ziehen normalerweise keine großen
Belohnungen an. Da immer mehr Bereiche dem wirklich globalen Wettbewerb
geöffnet werden, ist der Ertrag für gewöhnliche Leistungen zum Rückgang
verurteilt. Mittlere Talente gibt es im Überfluss, einige stammen von
Personen, die ihre Zeit zu einem Bruchteil der in führenden
Industrieländern geltenden Tarife vermieten können. Die Verlierer werden
die Spieler der Regionalliga sein, deren Reflexe eine halbe Sekunde zu
langsam sind, um mit denen der Bundesliga mitzuhalten. Anstatt ein
Vermögen zu machen, indem sie Tore schießen, werden sie 25.000 Dollar
verdienen, ohne zusätzliches Einkommen durch Werbeverträge. Andere
werden ganz ausfallen.

\begin{quote}
„Sobald sich ein Land dem globalen Markt öffnet, werden diejenigen
seiner Bürger, die über die entsprechenden Fähigkeiten verfügen, zu den
Gewinnern, und diejenigen, die dies nicht können, werden zu Verlierern
oder Zurückgebliebenen. Normalerweise behauptet eine Partei ... sie
könne sich der Globalisierung widersetzen oder deren Schmerz lindern.
Das sind Pat Buchanan in Amerika, die Kommunisten in Russland und jetzt
die Islamische Wohlfahrtspartei hier in der Türkei. Was also in der
Türkei passiert, ist viel komplizierter als nur eine fundamentalistische
Machtübernahme. Das ist, was passiert, wenn die sich ausweitende
Globalisierung immer mehr Verlierer hervorbringt, wenn sich die
Demokratisierung ausweitet und allen ein Stimmrecht gibt, während
religiöse Parteien diese Zufälligkeit effektiv ausnutzen, um an die
Macht zu gelangen`` \footnote{Friedman, ebenda.} - Thomas L. Friedman
\end{quote}

Wer werden in der Informationsgesellschaft die Verlierer sein? Im
Allgemeinen werden die Steuerverbraucher die Verlierer sein. Sie sind es
meist, die ihr Vermögen nicht durch den Umzug in eine andere
Rechtsprechungsregion vermehren können. Ein großer Teil ihres Einkommens
ist in den Regeln einer nationalen politischen Gerichtsbarkeit verankert
und nicht durch Marktbewertungen bestimmt. Daher scheint die Beseitigung
oder starke Reduzierung der Steuern, die sich negativ auf ihr
Nettovermögen auswirken, sie nicht viel besser stellen - der Preis für
eine geringere Besteuerung ist ein verminderter Strom an
Umverteilungszahlungen. Sie werden Einkommen verlieren, weil sie nicht
mehr auf politischen Zwang angewiesen sein können, um die Taschen von
Personen zu leeren, die produktiver sind als sie selbst. Diejenigen ohne
Ersparnisse, die sich auf die Regierung verlassen, um ihre Renten- und
Gesundheitsvorteile zu bezahlen, werden aller Wahrscheinlichkeit nach
einen Rückgang ihres Lebensstandards erleben. Dieser Verlust an
Einkommen führt zu einer Abwertung dessen, was der Finanzschriftsteller
Scott Burns als „transzendentales`` oder politisches Kapital bezeichnet
hat.\footnote{James Dale Davidson, \emph{The Squeeze} (New York: Summit
  Books, 1980), S. 38-55.} Dieses „transzendente`` oder imaginäre
Kapital basiert nicht auf dem ökonomischen Eigentum von Vermögenswerten,
sondern auf dem de facto Anspruch auf den durch politische Regeln und
Verordnungen etablierten Einkommensstrom. Zum Beispiel könnte das
erwartete Einkommen aus staatlichen Umverteilungsprogrammen in eine
Anleihe umgewandelt werden, die mit aktuellen Zinssätzen kapitalisiert
wurde. Diese imaginäre Anleihe, die von der imaginären Gemeinschaft
finanziert wird, ist transzendentales Kapital. Es wird plötzlich durch
die „große Transformation``, die dazu bestimmt ist, den Griff der
politischen Behörden auf den zur Einlösung ihrer Versprechen benötigten
Geldfluss zu verringern, entwertet.

\begin{quote}
„An Grenzen und auf hoher See, dort, wo niemand ein andauerndes Monopol
auf die Anwendung von Gewalt hatte, vermieden Händler die Zahlung von
Abgaben, die so hoch waren, dass der Schutz auf andere Weise günstiger
zu erwerben war.`` \footnote{Lane, \emph{Economic Consequences of
  Organized Violence}, S. 404.} - Frederic C. Lane
\end{quote}

Man braucht sich nicht allzu sehr anstrengen, um sich vorzustellen, dass
die Informationselite wahrscheinlich die Chancen zur Befreiung und
persönlichen Souveränität nutzen wird, die die neue Cyberwirtschaft
bietet. Ebenso ist zu erwarten, dass die „Zurückgebliebenen`` zunehmend
nationalistischer und unangenehmer werden, wenn der Einfluss der
Informationstechnologie im neuen Jahrtausend wächst. Es ist schwierig,
genau vorherzusagen, wann die Reaktion hässlich wird. Wir vermuten, dass
sich die Schuldzuweisungen intensivieren werden, wenn westliche Nationen
anfangen, auf ähnliche Art unmissverständlich auseinanderzufallen, wie
die ehemalige Sowjetunion.

Ebenso wird dies jedes Mal, wenn ein Nationalstaat auseinanderbricht, zu
einer weiteren Machtverlagerung führen und die Autonomie souveräner
Individuen fördern. Wir erwarten eine erhebliche Vervielfachung
souveräner Einheiten, da aus den Trümmern der Nationen eine Vielzahl von
Enklaven und Gerichtsbarkeiten entstehen, die eher mit Stadtstaaten
vergleichbar sind. Zu diesen neuen Einrichtungen werden viele gehören,
die wettbewerbsfähige Preise für Schutzdienstleistungen anbieten und
niedrige oder gar keine Steuern auf Einkommen und Kapital erheben. Es
ist fast sicher, dass diese neuen Einheiten ihre Schutzdienste
attraktiver anbieten als die führenden OECD-Nationen. Betrachtet man
dies einfach als Marktsegmentierung, ist der Bereich des Marktes, der am
schlechtesten bedient wird, das höchsteffiziente, kostengünstige
Segment. Wer hohe Steuern in Austausch für ein kompliziertes Angebot an
staatlichen Ausgaben zahlen möchte, hat dazu reichlich Gelegenheit.
Daher ist die vorteilhafteste und profitabelste Strategie für eine neue
Minisouveränität fast sicher ein höchsteffizientes, kostengünstiges
Alternativangebot. Eine solche Minisouveränität könnte nur mit großer
Mühe ein breiteres Angebot an Dienstleistungen bereitstellen als die
überlebenden Nationalstaaten. Da nicht alle Nationalstaaten gleichzeitig
zusammenbrechen werden, wird das staatliche Angebot gerade in der
Anfangszeit der Übergangsphase wahrscheinlich gut versorgt sein.
Andererseits kann ein günstiges Regime von erträglicher Ordnung relativ
kostengünstig bereitgestellt werden. Wenn soziale Unruhen und
Kriminalität in den alten Industrieländern in dem Maße zunehmen, wie wir
es erwarten, wird ein erträglicher Zustand von Recht und Ordnung in
einer Rechtsordnung attraktiver sein als ein nationales
Raumfahrtprogramm, ein staatlich gefördertes Frauenmuseum oder
subventionierte Umschulungsprogramme für entlassene Führungskräfte.

\section{DIE ENTNATIONALISIERUNG DES
INDIVIDUUMS}\label{die-entnationalisierung-des-individuums}

Die Staatsbürgerschaft wird in dem Maße unattraktiver und weniger
haltbar, wie neue Institutionen entstehen, die die Wahlmöglichkeiten bei
den Dienstleistungen erleichtern, für die der Staat jetzt zuständig ist,
angefangen beim Schutz. Dadurch wird es für Individuen praktisch,
aufzuhören, sich mit nationalen Begrifflichkeiten zu identifizieren.
Dennoch wird der Demystifizierungsprozess der Staatsbürgerschaft ein
langsamer sein. Sie werden ständig einer Flut von banalen Botschaften
ausgesetzt, die in den Routinen des täglichen Lebens darauf ausgelegt
sind, Ihre Identifikation mit Ihrem heimischen Nationalstaat zu
verstärken. Diese Botschaften machen es für Sie sehr unwahrscheinlich,
Ihre Nationalität zu „vergessen``. Für viele Leute ist die Nationalität
ein entscheidendes Identitätsmerkmal. „Wir`` werden belehrt, die Welt in
Bezug auf die Nationalität zu sehen. Es ist „unser`` Land, „unsere``
Athleten treten bei den Olympischen Spielen an. Wenn sie gewinnen, ist
es „unsere`` Flagge, die in der Zeremonie weht. „Unsere`` Hymne wird vor
den Juroren und anderen Teilnehmer bei der Preisverleihung gespielt.
„Wir`` werden dazu gebracht zu glauben, dass es „unser`` Sieg ist,
obwohl es nie ganz klar ist, wie „wir`` teilgenommen haben, außer dass
wir uns als Staatsbürger innerhalb desselben Territoriums befinden.

\subsection{Vom Wir zum Ich}\label{vom-wir-zum-ich}

Mit der zunehmenden Bedeutung der Informationstechnologie wird eine
globale Perspektive erleichtert und Wege geschaffen, durch die souveräne
Einzelpersonen die latenten Möglichkeiten der Informationstechnologie
nutzen können, um der nationalistischen Steuerlast zu entkommen.
Innerhalb der nächsten Jahrzehnte wird beispielsweise das Narrowcasting
das Broadcasting als Methode zur Nachrichtenbeschaffung für Individuen
ablösen. Dies hat erhebliche Auswirkungen. Es bedeutet eine Veränderung
in den Vorstellungen von Millionen Menschen vom Wir zum Ich. Wenn die
Individuen selbst beginnen, als ihre eigenen Nachrichtenredakteure zu
fungieren und auszuwählen, welche Themen und Nachrichten von Interesse
für sie sind, ist es weniger wahrscheinlich, dass sie sich selbst zur
Aufopferung für den Nationalstaat indoktrinieren werden. Ein ähnlicher
Effekt wird durch die Privatisierung des Bildungswesens entstehen, die
ebenfalls durch Technologie erleichtert wird. Im Mittelalter war die
Bildung fest in der Hand der Kirche. In der Moderne befindet sich die
Bildung in der Hand des Staates. In den Worten von Eric Hobsbawm hat
„staatliche Bildung Menschen in Staatsbürger eines bestimmten Landes
verwandelt: ‚Bauern zu Franzosen'„.\footnote{Eric Hobsbawm, \emph{The
  Nation as Invented Tradition}, in Hutchinson und Smith,
  \emph{Nationalism}, S. 77.} Im Informationszeitalter wird Bildung
privatisiert und individualisiert sein. Sie wird nicht mehr mit dem
politischen Ballast behaftet sein, der die Bildung in der industriellen
Ära prägte. Der Nationalismus wird nicht ständig in jeden Winkel des
geistigen Lebens hineinmassiert.

Der Wechsel zum Internet und zum World Wide Web wird auch die Bedeutung
des Standortes im Handel verringern. Es werden individuelle Adressen
geschaffen, die nicht territorial gebunden sind. Satellitengestützte
digitale Telefondienste werden sich über standortgebundene
Festnetzsysteme mit einem gemeinsamen internationalen Vorwahlcode
hinausentwickeln. Der Einzelne wird seine eigene, einzigartige globale
Telefonadresse haben, ähnlich wie eine Internetadresse, die ihn
erreicht, wo immer er sich gerade befindet. Im Laufe der Zeit werden
nationale Postmonopole zusammenbrechen, was den privatisierten
Postversand durch weltweite Dienste ohne besondere Bindungen an einen
bestehenden Nationalstaat ermöglicht.

Diese und andere scheinbar kleinen Schritte werden dazu beitragen, den
gewöhnlichen Verbraucher sowie die kognitive Elite von einer
routinehaften Identifikation mit dem Nationalstaat zu befreien. Die
Entmystifizierung der Staatsbürgerschaft wird am dramatischsten durch
das Aufkommen praktischer Alternativen zur Abwicklung innerhalb von
durch Staaten monopolisierten, abgegrenzten Territorien beschleunigt.
Die Bausteine der Cyber-Ökonomie - Cyber-Geld, Cyber-Bankwesen und ein
unregulierter globaler Cyber-Markt für Wertpapiere - werden fast
zwangsläufig in großem Maßstab entstehen. Während dies passiert, wird
die Fähigkeit der gierigen Regierungen, das Vermögen der „Staatsbürger``
zu beschlagnahmen, schrumpfen.

Während die führenden Staaten zweifellos versuchen werden, durch
Zusammenarbeit ein Kartell zur Aufrechterhaltung hoher Steuern und
Fiat-Währungen zu schaffen und die Verschlüsselung einzuschränken, um zu
verhindern, dass ihre Bürger ihren Herrschaftsbereichen entkommen, wird
dieser Versuch letztendlich scheitern. Die produktivsten Menschen auf
dem Planeten werden ihren Weg zur wirtschaftlichen Freiheit finden. Es
ist unwahrscheinlich, dass es dem Staat gelingen wird, Menschen effektiv
einzusperren, wo sie physisch zur Kasse gebeten werden können. Die
Unwirksamkeit der Bemühungen, illegale Einwanderer abzuschrecken, zeigt
eindrucksvoll, dass Nationalstaaten nicht in der Lage sein werden, ihre
Grenzen zu versiegeln, um erfolgreiche Menschen an der Flucht zu
hindern. Die Reichen werden mindestens genauso einfallsreich sein, um
auszuwandern, wie es angehende Taxifahrer und Kellner sind, um
einzuwandern.

Zum ersten Mal seit der mittelalterlichen fragmentierten Souveränität
werden Grenzen nicht klar definiert sein. Wie wir bereits früher
erörterten, wird es kein bemerkbar abgegrenztes Territorium geben, in
dem viele zukünftige Finanztransaktionen stattfinden werden. Anstatt
eine Erbschaft von Verpflichtungen aufgrund eines Geburtsunfalls zu
akzeptieren, werden immer mehr souveräne Individuen diesen
Mehrdeutigkeitsaspekt ausnutzen, um vor ihren Steuerpflichten
davonzulaufen, sich über ihre Staatsbürgerschaft hinauszubewegen und zu
Kunden zu werden. Sie werden private Steuerverträge als Kunden
aushandeln, wie es derzeit in der Schweiz möglich ist, wie wir bereits
in Kapitel 8 analysiert haben. Ein typischer privater Steuervertrag, der
mit den französischsprachigen Schweizer Kantonen ausgehandelt wird,
erlaubt einer Einzelperson oder Familie, gegen eine jährliche
Steuerzahlung von 50.000 Schweizer Franken (derzeit etwa 45.000 USD) zu
residieren. Beachten Sie, dass dies keine Pauschalsteuer ist, sondern
eine festgelegte Steuersumme, unabhängig vom Einkommen. Wenn Ihr
Jahresgehalt 50.000 Schweizer Franken beträgt (45.000 USD), sollten Sie
den Abschluss eines solchen privaten Steuervertrags nicht in Betracht
ziehen, da Ihre Steuerlast bei 100 Prozent liegt. Bei einem Einkommen
von 500.000 Schweizer Franken liegt Ihre Rate bei 10 Prozent. Bei
5.000.000 Schweizer Franken beträgt die Rate nur 1 Prozent. Bei 50
Millionen Schweizer Franken liegt die Steuerquote bei nur einem Zehntel
Prozent. Während dies im Vergleich zu einem Grenzsteuersatz von 58
Prozent in New York City ein unglaublich gutes Geschäft zu sein scheint,
so ist dies lediglich ein Maß dafür, wie räuberisch und monopolistisch
die Preisgestaltung für staatliche Dienstleistungen während des
Industriezeitalters allgemein wurde.

Tatsächlich ist ein Betrag von 50.000 Schweizer Franken ein angemessener
jährlicher Beitrag für die notwendigen und nützlichen Dienstleistungen
der Regierung. Die Schweizer machen sicherlich einen guten Gewinn aus
jedem Millionär, der einzieht und ihnen jährlich 50.000 Schweizer
Franken für das Privileg zahlt. In vielen Fällen sind die zusätzlichen
Kosten für die Regierung, einen weiteren Millionär in ihrer
Gerichtsbarkeit leben zu haben, praktisch null. Daher wird ihr
jährlicher Gewinn aus dem Geschäft fast 50.000 Schweizer Franken
betragen. Jeder Dienst, der unterboten werden kann und dem
kostengünstigen Anbieter dennoch annähernd 100 Prozent Gewinn
ermöglicht, ist extrem monopolisiert und überteuert. Bemerkenswert ist
nicht, dass der Steuersatz in diesem speziellen Fall als Prozentsatz des
Einkommens sinken sollte, sondern dass es jemals „fair`` erschien, dass
unterschiedliche Personen im zwanzigsten Jahrhundert stark
unterschiedliche Beträge für die Dienstleistungen der Regierung zahlen
sollten. Dies ist besonders merkwürdig, da diejenigen, die die
Regierungsdienste am meisten nutzen, am wenigsten zahlen und diejenigen,
die sie am wenigsten nutzen, am meisten zahlen. Alle werden einen
Vorteil als Wohnsitz gegenüber den Vereinigten Staaten bieten, der jedem
hochverdienenden Amerikaner über ein Leben lang Zehn Millionen wert ist.
Sofern die US-Steuern nicht reformiert werden, um mit denen anderer
Gerichtsbarkeiten konkurrieren zu können, und nicht mehr auf der
Grundlage der Nationalität erhoben werden, werden abwägende Personen die
US-Staatsbürgerschaft ablegen, ungeachtet der von Clintons
Ausreisesteuer auferlegten Hindernisse, um Pässe anzunehmen, die weniger
schwere Verbindlichkeiten mit sich bringen.

Regierungen im Industriezeitalter haben ihre Dienstleistungen auf der
Grundlage des Erfolgs des Steuerzahlers und nicht in Bezug auf die
Kosten oder den Wert der erbrachten Dienstleistungen berechnet. Die
Bewegung hin zu kaufmännischer Preisgestaltung für
Regierungsdienstleistungen wird zu einem zufriedenstellenderen Schutz
für einen weitaus niedrigeren Preis führen als dem, der von
herkömmlichen Nationalstaaten auferlegt wird.

\subsection{Die Staatsbürgerschaft geht den Weg der
Ritterlichkeit}\label{die-staatsbuxfcrgerschaft-geht-den-weg-der-ritterlichkeit}

Kurz gesagt, die Staatsbürgerschaft ist bestimmt, den Weg des Rittertums
einzuschlagen. Da die Grundlage, auf der Schutz geboten wird, wieder
umorganisiert wird, werden auch die Rationalisierungen und motivierenden
Ideologien, die das System ergänzen, unweigerlich verändert. Vor einem
halben Jahrtausend, am Ende des Mittelalters, als die Bereitstellung von
Schutz im Gegenzug für persönliche Dienstleistungen generell aufhörte
eine lohnende Angelegenheit zu sein, reagierten die Menschen auf
vorhersehbare Weise. Sie verließen das Rittertum. Eidschwüre und
persönliche Treue hörten auf, so ernst genommen zu werden, wie sie es
die vorherigen fünf Jahrhunderte waren. Jetzt verspricht die
Informationstechnologie, die Staatsbürgerschaft ebenso subversiv zu
beeinflussen. Der Nationalstaat und die Ansprüche des Nationalismus
werden entmystifiziert, genau wie die Ansprüche der Monopolkirche vor
fünf Jahrhunderten entmystifiziert wurden. Während Reaktionäre versuchen
werden, Innovatoren zu verunglimpfen und nationalistische Gefühle wieder
aufleben zu lassen, bezweifeln wir, dass der megapolitisch
untergegangene Nationalstaat eine ausreichend starke Loyalitätsbindung
ausüben kann, um dem durch die Informationstechnologie ausgelösten
Wettbewerbsdruck standzuhalten. Die meisten denkenden Individuen in
einer Welt von bankrotten Regierungen werden es vorziehen, als Kunden
von Schutzdiensten gut behandelt zu werden, anstatt als Bürger von
Nationalstaaten ausgeplündert zu werden.

Die wohlhabenden OECD-Länder legen den Einzelpersonen, die innerhalb
ihrer Grenzen Geschäfte betreiben, hohe Steuer- und Regulierungslasten
auf. Diese Kosten waren vielleicht noch vertretbar, als die
OECD-Nationalstaaten die einzigen Rechtssysteme waren, in denen man auf
einem angemessenen Komfortniveau Geschäfte tätigen und residieren
konnte. Doch diese Zeiten sind vergangen. Die Prämie, die gezahlt wird,
um als Einwohner der reichsten Nationalstaaten besteuert und reguliert
zu werden, lohnt sich nicht mehr. Sie wird in Zukunft noch weniger
akzeptabel sein, wenn der Wettbewerb zwischen den Rechtssystemen
zunimmt. Diejenigen mit der Verdienstfähigkeit und dem Kapital, um die
Wettbewerbsherausforderungen des Informationszeitalters zu meistern,
können sich überall niederlassen und Geschäfte betreiben. Mit einer
Auswahl an Wohnsitzen werden nur die patriotischsten oder dümmsten
Menschen weiterhin in den Ländern mit hohen Steuern leben bleiben.

Aus diesem Grund ist zu erwarten, dass ein oder mehrere Nationalstaaten
verdeckte Aktionen durchführen werden, um die Attraktivität der
Fluchtgründe zu untergraben. Die Reise könnte effektiv durch biologische
Kriegsführung, wie beispielsweise durch den Ausbruch einer tödlichen
Pandemie, entmutigt werden. Dies könnte nicht nur das Verlangen zu
reisen entmutigen, es könnte auch Rechtssystemen auf der ganzen Welt
einen Vorwand geben, ihre Grenzen zu versiegeln und die Einwanderung zu
begrenzen.

\subsection{Der Nachteil der
Staatsangehörigkeitsbesteuerung}\label{der-nachteil-der-staatsangehuxf6rigkeitsbesteuerung}

Sofern es nicht zu einer verblüffenden und nahezu wundersamen Änderung
der Politik kommt, wird der erfolgreiche Investor oder Unternehmer im
Informationszeitalter eine lebenslange Strafe in Höhe von zig Millionen,
Hunderten von Millionen oder sogar Milliarden von Dollar zahlen, um in
den Ländern zu leben, deren fiskalische Politiken den höchsten
Lebensstandard im zwanzigsten Jahrhundert gewährleistet haben.

Ohne eine radikale Veränderung wird die Strafe für Amerikaner am
höchsten sein. Die Vereinigten Staaten sind eine von nur drei
Gerichtsbarkeiten auf dem Planeten, die Steuern auf Grundlage der
Staatsangehörigkeit und nicht des Wohnsitzes erheben. Die anderen beiden
sind die Philippinen, eine ehemalige US-Kolonie, und Eritrea, dessen
einstiger exilierter Führer während seiner langen Rebellion gegen die
äthiopische Herrschaft in den Bann des IRS (Internal Revenue Service,
die amerikanische Bundessteuerbehörde) fiel. Eritrea erhebt jetzt eine
Staatsangehörigkeitssteuer von 3 Prozent. Obwohl dies nur ein blasses
Abbild der US-Steuersätze ist, macht selbst diese Last die eritreische
Staatsbürgerschaft im Informationszeitalter zur Haftung. Das geltende
Recht macht die US-Staatsbürgerschaft sogar zu einer noch größeren
Haftung. Das IRS ist zu einem der führenden Exportartikel Amerikas
geworden. Mehr als jedes andere Land reicht die USA bis in die
entlegensten Ecken der Erde, um Einkommen von ihren Staatsangehörigen
abzuziehen.

Wenn ein 747-Jumbojet, voll mit Investoren aus jedem Rechtssystem auf
der Erde, in einem neuen, unabhängigen Land landen würde und jeder
Investor 1.000 Dollar in ein Start-up-Unternehmen in der neuen
Wirtschaft riskieren würde, würde der Amerikaner eine wesentlich höhere
Steuer auf etwaige Gewinne zahlen als jeder andere. Die besondere
Strafsteuer von ausländischen Investitionen, beispielhaft durch die
sogenannte PFIC-Besteuerung (Passive Foreign Investment Company), sowie
die US-Staatsangehörigkeitssteuer, können zu Steuerschulden von 200
Prozent oder mehr auf langfristige, außerhalb der Vereinigten Staaten
gehaltene Vermögenswerte führen. Ein erfolgreicher Amerikaner könnte
seine Gesamtsteuerlast im Laufe seines Lebens als Bürger eines von über
280 anderen Ländern auf der Welt reduzieren.

Die Vereinigten Staaten haben das weltweit räuberischste und
reichtumabsorbierendste Steuersystem. Amerikaner, egal ob sie in den USA
oder im Ausland leben, werden mehr als Vermögenswerte und weniger als
Kunden betrachtet als die Bürger irgendeines anderen Landes. Das
amerikanische Steuerregime ist daher veralteter und weniger kompatibel
mit dem Erfolg im Informationszeitalter als selbst die von den notorisch
hoch besteuerten Wohlfahrtsstaaten Skandinaviens. Bürger Dänemarks oder
Schwedens stoßen auf wenige rechtliche Hindernisse, wenn sie ihre
wachsende technologische Autonomie als Individuen verwirklichen wollen.
Falls sie ihren eigenen Steuersatz aushandeln wollen, können sie sich
frei entscheiden, in der Schweiz durch ein privates Abkommen Steuern zu
zahlen, oder nach Bermuda zu ziehen und überhaupt keine
Einkommenssteuern zu zahlen. Ein Schwede oder Däne, der hohe Steuern
zahlen möchte, weil er glaubt, dass der skandinavische Wohlfahrtsstaat
sein Geld wert ist, trifft tatsächlich eine Wahl. Er kann wählen, in
jeder anderen zivilisierten oder unzivilisierten Weltregion zu dem dort
geltenden Steuersatz besteuert zu werden. Um seinen Steuersatz zu
ändern, muss er nur umziehen. Die Technologie erleichtert eine solche
Wahl von Moment zu Moment. Doch diese Option wird Amerikanern verwehrt.
Ein US-Pass wird zunehmend zum großen Hindernis für die Verwirklichung
der Möglichkeiten individueller Autonomie, die durch die
Informationsrevolution ermöglicht wird. Ein Amerikaner zu sein, war
während des industriellen Zeitalters ein glücklicher Zufall. Selbst
während der Anfangsphase des Informationszeitalters ist es zu einer
millionenschweren Belastung geworden.

Um zu sehen, wie groß diese Haftung ist, betrachten wir diesen
Vergleich. Unter vernünftigen Annahmen würde ein Neuseeländer, der die
gleiche Bruttoleistung wie der Durchschnitt der oberen 1 Prozent der
amerikanischen Steuerzahler hat, so viel weniger an Steuern zahlen, dass
die Verzinsung seiner alleinigen Steuereinsparungen ihn reicher machen
würde, als der Amerikaner es jemals sein könnte. Am Ende seines Lebens
hätte der Neuseeländer 73 Millionen Dollar mehr, die er seinen Kindern
oder Enkeln vermachen könnte. Und Neuseeland ist nicht einmal als
Steueroase bekannt. Mehr als vierzig andere Rechtssysteme verhängen
niedrigere Einkommens- und Kapitalsteuern als Neuseeland. Wenn unser
Argument richtig ist, wird die Anzahl der Niedrigsteuergerichtsbarkeiten
eher steigen als fallen. Alle diese Gerichtsbarkeiten bieten einen
Wohnsitzvorteil gegenüber den Vereinigten Staaten, der im Laufe eines
Lebens Zehn- oder gar Hundertmillionen wert ist. Solange die US-Steuern
nicht reformiert werden, um wettbewerbsfähiger mit denen anderer
Gerichtsbarkeiten zu werden, und nicht länger auf der Grundlage der
Nationalität erhoben werden, werden abwägende Personen die
US-Staatsbürgerschaft aufgeben, trotz der Hindernisse, die Clintons
Ausstiegssteuer darstellt.

Die Wettbewerbsbedingungen des Informationszeitalters werden es
ermöglichen, fast überall hohe Einkommen zu erzielen. Faktisch werden
die Standortmonopole, die von Nationalstaaten zur Erhebung von extrem
hohen Steuern ausgenutzt wurden, durch Technologie gebrochen. Sie
brechen bereits zusammen. Wenn sie weiter erodieren, ist es fast
unvermeidlich, dass der Wettbewerbsdruck die unternehmerischsten und
fähigsten Menschen dazu bringt, aus Ländern mit zu hohen Steuern zu
fliehen. Wie der ehemalige Chefredakteur des Economist, Norman Macrae,
es ausdrückte, werden solche Länder „überwiegend von Dummköpfen bewohnt
sein.``

\begin{quote}
„Bis zum Jahr 2012 werden die voraussichtlichen Ausgaben für
Sozialleistungen und Zinsen auf die Staatsschulden sämtliche
Steuereinnahmen der Bundesregierung aufbrauchen. \ldots{} Es wird nicht
ein Cent für Bildung, Kinderprogramme, Autobahnen, nationale
Verteidigung oder irgendein anderes willkürliches Programm übrig
bleiben.`` - Bipartisan U.S. Commission On Entitlement And Tax Reform
\end{quote}

Die Flucht der Reichen aus entwickelten Wohlfahrtsstaaten wird zu einem
demografisch denkbar ungünstigen Zeitpunkt eintreten. Anfang des 21.
Jahrhundert werden große Teile der alternden Bevölkerung in Europa und
Nordamerika nicht mehr über ausreichende Ersparnisse verfügen, um die
medizinischen Kosten zu decken und ihren Lebensstil im Ruhestand zu
finanzieren. Beispielsweise haben ganze 65 Prozent der Amerikaner
überhaupt keine Ersparnisse für die Rente. Gar keine. Und diejenigen,
die sparen, tun dies bei weitem nicht genug. Der Durchschnittsamerikaner
wird 65 Jahre alt und steht vor erwarteten medizinischen Kosten von mehr
als 200.000 Dollar vor dem Tod, und das bei einem Vermögen von weniger
als 75.000 Dollar. Selbst die Minderheit mit privater Altersvorsorge
wird sich kaum wohlfühlen. Die durchschnittliche Rente ersetzt nur 20
Prozent des Einkommens vor der Pensionierung. Die meisten Vermögenswerte
des typischen Rentners sind kein reales Vermögen, sondern
„transzendentales Kapital``, der erwartete Wert von Transferzahlungen.
Die meisten Leute sind darauf konditioniert, sich auf diese Transfers zu
verlassen, um die Lücke in ihren privaten Ressourcen zu schließen. Der
Haken ist, dass diese wahrscheinlich nicht kommen werden. Umlagesystemen
werden Cashflow und Ressourcen fehlen, um sie in Anspruch zu nehmen.
Eine Studie von Neil Howe zeigte, dass auch wenn die Bruttoeinkommen in
den USA schneller steigen würden als in den letzten zwanzig Jahren, die
Durchschnittseinkommen nach Steuern in den USA bis 2040 um 59 Prozent
sinken müssten, um die Sozialversicherungen und staatlichen
medizinischen Programme auf dem aktuellen Niveau zu finanzieren.

Dies ist kein Problem, das durch das Drehen von ein paar Stellschrauben
gelöst werden kann. Der Wohlfahrtsstaat steht vor der Insolvenz. Seine
Finanzierungsschwierigkeiten sind in Europa noch akuter als in
Nordamerika. Italien ist vielleicht das schlimmste Beispiel, dicht
gefolgt von Schweden und den anderen nordischen Wohlfahrtsstaaten, die
Maßstäbe für den großzügigen Umgang mit
Einkommensunterstützungsprogrammen setzen. Die Financial Times schätzt,
dass, wenn „der heutige Wert der italienischen Staatspensionen
inbegriffen wird, die Staatsverschuldung des Landes auf über 200 Prozent
des BIP ansteigen würde.`` \footnote{John Plender, \emph{Retirement
  Isn\textquotesingle t Working}, \emph{Financial Times}, 17.-18. Juni
  1995.}

Verschuldung auf solchem Niveau ist praktisch hoffnungslos. Eine
umfassende Studie zur kommerziellen Verschuldung von Unternehmen an der
Torontoer Börse, die vor einigen Jahren durchgeführt wurde, zeigte, dass
nur wenige Staaten Schuldenquoten überleben, die ein Viertel so extrem
sind wie die, denen die führenden Wohlfahrtsstaaten heute
gegenüberstehen.\footnote{V H. Atrill, \emph{How All Economies Work}
  (Calgary, Canada: Dimensionless Science Publications, 1979), S. 27f.}
Einfach ausgedrückt, sie sind pleite. Wenn diese Realität widerwillig,
aber unvermeidlich ins Auge gefasst wird, werden buchstäblich Billionen
ungedeckter Anspruchspflichten abgeschrieben werden.

Das ist die Logik der Cyberwirtschaft. Ein mögliches Hindernis könnte
die Trägheit sein, der Nestbauinstinkt, der den Menschen zögern lässt,
seine Zelte abzubauen und umzuziehen. Wenn es weitere Hindernisse gibt,
könnten sie fest in der menschlichen Natur verankert sein. Die
wirtschaftliche Logik des Einsatzes von Mitteln im Cyberspace könnte der
biologischen Logik zuwiderlaufen, die im tief verwurzelten Misstrauen
gegenüber Außenstehenden zum Ausdruck kommt. Kinder in jeder Kultur
zeigen eine Abneigung gegenüber Fremden. Gegner der Kommerzialisierung
der Souveränität werden ihr Bestes tun, um Zweifel an der neuen globalen
Kultur des Informationszeitalters und dem damit verbundenen Untergang
des Nationalstaates zu schüren. Ein weiterer möglicher Haken, der sich
aus der Epigenese oder genetisch beeinflussten motivationalen Faktoren
ergibt, ist die Aussicht, dass die „Verlierer und Zurückgebliebenen``
auf Entwicklungen reagieren werden, die den Nationalstaat untergraben,
mit der Wut von Jägern und Sammlern, die ihre Familien schützen. In
einer Umgebung, in der desorientierte und entfremdete Individuen eine
erhöhte Kraft haben, Störungen zu verursachen und zu zerstören, könnte
sich eine Reaktion gegen die Informationswirtschaft als gewalttätig und
unangenehm erweisen.

\begin{quote}
„Historisch gesehen ist kollektive Gewalt regelmäßig aus den zentralen
politischen Prozessen westlicher Länder hervorgegangen. Menschen, die
versuchen, die Hebel der Macht zu ergreifen, zu halten oder neu
auszurichten, haben kontinuierlich kollektive Gewalt als Teil ihrer
Kämpfe eingesetzt. Die Unterdrückten haben im Namen der Gerechtigkeit
zugeschlagen, die Privilegierten im Auftrag der Ordnung und die
dazwischen aus Angst. Große Veränderungen in den Machtverhältnissen
haben in der Regel außergewöhnliche Momente der kollektiven Gewalt
hervorgebracht und häufig von ihnen abhängig gemacht.`` \footnote{Tilly,
  \emph{Collective Violence in European Perspective}, S. 62.} Charles
Tilly
\end{quote}

\section{GEWALT IN DER PERSPEKTIVE}\label{gewalt-in-der-perspektive}

Es gibt mindestens zwei konkurrierende Theorien darüber, was Gewalt
unter Bedingungen von Veränderungen auslöst. Der Historiker Charles
Tilly fasst eine Theorie zusammen: „Der Anreiz zu kollektiver Gewalt
ergibt sich weitgehend aus den Ängsten der Menschen, wenn etablierte
Institutionen zusammenbrechen. Wenn Elend oder Gefahr die Angst
verstärkt, so die Theorie, wird die Reaktion umso gewalttätiger.`` In
Tillys Sicht ist Gewalt jedoch weniger ein Produkt der Angst als viel
eher ein äußerst rationaler Versuch, die Behörden dazu zu bringen, ihrer
Verantwortung nachzukommen, der von einem „Gefühl der verweigerten
Gerechtigkeit`` motiviert wird. Nach Tillys Deutung neigen „große
strukturelle Veränderungen`` dazu, kollektive Gewalt von „politischer``
Natur anzuregen. „Anstatt einen scharfen Bruch mit dem ‚normalen'
politischen Leben darzustellen, begleiten gewaltsame
Auseinandersetzungen darüber hinaus oft organisierte, friedliche
Bemühungen der gleichen Menschen, ihre Ziele zu erreichen. Sie gehören
zur gleichen Welt wie nicht gewalttätige Auseinandersetzungen.``
\footnote{Ebenda, S. 68.}

Unabhängig davon, welche Gewalttheorie zutreffender ist, scheinen die
Aussichten auf sozialen Frieden während der großen Transformation gering
zu sein. Der Zusammenbruch des Nationalstaates zählt sicherlich als
hervorstechendstes Beispiel für den Fall einer „etablierten
Institution``. Daher ist zu erwarten, dass die Ängste in voller Blüte
stehen werden, ebenso die politische Motivation für Gewalt. Dies könnte
insbesondere in den führenden Wohlfahrtsstaaten der Fall sein, in denen
die Bevölkerung an relative Einkommensgleichheit gewöhnt ist. Angesichts
der Tatsache, dass die Bevölkerungen in den frühen Stadien der
Informationswirtschaft im Industriezeitalter aufgewachsen sind, als
politische Behörden tatsächlich die Möglichkeit hatten, Beschwerden mit
materiellen Vorteilen zu beantworten, ist es vernünftig zu erwarten,
dass die „Zurückgebliebenen`` weiterhin materielle Vorteile fordern
werden. Es wird wahrscheinlich eine langsame, schmerzhafte Einarbeitung
in die Realitäten der Cyberwirtschaft benötigen, bis die Bevölkerung der
OECD-Staaten von der Erwartungshaltung abgewöhnt wurde, eine
Einkommensumverteilung in großem Maßstab erzwingen zu können. In beiden
Fällen, ob die Gewalt nun aus „Angst`` entsteht oder als kalkuliertes
Bemühen, sich die Vorteile des systematischen Zwangs zunutze zu machen,
scheinen die Bedingungen Gewalt wahrscheinlich zu machen.

\subsection{Wahlkreise der Verlierer}\label{wahlkreise-der-verlierer}

Der unausweichliche Zusammenbruch der erzwungenen Einkommensumverteilung
wird diejenigen erschüttern, die davon ausgingen, am Empfangsende von
Billionen in Transferprogrammen zu stehen. Meistens werden dies die
„Verlierer oder Zurückgebliebenen`` sein, Personen ohne die Fähigkeiten,
in globalen Märkten zu konkurrieren. Wie die Rentner der ehemaligen
Sowjetunion, die das Kernstück von Sjuganows kommunistischer
Unterstützung bildeten, werden die enttäuschten Rentner der sterbenden
Wohlfahrtsstaaten eine reaktionäre Wählerschaft bilden, die darauf
bedacht ist, die Souveränität der Nationalstaaten vor der Privatisierung
zu schützen und somit den Staat seiner Lizenz zum Stehlen zu berauben.
Wenn sie erkennen, dass die von ihnen einst kontrollierten Regierungen
ihre Souveränität über Ressourcen und die Fähigkeit zur Durchsetzung
großangelegter Einkommensumverteilungen verlieren, werden sie genauso
hartnäckig dagegen ankämpfen wie französische Beamte gegen einfache
Rechenaufgaben.

Sie erinnern sich vielleicht an die gewalttätige Reaktion, die
Premierminister Alain Juppés recht bescheidenen Vorschlägen
entgegengebracht wurde, die „demografisch nicht nachhaltigen``
Rentenleistungen von Staatsbediensteten zu verringern und den Betrieb
des verstaatlichten Eisenbahnsystems zu rationalisieren. Symbolisch für
den Unsinn des „Etat Providence``, wie die Franzosen ihr
Sozialversicherungssystem nennen, ist die Regel, die es „Ingenieuren auf
den computergesteuerten, Hochgeschwindigkeits-TGV-Zügen erlaubt, genau
wie ihre Vorgänger, die in kohlebetriebenen Lokomotiven schufteten,
bereits mit 50 in den Ruhestand zu gehen``.\footnote{Dick Howard,
  \emph{French Toast: Can Politicians Anywhere Tangle with Entitlements
  Without Getting Burned?}, \emph{The New Democrat}, Juli/August 1996,
  S. 39f.} Eine wütende Reaktion auf Kürzungen von unhaltbaren
Leistungen ist in jedem OECD-Land durchaus möglich. Und selbst dort, wo
die Bevölkerung weniger wütend reagiert, können Sie davon ausgehen, dass
die wahrscheinlichen Verlierer alles in ihrer Macht Stehende tun werden,
um die Aushöhlung des staatlichen Zwanges zu verhindern.

Dies wird zu einigen überraschenden Wendungen führen. In den Vereinigten
Staaten beispielsweise war der heimische Nationalstolz historisch
gesehen oft mit einem deutlichen Hauch von Rassismus versehen. Es
handelt sich dabei um eine Tradition, die mit den „White Caps`` und dem
Ku Klux Klan im 19. Jahrhundert begann. Nichtsdestotrotz sind Schwarze
als Gruppe die großen Nutznießer von Einkommenstransfers,
Chancengleichheit und anderen Früchten der politischen Regulierung. Sie
sind auch im US-Militär überproportional vertreten. Daher ist es
wahrscheinlich, dass sie zusammen mit weißen Arbeitern zu den
glühendsten Anhängern des amerikanischen Nationalismus werden.

Politiker, die bereit sind, den Unsicherheiten jener entgegenzutreten,
deren relative Talente am untersten Teil von Ammons Rübe zu verorten
sind, werden in fast jedem Land lautstark in den Vordergrund treten. Von
Slobodan Milosevic in Serbien über Pat Buchanan in den USA, Winston
Peters in Neuseeland bis hin zu Necmettin Erbakan, der türkischen
fundamentalistischen islamischen Wohlfahrtspartei, werden Demagogen
gegen die Globalisierung der Märkte, Einwanderung und Investmentfreiheit
wettern.

Diejenigen, die sich selbst als „Opfer der globalen Wirtschaft`` sehen,
werden sich besonders gegen Reiche und Einwanderer wenden. Wie Andrew
Heal es formuliert, werden sie „die Einreise von Einwanderern verachten,
deren Hauptkriterium für die Einreise ihr Reichtum oder ihr Mangel daran
zu sein scheint, was sie, so die fadenscheinige Logik, zu
Sozialhilfeempfängern macht.`` \footnote{Andrew Heal, \emph{New
  Zealand\textquotesingle s First}, \emph{Metro}, Juli 1996, S.86.}

\subsection{Angst vor der Freiheit}\label{angst-vor-der-freiheit}

Die Aussicht auf das Verschwinden des Nationalstaates Anfang des neuen
Jahrtausends scheint so getaktet zu sein, dass sie das Leben von
beeinflussbaren Menschen am stärksten beeinträchtigt. Dies wird zu weit
verbreiteten Unannehmlichkeiten führen. Mehr als einige Beobachter haben
ein Muster der Reaktion erkannt, das bei denen üblich ist, die sich
durch die Aussicht auf eine grenzenlose Welt ausgegrenzt fühlen. Während
die größeren, umfassenderen nationalen Gruppierungen beginnen, sich
aufzulösen, und die mobileren „Informationseliten`` ihre Angelegenheiten
globalisieren, verlassen sich die „Verlierer und Zurückgebliebenen`` auf
die Zugehörigkeit zu einer ethnischen Untergruppe, einem Stamm, einer
Gang oder einer religiösen oder sprachlichen Minderheit. Teilweise ist
dies eine praktische und pragmatische Reaktion auf den Zusammenbruch von
Dienstleistungen, einschließlich Recht und Ordnung, die früher vom Staat
bereitgestellt wurden. Für Personen mit wenigen verwertbaren Ressourcen
erweist es sich oft als schwierig, Zugang zu marktbasierten Alternativen
für gescheiterte öffentliche Dienstleistungen zu erwerben.

Die Umwandlung dessen, was früher als öffentliches Gut galt, wie
Bildung, Bereitstellung von sauberem Wasser und Nachbarschaftspolizei,
in private Güter, ist offensichtlich für diejenigen mit ausreichenden
Ressourcen leichter zu handhaben, um hochwertige private Alternativen zu
erwerben. Für diejenigen, die Geld brauchen, ist die praktischste
Alternative oft, sich auf Verwandte zu verlassen oder einer
gegenseitigen Hilfsgruppe beizutreten, die entlang ethnischer Linien
organisiert ist, wie die alte chinesische Ethnie „Hokkien`` in
Südostasien, oder durch eine religiöse Gemeinde. In den Teilen der Welt,
in denen dynamische, missionarische Religionen aktiv sind, hängt ein
Teil der Beliebtheit ihrer Programme von der Tatsache ab, dass sie dazu
tendieren, auf prämoderne Mechanismen zur Bereitstellung sozialer
Wohlfahrt und öffentlicher Güter zurückzugreifen. So haben
beispielsweise moslemgeführte Bürgerwehrgruppen eine führende Rolle in
der Bekämpfung gewalttätiger Banden in Kapstadt, Südafrika,
gespielt.\footnote{Roger Matthews, \emph{South Africa Calls Up Troops
  for War on Crime}, \emph{Financial Times}, 31. August / 1. September
  1996, S. 1.} Aber so praktisch und pragmatisch solche ethnischen und
religiösen Organisationsformen der Hilfe auch sein können, in der
rückwärtsgewandten Reaktion auf das Schwinden des Staates spielt mehr
eine Rolle. Es scheint auch eine starke psychologische Komponente in der
Reaktion gegen die Globalisierung zu geben.

Die Diskussion ist nicht unähnlich der psychologischen Erklärung für die
Anziehungskraft des Faschismus, die Erich Fromm in seinem berühmten Werk
„Die Furcht vor der Freiheit``, das erstmals 1941 veröffentlicht wurde,
entwickelt hat.\footnote{Erich Fromm, \emph{Fear of Freedom} (London:
  Routledge \& Kegan Paul, 1942).} Laut Fromm hatte die durch den
Kapitalismus eingeführte soziale Mobilität die festen Identitäten des
traditionellen Dorflebens zerstört. Dem Sohn eines Bauern war nicht mehr
klar, dass er zwangsläufig auch ein Bauer werden würde, oder sogar, dass
er dazu bestimmt war, auf demselben kargen Boden wie sein Vater zu
arbeiten. Er hatte nun eine breite Auswahl an Berufen. Er könnte Lehrer,
Kaufmann oder Soldat werden, Medizin studieren oder zur See fahren.
Selbst als Bauer könnte er in die Vereinigten Staaten, Kanada oder
Argentinien auswandern und sein Leben weit entfernt von der Heimat
seiner Vorfahren führen. Diese Freiheit, die der Kapitalismus den
Menschen gab, „ihre eigene Identität zu schaffen``, erwies sich für
diejenigen, die nicht bereit waren, sie kreativ zu nutzen, als
beängstigend. Wie Billig sagte, sehnten sie sich „nach der Sicherheit
einer soliden Identität`` und fühlten sich „zu der Einfachheit
nationalistischer und faschistischer Propaganda hingezogen``.\footnote{Billig,
  ebenda, S. 137.} Ebenso schreibt Billig über die Dämmerung der
industriellen Ära: „Es gibt eine globale Psychologie, die von oben auf
die Nation trifft, die Loyalitäten mit einem freien Spiel der
Identitäten welken lässt. Und dann gibt es die heiße Psychologie der
Kaste oder des Stammes, die mit einer mächtig intoleranten Hingabe und
emotionalen Heftigkeit die weiche Unterseite des Staates trifft.``
\footnote{Ebenda, S. 135.}

Andrew Heal betrachtet dasselbe Phänomen aus einer anderen Perspektive.
Er sieht zwei große „globale politische und wirtschaftliche Trends. . .
Trend eins ist das Wachstum der globalen Wirtschaft. . . Der zweite ist
die Zunahme nationalistischer, ethnischer und regionaler Stimmungen, sei
es Maori, schottisch, walisisch oder von Anti-Immigranten-Fraktionen,
die sich, auch wenn ihre Regierungen sie zu neuen, grenzenlosen
Horizonten drängen, umso stärker in die entgegengesetzte Richtung
ziehen.`` \footnote{Andrew Heal, \emph{New Zealand\textquotesingle s
  First}, S. 85.} Wie auch immer man sie betrachtet, ob als große
„Trends`` oder „psychologische Themen``, es ist offensichtlich, dass
sich weltweit eine starke reaktionäre Stimmung zugunsten des
Nationalismus und gegen den Fall der Grenzen und die Vertiefung der
Märkte breit macht.

\section{MULTIKULTURALISMUS UND
OPFERKULT}\label{multikulturalismus-und-opferkult}

In seiner Dämmerung, mit einer schwindenden Fähigkeit, Versprechen von
etwas für nichts aus leeren Taschen einzulösen, fand der Wohlfahrtsstaat
es zweckmäßig, neue Mythen der Diskriminierung zu fördern. Viele
Kategorien offiziell „unterdrückter`` Menschen wurden herausgestellt,
besonders in Nordamerika. Die Personen in den Gruppen, die als „Opfer``
bezeichnet wurden, wurden darüber informiert, dass sie für die
Unzulänglichkeiten in ihrem eigenen Leben nicht verantwortlich sind.
Vielmehr wurde gesagt, dass die Schuld bei den „alten weißen Männern``
europäischer Abstammung liegt und der angeblich unterdrückenden
Machtstruktur, die zum Nachteil der ausgeschlossenen Gruppen manipuliert
wurde. Wenn man schwarz, weiblich, homosexuell, Latino,
französischsprachig, behindert usw. war, hatte man Anspruch auf
Wiedergutmachung für vergangene Unterdrückung und Diskriminierung.

Wenn man Laschs Argument folgt, war das Ziel der Vertiefung eines
Opferkults, Nationen zu untergraben und es der neuen, ungebundenen
Informationselite zu erleichtern, die Verpflichtungen und Pflichten der
Staatsbürgerschaft zu umgehen. Wir sind jedoch nicht ganz davon
überzeugt, dass die neue Elite, und insbesondere die meisten Mitglieder
der Massenmedien, schlau genug sind, um zu einer solchen Haltung zu
gelangen. Es wäre fast beruhigend, das Gefühl zu haben, dass sie es
wären. Wir sehen die Zunahme der Opferrolle hauptsächlich als Versuch,
sozialen Frieden nicht nur durch die Erweiterung der Mitgliedschaft in
der Meritokratie, wie Lasch argumentiert, sondern auch durch die
Neugestaltung der Rechtfertigungen für die Umverteilung von Einkommen zu
kaufen. Der neue Trend des Opferkults entstand in seiner überspitztesten
Form in Nordamerika, weil die Informationstechnologie dort tiefer
vordrang. Wir vermuten jedoch, dass neue Mythen der Diskriminierung in
allen industriellen Gesellschaften in ihrem verfallenden Zustand
ziemlich verbreitet sein werden. Die multiethnischen Wohlfahrtsstaaten
in Nordamerika waren einfach anfälliger für die Versuchung, die Kosten
der Einkommensumverteilung auf den Privatsektor zu verlagern. Sie
konnten dies tun und gleichzeitig ein Gefühl von Unzufriedenheit und
Anspruch schüren, indem sie die Struktur der Gesellschaft als Ganzes und
weiße Männer im Allgemeinen für die wirtschaftlichen Unzulänglichkeiten
verschiedener Subkulturen innerhalb der Gesellschaft verantwortlich
machten.

\subsection{Die Megapolitik der
Innovation}\label{die-megapolitik-der-innovation-1}

Noch bevor die Informationstechnologie begann, durch die „kreative
Zerstörung`` die Industriewirtschaft zu bedrohen, hatte sie bereits
viele der geschätzten Mythen von Marxisten und Sozialisten überholt. Wir
haben die Megapolitik der Innovation in einem vorherigen Kapitel
untersucht. Der Punkt, den wir dort betonten, ist von Bedeutung, um die
sozialen Auswirkungen der Informationsrevolution in Perspektive zu
setzen. Der Präzedenzfall, dass die Technologie die
Beschäftigungsmöglichkeiten in den letzten Jahrhunderten erweitert hat,
scheint eine verlässliche Regel des Wirtschaftslebens zu sein, aber das
muss nicht so sein. Es ist möglich, dass sich die Gewinne in den Händen
einer wohlhabenden Minderheit konzentrieren.

\section{DIE REALEN LÖHNE FALLEN UM 50
PROZENT}\label{die-realen-luxf6hne-fallen-um-50-prozent}

Genau das ist in den ersten zwei Jahrhunderten oder mehr des modernen
Zeitalters tatsächlich passiert. Von der Zeit der Schießpulverrevolution
um 1500 bis 1700 sanken die Realeinkommen für die unteren 60-80 Prozent
der Bevölkerung in den meisten Ländern Westeuropas um 50 Prozent oder
mehr.\footnote{Robert Jutte, \emph{Poverty and Deviance in Early Modern
  Europe} (Cambridge: Cambridge University Press, 1994), S. 29, 74.} In
vielen Orten sank das Realeinkommen weiter bis 1750 und erreichte die
Werte von 1500 erst wieder um 1850.

Im Gegensatz zu den Erfahrungen der vergangenen 250 Jahre, wurden die
Einkommensgewinne der ersten Hälfte der modernen Zeit, einer Zeit der
dramatischen Expansion der westeuropäischen Wirtschaften, unter einer
kleinen Minderheit konzentriert. Die gegenwärtige Innovation von
Informationstechnologien unterscheidet sich stark von der Innovation von
Industrietechnologien, die die Welt in den letzten Jahrhunderten erlebt
hat. Der Unterschied liegt darin, dass die meisten aktuellen
technologischen Innovationen mit arbeitssparenden Eigenschaften dazu
neigen, qualifizierte Aufgaben zu schaffen und Skaleneffekte zu
reduzieren. Dies steht im Gegensatz zu den Erfahrungen seit etwa 1750.

Die industrielle Innovation eröffnete Arbeitsmöglichkeiten für
Ungelernte und erhöhte die Skaleneffekte von Unternehmen. Dies erhöhte
nicht nur das Einkommen der Armen ohne eigenes Zutun, sondern verstärkte
auch die Macht politischer Systeme und machte sie widerstandsfähiger
gegenüber Unruhen. Diejenigen, die in den frühen Phasen der
Industriellen Revolution durch Mechanisierung und Automatisierung
verdrängt wurden, waren eher qualifizierte Handwerker, Handwerksmeister
und Gesellen als Ungelernte. Dies galt sicherlich in der
Textilindustrie, die als erste mechanisierte Verfahren und
Stromausstattung im großen Stil einsetzte und eine gewaltsame Reaktion
der Ludditen hervorrief, die Textilmaschinen zerstörten und
Fabrikbesitzer in brutalen Amokläufen im frühen 19. Jahrhundert
ermordeten. Andererseits waren die Anhänger von Captain Swing, dem
mythischen Anführer eines Aufstandes von 1830 im Südosten Englands,
Tagelöhner. Ihre Forderungen beinhalteten die Auferlegung einer Abgabe
auf die lokalen Reichen, um sie selbst mit Geld oder Bier zu versorgen,
die Forderung nach einer Lohnerhöhung von den lokalen Arbeitgebern von
Tagelöhnern und das „Zerstören oder die Forderung nach der Zerstörung
von neuen Landmaschinen, insbesondere Dreschmaschinen``, die die
Nachfrage bei den Landwirten nach ländlicher Tagelohnarbeit
reduzierten.\footnote{Tilly, \emph{Collective Violence}, S. 77.}

Im Gegensatz zum romantischen Geschwätz von Marxisten und anderen, die
die gewalttätigen Gegner der arbeitssparenden Technologie zu Helden
machten, waren sie eine unangenehme und gewalttätige Bande, die aus rein
egoistischen Gründen der Einführung von Technologie entgegenstanden,
welche den weltweiten Lebensstandard erhöht hatte.

Während die gewalttätigen Anhänger von Ned Ludd und Captain Swing die
öffentliche Ordnung in England über viele Monate hinweg gefährdeten,
waren ihre Bewegungen, sobald sie von der Zentralgewalt unterdrückt
wurden, zum Scheitern verurteilt. Die arme, ungelernte Mehrheit dürfte
sich nicht lange von einer Sache angezogen fühlen, die versprach,
Maschinen zu zerstören, die ihnen Arbeitsplätze boten und auch ihren
Lebensstandard erhöhten, indem sie die Kosten für die von ihnen
benötigten Dinge wie warme Kleidung und Brot senkten.

\subsection{Höhere Einkommen für
Unqualifizierte}\label{huxf6here-einkommen-fuxfcr-unqualifizierte}

Mit der Zeit wurde die industrielle und landwirtschaftliche
Automatisierung für die sozial Schwachen attraktiv, da sie ihnen
Verdienstmöglichkeiten bot und ihre Lebenshaltungskosten senkte. Neue
Werkzeuge ermöglichten es auch den Unqualifizierten, Waren von gleicher
Qualität wie die von hochqualifizierten Arbeitern herzustellen. Ein
Genie und ein Dummkopf an der Montagelinie stellten beide dasselbe
Produkt her und verdienten den gleichen Lohn.

In den vergangenen zwei Jahrhunderten hat die industrielle
Automatisierung die Löhne für ungelernte Arbeit dramatisch gesteigert,
insbesondere in jenem kleinen Teil der Welt, in dem die Bedingungen
erstmals dem Kapitalismus ermöglichten, zu gedeihen. Die groß angelegten
fortschrittlichen Industrieunternehmen belohnten nicht nur ungelernte
Arbeitskräfte mit beispiellosen Löhnen, sie erleichterten auch die
Umverteilung von Einkommen.

Der Wohlfahrtsstaat entstand als logische Folge der Technologie des
Industriezeitalters. Aufgrund ihrer Größe und der hohen Kapitalkosten
waren die führenden Industriearbeitgeber die einfachsten Ziele für
Besteuerung. Und man konnte sich darauf verlassen, dass sie
Aufzeichnungen führten und die Pfändung von Löhnen durchsetzten, die die
Einkommenssteuer technologisch umsetzbar machte, wie es in vorherigen
Jahrhunderten bei dezentraleren Wirtschaften nicht möglich gewesen war.
Das Nettoergebnis war, dass das Wachstum der Skaleneffekte, das durch
industrielle Innovationen gefördert wurde, die Regierungen reicher
machte und vermutlich besser in der Lage war, Ordnung zu halten.

\subsection{Der Prozess wird
umgekehrt}\label{der-prozess-wird-umgekehrt}

Nach unserer Einschätzung findet heute das Gegenteil statt. Die
Informationstechnologie erhöht die Verdienstmöglichkeiten für Fachkräfte
und untergräbt Institutionen, die in großem Maßstab arbeiten,
einschließlich des Nationalstaates.

Dies weist auf eine weitere Ironie des Informationszeitalters hin,
nämlich die gespaltene und grundsätzlich hinderliche Haltung der
Kritiker des freien Marktes gegenüber dem Aufstieg und Fall
industrieller Arbeitsplätze. In den Anfangsphasen des
Industriezeitalters waren sie aufgrund des vermeintlichen Übels
industrieller Arbeitsplätze, die landlose Bauern von der „verlorenen
Welt`` weglockten, aufgebracht. Hört man den Kritikern zu, so war das
Aufkommen von Fabrikarbeitsplätzen ein beispielloses Übel und eine
„Ausbeutung`` der Arbeiterklasse. Aber jetzt scheint es, dass das einzig
Schlimmere als das Aufkommen von Fabrikarbeitsplätzen deren Verschwinden
ist. Die Urenkel derjenigen, die über die Einführung von
Fabrikarbeitsplätzen klagten, jammern jetzt über den Mangel an
Fabrikarbeitsplätzen, die hohe Löhne für gering qualifizierte Arbeit
bieten.

Der eine kohärente Faden, der sich durch diese Beschwerden zieht, ist
ein standhaftes Widerstandsvermögen gegen technologische Innovationen
und Marktveränderungen. In den frühen Stadien des Fabriksystems führte
dieser Widerstand zu Gewalt. Das könnte wieder geschehen.

Und das nicht, weil Kapitalisten die Arbeiter „ausbeuten``. Die Ankunft
des Computers als paradigmenbildende Technologie offenbarte die
Absurdität dieser Behauptung. Es könnte halbwegs glaubhaft für die
Unaufmerksamen erscheinen, dass ein kaum alphabetisierter Automechaniker
irgendwie bei der Herstellung eines Autos „ausgebeutet`` wurde und das
von Eigentümern, die die Unternehmen konzipierten, finanzierten und die
Arbeiter beschäftigten. Die entscheidende Rolle des konzeptionellen
Kapitals bei der Produktion und Vermarktung von physischen Produkten war
weniger offensichtlich als sie es bei Produkten des
Informationszeitalters ist, die eindeutig geistige Arbeit erfordert. Die
Annahme, dass die Unternehmer den Wert der von den Arbeitnehmern
geschaffenen Informationsprodukte irgendwie an sich gerissen haben, ist
daher sehr viel weniger plausibel. Da der Wert eindeutig durch geistige
Arbeit geschaffen wurde, wie bei der Produktion von Verbrauchersoftware,
war es geradezu absurd anzunehmen, dass es tatsächlich das Produkt von
jemand anderem als den qualifizierten Personen war, die es konzipiert
haben. Tatsächlich führte der offensichtliche und wachsende Trend weg
von ungelernter Beschäftigung zu einer zunehmenden Sorge um genau das
gegenteilige Problem: Ob ungelernte Arbeiter überhaupt noch einen
wirtschaftlichen Beitrag leisten können.\footnote{William Julius Wilson,
  \emph{When Work Disappears: The World of the New Urban Poor} (New
  York: Alfred A. Knopf, 1996).}

Daher verschiebt sich die Begründung für die Einkommensumverteilung von
der „Ausbeutung``, die eine produktive Kompetenz bei Personen mit
niedrigem Einkommen voraussetzte, hin zu „Diskriminierung``, die dies
nicht tat. „Diskriminierung`` wurde jedoch dafür verantwortlich gemacht,
dass es Personen mit geringen Fähigkeiten nicht gelingt, wertvollere zu
entwickeln.

Diese Diskriminierung wurde häufig dazu genutzt, die Einführung von
nicht optimalen Einstellungskriterien und anderen Kriterien zur
Schaffung von „Möglichkeiten`` zu rechtfertigen. Genauer gesagt, ging es
um die Umverteilung von Einkommen zugunsten der benachteiligten Gruppen.
In den USA zum Beispiel ermöglichte die Einstufung von Leistungs- und
Eignungstests aufgrund ethnischer Zugehörigkeit, dass Schwarze bessere
Ergebnisse erzielten als weiße und asiatische Bewerber, obwohl sie
objektiv gesehen niedrigere Punktzahlen erhielten. Durch diese und
andere Methoden wurden Arbeitgeber von den Regierungen gezwungen, mehr
Schwarze und andere offiziell als „Opfer`` bezeichnete Gruppen zu
höheren Löhnen einzustellen, als dies sonst der Fall gewesen wäre. Wer
sich nicht daran hielt, sah sich mit kostspieligen Gerichtsverfahren
konfrontiert, einschließlich Klagen mit hohen Schadenersatzforderungen.

Der Zweck der Benennung von Opfern bestand nicht darin, paranoide
Verfolgungswahnvorstellungen unter wichtigen Untergruppen der
Industriegesellschaft zu schüren, oder die Verbreitung von
kontraproduktiven Werten zu subventionieren. Es ging darum, den
bankrotten Staat vom fiskalischen Druck der Einkommensumverteilung zu
entlasten. Das Einflößen von Verfolgungswahnvorstellungen war lediglich
ein unglücklicher Nebeneffekt. Ironischerweise fiel das Anwachsen der
Sorge um „Diskriminierung`` mit den frühen Stadien einer technologischen
Revolution zusammen, die die tatsächliche willkürliche Diskriminierung
weitaus weniger zum Problem machen wird, als sie es jemals zuvor war. Im
Internet weiß oder kümmert sich niemand darum, ob der Autor eines neuen
Softwareprogramms schwarz, weiß, männlich, weiblich, homosexuell oder
ein vegetarischer Zwerg ist.

Während die Realität der Diskriminierung in der Zukunft wahrscheinlich
weniger bedrückend sein wird, wird dies nicht unbedingt den Druck auf
„Reparationen`` mindern, um verschiedene reale oder eingebildete
Ungerechtigkeiten auszugleichen. Jede Gesellschaft, unabhängig von ihren
objektiven Umständen, bringt eine oder mehrere Rechtfertigungen für die
Einkommensumverteilung hervor. Sie reichen von subtil bis absurd, vom
biblischen Gebot, deinen Nächsten wie dich selbst zu lieben, bis hin zu
Beschwörungen von schwarzer Magie. Zauberei, Hexerei und der Böse Blick
sind die Kehrseite des religiösen Gefühls, das geistige Äquivalent des
Finanzamts. Wenn die Menschen nicht durch Liebe dazu bewegt werden
können, die Armen zu subventionieren, werden die Armen selbst versuchen,
sicherzustellen, dass sie durch Angst bewegt werden. Manchmal nimmt dies
die Form einer regelrechten Erpressung an, ein Messer an der Kehle, eine
Waffe an den Kopf. Zu anderen Zeiten ist die Bedrohung versteckt oder
eingebildet. Es ist kein Zufall, dass die meisten „Hexen`` der frühen
Neuzeit Witwen oder unverheiratete Frauen mit wenigen Ressourcen waren.
Sie schreckten ihre Nachbarn mit Flüchen ab, die nicht selten dazu
führten, dass diese Nachbarn zahlten. Es ist keineswegs offensichtlich,
dass dies nur die Abergläubischen waren. Die böswillige Absicht des
Bösen Blicks war kein Aberglaube, sondern eine Tatsache. Sogar eine arme
Frau konnte Vieh befreien oder das Haus eines anderen in Brand setzen.
In diesem Sinne waren die Hexenprozesse der frühen Neuzeit nicht so
völlig absurd, wie sie scheinen. Während die Strafen grausam waren und
zweifellos viele Unschuldige unter den Halluzinationen von Nachbarn
unter dem Einfluss von Mutterkornvergiftung litten, kann die Verfolgung
von Hexen als indirekter Weg zur Verfolgung von Erpressung verstanden
werden.

Wir erwarten eine Wiederkehr von Erpressung, die durch den Wunsch
motiviert ist, an den Belohnungen des Erfolgs teilzuhaben, während das
Informationszeitalter sich entfaltet. Gruppen, die sich aufgrund
vergangener Diskriminierung benachteiligt fühlen, werden ihren
anscheinend wertvollen Status als Opfer wahrscheinlich nicht schnell
aufgeben, einfach weil ihre Forderungen an die Gesellschaft weniger
gerechtfertigt oder schwerer durchzusetzen werden. Sie werden ihre
Forderungen weiterhin verfolgen, bis Beweise in ihrer lokalen Umgebung
keinen Zweifel daran lassen, dass sie nicht länger belohnt werden.

Das Wachstum soziopathischen Verhaltens unter Afro-Amerikanern und
Afro-Kanadiern verdeutlicht dies. Es zeigt, dass zwischen dem Zorn der
Schwarzen und einer realistischen Bewertung des Ausmaßes, in dem die
Probleme der Schwarzen selbstverschuldete Folgen antisozialen Verhaltens
sind, wenig Ausgleich besteht. Der Zorn unter den Schwarzen hat
zugenommen, selbst als ihre Lebensstile immer dysfunktionaler geworden
sind. Die Geburtenrate außerhalb der Ehe hat stark zugenommen. Der
Bildungsabschluss hat abgenommen. Wachsende Anteile junger Schwarzer
sind in kriminelle Aktivitäten verwickelt, sodass es jetzt mehr schwarze
Männer in Gefängnissen gibt als in Hochschulen.

Diese perversen Ergebnisse könnten vorübergehend den Ressourcenfluss zu
benachteiligten Gemeinschaften während der Dämmerung der
Industrialisierung erhöht haben, indem sie die Bedrohung der
Gesellschaft insgesamt gesteigert haben. Aber der Effekt konnte nur
vorübergehend sein. Indem der Wohlfahrtsstaat den positiven Einfluss des
Wettbewerbs auf die Herausforderung von untätigen Personen, sich an
produktive Normen anzupassen, beseitigte, half er bei der Schaffung von
Legionen von dysfunktionalen, paranoiden und kulturell schlecht
angepassten Menschen, dem sozialen Äquivalent eines Pulverfasses. Der
Untergang des Nationalstaates und das Verschwinden der
Einkommensumverteilung in großem Maßstab werden zweifellos einigen der
psychopathischeren dieser unglücklichen Seelen dazu verleiten, gegen
jeden vorzugehen, der wohlhabender erscheint als sie. Daher ist es
sinnvoll anzunehmen, dass der soziale Frieden in Gefahr sein könnte,
wenn das Informationszeitalter beginnt, insbesondere in Nordamerika und
in multiethnischen Enklaven in Westeuropa.

\begin{quote}
„Wir werden niemals die Waffen niederlegen{[}\^{}bis{]} das House of
Commons ein Gesetz verabschiedet, das alle Maschinen abschafft, die der
Gemeinschaft schaden, und die Hinrichtung von Rahmenbrechern widerruft.
Aber wir. Wir bitten nicht mehr - das wird nicht ausreichen - es muss
gekämpft werden.`` unterzeichnet von dem General der Armee der
Wiedergutmacher Ned Ludd Clerk „Wiedergutmacher für immer, Amen``
\footnote{Tilly, \emph{Collective Violence}, S. 78.}
\end{quote}

\subsection{Neo-Ludditen}\label{neo-ludditen}

Angesichts der Erfahrungen mit antitechnologischer Rebellion im frühen
19. Jahrhundert und der langen Tradition kollektiver Gewalt in Europa
und Nordamerika sollte niemand überrascht sein, einen neo-luddistischen
Angriff auf Informationstechnologie und deren Nutzer zu sehen. Die
bereits erwähnten Ludditen waren Textilarbeiter in West Yorkshire,
England, die 1811-1812 eine terroristische Kampagne gegen automatisierte
Schneidemaschinen und die Fabrikbesitzer, die sie in ihre Produktion
aufnahmen, starteten.\footnote{Robert Reid, \emph{Land Of Lost Content:
  The Luddite Revolt 1812} (London: Penguin, 1986), S. 44.} Mit
geschwärzten Gesichtern zogen die Ludditen durch West Yorkshire,
verbrannten Fabriken und ermordeten Fabrikbesitzer, die es wagten, die
neue Technologie anzunehmen. Der Großteil der Gewalt ging von den
sogenannten „Scherern`` aus, hoch qualifizierten Handwerkern, deren
Arbeit im Umgang mit Riesenscheren von bis zu fünfundzwanzig Kilogramm
zuvor ein wichtiger Teil der Wollstoffproduktion war. Aber die
nachträgliche Arbeit, die die Scherer leisteten, „das Aufrauen der
Oberfläche mit Distelköpfen und das Schneiden des Stoffes mit Scheren``,
war laut Robert Reid, Autor des besten und umfassendsten
Diskussionsbeitrags über den Ludditen-Aufstand, \emph{Land of Lost
Content: The Luddite Revolt 1812}, zu einfach, um nicht mechanisiert zu
werden.\footnote{Ebenda, S. 45.} Leonardo da Vinci hatte einen Entwurf
für eine solche mechanisierte Schneidemaschine skizziert. Aber Leonardos
Entwurf für automatisches Schneiden blieb jahrhundertelang ungenutzt.
Schließlich wurde 1787 eine Vorrichtung wie die von Leonardo in England
neu erfunden und in Produktion gebracht. Wie Reid bemerkt, „waren die
Bestandteile der Technologie schon so lange bekannt, dass es
überraschend ist, dass sie nicht früher eingeführt wurden. Die neuen
Geräte der industriellen Revolution erforderten so wenig Stärke und
Fähigkeiten, dass viele Arbeitsplätze von Frauen und jungen Kindern zu
niedrigen Löhnen besetzt wurden. Eine dieser neuen Maschinen, auch von
relativ ungelernten Personen bedient, konnte jetzt in achtzehn Stunden
das erledigen, wofür ein qualifizierter Scherer mit Handscheren
achtundachtzig Stunden brauchte.`` \footnote{Ebenda, S. 26.}

Man beachte, dass die Arbeiter, die gegen die Mechanisierung
protestierten, bei ihrer Opposition gegen die neue Technologie sehr
differenziert vorgingen. Sie griffen nur die Technologien an und
bekämpften diejenigen, die ihre eigenen Arbeitsplätze bedrohten oder die
Nachfrage nach qualifizierter Arbeit minderten. Als ein Unternehmer
namens William Cooke Maschinen zur Teppichherstellung in den Bezirk West
Yorkshire einführte, löste dies überhaupt keine Gewalt aus. Es wurden
keine Versuche unternommen, Cookes Mühle zu zerstören bzw. seine
Maschinen zu sabotieren, geschweige denn ihn zu töten. Wie Robert Reid
in seiner Geschichte der Ludditen-Aufstände erklärt, rief Cookes neue
Technologie keinerlei Opposition hervor, denn Teppiche waren ein
Produkt, „auf dessen Herstellung bisher niemand im Tal spezialisiert
war``.\footnote{Ebenda.} Reid führt weiter aus: „Weil Cooke ein neues
Produkt einführte und Arbeitsplätze schuf, die auf keinerlei
traditionellen Praktiken basierten, blühte seine Mühle auf.``
\footnote{Ebenda.} Dies ist ein Beispiel mit wichtiger Anwendung für die
Zukunft. Es legt nahe, dass vorausdenkende Unternehmer im nächsten
Jahrtausend zunächst dramatische arbeitssparende Automatisierungen in
Regionen einführen werden, in denen es keine Tradition gibt, das
betreffende Produkt oder die betreffende Dienstleistung zu produzieren.

Wenn die Vergangenheit als Leitfaden dient, werden die gewalttätigsten
Terroristen der ersten Jahrzehnte des neuen Jahrtausends nicht
obdachlose Bettler sein, sondern verdrängte Arbeiter, die zuvor
Mittelschichtseinkommen und -status genossen haben. Dies war sicherlich
der Fall beim Ludditen-Aufstand von 1812, bei dem der Großteil der
Ludditen keine verarmten Proletarier, sondern qualifizierte Handwerker
waren, die es gewohnt waren, Einkommen zu erzielen, die fünfmal so hoch
oder höher waren als die eines durchschnittlichen Arbeiters. Die
äquivalente Gruppe heute wären wahrscheinlich verdrängte Fabrikarbeiter.
Leider findet man bei der Betrachtung der Bevölkerungsdaten der meisten
OECD-Länder mehr Gebiete, die als mögliche Standorte für gewalttätige
Reaktionen hervorgehoben werden könnten, als Gebiete, die nicht in diese
Kategorie fallen.

Die Nationalstaaten der Welt werden versuchen, der Cyberwirtschaft und
den souveränen Individuen, die in der Lage sind, diese zu ihrem Vorteil
zu nutzen und Reichtum anzuhäufen, entgegenzuwirken. Eine wütende
nationalistische Reaktion wird die Welt erfassen. Ein integraler
Bestandteil davon wird eine Antitechnologie-Reaktion sein, äquivalent zu
den Ludditen und anderen antitechnologischen Aufständen in
Großbritannien während der Industriellen Revolution. Dies sollte genau
betrachtet werden, denn es könnte ein Schlüssel zur Entwicklung des
Regierungswesens im neuen Jahrtausend sein. Eine der entscheidenden
Herausforderungen der bevorstehenden großen Veränderung wird darin
bestehen, die Ordnung angesichts eskalierender Gewalt aufrechtzuerhalten
oder alternativ ihren Auswirkungen zu entkommen. Individuen und
Unternehmen, die in besonderem Maße mit dem Aufkommen des
Informationszeitalters in Verbindung gebracht werden, einschließlich
derer im Silicon Valley und sogar der Lieferanten von Elektrizität, die
zur Stromversorgung der neuen Technologie erforderlich ist, werden
besondere Wachsamkeit gegenüber freiberuflichem Neo-Ludditen-Terrorismus
aufrechterhalten müssen.

Ein Verrückter wie der Unabomber wird leider wahrscheinlich Brigaden von
Nachahmern anspornen, da die Frustration über sinkende Einkommen und der
Unmut gegenüber Leistung zunehmen wird. Wir vermuten, dass ein Großteil
der kommenden Gewalt Bombenanschläge beinhalten wird. Wie in der New
York Times berichtet, stieg der innenpolitische Terrorismus in den USA
während der 1990er Jahre stark an. „Sie haben in den letzten fünf Jahren
um mehr als 50 Prozent zugenommen und haben sich im letzten Jahrzehnt
fast verdreifacht. Die Anzahl der kriminellen Explosionen und deren
Versuche ging von 1.103 im Jahr 1985 auf 3.163 im Jahr 1994 ... in
Kleinstädten und Vororten sowie unter innerstädtischen Straßengangs hat
es eine Art Verbreitung des „Gartenbombenlegers`` gegeben.`` \footnote{Timothy
  Egan, \emph{Terrorism Now Going Homespun as Bombings in the U.S.
  Spread}, \emph{New York Times}, 25. August 1996, S. 1.}

\subsection{Die Verteidigung wird zu einem privaten
Gut}\label{die-verteidigung-wird-zu-einem-privaten-gut}

Trotz der Strafsteuern, die von Nationalstaaten als Schutzgebühr
verhängt werden, ist es unwahrscheinlich, dass sie in den kommenden
Jahren eine effektive Abschreckung bieten können. Die abnehmende Rate
der Gewalt, die durch die neue Informationstechnologie angedeutet wird,
macht den Aufbau einer umfangreichen militärischen Infrastruktur weit
weniger nützlich. Das bedeutet nicht nur, dass Staaten weniger in der
Lage sein werden, ihre Bürger tatsächlich zu schützen, es weist auch
darauf hin, dass die scheinbare extraterritoriale Hegemonie der
Vereinigten Staaten als Weltmacht im nächsten Jahrhundert weniger
effektiv sein wird als die von Großbritannien im neunzehnten
Jahrhundert. Bis zum Ausbruch des Ersten Weltkriegs konnte die Macht
effektiv und entscheidend von der Kernregion zur Peripherie zu relativ
geringen Kosten projiziert werden. Im einundzwanzigsten Jahrhundert
werden die Bedrohungen, die Großmächte für die Sicherheit von Leben und
Eigentum darstellen, notwendigerweise mit der Rückkehr zur Gewalt
abnehmen. Die sinkenden Renditen der Gewalt legen nahe, dass
Nationalstaaten oder Imperien, die in der Lage sind, militärische Macht
in großem Maßstab auszuüben, im Informationszeitalter wahrscheinlich
nicht überleben oder entstehen werden.

Da der finanzielle Bedarf zur Bereitstellung einer angemessenen
Verteidigung sinkt, wird es immer glaubwürdiger, Schutzdienstleistungen
wie private Güter zu behandeln. Schließlich werden
Sicherheitsbedrohungen in geringerem Umfang zunehmend durch
Sicherheitskräfte abwehrbar sein, die kommerziell engagiert werden
können, etwa durch den Einsatz von Mauern, Zäunen und Sicherheitszonen
zur Abwehr von Störenfrieden. Weiterhin könnte ein wohlhabender
Einzelner oder ein Unternehmen in der Lage sein, sich gegen die meisten
Bedrohungen zu schützen, die im Informationszeitalter wahrscheinlich
aufkommen würden. An der Grenze wird die verminderte Größe der
militärischen Bedrohungen die Gefahr einer Anarchie oder konkurrierender
Gewalt innerhalb eines einzigen Territoriums erhöhen. Doch sie wird auch
den Wettbewerb zwischen den Zuständigkeiten bei der Bereitstellung von
Schutz zu wettbewerbsfähigen Bedingungen intensivieren. Dies wird dazu
führen, dass Schutzdienste, Pass- und Konsulatsdienste sowie die
Bereitstellung von Rechtsmitteln verstärkt eingekauft werden.

Auf lange Sicht werden die souveränen Individuen wahrscheinlich in der
Lage sein, mit nichtstaatlichen Dokumenten zu reisen, die, wie
Akkreditivschreiben, von privaten Organisationen und Interessengruppen
ausgestellt werden. Es ist nicht weit hergeholt, anzunehmen, dass eine
Gruppe wie eine Art Kaufmannsrepublik des Cyberspace, ähnlich wie die
mittelalterliche Hanseatische Liga, auftreten wird, um die Aushandlung
privater Verträge und Abkommen zwischen Rechtsordnungen zu erleichtern
sowie Schutz für ihre Mitglieder zu gewährleisten. Stellen Sie sich
einen speziellen Pass vor, ausgestellt von der Liga der souveränen
Individuen, der den Inhaber als Person unter dem Schutz der Liga
identifiziert.

Ein solches Dokument, sollte es entstehen, wäre nur ein vorübergehendes
Artefakt des Übergangs weg vom Nationalstaat und dem von ihm geförderten
bürokratischen Zeitalter. Vor der modernen Ära waren Pässe im
Allgemeinen unnötig zur Grenzüberschreitung, welche in den meisten
Fällen eher lose definiert waren. Zwar wurden in mittelalterlichen
Grenzgesellschaften gelegentlich Schutzbriefe verwendet, doch wurden
diese in der Regel von den Behörden ausgestellt, deren Reich besucht
werden sollte, und nicht von dem Land, aus dem der Reisende stammte.
Wichtiger als ein Pass waren Einführungs- und Kreditbriefe, die es einem
Reisenden ermöglichten, Unterkunft zu finden und geschäftlich zu
verhandeln. Diese Tage werden wieder kommen. Letztendlich werden
Personen ganz ohne Dokumente reisen können. Sie werden in der Lage sein,
sich auf einer narrensicheren biometrischen Basis durch
Stimmerkennungssysteme oder Netzhautscans zu identifizieren, die sie
eindeutig erkennen.

Kurz gesagt, erwarten wir, dass irgendwann in der ersten Hälfte des
nächsten Jahrhunderts die Welt die echte Privatisierung von Souveränität
erfahren wird. Dies wird Bedingungen begleiten, die den Raum des Zwangs
auf ein logisches Minimum schrumpfen lassen könnten. Dennoch wird es für
die säkularen Inquisitoren und Reaktionäre des nächsten Jahrtausends
sowohl ärgerlich als auch bedrohlich sein, die einst „heiligen``
Attribute der Nationalität auf eine Marktbasis zu stellen, um als eine
Frage der Kosten-Nutzen-Rechnung gekauft und verkauft zu werden.

In diesem Buch argumentieren wir, dass es keinen Nationalstaat mehr
braucht, um einen Informationskrieg zu führen. Solche Kriege könnten von
Computerprogrammierern ausgeführt werden, die große Mengen von „Bots``
oder digitalen Dienern bereitstellen. Bill Gates verfügt bereits über
eine größere Fähigkeit, Logikbomben in global anfälligen Systemen zu
zünden, als die meisten Nationalstaaten der Welt. Im Zeitalter des
Informationskriegs wäre jedes Softwareunternehmen oder sogar die Kirche
von Scientology ein mächtigerer Gegner als die gesammelte Bedrohung, die
von den meisten Staaten mit Sitzen in den Vereinten Nationen ausgeht.

Dieser Machtverlust der Nationalstaaten ist eine logische Folge des
Aufkommens kostengünstiger, fortschrittlicher Rechenkapazitäten.
Mikroverarbeitung reduziert nicht nur die Renditen von Gewalt, sondern
schafft erstmals auch einen Wettbewerbsmarkt für die Schutzdienste, für
die Regierungen in der Industriezeit Monopolpreise verlangt haben.

In der neuen Welt der kommerzialisierten Souveränität werden die
Menschen ihre Gerichtsbarkeiten wählen, ganz so wie viele heute ihre
Versicherungsanbieter oder ihre Religionen auswählen.\footnote{Siehe
  Stephen J. Duhner, \emph{Choosing My Religion}, New York Times
  Magazine, 31. März 1996, S. 36f.} Rechtssysteme, die es versäumen, ein
passendes Leistungsangebot bereitzustellen, was immer das sein mag,
werden vor dem Bankrott und der Liquidation stehen, genau wie
inkompetente Handelsunternehmen oder gescheiterte religiöse
Gemeinschaften. Der Wettbewerb wird daher die Bemühungen der lokalen
Gerichtsbarkeiten mobilisieren, ihre Fähigkeit zur wirtschaftlichen und
effektiven Bereitstellung von Dienstleistungen zu verbessern. In dieser
Hinsicht wird der Wettbewerb zwischen den Justizsystemen bei der
Bereitstellung öffentlicher Güter einen ähnlichen Einfluss haben, wie in
anderen Lebensbereichen zu beobachten ist. Wettbewerb führt in der Regel
zu einer erhöhten Kundenzufriedenheit.

\section{WETTBEWERB UND ANARCHIE}\label{wettbewerb-und-anarchie}

Es ist wichtig zu bedenken, dass der von uns erwartete Wettbewerb
zwischen Rechtssystemen nicht hauptsächlich ein Wettbewerb zwischen
Organisationen ist, die Gewalt im selben Territorium einsetzen. Wie
zuvor angegeben, führen wettbewerbsfähige Organisationen, die Gewalt
einsetzen, dazu, dass die Durchdringung von Gewalt im Leben zunimmt und
die wirtschaftlichen Möglichkeiten verringert. Wie Lane es ausdrückte:

\begin{quote}
Bei der Anwendung von Gewalt gab es offensichtlich erhebliche
Skaleneffekte im Wettbewerb mit rivalisierenden gewaltanwendenden
Unternehmen oder bei der Etablierung eines territorialen Monopols. Diese
Tatsache ist grundlegend für die wirtschaftliche Analyse eines Aspekts
der Regierung: Die gewaltanwendende, gewaltkontrollierende Industrie war
zumindest auf dem Land ein natürliches Monopol. Innerhalb territorialer
Grenzen konnte der von ihr erbrachte Dienst durch ein Monopol weitaus
kostengünstiger produziert werden. Sicherlich gab es Zeiten, in denen
gewaltanwendende Unternehmen im fast gleichen Territorium um
Schutzgelder konkurrierten, zum Beispiel während des Dreißigjährigen
Krieges in Deutschland. Aber eine solche Situation war noch
unwirtschaftlicher als ein Wettbewerb auf dem gleichen Territorium
zwischen konkurrierenden Telefonsystemen.\footnote{Lane, \emph{Economic
  Consequences of Organized Violence}, S. 402.}
\end{quote}

Lanes Kommentar liefert in zwei Aspekten aufschlussreiche Informationen.
Erstens stimmen wir mit seiner allgemeinen Schlussfolgerung überein,
dass Souveränitäten tendenziell territoriale Monopole ausüben werden, da
sie dadurch in der Lage sind, günstigere und effektivere
Schutzdienstleistungen anzubieten. Der zweite interessante Aspekt von
Lanes Bemerkung ist sein veralteter Vergleich mit dem
Monopol-Telefondienst. Offensichtlich wissen wir jetzt, dass
Telefonsysteme nicht zwangsläufig Monopole sein müssen. Dies führt zu
einer gewissen Vorsicht in der Analyse. Technologische Veränderungen
könnten die allgemeine Schlussfolgerung, dass Anarchie innerhalb
territorialer Grenzen nicht lebensfähig ist, teilweise hinfällig machen.
Beispielsweise könnten, wenn Cybervermögen in einem Bereich wachsen, der
sie außerhalb des Zwangsbereichs stellt, die Preise für
Schutzdienstleistungen weniger eine Frage der „Nachfrage`` und mehr eine
Frage der Marktaushandlung sein.

Trotzdem beziehen wir uns hier auf etwas anderes als auf generalisierte
Anarchie - nämlich auf den Wettbewerb zwischen Zuständigkeitsbereichen,
jeder mit einem Gewaltmonopol in seinem eigenen Territorium. Wir sehen,
dass diese Gerichtsbarkeiten darum konkurrieren, den größtmöglichen Wert
bei der kosteneffizienten Bereitstellung von Schutzdiensten zu bieten,
die für ihre „Kunden`` attraktiv sind. Zugegebenermaßen wird es
zweifellos größere Unklarheiten bei der Bereitstellung von
Schutzdiensten im Informationszeitalter geben, mit einer umfassenderen
privaten Bereitstellung von Polizei- und Verteidigungsdiensten, als wir
es gewohnt sind. Dennoch ist der Wettbewerb, den wir uns vorstellen,
anders als ein Konflikt mehrerer Schutzagenturen, die im großen Stil
darum kämpfen, verschiedenen Kunden im selben Territorium
Dienstleistungen anzubieten, was Anarchie wäre.

Trotzdem bedeutet die Vermehrung von Souveränitäten, bei der die
Individuen mehr und mehr die Rolle der Souveräne übernehmen, wenn sie
genügend Ressourcen akkumulieren, unweigerlich, dass der Spielraum für
Anarchie in der Welt zunimmt. Die Beziehungen zwischen den
Souveränitäten sind immer anarchistisch. Es gibt und gab nie eine
Weltregierung, die das Verhalten einzelner Souveränitäten, seien es
Mini-Staaten, Nationalstaaten oder Imperien, reguliert. Wie Jack
Hirshleifer schreibt: „Während Vereinigungen von primitiven Stämmen bis
zu modernen Nationalstaaten alle intern durch eine Form von Recht
regiert werden, bleiben ihre externen Beziehungen zueinander
hauptsächlich anarchistisch.`` \footnote{Jack Hirshleifer, \emph{Anarchy
  and Its Breakdown}, in Michelle R. Garfinkel und Stergios Skaperdas,
  eds., \emph{The Political Economy of Conflict and Appropriation}
  (Cambridge: Cambridge University Press, 1996), S. 15.} Wenn es mehr
souveräne Einheiten in der Welt gibt, entstehen unweigerlich mehr
Beziehungen in mehr als einer Gerichtsbarkeit, und diese sind daher
anarchistisch.

Es ist wichtig zu beachten, dass Anarchie, oder das Fehlen einer
übergeordneten Macht zur Schlichtung von Streitigkeiten, nicht mit
totalem Chaos oder der Abwesenheit von Form oder Organisation
gleichzusetzen ist. Hirshleifer merkt an, dass Anarchie analysiert
werden kann: „Auch stammesübergreifende oder internationale Systeme
haben ihre Gesetzmäßigkeiten und systematisch analysierbaren Muster.``
\footnote{Ebenda, S. 15.} Mit anderen Worten, so wie „Chaos`` in der
Mathematik eine komplexe und hoch geordnete Organisationsform darstellen
kann, ist „Anarchie`` nicht völlig formlos oder ungeordnet.

Hirshleifer analysiert eine Anzahl von anarchistischen Umgebungen. Dazu
gehören neben den Beziehungen zwischen Souveränitäten auch Gang-Kriege
im Chicago der Prohibitionszeit und „Bergleute gegen Claim-Jumper
(unehrliche Bergleute, die gegen Eigentumsansprüche verstoßen) während
des Goldrauschs in Kalifornien``. Beachten Sie, dass Kalifornien bereits
zum Beginn des Goldrauschs im Jahr 1849 Teil der Vereinigten Staaten
war, die Bedingungen in den Goldschürfgebieten jedoch als Anarchie
bezeichnet werden können. Wie Hirshleifer anmerkt, „waren die
offiziellen Organe des Gesetzes machtlos.`` \footnote{Ebenda, S. 34.} Er
argumentiert, dass topographische Bedingungen in den bergigen Lagern,
sowie effektive Selbstjustizorganisationen der Bergleute zur Bekämpfung
der Claim-Jumper, es für außenstehende Banden schwierig machten,
Goldminen zu beanspruchen, trotz der mangelnden effektiven
Strafverfolgung. Anders ausgedrückt, unter bestimmten Bedingungen kann
wertvolles Eigentum auch in Anarchie effektiv geschützt werden.

Die Frage ist, ob Hirshleifers theoretische Analyse der Dynamik der
spontanen Ordnung der darwinschen „natürlichen Wirtschaft`` für die
Wirtschaft des Informationszeitalters von Relevanz ist. Wir vermuten,
ja. Obwohl wir nicht überall eine allgemeine Anarchie oder
Goldschürfbedingungen erwarten, rechnen wir mit einer Zunahme der Anzahl
anarchistischer Beziehungen im Weltgefüge. Angesichts dieser Erwartung
ist Hirshleifers Argument über die Bedingungen, unter denen „zwei oder
mehr anarchistische Konkurrenten lebensfähige Anteile an den sozial
verfügbaren Ressourcen im Gleichgewicht`` behalten können,
suggestiv.\footnote{Ebenda, S. 17.} Insbesondere untersucht er, wann
Anarchie dazu neigt, in Tyrannei oder Dominanzhierarchien
„zusammenzubrechen``, was geschieht, wenn die anarchistischen Parteien
durch eine überwältigende Autorität unterdrückt werden können.

Diese Themen könnten im Informationszeitalter wichtiger zu verstehen
sein, als sie es im Industriezeitalter waren. Ein Teil des Grundes
dafür, dass die feineren Unterscheidungen bezüglich der Dynamik der
Anarchie in den letzten Jahrhunderten weniger kritisch waren als sie es
im neuen Jahrtausend sein könnten, liegt genau darin, dass die
Gewinnchancen durch Gewalt im modernen Zeitalter gestiegen sind. Dies
bedeutete, dass der Aufbau immer größerer Militärkräfte, wie es die
Nationalstaaten in den letzten Jahrhunderten getan haben, zu einer
entscheidenden Kriegsführung neigte. Eine entscheidende Kriegsführung
unterwirft fast zwangsläufig die Anarchie, indem sie die Konkurrenten um
die Kontrolle der Ressourcen unter die Dominanz einer mächtigeren
Autorität bringt. Andererseits trägt die abnehmende Entscheidungskraft
in der Schlacht, die der Überlegenheit der Verteidigung in der
Militärtechnologie entspricht, zur dynamischen Stabilität der Anarchie
bei. Daher sollte der offensichtliche Einfluss der
Informationstechnologie auf die Reduzierung der Entscheidungskraft
militärischer Aktionen die Anarchie zwischen Mini-Souveränitäten
stabiler machen und weniger anfällig dafür, durch Eroberung durch eine
große Regierung ersetzt zu werden. Weniger Entscheidung in der Schlacht
impliziert auch weniger Kämpfe, was eine ermutigende Schlussfolgerung
für die Welt im Informationszeitalter ist.

\subsection{Machbarkeit}\label{machbarkeit}

Eine weitere wichtige Bedingung für das Fortbestehen der Anarchie ist
die Machbarkeit oder ausreichende Einkommen. Individuen, denen ein
ausreichendes Einkommen zum Überleben fehlt, neigen dazu, entweder 1.
viel Mühe in den Kampf zu stecken, um genug Ressourcen zu erbeuten, um
zu überleben, oder 2. sich einem anderen Wettbewerber im Austausch für
Nahrung und Unterstützung zu ergeben. Etwas Ähnliches geschah mit dem
Aufstieg des Feudalismus während der Transformation des Jahres 1000. Wir
erwarten eine zunehmende Anzahl von Personen mit geringem Einkommen in
westlichen Ländern, die sich bisher auf Transferzahlungen vom Staat
verlassen haben, sich wohlhabenden Haushalten als Bedienstete
anzuschließen. Dennoch ist die reine Tatsache der Unmachbarkeit durch
einige Konkurrenten in einem Hobbes\textquotesingle schen Gemetzel (oder
Krieg jeder gegen jeden) nicht entscheidend. Wie Hirshleifer sagt: „Die
reine Tatsache von niedrigem Einkommen unter Anarchie, ... gibt für sich
genommen keinen klaren Hinweis darauf, was wahrscheinlich als nächstes
geschehen wird.`` \footnote{Ebenda, S. 37.}

\subsection{Die Beschaffenheit von
Vermögenswerten}\label{die-beschaffenheit-von-vermuxf6genswerten}

Noch ein weiterer interessanter Aspekt, der für die Aufrechterhaltung
der Anarchie notwendig ist, ist, dass Ressourcen „vorhersehbar und
verteidigbar`` sein müssen. In Hirshleifers Analyse ist die „Anarchie
ein gesellschaftliches Arrangement, in dem Rivalen darum kämpfen,
haltbare Ressourcen zu erobern und zu verteidigen``.\footnote{Ebenda, S.
  16.} Er definiert „Landgebiete oder bewegliche Kapitalgüter`` als
Inhalte von „haltbaren Ressourcen``.\footnote{Ebenda.} Im
Informationszeitalter könnten digitale Ressourcen zwar vorhersehbar
sein, aber sie werden nicht die Art von „haltbaren Ressourcen`` sein,
die Hirshleifer mit Territorialität und Anarchie verbindet. Tatsächlich
wäre die Eroberung des Territoriums, in dem eine Cyberbank eingetragen
ist, möglicherweise reine Zeitverschwendung, wenn digitales Geld mit
Lichtgeschwindigkeit an jedem Ort auf dem Planeten transferiert werden
kann. Nationalstaaten, die souveräne Individuen unterdrücken wollen,
müssten sowohl die Bankenhäfen der Welt als auch ihre Datenhäfen
gleichzeitig in Beschlag nehmen. Selbst dann könnten sie, wenn
verschlüsselte Systeme richtig konzipiert sind, nur bestimmte Summen
digitalen Geldes sabotieren oder zerstören, aber nicht beschlagnahmen.

Die Schlussfolgerung lautet, dass im kommenden Informationszeitalter,
die am meisten vorhersehbaren und verletzlichsten Vermögenswerte der
Reichen möglicherweise ihre physischen Personen sein könnten - mit
anderen Worten, ihr Leben. Dies erklärt unsere Angst vor Terrorismus im
Stile der Ludditen in den kommenden Jahrzehnten, von dem einiges
vielleicht verdeckt durch Provokateure gefördert wird, die im Auftrag
von Nationalstaaten handeln.

Langfristig bezweifeln wir jedoch, dass es den führenden Nationalstaaten
gelingen wird, diese souveränen Individuen zu unterdrücken. Vor allem
bestehende Staaten - besonders in Regionen mit geringem Kapital - werden
feststellen, dass sie mehr davon profitieren, diese souveränen
Einzelpersonen zu beherbergen, als Solidarität mit den nordatlantischen
Nationalstaaten zu wahren und die Heiligkeit des „internationalen``
Systems zu bewahren. Dass bankrotte, hoch besteuerte Wohlfahrtsstaaten
„ihre Staatsbürger`` und „ihr Kapital`` in „ihrem Land`` behalten
möchten, wird für hunderte von sich fragmentierenden Souveränitäten
anderswo kein überzeugendes Motiv sein.

Wir sagen dies trotz der Tatsache, dass es tausende multinationale
Organisationen gibt, die darauf abzielen, das Verhalten der
verschiedenen Souveränitäten der Welt zu beeinflussen. Es kann wenig
Zweifel daran bestehen, dass einige dieser Organisationen, wie die
Europäische Union und die Weltbank, einflussreich sind. Aber denken Sie
daran, dass die Rechtsbereiche, die souveräne Individuen willkommen
heißen, erheblich von ihrer Anwesenheit profitieren können. Selbst eine
starrköpfige Macht wie die Vereinigten Staaten, die durch aktuelle
Trends daran gebunden sind, energisch die Entstehung einer
Cyberwirtschaft außerhalb der Kontrolle der US-Regierung zu verhindern,
wird letztlich diejenigen Bewohner des Globus mit positiven
Bankguthaben, die keine Amerikaner sein wollen, nicht ausschließen
wollen. Dies ist insbesondere wahrscheinlich, da das Einkaufen
mittlerweile eine große Faszination für Reisende darstellt. Letztendlich
werden die Vereinigten Staaten oder Teile davon, wenn auch nach anderen,
aufgrund von Wettbewerbsdruck der Kommerzialisierung von Souveränität
beitreten.

\subsection{Nachfrage schafft Angebot}\label{nachfrage-schafft-angebot}

Dieser Druck wird zu Beginn in den Nationalstaaten mit den schwächsten
Bilanzen am stärksten zu spüren sein. Unter den neuen
„Offshore``-Zentren befinden sich Fragmente und Enklaven von aktuellen
Nationalstaaten, wie Kanada und Italien, die sich fast sicherlich lange
vor Ende des ersten Viertels des einundzwanzigsten Jahrhunderts auflösen
werden. Die Entstehung eines globalen Marktes für erstklassige,
kostenoptimierte Gerichtsbarkeiten wird dazu beitragen, diese ins Leben
zu rufen. Wie im regulären Handel werden kleine Mitbewerber beweglicher
und besser in der Lage sein, sich zu behaupten. Eine dünn besiedelte
Gerichtsbarkeit kann sich leichter so strukturieren, dass sie effizient
arbeitet.

Die Informationselite wird hochwertige Schutzverträge zu einem
vernünftigen Preis suchen. Auch wenn diese Gebühr weit unter dem liegt,
was erforderlich wäre, um einen spürbaren Nutzen für die gesamte
Bevölkerung von Nationalstaaten, wie sie jetzt strukturiert sind, mit
zehn Millionen bis hundert Millionen Bürgern, umzuverteilen, wäre dies
schon in einer Gerichtsbarkeit mit einer Bevölkerung von Zehntausenden
oder Hunderttausenden nicht unbedeutend. Die Steuerzahlungen und anderen
wirtschaftlichen Vorteile, die durch die Anwesenheit einer geringen
Anzahl von außerordentlich reichen Personen entstehen, deuten auf einen
wesentlich höheren Pro-Kopf-Nutzen in einem Gebiet mit kleiner
Bevölkerung im Vergleich zu einem mit großer Bevölkerung hin.

Da es praktisch unerheblich sein wird, wo ein Geschäft seinen Sitz hat,
außer im rein negativen Sinne, dass manche Adressen höhere
Verpflichtungen bedeuten als andere, wird es für kleine
Gerichtsbarkeiten leichter sein, kommerziell erfolgreiche Bedingungen
für den Schutz festzulegen. Daher werden Gerichtsbarkeiten mit kleinen
Bevölkerungen einen entschiedenen Vorteil bei der Formulierung einer,
für souveräne Individuen attraktiven, Fiskalpolitik genießen.

Wir glauben, dass das Zeitalter der Nationalstaaten vorbei ist, doch das
bedeutet nicht, dass die Anziehungskraft des Nationalismus als
emotionaler Einfluss auf den Menschen sofort verschwunden sein wird. Als
Ideologie ist der Nationalismus gut positioniert, um universelle
emotionale Bedürfnisse anzusprechen. Wir alle haben Momente der
Ehrfurcht erlebt, wie wenn man zum ersten Mal einen riesigen Wasserfall
sieht oder erstmals am Eingang einer großen Kathedrale steht. Wir haben
alle das Gefühl der Zugehörigkeit erlebt, sei es auf einer familiären
Weihnachtsfeier oder als Mitglied einer erfolgreichen Mannschaft in
irgendeinem Sport. Die menschliche Kultur verlangt eine Reaktion auf
beide dieser starken Emotionen. Wir werden erleuchtet von der
historischen Kultur unseres eigenen Landes, die wiederum Teil der
größeren Kultur der Menschheit ist. Wir finden Trost in dem Wissen, dass
wir zu einer kulturellen Gruppe gehören, die uns sowohl ein Gefühl der
Teilnahme als auch der Identität gibt.

Die Auswirkungen dieser kulturellen Symbole können den stärksten
emotionalen Effekt haben. Die amerikanischen Assoziationen mit der
Flagge, der Nationalhymne oder dem Familienfest an Thanksgiving, die
englische Verbindung zur Monarchie oder Cricket - all dies hat eine
echte Wirkung auf die Vorstellungskraft der Amerikaner und Engländer,
eine Wirkung, die durch Wiederholung verstärkt wird und tief ins
Unterbewusstsein vordringt. Solche Symbole helfen uns zu verstehen, was
für Menschen wir sind und erinnern uns an eine nationale Kultur. Als
Anti-Vietnamkriegsdemonstranten den Rest der Vereinigten Staaten
schockieren wollten, verbrannten sie die Flagge. Entfremdete Engländer
griffen die Monarchie an und sind sogar dafür bekannt, dass sie Löcher
in Cricketfelder graben.

Diese Auslöser sind oberflächlich, aber nicht unwichtig. Sie sind die
Assoziationen, für die wir gelernt haben zu brennen. Welcher Wechsel in
den megapolitischen Bedingungen oder sich daraus ergebende Änderungen in
den Institutionen auch immer eintreten, wahrscheinlich werden sie
weiterhin von Bedeutung für die Vorstellungen der Menschen bleiben, die,
so wie wir, im zwanzigsten Jahrhundert groß geworden sind.

\bookmarksetup{startatroot}

\chapter{DIE DÄMMERUNG DER
DEMOKRATIE}\label{die-duxe4mmerung-der-demokratie}

\begin{quote}
„Demokratische politische Systeme sind in historischer Hinsicht eine
neuere Angelegenheit. Sie hatten eine kurze Existenz in Griechenland und
Rom, bevor sie im 18. Jahrhundert, vor weniger als 200 Jahren, erneut
aufgetaucht sind. ~\ldots{} Ein Zyklus der Ablehnung könnte nun wieder
begonnen haben.`` \footnote{John Dunn, \emph{Western Political Theory in
  the Face of the Future}, Cambridge, Eng.: Cambridge University Press.
  1979, S. 2.} - William Pfaff
\end{quote}

Es ist kein Geheimnis, dass Demokratie in der Geschichte der Regierungen
relativ selten und vergänglich war. In den antiken und modernen Zeiten,
in denen sich die Demokratie durchgesetzt hat, hing ihr Erfolg von
megapolitischen Bedingungen ab, die die militärische Macht und Bedeutung
der Massen stärkten. Der Historiker Carroll Quigley hat diese
Charakteristika in seinem Werk „Waffensysteme und politische
Stabilität`` analysiert.\footnote{Carroll Quigley, \emph{Weapons Systems
  and Political Stability} (Washington, D.C.: University Press of
  America, 1983).}

Miteinbezogen wurde unter anderem Folgendes:

\begin{enumerate}
\def\labelenumi{\arabic{enumi}.}
\item
  \textbf{Günstige und weit verbreitete Waffen.} Demokratie neigt dazu,
  zu gedeihen, wenn die Kosten für den Kauf nützlicher Waffen niedrig
  sind.
\item
  \textbf{Waffen, die effektiv von Laien genutzt werden können.} Die
  Wahrscheinlichkeit von Demokratie steigt, wenn jeder in der Lage ist,
  effektive Waffen ohne langwierige Ausbildung zu benutzen.
\item
  \textbf{Ein militärischer Vorteil für eine große Anzahl von
  Fußsoldaten im Kampf.} Wie Quigley herausstellt: „Phasen der Dominanz
  der Infanterie waren Zeiten, in denen die politische Macht innerhalb
  der Gesellschaft stärker verteilt war und die Demokratie eine bessere
  Chance hatte, sich durchzusetzen.`` \footnote{Ebenda, S. 56.}
\end{enumerate}

Dies ist kaum ein vollständiger Katalog der Bedingungen, unter denen
Demokratie existieren kann. Wenn dies der Fall wäre, wäre die Demokratie
nicht zum triumphierenden System am Ende des zwanzigsten Jahrhunderts
geworden. Waffen waren zum Ende der Industrieära vermutlich teurer als
je zuvor. Und viele der wirksamsten Waffen erforderten definitiv
Spezialisten, um effektiv eingesetzt zu werden. Darüber hinaus bewies
der Golfkrieg zwischen den Vereinigten Staaten, ihren Verbündeten und
dem Irak, wie verwundbar große Infanterieeinheiten sind, selbst wenn sie
in Schützengräben und Befestigungen verschanzt sind. Warum also schien
die Demokratie unter diesen Bedingungen zu florieren, als das zwanzigste
Jahrhundert zu Ende ging?

\section{DEMOKRATIE, DER ZWILLINGBRUDER DES
KOMMUNISMUS?}\label{demokratie-der-zwillingbruder-des-kommunismus}

Wir haben in Kapitel 5 eine paradoxe Erklärung angeboten, nämlich dass
die Demokratie als Zwillingsbruder des Kommunismus gerade deshalb
aufblühte, weil sie eine ungehinderte Kontrolle der Ressourcen durch den
Staat ermöglichte. Dieser Schluss mag dem ‚gesunden Menschenverstand'
des industriellen Zeitalters albern erscheinen. Wir leugnen nicht, dass
demokratische Systeme und Kommunismus innerhalb der Grenzen der
industriellen Gesellschaft drastische Gegensätze darstellten. Doch aus
einer megapolitischen Perspektive betrachtet, wie sie wahrscheinlich
eher aus der Sicht des Informationszeitalters wahrgenommen wird, hatten
die beiden Systeme mehr gemeinsam, als man vermuten würde.

In einem Kontext, in dem Waffen grotesk teuer waren, wurde die
Demokratie zum Entscheidungsmechanismus, der die Kontrolle über
Ressourcen durch den Staat maximiert. Ähnlich wie der staatliche
Sozialismus, machten demokratische Systeme riesige Summen zur
Finanzierung einer massiven militärischen Infrastruktur verfügbar. Der
Unterschied bestand jedoch darin, dass der demokratische Wohlfahrtsstaat
sogar noch mehr Ressourcen in die Hände des Staates legte als es die
staatlich-sozialistischen Systeme tun konnten. Das ist durchaus
bemerkenswert, wenn man bedenkt, dass die staatlich-sozialistischen oder
kommunistischen Systeme praktisch jeden wertvollen Besitz beanspruchten.

Betrachtet man das Ganze nüchtern lediglich als Mechanismus zur
Ressourcensammlung, war der demokratische Staat dem staatlichen
Sozialismus als Rezept zur Bereicherung des Staates überlegen. Wie wir
bereits früher erklärten, stellte die Demokratie dem Militär erheblich
mehr Geld zur Verfügung, weil die Demokratie mit privatem Eigentum und
kapitalistischer Produktivität kompatibel war.

Das System des staatlichen Sozialismus basierte auf der Lehre, dass der
Staat alles besaß. Der demokratische Wohlfahrtsstaat hingegen stellte
zunächst begrenztere Ansprüche. Er gab vor, privates Eigentum
zuzulassen, wenn auch von einer bedingten Art, und nutzte so überlegene
Anreize zur Mobilisierung von Leistungen. Anstatt von Beginn an eine
umfassende Misswirtschaft zu betreiben, ließen demokratische Regierungen
im Westen den Individuen Eigentum besitzen und Reichtum ansammeln. Erst
nachdem der Reichtum geschaffen wurde, griffen die demokratischen
Nationalstaaten ein, um einen großen Teil davon wegzubesteuern.

Das Wort „groß`` sollte an dieser Stelle großgeschrieben werden. Zum
Beispiel stand im Jahr 1996 der lebenslange Bundessteuersatz in den
Vereinigten Staaten bei 73 Cent pro Dollar. Für Eigentümer von
Unternehmen, die ihr Einkommen durch Dividenden erhielten, lag der Satz
bei 83 Cent pro Dollar. Und für jede Person, die Geld an Enkelkinder
weitergeben oder verlassen wollte, betrug die Bundessteuerrate 99 Cent
pro Dollar. Wenn man die Bundes- und Landessteuern ebenfalls in Betracht
zieht, konfisziert eine demokratische Regierung auf allen Ebenen den
Löwenanteil jedes in den Vereinigten Staaten verdienten Dollars. Die
räuberischen Steuersätze machten den demokratischen Staat zu einem de
facto Partner mit einem Anteil von mindestens drei Vierteln bis neun
Zehnteln an allen Einnahmen. Das war sicherlich nicht dasselbe wie
Staatsozialismus. Aber es war eine enge Verwandtschaft.

Der demokratische Staat überlebte länger, weil er flexibler war und im
Vergleich zu Moskau oder Ost-Berlin über größere Mengen an Ressourcen
verfügte.

\subsection{„Ineffizienz, wo es darauf
ankam``}\label{ineffizienz-wo-es-darauf-ankam-1}

Wir haben die megapolitischen Vorteile der Demokratie als
Entscheidungsgrundlage für eine mächtige Regierung als „Ineffizienz, wo
es darauf ankam``, beschrieben. Verglichen mit dem Kommunismus war der
Wohlfahrtsstaat in der Tat ein viel effizienteres System. Aber im
Vergleich zu einer echten Laissez-Faire-Enklave wie Hongkong war der
Wohlfahrtsstaat ineffizient. Die Wachstumsraten in Hongkong waren
fabelhaft, aber ihre Überlegenheit lag genau in der Tatsache, dass der
Bewohner von Hongkong, und nicht die Regierung, in der Lage war, 85
Prozent der Vorteile des schnelleren Wachstums einzustecken.

Hongkong ist natürlich keine Demokratie. Tatsächlich ist es ein
geistiges Modell für die Art von Zuständigkeit, die wir im
Informationszeitalter zu sehen erwarten. Im Industriezeitalter musste
Hongkong keine Demokratie sein, da es die unangenehme Notwendigkeit
vermieden hat, Ressourcen zur Unterstützung einer mächtigen
militärischen Einrichtung zu sammeln. Hongkong wurde von außen
verteidigt, sodass es sich leisten konnte, eine wirklich freie
Wirtschaft zu erhalten.

Es war genau diese Fähigkeit, Ressourcen anzuhäufen, die der Demokratie
während der megapolitischen Bedingungen des Industriezeitalters die
Oberhand verlieh. Massendemokratie ging Hand in Hand mit dem
Industriekapitalismus. Wie Alvin Toffler sagte, ist die Massendemokratie
„der politische Ausdruck von Massenproduktion, Massenverteilung,
Massenkonsum, Massenbildung, Massenmedien, Massenunterhaltung, und allem
Anderen.`` \footnote{Zitiert in Kelly, ebenda, S. 46.}

Nun, da die Informationstechnologie die Massenproduktion verdrängt, ist
es logisch, den Niedergang der Massendemokratie zu erwarten. Das
entscheidende megapolitische Gebot, das die Massendemokratie im
Industriezeitalter zum Triumph führte, ist verschwunden. Es ist daher
nur eine Frage der Zeit, bis die Massendemokratie den Weg ihres Bruders,
des Kommunismus, geht.

\subsection{Massendemokratie unvereinbar mit dem
Informationszeitalter}\label{massendemokratie-unvereinbar-mit-dem-informationszeitalter}

Eine kurze Reflektion zeigt, dass die Technologie des
Informationszeitalters nicht zwangsläufig eine Massentechnologie ist. In
militärischer Hinsicht, wie wir angedeutet haben, eröffnet sie das
Potenzial für „intelligente Waffen`` und „Informationskrieg``, in dem
„Logikbomben`` zentralisierte Kommando- und Kontrollsysteme sabotieren
könnten. Die Informationstechnologie weist nicht nur eindeutig auf die
Perfektionierung von Waffen hin, die von Spezialisten bedient werden;
sie verringert auch die Entscheidungsfähigkeit der Kriegsführung und
verbessert die relative Position der Verteidigung. Die Mikrotechnologie
ermöglicht einen dramatischen Anstieg der militärischen Macht des
Einzelnen, während die Bedeutung von Masseninfanterieformationen
abnimmt. In einem Bericht der Rand Corporation an den
Verteidigungsminister heißt es: „Verbundene Netzwerke können nicht nur
von Staaten, sondern auch von nichtstaatlichen Akteuren, einschließlich
verstreuter Gruppen und sogar Einzelpersonen, angegriffen und gestört
werden.`` \footnote{Molander, et al.~\emph{Strategic Information
  Warfare}, ebenda, S. xv.} Darüber hinaus lässt dies darauf schließen,
dass Cyberkrieg das Potenzial der inhärenten Skalennachteile in großen
zentralisierten Systemen realisieren wird.

In den Worten der Rand-Experten: „Informationsbasierte Techniken machen
die geografische Entfernung irrelevant; Ziele auf dem amerikanischen
Festland sind genauso verwundbar wie Agenten im Kriegsgebiet.``
\footnote{Ebenda, S. xiv.} Früher war mit dem Wohnsitz innerhalb der
Grenzen großer, supermächtiger Nationalstaaten wie den Vereinigten
Staaten eine gewisse Sicherheit verbunden, im Informationszeitalter
könnte jedoch die Logik der Machtkonzentration umgekehrt werden. Peoria
mag weit von jeder potenziellen militärischen Front entfernt sein, aber
es wird nicht länger sicher vor Cyberattacken von nahezu jedem
potenziellen Gegner sein. Innerhalb der Grenzen einer Supermacht zu
leben, bedeutet, sich selbst ins Zentrum des Zieles zu setzen. Statt
sich zu vereinigen, könnten Orte ihre Sicherheit erhöhen, indem sie sich
aufspalten. Das Aufkommen des Cyberkrieges wird die Verwundbarkeit
zentralisierter Befehls- und Kontrollsysteme erhöhen, während die
Wettbewerbsfähigkeit verteilter Systeme gesteigert wird.

Die dadurch ausgelösten Rückkopplungsmechanismen könnten den
Dezentralisierungsprozess beschleunigen. Wie die Rand-Experten
vorschlagen, werden die Regierungen gezwungen sein, „die Nutzung neuer
Software-Verschlüsselungstechniken`` zu verstärken, um die Anfälligkeit
der Kommando- und Kontrollsysteme, die sich in der Spätphase des
Nationalstaates entwickelt haben, für Cyberangriffe zu verringern.
Dadurch werden diese hauptsächlich privatwirtschaftlichen Systeme
weitaus weniger sabotageanfällig und gleichzeitig wird die kommerzielle
Verbreitung der starken Verschlüsselung beschleunigt, was dazu beitragen
wird, sie von der staatlichen Vorherrschaft zu befreien. Auch dies wird
der Dezentralisierung Auftrieb geben. Wie wir bereits dargelegt haben,
wird dies die Verlagerung von Ressourcen in den Cyberspace weiter
vorantreiben, wo sie außerhalb der Reichweite der Politik liegen.

Letztendlich bedeutet dies das Ende der Massendemokratie, insbesondere
in ihrer vorherrschenden Form, der repräsentativen Fehlregierung, sei es
vom Bundes- oder vom Landtag.

\section{DIE MEGAPOLITIK DER
FEHLDARSTELLUNG}\label{die-megapolitik-der-fehldarstellung}

Wenn sich megapolitische Bedingungen so stark verändern, wie sie es
jetzt tun, verändert sich die Organisation der Regierung ebenfalls
unweigerlich. Tatsächlich war die Form der repräsentativen Regierung
traditionell ein Artefakt der Verteilung von roher Macht. Dies wird
durch die Tatsache gezeigt, dass Vertreter auf geographischer Basis
ausgewählt werden, und nicht auf irgendeine andere Weise.

Denken Sie mal darüber nach. Grundsätzlich wäre ein Parlament genauso
demokratisch, wenn seine Mitglieder entsprechend einer beliebigen
Aufteilung der Bevölkerung ausgewählt würden. Parlamentarische
Wahlbezirke oder kommunale Ebenen könnten auf Grundlage von Geburtstagen
oder sogar alphabetischen Wahlkreisen gebildet werden. Jeder, der am 1.
Januar geboren wurde, könnte aus einer Liste von Kandidaten wählen.
Diejenigen, die am 2. Januar geboren wurden, aus einer anderen. Oder
jede Person, deren Name mit ``Aa`` bis ``Af`` beginnt, könnte aus einer
Liste von Kandidaten wählen. Diejenigen, deren Namen mit ``Ag``
beginnen, würden aus einer anderen wählen. Und so weiter.

Ein solches System existiert aus mehreren Gründen derzeit nicht. Ein
erster und ausreichender Grund ist, dass es im 18. Jahrhundert
technologisch nicht umsetzbar war. Aber noch wichtiger ist die Tatsache,
dass Wahlkreise auf Grundlage von Geburtstagen oder dem Alphabet nicht
die Verteilung von roher Macht wiedergegeben oder sich ihr auch nur
angenähert hätten, die die Stimme zu dieser Zeit ausdrücken musste.
Personen, die nicht mehr als Geburtstage oder die ersten paar Buchstaben
ihrer Namen gemeinsam haben, waren und wären immer noch äußerst schwer
in irgendeiner kohärenten Machtbasis zu organisieren.

\subsection{Warum zählen geografische Querschnitte
mehr?}\label{warum-zuxe4hlen-geografische-querschnitte-mehr}

Die Abstimmung begann tatsächlich als ein Stellvertreter für einen
militärischen Konflikt. Und auch heute bleibt sie dies, wenn auch in
einer verdeckten Art und Weise. Solche Konflikte können entlang
geographischer Linien organisiert werden und seltener entlang von
Verwandtschafts- oder Religionslinien. Sie können jedoch nicht auf der
Grundlage von Geburtstagen oder Anfangsbuchstaben organisiert werden.
Auch können sie nicht effektiv nach Berufen organisiert werden, es sei
denn, diese Berufe sind in Erbgilden begrenzt, wie die Kasten in Indien,
oder regional gruppiert, wie die Landwirte in Iowa.

Der ganze Sinn der aktuellen Formeln der Vertretung besteht darin, dass
sie geographisch verankerte Interessen repräsentieren, anstatt entlang
einer anderen Dimension. Historisch gesehen war der Schlüssel zum
militärischen Erfolg die Kontrolle über Territorium. Alle militärischen
Bedrohungen haben sich lokal gebildet. Repräsentative Systeme sind
darauf ausgelegt, einen anderen Schauplatz für den Ausdruck dieser Macht
zu bieten. Die Tatsache, dass sie unweigerlich dazu neigen, lokale
Interessen zu fördern, ist ein Artefakt der Vertretungsformel.
Geographische Wahlkreise veranlassen Vertreter dazu, spezielle Gruppen
auf Kosten der allgemeinen Interessen zu bevorzugen, die alle Bewohner
eines Landes teilen.

\subsection{Neue Möglichkeiten
voraus}\label{neue-muxf6glichkeiten-voraus}

Wie Analysen von Public-Choice-Ökonomen gezeigt haben, haben scheinbar
geringfügige Veränderungen in der Form, in der eine Wahl strukturiert
ist, oder in der Art und Weise, wie die Stimmen berechnet werden, große
und vorhersehbare Auswirkungen auf das Ergebnis.\footnote{Dennis C.
  Mueller, \emph{Public Choice}, Vol. 2 (Cambridge: Cambridge University
  Press, 1989), S. 43-226.} Aus diesem Grund müssen ernsthafte Studenten
von Politikwissenschaften heute auch ernsthafte Studenten von
Verfassungen sein. Und es ist eine der Überlegungen, die uns dazu
veranlasst haben, über Verfassungen hinaus auf die ultimative
Metaverfassung zu schauen, die durch die vorherrschenden megapolitischen
Faktoren eines bestimmten Umfelds bestimmt wird.

Technologischer Wandel hat bereits einige der Grundlagen dafür
beseitigt, dass die Wahl auf geographische Wahlkreise begrenzt ist. Als
die modernen repräsentativen Systeme im achtzehnten und neunzehnten
Jahrhundert entstanden, waren fast alle Kommunikationen lokal. Die
meisten Menschen lebten und starben innerhalb weniger Kilometer von
ihrem Geburtsort, und ihr gesamter Handel und ihre Kommunikation wurden
lokal abgewickelt. Heute gibt es weltweite Sofortkommunikation. Man kann
mit jemandem fünftausend Kilometer entfernt nahezu genauso leicht
Geschäfte machen wie mit einem Nachbarn. In zunehmendem Maße überwindet
die Wirtschaft geographische Einschränkungen. Die Gesellschaft ist viel
mobiler geworden.

Und so ist es auch mit dem Reichtum im Informationszeitalter. Im
Unterschied zu einem Stahlwerk kann ein Computerprogramm kaum dem
hiesigen politischen Prozess zum Opfer fallen. Ein Stahlwerk kann nicht
einfach umziehen, wenn Gesetzgeber sich entscheiden, es zu besteuern
oder seine Besitzer zu regulieren. Ein Computerprogramm hingegen kann
per Modem mit Lichtgeschwindigkeit rund um die Welt übertragen werden.
Der Besitzer kann seinen Laptop packen und losfliegen. Auch das
untergräbt die megapolitischen Grundlagen von geografischen Wahlkreisen.

Ein wesentliches Problem, das alle repräsentativen Demokratiesysteme in
Anbetracht unserer Analyse teilen, ist, dass ihre geografischen
Wahlkreise zwangsläufig die etablierten Interessen von Unternehmen des
Industriezeitalters überrepräsentieren. Die „Verlierer`` oder
„Zurückgebliebenen`` sind perfekte Wähler, geografisch konzentriert und
politisch bedürftig. Die Geschichte der Industriedemokratie bestätigt
dies. Die „Gewinner`` der neuen Industrien waren in den
gesetzgeberischen Beratungen selbst in der Hochphase des
Industriezeitalters in den 1930er Jahren chronisch
unterrepräsentiert.\footnote{Michael A. Bernstein, \emph{The Great
  Depression: Delayed Recovery and Economic Change in America},
  1929-1939 (Cambridge: Cambridge University Press, 1987).} Die Tendenz
der Politiker, die bestehenden, etablierten Mitbewerber zu vertreten und
nicht die neu entstehenden Unternehmen oder die potenziellen Kunden
dieser Unternehmen, ist wahrscheinlich ein inhärentes Merkmal der
repräsentativen Regierung. Wie Mancur Olson in \emph{The Rise and
Decline of Nations} argumentierte, neigen langjährige Branchen dazu,
effektivere „Verteilungskoalitionen`` zu entwickeln, um Lobbyarbeit zu
betreiben und um politische Beute zu kämpfen.\footnote{Mancur Olson,
  \emph{The Rise and Decline of Nations: Economic Growth, Stagflation,
  and Social Rigidities} (New Haven: Yale University Press, 1982).}

In der Wirtschaft des Informationszeitalters verschärft sich dieses
Prinzip ins Unermessliche. Die kreativeren Teilnehmer an der neuen
Wirtschaft sind geografisch verteilt. Daher ist es unwahrscheinlich,
dass sie eine ausreichende Konzentration bilden, um die Aufmerksamkeit
von Gesetzgebern zu erregen, so wie es die Lachsfischer in Schottland
oder die Weizenbauern in Saskatchewan tun. Tatsächlich ist es
unwahrscheinlich, dass viele der dynamischen Persönlichkeiten der neuen
Wirtschaft Bürger selbst der umfassendsten Gerichtsbarkeit sind. Daher
werden sie kaum eine „Stimme`` in den legislativen Beratungen von
repräsentativen Demokratien haben. Als anschauliches Beispiel betrachten
Sie die fragwürdigen Bemühungen amerikanischer
Mathematik-Promovierender, ausländische Mathematiker daran zu hindern,
Arbeitsplätze in den Vereinigten Staaten anzunehmen.\footnote{Michael M.
  Phillips, \emph{Math Ph.D.s Add to Anti-Foreigner Wave: Scholars
  Facing High Jobless Rate Seek Immigration Curbs}, Wall Street Journal,
  4. September 1996, S. A2.} Ihre fremdenfeindlichen Forderungen an den
Kongress, die Arbeitgeber daran hindern sollen, Mitarbeiter auf der
Grundlage ihrer Verdienste einzustellen, werden mit großer
Wahrscheinlichkeit Gehör finden. Die antiquierte geografische
Vertretung, die aus dem Industriezeitalter übriggeblieben ist, nimmt
keine Rücksicht auf ausländische Mathematiker oder andere wichtige
Wohlstandsbringer, die keine Wähler sind.

„Warum glauben Menschen an die Legitimität demokratischer Institutionen?
Die Beantwortung dieser Frage ist fast ebenso schwierig wie die
Erklärung dafür, warum Menschen an bestimmte religiöse Dogmen glauben.
Denn wie bei religiösen Glaubensvorstellungen variiert das Verständnis,
der Skeptizismus und der Glaube stark innerhalb der Gesellschaft und
über die Zeit hinweg.`` \footnote{Juan J. Linz and Alfred Stepan, eds.,
  \emph{The Breakdown of Democratic Regimes} (Baltimore, Md.: The Johns
  Hopkins University Press, 1978), S. 18.} - Juan J. Linz

Wenige haben begonnen, auf konzertierte Weise über die Folgen des
technologischen Wandels nachzudenken, der den Industrialismus untergräbt
und die Einkommensverteilungen verändert. Offensichtlich wird die
Demokratie wahrscheinlich nicht viel mehr als ein Rezept für
legalisierten Parasitismus sein, wenn Einkommen so stark divergieren,
wie sie es in der Informationswirtschaft könnten. Noch weniger Menschen
haben die angedeutete Unverträglichkeit zwischen einigen Institutionen
der Industrieregierung und der Megapolitik der postindustriellen
Gesellschaft bemerkt. Ob diese Widersprüche nun explizit anerkannt
werden oder nicht, ihre Auswirkungen werden immer offensichtlicher, wenn
Beispiele für politisches Scheitern sich weltweit häufen.
Regierungsinstitutionen, die in der Moderne entstanden sind, spiegeln
die megapolitischen Bedingungen von vor einem oder mehreren
Jahrhunderten wider. Sie überlebten den Übergang von der
Agrargesellschaft zum urbanen Industrialismus. Aber das
Informationszeitalter könnte neue Mechanismen der Repräsentation
erfordern, um chronische Dysfunktion und sogar einen Zusammenbruch im
Stil der Sowjetunion zu vermeiden.

Man kann erwarten, dass Krisen der Fehlregierung in vielen Ländern
auftreten, da politische Versprechen entwertet werden und den
Regierungen der Kredit und die institutionelle Unterstützung ausgehen.
Letztlich werden neue institutionelle Formen entstehen müssen, die in
der Lage sind, die Freiheit unter den neuen technologischen Bedingungen
zu bewahren und gleichzeitig die gemeinsamen Interessen der Individuen
zum Ausdruck und zum Leben zu bringen.

All dies deutet auf das Ende der Massendemokratie hin, wie wir sie im
zwanzigsten Jahrhundert kannten. Die Frage ist: Was wird an ihre Stelle
treten? Wenn die einzige Alternative zur Massendemokratie eine Diktatur
wäre, in der der Einzelne kein Mitspracherecht über sein Schicksal hat,
könnte man versucht sein, sich der „Revolte gegen die Zukunft`` der
Neo-Ludditen anzuschließen.

\subsection{Neue Institutionen}\label{neue-institutionen}

Glücklicherweise ist die Diktatur jedoch nicht die einzige Alternative
zur Massendemokratie. Informationstechnologie ermöglicht
Wahlmöglichkeiten. Statt kollektiver Entscheidungen im eingeschränkten
Kontext der „Massenproduktion, Massenkonsum, Massenbildung,
Massenmedien, Massenunterhaltung und all dem Anderen``, wird die
Informationstechnologie echte, kundenspezifische
Souveränitätsdienstleistungen ermöglichen. Dies wird möglich sein, weil
es nicht mehr zwingend notwendig sein wird, in großem Maßstab zu
arbeiten. Wir glauben, dass die Technologie des Informationszeitalters
neue Formen des Regierens hervorbringen wird - so wie die
landwirtschaftliche Revolution und später das Industriezeitalter ihre
eigenen, unverwechselbaren Formen der sozialen Organisation
hervorgebracht haben.

Wie könnten solche neuen Institutionen aussehen? Um das zu verstehen,
müssen Sie alles vergessen, was Sie in falsch benannten,
„politikwissenschaftlichen`` Texten lesen. Die neuen Institutionen des
Regierens für das Informationszeitalter werden die Grenzen des
herkömmlichen Denkens überschreiten. Die Entwicklung hin zu solchen
Institutionen des neuen Zeitalters hat bereits begonnen. Es handelt sich
um wenig beachtete Improvisationen zur Umstrukturierung unzureichend
genutzter Güter, nämlich der Vorteile der Souveränität. Die
Nationalstaaten der Welt, die ängstlich auf die Sezessionsbewegungen und
die mächtigen Kräfte der Dezentralisierung blicken, haben sich
zusammengetan, um das stärkste Kartell zu bilden, das sie durchsetzen
können. Während die Zahl der neuen Staaten in der Welt in den 1990er
Jahren zugenommen hat, geschah dies hauptsächlich in zwei Gruppen, dank
des Zusammenbruchs der multiethnischen kommunistischen Diktaturen in der
ehemaligen Sowjetunion und Jugoslawien. Es ist bemerkenswert, dass
andere führende Nationalstaaten, einschließlich der Vereinigten Staaten,
sich bemühten, die Sowjetunion so lange wie möglich zu erhalten. Und nur
wenige Regierungen begrüßten den Zerfall Jugoslawiens. Die
Unabhängigkeit der ehemaligen jugoslawischen Republiken wurde erst
anerkannt, nachdem die Sezessionisten die Kontrolle errungen hatten, die
sie durch eigene militärische Anstrengungen durchsetzen konnten. Die
führenden Mächte gaben sich damit zufrieden, dass unbewaffnete oder
schlecht bewaffnete Separatisten von ihren serbischen Peinigern
abgeschlachtet wurden. Selbst das ferne China, ein mächtiger
Nationalstaat, der kein unmittelbares Interesse an der Erhaltung des
jugoslawischen Rumpfes hatte, widersetzte sich energisch den Bemühungen
um die Selbstbestimmung der unterdrückten ethnischen Albaner im Kosovo.
Ironischerweise ist es wahrscheinlicher, dass dieser Fetisch der
Grenzbefestigung den Weg zu einer zersplitterten Souveränität diktiert,
als dass er eine Dezentralisierung tatsächlich verhindert. Der
erbitterte Widerstand schwacher Nationalstaaten auf der ganzen Welt
gegen offene Sezession und politische Aufspaltung macht anerkannte
Souveränität zu einer wertvollen Form transzendentalen Kapitals, das von
den Staaten, die es besitzen, freiwillig fragmentiert und untervermietet
werden kann.

Ein Beispiel dafür, wie Souveränität freiwillig fragmentiert werden
kann, um eine im Wesentlichen private, steuerfreie Gerichtsbarkeit zu
schaffen, ist die Agulhas Bay Concession Free Zone, die fünfzig
Quadratkilometer der Insel Príncipe vor der Küste Westafrikas umfasst.
Obwohl das Gebiet innerhalb der Grenzen der Demokratischen Republik Sao
Tomé und Príncipe bleibt, ist die Verwaltung der Zone privatisiert. Die
Verwaltung wird durch einen Vertrag geregelt, der von der WADCO, der
West African Development Corporation Ltd.~verwaltet wird, einem privaten
Unternehmen mit Sitz in Südafrika. Die Amtssprache in der Zone ist nicht
die offizielle toméanische Sprache Portugiesisch, sondern Englisch. Die
offizielle Währung ist nicht das toméanische Monopol-Geld, die Dobra,
sondern das Geld der Welt, der US-Dollar. Für die Sicherheit sorgen
nicht die Polizeikräfte der Demokratischen Republik São Tomé und
Príncipe, sondern private Polizisten, die von der WADCO beschäftigt
werden. Das são-toméische Handelsrecht ist auf Handelsgeschäfte
innerhalb der Zone nicht anwendbar und die são-toméischen Gerichte sind
nicht zuständig. Alle Streitigkeiten müssen durch ein transnationales
Schiedsverfahren nach den Pariser ICC-Regeln beigelegt werden. Abgesehen
von einigen streng geregelten und unbedeutenden Ausnahmen gelten in der
Zone weder são-toméische Steuern noch offizielle Monopole. Die
Telekommunikation zum Beispiel ist innerhalb der Zone automatisch
dereguliert. Vorbehaltlich der Zahlung des Pachtzinses und der
Einhaltung anderer Konzessionsbedingungen ist WADCO berechtigt, den
Pachtvertrag über die private, fragmentierte Souveränität automatisch
und wiederholt um fünfzig Jahre zu verlängern, beginnend mit dem ersten
Verlängerungsdatum im Jahr 2047.

Was WADCO in Sao Tomé und Principe erreicht hat, können und werden
andere in vielen verschiedenen Ländern nachahmen. Einer der wahren
Pioniere des 21. Jahrhunderts, Joaquin Aguirre, hat eine ähnliche Zone
privater Souveränität in der Central Aguirre Portuaria im Osten
Boliviens geschaffen. Aguirre, ein Multimillionär, Schriftsteller und
Erfinder, Mitbegründer der Vereinten Nationen und ehemaliger Senator der
bolivianischen Republik, ist in mehrfacher Hinsicht ein Pionier. Ein
halbes Jahrhundert, nachdem er die Vereinten Nationen mitbegründet hat,
ist Aguirre nun der Prototyp des souveränen Individuums des 21.
Jahrhunderts. Seine Zona Franca, die frei von den meisten bolivianischen
Steuern, Zöllen und behördlichen Auflagen ist, weist den Weg zu der
neuen Form des privatisierten Stadtstaates, den erfolgreiche
Einzelpersonen im Informationszeitalter zunehmend erreichen werden. Es
zeigt auch eindeutig, dass das Leben der Massen, das von den Apologeten
von großen Regierungen so oft gepriesen wird, durch die wirtschaftliche
Entwicklung, die durch Freihandelszonen wie die von Señor Aguirre ins
Leben gerufen und katalysiert wird, dramatisch verbessert werden kann.
Im Laufe der Zeit wird sich die Zahl der De-facto-Stadtstaaten auf der
Welt erheblich vervielfachen. Wenn Sie als Einzelperson eine
ausreichende finanzielle Unabhängigkeit erreichen, werden Sie in der
Lage sein, die endgültige Unabhängigkeit zu erlangen, wie Joaquin
Aguirre. Für den Fall, dass Ihnen kein anderes Stück zersplitterter,
kommerzialisierter Souveränität eine bequeme Heimat bietet, können Sie
Ihren eigenen proprietären Ministaat gründen, der so unabhängig ist wie
jedes Herzogtum des Mittelalters. Anstatt mit Demagogen und politischen
Schreiberlingen Tauziehen zu spielen, um zu verhindern, dass Ihr
Vermögen den vielen zeternden Händen der Massendemokratie entrissen und
unter ihnen aufgeteilt wird, können Sie Ihr eigenes privates Reich der
Herrschaft errichten.

Der dramatische Phasenwechsel von der Massendemokratie zur ultimativen
Form der Selbstverwaltung, der individuellen Souveränität, muss weder
einen radikalen Wandel der öffentlichen Meinung noch ein wundersames
Votum enttäuschter Wähler voraussetzen, die sich für die Abschaffung der
Massendemokratie entscheiden. Eine solche Revolution kann beginnen, ja
sie hat bereits begonnen, unsichtbar, mit der Verpachtung von
Hoheitsgebieten zur Nutzung als steuerfreie Zonen, „Zona Francas`` und
Freihäfen. Zu gegebener Zeit wird die Souveränität immer wieder
zersplittert, bis sie so zerklüftet ist, dass eine weitere Aufteilung
keinen ausreichenden Wert mehr hätte, um die Transaktionskosten der
Dezentralisierung auszugleichen. Angesichts des Moore'schen Gesetzes und
des Gilder'schen Korollariums, wonach sich die Bandbreite jedes Jahr
verdreifacht, gibt es derzeit keinen Grund, ein baldiges Ende des
Dezentralisierungstrends vorauszusehen. Im Gegenteil, wir gehen davon
aus, dass die scheinbar solide Macht der Nationalstaaten, die sich
derzeit der Massendemokratie verschrieben hat, in Zehntausende von
Fragmenten in ein System zersplittern wird, das mehr an das Mittelalter
als an das moderne Industriezeitalter erinnert.

Zu gegebener Zeit werden selbst Nationalstaaten, die noch über
Rumpfinstitutionen der Massendemokratie verfügen, einen bedeutenden
Politikwandel erleben, um sich den neuen metakonstitutionellen
Realitäten anzupassen. Wie William Keech, ein treuer Verfechter der
Demokratie, in \emph{Economic Politics: The Costs of Democracy}
argumentiert: „Die Menschen lernen zu wollen, das bekommen zu können,
was sie sehen, aber sie können auch ihre Meinung ändern, wenn sie sehen,
dass ihnen nicht gefällt, was sie wollten und was sie bekommen haben.``
\footnote{William Keech, \emph{Economic Politics: The Costs of
  Democracy}. Cambridge: Cambridge University Press, 1995, S. 221.} Mit
anderen Worten: Die Tatsache, dass die Massendemokratie mit
konventionellen Institutionen der repräsentativen Regierung am Ende des
zwanzigsten Jahrhunderts überall bejubelt wird, könnte ein
„Verkaufssignal`` sein. Sie ist keineswegs eine Garantie dafür, dass
solche Entscheidungsregeln den Test der Zeit bestehen werden, nicht
einmal unter ihren eigenen Bedingungen. Denken Sie daran, dass es
außerhalb der Politik kaum Anzeichen dafür gibt, dass Führungskräfte,
Verwaltungsangestellte, Trainer oder andere professionelle
Führungskräfte auf demokratische Weise ausgewählt werden. Im Gegenteil,
die erfolgreichsten Führungspersönlichkeiten werden regelmäßig von den
Eigentümern im Rahmen von Auswahlverfahren eingestellt, bei denen
diejenigen, deren Interessen am stärksten im Vordergrund stehen, ein
ungleiches und unverhältnismäßig großes Mitspracherecht bei der
Bestimmung des Ergebnisses haben. Wenn die demokratische Auswahl
wirklich eine überlegene Methode wäre, um kompetente Führungskräfte zu
finden, würde man erwarten, dass sie als universelle Entscheidungsregel
gilt. Stattdessen ist sie fast ausschließlich auf den politischen
Bereich beschränkt. Kurz gesagt, die Annahme, dass die Erbringung
hoheitlicher Dienstleistungen durch die Dominanz demokratischer
Entscheidungsfindung behindert wird, ist nach den derzeitigen
Erkenntnissen vernünftiger als die gegenteilige Annahme, dass
Unternehmen und Wirtschaftsorganisationen darunter leiden, dass sie von
Führungskräften geleitet werden, die von den Eigentümern eingesetzt
werden, anstatt per Handzeichen.

Bis Mitte des 21. Jahrhunderts wird die Ausbreitung proprietärer
Gerichtsbarkeiten mit zersplitterter Souveränität wahrscheinlich die
Vorteile der proprietären Verwaltung endgültig demonstriert haben. Die
Wähler werden sehen, dass sie darunter leiden, dass ihnen die
Massendemokratie aufgebürdet wird. Daher werden sie, wie Professor Keech
vorschlägt, zu der Einsicht gelangen, dass die Vorteile, die sich aus
der Kontrolle der Regierung durch die Arbeitnehmer ergeben, durch deren
Kosten aufgewogen werden. Sie werden zu Reformen übergehen. Sogar die
Wähler in Europa und Nordamerika, die jetzt so stark gegen Reformen
eingestellt zu sein scheinen, könnten schließlich dafür stimmen, dass
ihre Regionen der Eigenverwaltung mehr entgegenkommen. Mehrheiten
könnten bereitwillig, ja sogar mit Freude die Farce der Politik
zugunsten einer eigenverantwortlichen Regierungsführung aufgeben, die
tatsächlich darauf abzielt, optimale Bedingungen für den Abschluss und
die Durchsetzung von Verträgen zu schaffen.

Sofern die Regierung mit ihren fadenscheinigen Ausstattungen überhaupt
überlebt, kann sie auf völlig neue Weise informiert werden. Irgendwo, in
irgendeinem Rechtssystem, irgendwann vor dem Jüngsten Tag, wird jemand
das Potential erkennen, das die Computertechnologie bietet, um eine
wirklich repräsentative Regierung zu ermöglichen. Das vermeintliche
Problem hoher Ausgaben für Wahlkampagnen und die zweifellos störenden
Dauerkampagnen könnten im Handstreich gelöst werden. Anstatt gewählt zu
werden, könnten Vertreter vollkommen zufällig durch Losverfahren
ausgewählt werden, mit einer hohen statistischen Wahrscheinlichkeit,
dass ihre Talente und Ansichten, denen der breiten Bevölkerung
entsprechen.

Dies wäre lediglich eine moderne Version des antiken griechischen
Systems der Auswahl durch Los. Wie E. S. Staveley in seiner
autoritativen Geschichte griechischer und römischer Wahlen und
Abstimmungen ausführlich beschreibt, wurden in Athen zahlreiche Ämter,
von den Magistraten bis zu den Archonten, durch Losentscheid als Ersatz
für Wahlen besetzt. Dies wurde geschickt erreicht, trotz mechanischer
Einschränkungen bei der Zufälligkeit der Chancen, durch die Verwendung
einer Losmaschine „oder, wie die Athener sie nannten, das Kleroterion.``
\footnote{E.S. Staveley, \emph{Greek and Roman Voting and Elections}
  (Ithaca, N.Y.: Cornell University Press, 1972), S. 62.}

Eine Reihe von schwarzen und weißen Bohnen wurden als zufällige
Zählmarken verwendet, um zu bestimmen, wer ausgewählt würde, um
verschiedene Ämter zu besetzen, sowie um „die Reihenfolge zu bestimmen,
in der die Stammesabschnitte im Rat ihre Runden als Prytanen machen
würden.`` \footnote{Ebenda, S. 65.} Die klassische Herkunft dieser Idee
könnte ihr zusätzliche Glaubwürdigkeit verleihen. Aber ihre
Hauptattraktivität besteht genau darin, dass sie die Nachteile der
Selbstauswahl in der Politik vermeiden würde. Sie würde statistisch
sicherstellen, dass weniger Anwälte und Egomanen das öffentliche
Geschäft monopolisieren.

Die Parlamente könnten aus echten Vertretern zusammengesetzt sein. Da
sie nicht durch eine Machtsuche zusammengeführt würden und sie ohnehin
nur eine sehr geringe Chance hätten, erneut per Losverfahren ausgewählt
zu werden, wären sie frei, die Regierungsgeschäfte zu führen und
Politiken auf der Grundlage einer rationalen Analyse der Angelegenheiten
zu formulieren.

\subsection{Direkte Provision}\label{direkte-provision}

Heutzutage haben Politiker, die darauf bedacht sind, Stimmen zu
optimieren, wenig Anreiz, Probleme kohärent zu analysieren. Es ist
deshalb kaum verwunderlich, dass ihre Erfolgsbilanz bei der eigentlichen
Problemlösung im Vergleich zu Unternehmern, Geschäftsführern und
Trainern von Sportteams, die nach ihrer Leistung belohnt werden, so
armselig ist. Eine leistungsbezogene Vergütung für Gesetzgeber würde
nicht jeden zufällig Ausgewählten so effizient machen wie Lee Kuan Yew.
Es gibt jedoch allen Grund zu der Annahme, dass die Leistung deutlich
verbessert würde, wenn das Gehalt der Gesetzgeber an eine objektive
Leistungsmaßnahme, wie beispielsweise das Wachstum des
Pro-Kopf-Einkommens nach Steuern, gekoppelt wäre. Wenn man sie nach
Leistung bezahlt, steigt die Wahrscheinlichkeit, dass sie Leistung
erbringen, um ein Tausendfaches.

Der Gewinn für die Gesellschaft durch Politiken, die das reale Einkommen
abzüglich Steuern verbessern, könnte enorm sein. Warum also nicht
Premierminister und Präsidenten sogar einen winzigen Anteil des Gewinns
zahlen, den ihre Politiken fördern? Die Mittel für solche Zahlungen
könnten durch eine kleine, unauffällige Steuer erhoben werden. Eine
solche Regelung würde die Gesellschaft von der Bedrohung befreien, der
sie heute durch ehrgeizige Männer mit besonderen politischen Fähigkeiten
wie Richard Nixon und Bill Clinton ausgesetzt ist.

\begin{quote}
„Sie brachten ihm Gold, Silber und Kleidung; aber der ‚Christ' verteilte
all diese Dinge an die Armen. Wenn Geschenke angeboten wurden,
verneigten sich er und seine Gefährtin und sprachen Gebete aus; aber
dann erhob er sich und befahl der Versammlung, ihn anzubeten. Später
organisierte er eine bewaffnete Bande, die er durch die Landschaft
führte, raubte Reisende, die sie auf dem Weg trafen, aus. Aber auch hier
war sein Ehrgeiz nicht, reich zu werden, sondern angebetet zu werden. Er
verteilte alle Beute an diejenigen, die nichts hatten, einschließlich,
so darf man annehmen, seiner eigenen Anhänger.`` \footnote{Norman Cohn,
  \emph{The Pursuit of the Millennium} (Oxford: Oxford University Press,
  1970), S. 41.} Norman Cohn
\end{quote}

\subsection{Messianische
Persönlichkeiten}\label{messianische-persuxf6nlichkeiten}

Es wurde zu wenig beachtet, dass die Wahlpolitik gestörte, messianische
Persönlichkeiten in Machtpositionen lockt. Solche Personen existierten
und stellten oft ernsthafte Bedrohungen für die soziale Ordnung dar,
sogar in Agrargemeinschaften vor dem Aufkommen demokratischer
politischer Systeme. Wenn man die Karrieren von Eudo de Stell, dem
Bretonischen Christus, Adelbert im 8. Jahrhundert, Eon im 11.
Jahrhundert, Tanchelm von Antwerpen, Melchior Hoffman und Bernt Rothmann
und ihren Gleichgesinnten betrachtet, sticht einiges hervor. Je
offensichtlicher ihre politischen Talente zu sein scheinen, desto größer
scheint der Schaden zu sein, den sie verursacht haben. Da der Staat noch
nicht damit beschäftigt war, weit verbreitete systematische
Zwangsmaßnahmen zu organisieren, nahmen diese frühen Protopolitiker es
oft auf sich, zu rauben und zu plündern, um Geld zu beschaffen, das sie
an ihre Anhänger unter den Armen verteilen konnten.

\subsection{Protopolitiker in Aktion}\label{protopolitiker-in-aktion}

Die Geschichten ihrer Eskapaden vermitteln einen Eindruck von Talenten,
die ihrer Zeit voraus waren, als würde man von zwei Meter großen Männern
lesen, die einen Platz hoch und runter laufen, bevor Basketball erfunden
wurde. Heute, dank der NBA, verdienen außergewöhnlich groß gewachsene
Männer Millionen, indem sie dribbeln und dunken. Würde es Basketball
nicht mehr geben, würden sie wieder in die Nischen der Gesellschaft
zurückfallen und wahrscheinlich hauptsächlich als Attraktionen im Zirkus
und in Schaubuden auftreten.

Bevor die Politik erfunden wurde, wurden Demagogen zur nächsten
Annäherung von Politik gezogen, die die agrarische Welt zu bieten hatte:
Wanderpredigt. Sie hielten Reden vor Menschenmengen und versprachen wie
Politiker denen, die ihnen folgen würden, ein besseres Leben. Damals wie
heute waren die Armen die Hauptziele der Demagogen. Norman Cohns große
Geschichte der Jahrtausendbewegungen, die Suche nach dem Millennium,
erzählt die Karrieren zahlreicher messianischer Führer bevor es
Wahlumfragen gab. In seinen Beschreibungen ist es leicht, die starken
Ähnlichkeiten im Persönlichkeitstyp mit dem charismatischen Politiker
der Moderne zu erkennen.

\begin{quote}
Der Anführer hat - wie der Pharao und viele andere „göttliche Könige`` -
alle Attribute eines idealen Vaters: Er ist vollkommen weise, er ist
vollkommen gerecht, er schützt die Schwachen. Aber andererseits ist er
auch der Sohn, dessen Aufgabe es ist, die Welt zu verändern, der
Messias, der einen neuen Himmel und eine neue Erde schaffen soll und von
sich selbst sagen kann: „Sieh her, ich mache alles neu!{}`` Und sowohl
als Vater als auch als Sohn ist diese Figur kolossal, übermenschlich,
allmächtig. Ihm werden so viele übernatürliche Kräfte zugeschrieben,
dass man sich vorstellt, sie strömten wie Licht hervor. \ldots{} Darüber
hinaus besitzt der eschatologische Anführer durch die Fülle dieses
göttlichen Geistes einzigartige wundertätige Kräfte. Seine Armeen werden
stets triumphierend und siegreich sein. Seine Anwesenheit wird die Erde
Ernten hervorbringen lassen, seine Herrschaft wird ein Zeitalter
vollkommener Harmonie sein, wie die alte, korrupte Welt es nie gekannt
hat.

Dieses Bild war natürlich ein rein fantastisches, im Sinne, dass es
keinen Bezug zur wirklichen Natur und Fähigkeit eines Menschen hatte,
der jemals existierte oder existieren könnte. Nichtsdestotrotz war es
ein Bild, das auf einen lebenden Menschen projiziert werden konnte; und
es gab immer Männer, die mehr als bereit waren, eine solche Projektion
anzunehmen, die tatsächlich leidenschaftlich darauf brannten, als
unfehlbare, wundertätige Erlöser gesehen zu werden. \ldots{} Und das
Geheimnis der Vorherrschaft, die sie ausübten, lag weder in ihrer
Herkunft noch zu einem großen Teil in ihrer Bildung, sondern immer in
ihren Persönlichkeiten. Zeitgenössische Berichte über diese Messiasse
der Armen betonen häufig ihre Beredsamkeit, ihre beherrschende Haltung
und ihr persönliches Charisma. Vor allem hat man den Eindruck, dass
sogar, wenn einige dieser Männer möglicherweise bewusste Betrüger
gewesen sein könnten, die meisten von ihnen sich wirklich als
inkarnierte Götter sahen. . . . Und diese totale Überzeugung würde sich
leicht genug den Massen mitteilen, deren tiefstes Verlangen sich genau
nach so einem eschatologischen Erlöser sehnte.\footnote{Ebenda, S.
  84-85.}
\end{quote}

Obwohl dieser Abschnitt wunderbar prägnant die selbsternannten
Weltuntergangsretter beschreibt, die häufig die mittelalterliche
Gesellschaft aufwühlten, kann er nicht das volle Ausmaß von Cohns
meisterhafter Untersuchung wiedergeben. Man kann das gesamte Werk nicht
lesen, ohne in den Possen dieser Propheten die bekannten Eigenschaften
des modernen Demagogen zu erkennen: die Beredsamkeit, die „persönliche
Ausstrahlung``, die „messianischen Ansprüche`` und das wiederkehrende
Verlangen, als Volkstribun der Armen verehrt zu werden.

Der Hauptunterschied zwischen der Rezeption dieser Hochstapler durch die
mittelalterliche Gesellschaft und der durch die Demokratie am Ende des
20. Jahrhunderts besteht darin, dass solche Personen im Mittelalter in
der Regel hingerichtet wurden, während die moderne demokratische Politik
ihnen am Ende des 20. Jahrhunderts einen offenen Kanal bereitstellt, um
auf legitime Weise die Macht des Nationalstaats zu ergreifen.

Ein System, das routinemäßig die Kontrolle über die größten, tödlichsten
Unternehmen auf der Erde dem Gewinner von Beliebtheitswettbewerben
zwischen charismatischen Demagogen übergibt, wird auf lange Sicht
darunter leiden.

\subsection{Bezahlt die Führungskräfte dafür, gute Arbeit zu
leisten}\label{bezahlt-die-fuxfchrungskruxe4fte-dafuxfcr-gute-arbeit-zu-leisten}

Wie oben angedeutet, wäre es ein Leichtes, eine überlegene Methode zur
Gewinnung talentierter Führungskräfte für eine Organisation festzulegen:
die Einstellung von Führungskräften. Dies ist die Methode, die in
wettbewerbsfähigen Volkswirtschaften am häufigsten und erfolgreichsten
eingesetzt wird. Ein rationaler Auswahlprozess in Verbindung mit einer
konstruktiven Anreizstruktur zur Belohnung positiver Führung würde
fähige Leute an die Spitze der Regierung bringen. Es würde auch neue
Arten von Talenten mobilisieren, die sich sonst nicht für die Probleme
des Regierens interessieren würden.

Die talentiertesten Führungskräfte der Welt könnten dazu verleitet
werden, strauchelnde Regierungen zu übernehmen, wenn sie auf der
Grundlage der Ergebnisse vergütet würden, die sie tatsächlich für die
Gesellschaft erzielen. Ein Anführer, der in der Lage wäre, das reale
Einkommen in irgendeiner führenden westlichen Nation signifikant zu
steigern, könnte mit Recht weitaus mehr verdienen als Michael Eisner. In
einer besseren Welt wäre jeder erfolgreiche Regierungschef ein
Multimillionär.

\subsection{Elektronische
Volksentscheide}\label{elektronische-volksentscheide}

Eine offensichtliche Alternative zur repräsentativen Fehlregierung wäre
die Durchführung von elektronischen Volksabstimmungen, bei denen Bürger,
möglicherweise eine repräsentative Auswahl, die durch
manipulationssichere Auslosungen ausgewählt wurde, ihre Stimmen direkt
zu gesetzgeberischen Vorschlägen abgeben könnten. Computertechnologie
ermöglicht es, Entscheidungen demokratisch zu treffen, mit
elektronischen Volksabstimmungen. Volksabstimmungen könnten leicht mit
Zuweisungen verknüpft werden, um die Anzahl der Stimmen zu bestimmten
Fragen zu verringern. Grundsätzlich ist es für die Wählerschaft weniger
herausfordernd, politische Fragen zu verstehen, als zu versuchen,
Politiker zu durchschauen und die Beurteilungen dieser Politiker zu den
gleichen Fragen zu bewerten, ganz zu schweigen davon, zu wissen, was
diese Politiker tatsächlich tun würden, wenn sie ihr Amt antreten. Dies
ist besonders schwierig, da Politiker und ihre Teams zunehmend
geschickter darin werden, die Bilder, die sie der Öffentlichkeit
präsentieren, zu verpacken und zu manipulieren.

\section{KOMMERZIALISIERTE
SOUVERÄNITÄT}\label{kommerzialisierte-souveruxe4nituxe4t}

Wir erwarten, dass etwas Neues entsteht, um die Politik zu ersetzen.
Obwohl alle Möglichkeiten, die wir oben diskutiert haben, mit einigem
Vorteil erprobt werden könnten, erwarten wir nicht, dass die Politik
reformiert oder verbessert wird, sondern dass sie überholt und in den
meisten Aspekten aufgegeben wird. Damit meinen wir nicht, dass wir eine
Diktatur erwarten, sondern eine unternehmerische Regierung - die
Kommerzialisierung der Souveränität.

Im Gegensatz zur Diktatur oder sogar zur Demokratie wird die
kommerzialisierte Souveränität nicht die Wahlfreiheit einschränken. Sie
wird jedem Einzelnen mehr Spielraum bieten, seine Meinungen
auszudrücken. Und für diejenigen, die das Talent haben, es für sich zu
nutzen, wird die kommerzialisierte Souveränität mehr praktischen
Spielraum für Entscheidungsfindung und Selbstbestimmung erlauben als
jede Form von sozialer Organisation, die bisher existiert hat.

\subsection{Kundengerechte Regierung}\label{kundengerechte-regierung}

Damit dies nicht so nach 2000 klingt, sollte man bedenken, dass die
Mikrotechnik miniaturisiert und disaggregiert. Sie erleichtert die
Anpassung an Kundenwünsche statt an die Massenproduktion. Man kann nun
in ein Geschäft gehen und Jeans kaufen, die nach einem Muster
geschnitten sind, das an die eigenen Maße angepasst und dann auf der
anderen Seite der Welt genäht wird. Wenn neue Institutionen endlich
entstehen, um sich den neuen megapolitischen Realitäten des
Informationszeitalters anzupassen, werden Sie in der Lage sein, eine
Regierungsführung zu erhalten, die mindestens so gut auf Ihre
persönlichen Bedürfnisse und Geschmäcker zugeschnitten ist wie Jeans.

Ausgerechnet Alvin Toffler hat die Idee kritisiert, dass die
Informationstechnologie die Bürger zu Kunden machen könnte. Toffler
sagt, unserer Meinung nach falsch: „Das ist ein viel zu enges Modell. Ob
es uns gefällt oder nicht, es gibt eine Welt voller Religion und Gefühl,
die nicht einfach auf Vertragsbeziehungen reduziert werden kann.``
\footnote{Zitiert in Kelly, ebenda, S. 46.} Aus Gründen, die wir zuvor
erörtert haben, stimmen wir zu, dass es schwierig sein wird, „die Welt
nationalistischer Empfindungen`` auf „Vertragsbeziehungen`` zu
reduzieren. Aber das heißt nicht, dass es unmöglich ist, und schon gar
nicht, dass es eine schlechte Lösung wäre. Etwas weniger irrationale
Begeisterung für den Nationalismus könnte Millionen von Menschenleben
retten.

\subsection{„Eintritt, Austritt`` und
„Stimme``}\label{eintritt-austritt-und-stimme}

Natürlich ist die Kommerzialisierung der Souveränität ein unvertrautes
Konzept, anscheinend sogar für Alvin Toffler. Aber seine zentrale Idee -
der wirtschaftliche Ausdruck - ist für die Menschen, die am Ende des 20.
Jahrhunderts leben, alltäglich. In jeder auch nur teilweise freien
Wirtschaft können Verbraucher ihre Wünsche direkt durch den Kauf von
Dienstleistungen und Produkten zum Ausdruck bringen. Oder durch das
Entziehen ihrer Kundschaft. Wenn Sie mit einer Version eines Produkts
oder einem Anbieter eines Dienstes unzufrieden sind, können Sie Ihre
Unzufriedenheit durch den „Austritt`` direkt zum Ausdruck bringen. Mit
anderen Worten, Sie können Ihr Geschäft woandershin verlegen.

In den vorherigen Kapiteln haben wir analysiert, wie der Fortschritt der
Informationstechnologie es bald ermöglichen wird, dass Sie
Vermögenswerte im Cyberspace erstellen können, die nahezu sicher vor
räuberischen Überfällen von Nationalstaaten sind. Dadurch wird eine
\emph{de facto} metakonstitutionelle Anforderung geschaffen, dass die
Regierungen Ihnen tatsächlich einen zufriedenstellenden Service bieten,
bevor Sie ihre Rechnungen bezahlen. Warum? Weil die
Einkommensbesteuerung in der Praxis fast genauso freiwillig sein wird,
wie sie es in der Theorie sein sollte.

\subsection{Vermeidung von „schwerfälligen politischen
Kanälen``}\label{vermeidung-von-schwerfuxe4lligen-politischen-kanuxe4len}

Im Endeffekt, wenn sich die Informationstechnologie so entwickelt, wie
sie könnte, wird sie gewährleisten, dass Regierungen tatsächlich von
ihren Bürgern kontrolliert werden. Als Bürgerin oder Bürger haben Sie
zuerst hunderte, dann tausende von Möglichkeiten, Ihre Schutzkosten
direkt zu senken, indem Sie einen privaten Steuervertrag mit einem
Nationalstaat abschließen oder sich vollständig von Nationalstaaten zu
entstehenden Mini-Souveränitäten abwenden. Diese „Eintritts-`` und
„Austritts``-Optionen sind der wirtschaftliche Ausdruck Ihrer Wünsche
als Kunde. Mit den Füßen und dem Geld abzustimmen hat den großen
Vorteil, dass es zu Ergebnissen führt, die man sich wünscht.

Wie verhalten sich Ihre Einstiegs- und Ausstiegsmöglichkeiten als Kunde
zu den politischen Ausdrucksformen in der Demokratie? Personen, die mit
einem Produkt oder einer Dienstleistung unzufrieden werden, insbesondere
wenn diese von der Regierung angeboten oder stark reguliert wird, können
ihrer Meinung eine „Stimme`` verleihen, indem sie beispielweise Briefe
an den Präsidenten der Vereinigten Staaten schreiben oder versuchen, ein
Treffen mit ihrem Abgeordneten oder einem anderen geeigneten gewählten
Amtsträger zu erwirken. Manchmal haben solche Liebesbriefe Erfolg. Aber
nicht immer. Normalerweise nicht. Wenn sie zunächst keinen Erfolg haben,
können Personen, die ihre „Stimme`` für Veränderungen einsetzen möchten,
dann Demonstrationen organisieren, eine ganzseitige Anzeige in einer
Zeitung schalten oder sogar selbst ein öffentliches Amt anstreben.

Die politische Ausdrucksweise bietet zwar die Möglichkeit, sich zu
artikulieren und zu reden. Sie bringt jedoch den Nachteil mit sich, dass
man nur selten Zufriedenheit erlangen oder seine Position durch eigenes
Handeln verbessern kann. Wenn man mit einem minderwertigen Produkt oder
einer minderwertigen Dienstleistung der Regierung konfrontiert wird, ist
man gezwungen, so lange dafür zu bezahlen, bis man den gesamten
politischen Prozess davon überzeugen kann, seiner Forderung nach einer
Änderung nachzukommen.

In westlichen Ländern, und mittlerweile praktisch auf der ganzen Welt,
hat dies dazu geführt, dass die Notwendigkeit besteht, die
Mehrheitsunterstützung eines demokratischen politischen Systems zu
sichern. Die Anforderung, eine Mehrheit einzubeziehen, verursacht enorme
Transaktionskosten zwischen Ihnen und dem Erreichen dessen, was aller
Wahrscheinlichkeit nach ein relativ unkompliziertes und rationales Ziel
ist.

Milton Friedman diskutierte die Vorteile der wirtschaftlichen, im
Gegensatz zur politischen Ausdrucksweise, um seinen Vorschlag für
Bildungsgutscheine in \emph{Capitalism and Freedom} voranzubringen:

\begin{quote}
Eltern könnten ihre Ansichten über Schulen direkt ausdrücken, indem sie
ihre Kinder von einer Schule abmelden und sie zu einer anderen schicken,
in einem viel größeren Ausmaß als es jetzt möglich ist. Im Allgemeinen
können sie diesen Schritt zurzeit nur unternehmen, indem sie ihren
Wohnort wechseln. Ansonsten können sie ihre Meinungen nur durch
umständliche politische Kanäle äußern.\footnote{Milton Friedman,
  \emph{Capitalism and Freedom} (Chicago: University of Chicago Press,
  1962), p.91. Discussed by Hirschman, ebenda, S. 16-17.}
\end{quote}

Albert O. Hirschman, der sich als Parteigänger der Politik äußerte, nahm
Anstoß an Friedmans Vorliebe für den „Ausstieg als ‚direkte'
Möglichkeit, seine ablehnende Haltung gegenüber einer Organisation zum
Ausdruck zu bringen. Eine in Wirtschaftsfragen weniger geschulte Person
könnte naiverweise annehmen, dass der direkte Weg, seine Meinung zu
äußern, darin besteht, sie zu äußern!{}`` \footnote{Hirschman, ebenda,
  S. 17.}

Ob es direkter oder effektiver ist, Ihre Meinungen durch Marktmittel
auszudrücken, beispielsweise indem Sie Ihre Unterstützung als Kunde
äußern oder zurückziehen, oder über „schwerfällige politische Kanäle``,
ist eine komplexe und umstrittene Frage. Verschiedene Personen werden
sie auf unterschiedliche Weise beantworten. Für diejenigen, deren
Hauptengagement bei der politischen Meinungsäußerung darin besteht,
Leistungen auf Kosten anderer zu fordern, mag der Wechsel zur
wirtschaftlichen Ausdrucksform tatsächlich eine trostlose Alternative
zum Schreiben an einen Politiker sein, um mehr zu verlangen.

\subsection{Wirtschaftlicher Ausdruck und „reziproke
Sozialität``}\label{wirtschaftlicher-ausdruck-und-reziproke-sozialituxe4t}

Für diejenigen, die beabsichtigen, ihre Mitmenschen in einer
„gegenseitigen`` anstatt einer „zwanghaften`` oder parasitären
Gesellschaft einzubeziehen, eröffnet die wirtschaftliche Ausdrucksform
die Aussicht, mit geringerem Aufwand an Zeit und Mühe eine weit größere
Zufriedenheit zu erreichen. Trotz Professor Hirschfields Einwänden,
lässt sich dies einfach demonstrieren.

Jede wirtschaftliche Ausdrucksform, bestehend aus Eintritt, laufenden
Verträgen und Austritt, könnte in eine politische „Stimme`` umgewandelt
werden, indem eine Vielzahl von Menschen in die Entscheidungsfindung
einbezogen wird. Versuchen Sie es doch einfach mal. Alles, was Sie für
das Experiment benötigen, sind ein paar hundert Menschen, die das Gefühl
haben, dass es in ihrem Leben nicht genug Politik gibt. Anstatt ihr
verfügbares Einkommen in Tausenden von Einzelkäufen über ein Jahr hinweg
auszugeben, würden sie diese Vielzahl von wirtschaftlichen
Entscheidungen in eine Handvoll politischer Entscheidungen umwandeln.

Zu Beginn würden alle zustimmen, ihr verfügbares Einkommen
zusammenzulegen und fortan individuelle Käufe zu unterlassen. Anstelle
von tausenden von Dollar, die individuell auf tausende Arten ausgegeben
werden könnten, erhält jeder eine Stimme oder vielleicht mehrere
Stimmen, abhängig von der Anzahl der zu besetzenden Posten. Anstatt
jederzeit Geld direkt auszugeben, um das zu bekommen, was man sich
wünscht, würde man seine Stimme oder Stimmen bei den wenigen
Gelegenheiten abgeben, wenn Wahlen abgehalten werden, um die
Repräsentanten zu wählen, die dann entscheiden würden, wie die nun
riesige gemeinsame Geldbörse ausgegeben werden sollte.

Sie würden zusammen mit den anderen an dem Konsum dieser Dinge
teilhaben, und nur an denen, die das herrschende Gremium im Namen der
Mehrheit genehmigt hat.

Erscheint das schon als „schwerfälliger politischer Kanal`` zur
Meinungsäußerung? Abwarten! Dieses Modell birgt das ganze Potential für
Redekunst und Überzeugungsarbeit, wie man es auf nationaler politischer
Ebene findet. Und die meiste Aussicht auf Frustration.

Wenn Sie zum Beispiel frischen Brokkoli mögen und die Gruppe eine
gewöhnliche Verteilung von Geschmacksvorlieben bei Lebensmitteln hat,
sind Sie in Schwierigkeiten. Es ist wahrscheinlich, dass einige oder die
meisten der anderen in Ihrer Gruppe es vorziehen würden, mehr von der
gemeinsamen Lebensmittelzuteilung für rotes Fleisch als für frisches
Gemüse auszugeben. Um zu verhindern, dass das Kantinenkomitee in ein
Lagerhaus geht und das gesamte jährliche Gemüsebudget für Dosen-Erbsen
und -Mais verpulvert, müssten Sie vielleicht vorstellig werden und Ihre
„Stimme`` erheben. Sie könnten die Aufmerksamkeit der Gruppe auf die
relativen Vorteile lenken, mehr Vitamine und Nährstoffe wie Sulforaphan
in Brokkoli zu sich zu nehmen, im Vergleich zu mehr gesättigten
Fettsäuren und Cholesterin aus rotem Fleisch.

Wie genau man diesen oder jeden anderen Punkt verständlich macht, ist in
diesem konstruierten politischen Modell natürlich ebenso rätselhaft wie
für die Befürworter jeder politischen Sache oder Kandidatur. Sie könnten
eine Rede halten, aber das setzt natürlich voraus, dass ein guter Teil
der Gruppe, die Sie überzeugen müssen, bereits irgendwo versammelt und
bereit ist, zuzuhören. Sie könnten Flugblätter drucken, vorausgesetzt,
dass die Hausregeln Ihres politischen Spiels eine solche
„Wahlkampfausgabe`` zulassen. Sie könnten Briefe schreiben. Aber beide
Möglichkeiten hängen davon ab, dass die anderen Teilnehmer lesen können.

\begin{quote}
„Es zeichnet das Bild einer Gesellschaft, in der die überwiegende
Mehrheit der Amerikaner nicht weiß, dass sie nicht die Fähigkeiten
besitzen, die sie benötigen, um in unserer zunehmend technologischen
Gesellschaft und im internationalen Markt ihren Lebensunterhalt zu
verdienen.`` Richard Riley; U.S. Bildungsminister, In „Adult Literacy in
America``
\end{quote}

\subsection{Neunzig Millionen
Alzheimer-Patienten?}\label{neunzig-millionen-alzheimer-patienten}

Wenn Ihre Gruppe in dieser politischen Modellübung zufällig aus
Amerikanern bestehen würde, hätten Sie erhebliche Schwierigkeiten, eine
überzeugende Botschaft zu verbreiten, insbesondere wenn die Mitglieder
der Gruppe dem gesamten US-Wählerpublikum ähneln. Die Wahrnehmung, dass
unverhältnismäßig viele Bürger der mächtigsten Nation der Welt
unterdurchschnittliche Leistungen erbringen, wurde durch die
gründlichste jemals durchgeführte Untersuchung der Kompetenz
amerikanischer Erwachsener traurig bestätigt. Die Studie, „Adult
Literacy in America``, zeigt, dass es keineswegs einfach ist, ein
gebildetes Publikum für irgendein politisches Argument zu finden. Ein
großer Teil, vielleicht sogar die Mehrheit der Amerikaner über fünfzehn,
fehlen grundlegende Fertigkeiten, die zur Bewertung von Ideen und zur
Urteilsbildung notwendig sind. Laut dem US-Bildungsministerium können 90
Millionen Amerikaner keinen Brief schreiben, keinen Busfahrplan
verstehen oder auch nur auf einem Taschenrechner addieren und
subtrahieren. Dies entspricht ungefähr dem, was man erwarten würde, wenn
90 Millionen Amerikaner sich in verschiedenen Stadien der
Alzheimer-Krankheit befänden. Dreißig Millionen wurden als so
inkompetent eingestuft, dass sie nicht einmal auf Fragen antworten
konnten.

Wenn also Ihre Gesundheitsbotschaft die Wende nicht geschafft hat, die
sich ansonsten ihren eigenen Weg sucht, könnten Sie bei
Tierrechtsaktivisten um Hilfe rufen. Vielleicht könnten Sie sie dazu
bringen, Ihre Gegner im Kantinenausschuss zu belagern oder bei den
Häusern einflussreicher Mitglieder einen Aufstand über das Unrecht des
Tötens von Kühen zu machen.

Dieses Beispiel könnte unendlich weitergeführt werden, was wohl weit
über das Maß der Geduld rationaler Menschen hinausgeht. Es zeigt
deutlich, dass 1. jeglicher wirtschaftlicher Ausdruck von Eintritt oder
Austritt in eine politische Äußerung der Stimme umgewandelt werden kann,
indem er zu einer kollektiven Entscheidung gemacht wird; und 2., dass
kollektive Entscheidungen, trotz der Einladung, die sie zur Eloquenz
bieten, tatsächlich oft schwerfällig und unhandlich sind.

Genau das hat die Erfahrung gezeigt. Es ist alles andere als einfach,
den Aufwand zu mobilisieren, der erforderlich ist, um den Kurs einer
Demokratie zu ändern. Um es noch einmal zu betonen, das könnte durchaus
der Grund sein, dass demokratische Wohlfahrtsstaaten Jahrhunderte des
Wettbewerbs mit alternativen Regierungsformen überlebt haben, um am Ende
der industriellen Ära zu dominieren. Demokratie hat sich als politisches
System gerade deshalb durchgesetzt, weil ihre Funktionsweise es den
Bürgern erschwerte, die Kontrolle über die Regierung zu übernehmen oder
die Ansprüche des Staates auf Ressourcen zu begrenzen.

Da jedoch eine unbeschränkte Partnerschaft mit dem Staat in Ihren
Angelegenheiten im Informationszeitalter keinen militärischen Vorteil
mehr bietet, werden einfallsreiche Menschen überlegene Wege finden, um
die wenigen wertvollen Dienstleistungen zu erhalten, die Regierungen
tatsächlich anbieten. Es ist wahrscheinlich, dass die tatsächliche
Leistung aus kollektiven Mechanismen, die sich nicht mehr rechnen,
ausgegliedert wird. Wir erwarten, dass Effizienz über gebündelte Macht
dominieren wird. Wie Neil Munro knapp ausdrückte, „ist es die
computerisierte Information - nicht Manpower oder Massenproduktion - die
zunehmend die US-Wirtschaft antreibt und die Kriege in einer Welt
gewinnen wird, die für 500 Fernsehkanäle verkabelt ist. Die
computergestützte Information existiert im Cyberspace - der neuen
Dimension, die durch die endlose Vermehrung von Computernetzwerken,
Satelliten, Modems, Datenbanken und dem öffentlichen Internet geschaffen
wird.`` \footnote{Neil Munro, \emph{The Pentagon\textquotesingle s New
  Nightmare: An Electronic Pearl Harbor,} Washington Post, 16. Juli
  1995, S. C3.}

In einer solchen Welt werden Massenarmeen wenig Bedeutung haben.
Effizienz wird noch wichtiger sein als zuvor. Wie wir bereits in Kapitel
6 und anderswo erläutert haben, schafft die Mikrotechnologie eine neue
Dimension des Schutzes. Erstmals in der menschlichen Existenz werden
Menschen in der Lage sein, Vermögenswerte zu schaffen und zu schützen,
die vollständig außerhalb des Hoheitsbereichs einzelner Regierungen
liegen. Diese Vermögenswerte werden daher hochempfindlich gegenüber
individueller Kontrolle sein. Es wird völlig legitim sein, wenn Sie und
eine bedeutende Anzahl anderer zukünftiger souveräner Individuen „mit
den Füßen abstimmen`` und sich dafür entscheiden, aus führenden Nationen
auszutreten, um mit einer abgelegenen Nation oder einer neuen
Minisouveränität einen Vertrag über persönlichen Schutz abzuschließen,
die nur einen kommerziell tolerierbaren Betrag verlangt und nicht den
Großteil Ihres Vermögens. Kurz gesagt, Sie würden wahrscheinlich 50
Millionen Dollar akzeptieren, um nach Bermuda zu ziehen.

\subsection{Zuerst aussteigen, später Verträge
abschließen}\label{zuerst-aussteigen-spuxe4ter-vertruxe4ge-abschlieuxdfen}

Der erste Impuls zur Kommerzialisierung der Souveränität muss von
Personen kommen, die sich durch ihre wirtschaftliche Abwanderung
ausdrücken. Diese Option wird besonders schwer in den Vereinigten
Staaten umsetzbar sein, wo sie aber auch am wertvollsten ist. Die
„Berliner Mauer`` für Kapitalisten, die von Präsident Bill Clinton und
dem republikanischen Kongress auferlegt wurde, widerspricht dem von
amerikanischen Nationalisten in den 1960er Jahren selbstbewusst
geäußerten Slogan „Love it or leave it``. Durch die Verhängung hoher
Steuern auf diejenigen, die auswandern, soll die Auswanderungssteuer
Loyalität erzwingen. Doch diese rachsüchtige Gesetzgebung, die an die
Strafen für fliehende Immobilienbesitzer in den letzten Tagen des
Römischen Reiches erinnert, könnte unabsichtlich den Rahmen für eine
rationalere Politik im späteren Informationszeitalter festlegen.

Es wird einen Zeitpunkt geben, an dem genügend fähige Personen das Land
verlassen und im Ausland bedeutende Vermögen angehäuft haben. Dann wird
es für die US-Behörden attraktiv erscheinen, ihren Bürgern oder Inhabern
einer Aufenthaltsgenehmigung zu gestatten, ihre zukünftigen
Steuerpflichten abzukaufen, indem sie eine Auswanderungssteuer zahlen,
ohne dabei jedoch tatsächlich auszuwandern. Mit anderen Worten, die
Auswanderungssteuer könnte zum Modell für eine einmalige
Abfindungszahlung werden. Eine Regierung, die eine solche
Auswanderungssteuer auferlegt, könnte erhebliche Vorteile daraus ziehen,
wenn sie diejenigen, die auswandern, unter den Bedingungen eines
privaten Abkommens, wie es derzeit in der Schweiz und anderen Ländern
existiert, wieder ansiedeln lässt.

Solche Schritte seitens der Vereinigten Staaten oder anderer Regierungen
wären rationale Gesten zur Optimierung der Einnahmen. Letztendlich wird
der Wettbewerb bei den Schutzdienstleistungen die Steuersätze senken und
die Steuerbedingungen an zivilisiertere Standards anpassen. Anstatt sich
auf den Gesetzgeber zu verlassen, um akzeptable Steuersysteme zu
erlassen, werden souveräne Einzelpersonen in Zukunft in der Lage sein,
akzeptable, maßgeschneiderte Maßnahmenpakete durch private Verträge
auszuhandeln.

\section{BELEIDIGUNG DER WAHREN
GLÄUBIGEN}\label{beleidigung-der-wahren-gluxe4ubigen}

Natürlich vertreten wir nicht für einen Moment die Vorstellung, dass
vieles davon populär sein wird. Die Entnationalisierung des Individuums
und die darin implizierte Kommerzialisierung der Souveränität wird die
verbleibenden wahren Gläubigen der Klischees der Politik des 20.
Jahrhunderts beleidigen. Wie der verstorbene Christopher Lasch sehen sie
in der Verkümmerung der Politik eine Bedrohung für das Wohlergehen eines
Großteils der Bevölkerung. Ihrer Ansicht nach könnte eine Wiederbelebung
der Politik des Industriezeitalters mit ihrem Engagement für die
Umverteilung von Einkommen eine Lösung für die Nöte sein, die so viele
Menschen angesichts des Wettbewerbsdrucks durch die
Informationstechnologie empfinden.

E. J. Dionne, Jr., ist ein politischer Reporter für die \emph{Washington
Post}. Ähnlich wie Lasch, blickt er nostalgisch auf die Politik zurück.
Er spricht sich auch für einen sozialdemokratischen Ausgleichsimpuls
aus, der in den kommenden Jahrzehnten zwangsläufig lauter werden wird,
wenn die neuen megapolitischen Realitäten des Informationszeitalters die
aus der modernen Welt übrig gebliebenen Institutionen immer stärker
untergraben. Dionne ist der Ansicht, dass die materiellen Verbesserungen
des Lebensstandards, die im 20. Jahrhundert in den reichen Ländern weit
verbreitet waren, in erster Linie der demokratischen Politik zu
verdanken sind und nicht der technologischen oder wirtschaftlichen
Entwicklung. Seine Botschaft lautet, dass die Hoffnung für die Zukunft
eine Ausweitung der Herrschaft der Politik über die Technologien des
Informationszeitalters erfordert:

\begin{quote}
„Das dringende Bedürfnis in den Vereinigten Staaten und in der gesamten
demokratischen Welt besteht in einem neuen Engagement für demokratische
Reformen, dem politischen Motor, der das Industriezeitalter so
erfolgreich gemacht hat. Die Technologien des Informationszeitalters
werden nicht von allein eine erfolgreiche Gesellschaft aufbauen, genauso
wenig wie der Industrialismus für sich allein die Welt verbessert hätte.
~\ldots Selbst die außergewöhnlichsten Durchbrüche in der Technologie
und die cleversten Anwendungen des Internets werden uns nicht vor
sozialem Zusammenbruch, Verbrechen oder Ungerechtigkeit retten. Nur die
Politik, die Kunst wie wir uns organisieren, kann überhaupt erst damit
anfangen, solche Aufgaben zu übernehmen.`` \footnote{E. J. Dionne,
  \emph{Why the Right Is Wrong}, Utne Reader, Juni 1996, S. 32.}
\end{quote}

Dionne und Leute wie er scheitern daran zu verstehen, dass die
Bedingungen, die das Leben im 20. Jahrhundert besonders förderlich für
systematischen Zwang gemacht haben, nicht von einer menschlichen Instanz
gewählt wurden. Die „Kunst, wie wir uns organisieren``, ist eine
Aussage, die vor der modernen Ära unverständlich gewesen wäre.
Gesellschaften sind zu komplex, um richtig als Ergebnis eines
willentlichen Selbstorganisationsversuchs betrachtet zu werden. Die
Nationalstaaten der modernen Ära entstanden spontan als zufälliges
Nebenprodukt der Industrietechnologie, die die Gewinnmöglichkeiten durch
Gewalt erhöhte. Nun reduziert die Informationstechnologie die
Gewinnmöglichkeiten durch Gewalt. Dies macht die Politik zu einem
Anachronismus und einer unwiederbringlichen Sache, unabhängig davon, wie
sehr die Menschen sie ins nächste Jahrtausend hinüberretten möchten.

\begin{quote}
„Nicht heut und gestern nur, die leben immer, Und niemand weiß, woher
sie sind gekommen.`` - Sophokles, Antigone
\end{quote}

\section{„SIE MACHEN ES NICHT MEHR SO WIE
FRÜHER``}\label{sie-machen-es-nicht-mehr-so-wie-fruxfcher}

Der brennende Wunsch, „Gesetze zu machen``, der so sehr Teil des
„gesunden Menschenverstands`` der Politik des 20. Jahrhunderts zu sein
scheint, ist keineswegs allen Kulturen eigen. Sein Verschwinden in der
Zukunft könnte als Teil eines Zyklus gesehen werden, der im Laufe der
Jahrhunderte gewachsen und geschrumpft ist. Zum Beispiel glaubten die
alten Griechen, unter anderem, dass Gesetze nicht gemacht werden
könnten. In den Worten des Philosophen Ernst Cassirer glaubten die
Griechen, „die ‚ungeschriebenen Gesetze', die Gesetze der Gerechtigkeit,
haben keinen Anfang in der Zeit``.\footnote{Ernst Cassirer, \emph{The
  Myth of the State} (New Haven: Yale University Press, 1946), S. 81.}
Wie andere präpolitische Völker fühlten sie, dass niemand die
natürlichen, „geometrischen`` Gesetze der Gerechtigkeit verbessern
konnte, die von keiner menschlichen Macht geschaffen wurden.

Sie glaubten nicht an einen „Gesetzgeber``. Wie Cassirer es ausdrückte:
„Es liegt am rationalen Denken, dass wir die Standards moralischen
Verhaltens finden, und es ist die Vernunft, und nur die Vernunft, die
ihnen ihre Autorität verleihen kann.`` In diesem Sinne wäre jeder
Versuch, Gesellschaften durch Gesetzgebung Gesetze aufzuerlegen, so, als
würde man versuchen, die Geometrie durch Gesetzgebung zu verändern.

\subsection{Gesetzgebung als Sakrileg}\label{gesetzgebung-als-sakrileg}

Aus völlig anderen Gründen herrschte während eines großen Teils des
Mittelalters ein ähnlicher Widerstand gegen die „Gesetzgebung``. Wie
John B. Morrall schreibt, „Für die Deutschen war das Gesetz etwas, das
seit Urzeiten existierte``. Es war „eine Garantie für die Rechte``
einzelner Mitglieder des Stammes.\footnote{John B. Morrall,
  \emph{Political Thought in Medieval Times} (New York: Harper
  Torchbooks, 1962), S. 15.} Könige und Räte

\begin{quote}
hatten bisher keine Absicht, neues Recht zu schaffen. Eine solche
Absicht wäre aus der Sicht des frühen Mittelalters nicht nur
überflüssig, sondern geradezu gotteslästerlich gewesen, denn das Recht
besaß ebenso wie das Königtum eine eigene sakrosankte Aura. Stattdessen
sahen sich König und Räte nur als Erklärer oder Klärer der wahren
Bedeutung des bereits existierenden und vollständigen Körpers des
Rechts. Der germanische Brauch überlieferte dem mittelalterlichen Geist
eine Idee, die er nie vergessen konnte, selbst wenn er in der Praxis
anders handelte. Diese Idee bestand darin, dass gute Gesetze
wiederentdeckt oder neu formuliert, aber niemals neu gemacht
wurden.\footnote{Ebenda, S. 16.}
\end{quote}

Nach den Exzessen der Gesetzgebung des zwanzigsten Jahrhunderts hat
diese alte Einstellung etwas Seltsames an sich. Der Wunsch, die
Zwangsgewalt des Staates für private Zwecke einzusetzen, insbesondere
für die Umverteilung von Einkommen, wurde fast zur zweiten Natur.

\subsection{Bedauern}\label{bedauern}

Es ist daher kaum verwunderlich, dass es traurige Lieder über die
Politik in ihren letzten Tagen gibt. Sie sind völlig vorhersehbar. Und
nicht nur, weil sie die Blindheit der meisten Denker gegenüber den
Imperativen der Megapolitik widerspiegeln. Wenige politische Reporter,
wie Dionne, sind bereit, die offensichtliche Rückbildung und den
Niedergang der Politik zu akzeptieren, wenn dies dazu führen könnte,
dass sie wieder auf der Verbrechensspur landen. Am Ende des Mittelalters
wurden Stimmen laut, die eine Wiederbelebung der Ritterlichkeit
unterstützten. Betrachten Sie \emph{Il Libro del Cortegiano} oder
\emph{Das Buch des Hofmanns}, das 1514 von Graf Baldassare Castiglione
geschrieben und 1528 von Aldus in Venedig veröffentlicht wurde.

Castigliones tiefgreifendes Verlangen nach einer Rückkehr zu den
Tugenden des Rittertums war stark spürbar, doch die Sehnsucht nach einer
erloschenen Lebensweise konnte sie im sechzehnten Jahrhundert nicht
zurückbringen. Und das wird sie auch im einundzwanzigsten Jahrhundert
nicht tun.

Wie wir bei der Erläuterung unserer Theorie der Megapolitik zu
vermitteln versucht haben, sind technologische Imperative und nicht die
öffentliche Meinung die wichtigsten Quellen für Veränderungen. Wenn
unsere Theorie der Megapolitik zutrifft, dann war der Grund dafür, dass
die Moderne mit ihrem Konzept der Staatsbürgerschaft und der auf den
Staat ausgerichteten Politik das Feudalsystem und das auf persönliche
Eide und Beziehungen ausgerichtete Rittertum verdrängt hat, nicht eine
Frage der Ideen, sondern der Verschiebung von Kosten und Nutzen aufgrund
neuer Technologien. Das Rittertum starb nicht aus, weil Castiglione oder
andere daran scheiterten, eine gleichgültige Bevölkerung davon zu
überzeugen, dass es in Staatsangelegenheiten keinen Bedarf an Ehre oder
Moral gab. Im Gegenteil, Castigliones ‚Der Hofmann' ist kritisch
gegenüber Fürsten und der von seinem Zeitgenossen Niccolò Machiavelli in
\emph{Il Principe} oder \emph{Der Fürst} gelobten Verhaltensweise. Und
was dann? Letztlich erreichte Machiavelli mit seinem Buch ein größeres
Publikum, nicht weil sein Argument in \emph{Der Fürst} überzeugender
war, sondern weil seine Ratschläge besser zu den megapolitischen
Bedingungen des modernen Zeitalters passten.

Wie der angesehene Philosoph des 20. Jahrhunderts, Ernst Cassirer, in
seiner Diskussion über „Das moralische Problem bei Machiavelli`` sagte,

\begin{quote}
Das Buch beschreibt, mit völliger Gleichgültigkeit, die Methoden und
Mittel, mit denen politische Macht erworben und aufrechterhalten werden
kann. Über den korrekten Gebrauch dieser Macht sagt es kein Wort.
~\ldots{} Niemand hatte jemals bezweifelt, dass das politische Leben,
wie es momentan aussieht, voller Verbrechen, Verrätereien und schweren
Vergehen ist. Aber kein Denker vor Machiavelli hatte sich daran gemacht,
die Kunst dieser Verbrechen zu lehren. Diese Dinge wurden getan, aber
sie wurden nicht gelehrt. Dass Machiavelli versprach, ein Lehrer in der
Kunst der Intrigen, Untreue und Grausamkeit zu werden, war bisher
unerhört.\footnote{Cassirer, ebenda, S. 142, 150.}
\end{quote}

Kurz gesagt, \emph{Der Fürst} war ein radikales Werk, das ein modernes
Rezept entwarf, wie ein aufstrebender Herrscher seine Karriere um jeden
Preis für andere vorantreiben konnte. Machiavelli befürwortete
Verhaltensweisen, die sich als gut geeignet für die Politik in einem
Zeitalter der Macht erwiesen. Aber die Kunst des Doppelspiels, die eine
kluge Politik für Politiker im modernen Zeitalter war, war empörend und
subversiv im Vergleich zur Kultur des Rittertums, die in den vorherigen
Jahrhunderten gewachsen war.

Wie wir bereits früher herausgefunden haben, zählten zu den Tugenden des
Rittertums ein besonderes Augenmerk auf extreme Treue zu Eiden. Dies war
eine Notwendigkeit in einer Gesellschaft, in der Schutz im Tausch gegen
persönliche Dienste organisiert wurde. Die Abmachungen, auf denen die
feudale Gesellschaft beruhte, waren nicht so beschaffen, dass sie unter
Menschen, die unter Zwang frei über ihre Interessen entscheiden konnten,
spontan wieder zustande gekommen wären. Daher mussten die feudalen
Verpflichtungen, die die Grundlage des Rittertums bildeten, durch ein
starkes Ehrgefühl ergänzt werden. In diesem Zusammenhang hätte kaum
etwas subversiver sein können als Machiavellis Vorschlag, der Fürst
solle nicht zögern zu lügen, zu betrügen und zu stehlen, wenn dies
seinen Interessen diene.

Als das 20. Jahrhundert sich dem Ende zuneigte, wurden Machiavellis
Argumente immer noch auf ihre Bedeutung für das Verständnis der modernen
Politik und verschiedener Verbrechen und Tyranneien des 20. Jahrhunderts
hin untersucht. Castigliones Werk hingegen ist nahezu vergessen. Im
Laufe eines Jahres wird \emph{Il Libro del Cortegiano} vielleicht von
einer Handvoll Literaturstudenten auf dem Postgraduiertenniveau und
einigen Kennern der Geschichte der Manieren von Anfang bis Ende gelesen.

Irgendwann in den nächsten Jahrzehnten wird die neue Megapolitik des
Informationszeitalters \emph{Der Fürst} veralten lassen. Das souveräne
Individuum wird ein neues Erfolgsrezept benötigen, das Ehre und
Geradlinigkeit bei der Bereitstellung von Ressourcen außerhalb des
Staates stark betont. Wir können voraussagen, dass solche Ratschläge von
E. J. Dionne Jr.~und den anderen lebenden Sozialdemokraten nicht mit
Vergnügen gelesen werden.

\subsection{Von Kunden festgelegte
Politik}\label{von-kunden-festgelegte-politik}

Dies wird besonders zu Beginn des Übergangs der Fall sein, wenn die
meisten Zuständigkeitsbereiche noch mit der Notwendigkeit belastet sein
werden, Politiken zu formulieren, deren Befürworter die Unterstützung
der Mehrheit der Bevölkerung gewinnen können. Später, wenn die
Demokratie nachlässt und der Markt für Souveränitätsdienste intensiver
wird, sollen die Marktbeschränkungen, die die „Politik`` beeinflussen,
breiter verstanden werden.

Was wir derzeit als „politische`` Führung verstehen, die immer im
Kontext eines Nationalstaates gedacht wird, wird zunehmend eher
unternehmerischer als politischer Natur. Unter diesen Bedingungen wird
der praktikable Spielraum für die Zusammenstellung eines „politischen``
Regelwerks für einen Gerichtsstand effektiv eingeschränkt, genauso wie
der Spielraum für Unternehmer bei der Planung eines erstklassigen Resort
Hotels oder eines ähnlichen Produkts oder einer Dienstleistung durch das
definiert ist, was Menschen dafür bezahlen würden. Ein Resort Hotel zum
Beispiel würde selten versuchen, unter Bedingungen zu agieren, die von
den Gästen verlangen, hart zu arbeiten, um seine Einrichtungen zu
reparieren und zu erweitern. Selbst ein Resort Hotel, das von seinen
Mitarbeitern besessen oder kontrolliert wird, ähnlich wie die typische
moderne Demokratie, würde vergeblich versuchen, seine Kunden dazu zu
zwingen, solchen Forderungen nachzukommen, insbesondere wenn bessere
Unterkünfte verfügbar werden. Wenn die Kunden lieber Golf spielen
würden, als schwere Arbeiten in der heißen Sonne zu verrichten, dann
bietet der Markt in dieser Frage zumindest wenig Spielraum für das
Aufzwingen willkürlicher Alternativen. Unter solchen Bedingungen werden
aktuell „politische`` Fragen zunehmend in unternehmerische
Entscheidungen übergehen, während die Gerichtsbarkeiten versuchen zu
entdecken, welche politischen Pakete Kunden anziehen werden.

\subsection{Die Verkümmerung der
Politik}\label{die-verkuxfcmmerung-der-politik}

In dem Maße, in dem dies verstanden wird, werden sich auch die
Einstellungen ändern. Die Bevölkerung in den dezentralisierten Ländern
wird nicht mehr erwarten, dass sie aus der gleichen Palette von
wunscherfüllenden politischen Optionen wählen kann, die im zwanzigsten
Jahrhundert die politische Debatte beherrschten. Da die
Einkommensmöglichkeiten stärker verteilt sind als im Industriezeitalter,
werden die Staaten dazu neigen, die Bedürfnisse derjenigen Kunden zu
befriedigen, deren Geschäft am wertvollsten ist und die die größte
Auswahl haben, wo sie es ausgeben können.

Unter solchen Bedingungen könnte es viel weniger bedeutsam sein, als wir
gewohnt sind anzunehmen, ob Politiken, die kommerziell optimal für einen
Zuständigkeitsbereich sind, dem „Durchschnittswähler`` in einer
Fokusgruppe gefallen würden. Kurz gesagt, die Kommerzialisierung der
Souveränität wird die Kontrolle der Regierungen durch ihre Kunden
erleichtern. Dies wird dazu tendieren, die Meinungen der Nicht-Kunden
irrelevant oder zumindest weniger relevant zu machen, so wie die
Meinungen der Big Mac-Esser über Stopfleber für den Erfolg von
französischen Drei-Sterne-Restaurants wie L\textquotesingle Arpège in
Paris irrelevant sind.

\section{„DER VERRAT AN DER
DEMOKRATIE``}\label{der-verrat-an-der-demokratie}

Wie der verstorbene Christopher Lasch werden Gegner nicht nur beklagen,
dass die Informationstechnologie Arbeitsplätze vernichtet; sie werden
auch beklagen, dass sie die Demokratie negiert, weil sie es dem
Einzelnen ermöglicht, seine Ressourcen außerhalb der Reichweite
politischer Zwänge einzusetzen. Aus diesem Grund wird die durch
Informationstechnologie erleichterte finanzielle Privatsphäre den
Reaktionären des neuen Jahrtausends besonders bedrohlich erscheinen. Sie
werden vor der Aussicht zurückschrecken, dass die Einkommens- und
Kapitalbesteuerung wirklich von „freiwilliger Übereinkunft`` abhängen
könnten. Sie werden neuartige und sogar drastische Mittel unterstützen,
um jedem, der wohlhabend zu sein scheint, die Mittel abzupressen, wie z.
B. „mutmaßliche Besteuerung`` und regelrechtes Erpressen von
wohlhabenden Personen.

\subsection{Gemeinschaftseigentum}\label{gemeinschaftseigentum}

Hinweise auf das, was noch kommen soll, liegen zum Zeitpunkt unserer
Schreibarbeit nahe an der Oberfläche. Frühe Anzeichen dafür, dass die
Fähigkeit der Regierungen, internationale Märkte zu kontrollieren, zu
schwinden beginnt, provozieren jene, die glauben, dass Individuen
rechtmäßig Vermögenswerte von Nationalstaaten sind. Sie möchten ihre
Fähigkeit durchsetzen, die Bürger eines Landes als Vermögenswerte und
nicht als Kunden zu behandeln. Die Reaktionäre sind der Überzeugung,
dass sämtliches Einkommen als Gemeinschaftseinnahmen angesehen werden
sollten, was bedeutet, dass es dem Staat zur Verfügung stehen
sollte.\footnote{Robert I Shapiro, \emph{Flat Wrong: New Tax Schemes
  Can\textquotesingle t Top Old Progressive Truths}, Washington Post,
  24. März 1996, S.C3, und Thomas L. Friedman, \emph{Politics in the Age
  of NAFTA,} New York Times, 7. April S. E11.}

Wir haben bereits die Argumente diskutiert, die Lasch in \emph{Revolt of
the Elites and the Betrayal of Democracy} vorgebracht hat. Aber das ist
nicht die einzige Streitschrift zugunsten des Nationalstaates. Der
Politikwissenschaftler der Harvard Universität, Michael Sandel,
argumentiert in \emph{Democracy in Discontent}, dass „eine heutige
Demokratie ohne eine Politik, die die globalen wirtschaftlichen Kräfte
kontrollieren kann, nicht möglich ist, denn ohne eine solche Kontrolle
ist es egal, wen die Menschen wählen, da die Konzerne regieren werden.``
\footnote{Zitiert von Friedman, ebenda.} Mit anderen Worten: Der Staat
muss seine parasitäre Macht über die Individuen beibehalten, um
sicherzustellen, dass politische Ergebnisse sich von Marktergebnissen
unterscheiden können. Andernfalls wären kollektive Entscheidungen zur
Herbeiführung unwirtschaftlicher Ergebnisse bedeutungslos.

In unserer Sichtweise ist Sandels Kritik, wie die von Lasch, nur zur
Hälfte richtig. Wir räumen ein, dass die Demokratie einen Großteil ihrer
Bedeutung verlieren wird, wenn Regierungen nicht die Macht haben,
Einzelpersonen zu zwingen, sich so zu verhalten, wie Politiker es
vorschreiben. Das ist offensichtlich. Tatsächlich ist die Demokratie,
wie wir sie aus dem 19. und 20. Jahrhundert kennen, dazu bestimmt, zu
verschwinden. Aber Sandel übersieht die wirkliche Bedeutung des Triumphs
der Märkte über den Zwang. Sein Aufruf zu einer „Herrschaft der
Konzerne`` als eine Gefahr, die mit dem Zusammenbruch des Nationalstaats
einhergeht, ist auffallend anachronistisch.

Konzerne werden kaum in der Lage sein, die Märkte der neuen globalen
Wirtschaft zu beherrschen. Tatsächlich ist, wie wir bereits angedeutet
haben, keineswegs offensichtlich, dass Konzerne auch in ihrer vertrauten
modernen Form weiter bestehen werden. Weit gefehlt. Konzerne sind quasi
dazu verpflichtet, sich in der megapolitischen Revolution zu verändern,
die mit der Einführung des Informationszeitalters einhergeht. Wie wir
zuvor diskutiert haben, wird die Mikroverarbeitung die
„Informationskosten`` verändern, die dazu beitragen, den „Nexus von
Verträgen`` zu bestimmen, der Unternehmen definiert. Wie die Ökonomen
Michael C. Jensen und William H. Meckling vorschlagen, sind Unternehmen
lediglich eine rechtliche Form, die „einen Nexus für eine Reihe von
Vertragsbeziehungen zwischen Individuen`` darstellt.\footnote{Siehe
  Louis Putterman und Randall S. Kroszner, \emph{The Economic Nature of
  the Firm: A New Introduction}, in Louis Putterman und Randall S.
  Kroszner, eds., \emph{The Economic Nature of the Firm: A Reader}
  (Cambridge: Cambridge University Press, 1996), S. 17.}

Ob Konzerne überhaupt noch überleben können, geschweige denn
„herrschen`` als „ein Bereich bürokratischer Leitung, der vor
Marktkräften abgeschirmt ist``, wird nach den Worten der
Wirtschaftswissenschaftler Louis Putterman und Randall S. Kroszner
wahrscheinlich selbst durch „die Vollständigkeit der Marktkräfte und die
Fähigkeit der Marktkräfte, die innerbetrieblichen Beziehungen zu
durchdringen`` bestimmt.\footnote{Ebenda.}

Wie wir bereits zuvor argumentiert haben, ist es zweifelhaft, dass
Firmen das zunehmende Eindringen von Marktkräften in das, was bisher
„innerbetriebliche Beziehungen`` waren, überleben können. Infolgedessen
werden Unternehmen dazu tendieren, sich aufzulösen, da die
Informationstechnologie es attraktiver macht, sich auf den
Preismechanismus und den Auktionsmarkt zu verlassen, um anstehende
Aufgaben zu erledigen, anstatt sie innerhalb einer formellen
Organisation zu internalisieren. Da die Informationstechnologie den
Produktionsprozess zunehmend automatisiert, wird sie einen Teil der
Daseinsberechtigung der Firma, die Notwendigkeit, Manager zu
beschäftigen und zu motivieren, um einzelne Arbeiter zu überwachen,
wegnehmen.

\subsection{„Warum gibt es Firmen?{}``}\label{warum-gibt-es-firmen}

Denken Sie daran, dass die Frage „Warum gibt es Firmen?{}`` nicht so
trivial ist, wie sie auf den ersten Blick erscheinen mag. Die
Mikroökonomie geht allgemein davon aus, dass der Preismechanismus das
effektivste Mittel zur Koordination von Ressourcen für deren am höchsten
bewerteten Einsatz ist. Wie Putterman und Kroszner feststellen, deutet
dies tendenziell darauf hin, dass Organisationen wie Firmen keine
inhärente „ökonomische Daseinsberechtigung`` \footnote{Ebenda, S. 9.}
haben. In diesem Sinne sind Firmen hauptsächlich Artefakte von
Informations- und Transaktionskosten, die Informationstechnologien
tendenziell drastisch reduzieren.

Daher wird das Informationszeitalter eher das Zeitalter von unabhängigen
Auftragnehmern ohne „Arbeitsplätze`` in langjährigen „Firmen`` sein. Da
Technologie die Transaktionskosten senkt, wird der gleiche Prozess der
Befreiung von politischer Beherrschung auch die „Herrschaft durch
Konzerne`` verhindern. Konzerne werden in einem Ausmaß mit „virtuellen
Konzernen`` überall auf der Welt konkurrieren, das alle außer Wenigen
beunruhigen und bedrohen wird. Die meisten Konzerne als Institutionen
werden mit dem verschärften Wettbewerb in immer vollständigeren Märkten
zu kämpfen haben, um zu überleben.

Die zu erwartende Konsequenz ist nicht, dass Individuen den Konzernen
ausgeliefert sein werden. Im Gegenteil. Konzerne als solche werden,
nicht mehr Macht haben, die Märkte zu manipulieren als Politiker. Es ist
vielmehr so, dass Individuen endlich frei sein werden, ihr eigenes
Schicksal in einem wirklich freien Markt zu bestimmen, der weder von
mächtigen Regierungen noch von Konzernhierarchien geregelt wird.

Diese Erosion der Transaktionskosten wird auch die kürzlich in Mode
gekommene Vorstellung vom „Stakeholder-Kapitalismus`` in Frage stellen.
Solche Konzepte, die Tony Blair von der britischen Labour Party ebenso
am Herzen liegen wie einigen in Bill Clintons Umfeld, beruhen auf der
Fähigkeit des Staates, die Konzerne zu manipulieren. Nach dem
Zusammenbruch des Sozialismus träumen die Interventionisten nun davon,
die Ziele des Sozialismus durch markteffizientere Mittel zu erreichen,
indem sie die Unternehmen stark regulieren. Diese neue umverteilende
These besagt, dass Geschäftsführung, Aktionäre, Mitarbeiter und „die
Gemeinschaft`` alle „Stakeholder`` der Firmen sind. Das Argument lautet,
dass sie alle Vorteile von langlebigen Firmen ableiten und sogar auf
diese Vorteile angewiesen sind. Daher sollte die Regulierung den Anteil
schützen, den Manager, Mitarbeiter und lokale Steuerbehörden an der
Fortsetzung ihrer historischen Beziehungen mit den Firmen haben.

„Stakeholder-Kapitalismus`` ist eine Doktrin, die letztendlich nicht nur
die Fähigkeit des Staates voraussetzt, die Entscheidungsfindung von
Konzernen zu manipulieren, sondern noch grundlegender die Existenz von
Konzernen als langjährige Organisationen voraussetzt, die unabhängig von
Preissignalen im Auktionsmarkt funktionieren können.

Wir vermuten, dass die Vertiefung der Märkte nicht nur die
Steuerkapazität des Nationalstaates verringern wird, sondern auch die
Fähigkeit der Politiker, den Eigentümern von Ressourcen durch
Regulierung willkürlich ihren Willen aufzuzwingen. In einer Welt, in der
juristische Vorteile Markttests unterzogen werden und viele lokale
Märkte für Wettbewerb aus aller Welt geöffnet werden, ist kaum zu
erwarten, dass lokale „Gemeinschaften`` viele wirksame Möglichkeiten
haben werden, bevorzugte Firmen vor globalem Wettbewerbsdruck zu
isolieren. Daher werden sie kaum Möglichkeiten haben, sicherzustellen,
dass Konzerne, die mit höheren Kosten belastet sind (zum Beispiel zum
Halten von unnötigen Angestellten und Managementpersonal und zur
Aufrechterhaltung von ungenutzten Einrichtungen, um lokalem politischem
Druck gerecht zu werden), in der Lage sind, diese Kosten auszugleichen
und im Geschäft zu bleiben. Im Industriezeitalter konnten Politiker
Märkte schließen und den Zugang für einige bevorzugte Firmen
einschränken, um Beschäftigung und andere Ziele zu erreichen. In der
Zukunft, wenn Informationen überall auf der Welt frei gehandelt werden
können, wird die Macht der Regierungen, lokale Unternehmen vor globalem
Wettbewerbsdruck zu schützen, minimal sein.

Auch ist es unwahrscheinlich, dass sich Forderungen nach einem „neuen
Gesellschaftsvertrag``, der sich auf einen so genannten unabhängigen
oder ehrenamtlichen Sektor konzentriert, der die Zeit von ansonsten
arbeitslosen oder marginalisierten Arbeitnehmern „in der Gemeinschaft``
auffangen soll, als realisierbar erweisen werden.\footnote{Siehe Jeremy
  Rifkin, \emph{The End of Work: The Decline of the Global Labor Force
  and the Dawn of the Post-Market Era} (New York: G.P.
  Putnam\textquotesingle s Sons, 1995).} Jeremy Rifkin stellt sich „eine
neue Partnerschaft zwischen der Regierung und dem dritten Sektor vor, um
die Sozialwirtschaft wieder aufzubauen. ~\ldots{} Die Armen zu ernähren,
eine medizinische Grundversorgung bereitzustellen, die Jugend der Nation
auszubilden, erschwingliche Wohnungen zu bauen und die Umwelt zu
erhalten. ~\ldots`` \footnote{Ebenda, S. 250.}

\subsection{Das Verschwinden der öffentlichen
Güter}\label{das-verschwinden-der-uxf6ffentlichen-guxfcter}

Natürlich werden Apologeten der Zwangsanwendung argumentieren, dass das
Abnehmen der Staatsmacht zu einem Unvermögen führen wird, öffentliche
Güter zu beschaffen oder zu genießen. Dies ist aus Wettbewerbsgründen
und anderen Gründen unwahrscheinlich. Zum einen werden, da
technologische Entwicklungen Standortvorteile zumeist aufheben,
Zuständigkeitsbereiche, die wesentliche öffentliche Güter wie die
Aufrechterhaltung von Recht und Ordnung nicht bereitstellen, schnell
Kunden verlieren. Im extremsten Falle, wie bereits in Somalia, Liberia,
Ruanda und dem ehemaligen Jugoslawien zu sehen, können Horden
mittelloser Flüchtlinge über Grenzen strömen, um eine
zufriedenstellendere Bereitstellung von Recht und Ordnung zu suchen.
Aber diese extremen Beispiele für Desertationen oder Abstimmungen mit
den Füßen unterscheiden sich nur in ihrer Dringlichkeit vom einfachen
Einkauf von Rechtssystemen. In jedem Fall werden Unternehmen lokale
Rechtssysteme dazu zwingen, die Bedürfnisse ihrer Kunden zu erfüllen.

\subsection{„Wettbewerbsfähige
Territorialklubs``}\label{wettbewerbsfuxe4hige-territorialklubs}

Dies ist mehr als nur eine Theorie, wie sie erstmals vom
Wirtschaftswissenschaftler Charles Tiebout im Jahr 1956 vorgestellt
wurde.\footnote{Siehe Charles M. Tiebout, \emph{A Pure Theory of Local
  Expenditure}, Journal of Political Economy 64 (1956), S. 416-24}
Wirtschaftswissenschaftler Fred Foldvary hat in \emph{Public Goods and
Private Communities: The Market Provision of Social Services}
dokumentiert, dass es keinen wesentlichen Grund dafür gibt, dass soziale
Dienstleistungen und viele öffentliche Güter mit politischen Mitteln
bereitgestellt werden müssen. Foldvarys Beispiele bestätigen unter
anderem das umstrittene Theorem des Nobelpreisträgers Ronald Coase, dass
„staatliche Interventionen nicht notwendig sind, um Probleme im
Zusammenhang mit externen Einflüssen``, z. B. Umweltverschmutzung, zu
lösen.\footnote{Mueller, ebenda, S .28-29.} Unternehmer können
kollektive Güter durch Marktmittel bereitstellen. Viele tun dies bereits
in der realen Welt. Foldvarys Fallstudien zeigen, wie die Privatisierung
von Gemeinschaften zu neuen Mechanismen für die Bereitstellung und
Finanzierung von öffentlichen Gütern und Dienstleistungen führen
kann.\footnote{Fred Foldvary, \emph{Public Goods and Private
  Communities: The Market Provision of Social Services} (Aldershot,
  Hants, England: Edward Elgar Publishing, Ltd., 1994).}

\subsection{Der Weg zum Wohlstand}\label{der-weg-zum-wohlstand}

Die Mikrotechnologie selbst wird neue Möglichkeiten zur Finanzierung und
Regulierung von Waren bereitstellen, die bisher als öffentliche Güter
behandelt wurden. Rückblickend werden einige dieser Güter sich als
getarnte private Güter herausstellen. Autobahnen sind ein
Paradebeispiel. Solange Staus ein geringes Problem waren, konnten
Straßen und Autobahnen als öffentliche Güter behandelt werden, auch wenn
sie den von Adam Smith geäußerten Kritikpunkten unterlagen, dass sie die
in der Nähe lebenden Menschen überproportional begünstigen, auf Kosten
derjenigen in entfernten Regionen, die gezwungen sind, dafür zu
bezahlen, während sie nur wenige der Vorteile genießen.

Im Informationszeitalter wird es technologisch möglich sein, Maut- sowie
Staugebühren, zu erheben, die den Zugang zu Autobahnen, Start- und
Landebahnen und anderen Infrastrukturen genau bepreisen, ohne den
Verkehrsfluss zu unterbrechen. Damit könnte die Bereitstellung der
Verkehrsinfrastruktur diskret privatisiert und direkt von den Nutzern
des Dienstes finanziert werden. Der Ökonom Paul Krugman schätzt, dass
die marktgerechte Preisgestaltung der US-Verkehrsinfrastruktur jährlich
zwischen 60 und 100 Milliarden US-Dollar zum CIDP (Corridor
Identification and Development Program) in den USA hinzufügen würde,
während gleichzeitig die Effizienz der Ressourcennutzung verbessert und
die Umweltverschmutzung reduziert würde.\footnote{Paul R. Krugman,
  \emph{The Tax-Reform Obsession}, New York Times Magazine, 7. April
  1996, S. 37.}

Darüber hinaus darf nicht vergessen werden, dass der kostspieligste Teil
dessen, was moderne Nationalstaaten tun - die Umverteilung von Einkommen
- nicht die Bereitstellung eines öffentlichen Gutes ist, sondern die
Bereitstellung privater Güter auf Kosten der Allgemeinheit. „Auf Kosten
der Allgemeinheit`` ist hier ein Euphemismus für „auf Kosten derjenigen,
die Steuern zahlen``.

Wie sieht es mit einem echten öffentlichen Gut aus, wie der
Bereitstellung einer Militärmacht, die in der Lage ist, einen Angriff
einer Großmacht abzuwehren? Solche Streitkräfte waren traditionell
teuer. Wie wir bereits untersucht haben, wäre eine Regierung, die nicht
die uneingeschränkte Fähigkeit hat, die Einkommen und das Eigentum ihrer
Bürger zu konfiszieren, nicht in der Lage, eine Beteiligung an einem
anderen Großmacht-Konflikt wie dem Zweiten Weltkrieg zu finanzieren.

Und doch stellt dieses fiskalische Limit weniger eine Bedrohung dar, als
reaktionäre Kräfte vorgeben, aus dem einfachen Grund, dass es keine
Konflikte mehr wie den Zweiten Weltkrieg geben wird. Die Technologie,
die Individuen befreit, wird dafür sorgen.

\subsection{Über die Politik hinweg}\label{uxfcber-die-politik-hinweg}

Anstatt die Qualität und den Charakter solcher Dienstleistungen dem
Belieben der Politik zu überlassen, können „Regierungen``
unternehmerisch geführt und in das umgewandelt werden, was Foldvary als
„wettbewerbsfähige Territorialklubs`` bezeichnet.\footnote{Foldvary,
  ebenda, S. 66f.} Wir vermuten, dass letztendlich der
Entscheidungsprozess, durch den solche „wettbewerbsfähige
Territorialklubs`` organisiert werden, viel weniger bedeutet als ihr
Erfolg bei der Bewältigung marktbasierter Leistungstests. Heutzutage
interessiert es nur wenige Verbraucher, wenn sie ein Produkt oder eine
Dienstleistung kaufen, ob das Unternehmen, das es verkauft, ein
Einzelunternehmen, eine Gesellschaft mit beschränkter Haftung oder eine
Aktiengesellschaft ist, die von externen, durch Pensionspläne ernannten
Direktoren kontrolliert wird. Ebenso bezweifeln wir, dass es den
rationalen Verbraucher von Staatsdienstleistungen im
Informationszeitalter interessiert, ob Singapur eine Massendemokratie
oder das Eigentum von Lee Kwan Yew ist.

\bookmarksetup{startatroot}

\chapter{MORAL UND VERBRECHEN IN DER „NATÜRLICHEN WIRTSCHAFT`` DES
INFORMATIONSZEITALTERS}\label{moral-und-verbrechen-in-der-natuxfcrlichen-wirtschaft-des-informationszeitalters}

\begin{quote}
„Korruption...ist weitaus verbreiteter und universeller als bisher
angenommen. Beweise dafür sind überall zu finden, sowohl in
Entwicklungsländern als auch, mit wachsender Häufigkeit, in
Industrieländern. ...Hochrangige politische Persönlichkeiten, darunter
Staatsoberhäupter und Minister, wurden der Korruption beschuldigt. ...In
gewisser Weise repräsentiert dies eine Privatisierung des Staates, bei
der die Macht nicht, wie Privatisierung normalerweise impliziert, dem
Markt überlassen wird, sondern an Regierungsbeamte und Bürokraten
geht.`` \footnote{Vito Tanzi, \emph{Corruption: Arm's-length
  Relationships and Markets}, in Gianluca Fiorentini und Sam Peltzman,
  eds., \emph{The Economics of Organized Crime} (Cambridge: Cambridge
  University Press, 1995), S. 167, 170.} - Viro Tanzi
\end{quote}

Wir gehen davon aus, dass sich der moderne Nationalstaat immer mehr
auflöst und „Barbaren der Neuzeit`` zunehmend die tatsächliche Macht
hinter den Kulissen ausüben. Gruppierungen wie die russischen
Mafiaclans, die die Überreste der ehemaligen Sowjetunion ausschlachten,
ethnische Verbrecherbanden, Nomenklaturen, Drogenbarone und abtrünnige
Geheimdienstagenturen werden immer mehr zu Rechtsstaaten in sich selbst.
Sie sind es bereits. Viel mehr als allgemein bekannt ist, haben die
modernen Barbaren bereits die Formen des Nationalstaates infiltriert,
ohne sein Erscheinungsbild allzu sehr zu verändern. Sie sind
Mikroparasiten, die sich von einem sterbenden System ernähren. Ebenso
gewalttätig und skrupellos wie ein Staat im Krieg, wenden diese Gruppen
die Techniken des Staates auf kleinerer Ebene an. Ihr wachsender
Einfluss und ihre Macht sind Teil der Verkleinerung der Politik. Durch
die Mikroverarbeitung wird die Größe, die Gruppen erreichen müssen, um
eine effektive Nutzung und Kontrolle der Gewalt zu gewährleisten,
reduziert. Mit dem Fortschreiten dieser technologischen Revolution wird
räuberische Gewalt zunehmend außerhalb zentraler Kontrollen organisiert.
Auch Bemühungen, Gewalt einzudämmen, werden sich mehr auf Effizienz als
auf die Größe der Macht stützen.

Die verstärkte heimliche kriminelle Aktivität und Korruption innerhalb
von Nationalstaaten wird eine wichtige Nebenhandlung in der Veränderung
der Welt bilden. Was Sie sehen könnten, wäre eine heimtückische und
düstere Variante eines schlechten Films, die \emph{Invasion der
Körperfresser}. Bevor die meisten Nationalstaaten sichtbar
zusammenbrechen, werden sie von neuzeitlichen Barbaren dominiert. Nicht
selten, wie in dem berühmten B-Film aus den 1950er-Jahren, werden sie
getarnte Barbaren sein. Die Körperfresser der Zukunft werden jedoch
keine Außerirdischen aus dem Weltraum sein, sondern Kriminelle
unterschiedlicher Zugehörigkeit, die offizielle Positionen bekleiden und
sich zumindest teilweise außerhalb der verfassungsmäßigen Ordnung
bewegen.

Das Ende einer Ära ist normalerweise eine Zeit intensiver Korruption.
Wenn sich die Bindungen an das alte System auflösen, löst sich auch das
soziale Ethos auf. Dies schafft eine Atmosphäre, in der Personen in
hohen Positionen öffentliche Anliegen mit privaten kriminellen
Aktivitäten vermischen können.

Leider können Sie sich nicht auf normale Informationskanäle verlassen,
um ein genaues und rechtzeitiges Verständnis vom Zerfall des
Nationalstaates zu erhalten. Die „hartnäckige Täuschung``, die den
Untergang des Römischen Reiches verschleiert hat, ist wahrscheinlich ein
typisches Merkmal für den Zerfall großer politischer Einheiten. Sie
verschleiert und maskiert nun den Zusammenbruch des Nationalstaates. Aus
verschiedenen Gründen kann man sich nicht immer darauf verlassen, dass
die Nachrichten die Wahrheit sagen. Viele sind konservativ in dem Sinne,
dass sie die Parteien der Vergangenheit repräsentieren. Einige sind
durch anachronistische ideologische Verpflichtungen zum Sozialismus und
zum Nationalstaat geblendet. Einige werden aus konkreteren Gründen Angst
haben, die Korruption zu offenbaren, die in einem zerfallenden System
wahrscheinlich immer mehr zum Vorschein kommen wird. Einigen fehlt der
physische Mut, der für eine solche Aufgabe erforderlich sein könnte.
Andere fürchten um ihren Arbeitsplatz oder haben Angst vor anderen
Vergeltungsmaßnahmen, wenn sie ihre Meinung sagen. Und natürlich gibt es
keinen Grund zu der Annahme, dass Reporter und Redakteure weniger
anfällig für korrupte Überlegungen sind als Bauinspektoren oder
italienische Straßenbauunternehmer. Wichtige Informationsorgane, die
scheinbar alles melden wollen, können sich in größerem Umfang als
erwartet als weniger verlässliche Informationsquellen erweisen, als
gemeinhin angenommen wird. Viele werden andere Beweggründe haben,
einschließlich der Unterstützung für ein schwankendes System, die sie
einer ehrlichen Information vorziehen. Sie werden wenig sehen und noch
weniger erklären.

\section{JENSEITS DER REALITÄT}\label{jenseits-der-realituxe4t}

Mit der stetigen Verbesserung der Technologien für künstliche Realität
und Computerspiele werden Sie sogar in der Lage sein, einen
Abendnachrichtenbericht zu bestellen, der genau die Nachrichten
simuliert, die Sie sehen möchten. Möchten Sie einen Bericht sehen, in
dem Sie selbst als Sieger des Zehnkampfs bei den Olympischen Spielen
dargestellt werden? Kein Problem. Es könnte die Schlagzeile von morgen
sein. Sie werden jede Geschichte, die Sie sich wünschen, ob wahr oder
falsch, auf Ihrem Fernseher/Computer sehen, und zwar mit größerer
Wahrhaftigkeit als alles, was NBC oder die BBC derzeit vorweisen können.

Wir bewegen uns rasant auf eine Welt zu, in der Informationen so weit
von den Grenzen der Realität befreit sein werden, wie die menschliche
Erfindungsgabe es ermöglicht. Sicherlich wird dies enorme Auswirkungen
auf die Qualität und den Charakter der Informationen haben, die Sie
erhalten. In einer Welt der künstlichen Realität und der sofortigen
Übertragung von allem und jedem wird die Integrität des Urteils und die
Fähigkeit, das Wahre vom Falschen zu unterscheiden, noch wichtiger.

Diese Veränderung wird jedoch weniger von unseren gegenwärtigen
Umständen abweichend sein, als es viele Menschen vermuten würden. Die
Unterscheidungen zwischen wahr und unwahr werden häufig aus Gründen
verwischt, die durch Technologie verstärkt wurden. Wir sagen dies in dem
Bewusstsein, dass viele der Folgen der Informationsrevolution befreiend
waren.

Die Technologie hat bereits begonnen, geographische Nähe und politische
Dominanz zu überwinden. Regierungen können Hindernisse errichten, um den
Handel mit Waren zu behindern, aber sie können viel weniger tun, um die
Übertragung von Informationen zu stoppen. Fast jeder Gast in einem
beliebigen Restaurant in Hongkong ist über das Mobiltelefon mit dem Rest
der Welt verbunden. Die Hardliner-Putschisten in Moskau konnten im
August 1991 Jelzins Kommunikation nicht unterbrechen, weil er
Mobiltelefone hatte.

\subsection{Mehr Information, weniger
Verständnis}\label{mehr-information-weniger-verstuxe4ndnis}

Da die Barrieren zur Informationsübertragung gefallen sind, gibt es nun
mehr davon, was gut ist. Aber es gibt auch mehr Verwirrung darüber, was
etwas bedeutet. Die moderne Technologie, die hilft, Informationen von
politischen Kontrollen und zeitlich-räumlichen Hindernissen zu befreien,
neigt auch dazu, den Wert des altmodischen Urteilsvermögens zu erhöhen.
Die Art von Einsicht, die hilft, das Wichtige und Wahre aus dem Berg von
Fakten und Fantasien herauszufiltern, gewinnt fast täglich an Wert. Dies
ist aus mindestens drei Gründen der Fall:

\begin{enumerate}
\def\labelenumi{\arabic{enumi}.}
\tightlist
\item
  Angesichts der Flut von Informationen, die heute zur Verfügung stehen,
  ist es wichtig, sich kurz zu fassen.
\end{enumerate}

Kürze führt zu Verkürzung. Verkürzung lässt das Unvertraute weg. Wenn
man viele Fakten zu verdauen und zahlreiche Telefonanrufe zu erledigen
hat, ist das natürliche Bedürfnis, jeden
Informationsverarbeitungsvorgang so knapp wie möglich zu gestalten.
Leider bietet verkürzte Information oft eine schlechte Grundlage für das
Verständnis. Die tieferen und reicheren Texturen der Geschichte sind
genau die Teile, die in den 25-Sekunden-Soundbites tendenziell
weggeschnitten und auf CNN missverstanden werden. Es ist viel einfacher,
eine Nachricht zu vermitteln, die eine Variation eines bereits
verstandenen Themas ist, als ein neues Verständnisparadigma zu
erforschen. Man kann viel einfacher über ein Baseball- oder Cricketspiel
berichten, als zu erklären, wie Baseball oder Cricket gespielt wird und
was das bedeutet.

\begin{enumerate}
\def\labelenumi{\arabic{enumi}.}
\setcounter{enumi}{1}
\item
  Der rasante technologische Wandel untergräbt die megapolitische
  Grundlage der sozialen und wirtschaftlichen Organisation. Dies hat zur
  Folge, dass ein umfassendes paradigmatisches Verständnis oder
  unausgesprochene Theorien darüber, wie die Welt funktioniert,
  schneller als in der Vergangenheit veraltet sind. Damit steigt die
  Bedeutung des Gesamtüberblicks und sinkt der Wert einzelner „Fakten``,
  die für fast jeden, der über ein Informationssuchsystem verfügt,
  leicht zugänglich sind.
\item
  Die zunehmende Tribalisierung und Marginalisierung des Lebens haben
  den Diskurs und sogar das Denken beeinträchtigt. Viele Menschen haben
  sich infolgedessen angewöhnt, vor Schlussfolgerungen zurückzuscheuen,
  die offensichtlich von den ihnen zur Verfügung stehenden Fakten
  nahegelegt werden. Eine kürzlich durchgeführte psychologische Studie,
  die als Meinungsumfrage getarnt war, zeigte, dass Mitglieder einzelner
  Berufsgruppen fast einhellig jede Schlussfolgerung ablehnten, die
  einen Einkommensverlust für sie implizierte, unabhängig davon, wie
  stichhaltig die sie unterstützende Logik war. Angesichts der
  zunehmenden Spezialisierung ist die Mehrzahl der interpretativen
  Informationen über die meisten spezialisierten Berufsgruppen darauf
  ausgelegt, den Interessen der Gruppen selbst zu dienen. Sie haben
  wenig Interesse an Ansichten, die unhöflich, unprofitabel oder
  politisch inkorrekt sein könnten. Es gibt kein besseres Beispiel für
  diese allgemeine Tendenz als den breiten Trommelwirbel von Ansichten,
  der eine rosige Zukunft für Investitionen an der Börse nahelegt. Die
  meisten dieser Informationen werden von Brokerhäusern generiert, von
  denen nur wenige Ihnen sagen werden, dass Aktien überbewertet sind.
  Ihr Einkommen stammt aus dem Transaktionsgeschäft, das davon abhängt,
  dass die Mehrheit der Kunden bereit ist zu kaufen. Unabhängige,
  gegensätzliche Stimmen sind selten zu hören.
\end{enumerate}

Aus diesen und anderen Gründen ist das Zeitalter der Information noch
nicht das Zeitalter des Verstehens geworden. Im Gegenteil, es hat einen
starken Rückgang der Rigorosität in der öffentlichen Diskussion gegeben.
Die Welt könnte jetzt mehr wissen als zu irgendeinem früheren Zeitpunkt.
Aber es gibt kaum noch eine öffentliche Stimme, die die Bedeutung von
Ereignissen bewerten und sagen kann, was wahr ist. Das ist der Grund,
warum wir fasziniert davon sind, das laue Interesse, insbesondere in den
US-Medien, an der Berichterstattung über Hinweise auf sensationelle
Korruption auf hohen Ebenen der US-Regierung zu sehen.

Ein zentrales Thema, mit dem wir uns in diesem Buch auseinandergesetzt
haben, ist wie sich verändernde Technologien und andere
„megapolitische`` Faktoren die „natürliche Wirtschaft`` verändern. Die
„natürliche Wirtschaft`` ist der darwinsche „Naturzustand``, in dem
Ergebnisse, manchmal ungerecht, durch körperliche Kraft bestimmt werden.
In der „natürlichen Wirtschaft`` ist ein wichtiger Verhaltensstrang das,
was Biologen als „Interferenz-Konkurrenz`` bezeichnen.

\subsection{Interferenz-Konkurrenz}\label{interferenz-konkurrenz}

„Interferenz-Konkurrenten``, wie Jack Hirshleifer es ausdrückte,
„gewinnen und behalten die Kontrolle über Ressourcen, indem sie ihre
Rivalen direkt bekämpfen oder behindern.`` \footnote{Hirshleifer,
  ebenda, S. 176.} So sehr wir uns auch wünschen mögen, dass
menschliches Verhalten stets dem Gesetz und „anderen sozial
durchgesetzten Spielregeln`` („politische Ökonomie``) unterliegt, gibt
es ausreichend Beweise dafür, dass viele Menschen nur „nach den Regeln
spielen``, wenn es ihnen passt. Hirshleifer, eine Autorität in Sachen
Konflikt, formulierte es so: „Die Beharrlichkeit von Verbrechen, Krieg
und Politik lehrt uns, dass tatsächliche menschliche Angelegenheiten
weiterhin stark den zugrunde liegenden Druckmitteln der natürlichen
Wirtschaft unterliegen.`` \footnote{Ebenda, S. 169.}

Mit anderen Worten werden wirtschaftliche Ergebnisse nicht nur durch das
friedliche und gesetzestreue Verhalten des in Lehrbüchern beschriebenen
Homo oeconomicus bestimmt, der Eigentumsrechte respektiert „und einfach
nicht nehmen wird, was ihm nicht gehört``.\footnote{Michelle R.
  Garfinkel und Stergios Skaperdas, Hrsg., \emph{The Political Economy
  of Conflict and Appropriation} (Cambridge: Cambridge University Press,
  1996), S. 1.} Tatsächliche Ergebnisse werden auch durch Konflikte
geformt, einschließlich offen ausgeübter Gewalt. Wie der Ökonom
Hirshleifer hervorhebt, wird „selbst unter Gesetz und Regierung der
rationale, selbstinteressierte Mensch zwischen legalen und illegalen
Mitteln zur Beschaffung von Ressourcen - zwischen Produktion und
Austausch einerseits und Diebstahl, Betrug und Erpressung andererseits -
ein Gleichgewicht finden.`` \footnote{Hirshleifer, ebenda, S. 173.}

\section{RAUBÜBERFALL IM
INFORMATIONSZEITALTER}\label{raubuxfcberfall-im-informationszeitalter}

Michelle R. Garfinkel und Stergios Skaperdas untersuchen dies in ihrem
aufschlussreichen Buch über Gewalt, Kriminalität und Politik, \emph{The
Political Economy of Conflict and Appropriation}: „Individuen und
Gruppen können entweder produzieren und somit Reichtum schaffen oder den
von anderen erschaffenen Reichtum an sich reißen.`` \footnote{Garfinkel
  und Skaperdas, ebenda, S. 1.} Sie zitieren eine Geschichte moderner
Interferenz-Konkurrenz, die ursprünglich vom \emph{Economist} berichtet
wurde: „Ein amerikanischer Geschäftsmann, der gerade in Moskau
angekommen war, um ein Büro zu eröffnen, wurde in seinem Hotel von fünf
Männern mit goldenen Uhren, Pistolen und einem Ausdruck des
Nettovermögens seiner Firma empfangen. Sie verlangten 7\% der künftigen
Einnahmen. Er nahm den ersten Flug nach New York, wo Straßenräuber
weniger raffiniert sind.`` \footnote{Ebenda} Diese Geschichte des
Raubüberfalls im Informationszeitalter verdankt der neuen Technologie
mehr, als der einfachen Tatsache, dass russische Schläger nun über das
Internet Zugang zu Finanzprofilen und Kreditauskünften ihrer Opfer
haben.

\subsection{Sinkende Entscheidungskraft der militärischen
Macht}\label{sinkende-entscheidungskraft-der-milituxe4rischen-macht}

Zum Guten wie zum Schlechten hat die Informationstechnologie durch die
Verringerung der Bedeutung von Großmächten das Entscheidungspotenzial
des Nationalstaates in einer unruhigen Welt drastisch reduziert. Wo
einst, wie Voltaire sagte, „Gott auf der Seite der größeren Bataillone
war``, scheint es mit jedem vergehenden Tag weniger göttliche
Unterstützung für die Erzeugung großer Gewaltrenditen zu geben.
Stattdessen sehen wir das Gegenteil - mehr Hinweise auf sinkende
Gewaltrenditen - was stark darauf hindeutet, dass große Konglomerate wie
der Nationalstaat ihre enormen Gemeinkosten nicht mehr rechtfertigen
können.

Das offensichtlichste Anzeichen für die sinkende Entscheidungsmacht
zentralisierter Gewalt ist der Anstieg des Terrorismus. Hochkarätige
Bombenanschläge in den USA in den mittleren neunziger Jahren zeigen,
dass selbst die größte militärische Supermacht der Welt nicht vor
Angriffen gefeit ist.

Eine weitere wichtige Erscheinung des Rückgangs der Gewaltbereitschaft
ist die weltweite Zunahme des Gangstertums und der organisierten
Kriminalität sowie der damit einhergehenden politischen
Vetternwirtschaft und Korruption. Sie spiegeln eine allgemein
unmoralische Atmosphäre wider, in der der Staat zwingen, aber nicht
schützen kann. In dem Maße, in dem das Gewaltmonopol zerfällt, drängen
neue Konkurrenten auf den Plan, wie die Schläger, die dem amerikanischen
Geschäftsmann in Moskau ihre eigenen privaten Steuern aufzwingen
wollten.

Kleine Gruppen, Stämme, Triaden, Banden, Gangster, Mafias, Milizen und
sogar einzelne Individuen haben zunehmend militärische Wirksamkeit
erlangt. Sie werden in der „natürlichen Wirtschaft`` des nächsten
Jahrtausends viel mehr reale Macht ausüben, als sie es im zwanzigsten
Jahrhundert getan haben. Waffen, die Mikrochips verwenden, neigen dazu,
das Machtgleichgewicht in Richtung Verteidigung zu verschieben, wodurch
entscheidende Aggressionen weniger lukrativ und daher weniger
wahrscheinlich werden. Intelligente Waffen, wie Stinger-Raketen
beispielsweise, neutralisieren effektiv einen Großteil des Vorteils, den
große, wohlhabende Staaten bisher beim Einsatz teurer Luftwaffen gegen
ärmere, kleinere Gruppen genossen haben.

\subsection{Informationskrieg voraus}\label{informationskrieg-voraus}

Unmittelbar bevor steht die heiß diskutierte, aber wenig verstandene
Möglichkeit des „Informationskrieges``. Dies weist auch auf abnehmende
Gewaltrenditen hin. „Logikbomben`` könnten Flugverkehrskontrollsysteme,
Eisenbahnweichensysteme, Stromerzeuger und -verteilungsnetze, Wasser-
und Abwassersysteme, Telefonrelais, sogar die eigenen
Kommunikationssysteme des Militärs lahmlegen oder sabotieren. Da
Gesellschaften immer mehr von computerisierten Kontrollsystemen abhängig
werden, könnten „Logikbomben`` fast genauso viel Schaden verursachen wie
physische Explosionen.

Im Gegensatz zu herkömmlichen Bomben könnten „Logikbomben`` nicht nur
von feindlichen Regierungen, sondern auch von Gruppen freiberuflicher
Computerprogrammierer und sogar talentierten einzelnen Hackern aus der
Ferne gezündet werden. Beachten Sie, dass 1996 ein argentinischer
Teenager verhaftet wurde, weil er wiederholt in die Computer des
Pentagons eingedrungen ist. Zwar haben Hacker bisher nicht dazu geneigt,
computergesteuerte Systeme auf zerstörerische Weise zu manipulieren,
aber das liegt nicht daran, dass es wirklich wirksame Mittel gibt, sie
zu stoppen.

Wenn das Zeitalter des Informationskrieges endlich anbricht, ist es
unwahrscheinlich, dass seine Antagonisten nur Regierungen sein werden.
Ein Unternehmen wie Microsoft hat sicherlich eine größere Fähigkeit,
einen Informationskrieg zu führen, als 90 Prozent der Nationalstaaten
der Welt.

\subsection{Das Zeitalter des souveränen
Individuums}\label{das-zeitalter-des-souveruxe4nen-individuums}

Dies ist Teil des Grundes, warum wir dieses Buch „Das souveräne
Individuum`` genannt haben. Da der Umfang der Kriegsführung abnimmt,
werden Verteidigung und Schutz auf kleinerer Ebene organisiert. Daher
werden es zunehmend private statt öffentliche Güter sein, die von
privaten Auftragnehmern auf Basis von Gewinnstreben bereitgestellt
werden. Dies zeigt sich bereits in der Privatisierung der Polizeiarbeit
in Nordamerika. Eine der am schnellsten wachsenden Berufsgruppen in den
USA ist der „Sicherheitswachmann``. Prognosen deuten darauf hin, dass
die Anzahl der privaten Sicherheitskräfte bis zum Jahr 2005 um 24 bis 40
Prozent über den Werten von 1990 zunehmen wird.\footnote{Hamish McRae,
  \emph{The World in 2020} (London: Harper Collins, 1995), S. 188.}

Die Privatisierung der Polizei ist bereits ein deutlich erkennbarer
Trend. Wie jedoch der anglo-irische Experte Hamish McRae hervorhebt, ist
dies kaum das Ergebnis einer bewussten Entscheidung der Regierung. Er
schreibt dazu in „Die Welt im Jahr 2020``:

\begin{quote}
Weder hat irgendeine Regierung die spezielle Entscheidung getroffen,
sich aus bestimmten Polizeiaufgaben zurückzuziehen, noch haben sie sich
tatsächlich zurückgezogen; der Privatsektor ist vielmehr
hineingegrätscht. Dies ist teilweise eine Folge der wahrgenommenen
Versäumnisse der Polizei, aber auch eine Folge anderer
gesellschaftlicher Veränderungen. Private Sicherheitsfirmen übernehmen
allmählich einen Großteil der Aufgabe, normale Bürger in ihren Büros
oder Einkaufszentren zu schützen. Wie die geschlossenen Wohnanlagen von
Los Angeles zeigen, bewegen sich die Menschen sogar wieder ein Stück
weit in Richtung des mittelalterlichen Stadt-Konzepts, wo die Bewohner
hinter Stadtmauern leben, die von Wachen patrouilliert werden, und der
Zugang nur an kontrollierten Toren möglich ist.\footnote{Ebenda, S.
  188-89.}
\end{quote}

Wir glauben, dass dies nur ein Vorgeschmack auf eine umfassendere
Privatisierung fast aller Funktionen ist, die von den Regierungen im
zwanzigsten Jahrhundert bislang ausgeübt wurden. Da die
Informationstechnologie die Fähigkeit zentralisierter Autoritäten
untergraben hat, Macht auszuüben und physische Sicherheit für Systeme zu
gewährleisten, die in großem Maßstab operieren, sinkt die optimale Größe
fast jedes Unternehmens in der „natürlichen Wirtschaft``.

Um auf diese technologische Veränderung zu reagieren, wird ein massiver
Investitionsbedarf (besser: Chance) erforderlich sein, um anfällige
Systeme neu zu gestalten, mit dezentralisierten anstatt konzentrierten
Fähigkeiten. Wenn Schwachstellen im großen Maßstab nicht beseitigt
werden, werden die Systeme, die sie beibehalten, einem katastrophalen
Ausfall unterliegen.

Früher oder später werden sich Dienstleistungen und Produkte, die von
großen bürokratischen Behörden und Unternehmen angeboten werden, in
stark wettbewerbsorientierte Märkte verwandeln, die nicht von einem
„Hauptquartier`` aus, sondern über ein verteiltes, dezentralisiertes
Netz verwaltet werden. Das Unternehmen mit einem Hauptsitz, der von
Streikposten umgeben oder von Terroristen sabotiert werden kann, wird
anfällig sein, bis er letztendlich zu einer „virtuellen Firma`` ohne
fixem Standort wird, „die gleichzeitig an vielen Orten wohnt``, wie
Kevin Kelly, leitender Redakteur des Wired Magazins, in \emph{Out of
Control} schreibt.\footnote{Kevin Kelly, \emph{Out of Control} (Reading,
  Mass.: Addison-Wesley, 1994), S. 189} Kelly weiß, dass die Technologie
die Notwendigkeit, Produktionsprozesse unter zentrale Kontrolle zu
bringen, verändert hat. „Während des größten Teils der industriellen
Revolution wurde ernsthafter Wohlstand durch die Zusammenführung von
Prozessen unter einem Dach geschaffen. Größer war effizienter.`` Heute
ist das nicht mehr der Fall.

Kelly sieht die Möglichkeit voraus, dass das Auto der Zukunft, das
Upstart Car, möglicherweise von nur einem Dutzend Menschen entworfen und
in Produktion gebracht werden könnte, die in einer virtuellen Firma
zusammenarbeiten.

In der Zukunft könnte übermäßige Größe nicht nur kontraproduktiv,
sondern auch gefährlich sein. Größere Unternehmen stellen verlockendere
Ziele dar. Wie Praktiker der Untergrundwirtschaft zeigen, ist einer der
Geheimnisse zur Vermeidung von Steuern, die Vermeidung von Entdeckung.
Dies wird für kleinere, „virtuelle Unternehmen`` viel einfacher sein als
für altmodische Konzerne, die von einer Hochhauszentrale aus operieren,
deren Name in Leuchtschrift über der Tür steht. Sie sind zwangsläufig
anfälliger für die Aufmerksamkeit von „Männern mit goldenen Uhren,
Pistolen und einem Ausdruck des Nettovermögens der Firma``, den
Gangstern, die in anderen Teilen der Welt ihre eigene Art der
Besteuerung durchsetzen werden, wie sie es in Russland tun. Unternehmen
aller Größenordnungen werden anfällig für kriminelle Ausplünderungen und
Auflagen durch organisierte kriminelle Banden.

\begin{quote}
„Betrachten Sie die Definition eines Erpressers als jemanden, der eine
Bedrohung schafft und dann Gebühren für deren Reduzierung verlangt. Im
Vergleich dazu kann oft die Bereitstellung von Schutz durch Regierungen
als Erpressung eingestuft werden.`` \footnote{Tilly, \emph{War Making
  and State Making as Organized Crime}, in Evans, Rueschemeyer, und
  Skoepol, ebenda, S. 171.} - Charles Tilly
\end{quote}

\subsection{Die Natur hasst Monopole}\label{die-natur-hasst-monopole}

Sobald das auf Gewalt basierende Monopol der „größeren Bataillone``
zerfällt, ist einer der ersten zu erwartenden Auswirkungen ein
zunehmender Wohlstand für das organisierte Verbrechen. Immerhin stellt
das organisierte Verbrechen die größte Konkurrenz für Nationalstaaten
dar, wenn es darum geht, Gewalt für räuberische Zwecke einzusetzen.
Obwohl es unhöflich ist, dies zu sagen, sollte nicht vergessen werden,
wie uns der Politikwissenschaftler Charles Tilly erinnert, dass
Regierungen selbst -- „die Schutzgelderpresser schlechthin mit dem
Vorteil der Legitimität - als unsere größten Beispiele für organisierte
Kriminalität gelten.`` \footnote{Ebenda, S. 169.}

Wenn Sie nichts anderes über die Welt wüssten, als dass ein wichtiges
Monopol gerade zusammenbricht, wäre eine der einfachsten und sichersten
Vorhersagen, die Sie treffen könnten, dass seine nächsten Wettbewerber
am meisten profitieren würden. Es ist daher kein Zufall, dass
Drogenkartelle, Banden, Mafias und Triaden verschiedenster Art auf der
ganzen Welt zunehmen.

\subsection{Sistema del Potere}\label{sistema-del-potere}

Von Russland über Japan bis hin zu den Vereinigten Staaten ist die
organisierte Kriminalität ein weit wichtigerer Faktor im Betrieb von
Volkswirtschaften, als Wirtschaftsbücher Sie glauben lassen würden. Was
die Sizilianer das „\emph{sistema del potere}``, das „System der
Macht``, der organisierten Kriminalität nennen, spielt eine zunehmend
wichtige Rolle dabei, zu bestimmen, wie Volkswirtschaften funktionieren.

Europäische Polizeibeamte berichten, dass internationale
Verbrechensyndikate, darunter die russische und italienische Mafia,
„eine dominante Rolle`` bei der Finanzierung der genozidalen Kriege
spielten, die in den letzten Jahren auf dem Balkan wüteten.

Drogenschmuggler haben ebenfalls eine Schlüsselrolle bei der
Finanzierung jüngster Bürgerkriege und Aufstände in anderen Teilen der
Welt gespielt. Julio Fernandez, Chef der spanischen Drogenpolizei in
Katalonien, sagt: „Von 1986 bis 1988 wurden 80 Prozent des Heroins in
Spanien von Tamil Tiger Guerillas transportiert, die mit pakistanischen
Bewohnern in Barcelona oder Madrid zusammenarbeiteten. Als wir dieses
Netzwerk durch Festnahmen zerschlagen haben, wurde es durch Kurden aus
der Türkei ersetzt, die es in den nächsten zwei Jahren völlig
dominierten.`` \footnote{Frank Viamo, \emph{The New Mafia Order}, Mother
  Jones, Mai/Juni 1995, S. 55.} Es ist sehr wahrscheinlich, dass immer
dann, wenn ein neuer Bürgerkrieg oder Aufstand beginnt, die
verzweifelten armen Kämpfer ihre militärischen Bemühungen durch den
Transport von Drogen und die Geldwäsche von Drogengeld finanzieren.

\subsection{Drogenfinanzierte Rabatte}\label{drogenfinanzierte-rabatte}

Die Aktivitäten organisierter krimineller Syndikate üben einen
Abwärtsdruck auf die Preise von Waren aus, mit Ausnahme von Drogen. Auf
mikroökonomischer Ebene subventionieren Verbrechersyndikate offenbar
legitime Geschäfte mit der Beute aus kriminellen Unternehmungen. Sie
können Drogengewinne und andere illegale Gelder waschen, indem sie
gewöhnliche Waren unterhalb der Kosten verkaufen, wodurch sie die Preise
ihrer sauberen Wettbewerber unterbieten und viele davon aus dem Geschäft
drängen.

\subsection{Yakuza-Deflation}\label{yakuza-deflation}

In Japan spielten die mächtigen Yakuza-Gangs eine Schlüsselrolle in der
hyperaktiven Immobilienblase der späten 1980er Jahre. Trotz der
Tatsache, dass die neunzigtausend Yakuza jährlich zwischen 10,19
Milliarden Dollar (offizielle Schätzung) und 71,35 Milliarden Dollar
(Schätzung von Professor Takatsugu Nato) verdienen, wurde ein hoher
Anteil der uneinbringlichen Kredite, die die Zahlungsfähigkeit der
japanischen Banken gefährdet haben, durch Yakuza-unterstützte Geschäfte
gemacht.\footnote{Siehe Velisarios Kattoulas,
  \emph{Japan\textquotesingle s Yakuza Claim Place Among Criminal
  Elite}, Washington Times, 25. November 1994, S. A22.} Der
Deflationsdruck - die „Preiszerstörung``, wie die Japaner es nennen -
die Japans Wirtschaft prägten, sind eine Folge davon.

\subsection{Ein Auge zudrücken}\label{ein-auge-zudruxfccken}

Die Mafias Russlands, wie Jelzin selbst zugegeben hat, sind mit
„kommerziellen Strukturen, Verwaltungsbehörden, Innenministerien und
städtischen Autoritäten ...`` \footnote{Viamo, ebenda, S. 49.}
verschmolzen. Aufgrund der Immunität, die die Mafias durch die
Verschmelzung mit der Polizei erreicht haben, sind sie in der Lage, die
Einziehung ihrer Privatsteuern durch offenkundige Gewalt zu erzwingen.
Autoritative Quellen weisen darauf hin, dass vier von fünf russischen
Unternehmen nun „Schutzgeld`` zahlen. „Laut einigen Berichten müssen
lokale Kleinunternehmen in Russland 30 bis 50 Prozent ihrer Gewinne an
Räuber abgeben, nicht nur die mageren 7 Prozent, die vom amerikanischen
Geschäftsmann verlangt werden.`` \footnote{Garfinkel und Skaperdas,
  ebenda, S. 2.}

Im Jahr 1993 gab es in Russland offiziell 355.500 als „gangsterhaft``
eingestufte Verbrechen, darunter fast „30.000 vorsätzliche Morde``,
hauptsächlich Gang-Attentate an Geschäftsleuten. Laut einem ehemaligen
Innenminister, General Viktor Yerin, „waren die meisten davon
Auftragsmorde aufgrund von Konflikten im Bereich der kommerziellen und
finanziellen Aktivitäten``. In den meisten Fällen drückten die Behörden
„ein Auge zu``. Kriminelle Organisationen spielen, „durch ihre Kontrolle
über Zwang und Korruption``, wie die Ökonomen Gianluca Fiorentini und
Sam Peltzman in \emph{The Economics of Organized Crime} schreiben, eine
Schlüsselrolle in der Wirtschaft.\footnote{Fiorentini und Peltzman,
  ebenda, S. 15.} Theoretisch kann dieser Einfluss manchmal vorteilhaft
sein, da er die Regulierung einschränkt und Regierungen möglicherweise
dazu ermutigt, ihre Bereitstellung öffentlicher Güter zu verbessern. Die
Präsenz einer mächtigen Mafia „schränkt die monopolistische Rolle der
Regierungsbehörden ein``.\footnote{Ebenda.} Regierungen in Gebieten mit
mächtigen organisierten Kriminalitätsgruppen können nur mit großer
Schwierigkeit Maßnahmen verfolgen, denen die Mafias widersprechen.

\subsection{Geheime Absprachen}\label{geheime-absprachen}

Tatsächlich ist es bemerkenswert, wie selten die meisten Regierungen
bereit sind, die Mafias, ihre Hauptkonkurrenten in der Anwendung
organisierter Zwangsmaßnahmen, direkt zu konfrontieren. Aus rein
wirtschaftlicher Hinsicht ist dies nicht überraschend. Die profitabelste
Vereinbarung, die „die gewählten Mitglieder der öffentlichen
Verwaltung`` treffen können, ist eine „geheime Absprache`` mit dem
organisierten Verbrechen. Fiorentini und Peltzman stellen fest, dass „es
Beweise für groß angelegte Vereinbarungen gibt, bei denen das
organisierte Verbrechen politische Unterstützung für Kandidatengruppen
sicherstellt, während diese den Gefallen durch eine günstige Verwaltung
öffentlicher Beschaffungen und die Bereitstellung öffentlicher
Dienstleistungen oder Subventionen erwidern.`` \footnote{Ebenda, S. 16.}

Im Gegensatz zu dem von Hollywood vermittelten Eindruck scheint es nun,
dass das Durchdringen und Betrügen von Regierungen zu einem der
Hauptziele krimineller Organisationen wie der sizilianischen Mafia
geworden ist. „Die meisten Forscher sind mittlerweile der Auffassung,
dass das größte Geschäft der sizilianischen Mafia gerade darin besteht,
sich verschiedene Quellen öffentlicher Ausgaben anzueignen und Betrug
gegen lokale, nationale und europäische Gemeinschaftsförderprogramme zu
organisieren.`` \footnote{Ebenda.}

\subsection{„Narco-Staaten``}\label{narco-staaten}

Wie wir in \emph{The Great Reckoning} gewarnt haben, sind viele
Regierungen auf der Welt durch Drogenbarone gründlich korrumpiert.
Mexiko ist ein unbestreitbares Beispiel. Der ehemalige mexikanische
Bundesvize-Generalstaatsanwalt Eduardo Valle Espinosa brachte das
mexikanische System in seiner Rücktrittserklärung auf den Punkt:
„Niemand kann ein politisches Projekt entwerfen, in dem die Köpfe des
Drogenhandels und ihre Finanziers nicht berücksichtigt werden. Denn wenn
Sie es tun, sterben Sie.`` Valle deutete an, dass Bestechungsgelder das
Amt des mexikanischen Polizeichefs so lukrativ machen, dass Kandidaten
bis zu 2 Millionen Dollar zahlen, nur um eingestellt zu werden. In einer
strikten Gewinn- und Verlustrechnung kann der Erwerb einer regionalen
Polizeibehörde eine lukrative Investition sein. Drogenkartelle sind
bereit, auch niederrangigen mexikanischen Beamten hohe Summen zu zahlen,
weil das Geld ihnen Immunität vor Strafverfolgung für ihre Verbrechen
bietet.

Kolumbien ist ein weiteres Land, in dem die obersten Ränge der Regierung
von Drogenbaronen dominiert werden. Die US-Behörden haben kürzlich das
US-Visum des kolumbianischen Präsidenten Ernesto Samper abgelehnt, mit
der Begründung, dass er wissentlich politische Spenden von
Drogenhändlern im Austausch für Gefälligkeiten erhalten habe.

\subsection{Ein Esel schimpft den anderen
Langohr}\label{ein-esel-schimpft-den-anderen-langohr}

Jeder, der die Berichte in unserem Newsletter, \emph{Strategic
Investment}, während der 1990er Jahre verfolgt hat, wird sofort die
Ironie in der Haltung der Clinton-Administration gegenüber Samper
erkennen. Es gibt glaubwürdige Beweise dafür, dass der US-Präsident Bill
Clinton selbst all das getan hat, was Samper vorgeworfen wird und noch
Schlimmeres. Selbst wenn Sie uns nicht glauben, wird Clintons
Hintergrund in schillernden Details in zwei gründlich recherchierten
Büchern von Autoren auf gegenüberliegenden Seiten des politischen
Spektrums hervorgehoben.

Roger Morris, der meist eine linke Perspektive einnimmt, war Beamter für
nationale Sicherheit in der Nixon-Regierung sowie ein leitender Berater
für Dean Acheson, Präsident Lyndon B. Johnson und Walter Mondale. Morris
hat einen Doktortitel von der Harvard University. Sein Buch,
\emph{Partners in Power}, enthüllt eine schmuddelige Vergangenheit für
Clinton, die Ernesto Samper wie einen Pfadfinder erscheinen lässt.

Morris erzählt von Clintons vaterloser Kindheit in Hot Springs,
Arkansas, einem Zentrum für Glücksspiel, Prostitution und organisiertem
Verbrechen, zu dem der Großteil seiner Familie eine gewisse Verbindung
hatte. Clintons Stiefonkel, Raymond Clinton, der für Bill Clinton eine
„Vaterfigur`` war, galt angeblich als eine führende Figur des „Paten``
in der Dixie-Mafia. Morris behauptet, dass Bill Clinton ein CIA-Rekrut
wurde und seine Studententage in Oxford damit verbrachte,
Anti-Vietnamkriegsaktivisten zu überwachen. So wie Morris die Dinge
sieht, blieb Clinton ein CIA-Agent während seiner Amtszeit als
Gouverneur, und erleichterte eine auf Mena, Arkansas, zentrierte
CIA-Drogen- und Waffenhandelsoperation. Morris scheint die gesamte CIA
des Drogenhandels zu beschuldigen, anstatt die Möglichkeit in Betracht
zu ziehen, dass Clinton sich mit einer korrupten Fraktion der Agentur
verbündet hat, was uns wahrscheinlicher erscheint. Jede Interpretation
deutet jedoch darauf hin, dass die Hauptgeheimdienstagentur der
US-Regierung entweder direkt oder indirekt an organisiertem Drogenhandel
im großen Maßstab beteiligt ist. Wenn die CIA nicht ein Anhängsel des
organisierten Verbrechens ist, ist sie gefährlich dicht daran, es zu
werden.\footnote{Für weitere eindeutige Beweise für die Beteiligung der
  CIA am Drogenhandel, siehe Michael Levine, The Big White Lie: The Deep
  Cover Operation That Exposed the CIA Sabotage of the Drug War (New
  York: Thunder\textquotesingle s Mouth Press, 1994).}

\subsection{Chance 1:250.000.000}\label{chance-1250.000.000}

Trotzdem enthält \emph{Partners in Power} Details, die jeden Studenten
über Korruption der modernen amerikanischen Politik interessieren
würden. Morris richtet seine Kritik jedoch keineswegs nur an Bill
Clinton. Auch seine Frau erhält einige kritische Aufmerksamkeit.
Betrachten Sie zum Beispiel diesen Auszug aus Morris' Bericht über
Hillary Clintons wundersamen Handel mit Rohstoffen: „Im Jahr 1995
führten Ökonomen der Universitäten Auburn und North Florida ein
ausgeklügeltes Computerstatistikmodell der Geschäfte der First Lady zur
Veröffentlichung in der Zeitschrift \emph{Journal of Economics and
Statistics} durch, wobei sie alle verfügbaren Unterlagen sowie
Marktinformationen aus dem \emph{Wall Street Journal} nutzten. Die
Wahrscheinlichkeit, dass Hillary Rodham ihre Geschäfte auf legitime
Weise gemacht hat, berechneten sie, war weniger als eins zu
250.000.000.`` \footnote{Roger Morris, \emph{Partners in Power} (New
  York: Henry Holt, 1996), S. 233.}

Morris führt viele belastende Details über den Drogenhandel und die
Geldwäsche auf, die unter Clinton in Arkansas florierten. „Durch das
schiere Ausmaß der Drogen und des Geldes, das durch die Flüge generiert
wurde, wurde das winzige Mena, Arkansas, in den 1980er Jahren zu einem
der weltweiten Zentren des Drogenhandels.`` \footnote{Ebenda, S. 393.}
Morris zitiert einen Vertrauten, der über Clinton aussagte: „Er wusste
Bescheid.``

Clinton wusste nicht nur vom Kokainschmuggel, sondern sagte laut dem
State Trooper L.D. Brown, einem ehemaligen Leibwächter, dem Clinton zu
einer Position bei der CIA verholfen hatte, dass der Drogenhandel keine
CIA-Operation sei. „‚Oh, nein', sagte Clinton, ‚Das ist ein Geschäft von
Lasater.'\,`` \footnote{Ebenda, S. 411.}

Dan Lasater, ein verurteilter Kokainhändler, war einer von Clintons
wichtigsten finanziellen Unterstützern. Ein Mann, der Millionen mit
Geschäften in Arkansas verdiente und angeblich dem damaligen Gouverneur
von Kentucky, John Y. Brown, 300.000 Dollar in bar in einer braunen
Papiertüte gab. Laut Morris war Lasater „nie nur ein weiterer großer
Spender, dem besondere Beachtung geschenkt wurde, sondern ein
außergewöhnlich enger Vertrauter, den Clinton regelmäßig in seinem
Maklerbüro besuchte und der nach Belieben zu seiner Villa kam.``
\footnote{Ebenda, S. 418.} Morris berichtet, dass Lasaters Chauffeur,
der ihn oft zur Villa brachte, „ein verurteilter Mörder war, der eine
Waffe trug und von dem allgemein bekannt war, dass er nebenbei mit
Drogen handelte.`` \footnote{Ebenda.} Laut Morris scheint der Präsident
der Vereinigten Staaten ein engeres Verhältnis zu einem Drogenhändler
gehabt zu haben als die behauptete Beziehung zwischen dem
kolumbianischen Präsidenten Ernesto Samper und dem Cali-Kartell.

\begin{quote}
„Puh! Bob sagt Dinge über Bill Clinton, die selbst Hillary nicht sagen
würde`` - P.J. O'Rourke
\end{quote}

R. Emmett Tyrell Jr., der Chefredakteur von \emph{The American
Spectator}, ist kein Linker wie Morris. Doch sein Bericht \emph{Boy
Clinton} enthält viele der gleichen Details, die auch Morris nutzt, um
ein Bild von Clinton als einem korrupten Politiker zu zeichnen, der eng
mit Drogenhandel und anderen Verbrechen verbunden ist. Im Prolog von
\emph{Boy Clinton} wird L. D. Brown, Clintons ehemaliger Leibwächter,
zitiert, der die sensationelle Behauptung aufstellt, Clinton sei in die
Aktivitäten der Todesschwadrone verwickelt gewesen, um Zeugen zu
beseitigen, die über den Drogenhandel in Mena Bescheid wussten.

Insbesondere bezeugt Brown, dass er persönlich am 18. Juni 1986 nach
Puerto Vallarta, Mexiko, entsandt wurde, mit einem belgischen leichten
automatischen F.A.L. Gewehr. Unter dem Pseudonym Michael Johnson sollte
Brown Terry Reed erschossen und getötet haben.

Reed, wie Sie sich vielleicht erinnern, geriet 1994 als Mitautor von
\emph{Compromised: Clinton, Bush and the CIA} in das öffentliche
Interesse. Die These von \emph{Compromised} ist, dass die CIA „die
Präsidentschaft vereinnahmt`` hat und dass ihre „verdeckten Operationen
wie ein Krebsgeschwür die Organe der Regierung metastasiert haben``.
Genauer gesagt, behaupten Reed und sein Mitautor, dass sowohl Clinton
als auch Bush tief in illegale Aktivitäten in Arkansas verwickelt waren,
einschließlich des Drogenhandels.

Brown hat Reed nicht, wie befohlen, getötet. Er und Reed schafften es,
zu überleben und zumindest einen Teil ihrer Geschichten zu erzählen,
womit sie mehr Glück hatten als andere, die damals und später mit
Clinton in Verbindung standen. Betrachten Sie den späten Jerry Parks,
der 1992 für die Sicherheit des Clinton-Gore Hauptquartiers zuständig
war und im September 1993 in einer gangsterhaften Ermordung erschossen
wurde. In einer weiteren bizarren Wendung dieser verdrehten Geschichte
hat der Londoner \emph{Sunday Telegraph} auf der Grundlage von
exklusiven Informationen, die von Parks' Witwe zur Verfügung gestellt
wurden, enthüllt, dass Parks von dem verstorbenen Vincent Foster
beauftragt wurde, Bill Clinton zu bespitzeln.

Warum Foster ein belastendes Dossier über Clinton anlegen wollte, kann
nur vermutet werden. (Er behauptete, er tue es für Hillary.) Aber in
jedem Fall widerlegt es die offizielle Darstellung von Foster als naiven
Landjungen, der durch die rücksichtslosen Methoden Washingtons so
schockiert war, dass er aus Verzweiflung Selbstmord beging. Diese
ohnehin schon unwahrscheinliche Geschichte wird mit jeder neuen
Enthüllung noch unwahrscheinlicher.\footnote{Gründlicher Überblick über
  die Foster-Geschichte, siehe Christopher Ruddy, Vincent Foster:
  \emph{The Ruddy Investigation}, verfügbar für \$19.95 von
  1-800-711-1968.}

\subsection{Der Mafia-Präsident}\label{der-mafia-pruxe4sident}

Während die Welt im Großen und Ganzen vor der verstörenden
Schlussfolgerung zurückschreckt, dass der Präsident der Vereinigten
Staaten durch eine enge Verbindung mit organisierter Kriminalität und
Kriminellen belastet ist, deutet das Beweismaterial genau darauf hin.
Morris zitiert einen ehemaligen US-Staatsanwalt, der organisierte
Verbrecher und deren Interessen verfolgte. Er behauptet, dass Clintons
Wahl zum Gouverneur im Jahr 1984 „die Wahl war, bei der die Mafia
wirklich in die Politik von Arkansas eintrat, die Jungs von den Hunde-
und Pferderennen, die Schmiergeldzahler, die eine Gelegenheit erkannten
... es ging weit über unsere alte Dixie-Mafia hinaus, die im Vergleich
dazu wie Kleinkinder wirkten. Es handelte sich um kriminelles Geld von
der Ost- und Westküste, das die Möglichkeiten genauso erkannte wie die
legalen Konzerne.`` \footnote{Morris, ebenda, S. 331.}

Anscheinend haben auch andere mit ähnlichen Einstellungen weiterhin die
Möglichkeiten mit Clinton wahrgenommen. Das \emph{New Yorker} Magazin
berichtet in Anlehnung an einen früheren Artikel in \emph{Readers'
Digest}, dass „die entscheidenden Verbündeten des Präsidenten in der
Gewerkschaftsbewegung ebenfalls Männer sind, die mit dem, was allem
Anschein nach einige der schmutzigsten, von der Mafia durchdrungensten
Gewerkschaften in Amerika sind, in Verbindung stehen.`` \footnote{Siehe
  Jeffrey Goldberg, \emph{Some of the President's New Union Pals Seem to
  Have Some Suspicious Pals of Their Own}, New York, 9. Juli, 1996, S.
  17.} Von besonderem Interesse ist Clintons enge Beziehung zu Arthur
Coia. Coia, der einer von Clintons „wichtigsten Geldbeschaffern`` ist,
ist Präsident der Laborers International Union of North America, „einer
der korruptesten Gewerkschaften in der Geschichte der
Arbeiterbewegung``.\footnote{Ebenda, S. 19.}

Offensichtlich hat das Justizministerium unter Clinton einen Deal mit
Coia geschlossen, den \emph{New Yorker} als „seltsam großzügig``
beschreibt, „damit er seinen Job behält, obwohl dasselbe
Justizministerium ihn seit langem der organisierten Kriminalität
zuordnet``.\footnote{Ebenda.}

Ob Terry Reeds These richtig ist, dass „die CIA die Präsidentschaft
übernommen hat``, ist ungewiss, aber es gibt offensichtlich eine starke
Versuchung für Individuen innerhalb einer verdeckten Organisation, die
autorisiert ist, „Verdeckte Operationen`` durchzuführen, sich der
rationalen Wahl von Professor Hirshleifer zu bedienen und „unrechtmäßige
Mittel zur Beschaffung von Ressourcen`` zu nutzen.

Angesichts des technologischen Wandels, der die Entscheidungsgewalt von
konzentrierter militärischer Macht in der Welt verringert, sollte man
vielleicht eine zunehmende Korruption erwarten, wenn nicht sogar eine
direkte Übernahme von Regierungen durch organisierte kriminelle
Unternehmungen.

Hirshleifer argumentiert, und wir stimmen zu, dass „die Institutionen
der politischen Ökonomie niemals so perfekt sein können, dass sie... die
zugrunde liegenden Realitäten der natürlichen Wirtschaft`` vollständig
verdrängen.\footnote{Hirshleifer, ebenda, S. 173.} Die Macht verlagert
sich in der „natürlichen Wirtschaft``. Dies impliziert weitreichende
Verschiebungen in den internen Machtverhältnissen der Gesellschaft.

Politische Korruption, so bemerkt Vito Tanzi scharfsinnig,
„repräsentiert eine Privatisierung des Staates, bei der seine Macht
nicht wie bei der Privatisierung üblich zum Markt verlagert wird,
sondern zu Regierungsbeamten und Bürokraten``.\footnote{Tanzi, ebenda,
  S. 167, 170.} In der Tat ist dies unter Clinton beim FBI und anderen
Polizeibehörden geschehen. Die „Rechtsstaatlichkeit`` wird immer mehr zu
dem, was Clinton und seine Kumpane wollen. Bis jetzt scheint es wenig
Beweise dafür zu geben, dass Einzelheiten zu diesen korrupten
Verbindungen für die Wähler von Bedeutung sein werden, selbst wenn sie
von den Massenmedien aufgegriffen und diskutiert würden. Im Gegenteil.
Es scheint wenig Sorge über Hinweise zu geben, dass der Präsident der
Vereinigten Staaten in Drogenschmuggel, Geldwäsche und Schlimmeres
verwickelt sein könnte.

Das erinnert an die späte Befürchtung von Walter Lippmann, dass Wählern
die Wahrnehmung fehlte, um das zu durchschauen, was er fiktive
Persönlichkeiten nannte. Er war der Ansicht, dass Wähler „mit
Schmeicheleien und Lobhudeleien schlecht bedient sind. Und sie werden
durch die unterwürfige Heuchelei verraten, die ihnen sagt, dass das, was
wahr und was falsch, was recht und was unrecht ist, durch ihre Stimme
bestimmt werden kann.`` \footnote{Walter Lippmann, \emph{The Public
  Philosophy} (New Brunswick, N.J.: Transaction Publishers, 1989), S.
  14.}

Lippmann nahm einen „Zusammenbruch der verfassungsmäßigen Ordnung``
wahr, der „die Ursache für den abrupten und katastrophalen Niedergang
der westlichen Gesellschaft sein könnte. Wir sind in kurzer Zeit weit
gefallen. ... Was wir gesehen haben, ist nicht nur Verfall - obwohl ein
großer Teil der alten Struktur sich auflöst - sondern etwas, das als
historische Katastrophe bezeichnet werden kann.`` \footnote{Ebenda, S.
  15.}

Das Problem besteht darin, dass politische Urteile weniger eine Reaktion
auf die reale Welt zu sein scheinen als auf eine von der breiten
Öffentlichkeit konstruierte Scheinrealität über Phänomene, die außerhalb
ihres direkten Wissens liegen.\footnote{Paul Roazen,
  \emph{Introduction}, in Lippmann, ebenda, S. xv.} Aber es ist ein
Fehler, sich von den Grenzen dessen leiten zu lassen, was andere sehen.
Selbst wenn es Sie kein bisschen interessiert, ob Vincent Foster
ermordet wurde und sein Mord von den höchsten Polizeibehörden und
verantwortlichen Beamten der US-Regierung, einschließlich sogar des
aktuellen Sonderermittlers Kenneth Starr, vertuscht wurde, könnten Sie
dennoch Beweise für das breitere Muster von Verbindungen zwischen
organisierter Kriminalität und dem Weißen Haus in Betracht ziehen
wollen.

Auf lange Sicht macht politische Korruption auf höchsten Ebenen den
konventionellen Lobgesang auf die Möglichkeiten der Demokratie zur
bewussten Lösung öffentlicher Probleme zunichte. Im
Informationszeitalter wird es viel weniger wichtig sein, dass eine
Regierung groß und mächtig ist, als dass sie ehrlich ist. Die meisten
der Dienstleistungen, die Regierungen historisch gesehen erbracht haben,
werden im nächsten Jahrtausend voraussichtlich auf den privaten Markt
verlagert. Doch angesichts der weltweiten Beweise ist es fraglich, ob
man sich langfristig auf ein korruptes System mit korrupten Anführern
für die Sicherheit seiner Familie und Investitionen verlassen kann.

Wie Morris sagt: „Die Clintons sind nicht nur symptomatisch, sondern sie
sind bezeichnend für das größere parteiübergreifende System in seiner
Sackgasse am Ende des Jahrhunderts``.\footnote{Morris, ebenda, S. 469.}
Vito Tanzi zeigt in seinem Aufsatz über Korruption, dass „die einzige
Möglichkeit, Korruption abzuwehren, darin besteht, den Umfang
öffentlicher Interventionen erheblich zu reduzieren``.\footnote{Fiorentini
  und Peltzman, ebenda, S. 16.} Die Informationsrevolution wird „den
Umfang öffentlicher Interventionen`` erheblich verringern und auf dieser
Grundlage besteht die Hoffnung auf eine Wiederbelebung von Moral und
Ehrlichkeit. Eine weitere offensichtliche Folge der
Informationsrevolution für Moral ist eine erhöhte Verwundbarkeit, die
mit der Möglichkeit des Cyberhandels und virtueller Unternehmen
einhergeht, die mit unknackbarer Verschlüsselung kommunizieren können.
Interne Diebe innerhalb einer Organisation, selbst einer virtuellen
Organisation, werden schwieriger aufzuspüren sein, und es wird so gut
wie unmöglich sein, gestohlenes oder heimlich erhaltenes Geld für den
Verkauf von Geschäftsgeheimnissen, Patenten oder anderen wertvollen
Wirtschaftsgütern wiederzuerlangen.

Kriminalität lohnt sich, und viele finden es attraktiv, rechtschaffene,
produktive Tätigkeiten durch gesetzlose, räuberische zu ergänzen. Im
Gegensatz zu der üblichen Situation, die in westlichen Gesellschaften
während der meisten der letzten zwei Jahrhunderte vorherrschte, sind
Kriminelle nicht bloß Außenseiter ohne sozialen Status. Wenn
Kriminalität sich auszahlt, neigt man dazu, eine bessere Klasse von
Kriminellen zu bekommen, da wenig gesellschaftlicher Makel mit
Verbrechen verbunden ist. Die sizilianische Mafia zum Beispiel und viele
Drogenhändler, die einheimische Arbeitskräfte zu überhöhten Preisen
beschäftigen, genießen in ihrem Heimatland Respekt und Unterstützung.

\section{DIE MORALISCHE ORDNUNG UND IHRE
FEINDE}\label{die-moralische-ordnung-und-ihre-feinde}

Alle starken Gesellschaften haben eine starke moralische Grundlage. Jede
Studie zur Geschichte der Wirtschaftsentwicklung zeigt die enge
Beziehung zwischen moralischen und wirtschaftlichen Faktoren. Länder und
Gruppierungen, die eine erfolgreiche Entwicklung erzielen, tun dies
teilweise, weil sie eine Ethik besitzen, die die wirtschaftlichen
Tugenden von Eigenverantwortung, harter Arbeit, familiärer und sozialer
Verantwortung, hohen Sparquoten und Ehrlichkeit fördert. Das gilt auch
für soziale Untergruppen. Der Geschäftserfolg von Juden, insbesondere
von religiösen Juden, von den Puritanern in Neuengland, von den Quäkern
im britischen Geschäft des 18. und 19. Jahrhunderts oder von den
Mormonen im modernen Amerika, zeigt die wirtschaftlichen Vorteile, die
aus Kulturen mit einem starken moralischen Rahmenwerk resultieren.

Man kann die Quäker als Beispiel nehmen. Die Quäker wurden aus mehreren
Gründen geschäftlich erfolgreich, insbesondere als Bankiers. Sie setzten
für sich selbst den höchstmöglichen Standard an Vertrauenswürdigkeit.
Sie leisteten keine Eide, sahen aber jede geschäftliche Verpflichtung
als ebenso bindend an wie einen Eid. „Mein Wort gilt`` war für sie ein
absolutes Prinzip. Sie glaubten an einen ruhigen Lebensstil, anständig,
aber sparsam. Aus religiöser Pflicht vermieden sie es, Geld für die
Eitelkeiten dieser Welt auszugeben. Sie mieden Streitigkeiten und
hielten Krieg immer für sündig. Sie glaubten, dass der Geschäftsmann die
moralische Verpflichtung hat, einen fairen Wert zu geben, und als
Händler erwarben sie den Ruf, hohe Qualität zu moderaten Preisen zu
bieten. „Caveat emptor`` - der Käufer muss aufpassen - war für sie nicht
gut genug. In einer Zeit, in der die meisten Händler die
Hohe-Preise-Hohe-Gewinne-Theorie des Handels verfolgten, führte die
Quäker-Moral sie auf natürliche Weise zu einer
Niedriger-Gewinn-Hoher-Umsatz-Politik. Wie Henry Ford später zeigte,
kann das potenziell wesentlich profitabler sein. Sie folgten dieser
Geschäftspolitik, weil sie glaubten, es sei ihre Pflicht, ihre Kunden
nicht zu betrügen, aber es stellte sich heraus, dass es der beste Weg
war, ihr Geschäft zu erweitern. Die Quäker erwiesen sich als gute
Geschäftspartner, also kamen ihre Kunden wieder; es gab Gewinne auf
beiden Seiten. Als eine Gemeinschaft, die viel sparte und ihre
Verpflichtungen ehrte, hatten die Quäker Vorteile als Bankiers, und die
Mitgliedschaft bei den Quäkern war an sich ein Geschäftsvermögen, das
Vertrauen weckte.

Leider können solche Geschäftsvorteile durch den Erfolg, den sie
hervorbringen, wieder zunichte gemacht werden. Länder durchlaufen einen
Zyklus, der die Grundlage von Adam Fergusons soziologischer Theorie im
18. Jahrhundert bildet - von Armut und harter Arbeit, zu Reichtum, zu
Luxus, zu Dekadenz und weiter zum Niedergang. Die alten Römer selbst
blickten zurück auf die Tugenden der republikanischen Ära, als das Reich
aufgebaut wurde, und beklagten den Luxus und die Faulheit, die sie als
Ursache ihres Niedergangs ansahen. Diese Erosion der fleißigen Tugenden
durch Wohlstand kann überraschend schnell passieren. Die Deutschen sind
immer noch ein fähiges und effizientes Volk, aber sie arbeiten bei
weitem nicht mehr so schwer wie in den Zeiten, als sie ihr Land nach der
verheerenden Niederlage im Jahr 1945 wieder aufbauten. Innerhalb von
zwei Generationen sind sie von langen Arbeitszeiten, fast mit bloßen
Händen, unter Bedingungen akuter Armut zu kurzen Arbeitszeiten für die
höchsten Löhne und den teuersten Wohlstand der Welt übergegangen. Im
Oktober 1995 wurde die Petersburger Erklärung von sechzehn deutschen
Arbeitgeberverbänden unterzeichnet. Sie ist ein Katalog gut begründeter
Beschwerden, die den Niedergang der Arbeitsmoral in Deutschland
widerspiegeln.

\begin{quote}
Die Steuerbelastung in Deutschland erreichte 1995 Rekordhöhen,
insbesondere aufgrund des Solidaritätszuschlags und der Beiträge zur
Pflegeversicherung. Mit einer Gesamtunternehmensbesteuerung von über 60
Prozent liegt Deutschland weit über der vergleichbaren internationalen
Ebene von 35 bis 40 Prozent. Gewohnheiten des öffentlichen Sektors wie
geregelte Beförderungen, Arbeitsplätze auf Lebenszeit und höhere
Rentenzahlungen müssen durch die Regeln des freien Marktes mit
leistungsgerechter Beförderung und Entlohnung ersetzt werden. Da
Deutschland die höchsten Arbeitskosten der Welt hat, muss die
Lohnpolitik zum Abbau der Arbeitslosigkeit beitragen, indem sie die
Kosten für die Unternehmen senkt...Lohnerhöhungen sollten an der
Wettbewerbsfähigkeit und Produktivität gemessen werden.\ldots{} Das
Verhalten der Gewerkschaften muss sich ändern. Das alljährliche Ritual
von Kampagnen, Forderungen, Arbeitnehmermobilisierung, Drohungen und
Warnstreiks ist schädlich.
\end{quote}

Diese Besorgnis, dass die Deutschen, insbesondere die jungen Leute und
die Erben des Wohlstands, die Gewohnheit zu arbeiten verloren haben,
wird auch von Bundeskanzler Kohl geteilt.

Der bestehende Arbeitsvertrag von Volkswagen sieht den höchsten Lohn für
Automobilarbeiter weltweit vor, zu dem noch Sozialabgaben hinzukommen,
im Austausch für eine 28-Stunden-Woche - vier Tage mit jeweils sieben
Stunden. Das Nachkriegsdeutschland ist nun ein großer Exporteur von
Arbeitsplätzen. Im 19. Jahrhundert galten die Briten als die
effizienteste Industrienation - einen Ruf, den sie einhundert Jahre
später sicherlich verloren hatten. Der Zyklus des Wohlstands untergräbt
zweifellos die Tugenden von harter Arbeit und bescheidenen Erwartungen,
die in den frühen Phasen der erfolgreichen industriellen Entwicklung
existieren. Nationen können ihre ursprünglichen Tugenden nicht bewahren,
so wie Individuen durch zu leichten Erfolg gierig und faul werden
können.

Globale Investitionen belohnen zweifelsohne diese fleißigen Tugenden und
bestrafen diejenigen, die gierig und faul werden, so wie es sich gehört.
Tatsächlich könnte man sagen, dass eine solide Investition sowohl auf
einer moralischen als auch auf einer rein finanziellen Beurteilung
basieren muss. Der Engländer im achtzehnten Jahrhundert, der das Kapital
einer Quäker-Bank zeichnete, hatte gute Aussichten auf Erfolg. Im
neunzehnten Jahrhundert investierten die Quäker in Schokoladengeschäfte,
da sie annahmen, dass Kakao gesünder ist als Alkohol. Das ist er
wahrscheinlich auch. Dennoch wäre eine Investition in Fry's oder
Cadbury's sicherlich eine gute Investition gewesen. Investoren sollten
darauf bedacht sein, Zeiten der Dekadenz zu vermeiden. Selbst wenn
Deutschland eine starke Position auf dem europäischen Markt und hohe
industrielle Fähigkeiten beibehält, haben hohe Arbeitskosten und kurze
Arbeitszeiten das zukünftige Potenzial Deutschlands bereits verringert.

Soziale Moral und wirtschaftlicher Erfolg sind untrennbar miteinander
verbunden. Aber welche Faktoren tragen dazu bei, die soziale Moral
aufrechtzuerhalten, und welche neigen dazu, sie zu untergraben? Arnold
Toynbee, der bedeutende Philosophiehistoriker der ersten Hälfte des
zwanzigsten Jahrhunderts, formulierte die Theorie der Herausforderung
und der Antwort. Gesellschaften werden durch Herausforderungen belebt
und entwickeln Tugenden, von denen sie nicht einmal wussten, dass sie
sie besitzen.

Es gibt immer eine menschliche Anerkennung, dass sich schwere Zeiten
entwickeln können, und diese normalerweise gesündere Reaktionen
hervorrufen als Perioden des Wohlstands. In unserem individuellen Leben
versuchen wir alle, es uns bequem zu machen, wir hoffen, in einem Haus
zu leben, das uns gefällt, einen Arbeitsplatz zu haben, der uns gefällt,
genug Geld auf der Bank zu haben und so weiter. Der Kampf, diese Ziele
zu erreichen, lohnt sich. Wir lernen in der Schule, wir bilden uns
weiter, wir arbeiten hart in unserem Unternehmen oder Beruf, um diese
Ziele zu erreichen.

Für allzu viele Menschen erweist sich die Erreichung dieser Ziele als
eine Art Falle. Der Kampf ist besser als die Errungenschaft. Der große
Schweizer Psychologe Carl Jung hatte Anfang dieses Jahrhunderts einen
amerikanischen Geschäftsmann als Patienten. Der Geschäftsmann hatte
genau diese Ambitionen als junger Mann. Er hatte daran gearbeitet, sein
eigenes Geschäft aufzubauen und genug Geld zu verdienen, um sich mit
vierzig zur Ruhe zu setzen. Er heiratete eine junge und attraktive Frau,
kaufte ein schönes Haus, hatte eine junge Familie, sein Geschäft war
sehr erfolgreich und mit vierzig Jahren konnte er tatsächlich verkaufen
und sich zur Ruhe setzen. Ein reicher und unabhängiger Mann mit
scheinbar keinen Sorgen. Anfangs genoss er seine Freiheit, konnte Dinge
tun, die er sich schon lange versprochen hatte. Er nahm seine Familie
mit nach Europa. Sie besuchten Kunstgalerien und so weiter. Allmählich
begannen diese Interessen und sein Freiheitsgefühl selbst zu verblassen.
Er begann die Zeit, in der er nicht frei hatte, in der er rund um die
Uhr für sein Geschäft arbeitete und all die üblichen Geschäftssorgen
hatte, als die glücklichste Zeit seines Lebens zu betrachten. Er verfiel
in eine Depression, aufgrund derer seine Frau ihn zu Jung als Patienten
brachte. Jung diagnostizierte ihn im Grunde genommen damit, dass er
keinen Ausweg für seine kreative Energie hatte, was auf ihn zurückfiel
und ihn zerstörte. Die Diagnose mag korrekt gewesen sein, führte aber
nicht zur Heilung. Der Geschäftsmann erholte sich nie von seinem
Nervenzusammenbruch.

Für den Menschen ist es der Kampf an sich, der zählt, und nicht so sehr
die Errungenschaft; wir sind für das Handeln geschaffen und der Erfolg
kann sich als große Enttäuschung erweisen. Die Ambitionen, was auch
immer sie sein mögen, setzen den Kampf in Gang, aber der Kampf ist
angenehmer als das eigentliche Ergebnis, selbst wenn das Ziel in vollem
Umfang erreicht ist. Und natürlich können die Ziele, für die meisten
Menschen, nur teilweise erreicht werden. Die meisten von uns haben nicht
so viel Geld, wie wir es gern hätten, und wir leben nicht in unserem
Traumhaus. Wir müssen uns mit weniger zufriedengeben.

Dieses Gefühl, dass Tugend dynamisch ist, dass sie sich eher in der
Anstrengung als im Ergebnis zeigt, entwickelte sich im 19. Jahrhundert
stark und auf verschiedene Weisen. Es gibt ein bekanntes Gedicht von
Arthur Hugh Clough, das vielen Menschen in dem Kampf um Leben und Tod
des Zweiten Weltkriegs Trost brachte. Es ist bemerkenswert, dass die
Suizidraten in den kriegführenden Ländern im Zweiten Weltkrieg gefallen
sind; selbst der Kampf des Krieges kann besser sein als die Depression
der Inaktivität.

\begin{quote}
\emph{Sage nicht, der Kampf sei vergebens,\\
die Arbeit und die Wunden seien umsonst,\\
der Feind ermattet und versagt nicht,\\
und wie die Dinge waren, so bleiben sie.}

\emph{Wenn Hoffnungen getäuscht wurden, könnten Ängste Lügner sein;\\
Es könnte sein, dass in jenem verborgenen Rauch,\\
deine Kameraden sogar jetzt die Fliehenden jagen,\\
und, wäre es nicht für dich, das Feld besitzen.}

\emph{Denn während die müden Wellen, vergeblich brechend,\\
scheinbar hier keinen schmerzhaften Zentimeter gewinnen,\\
kommt weit zurück, durch Bäche und Einlassstellen,\\
schweigend, das Hauptwasser hereinflutend.}

\emph{Und nicht nur durch die Fenster im Osten,\\
wenn der Tag anbricht, kommt das Licht,\\
vorne steigt die Sonne langsam, oh so langsam,\\
aber schau nach Westen, dort strahlt das Land hell.}
\end{quote}

Dieser aktive Wettbewerb spricht immer noch das moderne Empfinden an.
Tatsächlich ist es die Art und Weise, wie viele moderne Männer und
Frauen ihr Leben führen, in einem ständigen Kampf, um die Möglichkeiten
einer potenziell feindlichen Umgebung zu ergreifen. Wir alle leben in
einer Wettbewerbswelt und die meisten von uns möchten sich nicht davon
distanzieren. Natürlich gibt es auch das kontemplative spirituelle
Temperament, aber es ist ziemlich selten.

Eine ähnliche Wahrnehmung dieser dynamischen Moral des neunzehnten
Jahrhunderts wurde von William James, dem größten amerikanischen
Philosophen, in einer Rede vor dem Philosophischen Club von Yale im Jahr
1891 entwickelt:

\begin{quote}
Der tiefste Unterschied, praktisch gesehen, im moralischen Leben des
Menschen ist der Unterschied zwischen der sorglosen und der
anstrengenden Stimmung. Wenn wir uns in einer sorglosen Stimmung
befinden, ist das Vermeiden gegenwärtiger Übel unser vorherrschendes
Anliegen. Die anstrengende Stimmung hingegen macht uns gegenüber
gegenwärtigen Übeln völlig gleichgültig, solange nur das größere Ideal
erreicht wird. Die Fähigkeit zur anstrengenden Stimmung liegt
wahrscheinlich in jedem Menschen schlummernd, doch bei manchem fällt es
schwerer als bei anderen, sie zu wecken. Sie benötigt die wilderen
Leidenschaften, die großen Ängste, Liebe und Empörungen; oder den
tiefgreifenden Appell einer der höheren Treueschwüre, wie Gerechtigkeit,
Wahrheit und Freiheit. Ein starker Kontrast ist ein Erfordernis ihres
Sichtfelds; eine Welt, in der alle Berge eingeebnet und alle Täler
erhöht werden, ist kein für sie angemessener Wohnort. Deswegen könnte
diese Stimmung bei einem einsamen Denker für immer schlummern, ohne
jemals zu erwachen. Seine verschiedenen Ideale, von denen er weiß, dass
sie nur seine eigenen Vorlieben sind, liegen fast vom selben Wert: Er
kann mit ihnen nach Belieben umgehen. Aus diesem Grund ist in einer rein
menschlichen Welt ohne Gott der Appell an unsere moralische Energie
nicht vollkommen stimulierend. Das Leben ist sicherlich auch in einer
solchen Welt eine echte ethische Symphonie; aber sie wird nur innerhalb
von ein paar schwachen Oktaven gespielt, und die unendliche Skala der
Werte bleibt verschlossen.
\end{quote}

William James vertrat die Ansicht, dass die dynamische Moral, die darin
besteht, eher zu tun als zu sein, eher zu handeln als zu unterlassen,
auch auf den religiösen Bereich ausgedehnt werden kann. Eine starke
Entwicklung der Wettbewerbs- und Überlebensmoral findet sich auch in den
Werken von Adam Smith (1776), Thomas Malthus (1798) und Charles Darwin
(1859). Da dies die dominierende Morallehre der heutigen
Weltwirtschaftsordnung ist, bedarf deren zentrales Thema einer
sorgfältigen Betrachtung.

Die dominierende Idee des Darwinismus ist, dass Spezies durch Anpassung
an ihre Umgebung überleben und dieser Prozess der natürlichen Selektion
die Merkmale der Arten formt. Bei Tieren ist dieser Prozess das Ergebnis
von zufälligen Mutationen, die, wie wir heute wissen, zu einem
genetischen Prozess gehören, über den Darwin selbst nur spekulieren
konnte. Das Überleben menschlicher Gesellschaften hängt jedoch von
kulturellen Entscheidungen ab, die auf menschlicher Intelligenz
basieren. Kultur verändert die menschliche Gesellschaft, so wie Gene
andere Arten verändern. Veränderungen können daher viel schneller in
unseren Gesellschaften stattfinden. Es muss nicht über viele
Generationen hinweg arbeiten, wie es bei zufälligen genetischen
Mutationen der Fall ist. Anstelle der natürlichen Selektion bei Tieren
haben Menschen die kulturelle Selektion entwickelt, wobei einige
Kulturen zu bestimmten Zeiten der Menschheitsgeschichte neue
Technologien entwickelten, die ihnen einen entscheidenden Vorteil in der
Schaffung von Reichtum oder Macht verliehen. Der kulturelle Vorsprung
der neuen Technologien, wie ihn der eisenzeitliche Mensch gegenüber dem
bronzezeitlichen oder der elektronische Mensch gegenüber dem
mechanischen Menschen hatte, ist entscheidend. Adam Smith mag nicht der
erste wirtschaftswissenschaftliche Schriftsteller gewesen sein, der das
Wohlergehen von Nationen auf das Handeln von Einzelpersonen reduziert
hat, aber er hat es am prägnantesten und mit der größten Autorität
ausgedrückt:

\begin{quote}
Jeder Einzelne ist ständig bemüht, die vorteilhafteste Verwendung für
das Kapital zu finden, über das er verfügen kann. Es ist tatsächlich
sein eigener Vorteil, nicht der der Gesellschaft, den er im Blick hat.
Aber die Untersuchung seines eigenen Vorteils führt ihn natürlich, oder
eher zwangsläufig, dazu, jene Tätigkeit vorzuziehen, die für die
Gesellschaft am vorteilhaftesten ist.
\end{quote}

Thomas Malthus, der Begründer der Bevölkerungsstudien, erkannte, dass
das Argument von Adam Smith nicht nur auf die Entwicklung der Wirtschaft
von Nationen angewendet werden konnte, sondern auch auf das Überleben
der menschlichen Bevölkerungen. Er ist bekannt für seine These, dass
„die Bevölkerung, wenn sie unkontrolliert bleibt, in einem geometrischen
Verhältnis zunimmt. Die Lebensgrundlagen nehmen jedoch nur in einem
linearen Verhältnis zu. Wer sich ein wenig mit Zahlen auskennt, wird die
Unermesslichkeit der ersten Macht im Vergleich zur zweiten erkennen.``

Malthus hatte sogar lange vor Darwin erkannt, dass das gleiche Prinzip
in der gesamten Natur Anwendung findet:

\begin{quote}
Durch die Tier- und Pflanzenreiche hat die Natur das Leben mit
großzügiger Hand weit und breit verstreut. Sie war dagegen
vergleichsweise sparsam mit dem Raum und der Nahrung, die notwendig
sind, um sie zu pflegen. Die Lebenskeime, die in diesem Fleckchen Erde
enthalten sind, könnten mit ausreichend Nahrung und genügend Platz zum
Ausbreiten in ein paar tausend Jahren Millionen von Welten füllen. Die
Notwendigkeit, dieses zwingende und allgegenwärtige Gesetz der Natur,
hält sie innerhalb der vorgeschriebenen Grenzen.
\end{quote}

Die Art und Weise, wie sich die Welt entwickelt, wurde schon gegen Ende
des achtzehnten Jahrhunderts, zur Zeit von Adam Smith und Malthus, als
dynamisch verstanden - was sie tatsächlich immer schon war. Die
Menschheit, als eine Spezies unter vielen, ist gezwungen, um Ressourcen
zu konkurrieren. Diese Konkurrenz wird durch das Missverhältnis zwischen
ihrer unbegrenzten Fortpflanzungsfähigkeit und ihrer begrenzten
Kapazität, Nahrung zu produzieren, verursacht. Das Überleben
menschlicher Gesellschaften, wie auch das von Tierarten, hängt von einer
erfolgreichen Anpassung an die Umwelt ab. Eine dynamische Moral
beschäftigt sich daher mit der Überwindung der Anpassungsprobleme. Dies
gelingt am besten Individuen, die ihre Handlungen den Möglichkeiten der
Umwelt anpassen und daher die in der Gesellschaft verfügbaren Ressourcen
bestmöglich nutzen.

Malthus erkannte bereits, dass die Ideen von Adam Smith die Welt
verändert hatten, und er schrieb, dass sein neuer Ansatz zum Thema
Bevölkerung nicht neu war: „Die Prinzipien, auf denen sie basieren,
wurden zum Teil von Hume und zum Teil von Dr.~Adam Smith erklärt.`` Er
sah auch, dass dieser ständige Wettkampf ums Überleben eine moralische,
nicht nur eine praktische Angelegenheit war. Die letzte Passage des
„Essays`` von 1798 lautet:

\begin{quote}
Das Böse existiert in der Welt, nicht um Verzweiflung zu erzeugen,
sondern um Tätigkeit zu fördern. Wir sollen es nicht geduldig hinnehmen,
sondern uns bemühen, es zu vermeiden. Es ist nicht nur das Interesse,
sondern die Pflicht eines jeden Einzelnen, sich nach Kräften zu bemühen,
das Böse von sich selbst und von einem möglichst großen Kreis zu
entfernen. Je mehr er sich in dieser Pflicht übt, je weiser er seine
Bemühungen lenkt und je erfolgreicher diese Bemühungen sind, desto mehr
wird er wahrscheinlich seinen eigenen Geist verbessern und erhöhen, und
desto vollständiger scheint er den Willen seines Schöpfers zu erfüllen.
\end{quote}

Vielleicht kann man Darwins Sinn für die Wichtigkeit dieser Argumente
mit seiner Zusammenfassung der Inhalte von Kapitel 3 aus seinem
bahnbrechenden Buch „Über die Entstehung der Arten``, das erstmals 1859
veröffentlicht wurde, veranschaulichen. Er nannte dieses entscheidende
Kapitel \emph{Der Kampf ums Dasein}. Die Überschriften lauten: „Betrifft
die natürliche Auswahl - Der Begriff im weitesten Sinne verwendet -
Geometrische Vermehrungsrate - Schnelle Vermehrung von eingebürgerten
Tieren und Pflanzen - Art der Vermehrungskontrolle - Wettbewerb ist
universell - Auswirkungen des Klimas - Schutz vor der Anzahl der
Individuen - Komplexe Beziehungen zwischen allen Tieren und Pflanzen in
der gesamten Natur - Der Überlebenskampf ist am härtesten zwischen
Individuen und Varianten der gleichen Art; oft auch zwischen Arten der
gleichen Gattung - Die Beziehung von Organismus zu Organismus ist die
wichtigste aller Beziehungen.``

Seit 1776 ist es offensichtlich, dass der beste Weg zur Optimierung des
Wohlstands der Nationen darin besteht, Einzelpersonen zu erlauben, ihre
eigene Kapitalrendite unter Bedingungen des freien Wettbewerbs zu
optimieren. Seit 1798 ist es offensichtlich, dass das relative Überleben
von Populationen von ausreichendem wirtschaftlichen und politischen
Erfolg der Gesellschaften abhängt, um sich selbst ernähren, sich vor
ansteckenden Krankheiten schützen und die Bevölkerungen in Kriegen
schützen zu können. Seit 1859 ist es offensichtlich, dass das gesamte
Drama des Lebens, sei es im menschlichen, tierischen oder pflanzlichen
Reich, aus einem fortwährenden Kampf ums Überleben besteht, bei dem die
Arten oder Kulturen, die einander am nächsten sind, die größten Rivalen
sein können. Dieser Kampf erfordert eine dynamische Moral, die aktiv das
Böse abwehrt und nicht nur darauf reagiert, wenn es geschieht.

Diese Ideen waren so machtvoll, dass es unmöglich war, über die Natur
der Menschheit oder die Probleme der Moral seit der Zeit, in der sie
entwickelt wurden, nachzudenken, ohne auf sie zu reagieren. Karl Marx
glaubte genauso an den Kampf ums Überleben wie Charles Darwin, aber er
sah es als einen Krieg zwischen den Sozialklassen an, die selbst durch
ökonomische Kräfte geformt wurden. Adolf Hitler glaubte an den Kampf ums
Überleben und sah seine eigene politische Karriere fast ausschließlich
in diesen Begriffen. Aber er glaubte, dass der Kampf einer zwischen
verschiedenen Rassen war. Marx, Lenin, Stalin, Mao und Hitler können
alle als Sozialdarwinisten bezeichnet werden, insofern sie den
Überlebenskampf, „Mein Kampf`` wie Hitler es nannte, als die zentrale
politische Frage sahen. Die Marxisten sahen soziale Klassen, als wären
sie separate Spezies; die Nazis sahen Rassen im gleichen Licht.

Dies stellt jedoch keine dynamische Moral dar, wie Malthus sie sich
vorstellte, sondern eine dynamische Unmoral. Sowohl der Marxismus als
auch der Nationalsozialismus wollten das gleiche Problem lösen, das
Problem des Kampfes ums Überleben, jedoch durch die Zerstörung des
Wettbewerbs. Sie griffen fremde Gebiete an und stachelten Konflikte
zwischen verschiedenen Klassen, die um soziale Macht konkurrierten, oder
verschiedenen Rassen an, die entweder als wirtschaftliche Ausbeuter (der
normalerweise gegen Juden erhobene Vorwurf von Antisemiten) oder als
gefährliche Unterschicht (die Angst der Weißen vor den Schwarzen)
angesehen wurden. Der Zweite Weltkrieg war ein gescheiterter Versuch
Adolf Hitlers, dem deutschen Volk einen Überlebensvorteil zu
verschaffen, indem er potenzielle Konkurrenten, insbesondere Slawen und
Juden, vernichtete. Interessanterweise erwies sich die Kriegsniederlage
für Deutschland als vorteilhafter, als es der Sieg der Nazis jemals
hätte sein können.

Die Alternative zu destruktiver „Interferenz-Konkurrenz`` ist
kollaborative Konkurrenz, und kollaborative Konkurrenz ist die zentrale
Idee von Adam Smith, aber auch von Malthus und William James. Das
Urmodell des destruktiven Wettbewerbs ist der Eroberer. Er zerstört
seine Konkurrenten, um ihre Vermögenswerte zu ergreifen, was auch die
Übernahme ihrer Länder einschließen und die Versklavung ihrer Völker
beinhalten kann. Das Modell des kollaborativen Wettbewerbs ist der
Kaufmann. Es liegt im Interesse des Kaufmanns, dass der Kunde mit der
Transaktion zufrieden sein sollte, denn nur ein zufriedener Kunde kommt
für mehr Handel zurück. Es liegt auch im Interesse des Kaufmanns, dass
der Kunde wohlhabend ist, denn ein wohlhabender Kunde hat das Geld, um
weiter einzukaufen. Eroberung impliziert die Zerstörung der anderen
Partei; Handel impliziert die Zufriedenheit der anderen Partei. Da die
moderne Technologie die Eroberung zu einer außerordentlich gefährlichen
Politik gemacht hat, ist der Handel zum einzigen rationalen Ansatz für
die Lösung der Überlebensprobleme geworden.

Diese Interdependenz wird durch eine weitere zentrale Idee von Adam
Smith verstärkt, die nicht neu für ihn war, nämlich die Spezialisierung
der Funktionen. \emph{Der Wohlstand der Nationen} beginnt mit einem
berühmten Abschnitt, in dem Adam Smith bemerkt, dass „die größte
Verbesserung der Produktivkräfte der Arbeit und der größte Teil der
Gewandtheit, der Geschicklichkeit und des Urteilsvermögens, mit denen
sie überall gerichtet oder angewandt wird, die Auswirkungen der
Arbeitsteilung gewesen zu sein scheinen.`` Er weist darauf hin, dass
„das wichtige Geschäft der Herstellung eines Stecknadelkopfes auf diese
Weise in etwa achtzehn verschiedene Vorgänge unterteilt ist, die in
einigen Fabriken alle von verschiedenen Personen ausgeführt werden.`` Je
vollständiger die Spezialisierung der Funktionen, desto wahrscheinlicher
ist die Effizienz der Herstellung, aber offensichtlich ist eine solche
Wirtschaft stark voneinander abhängig. Wenn sie erfolgreich sein soll,
muss sie kollaborativ sein.

Eine erfolgreiche soziale Moral muss daher bestimmte Merkmale aufweisen.
Sie muss stark sein - eine schwache Moral ist verletzlich und
ineffektiv. Sie muss zum Überlebenskampf beitragen, aber auf eine
kollaborative Art und Weise, nicht mörderisch. Hitler hatte eine starke
Morallehre des Überlebens, doch ihre destruktive Qualität hätte seine
eigene Gesellschaft beinahe zerstört. Sie muss dynamisch sein, um den
dynamischen Veränderungen der modernen Technologie und tatsächlich aller
modernen sozialen Systeme gerecht zu werden. Sie muss wirtschaftlich
effizient sein. Die Mischung aus egalitären und autoritären Ideen im
leninistischen System hat einfach nicht funktioniert. Doch dies sind
nicht alle Merkmale, die man von einer solchen sozialen Moralität
erwarten könnte. Sie hat den breiteren Zweck, die Gesellschaft zu einem
guten Ort zum Leben zu machen und die Menschen zusammenzubinden. Auch
Moralvorstellungen müssen sich anpassen und überleben; eine brüchige
Moral wird vielleicht von unserer Generation akzeptiert, nur um in der
nächsten abgelehnt zu werden. Eine traditionelle soziale Moral mag zu
starr sein, um sich auf die aufeinanderfolgenden Veränderungen in der
Sozialstruktur einzustellen. Andererseits ist ein rein relativistisches
System überhaupt keine Moral; es gibt keine klaren Anweisungen, wie man
sich verhalten soll.

Wir können zunächst einmal die gesamte soziale Moral in einen Kontext
stellen. Eine starke Gemeinschaft, selbst eine virtuelle Gemeinschaft,
hängt davon ab, dass die Moral weitgehend akzeptiert wird. Die
erfolgreichsten Epochen in der Geschichte der Gesellschaften sind in der
Regel solche, in denen die kollektive Moral vollständig geteilt wird.
Eine solche Moral erfüllt nicht nur spezifische Funktionen wie die
Reduzierung von Verbrechen und die Unterstützung von Familien- und
Sozialstrukturen, sondern gibt den Bürgern auch ein Gefühl von Ziel und
Richtung. Historisch gesehen scheint ein Konsens über Moral davon
abhängig zu sein, dass es eine dominierende Religion gibt, ob es nun die
Staatsreligion der frühen Überlebensgemeinschaften für ein zerstreutes
Volk; die islamische Religion mit ihren sozialen Regeln; der
Katholizismus des Mittelalters; oder der Protestantismus des frühen
Neuenglands ist. Die drei Ideen eines Volkes, einer Moral und einer
Religion hängen voneinander ab, und jede tendiert dazu, die anderen zu
verstärken.

In einer solchen moralischen Gesellschaft ist es dem einzelnen Bürger
möglich, persönliche Ziele innerhalb eines Rahmens sozialer
Unterstützung zu verwirklichen. Zugegeben, die moralischen Gesetze
können etwas willkürlich sein, oder zumindest für Außenstehende so
erscheinen. Der orthodoxe Jude verliert die Freiheit, Schweinefleisch
oder Schalentiere zu essen oder am Sabbat zu arbeiten. Der treue
Katholik könnte die Freiheit verlieren, künstliche Verhütungsmittel zu
verwenden, geschweige denn, eine Abtreibung vorzunehmen. Der Moslem
könnte die Freiheit verlieren, Alkohol zu trinken. Der fromme
Konfuzianer könnte eine unangenehm lange Trauerzeit für seinen verehrten
Vater haben - selbst Konfuzius warnte, dass Trauerrituale übertrieben
werden könnten. Doch die Anhänger jedes dieser Glaubenssysteme
betrachten diese Beobachtungen als geringen Preis für ein gemeinsames
und kohärentes Verständnis der Weltordnung, in der der Einzelne einen
festen Platz hat. Ein orthodoxer Jude könnte gut argumentieren, dass die
Einhaltung des Sabbats ein kleiner Preis für die Vorteile des Gesetzes
oder die Stärke der jüdischen Familie ist. Eine gemeinsame Moral in
einer toleranten Gesellschaft war das Ideal von John Locke und den
frühen Philosophen der Freiheit. Sie glaubten keineswegs, dass eine
Gesellschaft jeglicher Art ohne Regeln existieren kann, aber sie
meinten, dass die Regeln der besten Vernunft unterliegen sollten und
dass Menschen nur zur Annahme der wichtigsten Regeln gezwungen werden
sollten. Sie erkannten, dass Zwang in der sozialen Moral unvermeidlich
ist, insbesondere beim Schutz von Leben oder von Eigentum, weil sie der
Ansicht waren, dass keine Gesellschaft überleben kann, wenn es keine
Sicherheit gibt. Sie zeigten eine fast absolute Toleranz gegenüber
Variationen bei persönlichen Entscheidungen, die das Wohl anderer nicht
beeinträchtigen. Der Konfuzianer, der seinen Vater vierzig Tage lang
betrauert, könnte neben dem Juden wohnen, der den Sabbat ehrt, ohne den
anderen zu stören oder ihn zwingen zu wollen, seine eigenen religiösen
Praktiken zu befolgen.

Aus dieser kombinierten Doktrin der sozialen Moral in wesentlichen
Angelegenheiten und der Toleranz in persönlichen Entscheidungen entsteht
tatsächlich ein moralischer Grundsatz, der allen Bürgern auferlegt
werden muss, und eine freiwillige Ethik, die die Bürger als
Einzelpersonen oder als Mitglieder von Untergruppen in der Gesellschaft
akzeptieren. Wenn ein Benediktinermönch Gelübde der Armut, der
Keuschheit und des Gehorsams ablegt, tut er dies als Mitglied einer
solchen Untergruppe. Er fordert nicht alle Katholiken, geschweige denn
alle seine Mitbürger, auf, die gleichen Gelübde abzulegen oder dieselben
Regeln zu befolgen. Er wird den Befehlen seines Abtes gehorchen, aber er
erwartet nicht, dass jemand außerhalb seines Klosters ihnen Beachtung
schenkt. Die Einhaltung dieser optionalen Teile der sozialen Moral muss
nicht universell sein, aber die Kernmoral muss geteilt werden, und
Menschen, die die Kernmoral nicht akzeptieren, schaden sowohl der
Gesellschaft als auch sich selbst. Im extremen Beispiel bietet eine
Gesellschaft, die von Räubern überrannt wird, die nicht zögern zu
morden, so wie es in großen Teilen Europas nach dem Fall des Römischen
Reiches der Fall war, niemandem ein zufriedenstellendes Leben, nicht
einmal den Räubern selbst; sie sind immer besonders bedroht durch andere
Mörder. Dies gilt gleichermaßen für einige innenstädtische Gebiete der
Vereinigten Staaten heute. Anarchie ist nicht die ideale Gesellschaft,
denn ohne die Durchsetzung des Rechts gibt es keine menschliche
Sicherheit.

Wenn man die Kräfte betrachtet, die der Moralethik der Gesellschaft
feindselig gegenüberstehen, muss man diese Kernmoral in Erwägung ziehen,
die in den meisten modernen religiösen Glaubenssystemen recht ähnlich
ist. Zumindest zwei der Zehn Gebote des Alten Testaments für Christen,
oder der Thora für Juden, können als universell für alles angesehen
werden, was man als eine Religion anerkennen könnte: „Du sollst nicht
morden`` und „Du sollst nicht stehlen``. Man kann sogar noch weiter
gehen. Fast alle ernsthaften Agnostiker würden sowohl Mord als auch
Diebstahl -- die ultimative Bedrohung des Lebens und die ultimative
Bedrohung des Eigentums - als verboten ansehen und würden akzeptieren,
dass die Gesellschaft das Recht hat, Menschen zu bestrafen, die morden
oder rauben. Sie könnten sich über die angemessene Bestrafung für ein
bestimmtes Verbrechen uneinig sein, aber nicht über das Recht der
Gesellschaft, als solche zu bestrafen.

Der ursprüngliche Ausdruck von John Locke trifft es genau. Jeder hat das
Recht auf „Leben, Freiheit und Eigentum``. 1776 fügte Thomas Jefferson
einen weiteren Ausdruck von John Locke hinzu: „das Streben nach Glück``.
Das ergibt einen sehr schönen Ausdruck, und ein sehr nobles Bestreben,
aber „Leben, Freiheit und Eigentum`` sind konkreter als „Leben, Freiheit
und das Streben nach Glück``. Die Gesellschaft hängt absolut vom Recht
auf Leben und dem Recht auf Eigentum ab. Die Praxis der Geschichte
zeigt, dass diese Rechte nur dann geschützt werden können, wenn Freiheit
herrscht. Ist der Staat allmächtig, wird er zum großen Feind des Lebens,
wie in Angriffskriegen, und des individuellen Eigentums, indem er einen
übermäßigen Anteil des nationalen Vermögens für seine eigenen, oft
unerwünschten und stets verschwenderischen Zwecke nimmt.

Die Kernmoral ist jedoch in den am weitesten fortgeschrittenen Nationen
unter Beschuss, teils durch die Kräfte der Modernität, die diesen
Nationen ihren technischen Vorsprung geben. Die Vereinigten Staaten sind
die führende technologische Macht der Welt. Viele Menschen,
einschließlich der meisten Amerikaner, hätten die Vereinigten Staaten zu
jeder Zeit bis in die frühen 1960er Jahre als moralisches Vorbild für
den Rest der Welt betrachtet. Heute wird diese Ansicht selten geäußert,
selbst von Amerikanern, die stolz auf ihr Land sind. Man konnte nicht,
wie es die Welt tat, dem Prozess gegen O. J. Simpson zuhören und die
Vereinigten Staaten als die einfache tugendhafte Republik betrachten,
die sie anfangs waren.

Wenn man auf die Kennzeichen des alten Amerika zurückblickt, spiegeln
sie die Bedürfnisse einer Frontier-Gesellschaft wider, die die
Einstellungen ihrer Bürger sogar in den Großstädten prägte. Grenzgebiete
sind demokratische Orte. Die Menschen fühlen sich als gleichberechtigt,
und die frühen Amerikaner warfen die Klassenhierarchien Europas ab.
Selbst Leibeigene, die als Gefangene aus England geschickt wurden,
etablierten sich als unabhängige Handwerker, Bauern oder freie Arbeiter,
sobald ihre Leibeigenschaft vorbei war. Die Löhne waren höher als in
Europa, und die Kosten für lebensnotwendige Güter waren gering, obwohl
importierte Waren teuer waren. Im Frontier selbst waren die Menschen
sehr voneinander abhängig, aber das Leben, wenn auch hart, war gemessen
an europäischen Standards gut. Einwanderer mussten vielleicht als
Niedriglohnempfänger in den Slums von Boston und New York beginnen,
entkamen diesen jedoch meist ziemlich schnell, und Generation für
Generation fand Wohlstand. Nach dem Bürgerkrieg sahen sich die Schwarzen
als eine weitere Einwanderergruppe und viele von ihnen teilten die
amerikanischen Werte und Ziele. Aus diesen entwickelte sich die schwarze
Mittelschicht.

Dieser Anspruch, gestärkt durch die tatsächliche Erfahrung der
Grenzgebiete und durch den Einfluss der Kirchen, sowohl protestantischer
als auch katholischer, prägte den Patriotismus der Amerikaner. Sie
glaubten, in Gottes eigenem Land zu leben, einer Vorstellung, die
einzigartig von demokratischen Idealen und christlichem Glauben geleitet
war und die erste und erfolgreichste der weltweiten Demokratien
darstellte. Das Bild ist uns sehr vertraut; es ist in dem Bild
verkörpert, das wir alle, oder fast alle, von Abraham Lincoln haben,
obwohl man noch immer einige Amerikaner im Süden finden kann, die
Lincoln als den Mann sehen, der die Schrecken des ersten modernen
Krieges entfesselte, um freie Staaten daran zu hindern, eine Union zu
verlassen, der sie nicht länger vertrauten.

Nichtsdestotrotz bleibt das Bild von Lincoln, zerklüftet, einfach,
ehrlich und eloquent, immer noch das oberste amerikanische Bild, und es
ist im Wesentlichen ein moralisches. Viele Amerikaner spüren immer noch
den ursprünglichen, lebhaften Kontrast zwischen der demokratischen
Energie des neuen Landes und den müden Hierarchien Europas. Dieses Ideal
einer im Wesentlichen dynamischen Leistungsgesellschaft ist für den
Ausländer schwer zu erkennen, wenn er das heutige Los Angeles, New York,
Houston oder Washington betrachtet, obwohl seine Spuren, und mehr als
nur Spuren, immer noch in den großen Vorstadtgebieten oder in ländlichen
Gegenden zu finden sind. Die amerikanische puritanische Ethik, mit all
ihrer historischen Bedeutung, überlebt am besten nördlich der
Schneegrenze, aber die unternehmerische Dynamik ist weiter verbreitet.

Amerikaner würden auf den Verfall der Großstädte hinweisen, die zu
Brutstätten für Kriminalität, insbesondere das Drogengeschäft, geworden
sind, als das schlimmste Symptom für den Rückgang eines
gemeinschaftlichen Moralgefühls. Die meisten Amerikaner erkennen auch,
dass es einen Zusammenstoß mehrerer unterschiedlicher moralischer
Kulturen gibt, die alle in ihren Ansprüchen und ihrer Autorität
konkurrieren. Die „politisch korrekte`` Kultur lehnt viele, aber nicht
alle moralischen Grundsätze ab, auf denen die alte Kultur beruhte. Sie
betont aggressiv die Rolle und die Rechte von Gruppen, die als
historisch von einer dominanten weißen männlichen Kultur ausgebeutet
angesehen werden und lehnt diese Kultur ab, obwohl sie die
Gründungskultur der Vereinigten Staaten ist.

Die dominante männliche Kultur der ersten Hälfte des 20. Jahrhunderts
konzentrierte sich auf das Überleben der Kernfamilie. Historisch gesehen
verschaffte dies dem Ehemann und Vater zumindest nominell Vorherrschaft
im Haus, obwohl das Haus in der Praxis oft von der Ehefrau und Mutter
geführt wurde, mit der oft sanften Akzeptanz des nominellen Herrn. Es
gab dem männlichen Chef eine echte Dominanz am Arbeitsplatz, eine
Dominanz, die die feministische Bewegung bislang herausgefordert, aber
noch nicht umgekehrt hat. Das Interesse der Familie und die historische
christliche Lehre verboten Abtreibung. Die alte Moral hielt Abtreibung
für unrechtmäßiges Töten, sie war nie erlaubt, und die Anhänger der
traditionellen Moral denken immer noch so. Anhänger der neuen Moral
denken das Gegenteil. Im Falle Roe v. Wade stützte der Oberste
Gerichtshof das verfassungsmäßige Recht auf Abtreibung, das bis dahin
als Frage der Einzelstaaten betrachtet wurde, auf die Doktrin eines
Rechts auf Privatsphäre, das weit entfernt von jeglicher Sprache ist,
die tatsächlich in der Verfassung oder ihren Änderungen zu finden ist.

Das Recht auf Privatsphäre einer Frau umfasste auch das Recht, Kinder zu
haben oder nicht zu haben, unabhängig von den Konsequenzen für den
Embryo. Der Oberste Gerichtshof betrachtete den Embryo nicht als Träger
verfassungsmäßiger Rechte - Embryos waren in der zweiten Hälfte des 20.
Jahrhunderts dieselben außerhalb der Verfassung stehenden Entitäten, wie
Sklaven es in der ersten Hälfte des 19. Jahrhunderts gewesen waren.
„Leben, Freiheit und das Streben nach Glück`` galt nicht für Sklaven,
und die Formulierungen der Unabhängigkeitserklärung wurden von den
Richtern in Roe v. Wade nicht auf Embryonen angewendet.

Die Abtreibungsdebatte ist das extremste Beispiel für den Konflikt
zwischen der alten und der neuen Moral, obwohl es ebenso bemerkenswerte
Konflikte in anderen Bereichen gibt, in denen die alte gesellschaftliche
Struktur mit ihrer Moral von der neuen herausgefordert worden ist. Die
traditionelle christliche Moral, gleichermaßen in der protestantischen
und katholischen Kirche, legte großen Wert auf sexuelle Rollen: Kein
heterosexueller Geschlechtsverkehr außerhalb oder vor der Ehe. Keine
homosexuellen Beziehungen. Lesbianismus wurde weniger betont, da die
Gesellschaft dessen Existenz kaum anerkannte. Als Königin Victoria
erstmals davon erzählt wurde, weigerte sie sich vehement zu glauben,
dass solche Dinge zwischen Frauen geschehen könnten. Politische
Korrektheit ist die Moral von vermeintlich unterdrückten Gruppen. Die
Homosexuellen reklamierten eine gleichwertige Gültigkeit für ihren
Lebensstil und stellten die traditionelle Opposition zu ihrem sexuellen
Verhalten in Frage. „Homophobie`` wurde als eine empörende Form von
Vorurteilen angesehen, ähnlich wie rassistische Diskriminierung. Es wird
von der neuen Moral als ebenso inakzeptabel angesehen, Homosexuelle zu
kritisieren, wie Schwarze, Juden oder Frauen.

Gleichzeitig wurden andere sexuelle Tabus aufgehoben oder abgeschafft.
In den 1960er Jahren gab es eine neue Welle der freien Liebe, teilweise
basierend auf der scheinbaren Sicherheit der Anti-Baby-Pille, aber auch
gefördert durch stimmungsverändernde Drogen und Popmusik. Dies führte zu
einer zunehmenden Anzahl nichtehelichen Zusammenlebens. In den 1990er
Jahren galt es in Großbritannien, einer eher altmodischen Gesellschaft
im Vergleich zu den meisten Teilen der Vereinigten Staaten, als absolut
normal, dass Prinz Edward mit seiner Freundin im Buckingham Palast
schlief - in der gleichen stabilen, aber unverheirateten Intimität, in
der Studenten seit den 60er Jahren miteinander in ihren Unterkünften
schliefen. Nur wenige fanden es bemerkenswert, dass Königin Elizabeth
II., das Oberhaupt der Church of England, das Verhalten ihres jüngsten
Sohnes billigte, nachdem die Ehen ihrer drei älteren Kinder bereits
gescheitert waren. Diejenigen, die sich beschwerten, wurden als
hoffnungslos veraltet und prüde angesehen. Und doch gab es immer noch
viele Menschen, die die alte Moral vorzogen, selbst wenn sie sie nicht
selbst praktizierten oder ernsthaft erwarteten, dass ihre Kinder dies
über ein bestimmtes Alter hinaus tun würden.

Die Bewegung der politischen Korrektheit hatte ebenfalls eine eigene
puritanische Seite. Da sie aus den wahrgenommenen Interessen der Frauen
hervorging, die als die am stärksten unterdrückte Gruppe angesehen
wurden, gab es eine gewisse Feindseligkeit gegenüber männlicher
Sexualität, sowohl in aggressiven als auch in zuvor als harmlos
angesehenen Formen. Einige Frauen waren der Ansicht, dass alle Männer
von Natur aus Vergewaltiger seien, und der natürliche Schrecken vor der
Vergewaltigung wurde in eine allgemeine Verleumdung des männlichen
Geschlechts übertrieben. Andere konzentrierten sich auf sexuelle
Belästigung, ein tatsächliches Übel - viele Männer haben sehr grobe
sexuelle Umgangsformen - welches in einigen trivialen Fällen lächerlich
wurde. Sexuelle Belästigung wurde sogar bei bloßen Blicken ohne jedes
gesprochene Wort oder physischen Kontakt angeklagt. Daher konnte die
neue Moral sehr zensurbehaftet sein. Weiße Menschen konnten aufgrund
ihrer Hautfarbe des Rassismus beschuldigt werden, nicht weil sie
rassistisch waren, sondern weil sie weiß waren. Männer konnten der
sexuellen Belästigung beschuldigt werden, weil ihre Mimik zeigte, dass
sie eine Frau attraktiv fanden, was in einer früheren Generation eher
als Kompliment denn als Beleidigung angesehen wurde.

Die politisch korrekten und die christlich-fundamentalistischen Gruppen
kritisieren sich gegenseitig heftig, doch in der modernen Welt sehen sie
sich ziemlich ähnlich. Beide setzen die Autorität einer bestimmten
moralischen Doktrin voraus, als ob sie universell wäre, obwohl ihre
moralischen Doktrinen unterschiedlich sind. Beide können tatsächlich
wegen des gleichen Mangels kritisiert werden: wegen eines übertriebenen
und übermäßig selbstsicheren Moralismus, dem es an Tiefe, historischem
Verständnis und Toleranz mangelt. Beide werden wegen ihrer angeblichen
Ähnlichkeit mit dem Puritanismus des siebzehnten Jahrhunderts
angegriffen, mit selbstbewussten Moralisten wie Oliver Cromwell in
England - er wäre beinahe nach Neuengland ausgewandert - oder den
Hexenjägern von Salem. Weder die Frauenbewegung in ihrer dogmatischeren
Form, noch die konservativen Prediger des Bible Belt können eines
Mangels an Moral beschuldigt werden, sondern ihrer Überentwicklung und
Starrheit. Das Herz dieser Moral scheint manchmal zu Stein geworden zu
sein. Diese Art der Verhärtung der moralischen Arterien ist ebenso
schädlich für die Konsensmoral der Gesellschaft wie die
„Alles-ist-erlaubt``-Anarchie, gegen die sie protestiert.

Es handelt sich um eine Verzerrung der moralischen Kräfte, eine
Verrohung in Selbstgerechtigkeit. Pharisäertum, die Überzeugung,
einzigartig tugendhaft zu sein, existiert so lange wie die Menschheit
selbst und war besonders für Jesus Christus anstößig. Die Aushöhlung der
Moral, der Glaube, dass ethische Entscheidungen eine reine Privatsache
sind, so wie die Wahl der Kleidung, ist ein jüngeres Phänomen. Dieser
Glaube spiegelt das Fehlen jeglicher gemeinsamen Moral wider. Er führt
das klassische Konzept der Freiheit auf eine ganz neue Ebene und
verwandelt „das Streben nach Glück``, wie es John Locke ursprünglich
gemeint und Jefferson es 1776 verstanden hatte, in einen Hedonismus, der
die Folgen nicht beachtet.

Der Ausdruck „das Streben nach Glück`` stammt aus John Lockes
\emph{Essay über menschliches Verständnis} (1691). „Die höchste
Vollkommenheit der intellektuellen Natur liegt in einer sorgfältigen
Verfolgung von wahrem und solidem Glück, so dass die Sorge um uns
selbst, dass wir imaginäres Glück nicht mit echtem Glück verwechseln,
die notwendige Grundlage unserer Freiheit ist.`` Er führt jedoch aus,
dass „nicht jeder sein Glück in derselben Sache sieht... der Geist hat
ebenso wie der Gaumen einen anderen Geschmack... Die Menschen können
unterschiedliche Dinge wählen, doch alle wählen richtig, vorausgesetzt,
sie sind nur wie eine Gruppe armer Insekten, von denen einige Bienen
sind, die sich an Blumen und ihrer Süße erfreuen, andere Käfer, die sich
an anderen Arten von Speisen erfreuen.`` Dennoch argumentiert er weiter,
dass die Vorliebe für Laster gegenüber Tugend „offensichtlich ein
falsches Urteil`` ist. Er legt besonderes Gewicht auf das religiöse
Argument, meint aber auch, dass „böse Menschen hier den schlechteren
Teil haben``. Er glaubt, dass „Moral, auf ihren wahren Grundlagen
errichtet, die Wahl in jedem, der nachdenken will, bestimmen kann``.

Die Lockesche Lehre der Freiheit gibt den menschlichen Vorlieben
zweifellos einen größeren Spielraum als autoritärere moralische Systeme,
die darauf abzielen, alle Menschen gleich zu behandeln und
Verhaltenseinheitlichkeit zu erzwingen. Doch schon bald erkennt die
klassische Freiheitslehre die Notwendigkeit kollektiver moralischer
Gebote an, einschließlich der Achtung anderer Menschen in der
Gesellschaft, insbesondere ihres Lebens und des friedlichen Besitzes
ihres Eigentums vor dem Gesetz. Ein allgemeiner Abbau der kollektiven
Moral bedroht die Freiheit, sowohl direkt, indem ein Element der
Anarchie eingeführt wird, als auch indirekt, indem die autoritärsten
Kräfte der Gesellschaft ermutigt werden. Wir können die Geschichte der
öffentlichen Moral als einen Zyklus zwischen Chaos und Autoritarismus
sehen; die moderne autoritäre Moral, sowohl der Feminismus als auch der
Fundamentalismus, sind als zyklische Reaktion auf den Hedonismus der
1960er Jahre entstanden.

Wir haben bereits einige Attribute der neuen Welt des nächsten
Jahrhunderts beschrieben. Sie wird durch zwei Hauptkräfte geprägt sein:
die Verschiebung der Technologie, die die Volkswirtschaften Asiens
öffnet, und die neuen globalen elektronischen Kommunikationsmittel, die
den Bürger immer unabhängiger von ihrer lokalen Regierung machen. Die
neue Technologie wird viele der mittleren menschlichen Qualifikationen
ersetzen oder hat sie bereits ersetzt - den Arbeiter am Fließband, den
Büroangestellten und jetzt zunehmend auch den mittleren Manager. Aber
sie hat die selteneren Fähigkeiten belohnt und eine internationale
kognitive Elite von hochqualifizierten Menschen geschaffen, für die die
neuen Kommunikationswege den größtmöglichen Markt für ihre Fähigkeiten
eröffnen. Wie die meisten Eliten neigen auch die Mitglieder der
kognitiven Elite dazu, sich selbst für etwas Besseres zu halten, sind
eher arrogant und glauben, sie könnten ihre eigenen Standards setzen.
Sie sind als Ergebnis davon von der Gesellschaft entfremdet.

Während der ersten Hälfte des nächsten Jahrhunderts wird ein massiver
Vermögenstransfer von dem alten Westen zum neuen Osten stattfinden.
Politische Fehler - und China ist immer noch ein politisch rückständiges
Land - könnten diesen Transfer von Reichtum und strategischer Macht
verzögern, aber es ist höchst unwahrscheinlich, dass sie ihn verhindern
können. Sie können ihn nicht umkehren.

Dieser Prozess der Wohlstandsverschiebung wird in jedem Fall den
größtmöglichen Druck auf die von Weißen dominierten Länder der
Nordhalbkugel, auf Europa und Nordamerika, ausüben. Derzeit gehören etwa
750 Millionen Menschen zu den fortschrittlichen Ländern dieser Region;
bis vor kurzem war Japan das einzige asiatische, nicht-weiße Land, das
den euro-amerikanischen Lebensstandard erreicht hat, obwohl es ethnisch
europäische Bevölkerungsgruppen in Neuseeland, Australien und in der
weißen Bevölkerung Südafrikas gab. Sogar 1990 bestand die
Gesamtbevölkerung der fortschrittlichen Industrieländer nur aus etwa 15
Prozent der Weltbevölkerung von 5 Milliarden. Die Form der Verteilung
des weltweiten Reichtums war 15 Prozent reich, 85 Prozent arm, sehr
ähnlich der Einkommensverteilung in fortschrittlichen
Industriegesellschaften vor hundert Jahren. Bis 2050, in einem
beschleunigenden Prozess, wird erwartet, dass die fortschrittlichen
Volkswirtschaften etwa 3 Milliarden Menschen aus einer Weltbevölkerung
umfassen werden, die vielleicht auf 7 Milliarden angestiegen ist, oder
eine Vermögensverteilung von 40 Prozent reich, 60 Prozent arm. Bis zum
Ende des Jahrhunderts könnten diese Zahlen gut umgekehrt sein, und die
Verteilung könnte 60 Prozent reich und 40 Prozent arm sein, wobei die
Armut besonders in Afrika konzentriert ist. Die Verschiebung zwischen
den Nationen wird hin zu einer größeren Gleichheit des Reichtums sein,
aber innerhalb der Nationen wird sie wahrscheinlich zu größerer
Ungleichheit tendieren. Die effizienten Nutzer von Talent und Kapital
werden einen entscheidenden Vorteil gegenüber denen mit mäßigen
Fähigkeiten oder wenig Kapital haben. Dieser Reichtum wird hochmobil
sein. Die Armen in der fortschrittlichen Welt werden die Reichen nicht
in dem Maß des zwanzigsten Jahrhunderts besteuern können; die Länder,
die dies versuchen, werden in einem intensiv wettbewerbsorientierten
Rennen zurückfallen.

Natürlich wird die Gesamtproduktivität der Weltwirtschaft weiter
steigen, möglicherweise um durchschnittlich 3 Prozent in der ganzen
Welt, wenn es keinen Weltkrieg gibt. Wenn das zutrifft, wird sich das
gesamte Weltprodukt alle fünfundzwanzig Jahre verdoppeln, was bedeutet,
dass es bis 2050 mehr als viermal so groß wie jetzt und bis 2100
sechzehn bis zwanzigmal so groß sein wird. Selbst wenn die
Weltbevölkerung bis 2100 auf 8 Milliarden angestiegen ist, wird das
weltweite BIP pro Kopf am Ende des Jahrhunderts zehnmal so hoch sein wie
jetzt. Ein solcher Anstieg des Wohlstands kann den Aufstieg der neuen
Industriegesellschaften und die Millioneneinkommen der kognitiven Elite
versorgen und trotzdem einen anständigen und steigenden Lebensstandard
für den Rest der fortgeschrittenen Belegschaft sichern. Aber die
Differenzen werden deutlich anders sein als im zwanzigsten Jahrhundert.
In globaler Hinsicht werden die armen Nationen ihre Einkommen viel
schneller wachsen sehen als die der reichen Nationen; national gesehen,
werden die Einkommen der Reichen, wie in Amerika in den 1990er Jahren,
viel schneller wachsen als mittlere oder niedrige Einkommen. Im nächsten
Jahrhundert werden wir die Schaffung einer weltweiten Superklasse
erleben, vielleicht von 500 Millionen sehr reichen Menschen, von denen
100 Millionen reich genug sein werden, um als souveräne Individuen
hervorzugehen.

Dieser Prozess wird eine unvermeidliche Konsequenz haben. Gesellschaften
werden viel weniger homogen werden; der Nationalstaat wird schwächer
werden oder ganz zerfallen; die kognitive Elite wird sich als
kosmopolitisch betrachten. Bereits jetzt entwickeln Menschen, die in den
gleichen globalen Funktionen arbeiten, eine Kultur, die ihrer Arbeit in
anderen Teilen der Welt näher steht, als ihren Mitbürgern in den alten
Nationalstaaten. Ein Londoner Investmentbanker wird sich wahrscheinlich
mehr in Seoul zu Hause fühlen als in Glasgow; ein Beamter aus Washington
könnte sich in Bonn mehr zu Hause fühlen als in schwarzen Vierteln von
Washington selbst. Wir können bereits den fragmentierenden Effekt
beobachten, den dieser Prozess auf moralische Werte hat. Die Moral des
Einzelnen wird zum Teil durch Bildung geprägt, durch das, was dem
Einzelnen als Kind beigebracht wurde; sie wird auch teilweise durch
Lebenserfahrungen geprägt. Sowohl die Bildung als auch die Erfahrungen
der kognitiven Elite werden kosmopolitisch sein und dazu neigen,
Menschen von ihren lokalen Gemeinschaften zu entfremden.

In dem Maße, wie wir uns auf das nächste Jahrhundert zubewegen, haben
viele Menschen in der wachsenden kognitiven Elite kaum religiöse oder
moralische Erziehung in der Familie bekommen. Die verbreitetste Religion
der Elite ist ein agnostischer Humanismus. Viele solcher Familien sind
selbst durch Scheidung, Wiederverheiratung und nachfolgende dritte Ehen
gespalten. Das Ehemuster von Hollywood ist nicht überall in den USA
anzutreffen, aber die kognitive Elite in Euro-Amerika hat eine hohe
Scheidungsquote, die wahrscheinlich ein Drittel oder mehr beträgt. Die
Kinder dieser geschiedenen Eltern haben selten eine grundlegende
religiöse Erziehung und sind sich der Variabilität der moralischen
Einstellung zwischen den Eltern, Stiefeltern und Stiefgeschwistern
bewusst. Vergleicht man die ursprüngliche moralische Erziehung dieser
Gruppe mit der einer irischen oder polnischen Dorfgemeinschaft, so
bietet die Dorferziehung offensichtlich die stärkere religiöse
Ausbildung. Eine gottlose, wurzellose und reiche Elite wird
wahrscheinlich weder glücklich sein noch geliebt werden.

Diese Unzulänglichkeit in der anfänglichen moralischen Bildung, die in
der dominanten Wirtschaftsgruppe des nächsten Jahrhunderts vorherrschen
wird, wird wahrscheinlich durch ihre Lebenserfahrungen verstärkt. Diese
Menschen werden die Disziplin einer fortgeschrittenen technischen
Ausbildung, in welcher Art auch immer, besitzen, um sich auf ihre neue
Rolle als Leiter des neuen elektronischen Universums vorzubereiten. Doch
sie werden daraus nur einige der moralischen Lektionen lernen, die
historisch das Gerüst für menschliches soziales Verhalten gebildet
haben. Nach den Maßstäben von Konfuzius, Buddha oder Plato (500 v.Chr.),
St.~Paulus (50 n.Chr.) oder Mohammed (600 n.Chr.) könnten sie als
moralisch ungebildet gelten. Sie werden die Lektionen wirtschaftlicher
Effizienz, den Einsatz von Ressourcen und die Verfolgung von Geld
vermittelt bekommen haben, aber nicht die Tugenden der Bescheidenheit
oder Selbstlosigkeit, geschweige denn der Keuschheit. Im Wesentlichen
werden die meisten von ihnen wie Heiden mit einem Wertesystem erzogen
worden sein, das näher an das der späten Römischen Republik als an das
Christentum heranreicht. Selbst diese Werte werden höchst individuell
sein, statt geteilt. Gesellschaften, wie wir argumentiert haben, können
nur stark sein, wenn echte moralische Werte weit verbreitet sind. Die
fortschrittlichen Nationen bewegen sich bereits auf eine Situation zu,
in der viele Menschen schwache oder begrenzte moralische Werte haben
werden, andere werden dies durch ein vehementes Festhalten an
irrationalen Werten kompensieren, und nur wenige Werte werden in der
gesamten Gesellschaft geteilt werden. Zweifellos werden einige der
„konkurrierenden territorialen Clubs``, die wir zuvor beschrieben haben,
anspruchsvolle moralische Standards für den Aufenthalt auferlegen.

Unterschiede im Wohlstand haben an sich in der Geschichte nicht zu
fundamentalen Unterschieden in religiösen Werten geführt. In dichten und
stabilen Gesellschaften mit starken Traditionen kann eine steile
hierarchische Struktur -- „der reiche Mann in seiner Burg, der arme Mann
vor seinem Tor`` - Werte verbergen, die sich durch die Hierarchie
ziehen. Dies hängt aber von der Stärke des gemeinschaftlichen Gefühls
der Reichen und Armen und der Stärke der sozialen Traditionen ab. Keine
dieser Bedingungen besteht heute, und sowohl das Gemeinschaftsgefühl als
auch die Traditionen werden durch die wirtschaftliche und technologische
Revolution, die gerade stattfindet, geschwächt. Die Leben der vielen und
der wenigen entfernen sich immer mehr voneinander. Die technologische
Revolution wurde durch den Bruch mit den alten Methoden erreicht. In
jedem Bereich waren es die Radikalen, die gesiegt haben und die
konventionellen Denker, die ins Hintertreffen geraten und buchstäblich
aus dem Rennen gefallen sind. Unsere Politik mag von konventionellen
Denkern - Bill Clinton, Helmut Kohl, John Major - geführt werden, aber
unsere erfolgreichsten Unternehmen werden von Radikalen mit einem tiefen
Verständnis für die neue technologische Welt angeführt; das
Paradebeispiel ist Bill Gates. Das konventionelle Denken wurde durch
seine Unfähigkeit, mit der Geschwindigkeit und der schieren Kraft des
Wandels umzugehen, diskreditiert.

Und doch ist Moral nicht so. Wenn wir die Wissenschaft des Moses nehmen,
die etwa 1000 v. Chr. entstanden ist, hat sie uns sehr wenig zu sagen.
Der Bericht über die Schöpfung im Buch Genesis mag eine theologische
Wahrheit enthalten - Gott schuf das Universum und die Menschheit, gibt
aber keinen wissenschaftlichen Bericht über die tatsächliche Entwicklung
physischer Strukturen. Wenn wir jedoch die Moral des Moses - die Zehn
Gebote - betrachten, hat sie uns eine Menge zu sagen.

Respekt für die Eltern und Treue in der Ehe sind die besten Wege, das
Familienleben zu bewahren; das Familienleben ist der beste Weg,
moralisch gesunde Kinder großzuziehen. Diebstahl schadet dem Dieb und
den Menschen, von denen Sachen gestohlen werden, und hemmt die Anreize
zu arbeiten und zu sparen. Die soziale Ordnung hängt von der Wahrheit
der Zeugen ab. Es ist falsch zu morden, und so weiter.

In der Wissenschaft haben drei Jahrtausende völlig verändert, was
menschliches Wissen ist; in der Moral könnten wir tatsächlich
zurückgefallen sein. Der durchschnittliche Psychotherapeut gibt dem
Patienten wahrscheinlich weniger gute moralische Ratschläge, wie er sein
Leben führen soll, als der durchschnittliche Jude von seinem Lehrer in
der Zeit von Moses erhalten hätte. Natürlich ist das Christentum selbst
noch immer vorhanden, aber es ist für den größten Teil der Welt nur ein
blasser Schatten seines früheren Selbst. Wenige Menschen haben den
Glauben der früheren Zeitalter, oder sogar der weniger ausgeklügelten
Gemeinschaften; man sucht nicht nach Heiligen auf der Park Avenue.

Die Zerstörung der Tradition war eine notwendige Bedingung für den
wissenschaftlichen Fortschritt. Wenn wir alle noch glauben würden, dass
die Sonne sich um die Erde dreht, hätten wir keine
Satellitenkommunikation entwickelt. Was wir tatsächlich für Wissenschaft
halten, ist nichts weiter als eine Reihe von Hypothesen, unvollkommene
Erklärungen, die durch andere, stärkere, aber immer noch unvollkommene
Erklärungen ersetzt werden. Und doch hat die Zerstörung der Tradition
ein Desaster für die moralische Ordnung der Welt bedeutet.

Konfuzius lehrte, dass wir immer Maß halten sollten (er nannte das
goldene Mittel \emph{Chum Yum}, zumindest wurde es so von Gelehrten des
siebzehnten Jahrhunderts übersetzt). Er lehrte auch, dass wir die
Autorität respektieren und andere so behandeln sollten, wie wir selbst
behandelt werden möchten. Diese Lehre ist zweihundertfünfzig Jahre alt.
Als Tradition beeinflusste sie China über die gesamte aufgezeichnete
Geschichte hinweg, doch dem Konfuzianismus messen viele moderne Chinesen
keine Bedeutung mehr bei, die keine Mäßigung schätzen, die Kraft statt
Autorität respektieren und sicherlich andere nicht so behandeln, wie sie
selbst behandelt werden möchten. Mit dem Verlust der Tradition können
Gesellschaften das ganze Vokabular ihres moralischen Konsenses
verlieren. China, mit all seiner aufstrebenden Macht, ist heute im
Vergleich zu Tibet ein moralisch rückständiges Land, so arm und
unterdrückt die Tibeter auch sein mögen.

Eine gute soziale Moral hat bestimmte Eigenschaften. Sie sollte zur
Überlebensfähigkeit der Gesellschaft und von Individuen beitragen, und
zwar auf dynamische und nicht auf statische Weise. Sie sollte Toleranz
beinhalten und Selbstgerechtigkeit vermeiden. Sie sollte religiös sein,
nicht bloß agnostisch. Sie sollte nicht vorgeben, Fragen
wissenschaftlicher Fakten entscheiden zu können. Sie sollte weder
anarchistisch noch autoritär sein. Sie sollte weit verbreitet und tief
verankert sein. Eine solche soziale Moral ist besonders wichtig für die
Familie und für die Erziehung von Kindern zu unabhängigen und
verantwortungsbewussten Erwachsenen. Sie bildet den Mittelpunkt einer
guten Gesellschaft.

Wir stellen fest, dass eine solche Moral von der Logik der gegenseitigen
Abhängigkeit getragen wird, die sich aus dem Handel und dem Miteinander
ergibt, aber durch die Angriffe eines oberflächlichen Szientismus, durch
die Entfremdung einer Ober- und einer Unterklasse, durch den Verlust der
Verwurzelung der alten geografischen Ökonomien bedroht ist. Vielleicht
wird es eine Reaktion gegen diese Tendenzen geben. Sie müssen als
äußerst gefährlich für die Gesellschaften des nächsten Jahrhunderts
erkannt werden.

Mit dem Ende des „schrecklichsten Jahrhunderts der abendländischen
Geschichte``, wie es Isaiah Berlin nannte, geht auch das Zeitalter des
Gigantismus in der Sozialstruktur zu Ende. Die letzten Tage des
zwanzigsten Jahrhunderts sind dazu bestimmt, eine Zeit der
Verkleinerung, der Dezentralisierung und der Neuorganisation zu sein. Es
wird die Zeit der sozialen Dinosaurier sein, die in der Teergrube
gefangen sind. Und eine Zeit der Aasfresser. Vögel werden die Knochen
der Dinosaurier picken. Regierungen, Unternehmen und Gewerkschaften
werden gezwungen sein, sich gegen ihre Neigungen hin zu neuen
Meta-Konstitutionsbedingungen, bedingt durch den Einzug der
Mikrotechnologie, anzupassen. Dies hat die Grenzen, innerhalb derer
Gewalt ausgeübt wird, tiefgreifend verschoben. Die heutige Welt hat sich
bereits mehr verändert, als wir allgemein verstehen, mehr als CNN und
die Zeitungen uns erzählen. Und sie hat sich genau in den Richtungen
verändert, wie sie eine Studie über megapolitische Zustände zeigt. Wie
wir zuerst in \emph{Blood in the Streets} und dann in \emph{The Great
Reckoning} argumentierten, wenn sich Technologie oder andere Faktoren,
die die Grenzen setzen, innerhalb derer Gewalt ausgeübt wird, verändern,
verändert sich unweigerlich auch der Charakter der Gesellschaft. Alles,
was mit der Art und Weise verbunden ist, wie Menschen interagieren,
einschließlich der Moral und des gesunden Menschenverstandes, wie wir
die Welt sehen, wird sich ebenfalls ändern. Nach einer Phase schwacher
Moral, die das Ende einer Ära anzeigt, werden wir ein Erwachen einer
strengeren Moral erleben, mit anspruchsvolleren Forderungen, um den
anspruchsvolleren Anforderungen einer Welt der Wettbewerbssouveränität
zu begegnen.

Mehrere Merkmale der neuen Moral sind absehbar. Zum einen wird sie die
Bedeutung der Produktivität und die Korrektheit der Einbehaltung von
Gewinnen durch diejenigen, die sie erwirtschaften, betonen. Ein weiterer
Schlüsselpunkt wird die Effizienz von Investitionen sein. Die Moral des
Informationszeitalters lobt Effizienz und erkennt den Vorteil an,
Ressourcen ihrer hochwertigsten Nutzung zuzuführen. Mit anderen Worten,
die Moral des Informationszeitalters wird die Moral des Marktes sein.
Wie James Bennett argumentiert hat, wird die Moral des
Informationszeitalters auch eine Moral des Vertrauens sein. Die
Cyberwirtschaft wird eine hochgradig vertrauenswürdige Gemeinschaft
sein. In einem Umfeld, in dem eine unknackbare Verschlüsselung es einem
Veruntreuer oder Dieb ermöglicht, die Erlöse aus seinen Verbrechen
sicher außerhalb des Bereichs der Wiedererlangung zu platzieren, wird es
einen sehr starken Anreiz geben, Verluste zu vermeiden, indem man mit
Dieben und Veruntreuern gar nicht erst Geschäfte macht. Ebenso, wie im
früher zitierten Beispiel der Quäker, wird ein Ruf der Ehrlichkeit ein
wichtiges Gut in der Cyberwirtschaft sein. In der Anonymität des
Cyberspace wird dieser Ruf vielleicht nicht immer einer bekannten Person
gelten, aber er wird zuverlässig verifizierbar sein durch die
Identifizierung von kryptographischen Schlüsseln. Die Möglichkeit, dass
es zu Ausweitungen von Schwierigkeiten kommt, wenn Verschlüsselung oder
Zertifizierung kryptografischer Identitäten durch Gangster oder andere
korrupte Personen missbraucht werden, ist entmutigend genug, dass es
stark gegen die Anstellung von Personen spricht, deren Verhalten auf
mangelnde Vertrauenswürdigkeit hinweist. Bennett stellt sich einen
„Gentlemen's Club des Cyberspace`` vor, geschützte Bereiche, die für die
Teilnahme erhöhte Sicherheitsmaßnahmen verlangen würden, „möglicherweise
mit biometrischer Überprüfung wie Stimmerkennung. Die Betreiber würden
die Verantwortung für die Identifizierung der Teilnehmer und bis zu
einem gewissen Grad für ihre Vertrauenswürdigkeit übernehmen und so
einen ‚Gentlemen's Club im Cyberspace' schaffen (obwohl Damen heutzutage
natürlich willkommen wären). In diesen Bereichen könnten Menschen
Geschäfte mit größerer Sicherheit und Vertrauen als im allgemeinen
Bereich des Cyberspace abwickeln. So könnte das 21. Jahrhundert eine
Rückkehr zu einem viktorianischen Schwerpunkt auf Vertrauenswürdigkeit
und Charakter in einer Umgebung sehen, die sich kein Viktorianer hätte
vorstellen können.``

Die geschützten Bereiche des Cyberspace könnten ebenfalls Garantien
anbieten, um Risiken zu mindern, ähnlich wie die extraterritorialen
Schutzgarantien, die die Grafen von Champagne anboten, um Händler auf
den Messen von Champagne zu schützen. Andere Gerichtsbarkeiten haben
tatsächlich „Händler auf Reisen vor jeglichen Verlusten geschützt, die
sie möglicherweise erleiden könnten, während sie durch das Territorium
unter der Gerichtsbarkeit des jeweiligen Adeligen ziehen.``

„Die Wächter der Messe``, ursprünglich von den Grafen ernannte Beamte,
boten Sicherheit und ein „Tribunal der Gerechtigkeit`` für die Händler
auf der Messe. Sie entwickelten sich schließlich zu eigenständigeren
Instanzen, mit einem separaten Siegel, das Verträge beglaubigte und
deren Erfüllung erzwang mit der Macht, „jeden Händler, der einer
Nichtzahlung seiner Schulden oder der Nichterfüllung seiner
vertraglichen Versprechen für schuldig befunden wurde, von zukünftigen
Messen auszuschließen. Dies war offensichtlich eine so schwere Strafe,
dass nur wenige bereit waren, diesen Entzug von zukünftigen
Gewinnmöglichkeiten freiwillig zu riskieren. Andernfalls könnten die
Wachen die Güter eines säumigen Schuldners beschlagnahmen und zugunsten
seiner Gläubiger verkaufen.`` \footnote{James Bennett, \emph{Cyberspace
  and the Return of Trust}, Strategic Investment, Oktober 1996.}

Die Ächtung als Mittel zur Durchsetzung von Verträgen verlor an
Bedeutung, als die Anzahl alternativer Märkte stieg. Mit der nun
verfügbaren neuen Informationstechnologie könnte die Ächtung von
Betrügern und Vertragsbrechern jedoch wieder zu einem starken
Durchsetzungsmechanismus in den fragmentierten Souveränitäten der
nächsten Gesellschaftsstufe werden. Computer-Verknüpfungen können den
Cyberspace mit unauslöschlichen Informationen über Kredit und Betrug
überwachen. Da die Welt in diesem Sinne eine besonders kleine
Gemeinschaft sein wird, werden Betrüger und Schwindler entmutigt werden.

Zusätzlich zur Betonung der Moral von Einnahmen und Effizienz und der
erneuten Betonung von Charakter und Vertrauenswürdigkeit, wird die neue
Moral vermutlich auch die Boshaftigkeit von Gewalt hervorheben,
insbesondere Entführung und Erpressung, die als Mittel zum „Ausnehmen``
von Individuen zunehmen werden, deren Ressourcen ohne diese Methoden
nicht leicht zur Beute von Kriminalität werden würden.

Ein weiterer wahrscheinlicher Antrieb zu strengerer Moral wird das Ende
von Ansprüchen und Einkommensumverteilung sein. Wenn die Hoffnung auf
Hilfe für diejenigen, die zurückfallen, hauptsächlich auf Aufrufen an
Privatpersonen und wohltätige Körperschaften beruht, wird es wichtiger
als im zwanzigsten Jahrhundert sein, dass die Empfänger von
Wohltätigkeit gegenüber denen, die freiwillig Wohltätigkeit leisten, als
moralisch würdig erscheinen.

\begin{quote}
„Subventionen, Geldgeschenke und die Aussicht auf wirtschaftliche
Möglichkeiten lassen die Notwendigkeit des Umweltschutzes in den
Hintergrund treten. Die Mantras von Demokratie, Umverteilung und
wirtschaftlicher Entwicklung steigern die Erwartungen und die
Fruchtbarkeitsraten, fördern das Bevölkerungswachstum und verstärken
damit eine ökologische und wirtschaftliche Abwärtsspirale.`` \footnote{Virginia
  Abernethy, \emph{Optimism and Overpopulation}, Atlantic Monthly
  December 1994, S. 88.} - Virginia Abernathy
\end{quote}

In gewisser Weise wird die neue Informationswelt besser in der Lage
sein, Ernsthaftigkeit in moralischen Fragen zu fördern. Die Versprechen
einer Einkommensumverteilung, die in den USA, Kanada und Westeuropa
Erwartungen unter den Unglücklichen und Erfolglosen geschürt haben,
haben auch international eine perverse Wirkung gehabt. Es gibt starke
Hinweise darauf, dass ausländische Hilfe und Versprechen von
Interventionen zur Abwendung von Hungersnöten und zur Steigerung der
Lebensstandards wichtige Faktoren für das Bevölkerungswachstum waren,
das die Tragfähigkeit von rückständigen Wirtschaften übersteigt. Das
erschreckende Wachstum der Weltbevölkerung seit dem Zweiten Weltkrieg,
mit seinen oft zerstörerischen Auswirkungen auf Wälder, Böden und
Wasserressourcen, kann auf eine globale Intervention zurückgeführt
werden. Diese Intervention hat die negativen Rückkopplungsfolgen
umgangen, die lange Zeit dafür gesorgt haben, dass lokale Bevölkerungen
im Einklang mit den Ressourcen standen, die zur Unterstützung von ihnen
notwendig waren.

Natürlich waren viele, die in lokalen Umgebungen mit wenigen Ressourcen
und wenig oder keinem Wachstum lebten, nur allzu erfreut zu hören, dass
die einschränkenden Begrenzungen ihres Dorflebens beiseite gelegt werden
könnten. Sie nahmen die optimistische Botschaft von internationalen
Hilfsarbeitern, Freiwilligen des Friedenskorps, lokalen Revolutionären
und den konkurrierenden Ideologen des Kalten Krieges, die jedem
versprachen, dass bessere Zeiten bevorstünden, begierig an. Dies war
genau die falsche Botschaft.

Eine wichtige Auswirkung der Umverteilung zwischen Kulturen besteht
darin, dass Menschen, die in nichtindustriellen Zivilisationen lebten
und nichtindustrielle Werte beibehielten, künstlich wettbewerbsfähig
gemacht wurden. Internationale Hilfe, Rettungseinsätze zur Bekämpfung
von Hungersnöten und Krankheiten sowie technische Interventionen
gaukelten vielen vor, dass ihre Lebensaussichten sich stark verbessert
hätten, ohne dass sie ihre Werte aktualisieren oder ihr Verhalten
signifikant ändern mussten.

Die internationale Einkommensumverteilung hat nicht nur zu einem nicht
nachhaltigen Bevölkerungswachstum in der Welt beigetragen, sondern auch
in wichtigen Aspekten zur kulturellen Relativität und zur weit
verbreiteten Verwirrung über die entscheidende Rolle der Kultur für die
Menschen, um in ihrer lokalen Umgebung zu gedeihen. Heutzutage glauben
die meisten Menschen, dass Kulturen eher Geschmacksache sind als
Ressourcen, die das Verhalten lenken und sowohl irreführen als auch
informieren können. Wir sind zu sehr darauf versessen zu glauben, dass
alle Kulturen gleich sind, und zu langsam, um die Nachteile
kontraproduktiver Kulturen zu erkennen. Das gilt insbesondere für die
hybriden Kulturen, die in diesem Jahrhundert in vielen Teilen der Welt
im Treibhaus von Subventionen und Interventionen zu entstehen begonnen
haben. Wie die kriminelle Subkultur in den inneren Städten Amerikas
behalten sie inkohärente Teile von Kulturen aus früheren Phasen der
wirtschaftlichen Entwicklung bei und kombinieren sie mit
Wertvorstellungen für das Verhalten im Informationszeitalter.

Die Informationsrevolution wird daher nicht nur das Genie in uns
entfesseln, sie wird auch den Geist der Nemesis entfachen. Beide werden
im kommenden Jahrtausend wie nie zuvor um Dominanz ringen.

Der Übergang von einer Industrie- zu einer Informationsgesellschaft wird
atemberaubend sein. Der Übergang von einer Wirtschaftsphase zur nächsten
hat immer eine Revolution mit sich gebracht. Wir glauben, dass die
Informationsrevolution wahrscheinlich die weitreichendste von allen sein
wird. Sie wird das Leben grundlegender neu organisieren als die
Agrarrevolution oder die industrielle Revolution. Und ihre Auswirkungen
werden in einem Bruchteil der Zeit zu spüren sein. Schnallen Sie sich
an.

\bookmarksetup{startatroot}

\chapter*{Afterword: Devolution and the law of diminishing
returns}\label{afterword-devolution-and-the-law-of-diminishing-returns}
\addcontentsline{toc}{chapter}{Afterword: Devolution and the law of
diminishing returns}

\markboth{Afterword: Devolution and the law of diminishing
returns}{Afterword: Devolution and the law of diminishing returns}

\begin{quote}
„Was über seine Maße aufgebläht ist, wird unweigerlich zusammenbrechen.
... Was konzentriert, kohärent und mit seiner Vergangenheit verbunden
ist, hat Kraft. Was zerstreut, geteilt und aufgebläht ist, verrottet und
fällt zu Boden. Je mehr es aufgebläht ist, desto heftiger ist der
Fall.'' --- Robert Greene und Joost Elffers, Die 48 Gesetze der Macht
\end{quote}

Bisher ist die Geschichte der menschlichen Gesellschaften durch eine
Tendenz gekennzeichnet, sich in Richtung größerer „Komplexität'' oder
sozialpolitischer Kontrolle zu entwickeln. Kleine Jagd- und
Sammelgruppen entwickelten sich zu agrarischen Staaten, die schließlich
größeren industriellen Nationalstaaten Platz machten. Wie der Archäologe
und Historiker Joseph A. Tainter in seinem Werk „Der Zusammenbruch
komplexer Gesellschaften'' schreibt: „Die Geschichte der Menschheit
insgesamt ist durch eine scheinbar unaufhaltsame Tendenz zu höheren
Komplexitäts-, Spezialisierungs- und sozialpolitischen Kontrollniveaus
gekennzeichnet. ...'' Doch nun verspricht das Aufkommen der nächsten
Entwicklungsstufe der Wirtschaft, der Informationsgesellschaft, eine
Umkehrung des scheinbar „unaufhaltsamen Trends'' zu höheren Graden der
Zentralisierung.

Tainers Arbeit wirft viele interessante Fragen auf, die für die Themen
dieses Buches relevant sind. Beispielsweise, wenn Tainer recht hat mit
der Annahme, dass die Dezentralisierung von Kontrolle und eine
verringerte Ressourcenumverteilung einen Zusammenbruch implizieren, dann
ist es unwahrscheinlich, dass der industrielle Nationalstaat in seiner
gegenwärtigen Form langfristig mit dezentralisierten Mikrostaaten
koexistieren kann, die souveräne Individuen beherbergen. Die
Nationalstaaten könnten möglicherweise nicht auf einer Ernährung aus
stabilen, geschweige denn verringerten Ressourcen überleben. Wie Tainer
detailliert beschreibt, tritt bei hypertrophierten Systemen, die ihr
Potenzial erschöpft haben -- wie wir glauben, dass es bei den
Nationalstaaten heute der Fall ist --, häufig „das Gesetz des
abnehmenden Grenzertrags'' ein. In „vielen entscheidenden Bereichen''
sinken die Erträge aus vermehrten Investitionen in die zentralisierte
sozialpolitische Kontrolle oder werden sogar negativ. Dies erklärt das
Phänomen „Parkinsons Gesetz'', bei dem die Zahl der Angestellten und die
Kosten für den Betrieb der britischen Admiralität im zwanzigsten
Jahrhundert in die Höhe schnellen, während die Anzahl der Schiffe in der
britischen Marine dramatisch zurückgeht.

Ähnliche Erscheinungsformen des „Gesetzes der abnehmenden Erträge'' sind
sicherlich zu beobachten, während das zwanzigste Jahrhundert sich dem
Ende zuneigt, sowohl in den Vereinigten Staaten als auch in anderen
führenden Volkswirtschaften. Wie Roger Lane, Professor für
Sozialwissenschaften am Haverford College, in seinem Aufsatz „Zur
sozialen Bedeutung der Mordtrends in Amerika'' schrieb: „Die alten
Institutionen der sozialen Kontrolle -- Gesetze, Schulen, Polizei,
Gefängnisse -- haben an Wirksamkeit verloren, trotz häufiger Zuführungen
von Arbeitskräften und Geld.'' Es gibt eindeutige Belege für steigende
Kosten, die mit den Gesamterfordernissen der Staatsführung verbunden
sind. So sind die Gesamteinnahmen aus Steuern von 27,8 Prozent des
mittleren Einkommens in den USA im Jahr 1957 auf 37,6 Prozent im Jahr
1997 gestiegen. Das ist ein starkes Indiz, wenn nicht sogar ein
eindeutiger Beweis für abnehmende Grenzerträge im gesamten Bereich
staatlicher Aktivitäten in den Vereinigten Staaten.

In der Vergangenheit waren stark abnehmende Grenzerträge häufig Vorboten
eines Zusammenbruchs. Die Argumentation dieses Buches besagt, dass die
gesteigerte Fähigkeit von Individuen, ihre Transaktionen und
Vermögenswerte vor raubtierhaften Steuern zu schützen, einen Rückgang
der Umverteilung von Ressourcen zur Folge hat, sowie eine geringere
zentrale soziale Kontrolle, weniger Regulierung und Vereinheitlichung
und letztlich eine Dezentralisierung des Territoriums. All diese
Entwicklungen haben sich historisch in Form eines „Zusammenbruchs''
manifestiert. In Tainers Worten bezeichnet „Zusammenbruch'' den Zustand,
wenn ein zentrales Kontrollsystem nicht mehr das wert ist, was es
kostet.

\begin{quote}
„Wann immer wir ein Schwellenphänomen beobachten, sei es in physischen,
biologischen oder sozialen Systemen, wird die Konfiguration des Systems
in dem Moment, in dem die Schwelle erreicht wird, instabil. Bereits die
geringste, selbst infinitesimale Veränderung in der Konfiguration des
Systems kann daher eine Veränderung im Verhalten eines einzelnen
Individuums, egal wie klein sie auch sein mag, auslösen, die in einem
instabilen sozialen Konfigurationsprozess zu einer begrenzten und
manchmal radikalen Veränderung führt.'' --- Nicholas Rashevsky, Die
Geschichte durch die Mathematik betrachten
\end{quote}

Während die meisten individuellen Anpassungen an Veränderungen
zugegebenermaßen marginal und evolutionär geprägt sind, kann es durchaus
revolutionäre „Paradigmenwechsel'' geben. Manchmal stürzen sogar große
Reiche als Folge hiervon. Die Grenzerträge aus weiteren Investitionen in
zentrale Kontrolle können so überwältigend negativ werden, dass es für
die meisten Individuen nicht mehr wirtschaftlich rational ist, das alte
System weiterhin zu unterstützen. Tainter erklärt den Fall des Römischen
Reiches mit diesen Worten: „Wenn man den Berichten Glauben schenken
darf, begrüßt zumindest ein Teil der überbesteuerten Landbevölkerung
offen die Erleichterung, die sie sich von den Barbaren erhofften, um die
Lasten der römischen Herrschaft loszuwerden. Ein wesentlich größerer
Teil war offensichtlich apathisch gegenüber dem bevorstehenden
Zusammenbruch. \ldots{} Die Kosten des Imperiums waren dramatisch
gestiegen, während angesichts der Erfolge der Barbaren der Schutz, den
der Staat vielen seiner Bürger bieten konnte, zunehmend ineffektiv
wurde. Für viele gab es schlichtweg keine verbleibenden Vorteile des
Imperiums, da sowohl Barbaren als auch Steuerbeamte ihr Land
durchquerten und verwüsteten. Wie Gunderson feststellt: ‚\ldots{} der
Nettowert lokaler Autonomie überstieg den der Mitgliedschaft im
Imperium.' Die Komplexität brachte keine Vorteile mehr, die die
Zersetzung übertrafen, und dennoch kostete sie erheblich mehr.''

Tainter zitiert andere Autoritäten, um seine These zu untermauern, dass
ein Zusammenbruch „eine entsprechende Steigerung des Grenzertrags von
sozialen Investitionen mit sich bringen kann.''

\begin{quote}
Zosimus, ein Schriftsteller aus der zweiten Hälfte des fünften
Jahrhunderts n.~Chr., schrieb über Thessalien und Makedonien, dass „...
infolge dieser Steuererhebungen Stadt und Land voller Klagen und
Beschwerden waren und alle die Barbaren anriefen und um ihre Hilfe
baten.'' \ldots{} „Im fünften Jahrhundert,'' schlussfolgert R.M. Adams,
„waren die Menschen bereit, die Zivilisation selbst aufzugeben, um dem
erschreckenden Steuerdruck zu entkommen.''
\end{quote}

Rashevskys Analyse der „Rolle des Determinismus versus Indeterminismus''
in der Geschichte hebt die Verwundbarkeit von Systemen für radikale
Veränderungen hervor, die sogar durch eine einzige Person ausgelöst
werden können, wenn das System instabil wird und einen „Schwellenwert''
erreicht. Wenn die Bedingungen für Veränderungen günstig sind
(beispielsweise wenn die Grenzerträge für die Unterstützung eines
zentralen Systems nicht mehr „überlegene Vorteile gegenüber der
Zersetzung'' bieten), ist die Möglichkeit eines radikalen Wandels so
stark, dass praktisch jeder ihn herbeiführen kann. Rashevsky schreibt:
„Die Person, die eine endliche Veränderung herbeiführt, muss kein
außergewöhnlicher Mensch sein. Sie kann jede beliebige Person sein. Die
Situation ist analog zu einem physikalischen System, in dem an einem
Punkt der Instabilität eine zufällige Verschiebung eines der Billionen
identischen Moleküle einen endlichen Übergang zu einem stabilen Zustand
auslöst.''

Wir können nicht genau bestimmen, wer den Zusammenbruch des überdehnten
Nationalstaatensystems herbeiführen wird oder wann dies geschehen wird.
Aber aus den Analysen von Tainter und Rashevsky über die Dynamik
sozialen Wandels können wir einen bevorstehenden Zusammenbruch ahnen.
Die am weitesten entwickelten und bislang erfolgreichen Nationalstaaten
zeichnen sich alle durch schrumpfende Bevölkerungen und massive, nicht
finanzierte Rentenverpflichtungen aus. Ohne beispiellose Einwanderung
aus unterentwickelten Ländern oder einen unerwarteten Zustrom von
Engeln, die bereit sind, Überstunden zu leisten und konfiskatorische
Steuersätze zu bezahlen, werden die führenden Staaten in Europa,
Nordamerika und Australasien bei den Einnahmen weit hinter dem
zurückbleiben, was nötig wäre, um die derzeitigen sozialen Leistungen
aufrechtzuerhalten. Aktuare prognostizieren steigende Steuern und
sinkende Vorteile, d.h. abnehmende Grenzerträge, insbesondere für
Unternehmer, die einen unverhältnismäßig hohen Anteil der Steuerlast
tragen.

Die Zahlen des IRS zeigen, dass im Jahr 1997 ein Zehntel Prozent der
Amerikaner den Großteil der Einkommensteuern in den Vereinigten Staaten
zahlte. Es sind genau diese Personen, denen effiziente
Mini-Souveränitäten neue Möglichkeiten für ihren Wohnsitz zu
vernachlässigbaren Steuern bieten können. Der Unterschied zwischen den
Schutzkosten einer kommerzialisierten Souveränität und den
ausbeuterischen Steuern, die von den alten Nationalstaaten auferlegt
werden, könnte sich auf viele Millionen oder sogar mehrere Milliarden
Dollar an lebenslangem Einkommen summieren.

Die herkömmliche Mikroökonomie beruht auf der Annahme, dass Individuen,
die einen \$100-Schein auf der Straße sehen, ihn aufheben werden.
Möglichkeiten, Millionen oder Milliarden zu sparen, wären zehntausend-
oder sogar millionenfach überzeugender. Menschen werden in der
beschriebenen Weise handeln, wenn sie vor der Wahl stehen, ihre
kostspielige Loyalität gegenüber Institutionen aufrechtzuerhalten, die
von abnehmenden Grenzerträgen betroffen sind, oder sich neuen Strukturen
zuzuwenden, die weniger verlangen und mehr versprechen.

\begin{quote}
„Von all den 36 Möglichkeiten, aus der Klemme zu kommen, ist der beste
Weg: geh einfach weg.'' --- Chinesisches Sprichwort
\end{quote}

Das Argument dieses Buches legt eindrücklich dar, warum es an der Zeit
ist, Ihr Kapital neu zu verteilen, sofern Sie über welches verfügen. Die
Staatsbürgerschaft ist überholt. Um Ihr lebenslanges Einkommen zu
optimieren und zu einem souveränen Individuum zu werden, müssen Sie
vielmehr Kunde von Regierung oder Schutzdiensten werden, anstatt sich
als Bürger zu verstehen. Anstatt die Steuerlast zu tragen, die Ihnen von
gierigen Politikern auferlegt wird, sind Sie besser aufgestellt, um im
Informationszeitalter erfolgreich zu sein, indem Sie sich die Freiheit
verschaffen, einen privaten Steuervertrag auszuhandeln, der Sie
verpflichtet, nicht mehr für staatliche Dienstleistungen zu zahlen, als
sie Ihnen tatsächlich wert sind.

Basierend auf der Geschichte anderer dominierender Systeme, die vor dem
Zusammenbruch stehen, werden diejenigen, die sich für das Ultimum
refugium entscheiden und frühzeitig aussteigen, davon profitieren. Dies
zeigt sich bereits in der Flut von Gesetzen, die in den 1990er Jahren
verabschiedet wurden, um Amerikaner zu bestrafen, die ihre
Staatsbürgerschaft niederlegen. Die Gefahren einer nationalistischen
Reaktion auf die Krise des Nationalstaats machen es wichtig, den
Spielraum für Tyrannei und Unheil nicht zu unterschätzen. Ungeachtet der
Tatsache, dass das Recht auf Auswanderung in der
Unabhängigkeitserklärung der USA verankert ist, ist es wahrscheinlich,
dass die USA eine der tyrannischeren Jurisdiktionen sein werden, die das
Aufkommen einer commercialisierten Souveränität blockieren. Sie sollten
stets darauf achten, Ihr Geld nicht in einer Jurisdiktion zu belassen,
die sich das Recht anmaßt, Sie, Ihre Kinder oder Ihre Enkel zu
verpflichten.

Egal wo Sie derzeit wohnen oder welche Nationalität Sie haben, um Ihren
Reichtum zu optimieren, sollten Sie anstreben, hauptsächlich in einem
anderen Land als dem Ihrer ersten Staatsbürgerschaft zu leben, während
Sie den Großteil Ihres Geldes in einer dritten Jurisdiktion anlegen,
idealerweise in einem Steuerparadies.

Um sich besser mit den Alternativen vertraut zu machen, empfehlen wir
Ihnen, umfassend zu reisen und attraktive Orte zu besuchen, an denen Sie
im Notfall das Recht auf Aufenthalt sichern möchten.

Wenn Sie wirklich ambitioniert sind, möchten Sie möglicherweise sogar
eine eigene Minisouveränität schaffen. In den Anhängen finden Sie
Kontakte, die Ihnen dabei helfen können, Ihre eigene steuerfreie Zone
oder zona franca zu verhandeln, von einer anerkannten Regierung, die
bereit ist, unter bestimmten Umständen Souveränität zu übertragen.

Angenommen, Sie stehen am Anfang\ldots{}

Angenommen, Sie stimmen den Grundannahmen dieses Buches zu und sind
begeistert von der Aussicht auf das Informationszeitalter, haben jedoch
nicht das nötige Kapital, um die Chancen zu nutzen und von der
kommerzialisierten Souveränität zu profitieren. Was tun Sie dann?

Jede Rezeptur für einfachen Erfolg wird zwangsläufig enttäuschen. Die
Möglichkeiten, erfolgreich zu sein, sind infolge der
Informationsrevolution in Hülle und Fülle vorhanden. Welche dieser
Möglichkeiten für Sie die richtige ist, lässt sich nicht pauschal sagen.
Wenn Sie entschlossen sind, Kapital anzusammeln, um Ihr volles Potenzial
als souveräne Person zu entfalten, sollten Sie es sich zur Priorität
machen, die Werke verschiedener Gurus zu studieren und zu bewerten, die
hilfreiche Tipps zum Erfolg anbieten.

Jede gute Fachbuchhandlung oder einer der Online-Buchhändler, wie zum
Beispiel Amazon, bietet eine große Auswahl an Ratgebern zum Thema
Erfolg. Lesen Sie so viele wie möglich, nicht mit der Vorstellung, dass
eine bestimmte Reihe von Regeln Ihnen automatisch finanzielle
Unabhängigkeit verschafft, sondern mit dem Verständnis, dass Erfolg eine
Entscheidung ist. Wenn Sie erfolgreich sein möchten, müssen Sie sich mit
der Perspektive und den Gewohnheiten ausstatten, die erfolgreiche
Menschen auszeichnen.

Wenn Sie sich noch in der Phase der Berufswahl befinden, widerstehen Sie
der Versuchung, zu der einfachen Schlussfolgerung zu gelangen, dass der
beste Weg zum Erfolg im Informationszeitalter darin besteht,
Computerprogrammierer zu werden. Ja, es stimmt, dass Programmierer
während der Informationsrevolution, die im letzten Viertel des
zwanzigsten Jahrhunderts stattfand, stark nachgefragt wurden. Doch mit
zunehmender Rechenleistung hat sich auch die künstliche Intelligenz
rasant entwickelt. Ein Unternehmen namens Authorgeneics hat bereits
gezeigt, dass es in der Lage ist, objektorientierte Software ohne
Programmierer zu erstellen. Sie werden nicht gut bezahlt, wenn Sie etwas
studieren, das mit einer magischen Lampe wie Aladdins Lampe erledigt
werden kann. Das Problem bei der Spezialisierung auf Software oder ein
anderes schnelllebiges Feld im Zentrum der Informationsrevolution ist,
dass Ihr Fachgebiet bald veraltet sein könnte.

Dies unterstreicht die Weisheit der traditionellen liberalen Bildung,
die darauf abzielte, die Studierenden zu ermutigen, ihre kritischen
Fähigkeiten und Denkfertigkeiten zu entwickeln. Erfolg im
Geschäftsleben, wie in den meisten Lebensbereichen, hängt davon ab,
Probleme lösen zu können. Wenn Sie lernen, wie man Probleme löst, steht
Ihnen eine vielversprechende Karriere bevor. Egal, wo Sie leben, es gibt
zahlreiche Probleme, die auf Lösungen warten. In den meisten Fällen sind
diejenigen, die von Lösungen profitieren könnten oder deren Probleme Sie
angehen, bereit, Sie gut zu bezahlen, um diese Lösungen zu finden.


\backmatter


\end{document}
